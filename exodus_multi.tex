\newparsha{שמות}
\threeverse{\aliya{שמות}}%Ex.1:1
{וְאֵ֗לֶּה שְׁמוֹת֙ בְּנֵ֣י יִשְׂרָאֵ֔ל הַבָּאִ֖ים מִצְרָ֑יְמָה אֵ֣ת יַעֲקֹ֔ב אִ֥ישׁ וּבֵית֖וֹ בָּֽאוּ׃
\rashi{\rashiDH{ואלה שמות בני ישראל. }אע״פ שמנאן בחייהן בשמותם, חזר ומנאן במיתתן, להודיע חבתן שנמשלו לכוכבים, שמוציאן ומכניסן במספר ובשמותם, שנאמר המו‍ֹצִיא בְמִסְפָּר צְבָאָם לְכֻלָם בְֹּשֵם יִקְרָא (ישעי׳ מ, כו.  שמו״ר א, ג)׃ 
}}
{וְאִלֵּין שְׁמָהָת בְּנֵי יִשְׂרָאֵל דְּעָאלוּ לְמִצְרָיִם עִם יַעֲקֹב גְּבַר וַאֲנָשׁ בֵּיתֵיהּ עָאלוּ׃}
{NOW THESE are the names of the sons of Israel, who came into Egypt with Jacob; every man came with his household:}{\Roman{chap}}
\threeverse{\arabic{verse}}%Ex.1:2
{רְאוּבֵ֣ן שִׁמְע֔וֹן לֵוִ֖י וִיהוּדָֽה׃}
{רְאוּבֵן שִׁמְעוֹן לֵוִי וִיהוּדָה׃}
{Reuben, Simeon, Levi, and Judah;}{\arabic{verse}}
\threeverse{\arabic{verse}}%Ex.1:3
{יִשָּׂשכָ֥ר זְבוּלֻ֖ן וּבִנְיָמִֽן׃}
{יִשָּׂשכָר זְבוּלוּן וּבִנְיָמִין׃}
{Issachar, Zebulun, and Benjamin;}{\arabic{verse}}
\threeverse{\arabic{verse}}%Ex.1:4
{דָּ֥ן וְנַפְתָּלִ֖י גָּ֥ד וְאָשֵֽׁר׃}
{דָּן וְנַפְתָּלִי גָּד וְאָשֵׁר׃}
{Dan and Naphtali, Gad and Asher.}{\arabic{verse}}
\threeverse{\arabic{verse}}%Ex.1:5
{וַֽיְהִ֗י כׇּל־נֶ֛פֶשׁ יֹצְאֵ֥י יֶֽרֶךְ־יַעֲקֹ֖ב שִׁבְעִ֣ים נָ֑פֶשׁ וְיוֹסֵ֖ף הָיָ֥ה בְמִצְרָֽיִם׃
\rashi{\rashiDH{ויוסף היה במצרים. }והלא הוא ובניו היו בכלל שבעים, ומה בא ללמדנו, וכי לא היינו יודעים שהוא היה במצרים, אלא להודיעך צדקתו של יוסף, הוא יוסף הרועה את צאן אביו, הוא יוסף שהיה במצרים ונעשה מלך ועמד בצדקו׃ }}
{וַהֲוַאָה כָּל נַפְשָׁתָא נָפְקֵי יִרְכָּא דְּיַעֲקֹב שִׁבְעִין נַפְשָׁן וְיוֹסֵף דַּהֲוָה בְמִצְרָיִם׃}
{And all the souls that came out of the loins of Jacob were seventy souls; and Joseph was in Egypt already.}{\arabic{verse}}
\threeverse{\arabic{verse}}%Ex.1:6
{וַיָּ֤מׇת יוֹסֵף֙ וְכׇל־אֶחָ֔יו וְכֹ֖ל הַדּ֥וֹר הַהֽוּא׃}
{וּמִית יוֹסֵף וְכָל אֲחוֹהִי וְכֹל דָּרָא הַהוּא׃}
{And Joseph died, and all his brethren, and all that generation.}{\arabic{verse}}
\threeverse{\arabic{verse}}%Ex.1:7
{וּבְנֵ֣י יִשְׂרָאֵ֗ל פָּר֧וּ וַֽיִּשְׁרְצ֛וּ וַיִּרְבּ֥וּ וַיַּֽעַצְמ֖וּ בִּמְאֹ֣ד מְאֹ֑ד וַתִּמָּלֵ֥א הָאָ֖רֶץ אֹתָֽם׃ \petucha 
\rashi{\rashiDH{וישרצו. }שהיו יולדות ששה בכרס אחד׃}}
{וּבְנֵי יִשְׂרָאֵל נְפִישׁוּ וְאִתְיַלַּדוּ וּסְגִיאוּ וּתְקִיפוּ לַחְדָּא לַחְדָּא וְאִתְמְלִיאַת אַרְעָא מִנְּהוֹן׃}
{And the children of Israel were fruitful, and increased abundantly, and multiplied, and waxed exceeding mighty; and the land was filled with them.}{\arabic{verse}}
\threeverse{\aliya{לוי}}%Ex.1:8
{וַיָּ֥קׇם מֶֽלֶךְ־חָדָ֖שׁ עַל־מִצְרָ֑יִם אֲשֶׁ֥ר לֹֽא־יָדַ֖ע אֶת־יוֹסֵֽף׃
\rashi{\rashiDH{ויקם מלך חדש. }רב ושמואל, חד אמר חדש ממש, וחד אמר שנתחדשו גזירותיו (סוטה יא.)׃ }\rashi{\rashiDH{אשר לא ידע. }עשה עצמו כאלו לא ידע׃}}
{וְקָם מַלְכָּא חֲדַתָּא עַל מִצְרָיִם דְּלָא מְקַיֵּים גְּזֵירַת יוֹסֵף׃}
{Now there arose a new king over Egypt, who knew not Joseph.}{\arabic{verse}}
\threeverse{\arabic{verse}}%Ex.1:9
{וַיֹּ֖אמֶר אֶל־עַמּ֑וֹ הִנֵּ֗ה עַ֚ם בְּנֵ֣י יִשְׂרָאֵ֔ל רַ֥ב וְעָצ֖וּם מִמֶּֽנּוּ׃}
{וַאֲמַר לְעַמֵּיהּ הָא עַמָּא בְּנֵי יִשְׂרָאֵל סָגַן וְתָקְפִין מִנַּנָא׃}
{And he said unto his people: ‘Behold, the people of the children of Israel are too many and too mighty for us;}{\arabic{verse}}
\threeverse{\arabic{verse}}%Ex.1:10
{הָ֥בָה נִֽתְחַכְּמָ֖ה ל֑וֹ פֶּן־יִרְבֶּ֗ה וְהָיָ֞ה כִּֽי־תִקְרֶ֤אנָה מִלְחָמָה֙ וְנוֹסַ֤ף גַּם־הוּא֙ עַל־שֹׂ֣נְאֵ֔ינוּ וְנִלְחַם־בָּ֖נוּ וְעָלָ֥ה מִן־הָאָֽרֶץ׃
\rashi{\rashiDH{הבה נתחכמה לו. }כל הבה לשון הכנה והזמנה לְדָבָר הוא, כלומר הזמינו עצמיכם לכך׃ }\rashi{\rashiDH{נתחכמה לו. }לעם, נתחכמה מה לעשות לו. ורבותינו דרשו, נתחכם למושיען של ישראל לדונם במים, שכבר נשבע שלא יביא מבול לעולם (שמו״ר א, יא). (והם לא הבינו שעל כל העולם אינו מביא, אבל הוא מביא על אומה אחת. ברש״י ישן)׃ }\rashi{\rashiDH{ועלה מן הארץ. }על כרחנו. ורבותינו דרשו, כאדם שמקלל עצמו, ותולה קללתו באחרים, והרי הוא כאלו כתב וְעָלִינוּ מן הארץ והם יירשוה׃ }}
{הַבוּ נִתְחַכַּם לְהוֹן דִּלְמָא יִסְגּוֹן וִיהֵי אֲרֵי יְעָרְעִנַּנָא קְרָב וְיִתּוֹסְפוּן אַף אִנּוּן עַל סָנְאַנָא וִיגִיחוּן בַּנָא קְרָב וְיִסְּקוּן מִן אַרְעָא׃}
{come, let us deal wisely with them, lest they multiply, and it come to pass, that, when there befalleth us any war, they also join themselves unto our enemies, and fight against us, and get them up out of the land.’}{\arabic{verse}}
\threeverse{\arabic{verse}}%Ex.1:11
{וַיָּשִׂ֤ימוּ עָלָיו֙ שָׂרֵ֣י מִסִּ֔ים לְמַ֥עַן עַנֹּת֖וֹ בְּסִבְלֹתָ֑ם וַיִּ֜בֶן עָרֵ֤י מִסְכְּנוֹת֙ לְפַרְעֹ֔ה אֶת־פִּתֹ֖ם וְאֶת־רַעַמְסֵֽס׃
\rashi{\rashiDH{עליו. }על העם׃}\rashi{\rashiDH{מסים. }לשון מס, שרים שגובין מהם המס. ומהו המס, שיבנו ערי מסכנות לפרעה׃ }\rashi{\rashiDH{למען ענותו בסבלתם. }של מצרים׃}\rashi{\rashiDH{ערי מסכנות. }כתרגומו, וכן לֶךְ בֹּא אֶל הַסֹּכֵן הַזֶה (ישעי׳ כב, טו), גזבר הממונה על האוצרות (שמו״ר א, יד)׃}\rashi{\rashiDH{את פתום ואת רעמסס. }שלא היו ראויות מתחלה לכך, ועשאום חזקות ובצורות לאוצר׃ }}
{וּמַנִּיאוּ עֲלֵיהוֹן שִׁלְטוֹנִין מַבְאֲשִׁין בְּדִיל לְעַנּוֹאֵיהוֹן בְּפוּלְחָנְהוֹן וּבְנוֹ קִרְוֵי בֵּית אוֹצְרֵי לְפַרְעֹה יָת פִּיתוֹם וְיָת רַעַמְסֵס׃}
{Therefore they did set over them taskmasters to afflict them with their burdens. And they built for Pharaoh store-cities, Pithom and Raamses.}{\arabic{verse}}
\threeverse{\arabic{verse}}%Ex.1:12
{וְכַאֲשֶׁר֙ יְעַנּ֣וּ אֹת֔וֹ כֵּ֥ן יִרְבֶּ֖ה וְכֵ֣ן יִפְרֹ֑ץ וַיָּקֻ֕צוּ מִפְּנֵ֖י בְּנֵ֥י יִשְׂרָאֵֽל׃
\rashi{\rashiDH{וכאשר יענו אותו. }בכל מה שהם נותנין לב לענות, כן לב הקב״ה להרבות ולהפריץ. }\rashi{\rashiDH{כן ירבה, }כן רבה וכן פרץ. ומדרשו, רוח הקודש אומרת כן, אתם אומרים פן ירבה, ואני אומר כן ירבה׃ }\rashi{\rashiDH{ויקצו. }קצו בחייהם. (ויש מפרשים המצרים בעיני עצמם וק״ל). ורבותינו דרשו, כקוצים היו בעיניהם׃ }}
{וּכְמָא דִּמְעַנַּן לְהוֹן כֵּן סָגַן וְכֵן תָּקְפִין וְעַקַת לְמִצְרָאֵי מִן קֳדָם בְּנֵי יִשְׂרָאֵל׃}
{But the more they afflicted them, the more they multiplied and the more they spread abroad. And they were adread because of the children of Israel.}{\arabic{verse}}
\threeverse{\aliya{ישראל}}%Ex.1:13
{וַיַּעֲבִ֧דוּ מִצְרַ֛יִם אֶת־בְּנֵ֥י יִשְׂרָאֵ֖ל בְּפָֽרֶךְ׃
\rashi{\rashiDH{בפרך. }בעבודה קשה המפרכת את הגוף ומשברתו׃}}
{וְאַפְלַחוּ מִצְרָאֵי יָת בְּנֵי יִשְׂרָאֵל בְּקַשְׁיוּ׃}
{And the Egyptians made the children of Israel to serve with rigour.}{\arabic{verse}}
\threeverse{\arabic{verse}}%Ex.1:14
{וַיְמָרְר֨וּ אֶת־חַיֵּיהֶ֜ם בַּעֲבֹדָ֣ה קָשָׁ֗ה בְּחֹ֙מֶר֙ וּבִלְבֵנִ֔ים וּבְכׇל־עֲבֹדָ֖ה בַּשָּׂדֶ֑ה אֵ֚ת כׇּל־עֲבֹ֣דָתָ֔ם אֲשֶׁר־עָבְד֥וּ בָהֶ֖ם בְּפָֽרֶךְ׃}
{וְאַמַּרוּ יָת חַיֵּיהוֹן בְּפוּלְחָנָא קַשְׁיָא בְּטִינָא וּבְלִבְנֵי וּבְכָל פּוּלְחָנָא בְּחַקְלָא יָת כָּל פּוּלְחָנְהוֹן דְּאַפְלַחוּ בְּהוֹן בְּקַשְׁיוּ׃}
{And they made their lives bitter with hard service, in mortar and in brick, and in all manner of service in the field; in all their service, wherein they made them serve with rigour.}{\arabic{verse}}
\threeverse{\arabic{verse}}%Ex.1:15
{וַיֹּ֙אמֶר֙ מֶ֣לֶךְ מִצְרַ֔יִם לַֽמְיַלְּדֹ֖ת הָֽעִבְרִיֹּ֑ת אֲשֶׁ֨ר שֵׁ֤ם הָֽאַחַת֙ שִׁפְרָ֔ה וְשֵׁ֥ם הַשֵּׁנִ֖ית פּוּעָֽה׃
\rashi{\rashiDH{למילדות. }הוא לשון מולידות, אלא שיש לשון קל ויש לשון כבד, כמו שובר ומשבר, דובר ומדבר, כך מוליד וּמְיַלֵּד׃ 
}\rashi{\rashiDH{שפרה. }זו יוכבד, על שם שֶׁמְשַׁפֶּרֶת את הולד׃ }\rashi{\rashiDH{פועה. }זו מרים, על שם שׁפּוֹעָה ומדברת והוגה לולד, כדרך הנשים המפייסות תינוק הבוכה (סוטה יא׃)׃ }\rashi{\rashiDH{פועה. }לשון צעקה, כמו כַּיּו‍ֹלֵדָה אֶפְעֶה (ישעי׳ מב, יד)׃ }}
{וַאֲמַר מַלְכָּא דְּמִצְרַיִם לְחָיָתָא יְהוּדַיָתָא דְּשׁוֹם חֲדָא שִׁפְרָה וְשׁוֹם תִּנְיֵיתָא פּוּעָה׃}
{And the king of Egypt spoke to the Hebrew midwives, of whom the name of the one was Shiphrah, and the name of the other Puah;}{\arabic{verse}}
\threeverse{\arabic{verse}}%Ex.1:16
{וַיֹּ֗אמֶר בְּיַלֶּדְכֶן֙ אֶת־הָֽעִבְרִיּ֔וֹת וּרְאִיתֶ֖ן עַל־הָאׇבְנָ֑יִם אִם־בֵּ֥ן הוּא֙ וַהֲמִתֶּ֣ן אֹת֔וֹ וְאִם־בַּ֥ת הִ֖וא וָחָֽיָה׃
\rashi{\rashiDH{בילדכן. }כמו בהולידכן׃}\rashi{\rashiDH{האבנים. }מושב האשה היולדת, ובמקום אחר קוראו משבר, וכמוהו עֹשֶה מְלָאכָה עַל הָאָבְנָיִם (ירמי׳ יח, ג), מושב כלי אומנות יוצר כלי חרס׃ }\rashi{\rashiDH{אם בן הוא וגו׳. }לא היה מקפיד אלא על הזכרים, שאמרו לו אִצְטַגְנִינָיו שעתיד להוולד בן המושיע אותם׃ 
}\rashi{\rashiDH{וחיה. }ותחיה׃}}
{וַאֲמַר כַּד תִּהְוְיָין מְיַלְּדָן יָת יְהוּדַיָתָא וְתִחְזְיָין עַל מַתְבְרָא אִם בְּרָא הוּא וְתִקְטְלָן יָתֵיהּ וְאִם בְּרַתָּא הִיא תְּקַיְּימֻנַּהּ׃}
{and he said: ‘When ye do the office of a midwife to the Hebrew women, ye shall look upon the birthstool: if it be a son, then ye shall kill him; but if it be a daughter, then she shall live.’}{\arabic{verse}}
\threeverse{\arabic{verse}}%Ex.1:17
{וַתִּירֶ֤אןָ הַֽמְיַלְּדֹת֙ אֶת־הָ֣אֱלֹהִ֔ים וְלֹ֣א עָשׂ֔וּ כַּאֲשֶׁ֛ר דִּבֶּ֥ר אֲלֵיהֶ֖ן מֶ֣לֶךְ מִצְרָ֑יִם וַתְּחַיֶּ֖יןָ אֶת־הַיְלָדִֽים׃
\rashi{\rashiDH{ותחיין את הילדים. }מְסַפְּקוֹת להם מים ומזון. (סוטה יא׃) תרגום הראשון וְקַיָּימא, והשני וְקַיֵּימְתּוּן, לפי שלשון עברית לנקבות רבות, תיבה זו וכיוצא בה משמשת לשון פעלו ולשון פעלתם, כגון ותאמרן איש מצרי, (שמות ב, יט) לשון עבר כמו ויאמרו לזכרים, וַתְּדַבֵּרְנָה בְּפִיכֶם (ירמי׳ מד, כה), לשון דברתם כמו ותדברו לזכרים, וכן וַתְּחַלֶלְנָה אֹתִי אֶל עַמִּי, (יחזקאל יג, יט) לשון עבר חללתם כמו ותחללו לזכרים׃ 
}}
{וּדְחִילָא חָיָתָא מִן קֳדָם יְיָ וְלָא עֲבַדָא כְּמָא דְּמַלֵּיל עִמְּהוֹן מַלְכָּא דְּמִצְרָיִם וְקַיִּימָא יָת בְּנַיָּא׃}
{But the midwives feared God, and did not as the king of Egypt commanded them, but saved the men-children alive.}{\arabic{verse}}
\threeverse{\aliya{שני}}%Ex.1:18
{וַיִּקְרָ֤א מֶֽלֶךְ־מִצְרַ֙יִם֙ לַֽמְיַלְּדֹ֔ת וַיֹּ֣אמֶר לָהֶ֔ן מַדּ֥וּעַ עֲשִׂיתֶ֖ן הַדָּבָ֣ר הַזֶּ֑ה וַתְּחַיֶּ֖יןָ אֶת־הַיְלָדִֽים׃}
{וּקְרָא מַלְכָּא דְּמִצְרַיִם לְחָיָתָא וַאֲמַר לְהוֹן מָא דֵין עֲבַדְתִּין פִּתְגָמָא הָדֵין וְקַיֵּימְתִּין יָת בְּנַיָא׃}
{And the king of Egypt called for the midwives, and said unto them: ‘Why have ye done this thing, and have saved the men-children alive?’}{\arabic{verse}}
\threeverse{\arabic{verse}}%Ex.1:19
{וַתֹּאמַ֤רְןָ הַֽמְיַלְּדֹת֙ אֶל־פַּרְעֹ֔ה כִּ֣י לֹ֧א כַנָּשִׁ֛ים הַמִּצְרִיֹּ֖ת הָֽעִבְרִיֹּ֑ת כִּֽי־חָי֣וֹת הֵ֔נָּה בְּטֶ֨רֶם תָּב֧וֹא אֲלֵהֶ֛ן הַמְיַלֶּ֖דֶת וְיָלָֽדוּ׃
\rashi{\rashiDH{כי חיות הנה. }בקיאות כמילדות, תרגום מילדות חַיָּתָא. ורבותינו דרשו, (סוטה יא׃  ושמו״ר) הרי הן משולות לחיות השדה שאינן צריכות מילדות, והיכן משולות לחיות, גור אריה, זאב יטרף, בכור שורו, אילה שלוחה, ומי שלא נכתב בו, הרי הכתוב כללן, ויברך אותם ועוד כתיב מָה אִמְּךָ לְבִיָא (יחזקאל יט, ב)׃ }}
{וַאֲמַרָא חָיָתָא לְפַרְעֹה אֲרֵי לָא כִנְשַׁיָּא מִצְרַיָתָא יְהוּדַיָתָא אֲרֵי חַכִּימָן אִנִּין עַד לָא עַלַת לְוָתְהוֹן חָיְתָא יָלְדָן׃}
{And the midwives said unto Pharaoh: ‘Because the Hebrew women are not as the Egyptian women; for they are lively, and are delivered ere the midwife come unto them.’}{\arabic{verse}}
\threeverse{\arabic{verse}}%Ex.1:20
{וַיֵּ֥יטֶב אֱלֹהִ֖ים לַֽמְיַלְּדֹ֑ת וַיִּ֧רֶב הָעָ֛ם וַיַּֽעַצְמ֖וּ מְאֹֽד׃
\rashi{\rashiDH{וייטב. }הטיב להם, וזה חלוק בתיבה שיסודה ב׳ אותיות ונתן לה וי״ו יו״ד בראשה, כשהיא באה לדבר בלשון ויפעל, הוא נקוד היו״ד בציר״י שהוא קמ״ץ קטן, (או בסגול שהוא פת״ח קטן)כגון וייטב אלהים למילדות, וַיֶרֶב בְּבַת יְהוּדָה (איכה ב, ה), הרבה תאניה, וכן וַיֶּגֶל הַשְּׁאֵרִית (דברי הימים־ב לו, כ), נבוזראדן הגלה את השארית, וַיֶפֶן זָנָב אֶל זָנָב (שופטים טו, ה), הפנה הזנבות זו לזו, כל אלו לשון מפעיל את אחרים, וכשהוא מדבר בלשון ויפעל, הוא נקוד היו״ד בחיר״ק, כגון וַיִּיטַב בְעֵינָיו (ויקרא י, כ), לשון הוטב, וכן וירב העם, נתרבה העם, ויגל יהודה, הגלה יהודה, ויפן כה וכה, פנה לכאן ולכאן. ואל תשיבני, וילך, וישב, וירד, ויצא, לפי שאינן מגזרתן של אלו, שהרי היו״ד יסוד בהן, ירד, יצא, ישב, ילך, יו״ד אות שלישית בו׃ }\rashi{\rashiDH{וייטב אלהים למילדת. }מהו הטובה׃}}
{וְאוֹטֵיב יְיָ לְחָיָתָא וּסְגִי עַמָּא וּתְקִיפוּ לַחְדָּא׃}
{And God dealt well with the midwives; and the people multiplied, and waxed very mighty.}{\arabic{verse}}
\threeverse{\arabic{verse}}%Ex.1:21
{וַיְהִ֕י כִּֽי־יָרְא֥וּ הַֽמְיַלְּדֹ֖ת אֶת־הָאֱלֹהִ֑ים וַיַּ֥עַשׂ לָהֶ֖ם בָּתִּֽים׃
\rashi{\rashiDH{ויעש להם בתים. }בתי כהונה ולויה ומלכות שקרויין בתים, ויבן את בית ה׳ ואת בית המלך, כהונה ולויה מיוכבד, ומלכות ממרים, כדאיתא במסכת סוטה (יא׃)׃ }}
{וַהֲוָה כַּד דְּחִילָא חָיָתָא מִן קֳדָם יְיָ וַעֲבַד לְהוֹן בָּתִּין׃}
{And it came to pass, because the midwives feared God, that He made them houses.}{\arabic{verse}}
\threeverse{\arabic{verse}}%Ex.1:22
{וַיְצַ֣ו פַּרְעֹ֔ה לְכׇל־עַמּ֖וֹ לֵאמֹ֑ר כׇּל־הַבֵּ֣ן הַיִּלּ֗וֹד הַיְאֹ֙רָה֙ תַּשְׁלִיכֻ֔הוּ וְכׇל־הַבַּ֖ת תְּחַיּֽוּן׃ \petucha 
\rashi{\rashiDH{לכל עמו. }אף עליהם גזר, יום שנולד משה אמרו לו אִצְטַגְנִינָיו, היום נולד מושיען, ואין אנו יודעים אם ממצרים אם מישראל, ורואין אנו שסופו ללקות במים, לפיכך גזר אותו היום אף על המצרים, שנאמר כל הבן הילוד, ולא נאמר הילוד לעברים, והם לא היו יודעים שסופו ללקות על מי מריבה׃ }}
{וּפַקֵּיד פַּרְעֹה לְכָל עַמֵּיהּ לְמֵימַר כָּל בְּרָא דְּיִתְיְלֵיד לִיהוּדָאֵי בְּנַהְרָא תִּרְמוֹנֵיהּ וְכָל בְּרַתָּא תְּקַיְּימוּן׃}
{And Pharaoh charged all his people, saying: ‘Every son that is born ye shall cast into the river, and every daughter ye shall save alive.’}{\arabic{verse}}
\newperek
\threeverse{\Roman{chap}}%Ex.2:1
{וַיֵּ֥לֶךְ אִ֖ישׁ מִבֵּ֣ית לֵוִ֑י וַיִּקַּ֖ח אֶת־בַּת־לֵוִֽי׃
\rashi{\rashiDH{ויקח את בת לוי. }פָּרוּשׁ היה ממנה מפני גזירת פרעה, (וחזר ולקחה, וזהו וילך, שהלך בעצת בתו שאמרה לו גזרתך קשה משל פרעה, אם פרעה גזר על הזכרים ואתה גם כן על הנקבות. ברש״י ישן) והחזירה ועשה בה לקוחין שניים, ואף היא נהפכה להיות נערה. ובת ק״ל שנה היתה, שנולדה בבואה למצרים בין החומות, ומאתים ועשר נשתהו שם, וכשיצאו היה משה בן שמונים שנה, אם כן כשנתעברה ממנו היתה בת מאה ושלשים, וקורא אותה בת לוי׃ }}
{וַאֲזַל גּוּבְרָא מִדְּבֵית לֵוִי וּנְסֵיב יָת בַּת לֵוִי׃}
{And there went a man of the house of Levi, and took to wife a daughter of Levi.}{\Roman{chap}}
\threeverse{\arabic{verse}}%Ex.2:2
{וַתַּ֥הַר הָאִשָּׁ֖ה וַתֵּ֣לֶד בֵּ֑ן וַתֵּ֤רֶא אֹתוֹ֙ כִּי־ט֣וֹב ה֔וּא וַֽתִּצְפְּנֵ֖הוּ שְׁלֹשָׁ֥ה יְרָחִֽים׃
\rashi{\rashiDH{כי טוב הוא. }כשנולד נתמלא הבית כֻּלּוֹ אורה (סוטה יב.)׃}}
{וְעַדִּיאַת אִתְּתָא וִילֵידַת בַּר וַחֲזָת יָתֵיהּ אֲרֵי טָב הוּא וְאַטְמַרְתֵּיהּ תְּלָתָא יַרְחִין׃}
{And the woman conceived, and bore a son; and when she saw him that he was a goodly child, she hid him three months.}{\arabic{verse}}
\threeverse{\arabic{verse}}%Ex.2:3
{וְלֹא־יָכְלָ֣ה עוֹד֮ הַצְּפִינוֹ֒ וַתִּֽקַּֽח־לוֹ֙ תֵּ֣בַת גֹּ֔מֶא וַתַּחְמְרָ֥ה בַחֵמָ֖ר וּבַזָּ֑פֶת וַתָּ֤שֶׂם בָּהּ֙ אֶת־הַיֶּ֔לֶד וַתָּ֥שֶׂם בַּסּ֖וּף עַל־שְׂפַ֥ת הַיְאֹֽר׃
\rashi{\rashiDH{ולא יכלה עוד הצפינו. }שמנו לה המצריים מיום שהחזירה, והיא ילדתו לששה חדשים ויום א׳, שהיולדת לשבעה יולדת למקוטעין, והם בדקו אחריה לסוף ט׳׃ }\rashi{\rashiDH{גמא. }גמי בלשון משנה ובלע״ז יונ״קו, ודבר רך הוא, ועומד בפני רך ובפני קשה׃ }\rashi{\rashiDH{בחמר ובזפת. }זפת מבחוץ וטיט מבפנים, כדי שלא יריח אותו צדיק ריח רע של זפת׃ }\rashi{\rashiDH{ותשם בסוף. }הוא לשון אגם רושי״ל בלע״ז, ודומה לו קָנֶה וָסוּף קָמֵלוּ (ישעי׳ יט, ו)׃ }}
{וְלָא יְכֵילַת עוֹד לְאַטְמָרוּתֵיהּ וּנְסֵיבַת לֵיהּ תֵּיבְתָא דְּגֹומֶא וַחֲפָתַהּ בְּחֵימָרָא וּבְזִפְתָּא וְשַׁוִּיאַת בַּהּ יָת רָבְיָא וְשַׁוִּיתַהּ בְּיַעְרָא עַל כֵּיף נַהְרָא׃}
{And when she could not longer hide him, she took for him an ark of bulrushes, and daubed it with slime and with pitch; and she put the child therein, and laid it in the flags by the river’s brink.}{\arabic{verse}}
\threeverse{\arabic{verse}}%Ex.2:4
{וַתֵּתַצַּ֥ב אֲחֹת֖וֹ מֵרָחֹ֑ק לְדֵעָ֕ה מַה־יֵּעָשֶׂ֖ה לֽוֹ׃}
{וְאִתְעַתַּדַת אֲחָתֵיהּ מֵרַחִיק לְמִדַּע מָא יִתְעֲבֵיד לֵיהּ׃}
{And his sister stood afar off, to know what would be done to him.}{\arabic{verse}}
\threeverse{\arabic{verse}}%Ex.2:5
{וַתֵּ֤רֶד בַּת־פַּרְעֹה֙ לִרְחֹ֣ץ עַל־הַיְאֹ֔ר וְנַעֲרֹתֶ֥יהָ הֹלְכֹ֖ת עַל־יַ֣ד הַיְאֹ֑ר וַתֵּ֤רֶא אֶת־הַתֵּבָה֙ בְּת֣וֹךְ הַסּ֔וּף וַתִּשְׁלַ֥ח אֶת־אֲמָתָ֖הּ וַתִּקָּחֶֽהָ׃
\rashi{\rashiDH{לרחץ על היאור. }סרס המקרא ופרשהו, ותרד בת פרעה על היאור לרחוץ בו׃ }\rashi{\rashiDH{על יד היאור. }אצל היאור, כמו רְאוּ חֶלְקַת יו‍ֹאָב אֶל יָדִי (שמואל־ב יד, ל), והוא לשון יד ממש, שיד האדם סמוכה לו. ורבותינו דרשו (סוטה יב׃) הולכות לשון מיתה, כמו הִנֵּה אָנֹכִי הו‍ֹלֵךְ לָמוּת (בראשית כה, לב), הולכות למות לפי שמיחו בה, והכתוב מסייען, כי למה לנו לכתוב ונערותיה הולכות׃ }\rashi{\rashiDH{את אמתה. }את שפחתה. ורבותינו דרשו לשון יד, אבל לפי דקדוק לשון הקודש היה לו להנקד אמתה מ״ם דגושה, והם דרשו את אמתה, את ידה ונשתרבבה אמתה אמות הרבה (סוטה שם)׃ }}
{וּנְחַתַת בַּת פַּרְעֹה לְמִסְחֵי עַל נַהְרָא וְעוּלֵימְתַהָא מְהַלְּכָן עַל כֵּיף נַהְרָא וַחֲזָת יָת תֵּיבְתָא בְּגוֹ יַעְרָא וְאוֹשֵׁיטַת יָת אַמְּתַהּ וּנְסֵיבְתַּהּ׃}
{And the daughter of Pharaoh came down to bathe in the river; and her maidens walked along by the river-side; and she saw the ark among the flags, and sent her handmaid to fetch it.}{\arabic{verse}}
\threeverse{\arabic{verse}}%Ex.2:6
{וַתִּפְתַּח֙ וַתִּרְאֵ֣הוּ אֶת־הַיֶּ֔לֶד וְהִנֵּה־נַ֖עַר בֹּכֶ֑ה וַתַּחְמֹ֣ל עָלָ֔יו וַתֹּ֕אמֶר מִיַּלְדֵ֥י הָֽעִבְרִ֖ים זֶֽה׃
\rashi{\rashiDH{ותפתח ותראהו. }את מי ראתה, הילד, זהו פשוטו. ומדרשו, שראתה עמו שכינה (סוטה שם  שמו״ר א, כח)׃ 
}\rashi{\rashiDH{והנה נער בוכה. }קולו כנער׃}}
{וּפְתַחַת וַחֲזָת יָת רָבְיָא וְהָא עוּלֵימָא בָּכֵי וְחַסַת עֲלוֹהִי וַאֲמַרַת מִבְּנֵי יְהוּדָאֵי הוּא דֵין׃}
{And she opened it, and saw it, even the child; and behold a boy that wept. And she had compassion on him, and said: ‘This is one of the Hebrews’ children.’}{\arabic{verse}}
\threeverse{\arabic{verse}}%Ex.2:7
{וַתֹּ֣אמֶר אֲחֹתוֹ֮ אֶל־בַּת־פַּרְעֹה֒ הַאֵלֵ֗ךְ וְקָרָ֤אתִי לָךְ֙ אִשָּׁ֣ה מֵינֶ֔קֶת מִ֖ן הָעִבְרִיֹּ֑ת וְתֵינִ֥ק לָ֖ךְ אֶת־הַיָּֽלֶד׃
\rashi{\rashiDH{מן העבריות. }שהחזירתו על מצריות הרבה לינק ולא ינק, לפי שהיה עתיד לדבר עם השכינה (שמו״ר א, ל.  סוטה שם)׃ }}
{וַאֲמַרַת אֲחָתֵיהּ לְבַת פַּרְעֹה הַאֵיזֵיל וְאֶקְרֵי לִיךְ אִתְּתָא מֵינִקְתָּא מִן יְהוּדַיָתָא וְתוֹנִיק לִיךְ יָת רָבְיָא׃}
{Then said his sister to Pharaoh’s daughter: ‘Shall I go and call thee a nurse of the Hebrew women, that she may nurse the child for thee?’}{\arabic{verse}}
\threeverse{\arabic{verse}}%Ex.2:8
{וַתֹּֽאמֶר־לָ֥הּ בַּת־פַּרְעֹ֖ה לֵ֑כִי וַתֵּ֙לֶךְ֙ הָֽעַלְמָ֔ה וַתִּקְרָ֖א אֶת־אֵ֥ם הַיָּֽלֶד׃
\rashi{\rashiDH{ותלך העלמה. }הלכה בזריזות ועלמות כעלם׃}}
{וַאֲמַרַת לַהּ בַּת פַּרְעֹה אִיזִילִי וַאֲזַלַת עוּלֵימְתָא וּקְרָת יָת אִמֵּיהּ דְּרָבְיָא׃}
{And Pharaoh’s daughter said to her: ‘Go.’ And the maiden went and called the child’s mother.}{\arabic{verse}}
\threeverse{\arabic{verse}}%Ex.2:9
{וַתֹּ֧אמֶר לָ֣הּ בַּת־פַּרְעֹ֗ה הֵילִ֜יכִי אֶת־הַיֶּ֤לֶד הַזֶּה֙ וְהֵינִקִ֣הוּ לִ֔י וַאֲנִ֖י אֶתֵּ֣ן אֶת־שְׂכָרֵ֑ךְ וַתִּקַּ֧ח הָאִשָּׁ֛ה הַיֶּ֖לֶד וַתְּנִיקֵֽהוּ׃
\rashi{\rashiDH{היליכי. }נתנבאה ולא ידעה מה נתנבאה, הי שליכי (שמו״ר שם  סוטה שם)׃ }}
{וַאֲמַרַת לַהּ בַּת פַּרְעֹה הָלִיכִי יָת רָבְיָא הָדֵין וְאוֹנִיקִיהוּ לִי וַאֲנָא אֶתֵּין יָת אַגְרִיךְ וּנְסֵיבַת אִתְּתָא רָבְיָא וְאוֹנִיקְתֵּיהּ׃}
{And Pharaoh’s daughter said unto her: ‘Take this child away, and nurse it for me, and I will give thee thy wages.’ And the woman took the child, and nursed it.}{\arabic{verse}}
\threeverse{\arabic{verse}}%Ex.2:10
{וַיִּגְדַּ֣ל הַיֶּ֗לֶד וַתְּבִאֵ֙הוּ֙ לְבַת־פַּרְעֹ֔ה וַֽיְהִי־לָ֖הּ לְבֵ֑ן וַתִּקְרָ֤א שְׁמוֹ֙ מֹשֶׁ֔ה וַתֹּ֕אמֶר כִּ֥י מִן־הַמַּ֖יִם מְשִׁיתִֽהוּ׃
\rashi{\rashiDH{משיתהו. }כתרגומו שְׁחַלְתֵּיהּ, והוא לשון הוצאה בלשון ארמי, כְּמִשְׁחַל בִּנִּיתָא מֵחַלְבָא, ובלשון עברי משיתהו, לשון הסירותיו, כמו לֹא יָמוּש (יהושע א, ח) לא משו, כך חברו מנחם. ואני אומר שאינו ממחברת מש, וימוש, אלא מגזרת מָשָׁה, ולשון הוצאה הוא, וכן יַמְשֵנִי מִמַּיִם רַבִּים (שמואל־ב כב, יז), שאלו היה ממחברת מש, לא יתכן לומר משיתיהו אלא הֲמִישׁוֹתִיהוּ, כאשר יאמר מן קם הקימותי, ומן שב השיבותי, ומן בא הביאותי, או משתיהו, כמו וּמַֹש ְתִּי אֶת עֲו‍ֹן הָאָרֶץ (זכריה ג, ט), אבל משיתי, אינו אלא מגזרת תיבה שפעל שלה מיוסד בה״א בסוף התיבה, כגון משה, בנה, עשה, צוה, פנה, כשיבא לומר בהם פעלתי, תבא היו״ד במקום ה״א, כמו עשיתי, בניתי, פניתי, צויתי׃ }}
{וּרְבָא רָבְיָא וְאֵיתִיתֵיהּ לְבַת פַּרְעֹה וַהֲוָה לַהּ לְבַר וּקְרָאת שְׁמֵיהּ מֹשֶׁה וַאֲמַרַת אֲרֵי מִן מַיָּא שְׁחַלְתֵּיהּ׃}
{And the child grew, and she brought him unto Pharaoh’s daughter, and he became her son. And she called his name Moses, and said: ‘Because I drew him out of the water.’}{\arabic{verse}}
\threeverse{\aliya{שלישי}}%Ex.2:11
{וַיְהִ֣י \legarmeh  בַּיָּמִ֣ים הָהֵ֗ם וַיִּגְדַּ֤ל מֹשֶׁה֙ וַיֵּצֵ֣א אֶל־אֶחָ֔יו וַיַּ֖רְא בְּסִבְלֹתָ֑ם וַיַּרְא֙ אִ֣ישׁ מִצְרִ֔י מַכֶּ֥ה אִישׁ־עִבְרִ֖י מֵאֶחָֽיו׃
\rashi{\rashiDH{ויגדל משה. }והלא כבר כתב ויגדל הילד, א״ר יהודא בר״א הראשון לקומה והשני לגדולה, שמינהו פרעה על ביתו׃ }\rashi{\rashiDH{וירא בסבלותם. }נתן עיניו ולבו להיות מיצר עליהם (שמו״ר א, כז)׃ }\rashi{\rashiDH{איש מצרי. }נוגש היה, ממונה על שוטרי ישראל, והיה מעמידם מקרות הגבר למלאכתם (שמו״ר א, כח)׃ }\rashi{\rashiDH{מכה איש עברי. }מלקהו ורודהו. ובעלה של שלומית בת דברי היה, ונתן עיניו בה, ובלילה העמידו והוציאו מביתו, והוא חזר ונכנס לבית ובא על אשתו, כסבורה שהוא בעלה, וחזר האיש לביתו והרגיש בדבר, וכשראה אותו מצרי שהרגיש בדבר, היה מכהו ורודהו כל היום׃ }}
{וַהֲוָה בְּיוֹמַיָּא הָאִנּוּן וּרְבָא מֹשֶׁה וּנְפַק לְוָת אֲחוֹהִי וַחֲזָא בְּפוּלְחָנְהוֹן וַחֲזָא גְּבַר מִצְרַי מָחֵי לִגְבַר יְהוּדַי מֵאֲחוֹהִי׃}
{And it came to pass in those days, when Moses was grown up, that he went out unto his brethren, and looked on their burdens; and he saw an Egyptian smiting a Hebrew, one of his brethren.}{\arabic{verse}}
\threeverse{\arabic{verse}}%Ex.2:12
{וַיִּ֤פֶן כֹּה֙ וָכֹ֔ה וַיַּ֖רְא כִּ֣י אֵ֣ין אִ֑ישׁ וַיַּךְ֙ אֶת־הַמִּצְרִ֔י וַֽיִּטְמְנֵ֖הוּ בַּחֽוֹל׃
\rashi{\rashiDH{ויפן כה וכה. }ראה מה עשה לו בבית ומה עשה לו בשדה. ולפי פשוטו כמשמעו׃}\rashi{\rashiDH{וירא כי אין איש. }שאין איש עתיד לצאת ממנו שיתגייר׃}}
{וְאִתְפְּנִי לְכָא וּלְכָא וַחֲזָא אֲרֵי לֵית אֲנָשׁ וּמְחָא יָת מִצְרָאָה וְטַמְרֵיהּ בְּחָלָא׃}
{And he looked this way and that way, and when he saw that there was no man, he smote the Egyptian, and hid him in the sand.}{\arabic{verse}}
\threeverse{\arabic{verse}}%Ex.2:13
{וַיֵּצֵא֙ בַּיּ֣וֹם הַשֵּׁנִ֔י וְהִנֵּ֛ה שְׁנֵֽי־אֲנָשִׁ֥ים עִבְרִ֖ים נִצִּ֑ים וַיֹּ֙אמֶר֙ לָֽרָשָׁ֔ע לָ֥מָּה תַכֶּ֖ה רֵעֶֽךָ׃
\rashi{\rashiDH{שני אנשים עברים. }דתן ואבירם הם, שהותירו מן המן׃ }\rashi{\rashiDH{נצים. }מריבים׃}\rashi{\rashiDH{למה תכה. }אע״פ שלא הכהו, נקרא רשע בהרמת יד׃ }\rashi{\rashiDH{רעך. }רשע כמותך׃ 
}}
{וּנְפַק בְּיוֹמָא תִּנְיָנָא וְהָא תְּרֵין גּוּבְרִין יְהוּדָאִין נָצַן וַאֲמַר לְחַיָּבָא לְמָא אַתְּ מָחֵי לְחַבְרָךְ׃}
{And he went out the second day, and, behold, two men of the Hebrews were striving together; and he said to him that did the wrong: ‘Wherefore smitest thou thy fellow?’}{\arabic{verse}}
\threeverse{\arabic{verse}}%Ex.2:14
{וַ֠יֹּ֠אמֶר מִ֣י שָֽׂמְךָ֞ לְאִ֨ישׁ שַׂ֤ר וְשֹׁפֵט֙ עָלֵ֔ינוּ הַלְהׇרְגֵ֙נִי֙ אַתָּ֣ה אֹמֵ֔ר כַּאֲשֶׁ֥ר הָרַ֖גְתָּ אֶת־הַמִּצְרִ֑י וַיִּירָ֤א מֹשֶׁה֙ וַיֹּאמַ֔ר אָכֵ֖ן נוֹדַ֥ע הַדָּבָֽר׃
\rashi{\rashiDH{מי שמך לאיש. }והרי עודך נער׃}\rashi{\rashiDH{הלהרגני אתה אומר. }מכאן אנו למדים שהרגו בשם המפורש׃}\rashi{\rashiDH{ויירא משה. }כפשוטו. ומדרשו, דאג לו על שראה בישראל רשעים דֵּילָטוֹרִין, אמר, מעתה שמא אינם ראויין להגאל׃ }\rashi{\rashiDH{אכן נודע הדבר. }כמשמעו. ומדרשו, נודע לי הדבר שהייתי תמה עליו, מה חטאו ישראל מכל שבעים אומות להיות נרדים בעבודת פרך, אבל רואה אני שהם ראויים לכך׃ }}
{וַאֲמַר מַן שַׁוְיָךְ לִגְבַר רַב וְדַיָּין עֲלַנָא הַלְמִקְטְלִי אַתְּ אָמַר כְּמָא דִּקְטַלְתָּא יָת מִצְרָאָה וּדְחֵיל מֹשֶׁה וַאֲמַר בְּקוּשְׁטָא אִתְיְדַע פִּתְגָמָא׃}
{And he said: ‘Who made thee a ruler and a judge over us? thinkest thou to kill me, as thou didst kill the Egyptian?’ And Moses feared, and said: ‘Surely the thing is known.’}{\arabic{verse}}
\threeverse{\arabic{verse}}%Ex.2:15
{וַיִּשְׁמַ֤ע פַּרְעֹה֙ אֶת־הַדָּבָ֣ר הַזֶּ֔ה וַיְבַקֵּ֖שׁ לַהֲרֹ֣ג אֶת־מֹשֶׁ֑ה וַיִּבְרַ֤ח מֹשֶׁה֙ מִפְּנֵ֣י פַרְעֹ֔ה וַיֵּ֥שֶׁב בְּאֶֽרֶץ־מִדְיָ֖ן וַיֵּ֥שֶׁב עַֽל־הַבְּאֵֽר׃
\rashi{\rashiDH{וישמע פרעה. }הם הלשינו עליו׃}\rashi{\rashiDH{ויבקש להרוג את משה. }מסרו לְקוֹסְטֵינָר להרגו ולא שלטה בו החרב, הוא שאמר משה, ויצילני מחרב פרעה׃ }\rashi{\rashiDH{(וישב בארץ מדין. }נתעכב שם, כמו וישב יעקב׃) }\rashi{\rashiDH{וישב על הבאר. }למד מיעקב שנזדווג לו זווגו על הבאר׃}}
{וּשְׁמַע פַּרְעֹה יָת פִּתְגָמָא הָדֵין וּבְעָא לְמִקְטַל יָת מֹשֶׁה וַעֲרַק מֹשֶׁה מִן קֳדָם פַּרְעֹה וִיתֵיב בְּאַרְעָא דְּמִדְיָן וִיתֵיב עַל בֵּירָא׃}
{Now when Pharaoh heard this thing, he sought to slay Moses. But Moses fled from the face of Pharaoh, and dwelt in the land of Midian; and he sat down by a well.}{\arabic{verse}}
\threeverse{\arabic{verse}}%Ex.2:16
{וּלְכֹהֵ֥ן מִדְיָ֖ן שֶׁ֣בַע בָּנ֑וֹת וַתָּבֹ֣אנָה וַתִּדְלֶ֗נָה וַתְּמַלֶּ֙אנָה֙ אֶת־הָ֣רְהָטִ֔ים לְהַשְׁק֖וֹת צֹ֥אן אֲבִיהֶֽן׃
\rashi{\rashiDH{ולכהן מדין. }רב שבהן, ופירש לו מעבודת אלילים ונידוהו מאצלם׃ }\rashi{\rashiDH{את הרהטים. }את בריכות מרוצות המים העשויות בארץ׃}}
{וּלְרַבָּא דְּמִדְיָן שְׁבַע בְּנָן וַאֲתַאָה וּדְלַאָה וּמְלַאָה יָת רָטַיָּא לְאַשְׁקָאָה עָנָא דַּאֲבוּהוֹן׃}
{Now the priest of Midian had seven daughters; and they came and drew water, and filled the troughs to water their father’s flock.}{\arabic{verse}}
\threeverse{\arabic{verse}}%Ex.2:17
{וַיָּבֹ֥אוּ הָרֹעִ֖ים וַיְגָרְשׁ֑וּם וַיָּ֤קׇם מֹשֶׁה֙ וַיּ֣וֹשִׁעָ֔ן וַיַּ֖שְׁקְ אֶת־צֹאנָֽם׃
\rashi{\rashiDH{ויגרשום. }מפני הנידוי׃ 
}}
{וַאֲתוֹ רָעַיָּא וּטְרַדוּנִין וְקָם מֹשֶׁה וּפְרַקִנִּין וְאַשְׁקִי יָת עָנְהוֹן׃}
{And the shepherds came and drove them away; but Moses stood up and helped them, and watered their flock.}{\arabic{verse}}
\threeverse{\arabic{verse}}%Ex.2:18
{וַתָּבֹ֕אנָה אֶל־רְעוּאֵ֖ל אֲבִיהֶ֑ן וַיֹּ֕אמֶר מַדּ֛וּעַ מִהַרְתֶּ֥ן בֹּ֖א הַיּֽוֹם׃}
{וַאֲתַאָה לְוָת רְעוּאֵל אֲבוּהוֹן וַאֲמַר מָדֵין אוֹחִיתִין לְמֵיתֵי יוֹמָא דֵין׃}
{And when they came to Reuel their father, he said: ‘How is it that ye are come so soon to-day?’}{\arabic{verse}}
\threeverse{\arabic{verse}}%Ex.2:19
{וַתֹּאמַ֕רְןָ אִ֣ישׁ מִצְרִ֔י הִצִּילָ֖נוּ מִיַּ֣ד הָרֹעִ֑ים וְגַם־דָּלֹ֤ה דָלָה֙ לָ֔נוּ וַיַּ֖שְׁקְ אֶת־הַצֹּֽאן׃}
{וַאֲמַרָא גּוּבְרָא מִצְרָאָה שֵׁיזְבַנָא מִיְּדָא דְּרָעַיָּא וְאַף מִדְלָא דְּלָא לַנָא וְאַשְׁקִי יָת עָנָא׃}
{And they said: ‘An Egyptian delivered us out of the hand of the shepherds, and moreover he drew water for us, and watered the flock.’}{\arabic{verse}}
\threeverse{\arabic{verse}}%Ex.2:20
{וַיֹּ֥אמֶר אֶל־בְּנֹתָ֖יו וְאַיּ֑וֹ לָ֤מָּה זֶּה֙ עֲזַבְתֶּ֣ן אֶת־הָאִ֔ישׁ קִרְאֶ֥ן ל֖וֹ וְיֹ֥אכַל לָֽחֶם׃
\rashi{\rashiDH{למה זה עזבתן. }הכיר בו שהוא מזרעו של יעקב, שהמים עולים לקראתו׃ }\rashi{\rashiDH{ויאכל לחם. }שמא ישא אחת מכם, כמה דאת אמר כי אם הלחם אשר הוא אוכל׃ }}
{וַאֲמַר לִבְנָתֵיהּ וְאָן הוּא לְמָא דְּנָן שְׁבַקְתִּין יָת גּוּבְרָא קְרַן לֵיהּ וְיֵיכוֹל לַחְמָא׃}
{And he said unto his daughters: ‘And where is he? Why is it that ye have left the man? call him, that he may eat bread.’}{\arabic{verse}}
\threeverse{\arabic{verse}}%Ex.2:21
{וַיּ֥וֹאֶל מֹשֶׁ֖ה לָשֶׁ֣בֶת אֶת־הָאִ֑ישׁ וַיִּתֵּ֛ן אֶת־צִפֹּרָ֥ה בִתּ֖וֹ לְמֹשֶֽׁה׃
\rashi{\rashiDH{ויואל. }כתרגומו, (ס״א כמשמעו) ודומה לו הֹואֶל נָא וְלִין (שופטים יט, ו), ולו הואלנו, הואלתי לדבר. ומדרשו לשון אָלָה, נשבע לו שלא יזוז ממדין כי אם ברשותו׃ 
}}
{וּצְבִי מֹשֶׁה לְמִתַּב עִם גּוּבְרָא וִיהַב יָת צִפּוֹרָה בְּרַתֵּיהּ לְמֹשֶׁה׃}
{And Moses was content to dwell with the man; and he gave Moses Zipporah his daughter.}{\arabic{verse}}
\threeverse{\arabic{verse}}%Ex.2:22
{וַתֵּ֣לֶד בֵּ֔ן וַיִּקְרָ֥א אֶת־שְׁמ֖וֹ גֵּרְשֹׁ֑ם כִּ֣י אָמַ֔ר גֵּ֣ר הָיִ֔יתִי בְּאֶ֖רֶץ נׇכְרִיָּֽה׃ \petucha }
{וִילֵידַת בַּר וּקְרָא יָת שְׁמֵיהּ גֵּרְשׁוֹם אֲרֵי אֲמַר דַּיָּר הֲוֵיתִי בַּאֲרַע נוּכְרָאָה׃}
{And she bore a son, and he called his name Gershom; for he said: ‘I have been a stranger in a strange land.’}{\arabic{verse}}
\threeverse{\arabic{verse}}%Ex.2:23
{וַיְהִי֩ בַיָּמִ֨ים הָֽרַבִּ֜ים הָהֵ֗ם וַיָּ֙מׇת֙ מֶ֣לֶךְ מִצְרַ֔יִם וַיֵּאָנְח֧וּ בְנֵֽי־יִשְׂרָאֵ֛ל מִן־הָעֲבֹדָ֖ה וַיִּזְעָ֑קוּ וַתַּ֧עַל שַׁוְעָתָ֛ם אֶל־הָאֱלֹהִ֖ים מִן־הָעֲבֹדָֽה׃
\rashi{\rashiDH{ויהי בימים הרבים ההם. }שהיה משה גָּר במדין, וימת מלך מצרים והוצרכו ישראל לתשועה, ומשה היה רועה וגו׳ ובאת תשועה על ידו, ולכך נסמכו פרשיות הללו. (בר״י)׃ }\rashi{\rashiDH{וימת מלך מצרים. }נצטרע, והיה שוחט תינוקות ישראל ורוחץ בדמם (שמו״ר א, לד)׃ }}
{וַהֲוָה בְּיוֹמַיָּא סַגִּיאַיָּא הָאִנּוּן וּמִית מַלְכָּא דְּמִצְרַיִם וְאִתְאָנַחוּ בְּנֵי יִשְׂרָאֵל מִן פּוּלְחָנָא דַּהֲוָה קְשֵׁי עֲלֵיהוֹן וּזְעִיקוּ וּסְלֵיקַת קְבִילַתְהוֹן לִקְדָם יְיָ מִן פּוּלְחָנָא׃}
{And it came to pass in the course of those many days that the king of Egypt died; and the children of Israel sighed by reason of the bondage, and they cried, and their cry came up unto God by reason of the bondage.}{\arabic{verse}}
\threeverse{\arabic{verse}}%Ex.2:24
{וַיִּשְׁמַ֥ע אֱלֹהִ֖ים אֶת־נַאֲקָתָ֑ם וַיִּזְכֹּ֤ר אֱלֹהִים֙ אֶת־בְּרִית֔וֹ אֶת־אַבְרָהָ֖ם אֶת־יִצְחָ֥ק וְאֶֽת־יַעֲקֹֽב׃
\rashi{\rashiDH{נאקתם. }צעקתם, וכן מֵעִיר מְתִים יִנְאָקוּ (איוב כד, יב)׃ }\rashi{\rashiDH{את בריתו את אברהם. }עם אברהם׃}}
{וּשְׁמִיעַ קֳדָם יְיָ יָת קְבִילַתְהוֹן וּדְכִיר יְיָ יָת קְיָמֵיהּ דְּעִם אַבְרָהָם דְּעִם יִצְחָק וּדְעִם יַעֲקֹב׃}
{And God heard their groaning, and God remembered His covenant with Abraham, with Isaac, and with Jacob.}{\arabic{verse}}
\threeverse{\arabic{verse}}%Ex.2:25
{וַיַּ֥רְא אֱלֹהִ֖ים אֶת־בְּנֵ֣י יִשְׂרָאֵ֑ל וַיֵּ֖דַע אֱלֹהִֽים׃ \setuma         
\rashi{\rashiDH{וידע אלהים. }נתן עליהם לב ולא העלים עיניו׃ 
}}
{וּגְלֵי קֳדָם יְיָ שִׁעְבּוּדָא דִּבְנֵי יִשְׂרָאֵל וַאֲמַר בְּמֵימְרֵיהּ לְמִפְרַקְהוֹן יְיָ׃}
{And God saw the children of Israel, and God took cognizance of them.}{\arabic{verse}}
\newperek
\threeverse{\aliya{רביעי}}%Ex.3:1
{וּמֹשֶׁ֗ה הָיָ֥ה רֹעֶ֛ה אֶת־צֹ֛אן יִתְר֥וֹ חֹתְנ֖וֹ כֹּהֵ֣ן מִדְיָ֑ן וַיִּנְהַ֤ג אֶת־הַצֹּאן֙ אַחַ֣ר הַמִּדְבָּ֔ר וַיָּבֹ֛א אֶל־הַ֥ר הָאֱלֹהִ֖ים חֹרֵֽבָה׃
\rashi{\rashiDH{אחר המדבר. }להתרחק מן הגזל, שלא ירעו בשדות אחרים׃ }\rashi{\rashiDH{אל הר האלהים. }על שם העתיד׃}}
{וּמֹשֶׁה הֲוָה רָעֵי יָת עָנָא דְּיִתְרוֹ חֲמוּהִי רַבָּא דְּמִדְיָן וְדַבַּר יָת עָנָא לַאֲתַר שְׁפַר רִעְיָא לְמַדְבְּרָא וַאֲתָא לְטוּרָא דְּאִתְגְּלִי עֲלוֹהִי יְקָרָא דַּייָ לְחוֹרֵב׃}
{Now Moses was keeping the flock of Jethro his father-in-law, the priest of Midian; and he led the flock to the farthest end of the wilderness, and came to the mountain of God, unto Horeb.}{\Roman{chap}}
\threeverse{\arabic{verse}}%Ex.3:2
{וַ֠יֵּרָ֠א מַלְאַ֨ךְ יְהֹוָ֥ה אֵלָ֛יו בְּלַבַּת־אֵ֖שׁ מִתּ֣וֹךְ הַסְּנֶ֑ה וַיַּ֗רְא וְהִנֵּ֤ה הַסְּנֶה֙ בֹּעֵ֣ר בָּאֵ֔שׁ וְהַסְּנֶ֖ה אֵינֶ֥נּוּ אֻכָּֽל׃
\rashi{\rashiDH{בלבת אש. }בשלהבת אש לבו של אש, כמו לב השמים, בְּלֵב הָאֵלָה, (שמואל־ב יח, יד) ואל תתמה על התי״ו, שיש לנו כיוצא בו, מָה אֲמֻלָה לִבָּתֵךְ (יחזקאל טז, ל)׃ 
}\rashi{\rashiDH{מתוך הסנה. }ולא אילן אחר, משום עִמֹּו אָנֹכִי בְצָרָה׃ }\rashi{\rashiDH{אכל. }נאכל, כמו לא עבד בה, אשר לקח משם׃ }}
{וְאִתְגְּלִי מַלְאֲכָא דַּייָ לֵיהּ בְּשַׁלְהוֹבִית אִישָׁתָא מִגּוֹ אֲסַנָּא וַחֲזָא וְהָא אֲסַנָּא בָּעַר בְּאִישָׁתָא וַאֲסַנָּא לָיְתוֹהִי מִתְאֲכִיל׃}
{And the angel of the \lord\space appeared unto him in a flame of fire out of the midst of a bush; and he looked, and, behold, the bush burned with fire, and the bush was not consumed.}{\arabic{verse}}
\threeverse{\arabic{verse}}%Ex.3:3
{וַיֹּ֣אמֶר מֹשֶׁ֔ה אָסֻֽרָה־נָּ֣א וְאֶרְאֶ֔ה אֶת־הַמַּרְאֶ֥ה הַגָּדֹ֖ל הַזֶּ֑ה מַדּ֖וּעַ לֹא־יִבְעַ֥ר הַסְּנֶֽה׃
\rashi{\rashiDH{אסורה נא. }אסורה מכאן להתקרב שם׃}}
{וַאֲמַר מֹשֶׁה אֶתְפְּנֵי כְּעַן וְאֶחְזֵי יָת חֶזְוָנָא רַבָּא הָדֵין מָא דֵין לָא מִתּוֹקַד אֲסַנָּא׃}
{And Moses said: ‘I will turn aside now, and see this great sight, why the bush is not burnt.’}{\arabic{verse}}
\threeverse{\arabic{verse}}%Ex.3:4
{וַיַּ֥רְא יְהֹוָ֖ה כִּ֣י סָ֣ר לִרְא֑וֹת וַיִּקְרָא֩ אֵלָ֨יו אֱלֹהִ֜ים מִתּ֣וֹךְ הַסְּנֶ֗ה וַיֹּ֛אמֶר מֹשֶׁ֥ה מֹשֶׁ֖ה וַיֹּ֥אמֶר הִנֵּֽנִי׃}
{וַחֲזָא יְיָ אֲרֵי אִתְפְּנֵי לְמִחְזֵי וּקְרָא לֵיהּ יְיָ מִגּוֹ אֲסַנָּא וַאֲמַר מֹשֶׁה מֹשֶׁה וַאֲמַר הָאֲנָא׃}
{And when the \lord\space saw that he turned aside to see, God called unto him out of the midst of the bush, and said: ‘Moses, Moses.’ And he said: ‘Here am I.’}{\arabic{verse}}
\threeverse{\arabic{verse}}%Ex.3:5
{וַיֹּ֖אמֶר אַל־תִּקְרַ֣ב הֲלֹ֑ם שַׁל־נְעָלֶ֙יךָ֙ מֵעַ֣ל רַגְלֶ֔יךָ כִּ֣י הַמָּק֗וֹם אֲשֶׁ֤ר אַתָּה֙ עוֹמֵ֣ד עָלָ֔יו אַדְמַת־קֹ֖דֶשׁ הֽוּא׃
\rashi{\rashiDH{של. }שְׁלוֹף והוצא, כמו וְנָשַל הַבַּרְזָל (דברים יט, ה), כי ישל זיתך׃ }\rashi{\rashiDH{אדמת קודש הוא. }המקום׃ 
}}
{וַאֲמַר לָא תִּקְרַב הָלְכָא שְׁרִי סֵינָךְ מֵעַל רַגְלָךְ אֲרֵי אַתְרָא דְּאַתְּ קָאֵים עֲלווֹהִי אֲתַר קַדִּישׁ הוּא׃}
{And He said: ‘Draw not nigh hither; put off thy shoes from off thy feet, for the place whereon thou standest is holy ground.’}{\arabic{verse}}
\threeverse{\arabic{verse}}%Ex.3:6
{וַיֹּ֗אמֶר אָנֹכִי֙ אֱלֹהֵ֣י אָבִ֔יךָ אֱלֹהֵ֧י אַבְרָהָ֛ם אֱלֹהֵ֥י יִצְחָ֖ק וֵאלֹהֵ֣י יַעֲקֹ֑ב וַיַּסְתֵּ֤ר מֹשֶׁה֙ פָּנָ֔יו כִּ֣י יָרֵ֔א מֵהַבִּ֖יט אֶל־הָאֱלֹהִֽים׃}
{וַאֲמַר אֲנָא אֱלָהָא דַּאֲבוּךְ אֱלָהֵיהּ דְּאַבְרָהָם אֱלָהֵיהּ דְּיִצְחָק וֵאלָהֵיהּ דְּיַעֲקֹב וּכְבַשִׁנּוּן מֹשֶׁה לְאַפּוֹהִי אֲרֵי דְּחֵיל מִלְּאִסְתַּכָּלָא בְּצֵית יְקָרָא דַּייָ׃}
{Moreover He said: ‘I am the God of thy father, the God of Abraham, the God of Isaac, and the God of Jacob.’ And Moses hid his face; for he was afraid to look upon God.}{\arabic{verse}}
\threeverse{\arabic{verse}}%Ex.3:7
{וַיֹּ֣אמֶר יְהֹוָ֔ה רָאֹ֥ה רָאִ֛יתִי אֶת־עֳנִ֥י עַמִּ֖י אֲשֶׁ֣ר בְּמִצְרָ֑יִם וְאֶת־צַעֲקָתָ֤ם שָׁמַ֙עְתִּי֙ מִפְּנֵ֣י נֹֽגְשָׂ֔יו כִּ֥י יָדַ֖עְתִּי אֶת־מַכְאֹבָֽיו׃
\rashi{\rashiDH{כי ידעתי את מכאוביו. }כמו וַיֵדַּע אֱלֹהִים, כלומר כי שמתי לב להתבונן ולדעת את מכאוביו, ולא העלמתי עיני ולא אאטום את אזני מצעקתם׃ 
}}
{וַאֲמַר יְיָ מִגְלָא גְּלֵי קֳדָמַי שִׁעְבּוּד עַמִּי דִּבְמִצְרָיִם וְיָת קְבִילַתְהוֹן שְׁמִיעַ קֳדָמַי מִן קֳדָם מַפְלְחֵיהוֹן אֲרֵי גְּלֵי קֳדָמַי כֵּיבֵיהוֹן׃}
{And the \lord\space said: ‘I have surely seen the affliction of My people that are in Egypt, and have heard their cry by reason of their taskmasters; for I know their pains;}{\arabic{verse}}
\threeverse{\arabic{verse}}%Ex.3:8
{וָאֵרֵ֞ד לְהַצִּיל֣וֹ \legarmeh  מִיַּ֣ד מִצְרַ֗יִם וּֽלְהַעֲלֹתוֹ֮ מִן־הָאָ֣רֶץ הַהִוא֒ אֶל־אֶ֤רֶץ טוֹבָה֙ וּרְחָבָ֔ה אֶל־אֶ֛רֶץ זָבַ֥ת חָלָ֖ב וּדְבָ֑שׁ אֶל־מְק֤וֹם הַֽכְּנַעֲנִי֙ וְהַ֣חִתִּ֔י וְהָֽאֱמֹרִי֙ וְהַפְּרִזִּ֔י וְהַחִוִּ֖י וְהַיְבוּסִֽי׃}
{וְאִתְגְּלִיתִי לְשֵׁיזָבוּתְהוֹן מִיְּדָא דְּמִצְרָאֵי וּלְאַסָּקוּתְהוֹן מִן אַרְעָא הַהִיא לַאֲרַע טָבָא וּפַתְיָא לַאֲרַע עָבְדָא חֲלָב וּדְבָשׁ לַאֲתַר כְּנַעֲנָאֵי וְחִתָּאֵי וֶאֱמוֹרָאֵי וּפְרִזָּאֵי וְחִוָּאֵי וִיבוּסָאֵי׃}
{and I am come down to deliver them out of the hand of the Egyptians, and to bring them up out of that land unto a good land and a large, unto a land flowing with milk and honey; unto the place of the Canaanite, and the Hittite, and the Amorite, and the Perizzite, and the Hivite, and the Jebusite.}{\arabic{verse}}
\threeverse{\arabic{verse}}%Ex.3:9
{וְעַתָּ֕ה הִנֵּ֛ה צַעֲקַ֥ת בְּנֵי־יִשְׂרָאֵ֖ל בָּ֣אָה אֵלָ֑י וְגַם־רָאִ֙יתִי֙ אֶת־הַלַּ֔חַץ אֲשֶׁ֥ר מִצְרַ֖יִם לֹחֲצִ֥ים אֹתָֽם׃}
{וּכְעַן הָא קְבִילַת בְּנֵי יִשְׂרָאֵל עַלַת לִקְדָמַי וְאַף גְּלֵי קֳדָמַי דּוּחְקָא דְּמִצְרָאֵי דָּחֲקִין לְהוֹן׃}
{And now, behold, the cry of the children of Israel is come unto Me; moreover I have seen the oppression wherewith the Egyptians oppress them.}{\arabic{verse}}
\threeverse{\arabic{verse}}%Ex.3:10
{וְעַתָּ֣ה לְכָ֔ה וְאֶֽשְׁלָחֲךָ֖ אֶל־פַּרְעֹ֑ה וְהוֹצֵ֛א אֶת־עַמִּ֥י בְנֵֽי־יִשְׂרָאֵ֖ל מִמִּצְרָֽיִם׃
\rashi{\rashiDH{ועתה לכה ואשלחך אל פרעה. }ואם תאמר מה תועיל, והוצא את עמי, יועילו דבריך ותוציאם משם׃ }}
{וּכְעַן אֵיתַא וְאֶשְׁלְחִנָּךְ לְוָת פַּרְעֹה וְאַפֵּיק יָת עַמִּי בְנֵי יִשְׂרָאֵל מִמִּצְרָיִם׃}
{Come now therefore, and I will send thee unto Pharaoh, that thou mayest bring forth My people the children of Israel out of Egypt.’}{\arabic{verse}}
\threeverse{\arabic{verse}}%Ex.3:11
{וַיֹּ֤אמֶר מֹשֶׁה֙ אֶל־הָ֣אֱלֹהִ֔ים מִ֣י אָנֹ֔כִי כִּ֥י אֵלֵ֖ךְ אֶל־פַּרְעֹ֑ה וְכִ֥י אוֹצִ֛יא אֶת־בְּנֵ֥י יִשְׂרָאֵ֖ל מִמִּצְרָֽיִם׃
\rashi{\rashiDH{מי אנכי. }מה אני חשוב לדבר עם המלכים׃}\rashi{\rashiDH{וכי אוציא את בני ישראל. }ואף אם חשוב אני, מה זכו ישראל שֶׁתַּעֲשֶׂה להם נס ואוציאם ממצרים׃ }}
{וַאֲמַר מֹשֶׁה קֳדָם יְיָ מַן אֲנָא אֲרֵי אֵיזֵיל לְוָת פַּרְעֹה וַאֲרֵי אַפֵּיק יָת בְּנֵי יִשְׂרָאֵל מִמִּצְרָיִם׃}
{And Moses said unto God: ‘Who am I, that I should go unto Pharaoh, and that I should bring forth the children of Israel out of Egypt?’}{\arabic{verse}}
\threeverse{\arabic{verse}}%Ex.3:12
{וַיֹּ֙אמֶר֙ כִּֽי־אֶֽהְיֶ֣ה עִמָּ֔ךְ וְזֶה־לְּךָ֣ הָא֔וֹת כִּ֥י אָנֹכִ֖י שְׁלַחְתִּ֑יךָ בְּהוֹצִֽיאֲךָ֤ אֶת־הָעָם֙ מִמִּצְרַ֔יִם תַּֽעַבְדוּן֙ אֶת־הָ֣אֱלֹהִ֔ים עַ֖ל הָהָ֥ר הַזֶּֽה׃
\rashi{\rashiDH{ויאמר כי אהיה עמך. }השיבו על ראשון ראשון ועל אחרון אחרון, שאמרת מי אנכי כי אלך אל פרעה, לא שלך היא, כי אם משלי, כי אהיה עמך, וזה המראה אשר ראית בסנה, לך האות כי אנכי שלחתיך, ותצליח בשליחותי וכדאי אני להציל, כאשר ראית הסנה עושה שליחותי ואיננו אֻכָּל, כך תלך בשליחותי ואינך ניזוק, וששאלת מה זכות יש לישראל שיצאו ממצרים, דבר גדול יש לי על הוצאה זו, שהרי עתידים לקבל התורה על ההר הזה לסוף שלשה חדשים שיצאו מצרים. דבר אחר כי אהיה עמך, וזה שתצליח בשליחותך, לך האות על הבטחה אחרת שאני מבטיחך, שכשתוציאם ממצרים תעבדון אותי על ההר הזה, שתקבלו התורה עליו, והיא הזכות העומדת לישראל. ודוגמת לשון זה מצינו, וְזֶה לְּךָ הָאֹות אָכֹול הַשָּׁנָה סָפִיחַ וגו׳ (ישעי׳ לז, ל), מפלת סנחריב תהיה לך לאות על הבטחה אחרת, שארצכם חריבה מפירות ואני אברך הספיחים׃ }}
{וַאֲמַר אֲרֵי יְהֵי מֵימְרִי בְּסַעֲדָךְ וְדֵין לָךְ אָתָא אֲרֵי אֲנָא שְׁלַחְתָּךְ בְּאַפָּקוּתָךְ יָת עַמָּא מִמִּצְרַיִם תִּפְלְחוּן קֳדָם יְיָ עַל טוּרָא הָדֵין׃}
{And He said: ‘Certainly I will be with thee; and this shall be the token unto thee, that I have sent thee: when thou hast brought forth the people out of Egypt, ye shall serve God upon this mountain.’}{\arabic{verse}}
\threeverse{\arabic{verse}}%Ex.3:13
{וַיֹּ֨אמֶר מֹשֶׁ֜ה אֶל־הָֽאֱלֹהִ֗ים הִנֵּ֨ה אָנֹכִ֣י בָא֮ אֶל־בְּנֵ֣י יִשְׂרָאֵל֒ וְאָמַרְתִּ֣י לָהֶ֔ם אֱלֹהֵ֥י אֲבוֹתֵיכֶ֖ם שְׁלָחַ֣נִי אֲלֵיכֶ֑ם וְאָֽמְרוּ־לִ֣י מַה־שְּׁמ֔וֹ מָ֥ה אֹמַ֖ר אֲלֵהֶֽם׃}
{וַאֲמַר מֹשֶׁה קֳדָם יְיָ הָא אֲנָא אָתֵי לְוָת בְּנֵי יִשְׂרָאֵל וְאֵימַר לְהוֹן אֱלָהָא דַּאֲבָהָתְכוֹן שַׁלְחַנִי לְוָתְכוֹן וְיֵימְרוּן לִי מַן שְׁמֵיהּ מָא אֵימַר לְהוֹן׃}
{And Moses said unto God: ‘Behold, when I come unto the children of Israel, and shall say unto them: The God of your fathers hath sent me unto you; and they shall say to me: What is His name? what shall I say unto them?’}{\arabic{verse}}
\threeverse{\arabic{verse}}%Ex.3:14
{וַיֹּ֤אמֶר אֱלֹהִים֙ אֶל־מֹשֶׁ֔ה אֶֽהְיֶ֖ה אֲשֶׁ֣ר אֶֽהְיֶ֑ה וַיֹּ֗אמֶר כֹּ֤ה תֹאמַר֙ לִבְנֵ֣י יִשְׂרָאֵ֔ל אֶֽהְיֶ֖ה שְׁלָחַ֥נִי אֲלֵיכֶֽם׃
\rashi{\rashiDH{אהיה אשר אהיה. }אהיה עמם בצרה זאת, אשר אהיה עמם בשעבוד שאר מלכיות, אמר לפניו רבש״ע מה אני מזכיר להם צרה אחרת, דיים בצרה זו, אמר לו יפה אמרת, כה תאמר וגו׳. (ברכות ט׃  שמו״ר ג, ז) (לא שהשכיל חלילה משה ביותר, אלא שלא הבין דברי השי״ת, כי לא מחשבתו מחשבת השי״ת, שמאז כך היתה דעתו באומרו יתברך אהיה אשר אהיה, למשה לבדו הגיד, ולא שיגיד לישראל, וזהו יפה אמרת, שגם דעתי מתחלה כך היתה, שלא תגיד לבני ישראל כדברים האלה, אלא כה תאמר לבני ישראל אהיה פעם אחת. וכן משמע במסכת ברכות ודו״ק)׃ }}
{וַאֲמַר יְיָ לְמֹשֶׁה אֶהְיֶה אֲשֶׁר אֶהְיֶה וַאֲמַר כִּדְנָן תֵּימַר לִבְנֵי יִשְׂרָאֵל אֶהְיֶה שַׁלְחַנִי לְוָתְכוֹן׃}
{And God said unto Moses: ‘\textsc{I Am That I Am}’; and He said: ‘Thus shalt thou say unto the children of Israel: \textsc{I Am} hath sent me unto you.’}{\arabic{verse}}
\threeverse{\arabic{verse}}%Ex.3:15
{וַיֹּ֩אמֶר֩ ע֨וֹד אֱלֹהִ֜ים אֶל־מֹשֶׁ֗ה כֹּֽה־תֹאמַר֮ אֶל־בְּנֵ֣י יִשְׂרָאֵל֒ יְהֹוָ֞ה אֱלֹהֵ֣י אֲבֹתֵיכֶ֗ם אֱלֹהֵ֨י אַבְרָהָ֜ם אֱלֹהֵ֥י יִצְחָ֛ק וֵאלֹהֵ֥י יַעֲקֹ֖ב שְׁלָחַ֣נִי אֲלֵיכֶ֑ם זֶה־שְּׁמִ֣י לְעֹלָ֔ם וְזֶ֥ה זִכְרִ֖י לְדֹ֥ר דֹּֽר׃
\rashi{\rashiDH{זה שמי לעלם. }חסר וי״ו לומר, העלימהו, שלא יקרא ככתבו (שמו״ר ג, ט)׃ }\rashi{\rashiDH{וזה זכרי. }למדו היאך נקרא, וכן דוד הוא אומר, ה׳ שמך לעולם ה׳ זכרך לדור ודור׃ }}
{וַאֲמַר עוֹד יְיָ לְמֹשֶׁה כִּדְנָן תֵּימַר לִבְנֵי יִשְׂרָאֵל יְיָ אֱלָהָא דַּאֲבָהָתְכוֹן אֱלָהֵיהּ דְּאַבְרָהָם אֱלָהֵיהּ דְּיִצְחָק וֵאלָהֵיהּ דְּיַעֲקֹב שַׁלְחַנִי לְוָתְכוֹן דֵּין שְׁמִי לְעָלַם וְדֵין דּוּכְרָנִי לְכָל דָר וְדָר׃}
{And God said moreover unto Moses: ‘Thus shalt thou say unto the children of Israel: The \lord, the God of your fathers, the God of Abraham, the God of Isaac, and the God of Jacob, hath sent me unto you; this is My name for ever, and this is My memorial unto all generations.}{\arabic{verse}}
\threeverse{\aliya{חמישי}}%Ex.3:16
{לֵ֣ךְ וְאָֽסַפְתָּ֞ אֶת־זִקְנֵ֣י יִשְׂרָאֵ֗ל וְאָמַרְתָּ֤ אֲלֵהֶם֙ יְהֹוָ֞ה אֱלֹהֵ֤י אֲבֹֽתֵיכֶם֙ נִרְאָ֣ה אֵלַ֔י אֱלֹהֵ֧י אַבְרָהָ֛ם יִצְחָ֥ק וְיַעֲקֹ֖ב לֵאמֹ֑ר פָּקֹ֤ד פָּקַ֙דְתִּי֙ אֶתְכֶ֔ם וְאֶת־הֶעָשׂ֥וּי לָכֶ֖ם בְּמִצְרָֽיִם׃
\rashi{\rashiDH{את זקני ישראל. }מיוחדים לישיבה. ואם תאמר זקנים סתם, היאך אפשר לו לאסוף זקנים של ס׳ רבוא׃ }}
{אִיזֵיל וְתִכְנוֹשׁ יָת סָבֵי יִשְׂרָאֵל וְתֵימַר לְהוֹן יְיָ אֱלָהָא דַּאֲבָהָתְכוֹן אִתְגְּלִי לִי אֱלָהֵיהּ דְּאַבְרָהָם יִצְחָק וְיַעֲקֹב לְמֵימַר מִדְכָר דְּכִירְנָא יָתְכוֹן וְיָת דְּאִתְעֲבֵיד לְכוֹן בְּמִצְרָיִם׃}
{Go, and gather the elders of Israel together, and say unto them: The \lord, the God of your fathers, the God of Abraham, of Isaac, and of Jacob, hath appeared unto me, saying: I have surely remembered you, and seen that which is done to you in Egypt.}{\arabic{verse}}
\threeverse{\arabic{verse}}%Ex.3:17
{וָאֹמַ֗ר אַעֲלֶ֣ה אֶתְכֶם֮ מֵעֳנִ֣י מִצְרַ֒יִם֒ אֶל־אֶ֤רֶץ הַֽכְּנַעֲנִי֙ וְהַ֣חִתִּ֔י וְהָֽאֱמֹרִי֙ וְהַפְּרִזִּ֔י וְהַחִוִּ֖י וְהַיְבוּסִ֑י אֶל־אֶ֛רֶץ זָבַ֥ת חָלָ֖ב וּדְבָֽשׁ׃}
{וַאֲמַרִית אַסֵּיק יָתְכוֹן מִשִּׁעְבּוּד מִצְרָאֵי לַאֲרַע כְּנַעֲנָאֵי וְחִתָּאֵי וֶאֱמוֹרָאֵי וּפְרִזָּאֵי וְחִוָּאֵי וִיבוּסָאֵי לַאֲרַע עָבְדָא חֲלָב וּדְבָשׁ׃}
{And I have said: I will bring you up out of the affliction of Egypt unto the land of the Canaanite, and the Hittite, and the Amorite, and the Perizzite, and the Hivite, and the Jebusite, unto a land flowing with milk and honey.}{\arabic{verse}}
\threeverse{\arabic{verse}}%Ex.3:18
{וְשָׁמְע֖וּ לְקֹלֶ֑ךָ וּבָאתָ֡ אַתָּה֩ וְזִקְנֵ֨י יִשְׂרָאֵ֜ל אֶל־מֶ֣לֶךְ מִצְרַ֗יִם וַאֲמַרְתֶּ֤ם אֵלָיו֙ יְהֹוָ֞ה אֱלֹהֵ֤י הָֽעִבְרִיִּים֙ נִקְרָ֣ה עָלֵ֔ינוּ וְעַתָּ֗ה נֵֽלְכָה־נָּ֞א דֶּ֣רֶךְ שְׁלֹ֤שֶׁת יָמִים֙ בַּמִּדְבָּ֔ר וְנִזְבְּחָ֖ה לַֽיהֹוָ֥ה אֱלֹהֵֽינוּ׃
\rashi{\rashiDH{ושמעו לקולך. }מאליהם, מכיון שתאמר להם לשון זה ישמעו לקולך, שכבר סימן זה מסור בידם מיעקב ומיוסף שבלשון זה הם נגאלים, יעקב אמר להם ואלהים פקוד יפקוד אתכם, יוסף אמר להם פָּקֹד יִפְקֹד אֶלֹהִים אֶתְכֶם (בראשית נ, כה)׃ }\rashi{\rashiDH{נקרה עלינו. }לשון מקרה, וכן וַיִּקָּר אֶלֹהִים (במדבר כג, ד), ואנכי אקרה כה, אהא נקרה מאתו הלום׃}\rashi{\rashiDH{(אלהי העבריים. }יו״ד יתירה, רמז לי׳ מכות. ברש״י ישן)׃  }}
{וִיקַבְּלוּן מִנָּךְ וְתֵיתֵי אַתְּ וְסָבֵי יִשְׂרָאֵל לְוָת מַלְכָּא דְּמִצְרַיִם וְתֵימְרוּן לֵיהּ יְיָ אֱלָהָא דִּיהוּדָאֵי אִתְקְרִי עֲלַנָא וּכְעַן נֵיזֵיל כְּעַן מַהְלַךְ תְּלָתָא יוֹמִין בְּמַדְבְּרָא וּנְדַבַּח קֳדָם יְיָ אֱלָהַנָא׃}
{And they shall hearken to thy voice. And thou shalt come, thou and the elders of Israel, unto the king of Egypt, and ye shall say unto him: The \lord, the God of the Hebrews, hath met with us. And now let us go, we pray thee, three days’ journey into the wilderness, that we may sacrifice to the \lord\space our God.}{\arabic{verse}}
\threeverse{\arabic{verse}}%Ex.3:19
{וַאֲנִ֣י יָדַ֔עְתִּי כִּ֠י לֹֽא־יִתֵּ֥ן אֶתְכֶ֛ם מֶ֥לֶךְ מִצְרַ֖יִם לַהֲלֹ֑ךְ וְלֹ֖א בְּיָ֥ד חֲזָקָֽה׃
\rashi{\rashiDH{לא יתן אתכם מלך מצרים להלוך. }אם אין אני מראה לו ידי החזקה, כלומר כל עוד שאין אני מודיעו ידי החזקה לא יתן אתכם להלוך׃ }\rashi{\rashiDH{לא יתן. }לָא יִשְׁבּוֹק, כמו עַל כֵּן לֹא נְתַתִּיךָ (בראשית כ, ו), לֹא נְתָנֹו אֱלֹהִים לְהָרַע עִמָּדִי (שם לא, ח), וכלן לשון נתינה הם. וי״מ ולא ביד חזקה. ולא בשביל שידו חזקה כי מאז אשלח את ידי והכיתי את מצרים וגו׳ ומתרגמינן אותו וְלָא מִן קֳדָם דְּחֵילֵהּ תַּקִּיף. משמו של רבי יעקב ברבי מנחם נאמר לי׃ }}
{וּקְדָמַי גְּלֵי אֲרֵי לָא יִשְׁבּוֹק יָתְכוֹן מַלְכָּא דְּמִצְרַיִם לְמֵיזַל וְלָא מִן קֳדָם דְּחֵילֵיהּ תַּקִּיף׃}
{And I know that the king of Egypt will not give you leave to go, except by a mighty hand.}{\arabic{verse}}
\threeverse{\arabic{verse}}%Ex.3:20
{וְשָׁלַחְתִּ֤י אֶת־יָדִי֙ וְהִכֵּיתִ֣י אֶת־מִצְרַ֔יִם בְּכֹל֙ נִפְלְאֹתַ֔י אֲשֶׁ֥ר אֶֽעֱשֶׂ֖ה בְּקִרְבּ֑וֹ וְאַחֲרֵי־כֵ֖ן יְשַׁלַּ֥ח אֶתְכֶֽם׃}
{וְאֶשְׁלַח יָת מַחַת גְּבוּרְתִי וְאֶמְחֵי יָת מִצְרָאֵי בְּכֹל פְּרִישָׁתַי דְּאַעֲבֵיד בֵּינֵיהוֹן וּבָתַר כֵּן יְשַׁלַּח יָתְכוֹן׃}
{And I will put forth My hand, and smite Egypt with all My wonders which I will do in the midst thereof. And after that he will let you go.}{\arabic{verse}}
\threeverse{\arabic{verse}}%Ex.3:21
{וְנָתַתִּ֛י אֶת־חֵ֥ן הָֽעָם־הַזֶּ֖ה בְּעֵינֵ֣י מִצְרָ֑יִם וְהָיָה֙ כִּ֣י תֵֽלֵכ֔וּן לֹ֥א תֵלְכ֖וּ רֵיקָֽם׃}
{וְאֶתֵּין יָת עַמָּא הָדֵין לְרַחֲמִין בְּעֵינֵי מִצְרָאֵי וִיהֵי אֲרֵי תְּהָכוּן לָא תְּהָכוּן רֵיקָנִין׃}
{And I will give this people favour in the sight of the Egyptians. And it shall come to pass, that, when ye go, ye shall not go empty;}{\arabic{verse}}
\threeverse{\arabic{verse}}%Ex.3:22
{וְשָׁאֲלָ֨ה אִשָּׁ֤ה מִשְּׁכֶנְתָּהּ֙ וּמִגָּרַ֣ת בֵּיתָ֔הּ כְּלֵי־כֶ֛סֶף וּכְלֵ֥י זָהָ֖ב וּשְׂמָלֹ֑ת וְשַׂמְתֶּ֗ם עַל־בְּנֵיכֶם֙ וְעַל־בְּנֹ֣תֵיכֶ֔ם וְנִצַּלְתֶּ֖ם אֶת־מִצְרָֽיִם׃
\rashi{\rashiDH{ומגרת ביתה. }מאותה שהיא גרה אִתָּהּ בבית׃}\rashi{\rashiDH{ונצלתם. }כתרגומו וּתְרוֹקִנּוּן, וכן וַיְנַצְּלוּ אֶת מִצְרָיִם (שמות יב, לו), וַיִּתְנַצְּלוּ בְּנֵי יִשְֹרָאֵל אֶת עֶדְיָם (שם לג, ו), והנו״ן בו יסוד. ומנחם חברו במחברת צד״י, עם וַיַצֵּל אֱלֹהִים אֶת מִקְנֵה אֲבִיכֶם (בראשית לא, ט), אשר הציל אלהים מאבינו, ולא יאמנו דבריו, כי אם לא היתה הנו״ן יסוד והיא נקודה בחיר״ק, לא תהא משמשת בלשון ופעלתם, אלא בלשון ונפעלתם, כמו וְנִסַּחְתֶּם מֵעַל הָאֲדָמָה (דברים כח, סג), ונתתם ביד אויב, וְנִגַּפְתֶּם לִפְנֵי אֹיְבֵיכֶם (ויקרא כו, יז), וְנִתַּכְתֶּם בְּתֹוכָהּ (יחזקאל כב, כא), ואמרתם נצלנו, לשון נפעלנו, וכל נו״ן שהיא באה בתיבה לפרקים, ונופלת ממנה, כנו״ן של נוגף, נושא, נותן, נושך, כשהיא מדברת לשון ופעלתם, תנקד בשו״א בחטף, כגון וּנְשָֹאתֶם אֶת אֲבִיכֶם (בראשית מה, יט), וּנְתַתֶּם לָהֶם אֶת אֶרֶץ הַגִּלְעָד (במדבר לב, כט), ונמלתם את בשר ערלתכם. לכן אני אומר, שזאת הנקודה בחיר״ק מן היסוד היא, ויסוד שם דבר נצול, והוא מן הלשונות הכבדים, כמו דִּבּוּר, כִּפּוּר, לִמּוּד, כשידבר בלשון ופעלתם ינקד בחיר״ק, כמו וְדִבַּרְתֶּם אֶל הַסֶּלַע (שם כ, ח), וְכִפַּרְתֶּם אֶת הַבָּית (יחזקאל מה, כ), וְלִמַּדְתֶּם אֹתָם אֶת בְּנֵיכֶם (דברים יא, יט)׃ 
}}
{וְתִשְׁאַל אִתְּתָא מִשֵּׁיבָבְתַהּ וּמִקָּרִיבַת בֵּיתַהּ מָנִין דִּכְסַף וּמָנִין דִּדְהַב וּלְבוּשִׁין וּתְשַׁוּוֹן עַל בְּנֵיכוֹן וְעַל בְּנָתְכוֹן וּתְרוֹקְנוּן יָת מִצְרָיִם׃}
{but every woman shall ask of her neighbour, and of her that sojourneth in her house, jewels of silver, and jewels of gold, and raiment; and ye shall put them upon your sons, and upon your daughters; and ye shall spoil the Egyptians.’}{\arabic{verse}}
\newperek
\threeverse{\Roman{chap}}%Ex.4:1
{וַיַּ֤עַן מֹשֶׁה֙ וַיֹּ֔אמֶר וְהֵן֙ לֹֽא־יַאֲמִ֣ינוּ לִ֔י וְלֹ֥א יִשְׁמְע֖וּ בְּקֹלִ֑י כִּ֣י יֹֽאמְר֔וּ לֹֽא־נִרְאָ֥ה אֵלֶ֖יךָ יְהֹוָֽה׃}
{וַאֲתֵיב מֹשֶׁה וַאֲמַר וְהָא לָא יְהֵימְנוּן לִי וְלָא יְקַבְּלוּן מִנִּי אֲרֵי יֵימְרוּן לָא אִתְגְּלִי לָךְ יְיָ׃}
{And Moses answered and said: ‘But, behold, they will not believe me, nor hearken unto my voice; for they will say: The lord hath not appeared unto thee.’}{\Roman{chap}}
\threeverse{\arabic{verse}}%Ex.4:2
{וַיֹּ֧אמֶר אֵלָ֛יו יְהֹוָ֖ה \qk{מַה־זֶּ֣ה}{מזה} בְיָדֶ֑ךָ וַיֹּ֖אמֶר מַטֶּֽה׃
\rashi{\rashiDH{מזה בידך. }לכך נכתב תיבה אחת לדרוש, מִזֶּה שבידך אתה חייב ללקות, שחשדת בכשרים. ופשוטו, כאדם שאומר לחבירו, מודה אתה שזו שלפניך אבן היא, אומר לו הן, אמר לו הריני עושה אותה עץ׃ }}
{וַאֲמַר לֵיהּ יְיָ מָא דֵּין בִּידָךְ וַאֲמַר חוּטְרָא׃}
{And the \lord\space said unto him: ‘What is that in thy hand?’ And he said: ‘A rod.’}{\arabic{verse}}
\threeverse{\arabic{verse}}%Ex.4:3
{וַיֹּ֙אמֶר֙ הַשְׁלִיכֵ֣הוּ אַ֔רְצָה וַיַּשְׁלִכֵ֥הוּ אַ֖רְצָה וַיְהִ֣י לְנָחָ֑שׁ וַיָּ֥נׇס מֹשֶׁ֖ה מִפָּנָֽיו׃
\rashi{\rashiDH{ויהי לנחש. }רמז לו שסיפר לשון הרע על ישראל (באומרו לא יאמינו לי), ותפש אומנתו של נחש׃ }}
{וַאֲמַר רְמוֹהִי לְאַרְעָא וּרְמָהִי לְאַרְעָא וַהֲוָה לְחִוְיָא וַעֲרַק מֹשֶׁה מִן קֳדָמוֹהִי׃}
{And He said: ‘Cast it on the ground.’ And he cast it on the ground, and it became a serpent; and Moses fled from before it.}{\arabic{verse}}
\threeverse{\arabic{verse}}%Ex.4:4
{וַיֹּ֤אמֶר יְהֹוָה֙ אֶל־מֹשֶׁ֔ה שְׁלַח֙ יָֽדְךָ֔ וֶאֱחֹ֖ז בִּזְנָב֑וֹ וַיִּשְׁלַ֤ח יָדוֹ֙ וַיַּ֣חֲזֶק בּ֔וֹ וַיְהִ֥י לְמַטֶּ֖ה בְּכַפּֽוֹ׃
\rashi{\rashiDH{ויחזק בו. }לשון אחיזה הוא, והרבה יש במקרא, וַיַחֲזִקוּ הָאֲנָֹשִים בְּיָדֹו (בראשית יט, טז) וְהֶחֱזִיקָה בִּמְבֹֻשָיו (דברים כה, יא), וְהֶחֱזַקְתִּי בִּזְקָנֹו (שמואל־א יז, לה), כל לשון חזוק הדבוק לבי״ת, לשון אחיזה הוא׃ 
}}
{וַאֲמַר יְיָ לְמֹשֶׁה אוֹשֵׁיט יְדָךְ וְאוֹחֵיד בְּדַנְבֵּיהּ וְאוֹשֵׁיט יְדֵיהּ וְאַתְקֵיף בֵּיהּ וַהֲוָה לְחוּטְרָא בִּידֵיהּ׃}
{And the \lord\space said unto Moses: ‘Put forth thy hand, and take it by the tail—and he put forth his hand, and laid hold of it, and it became a rod in his hand—}{\arabic{verse}}
\threeverse{\arabic{verse}}%Ex.4:5
{לְמַ֣עַן יַאֲמִ֔ינוּ כִּֽי־נִרְאָ֥ה אֵלֶ֛יךָ יְהֹוָ֖ה אֱלֹהֵ֣י אֲבֹתָ֑ם אֱלֹהֵ֧י אַבְרָהָ֛ם אֱלֹהֵ֥י יִצְחָ֖ק וֵאלֹהֵ֥י יַעֲקֹֽב׃}
{בְּדִיל דִּיהֵימְנוּן אֲרֵי אִתְגְּלִי לָךְ יְיָ אֱלָהָא דַּאֲבָהָתְהוֹן אֱלָהֵיהּ דְּאַבְרָהָם אֱלָהֵיהּ דְּיִצְחָק וֵאלָהֵיהּ דְּיַעֲקֹב׃}
{that they may believe that the \lord, the God of their fathers, the God of Abraham, the God of Isaac, and the God of Jacob, hath appeared unto thee.’}{\arabic{verse}}
\threeverse{\arabic{verse}}%Ex.4:6
{וַיֹּ֩אמֶר֩ יְהֹוָ֨ה ל֜וֹ ע֗וֹד הָֽבֵא־נָ֤א יָֽדְךָ֙ בְּחֵיקֶ֔ךָ וַיָּבֵ֥א יָד֖וֹ בְּחֵיק֑וֹ וַיּ֣וֹצִאָ֔הּ וְהִנֵּ֥ה יָד֖וֹ מְצֹרַ֥עַת כַּשָּֽׁלֶג׃
\rashi{\rashiDH{מצורעת כשלג. }דרך צרעת להיות לבנה, אם בהרת לבנה היא, אף באות זה רמז לו שלשון הרע סִיפֵּר באומרו לא יאמינו לי, לפיכך הלקהו בצרעת, כמו שלקתה מרים על לשון הרע׃ }}
{וַאֲמַר יְיָ לֵיהּ עוֹד אַעֵיל כְּעַן יְדָךְ בְּעִטְפָךְ וְאַעֵיל יְדֵיהּ בְּעִטְפֵיהּ וְאַפְּקַהּ וְהָא יְדֵיהּ חָוְרָא כְּתַלְגָּא׃}
{And the \lord\space said furthermore unto him: ‘Put now thy hand into thy bosom.’ And he put his hand into his bosom; and when he took it out, behold, his hand was leprous, as white as snow.}{\arabic{verse}}
\threeverse{\arabic{verse}}%Ex.4:7
{וַיֹּ֗אמֶר הָשֵׁ֤ב יָֽדְךָ֙ אֶל־חֵיקֶ֔ךָ וַיָּ֥שֶׁב יָד֖וֹ אֶל־חֵיק֑וֹ וַיּֽוֹצִאָהּ֙ מֵֽחֵיק֔וֹ וְהִנֵּה־שָׁ֖בָה כִּבְשָׂרֽוֹ׃
\rashi{\rashiDH{ויוציאה מחיקו והנה שבה וגו׳. }מכאן, שמדה טובה ממהרת לבא ממדת פורעניות, שהרי בראשונה לא נאמר מחיקו (שמו״ר ג, יח)׃ }}
{וַאֲמַר אֲתֵיב יְדָךְ לְעִטְפָךְ וַאֲתֵיב יְדֵיהּ לְעִטְפֵיהּ וְאַפְּקַהּ מֵעִטְפֵיהּ וְהָא תַּבַת הֲוָת כְּבִשְׂרֵיהּ׃}
{And He said: ‘Put thy hand back into thy bosom.—And he put his hand back into his bosom; and when he took it out of his bosom, behold, it was turned again as his other flesh.—}{\arabic{verse}}
\threeverse{\arabic{verse}}%Ex.4:8
{וְהָיָה֙ אִם־לֹ֣א יַאֲמִ֣ינוּ לָ֔ךְ וְלֹ֣א יִשְׁמְע֔וּ לְקֹ֖ל הָאֹ֣ת הָרִאשׁ֑וֹן וְהֶֽאֱמִ֔ינוּ לְקֹ֖ל הָאֹ֥ת הָאַחֲרֽוֹן׃
\rashi{\rashiDH{והאמינו לקול האות האחרון. }משתאמר להם בשבילכם לקיתי על שספרתי עליכם לשון הרע, יאמינו לך, שכבר למדו בכך שהמזדווגין להרע להם לוקים בנגעים, כגון פרעה ואבימלך בשביל שרה׃ }}
{וִיהֵי אִם לָא יְהֵימְנוּן לָךְ וְלָא יְקַבְּלוּן לְקָל אָתָא קַדְמָאָה וִיהֵימְנוּן לְקָל אָתָא בָּתְרָאָה׃}
{And it shall come to pass, if they will not believe thee, neither hearken to the voice of the first sign, that they will believe the voice of the latter sign.}{\arabic{verse}}
\threeverse{\arabic{verse}}%Ex.4:9
{וְהָיָ֡ה אִם־לֹ֣א יַאֲמִ֡ינוּ גַּם֩ לִשְׁנֵ֨י הָאֹת֜וֹת הָאֵ֗לֶּה וְלֹ֤א יִשְׁמְעוּן֙ לְקֹלֶ֔ךָ וְלָקַחְתָּ֙ מִמֵּימֵ֣י הַיְאֹ֔ר וְשָׁפַכְתָּ֖ הַיַּבָּשָׁ֑ה וְהָי֤וּ הַמַּ֙יִם֙ אֲשֶׁ֣ר תִּקַּ֣ח מִן־הַיְאֹ֔ר וְהָי֥וּ לְדָ֖ם בַּיַּבָּֽשֶׁת׃
\rashi{\rashiDH{ולקחת ממימי היאור. }רמז להם שבמכה ראשונה נפרע מאלהותם, (פירוש, כשהקב״ה נפרע מן האומות, נפרע מאלהותם תחלה, שהיו עובדים לנילוס המחיה אותם, והפכם לדם. ברש״י ישן)׃}\rashi{\rashiDH{והיו המים וגו׳. }והיו, והיו, שני פעמים, נראה בעיני, אלו נאמר והיו המים אשר תקח מן היאור לדם ביבשת, שומע אני שבידו הם נהפכים לדם, ואז כשירדו לארץ יהיו בהוייתן, אבל עכשיו מלמדנו, שלא יהיו דם עד שיהיו ביבשת׃  }}
{וִיהֵי אִם לָא יְהֵימְנוּן אַף לִתְרֵין אָתַיָּא הָאִלֵּין וְלָא יְקַבְּלוּן מִנָּךְ וְתִסַּב מִמַּיָּא דִּבְנַהְרָא וְתֵישׁוֹד לְיַבֶּשְׁתָּא וִיהוֹן מַיָּא דְּתִסַּב מִן נַהְרָא וִיהוֹן לִדְמָא בְּיַבֶּשְׁתָּא׃}
{And it shall come to pass, if they will not believe even these two signs, neither hearken unto thy voice, that thou shalt take of the water of the river, and pour it upon the dry land; and the water which thou takest out of the river shall become blood upon the dry land.’}{\arabic{verse}}
\threeverse{\arabic{verse}}%Ex.4:10
{וַיֹּ֨אמֶר מֹשֶׁ֣ה אֶל־יְהֹוָה֮ בִּ֣י אֲדֹנָי֒ לֹא֩ אִ֨ישׁ דְּבָרִ֜ים אָנֹ֗כִי גַּ֤ם מִתְּמוֹל֙ גַּ֣ם מִשִּׁלְשֹׁ֔ם גַּ֛ם מֵאָ֥ז דַּבֶּרְךָ֖ אֶל־עַבְדֶּ֑ךָ כִּ֧י כְבַד־פֶּ֛ה וּכְבַ֥ד לָשׁ֖וֹן אָנֹֽכִי׃
\rashi{\rashiDH{גם מתמול וגו׳. }למדנו שכל שבעה ימים היה הקב״ה מפתה את משה בסנה לילך בשליחותו, מתמול שלשום מאז דברך הרי שלשה, ושלשה גמין רבויין הם, הרי ששה, והוא היה עומד ביום הז׳ כשאמר לו זאת, עוד שלח נא ביד תשלח, עד שחרה בו וקבל עליו. (שמו״ר ג, טז) וכל זה, שלא היה רוצה ליטול גדולה על אהרן אחיו שהיה גדול הימנו, ונביא היה, שנאמר (הלא אהרן אחיך הלוי וגו׳, ועוד נאמר לעלי הכהן) הֲנִגְלֹה נִגְלֵיתִי אֶל בֵּית אָבִיךָ בְּהיֹותָם בְּמִצְרַיִם (שמואל־א ב, כז), הוא אהרן, וכן וָאִוָּדַע לָהֶם בְּאֶרֶץ מִצְרָיִם וגו׳ (יחזקאל כ, ה) ואָֹמַר אֲלְיהֶם אִיש שִקּוּצֵי עֵינָיו הַֹשְלִיכוּ, ואותה נבואה לאהרן נאמרה׃ }\rashi{\rashiDH{כבד פה. }בכבידות אני מדבר, ובלשון לע״ז בלב״ו }}
{וַאֲמַר מֹשֶׁה קֳדָם יְיָ בְּבָעוּ יְיָ לָא גְּבַר דְּמִלּוּל אֲנָא אַף מֵאֶתְמָלִי אַף מִדְּקַמּוֹהִי אַף מֵעִדָּן דְּמַלֵּילְתָּא עִם עַבְדָּךְ אֲרֵי יַקִּיר מַמְלַל וְעַמִּיק לִישָׁן אֲנָא׃}
{And Moses said unto the \lord: ‘Oh Lord, I am not a man of words, neither heretofore, nor since Thou hast spoken unto Thy servant; for I am slow of speech, and of a slow tongue.’}{\arabic{verse}}
\threeverse{\arabic{verse}}%Ex.4:11
{וַיֹּ֨אמֶר יְהֹוָ֜ה אֵלָ֗יו מִ֣י שָׂ֣ם פֶּה֮ לָֽאָדָם֒ א֚וֹ מִֽי־יָשׂ֣וּם אִלֵּ֔ם א֣וֹ חֵרֵ֔שׁ א֥וֹ פִקֵּ֖חַ א֣וֹ עִוֵּ֑ר הֲלֹ֥א אָנֹכִ֖י יְהֹוָֽה׃
\rashi{\rashiDH{מי שם פה וגו׳. }מי למדך לְדַבֵּר כשהיית נדון לפני פרעה על המצרי׃}\rashi{\rashiDH{או מי ישום אלם. }מי עשה פרעה אלם שלא נתאמץ במצות הריגתך, ואת משרתיו חרשים שלא שמעו בצוותו עליך, וְלָאִסְפַּקְלָטוֹרִין (שבת קח.) ההורגים מי עשאם עִוְרִים, שלא ראו כשברחת מן הבימה ונמלטת (תנחומא שמות י)׃ }\rashi{\rashiDH{הלא אנכי. }ששמי ה׳ עשיתי כל זאת׃}}
{וַאֲמַר יְיָ לֵיהּ מַן שַׁוִּי פֻּמָּא לַאֲנָשָׁא אוֹ מַן שַׁוִּי אִלֵּימָא אוֹ חֶרְשָׁא אוֹ פְתִיחָא אוֹ עֲוִירָא הֲלָא אֲנָא יְיָ׃}
{And the \lord\space said unto him: ‘Who hath made man’s mouth? or who maketh a man dumb, or deaf, or seeing, or blind? is it not I the \lord?}{\arabic{verse}}
\threeverse{\arabic{verse}}%Ex.4:12
{וְעַתָּ֖ה לֵ֑ךְ וְאָנֹכִי֙ אֶֽהְיֶ֣ה עִם־פִּ֔יךָ וְהוֹרֵיתִ֖יךָ אֲשֶׁ֥ר תְּדַבֵּֽר׃}
{וּכְעַן אִיזֵיל וּמֵימְרִי יְהֵי עִם פֻּמָּךְ וְאַלְּפִנָּךְ דִּתְמַלֵּיל׃}
{Now therefore go, and I will be with thy mouth, and teach thee what thou shalt speak.’}{\arabic{verse}}
\threeverse{\arabic{verse}}%Ex.4:13
{וַיֹּ֖אמֶר בִּ֣י אֲדֹנָ֑י שְֽׁלַֽח־נָ֖א בְּיַד־תִּשְׁלָֽח׃
\rashi{\rashiDH{ביד תשלח. }ביד מי שאתה רגיל לשלוח והוא אהרן. דבר אחר, ביד אחר שתרצה לשלוח, שאין סופי להכניסם לארץ ולהיות גואלם לעתיד, יש לך שלוחים הרבה׃ }}
{וַאֲמַר בְּבָעוּ יְיָ שְׁלַח כְּעַן בְּיַד מַן דְּכָשַׁר לְמִשְׁלַח׃}
{And he said: ‘Oh Lord, send, I pray Thee, by the hand of him whom Thou wilt send.’}{\arabic{verse}}
\threeverse{\arabic{verse}}%Ex.4:14
{וַיִּֽחַר־אַ֨ף יְהֹוָ֜ה בְּמֹשֶׁ֗ה וַיֹּ֙אמֶר֙ הֲלֹ֨א אַהֲרֹ֤ן אָחִ֙יךָ֙ הַלֵּוִ֔י יָדַ֕עְתִּי כִּֽי־דַבֵּ֥ר יְדַבֵּ֖ר ה֑וּא וְגַ֤ם הִנֵּה־הוּא֙ יֹצֵ֣א לִקְרָאתֶ֔ךָ וְרָאֲךָ֖ וְשָׂמַ֥ח בְּלִבּֽוֹ׃
\rashi{ויחר אף. (זבחים קב.) רבי יהושע בן קרחה אומר, כל חרון אף שבתורה נאמר בו רושם, וזה לא נאמר בו רושם, ולא מצינו שבא עונש על ידי אותו חרון, אמר לו רבי יוסי אף בזו נאמר בו רושם, הלא אהרן אחיך הלוי, שהיה עתיד להיות לוי ולא כהן, והכהונה הייתי אומר לצאת ממך, מעתה לא יהיה כן, אלא הוא יהיה כהן ואתה הלוי, שנאמר וּמשֶׁה אִישׁ הָאֱלֹהִים בָּנָיו יִקָּרְאוּ עַל שֵׁבֶט הַלֵּוִי (דברי הימים־א כג, יד)׃ }\rashi{\rashiDH{הנה הוא יצא לקראתך. }כשתלך למצרים׃ 
}\rashi{\rashiDH{וראך ושמח בלבו. }לא כשאתה סבור שיהא מקפיד עליך שאתה עולה לגדולה, ומשם זכה אהרן לַעֲדִי החשן הנתון על הלב׃ }}
{וּתְקֵיף רוּגְזָא דַּייָ בְּמֹשֶׁה וַאֲמַר הֲלָא אַהֲרֹן אֲחוּךְ לֵיוָאָה גְּלֵי קֳדָמַי אֲרֵי מַלָּלָא יְמַלֵּיל הוּא וְאַף הָא הוּא נָפֵיק לְקַדָּמוּתָךְ וְיִחְזֵינָךְ וְיִחְדֵּי בְּלִבֵּיהּ׃}
{And the anger of the \lord\space was kindled against Moses, and He said: ‘Is there not Aaron thy brother the Levite? I know that he can speak well. And also, behold, he cometh forth to meet thee; and when he seeth thee, he will be glad in his heart.}{\arabic{verse}}
\threeverse{\arabic{verse}}%Ex.4:15
{וְדִבַּרְתָּ֣ אֵלָ֔יו וְשַׂמְתָּ֥ אֶת־הַדְּבָרִ֖ים בְּפִ֑יו וְאָנֹכִ֗י אֶֽהְיֶ֤ה עִם־פִּ֙יךָ֙ וְעִם־פִּ֔יהוּ וְהוֹרֵיתִ֣י אֶתְכֶ֔ם אֵ֖ת אֲשֶׁ֥ר תַּעֲשֽׂוּן׃}
{וּתְמַלֵּיל עִמֵּיהּ וּתְשַׁוֵּי יָת פִּתְגָמַיָּא בְּפֻמֵּיהּ וּמֵימְרִי יְהֵי עִם פֻּמָּךְ וְעִם פֻּמֵּיהּ וְאַלֵּיף יָתְכוֹן יָת דְּתַעְבְּדוּן׃}
{And thou shalt speak unto him, and put the words in his mouth; and I will be with thy mouth, and with his mouth, and will teach you what ye shall do.}{\arabic{verse}}
\threeverse{\arabic{verse}}%Ex.4:16
{וְדִבֶּר־ה֥וּא לְךָ֖ אֶל־הָעָ֑ם וְהָ֤יָה הוּא֙ יִֽהְיֶה־לְּךָ֣ לְפֶ֔ה וְאַתָּ֖ה תִּֽהְיֶה־לּ֥וֹ לֵֽאלֹהִֽים׃
\rashi{\rashiDH{ודבר הוא לך. }בשבילך ידבר אל העם, וזה יוכיח על כל לך ולי ולו ולכם ולהם הסמוכים לדבור, שכולם לשון על הם׃ }\rashi{\rashiDH{יהיה לך לפה. }למליץ, לפי שאתה כבד פה׃ 
}\rashi{\rashiDH{לאלהים. }לרב ולשר׃}}
{וִימַלֵּיל הוּא לָךְ עִם עַמָּא וִיהֵי הוּא יְהֵי לָךְ לִמְתוּרְגְּמָן וְאַתְּ תְּהֵי לֵיהּ לְרָב׃}
{And he shall be thy spokesman unto the people; and it shall come to pass, that he shall be to thee a mouth, and thou shalt be to him in God’s stead.}{\arabic{verse}}
\threeverse{\arabic{verse}}%Ex.4:17
{וְאֶת־הַמַּטֶּ֥ה הַזֶּ֖ה תִּקַּ֣ח בְּיָדֶ֑ךָ אֲשֶׁ֥ר תַּעֲשֶׂה־בּ֖וֹ אֶת־הָאֹתֹֽת׃ \petucha }
{וְיָת חוּטְרָא הָדֵין תִּסַּב בִּידָךְ דְּתַעֲבֵיד בֵּיהּ יָת אָתַיָּא׃}
{And thou shalt take in thy hand this rod, wherewith thou shalt do the signs.’}{\arabic{verse}}
\threeverse{\aliya{ששי}}%Ex.4:18
{וַיֵּ֨לֶךְ מֹשֶׁ֜ה וַיָּ֣שׇׁב \legarmeh  אֶל־יֶ֣תֶר חֹֽתְנ֗וֹ וַיֹּ֤אמֶר לוֹ֙ אֵ֣לְכָה נָּ֗א וְאָשׁ֙וּבָה֙ אֶל־אַחַ֣י אֲשֶׁר־בְּמִצְרַ֔יִם וְאֶרְאֶ֖ה הַעוֹדָ֣ם חַיִּ֑ים וַיֹּ֧אמֶר יִתְר֛וֹ לְמֹשֶׁ֖ה לֵ֥ךְ לְשָׁלֽוֹם׃
\rashi{\rashiDH{וישב אל יתר חתנו. }ליטול רשות, שהרי נשבע לו (שלא יזוז ממדין כי אם ברשותו). (מכילתא יתרו) ושבעה שמות היו לו, רעואל, יתר, יתרו, קיני, וכו׳׃ }}
{וַאֲזַל מֹשֶׁה וְתָב לְוָת יֶתֶר חֲמוּהִי וַאֲמַר לֵיהּ אֵיזֵיל כְּעַן וַאֲתוּב לְוָת אַחַי דִּבְמִצְרַיִם וְאֶחְזֵי הַעַד כְּעַן קַיָּמִין וַאֲמַר יִתְרוֹ לְמֹשֶׁה אִיזֵיל לִשְׁלָם׃}
{And Moses went and returned to Jethro his father-in-law, and said unto him: ‘Let me go, I pray thee, and unto my brethren that are in Egypt, and see whether they be yet alive.’ And Jethro said to Moses: ‘Go in peace.’}{\arabic{verse}}
\threeverse{\arabic{verse}}%Ex.4:19
{וַיֹּ֨אמֶר יְהֹוָ֤ה אֶל־מֹשֶׁה֙ בְּמִדְיָ֔ן לֵ֖ךְ שֻׁ֣ב מִצְרָ֑יִם כִּי־מֵ֙תוּ֙ כׇּל־הָ֣אֲנָשִׁ֔ים הַֽמְבַקְשִׁ֖ים אֶת־נַפְשֶֽׁךָ׃
\rashi{\rashiDH{כי מתו כל האנשים. }מי הם, דתן ואבירם, חיים היו, אלא שירדו מנכסיהם, והעני חשוב כמת (נדרים סד׃)׃ 
}}
{וַאֲמַר יְיָ לְמֹשֶׁה בְּמִדְיָן אִיזֵיל תּוּב לְמִצְרָיִם אֲרֵי מִיתוּ כָל גּוּבְרַיָּא דִּבְעוֹ לְמִקְטְלָךְ׃}
{And the \lord\space said unto Moses in Midian: ‘Go, return into Egypt; for all the men are dead that sought thy life.’}{\arabic{verse}}
\threeverse{\arabic{verse}}%Ex.4:20
{וַיִּקַּ֨ח מֹשֶׁ֜ה אֶת־אִשְׁתּ֣וֹ וְאֶת־בָּנָ֗יו וַיַּרְכִּבֵם֙ עַֽל־הַחֲמֹ֔ר וַיָּ֖שׇׁב אַ֣רְצָה מִצְרָ֑יִם וַיִּקַּ֥ח מֹשֶׁ֛ה אֶת־מַטֵּ֥ה הָאֱלֹהִ֖ים בְּיָדֽוֹ׃
\rashi{\rashiDH{על החמור. }חמור המיוחד, הוא החמור שחבש אברהם לעקידת יצחק, והוא שעתיד מלך המשיח להגלות עליו, שנאמר עָנִי וְרֹכֵב עַל חֲמֹור (זכריה ט, ט)׃ }\rashi{\rashiDH{וישב ארצה מצרים ויקח משה את מטה. }אין מוקדם ומאוחר מדוקדקים במקרא׃}}
{וּדְבַר מֹשֶׁה יָת אִתְּתֵיהּ וְיָת בְּנוֹהִי וְאַרְכֵּיבִנּוּן עַל חֲמָרָא וְתָב לְאַרְעָא דְּמִצְרָיִם וּנְסֵיב מֹשֶׁה יָת חוּטְרָא דְּאִתְעֲבִידוּ בֵּיהּ נִסִּין מִן קֳדָם יְיָ בִּידֵיהּ׃}
{And Moses took his wife and his sons, and set them upon an ass, and he returned to the land of Egypt; and Moses took the rod of God in his hand.}{\arabic{verse}}
\threeverse{\arabic{verse}}%Ex.4:21
{וַיֹּ֣אמֶר יְהֹוָה֮ אֶל־מֹשֶׁה֒ בְּלֶכְתְּךָ֙ לָשׁ֣וּב מִצְרַ֔יְמָה רְאֵ֗ה כׇּל־הַמֹּֽפְתִים֙ אֲשֶׁר־שַׂ֣מְתִּי בְיָדֶ֔ךָ וַעֲשִׂיתָ֖ם לִפְנֵ֣י פַרְעֹ֑ה וַאֲנִי֙ אֲחַזֵּ֣ק אֶת־לִבּ֔וֹ וְלֹ֥א יְשַׁלַּ֖ח אֶת־הָעָֽם׃
\rashi{\rashiDH{בלכתך לשוב מצרימה וגו׳. }דע, שעל מנת כן תלך, שתהא גבור בשליחותי לעשות כל מופתי לפני פרעה ולא תירא ממנו׃ }\rashi{\rashiDH{אשר שמתי בידך. }לא על שלשה אותות האמורות למעלה, שהרי לא לפני פרעה צוה לעשותם אלא לפני ישראל שיאמינו לו, ולא מצינו שעשאם לפניו, אלא מופתים שאני עתיד לשום בידך במצרים, כמו כִּי יְדַבֵּר אֲלֵכֶם פַּרְעֹה וגו׳ (שמות ז, ט), ואל תתמה על אשר כתיב אשר שמתי, שכן משמעו, כשתדבר עמו כבר שַׂמְתִּים בידך׃ }}
{וַאֲמַר יְיָ לְמֹשֶׁה בִּמְהָכָךְ לִמְתָּב לְמִצְרַיִם חֲזִי כָּל מוֹפְתַיָּא דְּשַׁוִּיתִי בִּידָךְ וְתַעֲבֵידִנּוּן קֳדָם פַּרְעֹה וַאֲנָא אֲתַקֵּיף יָת לִבֵּיהּ וְלָא יְשַׁלַּח יָת עַמָּא׃}
{And the \lord\space said unto Moses: ‘When thou goest back into Egypt, see that thou do before Pharaoh all the wonders which I have put in thy hand; but I will harden his heart, and he will not let the people go.}{\arabic{verse}}
\threeverse{\arabic{verse}}%Ex.4:22
{וְאָמַרְתָּ֖ אֶל־פַּרְעֹ֑ה כֹּ֚ה אָמַ֣ר יְהֹוָ֔ה בְּנִ֥י בְכֹרִ֖י יִשְׂרָאֵֽל׃
\rashi{\rashiDH{ואמרת אל פרעה. }כשתשמע שלבו חזק וימאן לשלוח, אמור לו כן׃ }\rashi{\rashiDH{בני בכרי. }לשון גדולה, כמו אַף אָנִי בְּכֹור אֶתְּנֵהוּ (תהלים פט, כח), זו פשוטו. ומדרשו, כאן חתם הקב״ה על מכירת הבכורה שלקח יעקב מעשו׃ }}
{וְתֵימַר לְפַרְעֹה כִּדְנָן אֲמַר יְיָ בְּרִי בּוּכְרִי יִשְׂרָאֵל׃}
{And thou shalt say unto Pharaoh: Thus saith the \lord: Israel is My son, My first-born.}{\arabic{verse}}
\threeverse{\arabic{verse}}%Ex.4:23
{וָאֹמַ֣ר אֵלֶ֗יךָ שַׁלַּ֤ח אֶת־בְּנִי֙ וְיַֽעַבְדֵ֔נִי וַתְּמָאֵ֖ן לְשַׁלְּח֑וֹ הִנֵּה֙ אָנֹכִ֣י הֹרֵ֔ג אֶת־בִּנְךָ֖ בְּכֹרֶֽךָ׃
\rashi{\rashiDH{ואומר אליך. }בשליחותו של מקום׃}\rashi{\rashiDH{שלח את בני וגו׳. הנה אנכי הרג וגו׳. }היא מכה אחרונה, ובה התרהו תחלה מפני שהיא קשה, וזה הוא שנאמר באיוב הֶן אֵל יַשְׂגִּיב בְּכֹחֹו, לפיכך, מִי כָמֹהוּ מֹורֶה (איוב לו, כב), בשר ודם המבקש להנקם מחבירו, מעלים את דבריו שלא יבקש הצלה, אבל הקב״ה ישגיב בכחו ואין יכולת להמלט מידו כי אם בשובו אליו, לפיכך הוא מורהו ומתרה בו לשוב׃ }}
{וַאֲמַרִית לָךְ שַׁלַּח יָת בְּרִי וְיִפְלַח קֳדָמַי וּמְסָרֵיב אַתְּ לְשַׁלָּחוּתֵיהּ הָא אֲנָא קָטֵיל יָת בְּרָךְ בּוּכְרָךְ׃}
{And I have said unto thee: Let My son go, that he may serve Me; and thou hast refused to let him go. ‘Behold, I will slay thy first-born.’}{\arabic{verse}}
\threeverse{\arabic{verse}}%Ex.4:24
{וַיְהִ֥י בַדֶּ֖רֶךְ בַּמָּל֑וֹן וַיִּפְגְּשֵׁ֣הוּ יְהֹוָ֔ה וַיְבַקֵּ֖שׁ הֲמִיתֽוֹ׃
\rashi{\rashiDH{ויהי בדרך במלון. }משה׃}\rashi{\rashiDH{ויבקש המיתו. }למשה, לפי שלא מל את אליעזר בנו, ועל שנתרשל נענש עונש מיתה. תניא אמר רבי יוסי ח״ו לא נתרשל, אלא אמר, אמול ואצא לדרך, סכנה היא לתינוק עד שלשה ימים, אמול ואשהה שלשה ימים, הקב״ה צוני לך שוב מצרים, ומפני מה נענש מיתה, לפי שנתעסק במלון תחלה (במסכת נדרים לא׃), והיה המלאך נעשה כמין נחש, ובולעו מראשו ועד יריכיו, וחוזר ובולעו מרגליו ועד אותו מקום, הבינה צפורה שבשביל המילה הוא׃ 
}}
{וַהֲוָה בְּאוֹרְחָא בְּבֵית מְבָתָא וְעָרַע בֵּיהּ מַלְאֲכָא דַּייָ וּבְעָא לְמִקְטְלֵיהּ׃}
{And it came to pass on the way at the lodging-place, that the \lord\space met him, and sought to kill him.}{\arabic{verse}}
\threeverse{\arabic{verse}}%Ex.4:25
{וַתִּקַּ֨ח צִפֹּרָ֜ה צֹ֗ר וַתִּכְרֹת֙ אֶת־עׇרְלַ֣ת בְּנָ֔הּ וַתַּגַּ֖ע לְרַגְלָ֑יו וַתֹּ֕אמֶר כִּ֧י חֲתַן־דָּמִ֛ים אַתָּ֖ה לִֽי׃
\rashi{\rashiDH{ותגע לרגליו. }השליכתו לפני רגליו של משה׃}\rashi{\rashiDH{ותאמר. }על בנה׃}\rashi{\rashiDH{כי חתן דמים אתה לי. }אתה היית גורם להיות החתן שלי נרצח עליך. הורג אישי אתה לי׃}}
{וּנְסֵיבַת צִפּוֹרָה טִנָּרָא וּגְזַרַת יָת עָרְלַת בְּרַהּ וְקָרֵיבַת לִקְדָמוֹהִי וַאֲמַרַת בִּדְמָא דִּמְהוּלְתָּא הָדֵין אִתְיְהֵיב חַתְנָא לַנָא׃}
{Then Zipporah took a flint, and cut off the foreskin of her son, and cast it at his feet; and she said: ‘Surely a bridegroom of blood art thou to me.’}{\arabic{verse}}
\threeverse{\arabic{verse}}%Ex.4:26
{וַיִּ֖רֶף מִמֶּ֑נּוּ אָ֚ז אָֽמְרָ֔ה חֲתַ֥ן דָּמִ֖ים לַמּוּלֹֽת׃ \petucha 
\rashi{\rashiDH{וירף. }המלאך ממנו. אז, בינה שעל המילה בא להורגו׃ }\rashi{\rashiDH{אמרה חתן דמים למולת. }חתני היה נרצח על דבר המילה. (שינה רש״י בלשונו, לעיל כתב אתה היית גורם, דקשה לרש״י, מה זה אז אמרה חתן דמים, והלא גם לעיל אמרה חתן דמים, אלא מתחלה סברה דזה וזה גורם, חטא המילה וחטא אחר, אח״כ כשראתה וירף לגמרי, אז הבינה דעל דבר המילה לבד בא, ובזה מתורץ גם כן שינוי לשון בתרגום אונקלוס בחתן דמים, ודו״ק כנ״ל)׃ }\rashi{\rashiDH{למולת. }על דבר המולות, שם דבר הוא, והלמ״ד משמשת בלשון על, כמו וְאָמַר פַּרְעֹה לִבְנֵי יִשְׂרָאֵל (שמות יד, ג). ואונקלוס תרגם דמים, על דם המילה׃ 
}}
{וְנָח מִנֵּיהּ בְּכֵן אֲמַרַת אִלּוּלֵי דְּמָא דִּמְהוּלְתָּא הָדֵין אִתְחַיַּיב חַתְנָא קְטוֹל׃}
{So He let him alone. Then she said: ‘A bridegroom of blood in regard of the circumcision.’}{\arabic{verse}}
\threeverse{\arabic{verse}}%Ex.4:27
{וַיֹּ֤אמֶר יְהֹוָה֙ אֶֽל־אַהֲרֹ֔ן לֵ֛ךְ לִקְרַ֥את מֹשֶׁ֖ה הַמִּדְבָּ֑רָה וַיֵּ֗לֶךְ וַֽיִּפְגְּשֵׁ֛הוּ בְּהַ֥ר הָאֱלֹהִ֖ים וַיִּשַּׁק־לֽוֹ׃}
{וַאֲמַר יְיָ לְאַהֲרֹן אִיזֵיל לְקַדָּמוּת מֹשֶׁה לְמַדְבְּרָא וַאֲזַל וְעָרְעֵיהּ בְּטוּרָא דְּאִתְגְּלִי עֲלוֹהִי יְקָרָא דַּייָ וְנַשֵּׁיק לֵיהּ׃}
{And the \lord\space said to Aaron: ‘Go into the wilderness to meet Moses.’ And he went, and met him in the mountain of God, and kissed him.}{\arabic{verse}}
\threeverse{\arabic{verse}}%Ex.4:28
{וַיַּגֵּ֤ד מֹשֶׁה֙ לְאַֽהֲרֹ֔ן אֵ֛ת כׇּל־דִּבְרֵ֥י יְהֹוָ֖ה אֲשֶׁ֣ר שְׁלָח֑וֹ וְאֵ֥ת כׇּל־הָאֹתֹ֖ת אֲשֶׁ֥ר צִוָּֽהוּ׃}
{וְחַוִּי מֹשֶׁה לְאַהֲרֹן יָת כָּל פִּתְגָמַיָּא דַּייָ דְּשַׁלְחֵיהּ וְיָת כָּל אָתַיָּא דְּפַקְּדֵיהּ׃}
{And Moses told Aaron all the words of the \lord\space wherewith He had sent him, and all the signs wherewith He had charged him.}{\arabic{verse}}
\threeverse{\arabic{verse}}%Ex.4:29
{וַיֵּ֥לֶךְ מֹשֶׁ֖ה וְאַהֲרֹ֑ן וַיַּ֣אַסְפ֔וּ אֶת־כׇּל־זִקְנֵ֖י בְּנֵ֥י יִשְׂרָאֵֽל׃}
{וַאֲזַל מֹשֶׁה וְאַהֲרֹן וּכְנַשׁוּ יָת כָּל סָבֵי בְּנֵי יִשְׂרָאֵל׃}
{And Moses and Aaron went and gathered together all the elders of the children of Israel.}{\arabic{verse}}
\threeverse{\arabic{verse}}%Ex.4:30
{וַיְדַבֵּ֣ר אַהֲרֹ֔ן אֵ֚ת כׇּל־הַדְּבָרִ֔ים אֲשֶׁר־דִּבֶּ֥ר יְהֹוָ֖ה אֶל־מֹשֶׁ֑ה וַיַּ֥עַשׂ הָאֹתֹ֖ת לְעֵינֵ֥י הָעָֽם׃}
{וּמַלֵּיל אַהֲרֹן יָת כָּל פִּתְגָמַיָּא דְּמַלֵּיל יְיָ עִם מֹשֶׁה וַעֲבַד אָתַיָּא לְעֵינֵי עַמָּא׃}
{And Aaron spoke all the words which the \lord\space had spoken unto Moses, and did the signs in the sight of the people.}{\arabic{verse}}
\threeverse{\arabic{verse}}%Ex.4:31
{וַֽיַּאֲמֵ֖ן הָעָ֑ם וַֽיִּשְׁמְע֡וּ כִּֽי־פָקַ֨ד יְהֹוָ֜ה אֶת־בְּנֵ֣י יִשְׂרָאֵ֗ל וְכִ֤י רָאָה֙ אֶת־עׇנְיָ֔ם וַֽיִּקְּד֖וּ וַיִּֽשְׁתַּחֲוֽוּ׃}
{וְהֵימֵין עַמָּא וּשְׁמַעוּ אֲרֵי דְּכִיר יְיָ יָת בְּנֵי יִשְׂרָאֵל וַאֲרֵי גְּלֵי קֳדָמוֹהִי שִׁעְבּוּדְהוֹן וּכְרַעוּ וּסְגִידוּ׃}
{And the people believed; and when they heard that the \lord\space had remembered the children of Israel, and that He had seen their affliction, then they bowed their heads and worshipped.}{\arabic{verse}}
\newperek
\threeverse{\aliya{שביעי}}%Ex.5:1
{וְאַחַ֗ר בָּ֚אוּ מֹשֶׁ֣ה וְאַהֲרֹ֔ן וַיֹּאמְר֖וּ אֶל־פַּרְעֹ֑ה כֹּֽה־אָמַ֤ר יְהֹוָה֙ אֱלֹהֵ֣י יִשְׂרָאֵ֔ל שַׁלַּח֙ אֶת־עַמִּ֔י וְיָחֹ֥גּוּ לִ֖י בַּמִּדְבָּֽר׃
\rashi{\rashiDH{ואחר באו משה ואהרן וגו׳. }אבל הזקנים נשמטו אחד אחד מאחר משה ואהרן, עד שנשמטו כולם קודם שהגיעו לפלטין, לפי שיראו ללכת (שמו״ר ה, יד), ובסיני נפרע להם, ונגש משה לבדו והם לא יגשו, החזירם לאחוריהם׃ }}
{וּבָתָר כֵּן עָאלוּ מֹשֶׁה וְאַהֲרֹן וַאֲמַרוּ לְפַרְעֹה כִּדְנָן אֲמַר יְיָ אֱלָהָא דְּיִשְׂרָאֵל שַׁלַּח יָת עַמִּי וְיֵיחֲגוּן קֳדָמַי בְּמַדְבְּרָא׃}
{And afterward Moses and Aaron came, and said unto Pharaoh: ‘Thus saith the \lord, the God of Israel: Let My people go, that they may hold a feast unto Me in the wilderness.’}{\Roman{chap}}
\threeverse{\arabic{verse}}%Ex.5:2
{וַיֹּ֣אמֶר פַּרְעֹ֔ה מִ֤י יְהֹוָה֙ אֲשֶׁ֣ר אֶשְׁמַ֣ע בְּקֹל֔וֹ לְשַׁלַּ֖ח אֶת־יִשְׂרָאֵ֑ל לֹ֤א יָדַ֙עְתִּי֙ אֶת־יְהֹוָ֔ה וְגַ֥ם אֶת־יִשְׂרָאֵ֖ל לֹ֥א אֲשַׁלֵּֽחַ׃}
{וַאֲמַר פַּרְעֹה שְׁמָא דַּייָ לָא אִתְגְּלִי לִי דַּאֲקַבֵּיל לְמֵימְרֵיהּ לְשַׁלָּחָא יָת יִשְׂרָאֵל לָא אִתְגְּלִי לִי שְׁמָא דַּייָ וְאַף יָת יִשְׂרָאֵל לָא אֲשַׁלַּח׃}
{And Pharaoh said: ‘Who is the \lord, that I should hearken unto His voice to let Israel go? I know not the \lord, and moreover I will not let Israel go.’}{\arabic{verse}}
\threeverse{\arabic{verse}}%Ex.5:3
{וַיֹּ֣אמְר֔וּ אֱלֹהֵ֥י הָעִבְרִ֖ים נִקְרָ֣א עָלֵ֑ינוּ נֵ֣לְכָה נָּ֡א דֶּ֩רֶךְ֩ שְׁלֹ֨שֶׁת יָמִ֜ים בַּמִּדְבָּ֗ר וְנִזְבְּחָה֙ לַֽיהֹוָ֣ה אֱלֹהֵ֔ינוּ פֶּ֨ן־יִפְגָּעֵ֔נוּ בַּדֶּ֖בֶר א֥וֹ בֶחָֽרֶב׃
\rashi{\rashiDH{פן יפגענו. }פן יפגעך היו צריכים לומר, אלא שחלקו כבוד למלכות. פגיעה זו, לשון מקרה מות הוא׃ }}
{וַאֲמַרוּ אֱלָהָא דִּיהוּדָאֵי אִתְגְלִי עֲלַנָא נֵיזֵיל כְּעַן מַהְלַךְ תְּלָתָא יוֹמִין בְּמַדְבְּרָא וּנְדַבַּח קֳדָם יְיָ אֱלָהַנָא דִּלְמָא יְעָרְעִנַּנָא בְּמוֹת אוֹ בִּקְטוֹל׃}
{And they said: ‘The God of the Hebrews hath met with us. Let us go, we pray thee, three days’ journey into the wilderness, and sacrifice unto the \lord\space our God; lest He fall upon us with pestilence, or with the sword.’}{\arabic{verse}}
\threeverse{\arabic{verse}}%Ex.5:4
{וַיֹּ֤אמֶר אֲלֵהֶם֙ מֶ֣לֶךְ מִצְרַ֔יִם לָ֚מָּה מֹשֶׁ֣ה וְאַהֲרֹ֔ן תַּפְרִ֥יעוּ אֶת־הָעָ֖ם מִמַּֽעֲשָׂ֑יו לְכ֖וּ לְסִבְלֹתֵיכֶֽם׃
\rashi{\rashiDH{תפריעו את העם ממעשיו. }תבדילו ותרחיקו אותם ממלאכתם, ששומעין לכם וסבורים לנוח מן המלאכה, וכן פְּרַעֵהוּ אַל תַּעַבָר בֹּו (משלי ד, טו), רחקהו, וכן וַתִּפְרְעוּ כָל עֲצָתִי (שם א, כה), כִּי פָּרֻע הוּא (שמות לב, כה), נרחק ונתעב׃ 
}\rashi{\rashiDH{לכו לסבלותיכם. }לכו למלאכתכם שיש לכם לעשות בבתיכם, אבל מלאכת שעבוד מצרים לא היתה על שבטו של לוי, ותדע לך, שהרי משה ואהרן יוצאים ובאים שלא ברשות׃ }}
{וַאֲמַר לְהוֹן מַלְכָּא דְּמִצְרַיִם לְמָא מֹשֶׁה וְאַהֲרֹן תְּבַטְּלוּן יָת עַמָּא מֵעֲבִידַתְהוֹן אִיזִילוּ לְפוּלְחָנְכוֹן׃}
{And the king of Egypt said unto them: ‘Wherefore do ye, Moses and Aaron, cause the people to break loose from their work? get you unto your burdens.’}{\arabic{verse}}
\threeverse{\arabic{verse}}%Ex.5:5
{וַיֹּ֣אמֶר פַּרְעֹ֔ה הֵן־רַבִּ֥ים עַתָּ֖ה עַ֣ם הָאָ֑רֶץ וְהִשְׁבַּתֶּ֥ם אֹתָ֖ם מִסִּבְלֹתָֽם׃
\rashi{\rashiDH{הן רבים עתה עם הארץ. }שהעבודה מוטלת עליהם, ואתם משביתים אותם מסבלותם, הפסד גדול הוא זה׃ }}
{וַאֲמַר פַּרְעֹה הָא מִדְּסַגִּיאִין כְּעַן עַמָּא דְּאַרְעָא וּתְבַטְּלוּן יָתְהוֹן מִפּוּלְחָנְהוֹן׃}
{And Pharaoh said: ‘Behold, the people of the land are now many, and will ye make them rest from their burdens?’}{\arabic{verse}}
\threeverse{\arabic{verse}}%Ex.5:6
{וַיְצַ֥ו פַּרְעֹ֖ה בַּיּ֣וֹם הַה֑וּא אֶת־הַנֹּגְשִׂ֣ים בָּעָ֔ם וְאֶת־שֹׁטְרָ֖יו לֵאמֹֽר׃
\rashi{\rashiDH{הנוגשים. }מצריים היו, והשוטרים היו ישראלים, הנוגש ממונה על כמה שוטרים, והשוטר ממונה לִרְדּוֹת בעושי המלאכה׃ }}
{וּפַקֵּיד פַּרְעֹה בְּיוֹמָא הַהוּא יָת שִׁלְטוֹנֵי עַמָּא וְיָת סָרְכוֹהִי לְמֵימַר׃}
{And the same day Pharaoh commanded the taskmasters of the people, and their officers, saying:}{\arabic{verse}}
\threeverse{\arabic{verse}}%Ex.5:7
{לֹ֣א תֹאסִפ֞וּן לָתֵ֨ת תֶּ֧בֶן לָעָ֛ם לִלְבֹּ֥ן הַלְּבֵנִ֖ים כִּתְמ֣וֹל שִׁלְשֹׁ֑ם הֵ֚ם יֵֽלְכ֔וּ וְקֹשְׁשׁ֥וּ לָהֶ֖ם תֶּֽבֶן׃
\rashi{\rashiDH{תבן. }אשטו״בלא, היו גובלין אותו עם הטיט׃ }\rashi{\rashiDH{לבנים. }טיוו״לש בלע״ז, שעושים מטיט, ומיבשין אותן בחמה, ויש ששורפין אותן בכבשן׃ }\rashi{\rashiDH{כתמול שלשם. }כאשר הייתם עושים עד הנה׃ 
}\rashi{\rashiDH{וקששו. }ולקטו׃}}
{לָא תֵּיסְפוּן לְמִתַּן תִּבְנָא לְעַמָּא לְמִרְמֵי לִבְנִין כְּמֵאֶתְמָלִי וּמִדְּקַמּוֹהִי אִנּוּן יֵיזְלוּן וִיגָבְבוּן לְהוֹן תִּבְנָא׃}
{’Ye shall no more give the people straw to make brick, as heretofore. Let them go and gather straw for themselves.}{\arabic{verse}}
\threeverse{\arabic{verse}}%Ex.5:8
{וְאֶת־מַתְכֹּ֨נֶת הַלְּבֵנִ֜ים אֲשֶׁ֣ר הֵם֩ עֹשִׂ֨ים תְּמ֤וֹל שִׁלְשֹׁם֙ תָּשִׂ֣ימוּ עֲלֵיהֶ֔ם לֹ֥א תִגְרְע֖וּ מִמֶּ֑נּוּ כִּֽי־נִרְפִּ֣ים הֵ֔ם עַל־כֵּ֗ן הֵ֤ם צֹֽעֲקִים֙ לֵאמֹ֔ר נֵלְכָ֖ה נִזְבְּחָ֥ה לֵאלֹהֵֽינוּ׃
\rashi{\rashiDH{ואת מתכנת הלבנים׃ }סכום חשבון הַלְּבֵנִים שהיה כל אחד עושה ליום כשהיה התבן נתן להם, אותו סכום תשימו עליהם גם עתה, למען תכבד העבודה עליהם׃ }\rashi{\rashiDH{כי נרפים. }מן העבודה הם, לכך לבם פונה אל הבטלה וצועקים לאמר נלכה וגו׳׃ }\rashi{\rashiDH{מתכנת. }ותכן לבנים, ולו נתכנו עלילות, את הכסף המתוכן, כולן לשון חשבון הם׃ }\rashi{\rashiDH{נרפים. }המלאכה רפויה בידם ועזובה מהם, והם נרפים ממנה רטר״ייש בלע״ז }}
{וְיָת סְכוֹם לִבְנַיָּא דְּאִנּוּן עָבְדִין מֵאֶתְמָלִי וּמִדְּקַמּוֹהִי תְּמַנּוֹן עֲלֵיהוֹן לָא תִּמְנְעוּן מִנֵּיהּ אֲרֵי בַּטְלָנִין אִנּוּן עַל כֵּן אִנּוּן מַצְוְחִין לְמֵימַר נֵיזֵיל נְדַבַּח קֳדָם אֱלָהַנָא׃}
{And the tale of the bricks, which they did make heretofore, ye shall lay upon them; ye shall not diminish aught thereof; for they are idle; therefore they cry, saying: Let us go and sacrifice to our God.}{\arabic{verse}}
\threeverse{\arabic{verse}}%Ex.5:9
{תִּכְבַּ֧ד הָעֲבֹדָ֛ה עַל־הָאֲנָשִׁ֖ים וְיַעֲשׂוּ־בָ֑הּ וְאַל־יִשְׁע֖וּ בְּדִבְרֵי־שָֽׁקֶר׃
\rashi{\rashiDH{ואל ישעו בדברי שקר. }ואל יהגו וידברו תמיד בדברי רוח, לאמר נלכה נזבחה, ודומה לו ואשעה בחקיך תמיד, למשל ולשנינה מתרגמינן וּלְשׁוֹעִין, וַיְסַפֵּר וְאִשְׁתָּעֵי, ואי אפשר לומר ואל ישעו לשון וישע ה׳ אל הבל וגו׳ ואל קין ואל מנחתו לא שעה, ולפרש אל ישעו אל יפנו, שא״כ היה לו לכתוב ואל ישעו אל דברי שקר, או לדברי שקר, כי כן גזרת כלם, יִשְׁעֶה הָאָדָם עַל עֹשֵׂהוּ (ישעיה יז, ז), וְלֹא שָׁעוּ עַל קְדֹוש יִשְׂרָאֵל (שם לא, א), וְלֹא יִשְׁעֶה אֶל הַמִזְבְּחֹות (שם יז, ח), ולא מצאתי שמוש של בי״ת סמוכה לאחריהם, אבל אחר לשון דבור כמתעסק לדבר בדבר, נופל לשון שמוש בי״ת, כגון הַנִּדְבָּרִים בְּךָ (יחזקאל לג, ל), וַתְּדַבֵּר מִרְיָם וְאַהֲרֹן בְּמֹשֶה (במדבר יב, א), הַמַּלְאָךְ הַדֹּבֵר בִּי (זכריה ד, א), לְדַבֵּר בָּם (דברים יא, יט), וַאֲדַבְּרָה בְעֵדֹותֶיךָ (תהלים קיט, מה), אף כאן אל ישעו בדברי שקר, אל יהיו נדברים בדברי שוא והבאי׃ }}
{יִתְקַף פּוּלְחָנָא עַל גּוּבְרַיָּא וְיִתְעַסְּקוּן בַּהּ וְלָא יִתְעַסְּקוּן בְּפִתְגָמִין בְּטֵילִין׃}
{Let heavier work be laid upon the men, that they may labour therein; and let them not regard lying words.’}{\arabic{verse}}
\threeverse{\arabic{verse}}%Ex.5:10
{וַיֵּ֨צְא֜וּ נֹגְשֵׂ֤י הָעָם֙ וְשֹׁ֣טְרָ֔יו וַיֹּאמְר֥וּ אֶל־הָעָ֖ם לֵאמֹ֑ר כֹּ֚ה אָמַ֣ר פַּרְעֹ֔ה אֵינֶ֛נִּי נֹתֵ֥ן לָכֶ֖ם תֶּֽבֶן׃}
{וּנְפַקוּ שִׁלְטוֹנֵי עַמָּא וְסָרְכוֹהִי וַאֲמַרוּ לְעַמָּא לְמֵימַר כִּדְנָן אֲמַר פַּרְעֹה לֵית אֲנָא יָהֵיב לְכוֹן תִּבְנָא׃}
{And the taskmasters of the people went out, and their officers, and they spoke to the people, saying: ‘Thus saith Pharaoh: I will not give you straw.}{\arabic{verse}}
\threeverse{\arabic{verse}}%Ex.5:11
{אַתֶּ֗ם לְכ֨וּ קְח֤וּ לָכֶם֙ תֶּ֔בֶן מֵאֲשֶׁ֖ר תִּמְצָ֑אוּ כִּ֣י אֵ֥ין נִגְרָ֛ע מֵעֲבֹדַתְכֶ֖ם דָּבָֽר׃
\rashi{\rashiDH{אתם לכו קחו לכם תבן. }וצריכים אתם לילך בזריזות, כי אין נגרע דבר מכל סכום לְבֵנִים שהייתם עושים ליום בהיות התבן נִתַּן לכם מזומן מבית המלך׃ }}
{אַתּוּן אִיזִילוּ סַבוּ לְכוֹן תִּבְנָא מֵאֲתַר דְּתַשְׁכְּחוּן אֲרֵי לָא יִתְמְנַע מִפּוּלְחָנְכוֹן מִדָּעַם׃}
{Go yourselves, get you straw where ye can find it; for nought of your work shall be diminished.’}{\arabic{verse}}
\threeverse{\arabic{verse}}%Ex.5:12
{וַיָּ֥פֶץ הָעָ֖ם בְּכׇל־אֶ֣רֶץ מִצְרָ֑יִם לְקֹשֵׁ֥שׁ קַ֖שׁ לַתֶּֽבֶן׃
\rashi{\rashiDH{לקשש קש לתבן. }לאסוף אסיפה, ללקוט לקט לצורך תבן הטיט׃ }\rashi{\rashiDH{קש. }לשון לקוט, על שם שדבר המתפזר הוא וצריך לקוששו, קרוי קש בשאר מקומות׃ }}
{וְאִתְבַּדַּר עַמָּא בְּכָל אַרְעָא דְּמִצְרָיִם לְגָבָבָא גִּלֵּי לְתִבְנָא׃}
{So the people were scattered abroad throughout all the land of Egypt to gather stubble for straw.}{\arabic{verse}}
\threeverse{\arabic{verse}}%Ex.5:13
{וְהַנֹּגְשִׂ֖ים אָצִ֣ים לֵאמֹ֑ר כַּלּ֤וּ מַעֲשֵׂיכֶם֙ דְּבַר־י֣וֹם בְּיוֹמ֔וֹ כַּאֲשֶׁ֖ר בִּהְי֥וֹת הַתֶּֽבֶן׃
\rashi{\rashiDH{אצים. }דוחקים׃}\rashi{\rashiDH{דבר יום ביומו. }חשבון של כל יום כַּלּוּ ביומו, כאשר עשיתם בהיות התבן מוכן׃ 
}}
{וְשִׁלְטוֹנַיָּא דָּחֲקִין לְמֵימַר אַשְׁלִימוּ עֲבִידַתְכוֹן פִּתְגָּם יוֹם בְּיוֹמֵיהּ כְּמָא דַּהֲוֵיתוֹן עָבְדִין כַּד מִתְיְהֵיב לְכוֹן תִּבְנָא׃}
{And the taskmasters were urgent, saying: ‘Fulfil your work, your daily task, as when there was straw.’}{\arabic{verse}}
\threeverse{\arabic{verse}}%Ex.5:14
{וַיֻּכּ֗וּ שֹֽׁטְרֵי֙ בְּנֵ֣י יִשְׂרָאֵ֔ל אֲשֶׁר־שָׂ֣מוּ עֲלֵהֶ֔ם נֹגְשֵׂ֥י פַרְעֹ֖ה לֵאמֹ֑ר מַדּ֡וּעַ לֹא֩ כִלִּיתֶ֨ם חׇקְכֶ֤ם לִלְבֹּן֙ כִּתְמ֣וֹל שִׁלְשֹׁ֔ם גַּם־תְּמ֖וֹל גַּם־הַיּֽוֹם׃
\rashi{\rashiDH{ויכו שטרי בני ישראל. }השוטרים ישראלים היו, וחסים על חבריהם מלדחקם, וכשהיו משלימין הַלְּבֵנִים לנוגשים שהם מצריים, והיה חסר מן הסכום, היו מלקין אותם על שלא דחקו את עושי המלאכה, לפיכך זכו אותן שוטרים להיות סנהדרין, ונאצל מן הרוח אשר על משה והושם עליהם, שנאמר אספה לי שבעים איש מזקני ישראל, מאותן שידעת הטובה שעשו במצרים, כי הם זקני העם ושוטריו׃ }\rashi{\rashiDH{ויכו שטרי בני ישראל. }אשר שָׂמוּ נגשי פרעה אותם לשוטרים עליהם, לאמר מדוע וגו׳, למה ויכו, שהיו אומרים להם מדוע לא כליתם גם תמול גם היום, חק הקצוב עליכם ללבון כתמול השלישי, שהוא יום שלפני אתמול, והוא היה בהיות התבן נתן להם׃ }\rashi{\rashiDH{ויכו. }לשון וַיּוּפְעֲלוּ, הוכו מיד אחרים, הנוגשים הכום׃ }}
{וּלְקוֹ סָרְכֵי בְּנֵי יִשְׂרָאֵל דְּמַנִּיאוּ עֲלֵיהוֹן שִׁלְטוֹנֵי פַּרְעֹה לְמֵימַר מָדֵין לָא אַשְׁלֵימְתּוּן גְּזֵירַתְכוֹן לְמִרְמֵי לִבְנִין כְּמֵאֶתְמָלִי וּמִדְּקַמּוֹהִי אַף תְּמָלִי אַף יוֹמָא דֵּין׃}
{And the officers of the children of Israel, whom Pharaoh’s taskmasters had set over them, were beaten, saying: ‘Wherefore have ye not fulfilled your appointed task in making brick both yesterday and today as heretofore?’}{\arabic{verse}}
\threeverse{\arabic{verse}}%Ex.5:15
{וַיָּבֹ֗אוּ שֹֽׁטְרֵי֙ בְּנֵ֣י יִשְׂרָאֵ֔ל וַיִּצְעֲק֥וּ אֶל־פַּרְעֹ֖ה לֵאמֹ֑ר לָ֧מָּה תַעֲשֶׂ֦ה כֹ֖ה לַעֲבָדֶֽיךָ׃}
{וַאֲתוֹ סָרְכֵי בְּנֵי יִשְׂרָאֵל וּצְוַחוּ קֳדָם פַּרְעֹה לְמֵימַר לְמָא מִתְעֲבֵיד כְּדֵין לְעַבְדָךְ׃}
{Then the officers of the children of Israel came and cried unto Pharaoh, saying: ‘Wherefore dealest thou thus with thy servants?}{\arabic{verse}}
\threeverse{\arabic{verse}}%Ex.5:16
{תֶּ֗בֶן אֵ֤ין נִתָּן֙ לַעֲבָדֶ֔יךָ וּלְבֵנִ֛ים אֹמְרִ֥ים לָ֖נוּ עֲשׂ֑וּ וְהִנֵּ֧ה עֲבָדֶ֛יךָ מֻכִּ֖ים וְחָטָ֥את עַמֶּֽךָ׃
\rashi{\rashiDH{ולבנים אומרים לנו עשו. }הנוגשים אומרים לנו עשו לְבֵנִים כמנין הראשון׃}\rashi{\rashiDH{וחטאת עמך. }אלו היה נקוד פתח, הייתי אומר שהוא דבוק, ודבר זה חטאת עמך הוא, עכשיו שהוא קמץ, שם דבר הוא, וכך פירושו, ודבר זה מביא חטאת על עמך, כאילו כתוב וחטאת לעמך, כמו כְּבֹואָנָה בֵּית לֶחֶם (רות א, יט), שהוא כמו לבית לחם, וכן הרבה׃ }}
{תִּבְנָא לָא מִתְיְהֵיב לְעַבְדָּךְ וְלִבְנַיָּא אָמְרִין לַנָא עֲבִידוּ וְהָא עַבְדָּךְ לָקַן וְחָטַן עֲלֵיהוֹן עַמָּךְ׃}
{There is no straw given unto thy servants, and they say to us: Make brick; and, behold, thy servants are beaten, but the fault is in thine own people.’}{\arabic{verse}}
\threeverse{\arabic{verse}}%Ex.5:17
{וַיֹּ֛אמֶר נִרְפִּ֥ים אַתֶּ֖ם נִרְפִּ֑ים עַל־כֵּן֙ אַתֶּ֣ם אֹֽמְרִ֔ים נֵלְכָ֖ה נִזְבְּחָ֥ה לַֽיהֹוָֽה׃}
{וַאֲמַר בַּטְלָנִין אַתּוּן בַּטְלָנִין עַל כֵּן אַתּוּן אָמְרִין נֵיזֵיל נְדַבַּח קֳדָם יְיָ׃}
{But he said: ‘Ye are idle, ye are idle; therefore ye say: Let us go and sacrifice to the \lord.}{\arabic{verse}}
\threeverse{\arabic{verse}}%Ex.5:18
{וְעַתָּה֙ לְכ֣וּ עִבְד֔וּ וְתֶ֖בֶן לֹא־יִנָּתֵ֣ן לָכֶ֑ם וְתֹ֥כֶן לְבֵנִ֖ים תִּתֵּֽנוּ׃
\rashi{\rashiDH{ותכן לבנים. }חשבון הַלְּבֵנִים, וכן אֶת הַכֶּסֶף הַמְתֻכָּן (מלכים־ ב יב, יב), המנוי, כמו שאמר בענין וַיָצֻרוּ וַיִּמְנוּ אֶת הַכֶּסֶף (שם יא)׃ }}
{וּכְעַן אִיזִילוּ פְלַחוּ וְתִבְנָא לָא יִתְיְהֵיב לְכוֹן וּסְכוֹם לִבְנַיָּא תִּתְּנוּן׃}
{Go therefore now, and work; for there shall no straw be given you, yet shall ye deliver the tale of bricks.’}{\arabic{verse}}
\threeverse{\arabic{verse}}%Ex.5:19
{וַיִּרְא֞וּ שֹֽׁטְרֵ֧י בְנֵֽי־יִשְׂרָאֵ֛ל אֹתָ֖ם בְּרָ֣ע לֵאמֹ֑ר לֹא־תִגְרְע֥וּ מִלִּבְנֵיכֶ֖ם דְּבַר־י֥וֹם בְּיוֹמֽוֹ׃
\rashi{\rashiDH{ויראו שוטרי בני ישראל. }את חבריהם הנרדים על ידם׃}\rashi{\rashiDH{ברע.} ראו אותם ברעה וצרה המוצאת אותם, בהכבידם העבודה עליהם לאמר לא תגרעו וגו׳׃ }}
{וַחֲזוֹ סָרְכֵי בְנֵי יִשְׂרָאֵל יָתְהוֹן בְּבִישׁ לְמֵימַר לָא תִּמְנְעוּן מִלִּבְנֵיכוֹן פִּתְגָם יוֹם בְּיוֹמֵיהּ׃}
{And the officers of the children of Israel did see that they were set on mischief, when they said: ‘Ye shall not diminish aught from your bricks, your daily task.’}{\arabic{verse}}
\threeverse{\arabic{verse}}%Ex.5:20
{וַֽיִּפְגְּעוּ֙ אֶת־מֹשֶׁ֣ה וְאֶֽת־אַהֲרֹ֔ן נִצָּבִ֖ים לִקְרָאתָ֑ם בְּצֵאתָ֖ם מֵאֵ֥ת פַּרְעֹֽה׃
\rashi{\rashiDH{ויפגעו. }אנשים מישראל את משה ואת אהרן וגו׳. ורבותינו דרשו, כל נצים ונצבים דתן ואבירם היו, שנאמר בהם יצאו נצבים׃ }}
{וְעָרַעוּ יָת מֹשֶׁה וְיָת אַהֲרֹן קָיְמִין לְקַדָּמוּתְהוֹן בְּמִפַּקְהוֹן מִלְּוָת פַּרְעֹה׃}
{And they met Moses and Aaron, who stood in the way, as they came forth from Pharaoh;}{\arabic{verse}}
\threeverse{\arabic{verse}}%Ex.5:21
{וַיֹּאמְר֣וּ אֲלֵהֶ֔ם יֵ֧רֶא יְהֹוָ֛ה עֲלֵיכֶ֖ם וְיִשְׁפֹּ֑ט אֲשֶׁ֧ר הִבְאַשְׁתֶּ֣ם אֶת־רֵיחֵ֗נוּ בְּעֵינֵ֤י פַרְעֹה֙ וּבְעֵינֵ֣י עֲבָדָ֔יו לָֽתֶת־חֶ֥רֶב בְּיָדָ֖ם לְהׇרְגֵֽנוּ׃}
{וַאֲמַרוּ לְהוֹן יִתְגְּלֵי יְיָ עֲלֵיכוֹן וְיִתְפְּרַע דְּאַבְאֵישְׁתּוּן יָת רֵיחַנָא בְּעֵינֵי פַּרְעֹה וּבְעֵינֵי עַבְדּוֹהִי לְמִתַּן חַרְבָּא בְּיַדְהוֹן לְמִקְטְלַנָא׃}
{and they said unto them: ‘The \lord\space look upon you, and judge; because ye have made our savour to be abhorred in the eyes of Pharaoh, and in the eyes of his servants, to put a sword in their hand to slay us.’}{\arabic{verse}}
\threeverse{\aliya{מפטיר}}%Ex.5:22
{וַיָּ֧שׇׁב מֹשֶׁ֛ה אֶל־יְהֹוָ֖ה וַיֹּאמַ֑ר אֲדֹנָ֗י לָמָ֤ה הֲרֵעֹ֙תָה֙ לָעָ֣ם הַזֶּ֔ה לָ֥מָּה זֶּ֖ה שְׁלַחְתָּֽנִי׃
\rashi{\rashiDH{למה הרעתה לעם הזה. }ואם תאמר מה איכפת לך, קוֹבֵל אני על ששלחתני (שמו״ר ה, כב)׃ }}
{וְתָב מֹשֶׁה לִקְדָם יְיָ וַאֲמַר יְיָ לְמָא אַבְאֵישְׁתָּא לְעַמָּא הָדֵין לְמָא דְּנָן שְׁלַחְתָּנִי׃}
{And Moses returned unto the \lord, and said: ‘Lord, wherefore hast Thou dealt ill with this people? why is it that Thou hast sent me?}{\arabic{verse}}
\threeverse{\arabic{verse}}%Ex.5:23
{וּמֵאָ֞ז בָּ֤אתִי אֶל־פַּרְעֹה֙ לְדַבֵּ֣ר בִּשְׁמֶ֔ךָ הֵרַ֖ע לָעָ֣ם הַזֶּ֑ה וְהַצֵּ֥ל לֹא־הִצַּ֖לְתָּ אֶת־עַמֶּֽךָ׃
\rashi{\rashiDH{הרע. }לשון הפעיל הוא, הרבה רעה עליהם, ותרגומו אַבְאֵישׁ׃ }}
{וּמֵעִדָּן דְּעַלִית לְוָת פַּרְעֹה לְמַלָּלָא בִּשְׁמָךְ אַבְאֵישׁ לְעַמָּא הָדֵין וְשֵׁיזָבָא לָא שֵׁיזֵיבְתָּא יָת עַמָּךְ׃}
{For since I came to Pharaoh to speak in Thy name, he hath dealt ill with this people; neither hast Thou delivered Thy people at all.’}{\arabic{verse}}
\newperek
\threeverse{\arabic{verse}}%Ex.6:1
{וַיֹּ֤אמֶר יְהֹוָה֙ אֶל־מֹשֶׁ֔ה עַתָּ֣ה תִרְאֶ֔ה אֲשֶׁ֥ר אֶֽעֱשֶׂ֖ה לְפַרְעֹ֑ה כִּ֣י בְיָ֤ד חֲזָקָה֙ יְשַׁלְּחֵ֔ם וּבְיָ֣ד חֲזָקָ֔ה יְגָרְשֵׁ֖ם מֵאַרְצֽוֹ׃ \setuma         
\rashi{עתה תראה וגו׳. (סנהדרין קיא.) הרהרת על מדותי, לא כאברהם שאמרתי לו כִּי בְיִצְחָק יִקָּרֵא לְךָ זָרַע (בראשית כא, יב), ואחר כך אמרתי לו העלהו לעולה, ולא הרהר אחרי מדותי, לפיכך עתה תראה, העשוי לפרעה תראה, ולא העשוי למלכי שבעה אומות כשאביאם לארץ׃ }\rashi{\rashiDH{כי ביד חזקה ישלחם. }מפני ידי החזקה שתחזק עליו, ישלחם׃ 
}\rashi{\rashiDH{וביד חזקה יגרשם מארצו. }על כרחם של ישראל יגרשם, ולא יספיקו לעשות להם צדה, וכן הוא אומר ותחזק מצרים על העם למהר לשלחם וגו׳׃ 
}}
{וַאֲמַר יְיָ לְמֹשֶׁה כְּעַן תִּחְזֵי דְּאַעֲבֵיד לְפַרְעֹה אֲרֵי בְּיַד תַּקִּיפָא יְשַׁלְּחִנּוּן וּבְיַד תַּקִּיפָא יְתָרֵיכִנּוּן מֵאַרְעֵיהּ׃}
{And the \lord\space said unto Moses: ‘Now shalt thou see what I will do to Pharaoh; for by a strong hand shall he let them go, and by a strong hand shall he drive them out of his land.’}{\Roman{chap}}
\newparsha{וארא}
\threeverse{\aliya{וארא}}%Ex.6:2
{וַיְדַבֵּ֥ר אֱלֹהִ֖ים אֶל־מֹשֶׁ֑ה וַיֹּ֥אמֶר אֵלָ֖יו אֲנִ֥י יְהֹוָֽה׃
\rashi{\rashiDH{וידבר אלהים אל משה. }דִּבֶּר אתו משפט, על שהקשה לדבר ולומר למה הרעותה לעם הזה׃ }\rashi{\rashiDH{ויאמר אליו אני ה׳. }נאמן לשלם שכר טוב למתהלכים לפני, ולא לחנם שלחתיך כי אם לקיים דברי שדברתי לאבות הראשונים. ובלשון הזה מצינו שהוא נדרש בכמה מקומות אני ה׳ נאמן ליפרע, כשהוא אומר אצל עונש, כגון וחללת את שם אלהיך אני ה׳, וכשהוא אומר אצל קיום מצות, כגון ושמרתם מצותי ועשיתם אותם אני ה׳, נאמן ליתן שכר׃ }}
{וּמַלֵּיל יְיָ עִם מֹשֶׁה וַאֲמַר לֵיהּ אֲנָא יְיָ׃}
{And God spoke unto Moses, and said unto him: ‘I am the \lord;}{\arabic{verse}}
\threeverse{\arabic{verse}}%Ex.6:3
{וָאֵרָ֗א אֶל־אַבְרָהָ֛ם אֶל־יִצְחָ֥ק וְאֶֽל־יַעֲקֹ֖ב בְּאֵ֣ל שַׁדָּ֑י וּשְׁמִ֣י יְהֹוָ֔ה לֹ֥א נוֹדַ֖עְתִּי לָהֶֽם׃
\rashi{\rashiDH{וארא. }אל האבות׃}\rashi{\rashiDH{באל שדי. }הבטחתים הבטחות, ובכולן אמרתי להם אני אל שדי׃ }\rashi{\rashiDH{ושמי ה׳ לא נודעתי להם. }לא הודעתי אין כתיב כאן, אלא לא נודעתי, לא נִכַּרְתִּי להם במדת אמיתית שלי שעליה נקרא שמי ה׳, נאמן לְאַמֵּת דברי, שהרי הבטחתים ולא קיימתי׃ }}
{וְאִתְגְּלִיתִי לְאַבְרָהָם לְיִצְחָק וּלְיַעֲקֹב בְּאֵל שַׁדָּי וּשְׁמִי יְיָ לָא הוֹדַעִית לְהוֹן׃}
{and I appeared unto Abraham, unto Isaac, and unto Jacob, as God Almighty, but by My name \englishshem I made Me not known to them.}{\arabic{verse}}
\threeverse{\arabic{verse}}%Ex.6:4
{וְגַ֨ם הֲקִמֹ֤תִי אֶת־בְּרִיתִי֙ אִתָּ֔ם לָתֵ֥ת לָהֶ֖ם אֶת־אֶ֣רֶץ כְּנָ֑עַן אֵ֛ת אֶ֥רֶץ מְגֻרֵיהֶ֖ם אֲשֶׁר־גָּ֥רוּ בָֽהּ׃
\rashi{\rashiDH{וגם הקמתי את בריתי וגו׳. }וגם כשנראיתי להם באל שדי, הצבתי והעמדתי בריתי ביני וביניהם׃ }\rashi{\rashiDH{לתת להם את ארץ כנען. }לאברהם בפרשת מילה נאמר, אֲנִי אֵל שַדַּי וגו׳ וְנָתַתִּי לְךָ וּלְזַרְעֲךָ אַחֲרֶיךָ אֵת אֶרֶץ מְגֻרֶיךָ (בראשית יז, אח). ליצחק, כִּי לְךָ וּלְזַרְעֲךָ אֶתֵּן אֶת כָּל הָאֲרצֹת הָאֵל וַהֲקִימֹתִּי אֶת הַשְׁבֻעָה אֲשֶׁר נִשְׁבַּעְתִּי לְאַבְרָהָם (שם כו, ג), ואותה שבועה שנשבעתי לאברהם באל שדי, אמרתי ליעקב אֲנִי אֵל שַׁדַּי פְּרֵה וּרְבֵה וגו׳ (שם לה, יא), וְאֶת הָאָרֶץ אֲשֶׁר וגו׳ (שם יב), הרי שנדרתי להם ולא קיימתי׃ }}
{וְאַף אֲקֵימִית יָת קְיָמִי עִמְּהוֹן לְמִתַּן לְהוֹן יָת אַרְעָא דִּכְנָעַן יָת אֲרַע תּוֹתָבוּתְהוֹן דְּאִתּוֹתַבוּ בַהּ׃}
{And I have also established My covenant with them, to give them the land of Canaan, the land of their sojournings, wherein they sojourned.}{\arabic{verse}}
\threeverse{\arabic{verse}}%Ex.6:5
{וְגַ֣ם \legarmeh  אֲנִ֣י שָׁמַ֗עְתִּי אֶֽת־נַאֲקַת֙ בְּנֵ֣י יִשְׂרָאֵ֔ל אֲשֶׁ֥ר מִצְרַ֖יִם מַעֲבִדִ֣ים אֹתָ֑ם וָאֶזְכֹּ֖ר אֶת־בְּרִיתִֽי׃
\rashi{\rashiDH{וגם אני. }כמו שהצבתי והעמדתי הברית יש עלי לקיים, לפיכך \rashiDH{שמעתי את נאקת בני ישראל} הנואקים׃ \rashiDH{אשר מצרים מעבידים אתם ואזכור. }אותו הברית, כי בברית בין הבתרים אמרתי לו וְגַם אֶת הַגֹוי אַשֶר יַעֲבֹדוּ דָן אָנֹכִי (בראשית טו, יד)׃ }}
{וְאַף קֳדָמַי שְׁמִיעַ יָת קְבִילַת בְּנֵי יִשְׂרָאֵל דְּמִצְרָאֵי מַפְלְחִין בְּהוֹן וּדְכִירְנָא יָת קְיָמִי׃}
{And moreover I have heard the groaning of the children of Israel, whom the Egyptians keep in bondage; and I have remembered My covenant.}{\arabic{verse}}
\threeverse{\aliya{לוי}}%Ex.6:6
{לָכֵ֞ן אֱמֹ֥ר לִבְנֵֽי־יִשְׂרָאֵל֮ אֲנִ֣י יְהֹוָה֒ וְהוֹצֵאתִ֣י אֶתְכֶ֗ם מִתַּ֙חַת֙ סִבְלֹ֣ת מִצְרַ֔יִם וְהִצַּלְתִּ֥י אֶתְכֶ֖ם מֵעֲבֹדָתָ֑ם וְגָאַלְתִּ֤י אֶתְכֶם֙ בִּזְר֣וֹעַ נְטוּיָ֔ה וּבִשְׁפָטִ֖ים גְּדֹלִֽים׃
\rashi{\rashiDH{לכן. }על פי אותה השבועה׃}\rashi{\rashiDH{אמור לבני ישראל אני ה׳. }הנאמן בהבטחתי׃}\rashi{\rashiDH{והוצאתי אתכם. }כי כן הבטחתיו (שם), וְאַחֲרֵי כֵן יֵצְאוּ בִּרְכֻשׁ גָדֹול׃ 
}\rashi{\rashiDH{סבלות מצרים. }טורח משא מצרים׃ 
}}
{בְּכֵן אֵימַר לִבְנֵי יִשְׂרָאֵל אֲנָא יְיָ וְאַפֵּיק יָתְכוֹן מִגּוֹ דְּחוֹק פּוּלְחַן מִצְרָאֵי וַאֲשֵׁיזֵיב יָתְכוֹן מִפּוּלְחָנְהוֹן וְאֶפְרוֹק יָתְכוֹן בִּדְרָע מְרָמַם וּבְדִינִין רַבְרְבִין׃}
{Wherefore say unto the children of Israel: I am the \lord, and I will bring you out from under the burdens of the Egyptians, and I will deliver you from their bondage, and I will redeem you with an outstretched arm, and with great judgments;}{\arabic{verse}}
\threeverse{\arabic{verse}}%Ex.6:7
{וְלָקַחְתִּ֨י אֶתְכֶ֥ם לִי֙ לְעָ֔ם וְהָיִ֥יתִי לָכֶ֖ם לֵֽאלֹהִ֑ים וִֽידַעְתֶּ֗ם כִּ֣י אֲנִ֤י יְהֹוָה֙ אֱלֹ֣הֵיכֶ֔ם הַמּוֹצִ֣יא אֶתְכֶ֔ם מִתַּ֖חַת סִבְל֥וֹת מִצְרָֽיִם׃}
{וַאֲקָרֵיב יָתְכוֹן קֳדָמַי לְעַם וְאֶהְוֵי לְכוֹן לֶאֱלָהּ וְתִדְּעוּן אֲרֵי אֲנָא יְיָ אֱלָהֲכוֹן דְּאַפֵּיק יָתְכוֹן מִגּוֹ דְּחוֹק פּוּלְחַן מִצְרָאֵי׃}
{and I will take you to Me for a people, and I will be to you a God; and ye shall know that I am the \lord\space your God, who brought you out from under the burdens of the Egyptians.}{\arabic{verse}}
\threeverse{\arabic{verse}}%Ex.6:8
{וְהֵבֵאתִ֤י אֶתְכֶם֙ אֶל־הָאָ֔רֶץ אֲשֶׁ֤ר נָשָׂ֙אתִי֙ אֶת־יָדִ֔י לָתֵ֣ת אֹתָ֔הּ לְאַבְרָהָ֥ם לְיִצְחָ֖ק וּֽלְיַעֲקֹ֑ב וְנָתַתִּ֨י אֹתָ֥הּ לָכֶ֛ם מוֹרָשָׁ֖ה אֲנִ֥י יְהֹוָֽה׃
\rashi{\rashiDH{נשאתי את ידי. }הרימותיה לישבע בכסאי׃}}
{וְאַעֵיל יָתְכוֹן לְאַרְעָא דְּקַיֵּימִית בְּמֵימְרִי לְמִתַּן יָתַהּ לְאַבְרָהָם לְיִצְחָק וּלְיַעֲקֹב וְאֶתֵּין יָתַהּ לְכוֹן יְרוּתָּא אֲנָא יְיָ׃}
{And I will bring you in unto the land, concerning which I lifted up My hand to give it to Abraham, to Isaac, and to Jacob; and I will give it you for a heritage: I am the \lord.’}{\arabic{verse}}
\threeverse{\arabic{verse}}%Ex.6:9
{וַיְדַבֵּ֥ר מֹשֶׁ֛ה כֵּ֖ן אֶל־בְּנֵ֣י יִשְׂרָאֵ֑ל וְלֹ֤א שָֽׁמְעוּ֙ אֶל־מֹשֶׁ֔ה מִקֹּ֣צֶר ר֔וּחַ וּמֵעֲבֹדָ֖ה קָשָֽׁה׃ \petucha 
\rashi{\rashiDH{ולא שמעו אל משה. }לא קבלו תנחומין׃}\rashi{\rashiDH{מקצר רוח. }כל מי שהוא מיצר רוחו ונשימתו קצרה ואינו יכול להאריך בנשימתו. קרוב לענין זה שמעתי בפרשה זו מרבי ברוך בר׳ אליעזר, והביא לי ראיה ממקרא זה, בַּפַּעַם הַזֹּאת אֹודִיעֵם אֶת יָדִי וְאֶת גְבוּרָתִי וְיָדְעוּ כִּי שְׁמִי ה׳ (ירמי׳ טז, כא), למדנו כשהקב״ה מְאַמֵּן את דבריו אפילו לפורענות, מודיע ששמו ה׳, וכל שכן האמנה לטובה. ורבותינו דרשוהו (שמו״ר ו, ד.  סנהדרין קיא.) לענין של מעלה, שאמר משה לָמָה הֲרֵעֹתָה (שמות ה, כב), אמר לו הקב״ה חבל על דאבדין ולא משתכחין, יש לי להתאונן על מיתת האבות, הרבה פעמים נגליתי עליהם באל שדי, ולא אמרו לי מה שמך, ואתה אמרת מה שמו מה אומר אליהם׃ }\rashi{\rashiDH{וגם הקימותי וגו׳. }וכשבקש אברהם לקבור את שרה, לא מצא קרקע עד שקנה בדמים מרובים, וכן ביצחק ערערו עליו על הבארות אשר חפר, וכן ביעקב וַיִקֶן אֶת חֶלְקַת הַשָׂדֶה לנטות אהלו (בראשית לג, יט), ולא הרהרו אחר מדותי, ואתה אמרת למה הרעותה. ואין המדרש מתישב אחר המקרא מפני כמה דברים, אחת, שלא נאמר ושמי ה׳ לא שאלו לי, ואם תאמר לא הודיעם שכך שמו, הרי תחלה כשנגלה לאברהם בין הבתרים נאמר אֲנִי ה׳ אֲשֶׁר הֹוצֵאתִיךָ מֵאוּר כַּשְׁדִּים (שם טו, ז), ועוד, היאך הסמיכה נמשכת בדברים שהוא סומך לכאן וגם אני שמעתי וגו׳, לכן אמור לבני ישראל, לכך אני אומר יתישב המקרא על פשוטו דָּבָר דָּבֻר עַל אָפְנָיו (משלי כה, יא), והדרש תדרש, שנאמר הֲלֹוא כֹה דְבָרִי כָּאֵש נְאֻם ה׳ וּכְפַטִּיש יְפֹצֵץ סָלַע (ירמיה כג, כט), מתחלק לכמה ניצוצות׃ 
}}
{וּמַלֵּיל מֹשֶׁה כֵּן עִם בְּנֵי יִשְׂרָאֵל וְלָא קַבִּילוּ מִן מֹשֶׁה מֵעֲיָק רוּחַ וּמִפּוּלְחָנָא דַּהֲוָה קְשֵׁי עֲלֵיהוֹן׃}
{And Moses spoke so unto the children of Israel; but they hearkened not unto Moses for impatience of spirit, and for cruel bondage.}{\arabic{verse}}
\threeverse{\aliya{ישראל}}%Ex.6:10
{וַיְדַבֵּ֥ר יְהֹוָ֖ה אֶל־מֹשֶׁ֥ה לֵּאמֹֽר׃}
{וּמַלֵּיל יְיָ עִם מֹשֶׁה לְמֵימַר׃}
{And the \lord\space spoke unto Moses, saying:}{\arabic{verse}}
\threeverse{\arabic{verse}}%Ex.6:11
{בֹּ֣א דַבֵּ֔ר אֶל־פַּרְעֹ֖ה מֶ֣לֶךְ מִצְרָ֑יִם וִֽישַׁלַּ֥ח אֶת־בְּנֵֽי־יִשְׂרָאֵ֖ל מֵאַרְצֽוֹ׃}
{עוֹל מַלֵּיל עִם פַּרְעֹה מַלְכָּא דְּמִצְרָיִם וִישַׁלַּח יָת בְּנֵי יִשְׂרָאֵל מֵאַרְעֵיהּ׃}
{’Go in, speak unto Pharaoh king of Egypt, that he let the children of Israel go out of his land.’}{\arabic{verse}}
\threeverse{\arabic{verse}}%Ex.6:12
{וַיְדַבֵּ֣ר מֹשֶׁ֔ה לִפְנֵ֥י יְהֹוָ֖ה לֵאמֹ֑ר הֵ֤ן בְּנֵֽי־יִשְׂרָאֵל֙ לֹֽא־שָׁמְע֣וּ אֵלַ֔י וְאֵיךְ֙ יִשְׁמָעֵ֣נִי פַרְעֹ֔ה וַאֲנִ֖י עֲרַ֥ל שְׂפָתָֽיִם׃ \petucha 
\rashi{\rashiDH{ערל שפתים. }אטום שפתים, וכן כל לשון ערלה אני אומר שהוא אטום. עֲרֵלָה אָזְנָם (שם ו, י), אטומה משמוע. עַרְלֵי לֵב (שם ט, כה), אטומים מהבין. שְׁתֵה גַּם אַתָּה וְהֵעָרֵל (חבקוק ב, טז), והאטם משכרות כוס הקללה (נ״א התרעלה). וְעֶרֶל בָּשָׁר (יחזקאל מד, ט) שהגיד אטום ומכוסה בה. וַעֲרַלְתֶּם עָרְלָתֹו (ויקרא יט, כג), עשו לו אוטם וכיסוי, איסור שיבדיל בפני אכילתו. שָׁלשׁ שָׁנִים יִהְיֶה לָכֶם עֲרֵלִים (שם), אטום ומכוסה ומובדל מלאכלו׃ }\rashi{\rashiDH{ואיך ישמעני פרעה. }זה אחד מעשרה ק״ו שבתורה (ב״ר צב, ז)׃ 
}}
{וּמַלֵּיל מֹשֶׁה קֳדָם יְיָ לְמֵימַר הָא בְּנֵי יִשְׂרָאֵל לָא קַבִּילוּ מִנִּי וְאֵיכְדֵין יְקַבֵּיל מִנִּי פַּרְעֹה וַאֲנָא יַקִּיר מַמְלַל׃}
{And Moses spoke before the \lord, saying: ‘Behold, the children of Israel have not hearkened unto me; how then shall Pharaoh hear me, who am of uncircumcised lips?’}{\arabic{verse}}
\threeverse{\arabic{verse}}%Ex.6:13
{וַיְדַבֵּ֣ר יְהֹוָה֮ אֶל־מֹשֶׁ֣ה וְאֶֽל־אַהֲרֹן֒ וַיְצַוֵּם֙ אֶל־בְּנֵ֣י יִשְׂרָאֵ֔ל וְאֶל־פַּרְעֹ֖ה מֶ֣לֶךְ מִצְרָ֑יִם לְהוֹצִ֥יא אֶת־בְּנֵֽי־יִשְׂרָאֵ֖ל מֵאֶ֥רֶץ מִצְרָֽיִם׃ \setuma         
\rashi{\rashiDH{וידבר ה׳ אל משה ואל אהרן. }לפי שאמר משה ואני ערל שפתים, צירף לו הקב״ה את אהרן להיות לו לְפֶה ולמליץ׃ }\rashi{\rashiDH{ויצום אל בני ישראל. }צוה עליהם להנהיגם בנחת ולסבול אותם (שמו״ר ז, ג)׃ 
}\rashi{\rashiDH{ואל פרעה מלך מצרים. }צִוָּם עליו לחלוק לו כבוד בדבריהם, זה מדרשו. ופשוטו, צִוָּם על דבר ישראל ועל שליחותו אל פרעה. ודבר הצווי מהו, מפורש בפרשה שניה לאחר סדר היחס, אלא מתוך שהזכיר משה ואהרן, הפסיק הענין באלה ראשי בית אבותם, ללמדנו היאך נולדו משה ואהרן, ובמי נתיחסו׃ }}
{וּמַלֵּיל יְיָ עִם מֹשֶׁה וּלְאַהֲרֹן וּפַקֵּידִנּוּן לְוָת בְּנֵי יִשְׂרָאֵל וּלְוָת פַּרְעֹה מַלְכָּא דְּמִצְרָיִם לְאַפָּקָא יָת בְּנֵי יִשְׂרָאֵל מֵאַרְעָא דְּמִצְרָיִם׃}
{And the \lord\space spoke unto Moses and unto Aaron, and gave them a charge unto the children of Israel, and unto Pharaoh king of Egypt, to bring the children of Israel out of the land of Egypt.}{\arabic{verse}}
\threeverse{\aliya{שני}}%Ex.6:14
{אֵ֖לֶּה רָאשֵׁ֣י בֵית־אֲבֹתָ֑ם בְּנֵ֨י רְאוּבֵ֜ן בְּכֹ֣ר יִשְׂרָאֵ֗ל חֲנ֤וֹךְ וּפַלּוּא֙ חֶצְרֹ֣ן וְכַרְמִ֔י אֵ֖לֶּה מִשְׁפְּחֹ֥ת רְאוּבֵֽן׃
\rashi{\rashiDH{אלה ראשי בית אבותם. }מתוך שהוזקק ליחס שבטו של לוי עד משה ואהרן בשביל משה ואהרן, התחיל ליחסם דרך תולדותם מראובן. (ובפסיקתא גדולה ראיתי, לפי שֶׁקִּנְטְרָם יעקב אבינו לשלשה שבטים הללו בשעת מותו, חזר הכתוב ויחסם כאן לבדם, לומר שחשובים הם)׃ }}
{אִלֵּין רֵישֵׁי בֵית אֲבָהָתְהוֹן בְּנֵי רְאוּבֵן בּוּכְרָא דְּיִשְׂרָאֵל חֲנוֹךְ וּפַלּוּא חֶצְרוֹן וְכַרְמִי אִלֵּין זַרְעֲיָת רְאוּבֵן׃}
{These are the heads of their fathers’ houses: the sons of Reuben the first-born of Israel: Hanoch, and Pallu, Hezron, and Carmi. These are the families of Reuben.}{\arabic{verse}}
\threeverse{\arabic{verse}}%Ex.6:15
{וּבְנֵ֣י שִׁמְע֗וֹן יְמוּאֵ֨ל וְיָמִ֤ין וְאֹ֙הַד֙ וְיָכִ֣ין וְצֹ֔חַר וְשָׁא֖וּל בֶּן־הַֽכְּנַעֲנִ֑ית אֵ֖לֶּה מִשְׁפְּחֹ֥ת שִׁמְעֽוֹן׃}
{וּבְנֵי שִׁמְעוֹן יְמוּאֵל וְיָמִין וְאֹהַד וְיָכִין וְצֹחַר וְשָׁאוּל בַּר כְּנַעֲנֵיתָא אִלֵּין זַרְעֲיָת שִׁמְעוֹן׃}
{And the sons of Simeon: Jemuel, and Jamin, and Ohad, and Jachin, and Zohar, and Shaul the son of a Canaanitish woman. These are the families of Simeon.}{\arabic{verse}}
\threeverse{\arabic{verse}}%Ex.6:16
{וְאֵ֨לֶּה שְׁמ֤וֹת בְּנֵֽי־לֵוִי֙ לְתֹ֣לְדֹתָ֔ם גֵּרְשׁ֕וֹן וּקְהָ֖ת וּמְרָרִ֑י וּשְׁנֵי֙ חַיֵּ֣י לֵוִ֔י שֶׁ֧בַע וּשְׁלֹשִׁ֛ים וּמְאַ֖ת שָׁנָֽה׃
\rashi{\rashiDH{ושני חיי לוי וגו׳. }למה נמנו שנותיו של לוי, להודיע כמה ימי השעבוד, שכל זמן שאחד מן השבטים קיים, לא היה שעבוד, שנאמר וַיָּמָת יֹוסֵף וְכָל אֶחָיו (שמות א, ו), ואח״כ וַיָּקָם מֶלֶךְ חָדָשׁ, ולוי האריך ימים על כולם׃ 
}}
{וְאִלֵּין שְׁמָהָת בְּנֵי לֵוִי לְתוֹלְדָתְהוֹן גֵּרְשׁוֹן וּקְהָת וּמְרָרִי וּשְׁנֵי חַיֵּי לֵוִי מְאָה וּתְלָתִין וּשְׁבַע שְׁנִין׃}
{And these are the names of the sons of Levi according to their generations: Gershon and Kohath, and Merari. And the years of the life of Levi were a hundred thirty and seven years.}{\arabic{verse}}
\threeverse{\arabic{verse}}%Ex.6:17
{בְּנֵ֥י גֵרְשׁ֛וֹן לִבְנִ֥י וְשִׁמְעִ֖י לְמִשְׁפְּחֹתָֽם׃}
{בְּנֵי גֵּרְשׁוֹן לִבְנִי וְשִׁמְעִי לְזַרְעֲיָתְהוֹן׃}
{The sons of Gershon: Libni and Shimei, according to their families.}{\arabic{verse}}
\threeverse{\arabic{verse}}%Ex.6:18
{וּבְנֵ֣י קְהָ֔ת עַמְרָ֣ם וְיִצְהָ֔ר וְחֶבְר֖וֹן וְעֻזִּיאֵ֑ל וּשְׁנֵי֙ חַיֵּ֣י קְהָ֔ת שָׁלֹ֧שׁ וּשְׁלֹשִׁ֛ים וּמְאַ֖ת שָׁנָֽה׃
\rashi{\rashiDH{ושני חיי קהת. ושני חיי עמרם וגו׳. }מחשבון זה אנו למדים על מושב בני ישראל ארבע מאות שנה שאמר הכתוב, שלא בארץ מצרים לבדה היו, אלא מיום שנולד יצחק, שהרי קהת מיורדי מצרים היה, חשוב כל שנותיו ושנות עמרם ושמונים של משה, לא תמצאם ד׳ מאות שנה, והרבה שנים נבלעים לבנים בשני האבות׃ }}
{וּבְנֵי קְהָת עַמְרָם וְיִצְהָר וְחֶבְרוֹן וְעֻזִּיאֵל וּשְׁנֵי חַיֵּי קְהָת מְאָה וּתְלָתִין וּתְלָת שְׁנִין׃}
{And the sons of Kohath: Amram, and Izhar, and Hebron, and Uzziel. And the years of the life of Kohath were a hundred thirty and three years.}{\arabic{verse}}
\threeverse{\arabic{verse}}%Ex.6:19
{וּבְנֵ֥י מְרָרִ֖י מַחְלִ֣י וּמוּשִׁ֑י אֵ֛לֶּה מִשְׁפְּחֹ֥ת הַלֵּוִ֖י לְתֹלְדֹתָֽם׃}
{וּבְנֵי מְרָרִי מַחְלִי וּמוּשִׁי אִלֵּין זַרְעֲיָת לֵוִי לְתוֹלְדָתְהוֹן׃}
{And the sons of Merari: Mahli and Mushi. These are the families of the Levites according to their generations.}{\arabic{verse}}
\threeverse{\arabic{verse}}%Ex.6:20
{וַיִּקַּ֨ח עַמְרָ֜ם אֶת־יוֹכֶ֤בֶד דֹּֽדָתוֹ֙ ל֣וֹ לְאִשָּׁ֔ה וַתֵּ֣לֶד ל֔וֹ אֶֽת־אַהֲרֹ֖ן וְאֶת־מֹשֶׁ֑ה וּשְׁנֵי֙ חַיֵּ֣י עַמְרָ֔ם שֶׁ֧בַע וּשְׁלֹשִׁ֛ים וּמְאַ֖ת שָׁנָֽה׃
\rashi{\rashiDH{יוכבד דדתו. }אחת אבוהי, בת לוי אחות קהת׃ 
}}
{וּנְסֵיב עַמְרָם יָת יוֹכֶבֶד אֲחָת אֲבוּהִי לֵיהּ לְאִתּוּ וִילֵידַת לֵיהּ יָת אַהֲרֹן וְיָת מֹשֶׁה וּשְׁנֵי חַיֵּי עַמְרָם מְאָה וּתְלָתִין וּשְׁבַע שְׁנִין׃}
{And Amram took him Jochebed his father’s sister to wife; and she bore him Aaron and Moses. And the years of the life of Amram were a hundred and thirty and seven years.}{\arabic{verse}}
\threeverse{\arabic{verse}}%Ex.6:21
{וּבְנֵ֖י יִצְהָ֑ר קֹ֥רַח וָנֶ֖פֶג וְזִכְרִֽי׃}
{וּבְנֵי יִצְהָר קֹרַח וָנֶפֶג וְזִכְרִי׃}
{And the sons of Izhar: Korah, and Nepheg, and Zichri.}{\arabic{verse}}
\threeverse{\arabic{verse}}%Ex.6:22
{וּבְנֵ֖י עֻזִּיאֵ֑ל מִֽישָׁאֵ֥ל וְאֶלְצָפָ֖ן וְסִתְרִֽי׃}
{וּבְנֵי עֻזִּיאֵל מִישָׁאֵל וְאֶלְצָפָן וְסִתְרִי׃}
{And the sons of Uzziel: Mishael, and Elzaphan, and Sithri.}{\arabic{verse}}
\threeverse{\arabic{verse}}%Ex.6:23
{וַיִּקַּ֨ח אַהֲרֹ֜ן אֶת־אֱלִישֶׁ֧בַע בַּת־עַמִּינָדָ֛ב אֲח֥וֹת נַחְשׁ֖וֹן ל֣וֹ לְאִשָּׁ֑ה וַתֵּ֣לֶד ל֗וֹ אֶת־נָדָב֙ וְאֶת־אֲבִיה֔וּא אֶת־אֶלְעָזָ֖ר וְאֶת־אִֽיתָמָֽר׃
\rashi{\rashiDH{אחות נחשון. }מכאן למדנו, הנושא אשה צריך לבדוק באחיה (ב״ב קי.  שמו״ר ז, ד)׃ 
}}
{וּנְסֵיב אַהֲרֹן יָת אֱלִישֶׁבַע בַּת עַמִּינָדָב אֲחָתֵיהּ דְּנַחְשׁוֹן לֵיהּ לְאִתּוּ וִילֵידַת לֵיהּ יָת נָדָב וְיָת אֲבִיהוּא יָת אֶלְעָזָר וְיָת אִיתָמָר׃}
{And Aaron took him Elisheba, the daughter of Amminadab, the sister of Nahshon, to wife; and she bore him Nadab and Abihu, Eleazar and Ithamar.}{\arabic{verse}}
\threeverse{\arabic{verse}}%Ex.6:24
{וּבְנֵ֣י קֹ֔רַח אַסִּ֥יר וְאֶלְקָנָ֖ה וַאֲבִיאָסָ֑ף אֵ֖לֶּה מִשְׁפְּחֹ֥ת הַקׇּרְחִֽי׃}
{וּבְנֵי קֹרַח אַסִּיר וְאֶלְקָנָה וַאֲבִיאָסָף אִלֵּין זַרְעֲיָת קֹרַח׃}
{And the sons of Korah: Assir, and Elkanah, and Abiasaph; these are the families of the Korahites.}{\arabic{verse}}
\threeverse{\arabic{verse}}%Ex.6:25
{וְאֶלְעָזָ֨ר בֶּֽן־אַהֲרֹ֜ן לָקַֽח־ל֨וֹ מִבְּנ֤וֹת פּֽוּטִיאֵל֙ ל֣וֹ לְאִשָּׁ֔ה וַתֵּ֥לֶד ל֖וֹ אֶת־פִּֽינְחָ֑ס אֵ֗לֶּה רָאשֵׁ֛י אֲב֥וֹת הַלְוִיִּ֖ם לְמִשְׁפְּחֹתָֽם׃
\rashi{\rashiDH{מבנות פוטיאל. }מזרע יתרו שֶׁפִּטֵּם עגלים לעבודת אלילים, ומזרע יוסף שפטפט ביצרו (ב״ב קט׃)׃ 
}}
{וְאֶלְעָזָר בַּר אַהֲרֹן נְסֵיב לֵיהּ מִבְּנָת פּוּטִיאֵל לֵיהּ לְאִתּוּ וִילֵידַת לֵיהּ יָת פִּינְחָס אִלֵּין רֵישֵׁי אֲבָהָת לֵיוָאֵי לְזַרְעֲיָתְהוֹן׃}
{And Eleazar Aaron’s son took him one of the daughters of Putiel to wife; and she bore him Phinehas. These are the heads of the fathers’ houses of the Levites according to their families.}{\arabic{verse}}
\threeverse{\arabic{verse}}%Ex.6:26
{ה֥וּא אַהֲרֹ֖ן וּמֹשֶׁ֑ה אֲשֶׁ֨ר אָמַ֤ר יְהֹוָה֙ לָהֶ֔ם הוֹצִ֜יאוּ אֶת־בְּנֵ֧י יִשְׂרָאֵ֛ל מֵאֶ֥רֶץ מִצְרַ֖יִם עַל־צִבְאֹתָֽם׃
\rashi{\rashiDH{הוא אהרן ומשה. }אלו שהוזכרו למעלה שילדה יוכבד לעמרם. הוא אהרן ומשה אשר אמר ה׳, יש מקומות שמקדים אהרן למשה ויש מקומות שמקדים משה לאהרן, לומר לך ששקולין כאחד׃ }\rashi{\rashiDH{על צבאותם. }בצבאותם, כל צבאם לשבטיהם, יש על, שאינו אלא במקום אות אחת, וְעַל חַרְבְּךָ תִּחְיֶה (בראשית כז, מ), כמו בחרבך. עֲמַדְתֶּם עַל חַרְבְּכֶם (יחזקאל לג, כו), כמו בחרבכם׃ }}
{הוּא אַהֲרֹן וּמֹשֶׁה דַּאֲמַר יְיָ לְהוֹן אַפִּיקוּ יָת בְּנֵי יִשְׂרָאֵל מֵאַרְעָא דְּמִצְרַיִם עַל חֵילֵיהוֹן׃}
{These are that Aaron and Moses, to whom the \lord\space said: ‘Bring out the children of Israel from the land of Egypt according to their hosts.’}{\arabic{verse}}
\threeverse{\arabic{verse}}%Ex.6:27
{הֵ֗ם הַֽמְדַבְּרִים֙ אֶל־פַּרְעֹ֣ה מֶֽלֶךְ־מִצְרַ֔יִם לְהוֹצִ֥יא אֶת־בְּנֵֽי־יִשְׂרָאֵ֖ל מִמִּצְרָ֑יִם ה֥וּא מֹשֶׁ֖ה וְאַהֲרֹֽן׃
\rashi{\rashiDH{הם המדברים וגו׳. }הם שנצטוו הם שקיימו׃}\rashi{\rashiDH{הוא משה ואהרן. }הם בשליחותם ובצדקתם מתחלה ועד סוף׃}}
{אִנּוּן דִּמְמַלְּלִין עִם פַּרְעֹה מַלְכָּא דְּמִצְרַיִם לְאַפָּקָא יָת בְּנֵי יִשְׂרָאֵל מִמִּצְרָיִם הוּא מֹשֶׁה וְאַהֲרֹן׃}
{These are they that spoke to Pharaoh king of Egypt, to bring out the children of Israel from Egypt. These are that Moses and Aaron.}{\arabic{verse}}
\threeverse{\arabic{verse}}%Ex.6:28
{וַיְהִ֗י בְּי֨וֹם דִּבֶּ֧ר יְהֹוָ֛ה אֶל־מֹשֶׁ֖ה בְּאֶ֥רֶץ מִצְרָֽיִם׃ \setuma         
\rashi{\rashiDH{ויהי ביום דבר וגו׳. }מחובר למקרא שלאחריו׃}}
{וַהֲוָה בְּיוֹמָא דְּמַלֵּיל יְיָ עִם מֹשֶׁה בְּאַרְעָא דְּמִצְרָיִם׃}
{And it came to pass on the day when the \lord\space spoke unto Moses in the land of Egypt,}{\arabic{verse}}
\threeverse{\aliya{שלישי}}%Ex.6:29
{וַיְדַבֵּ֧ר יְהֹוָ֛ה אֶל־מֹשֶׁ֥ה לֵּאמֹ֖ר אֲנִ֣י יְהֹוָ֑ה דַּבֵּ֗ר אֶל־פַּרְעֹה֙ מֶ֣לֶךְ מִצְרַ֔יִם אֵ֛ת כׇּל־אֲשֶׁ֥ר אֲנִ֖י דֹּבֵ֥ר אֵלֶֽיךָ׃
\rashi{\rashiDH{וידבר ה׳. }הוא הדבור עצמו האמור למעלה בא דבר אל פרעה מלך מצרים, אלא מתוך שהפסיק הענין כדי ליחסם, חזר הענין עליו להתחיל בו׃ }\rashi{\rashiDH{אני ה׳. }כדאי אני לשלחך ולקיים דברי שליחותי׃}}
{וּמַלֵּיל יְיָ עִם מֹשֶׁה לְמֵימַר אֲנָא יְיָ מַלֵּיל עִם פַּרְעֹה מַלְכָּא דְּמִצְרַיִם יָת כָּל דַּאֲנָא מְמַלֵּיל עִמָּךְ׃}
{that the \lord\space spoke unto Moses, saying: ‘I am the \lord; speak thou unto Pharaoh king of Egypt all that I speak unto thee.’}{\arabic{verse}}
\threeverse{\arabic{verse}}%Ex.6:30
{וַיֹּ֥אמֶר מֹשֶׁ֖ה לִפְנֵ֣י יְהֹוָ֑ה הֵ֤ן אֲנִי֙ עֲרַ֣ל שְׂפָתַ֔יִם וְאֵ֕יךְ יִשְׁמַ֥ע אֵלַ֖י פַּרְעֹֽה׃ \petucha 
\rashi{\rashiDH{ויאמר משה לפני ה׳. }היא האמירה שאמר למעלה הן בני ישראל לא שמעו אלי, וְשָׁנָה הכתוב כאן, כיון שהפסיק הענין, וכך היא הַשִּׁיטָה, כאדם האומר נחזור על הראשונות׃ }}
{וַאֲמַר מֹשֶׁה קֳדָם יְיָ הָא אֲנָא יַקִּיר מַמְלַל וְאֵיכְדֵין יְקַבֵּיל מִנִּי פַּרְעֹה׃}
{And Moses said before the \lord: ‘Behold, I am of uncircumcised lips, and how shall Pharaoh hearken unto me?’}{\arabic{verse}}
\newperek
\threeverse{\Roman{chap}}%Ex.7:1
{וַיֹּ֤אמֶר יְהֹוָה֙ אֶל־מֹשֶׁ֔ה רְאֵ֛ה נְתַתִּ֥יךָ אֱלֹהִ֖ים לְפַרְעֹ֑ה וְאַהֲרֹ֥ן אָחִ֖יךָ יִהְיֶ֥ה נְבִיאֶֽךָ׃
\rashi{\rashiDH{נתתיך אלהים לפרעה. }שופט ורודה לרדותו במכות ויסורין׃}\rashi{\rashiDH{יהיה נביאך. }כתרגומו יְהֵי מְתוּרְגְּמָנָךְ, וכן כל לשון נבואה, אדם המכריז ומשמיע לעם דברי תוכחות, והוא מגזרת נִיב שְׂפָתָיִם (ישעי׳ נז, יט), יָנוּב חָכְמָה (משלי י, לא), וַיְכַל מֵהִתְנַבֹּות דשמואל (שמואל־א י, יג), ובלע״ז קוראין לו פרידי״גר }}
{וַאֲמַר יְיָ לְמֹשֶׁה חֲזִי דְּמַנִּיתָךְ רָב לְפַרְעֹה וְאַהֲרֹן אֲחוּךְ יְהֵי מְתוּרְגְּמָנָךְ׃}
{And the \lord\space said unto Moses: ‘See, I have set thee in God’s stead to Pharaoh; and Aaron thy brother shall be thy prophet.}{\Roman{chap}}
\threeverse{\arabic{verse}}%Ex.7:2
{אַתָּ֣ה תְדַבֵּ֔ר אֵ֖ת כׇּל־אֲשֶׁ֣ר אֲצַוֶּ֑ךָּ וְאַהֲרֹ֤ן אָחִ֙יךָ֙ יְדַבֵּ֣ר אֶל־פַּרְעֹ֔ה וְשִׁלַּ֥ח אֶת־בְּנֵֽי־יִשְׂרָאֵ֖ל מֵאַרְצֽוֹ׃
\rashi{\rashiDH{אתה תדבר. }פעם אחת כל שליחות ושליחות כפי ששמעת מפי, ואהרן אחיך ימליצנו ויטעימנו באזני פרעה׃ }}
{אַתְּ תְּמַלֵּיל יָת כָּל דַּאֲפַקְּדִנָּךְ וְאַהֲרֹן אֲחוּךְ יְמַלֵּיל עִם פַּרְעֹה וִישַׁלַּח יָת בְּנֵי יִשְׂרָאֵל מֵאַרְעֵיהּ׃}
{Thou shalt speak all that I command thee; and Aaron thy brother shall speak unto Pharaoh, that he let the children of Israel go out of his land.}{\arabic{verse}}
\threeverse{\arabic{verse}}%Ex.7:3
{וַאֲנִ֥י אַקְשֶׁ֖ה אֶת־לֵ֣ב פַּרְעֹ֑ה וְהִרְבֵּיתִ֧י אֶת־אֹתֹתַ֛י וְאֶת־מוֹפְתַ֖י בְּאֶ֥רֶץ מִצְרָֽיִם׃
\rashi{\rashiDH{ואני אקשה. }מאחר שהרשיע והתריס כנגדי, וגלוי לפני שאין נחת רוח באומות עובדי אלילים, לתת לב שלם לשוב, טוב לי שיתקשה לבו, למען הרבות בו אותותי ותכירו את גבורותי, וכן מדתו של הקדוש ב״ה, מביא פורענות על האומות עובדי אלילים, כדי שישמעו ישראל וייראו, שנאמר הִכְרַתִּי גֹויִם נָשַמּוּ פִּנֹּותָם וגו׳ (צפניה ג, ו), אָמַרְתִּי אַךְ תִּירְאִי אֹותִי תִּקְחִי מוּסָר (שם ז), ואף על פי כן בחמש מכות הראשונות לא נאמר ויחזק ה׳ את לב פרעה, אלא ויחזק לב פרעה. (ועיין ברא״ם שגורס כאן דבור המתחיל בלכתך לשוב עד שמתים בידך, וכדלעיל בפרשת שמות בפסוק בלכתך לשוב ע״ש)׃ }}
{וַאֲנָא אַקְשֵׁי יָת לִבָּא דְּפַרְעֹה וְאַסְגֵּי יָת אָתְוָתַי וְיָת מוֹפְתַי בְּאַרְעָא דְּמִצְרָיִם׃}
{And I will harden Pharaoh’s heart, and multiply My signs and My wonders in the land of Egypt.}{\arabic{verse}}
\threeverse{\arabic{verse}}%Ex.7:4
{וְלֹֽא־יִשְׁמַ֤ע אֲלֵכֶם֙ פַּרְעֹ֔ה וְנָתַתִּ֥י אֶת־יָדִ֖י בְּמִצְרָ֑יִם וְהוֹצֵאתִ֨י אֶת־צִבְאֹתַ֜י אֶת־עַמִּ֤י בְנֵֽי־יִשְׂרָאֵל֙ מֵאֶ֣רֶץ מִצְרַ֔יִם בִּשְׁפָטִ֖ים גְּדֹלִֽים׃
\rashi{\rashiDH{את ידי. }יד ממש, להכות בהם׃ 
}}
{וְלָא יְקַבֵּיל מִנְּכוֹן פַּרְעֹה וְאֶתֵּין יָת מַחַת גְּבוּרְתִי בְּמִצְרָיִם וְאַפֵּיק יָת חֵילַי יָת עַמִּי בְנֵי יִשְׂרָאֵל מֵאַרְעָא דְּמִצְרַיִם בְּדִינִין רַבְרְבִין׃}
{But Pharaoh will not hearken unto you, and I will lay My hand upon Egypt, and bring forth My hosts, My people the children of Israel, out of the land of Egypt, by great judgments.}{\arabic{verse}}
\threeverse{\arabic{verse}}%Ex.7:5
{וְיָדְע֤וּ מִצְרַ֙יִם֙ כִּֽי־אֲנִ֣י יְהֹוָ֔ה בִּנְטֹתִ֥י אֶת־יָדִ֖י עַל־מִצְרָ֑יִם וְהוֹצֵאתִ֥י אֶת־בְּנֵֽי־יִשְׂרָאֵ֖ל מִתּוֹכָֽם׃}
{וְיִדְּעוּן מִצְרָאֵי אֲרֵי אֲנָא יְיָ כַּד אֲרֵים יָת מַחַת גְּבוּרְתִי עַל מִצְרָיִם וְאַפֵּיק יָת בְּנֵי יִשְׂרָאֵל מִבֵּינֵיהוֹן׃}
{And the Egyptians shall know that I am the \lord, when I stretch forth My hand upon Egypt, and bring out the children of Israel from among them.’}{\arabic{verse}}
\threeverse{\arabic{verse}}%Ex.7:6
{וַיַּ֥עַשׂ מֹשֶׁ֖ה וְאַהֲרֹ֑ן כַּאֲשֶׁ֨ר צִוָּ֧ה יְהֹוָ֛ה אֹתָ֖ם כֵּ֥ן עָשֽׂוּ׃}
{וַעֲבַד מֹשֶׁה וְאַהֲרֹן כְּמָא דְּפַקֵּיד יְיָ יָתְהוֹן כֵּן עֲבַדוּ׃}
{And Moses and Aaron did so; as the \lord\space commanded them, so did they.}{\arabic{verse}}
\threeverse{\arabic{verse}}%Ex.7:7
{וּמֹשֶׁה֙ בֶּן־שְׁמֹנִ֣ים שָׁנָ֔ה וְאַֽהֲרֹ֔ן בֶּן־שָׁלֹ֥שׁ וּשְׁמֹנִ֖ים שָׁנָ֑ה בְּדַבְּרָ֖ם אֶל־פַּרְעֹֽה׃ \petucha }
{וּמֹשֶׁה בַּר תְּמָנַן שְׁנִין וְאַהֲרֹן בַּר תְּמָנַן וּתְלָת שְׁנִין בְּמַלָּלוּתְהוֹן עִם פַּרְעֹה׃}
{And Moses was fourscore years old, and Aaron fourscore and three years old, when they spoke unto Pharaoh.}{\arabic{verse}}
\threeverse{\aliya{רביעי}}%Ex.7:8
{וַיֹּ֣אמֶר יְהֹוָ֔ה אֶל־מֹשֶׁ֥ה וְאֶֽל־אַהֲרֹ֖ן לֵאמֹֽר׃}
{וַאֲמַר יְיָ לְמֹשֶׁה וּלְאַהֲרֹן לְמֵימַר׃}
{And the \lord\space spoke unto Moses and unto Aaron, saying:}{\arabic{verse}}
\threeverse{\arabic{verse}}%Ex.7:9
{כִּי֩ יְדַבֵּ֨ר אֲלֵכֶ֤ם פַּרְעֹה֙ לֵאמֹ֔ר תְּנ֥וּ לָכֶ֖ם מוֹפֵ֑ת וְאָמַרְתָּ֣ אֶֽל־אַהֲרֹ֗ן קַ֧ח אֶֽת־מַטְּךָ֛ וְהַשְׁלֵ֥ךְ לִפְנֵֽי־פַרְעֹ֖ה יְהִ֥י לְתַנִּֽין׃
\rashi{\rashiDH{מופת. }אות, להודיע שיש צורך (צרוך) במי ששולח אתכם׃ }}
{אֲרֵי יְמַלֵּיל עִמְּכוֹן פַּרְעֹה לְמֵימַר הַבוּ לְכוֹן אָתָא וְתֵימַר לְאַהֲרֹן סַב יָת חוּטְרָךְ וּרְמִי קֳדָם פַּרְעֹה יְהֵי לְתַנִּינָא׃}
{’When Pharaoh shall speak unto you, saying: Show a wonder for you; then thou shalt say unto Aaron: Take thy rod, and cast it down before Pharaoh, that it become a serpent.’}{\arabic{verse}}
\threeverse{\arabic{verse}}%Ex.7:10
{וַיָּבֹ֨א מֹשֶׁ֤ה וְאַהֲרֹן֙ אֶל־פַּרְעֹ֔ה וַיַּ֣עֲשׂוּ כֵ֔ן כַּאֲשֶׁ֖ר צִוָּ֣ה יְהֹוָ֑ה וַיַּשְׁלֵ֨ךְ אַהֲרֹ֜ן אֶת־מַטֵּ֗הוּ לִפְנֵ֥י פַרְעֹ֛ה וְלִפְנֵ֥י עֲבָדָ֖יו וַיְהִ֥י לְתַנִּֽין׃
\rashi{\rashiDH{לתנין. }נחש׃ 
}}
{וְעָאל מֹשֶׁה וְאַהֲרֹן לְוָת פַּרְעֹה וַעֲבַדוּ כֵן כְּמָא דְּפַקֵּיד יְיָ וּרְמָא אַהֲרֹן יָת חוּטְרֵיהּ קֳדָם פַּרְעֹה וּקְדָם עַבְדּוֹהִי וַהֲוָה לְתַנִּינָא׃}
{And Moses and Aaron went in unto Pharaoh, and they did so, as the \lord\space had commanded; and Aaron cast down his rod before Pharaoh and before his servants, and it became a serpent.}{\arabic{verse}}
\threeverse{\arabic{verse}}%Ex.7:11
{וַיִּקְרָא֙ גַּם־פַּרְעֹ֔ה לַֽחֲכָמִ֖ים וְלַֽמְכַשְּׁפִ֑ים וַיַּֽעֲשׂ֨וּ גַם־הֵ֜ם חַרְטֻמֵּ֥י מִצְרַ֛יִם בְּלַהֲטֵיהֶ֖ם כֵּֽן׃
\rashi{\rashiDH{בלהטיהם }בְּלַחֲשֵׁיהוֹן, ואין לו דמיון במקרא, ויש לדמות לו לַהַט הַחֶרֶב הַמִּתְהַפֶּכֶת (בראשית ג, כד), דומה שהיא מתהפכת על ידי לחש׃ }}
{וּקְרָא אַף פַּרְעֹה לְחַכִּימַיָּא וּלְחָרָשַׁיָּא וַעֲבַדוּ אַף אִנּוּן חָרָשֵׁי מִצְרַיִם בְּלַחֲשֵׁיהוֹן כֵּן׃}
{Then Pharaoh also called for the wise men and the sorcerers; and they also, the magicians of Egypt, did in like manner with their secret arts.}{\arabic{verse}}
\threeverse{\arabic{verse}}%Ex.7:12
{וַיַּשְׁלִ֙יכוּ֙ אִ֣ישׁ מַטֵּ֔הוּ וַיִּהְי֖וּ לְתַנִּינִ֑ם וַיִּבְלַ֥ע מַטֵּֽה־אַהֲרֹ֖ן אֶת־מַטֹּתָֽם׃
\rashi{\rashiDH{ויבלע מטה אהרן. }מאחר שחזר ונעשה מטה בלע את כלן (שבת צז)׃}}
{וּרְמוֹ גְּבַר חוּטְרֵיהּ וַהֲווֹ לְתַנִּינִין וּבְלַע חוּטְרָא דְּאַהֲרֹן יָת חוּטְרֵיהוֹן׃}
{For they cast down every man his rod, and they became serpents; but Aaron’s rod swallowed up their rods.}{\arabic{verse}}
\threeverse{\arabic{verse}}%Ex.7:13
{וַיֶּחֱזַק֙ לֵ֣ב פַּרְעֹ֔ה וְלֹ֥א שָׁמַ֖ע אֲלֵהֶ֑ם כַּאֲשֶׁ֖ר דִּבֶּ֥ר יְהֹוָֽה׃ \setuma         }
{וְאִתַּקַּף לִבָּא דְּפַרְעֹה וְלָא קַבֵּיל מִנְּהוֹן כְּמָא דְּמַלֵּיל יְיָ׃}
{And Pharaoh’s heart was hardened, and he hearkened not unto them; as the \lord\space had spoken.}{\arabic{verse}}
\threeverse{\arabic{verse}}%Ex.7:14
{וַיֹּ֤אמֶר יְהֹוָה֙ אֶל־מֹשֶׁ֔ה כָּבֵ֖ד לֵ֣ב פַּרְעֹ֑ה מֵאֵ֖ן לְשַׁלַּ֥ח הָעָֽם׃
\rashi{\rashiDH{כבד. }תרגומו יַקִּיר, ולא אתיקר, מפני שהוא שם דבר, כמו כִּי כָּבֵד מִמְּךָ הַדָּבָר (שמות יח, יח)׃ 
}}
{וַאֲמַר יְיָ לְמֹשֶׁה אִתְיַקַּר לִבָּא דְּפַרְעֹה סָרֵיב לְשַׁלָּחָא עַמָּא׃}
{And the \lord\space said unto Moses: ‘Pharaoh’s heart is stubborn, he refuseth to let the people go.}{\arabic{verse}}
\threeverse{\arabic{verse}}%Ex.7:15
{לֵ֣ךְ אֶל־פַּרְעֹ֞ה בַּבֹּ֗קֶר הִנֵּה֙ יֹצֵ֣א הַמַּ֔יְמָה וְנִצַּבְתָּ֥ לִקְרָאת֖וֹ עַל־שְׂפַ֣ת הַיְאֹ֑ר וְהַמַּטֶּ֛ה אֲשֶׁר־נֶהְפַּ֥ךְ לְנָחָ֖שׁ תִּקַּ֥ח בְּיָדֶֽךָ׃
\rashi{\rashiDH{הנה יצא המימה. }לנקביו, שהיה עושה עצמו אלוה, ואומר שאינו צריך לנקביו, ומשכים ויוצא לנילוס ועושה שם צרכיו (שמו״ר ט, ז)׃ }}
{אִיזֵיל לְוָת פַּרְעֹה בְּצַפְרָא הָא נָפֵיק לְמַיָּא וְתִתְעַתַּד לְקַדָּמוּתֵיהּ עַל כֵּיף נַהְרָא וְחוּטְרָא דְּאִתְהֲפֵיךְ לְחִוְיָא תִּסַּב בִּידָךְ׃}
{Get thee unto Pharaoh in the morning; lo, he goeth out unto the water; and thou shalt stand by the river’s brink to meet him; and the rod which was turned to a serpent shalt thou take in thy hand.}{\arabic{verse}}
\threeverse{\arabic{verse}}%Ex.7:16
{וְאָמַרְתָּ֣ אֵלָ֗יו יְהֹוָ֞ה אֱלֹהֵ֤י הָעִבְרִים֙ שְׁלָחַ֤נִי אֵלֶ֙יךָ֙ לֵאמֹ֔ר שַׁלַּח֙ אֶת־עַמִּ֔י וְיַֽעַבְדֻ֖נִי בַּמִּדְבָּ֑ר וְהִנֵּ֥ה לֹא־שָׁמַ֖עְתָּ עַד־כֹּֽה׃
\rashi{\rashiDH{עד כה. }עד הנה. ומדרשו, עד שתשמע ממני מכת בכורות, שאפתח בה בכה כֹּה אָמַר ה׳ כֲַּחצֹת הַלַּיְלָה׃ }}
{וְתֵימַר לֵיהּ יְיָ אֱלָהָא דִּיהוּדָאֵי שַׁלְחַנִי לְוָתָךְ לְמֵימַר שַׁלַּח יָת עַמִּי וְיִפְלְחוּן קֳדָמַי בְּמַדְבְּרָא וְהָא לָא קַבֵּילְתָּא עַד כְּעַן׃}
{And thou shalt say unto him: The \lord, the God of the Hebrews, hath sent me unto thee, saying: Let My people go, that they may serve Me in the wilderness; and, behold, hitherto thou hast not hearkened;}{\arabic{verse}}
\threeverse{\arabic{verse}}%Ex.7:17
{כֹּ֚ה אָמַ֣ר יְהֹוָ֔ה בְּזֹ֣את תֵּדַ֔ע כִּ֖י אֲנִ֣י יְהֹוָ֑ה הִנֵּ֨ה אָנֹכִ֜י מַכֶּ֣ה \legarmeh  בַּמַּטֶּ֣ה אֲשֶׁר־בְּיָדִ֗י עַל־הַמַּ֛יִם אֲשֶׁ֥ר בַּיְאֹ֖ר וְנֶהֶפְכ֥וּ לְדָֽם׃
\rashi{\rashiDH{ונהפכו לדם. }לפי שאין גשמים יורדים במצרים, ונילוס עולה ומשקה את הארץ, ומצרים עובדים לנילוס, לפיכך הלקה את יראתם ואחר כך הלקה אותם׃ }}
{כִּדְנָן אֲמַר יְיָ בְּדָא תִדַּע אֲרֵי אֲנָא יְיָ הָא אֲנָא מָחֵי בְחוּטְרָא דִּבְיְדִי עַל מַיָּא דִּבְנַהְרָא וְיִתְהַפְכוּן לִדְמָא׃}
{thus saith the \lord: In this thou shalt know that I am the \lord—behold, I will smite with the rod that is in my hand upon the waters which are in the river, and they shall be turned to blood.}{\arabic{verse}}
\threeverse{\arabic{verse}}%Ex.7:18
{וְהַדָּגָ֧ה אֲשֶׁר־בַּיְאֹ֛ר תָּמ֖וּת וּבָאַ֣שׁ הַיְאֹ֑ר וְנִלְא֣וּ מִצְרַ֔יִם לִשְׁתּ֥וֹת מַ֖יִם מִן־הַיְאֹֽר׃ \setuma         
\rashi{\rashiDH{ונלאו מצרים. }לבקש רפואה למי היאור שיהיו ראויין לשתות׃}}
{וְנוּנֵי דִּבְנַהְרָא יְמוּתוּן וְיִסְרֵי נַהְרָא וְיִלְאוֹן מִצְרָאֵי לְמִשְׁתֵּי מַיָּא מִן נַהְרָא׃}
{And the fish that are in the river shall die, and the river shall become foul; and the Egyptians shall loathe to drink water from the river.’}{\arabic{verse}}
\threeverse{\arabic{verse}}%Ex.7:19
{וַיֹּ֨אמֶר יְהֹוָ֜ה אֶל־מֹשֶׁ֗ה אֱמֹ֣ר אֶֽל־אַהֲרֹ֡ן קַ֣ח מַטְּךָ֣ וּנְטֵֽה־יָדְךָ֩ עַל־מֵימֵ֨י מִצְרַ֜יִם עַֽל־נַהֲרֹתָ֣ם \legarmeh  עַל־יְאֹרֵיהֶ֣ם וְעַל־אַגְמֵיהֶ֗ם וְעַ֛ל כׇּל־מִקְוֵ֥ה מֵימֵיהֶ֖ם וְיִֽהְיוּ־דָ֑ם וְהָ֤יָה דָם֙ בְּכׇל־אֶ֣רֶץ מִצְרַ֔יִם וּבָעֵצִ֖ים וּבָאֲבָנִֽים׃
\rashi{\rashiDH{אמר אל אהרן. }לפי שהגין היאור על משה כשנשלך לתוכו, לפיכך לא לקה על ידו לא בדם ולא בצפרדעים, ולקה על ידי אהרן׃ }\rashi{\rashiDH{נהרותם. }הם נהרות המושכים כעין נהרות שלנו׃}\rashi{\rashiDH{יאוריהם. }הם בריכות נגרים העשויות בידי אדם משפת הנהר לשדות, ונילוס מימיו מתברכים ועולה דרך היאורים ומשקה השדות׃ }\rashi{\rashiDH{אגמיהם. }קבוצת מים שאינן נובעין ואין מושכין, אלא עומדין במקום אחד, וקורין לו אשטנ״ק׃ }\rashi{\rashiDH{בכל ארץ מצרים. }אף במרחצאות ובאמבטאות שבבתים׃}\rashi{\rashiDH{ובעצים ובאבנים. }מים שבכלי עץ ובכלי אבן׃ 
}}
{וַאֲמַר יְיָ לְמֹשֶׁה אֵימַר לְאַהֲרֹן סַב חוּטְרָךְ וַאֲרֵים יְדָךְ עַל מַיָּא דְּמִצְרָאֵי עַל נַהְרֵיהוֹן עַל אֲרִתֵּיהוֹן וְעַל אַגְמֵיהוֹן וְעַל כָּל בֵּית כְּנֵישָׁת מֵימֵיהוֹן וִיהוֹן דְּמָא וִיהֵי דְמָא בְּכָל אַרְעָא דְּמִצְרַיִם וּבְמָנֵי אָעָא וּבְמָנֵי אַבְנָא׃}
{And the \lord\space said unto Moses: ‘Say unto Aaron: Take thy rod, and stretch out thy hand over the waters of Egypt, over their rivers, over their streams, and over their pools, and over all their ponds of water, that they may become blood; and there shall be blood throughout all the land of Egypt, both in vessels of wood and in vessels of stone.’}{\arabic{verse}}
\threeverse{\arabic{verse}}%Ex.7:20
{וַיַּֽעֲשׂוּ־כֵן֩ מֹשֶׁ֨ה וְאַהֲרֹ֜ן כַּאֲשֶׁ֣ר \legarmeh  צִוָּ֣ה יְהֹוָ֗ה וַיָּ֤רֶם בַּמַּטֶּה֙ וַיַּ֤ךְ אֶת־הַמַּ֙יִם֙ אֲשֶׁ֣ר בַּיְאֹ֔ר לְעֵינֵ֣י פַרְעֹ֔ה וּלְעֵינֵ֖י עֲבָדָ֑יו וַיֵּהָ֥פְכ֛וּ כׇּל־הַמַּ֥יִם אֲשֶׁר־בַּיְאֹ֖ר לְדָֽם׃}
{וַעֲבַדוּ כֵן מֹשֶׁה וְאַהֲרֹן כְּמָא דְּפַקֵּיד יְיָ וַאֲרֵים בְּחוּטְרָא וּמְחָא יָת מַיָּא דִּבְנַהְרָא לְעֵינֵי פַרְעֹה וּלְעֵינֵי עַבְדּוֹהִי וְאִתְהֲפִיכוּ כָּל מַיָּא דִּבְנַהְרָא לִדְמָא׃}
{And Moses and Aaron did so, as the \lord\space commanded; and he lifted up the rod, and smote the waters that were in the river, in the sight of Pharaoh, and in the sight of his servants; and all the waters that were in the river were turned to blood.}{\arabic{verse}}
\threeverse{\arabic{verse}}%Ex.7:21
{וְהַדָּגָ֨ה אֲשֶׁר־בַּיְאֹ֥ר מֵ֙תָה֙ וַיִּבְאַ֣שׁ הַיְאֹ֔ר וְלֹא־יָכְל֣וּ מִצְרַ֔יִם לִשְׁתּ֥וֹת מַ֖יִם מִן־הַיְאֹ֑ר וַיְהִ֥י הַדָּ֖ם בְּכׇל־אֶ֥רֶץ מִצְרָֽיִם׃}
{וְנוּנֵי דִּבְנַהְרָא מִיתוּ וּסְרִי נַהְרָא וְלָא יְכִילוּ מִצְרָאֵי לְמִשְׁתֵּי מַיָּא מִן נַהְרָא וַהֲוָה דְּמָא בְּכָל אַרְעָא דְּמִצְרָיִם׃}
{And the fish that were in the river died; and the river became foul, and the Egyptians could not drink water from the river; and the blood was throughout all the land of Egypt.}{\arabic{verse}}
\threeverse{\arabic{verse}}%Ex.7:22
{וַיַּֽעֲשׂוּ־כֵ֛ן חַרְטֻמֵּ֥י מִצְרַ֖יִם בְּלָטֵיהֶ֑ם וַיֶּחֱזַ֤ק לֵב־פַּרְעֹה֙ וְלֹא־שָׁמַ֣ע אֲלֵהֶ֔ם כַּאֲשֶׁ֖ר דִּבֶּ֥ר יְהֹוָֽה׃
\rashi{\rashiDH{בלטיהם. }לחש שאומרין אותו בלט ובחשאי. ורבותינו אמרו, בלטיהם מעשה שדים, בלהטיהם מעשה כשפים (סנהדרין סז׃)׃ }\rashi{\rashiDH{ויחזק לב פרעה. }לומר על ידי מכשפות אתם עושים כן, תבן אתם מכניסין לָעֲפָרִיִּים (מנחות פה.) עיר שכולה תבן, אף אתם מביאין מכשפות למצרים שכולה כשפים׃ }}
{וַעֲבַדוּ כֵן חָרָשֵׁי מִצְרַיִם בְּלַחֲשֵׁיהוֹן וְאִתַּקַּף לִבָּא דְּפַרְעֹה וְלָא קַבֵּיל מִנְּהוֹן כְּמָא דְּמַלֵּיל יְיָ׃}
{And the magicians of Egypt did in like manner with their secret arts; and Pharaoh’s heart was hardened, and he hearkened not unto them; as the \lord\space had spoken.}{\arabic{verse}}
\threeverse{\arabic{verse}}%Ex.7:23
{וַיִּ֣פֶן פַּרְעֹ֔ה וַיָּבֹ֖א אֶל־בֵּית֑וֹ וְלֹא־שָׁ֥ת לִבּ֖וֹ גַּם־לָזֹֽאת׃
\rashi{\rashiDH{גם לזאת. }למופת המטה שנהפך לתנין ולא לזה של דם׃}}
{וְאִתְפְּנִי פַרְעֹה וְעָאל לְבֵיתֵיהּ וְלָא שַׁוִּי לִבֵּיהּ אַף לְדָא׃}
{And Pharaoh turned and went into his house, neither did he lay even this to heart.}{\arabic{verse}}
\threeverse{\arabic{verse}}%Ex.7:24
{וַיַּחְפְּר֧וּ כׇל־מִצְרַ֛יִם סְבִיבֹ֥ת הַיְאֹ֖ר מַ֣יִם לִשְׁתּ֑וֹת כִּ֣י לֹ֤א יָֽכְלוּ֙ לִשְׁתֹּ֔ת מִמֵּימֵ֖י הַיְאֹֽר׃}
{וַחֲפַרוּ כָל מִצְרָאֵי סַחְרָנוּת נַהְרָא מַיָּא לְמִשְׁתֵּי אֲרֵי לָא יְכִילוּ לְמִשְׁתֵּי מִמַּיָּא דִּבְנַהְרָא׃}
{And all the Egyptians digged round about the river for water to drink; for they could not drink of the water of the river.}{\arabic{verse}}
\threeverse{\arabic{verse}}%Ex.7:25
{וַיִּמָּלֵ֖א שִׁבְעַ֣ת יָמִ֑ים אַחֲרֵ֥י הַכּוֹת־יְהֹוָ֖ה אֶת־הַיְאֹֽר׃ \petucha 
\rashi{\rashiDH{וימלא. }מנין שבעת ימים שלא שב היאור לקדמותו, שהיתה המכה משמשת רביע חדש, ושלשה חלקים היה מעיד ומתרה בהם (שמו״ר ט, יב)׃ }}
{וּשְׁלִימוּ שִׁבְעָא יוֹמִין בָּתַר דִּמְחָא יְיָ יָת נַהְרָא׃}
{And seven days were fulfilled, after that the \lord\space had smitten the river.}{\arabic{verse}}
\threeverse{\arabic{verse}}%Ex.7:26
{וַיֹּ֤אמֶר יְהֹוָה֙ אֶל־מֹשֶׁ֔ה בֹּ֖א אֶל־פַּרְעֹ֑ה וְאָמַרְתָּ֣ אֵלָ֗יו כֹּ֚ה אָמַ֣ר יְהֹוָ֔ה שַׁלַּ֥ח אֶת־עַמִּ֖י וְיַֽעַבְדֻֽנִי׃}
{וַאֲמַר יְיָ לְמֹשֶׁה עוֹל לְוָת פַּרְעֹה וְתֵימַר לֵיהּ כִּדְנָן אֲמַר יְיָ שַׁלַּח יָת עַמִּי וְיִפְלְחוּן קֳדָמָי׃}
{And the \lord\space spoke unto Moses: ‘Go in unto Pharaoh, and say unto him: Thus saith the \lord: Let My people go, that they may serve Me.}{\arabic{verse}}
\threeverse{\arabic{verse}}%Ex.7:27
{וְאִם־מָאֵ֥ן אַתָּ֖ה לְשַׁלֵּ֑חַ הִנֵּ֣ה אָנֹכִ֗י נֹגֵ֛ף אֶת־כׇּל־גְּבוּלְךָ֖ בַּֽצְפַרְדְּעִֽים׃
\rashi{\rashiDH{ואם מאן אתה. }ואם סרבן אתה. מאן כמו ממאן, מסרב, אלא כינה האדם על שם המפעל, כמו שָׁלֵו (איוב טז, יב) וְשֹׁקֵט (ירמיה מח, יא), סַר וְזָעֵף (מלכים ־א כ, מג)׃ }\rashi{\rashiDH{נגף את כל גבולך. }מכה, וכן כל לשון מגפה אינו לשון מיתה אלא לשון מכה, וכן וְנָגְפוּ אִשָּׁה הָרָה (שמות כא, כב) אינו לשון מיתה, וכן וּבְטֶרֶם יִתְנַגְּפוּ רגליכם (ירמי׳ יג, טז), פֶּן תִּגֹּף בָּאֶבֶן רַגְלֶךָ (תהלים צא, יב), וּלְאֶבֶן נֶגֶף (ישעי׳ ח, יד)׃ }}
{וְאִם מְסָרֵיב אַתְּ לְשַׁלָּחָא הָא אֲנָא מָחֵי יָת כָּל תְּחוּמָךְ בְּעוּרְדְּעָנַיָּא׃}
{And if thou refuse to let them go, behold, I will smite all thy borders with frogs.}{\arabic{verse}}
\threeverse{\arabic{verse}}%Ex.7:28
{וְשָׁרַ֣ץ הַיְאֹר֮ צְפַרְדְּעִים֒ וְעָלוּ֙ וּבָ֣אוּ בְּבֵיתֶ֔ךָ וּבַחֲדַ֥ר מִשְׁכָּבְךָ֖ וְעַל־מִטָּתֶ֑ךָ וּבְבֵ֤ית עֲבָדֶ֙יךָ֙ וּבְעַמֶּ֔ךָ וּבְתַנּוּרֶ֖יךָ וּבְמִשְׁאֲרוֹתֶֽיךָ׃
\rashi{\rashiDH{ועלו. }מן היאור׃}\rashi{\rashiDH{בביתך. }ואחר כך בבתי עבדיך, הוא התחיל בעצה תחלה, ויאמר אל עמו, וממנו התחילה הפורענות (סוטה יא.  שמו״ר י, ד)׃ }}
{וִירַבֵּי נַהְרָא עוּרְדְּעָנַיָּא וְיִסְּקוּן וְיֵיעֲלוּן בְּבֵיתָךְ וּבְאִדְּרוֹן בֵּית מִשְׁכְּבָךְ וְעַל עַרְסָתָךְ וּבְבֵית עַבְדָךְ וּבְעַמָּךְ וּבְתַנּוּרָךְ וּבְאָצְוָתָךְ׃}
{And the river shall swarm with frogs, which shall go up and come into thy house, and into thy bed-chamber, and upon thy bed, and into the house of thy servants, and upon thy people, and into thine ovens, and into thy kneading-troughs.}{\arabic{verse}}
\threeverse{\arabic{verse}}%Ex.7:29
{וּבְכָ֥ה וּֽבְעַמְּךָ֖ וּבְכׇל־עֲבָדֶ֑יךָ יַעֲל֖וּ הַֽצְפַרְדְּעִֽים׃
\rashi{\rashiDH{ובכה ובעמך. }בתוך מעיהם נכנסים ומקרקרין׃}}
{וּבָךְ וּבְעַמָּךְ וּבְכָל עַבְדָךְ יִסְּקוּן עוּרְדְּעָנַיָּא׃}
{And the frogs shall come up both upon thee, and upon thy people, and upon all thy servants.’}{\arabic{verse}}
\newperek
\threeverse{\Roman{chap}}%Ex.8:1
{וַיֹּ֣אמֶר יְהֹוָה֮ אֶל־מֹשֶׁה֒ אֱמֹ֣ר אֶֽל־אַהֲרֹ֗ן נְטֵ֤ה אֶת־יָדְךָ֙ בְּמַטֶּ֔ךָ עַ֨ל־הַנְּהָרֹ֔ת עַל־הַיְאֹרִ֖ים וְעַל־הָאֲגַמִּ֑ים וְהַ֥עַל אֶת־הַֽצְפַרְדְּעִ֖ים עַל־אֶ֥רֶץ מִצְרָֽיִם׃}
{וַאֲמַר יְיָ לְמֹשֶׁה אֵימַר לְאַהֲרֹן אֲרֵים יָת יְדָךְ בְּחוּטְרָךְ עַל נַהְרַיָּא עַל אֲרִתַּיָּא וְעַל אַגְמַיָּא וְאַסֵּיק יָת עוּרְדְּעָנַיָּא עַל אַרְעָא דְּמִצְרָיִם׃}
{And the \lord\space said unto Moses: ‘Say unto Aaron: Stretch forth thy hand with thy rod over the rivers, over the canals, and over the pools, and cause frogs to come up upon the land of Egypt.’}{\Roman{chap}}
\threeverse{\arabic{verse}}%Ex.8:2
{וַיֵּ֤ט אַהֲרֹן֙ אֶת־יָד֔וֹ עַ֖ל מֵימֵ֣י מִצְרָ֑יִם וַתַּ֙עַל֙ הַצְּפַרְדֵּ֔עַ וַתְּכַ֖ס אֶת־אֶ֥רֶץ מִצְרָֽיִם׃
\rashi{\rashiDH{ותעל הצפרדע. }צפרדע אחת היתה, והיו מַכִּין אותה והיא מתזת נחילים נחילים, זהו מדרשו (שמו״ר י, ה). ופשוטו יש לומר, שרוץ הצפרדעים קורא לשון יחידות, וכן ותהי הכנם, הרחישה גדוליר״א בלע״ז, ואף ותעל הצפרדע גרינולי״רא בלע״ז }}
{וַאֲרֵים אַהֲרֹן יָת יְדֵיהּ עַל מַיָּא דְּמִצְרָאֵי וּסְלִיקוּ עוּרְדְּעָנַיָּא וַחֲפוֹ יָת אַרְעָא דְּמִצְרָיִם׃}
{And Aaron stretched out his hand over the waters of Egypt; and the frogs came up, and covered the land of Egypt.}{\arabic{verse}}
\threeverse{\arabic{verse}}%Ex.8:3
{וַיַּֽעֲשׂוּ־כֵ֥ן הַֽחַרְטֻמִּ֖ים בְּלָטֵיהֶ֑ם וַיַּעֲל֥וּ אֶת־הַֽצְפַרְדְּעִ֖ים עַל־אֶ֥רֶץ מִצְרָֽיִם׃}
{וַעֲבַדוּ כֵן חָרָשַׁיָּא בְּלַחֲשֵׁיהוֹן וְאַסִּיקוּ יָת עוּרְדְּעָנַיָּא עַל אַרְעָא דְּמִצְרָיִם׃}
{And the magicians did in like manner with their secret arts, and brought up frogs upon the land of Egypt.}{\arabic{verse}}
\threeverse{\arabic{verse}}%Ex.8:4
{וַיִּקְרָ֨א פַרְעֹ֜ה לְמֹשֶׁ֣ה וּֽלְאַהֲרֹ֗ן וַיֹּ֙אמֶר֙ הַעְתִּ֣ירוּ אֶל־יְהֹוָ֔ה וְיָסֵר֙ הַֽצְפַרְדְּעִ֔ים מִמֶּ֖נִּי וּמֵֽעַמִּ֑י וַאֲשַׁלְּחָה֙ אֶת־הָעָ֔ם וְיִזְבְּח֖וּ לַיהֹוָֽה׃}
{וּקְרָא פַרְעֹה לְמֹשֶׁה וּלְאַהֲרֹן וַאֲמַר צַלּוֹ קֳדָם יְיָ וְיַעְדֵּי עוּרְדְּעָנַיָּא מִנִּי וּמֵעַמִּי וַאֲשַׁלַּח יָת עַמָּא וִידַבְּחוּן קֳדָם יְיָ׃}
{Then Pharaoh called for Moses and Aaron, and said: ‘Entreat the \lord, that He take away the frogs from me, and from my people; and I will let the people go, that they may sacrifice unto the \lord.’}{\arabic{verse}}
\threeverse{\arabic{verse}}%Ex.8:5
{וַיֹּ֨אמֶר מֹשֶׁ֣ה לְפַרְעֹה֮ הִתְפָּאֵ֣ר עָלַי֒ לְמָתַ֣י \legarmeh  אַעְתִּ֣יר לְךָ֗ וְלַעֲבָדֶ֙יךָ֙ וּֽלְעַמְּךָ֔ לְהַכְרִית֙ הַֽצְפַרְדְּעִ֔ים מִמְּךָ֖ וּמִבָּתֶּ֑יךָ רַ֥ק בַּיְאֹ֖ר תִּשָּׁאַֽרְנָה׃
\rashi{\rashiDH{התפאר עלי. }כמו הֲיִתְפָּאֵר הַגַרְזֶן עַל הַחֹצֵב בֹּו (ישעי׳ י, טו), משתבח לומר אני גדול ממך ונטי״ר בלע״ז, וכן התפאר עלי, השתבח להתחכם ולשאול דבר גדול ולומר שלא אוכל לעשותו׃ }\rashi{\rashiDH{למתי אעתיר לך. }את אשר אעתיר לך היום על הכרתת הצפרדעים, למתי תרצה שיכרתו, ותראה אם אשלים דברי למועד שתקבע לי. אלו נאמר מתי אעתיר, היה משמע מתי אתפלל, עכשיו שנאמר למתי, משמע אני היום אתפלל עליך שיכרתו הצפרדעים לזמן שתקבע עלי, אמור לאיזה יום תרצה שיכרתו. אעתיר העתירו והעתרתי, ולא נאמר אעתר עתרו ועתרתי, מפני שכל לשון עתר הרבות פלל הוא, וכאשר יאמר הרבו ארבה והרביתי לשון הפעיל, כך יאמר, אעתיר העתירו והעתרתי דברים, ואב לכולם וְהַעְתַּרְתֶּם עָלַי דִּבְרֵיכֶם (יחזקאל לה, יג), הרביתם׃ }}
{וַאֲמַר מֹשֶׁה לְפַרְעֹה שְׁאַל לָךְ גְּבוּרָא הַב לָךְ זְמָן לְאִמַּתִּי אֲצַלֵּי עֲלָךְ וְעַל עַבְדָךְ וְעַל עַמָּךְ לְשֵׁיצָאָה עוּרְדְּעָנַיָּא מִנָּךְ וּמִבָּתָּךְ לְחוֹד דִּבְנַהְרָא יִשְׁתְּאַרוּן׃}
{And Moses said unto Pharaoh: ‘Have thou this glory over me; against what time shall I entreat for thee, and for thy servants, and for thy people, that the frogs be destroyed from thee and thy houses, and remain in the river only?’}{\arabic{verse}}
\threeverse{\arabic{verse}}%Ex.8:6
{וַיֹּ֖אמֶר לְמָחָ֑ר וַיֹּ֙אמֶר֙ כִּדְבָ֣רְךָ֔ לְמַ֣עַן תֵּדַ֔ע כִּי־אֵ֖ין כַּיהֹוָ֥ה אֱלֹהֵֽינוּ׃
\rashi{\rashiDH{ויאמר למחר. }התפלל היום שיכרתו למחר׃ 
}}
{וַאֲמַר לִמְחַר וַאֲמַר כְּפִתְגָמָךְ בְּדִיל דְּתִדַּע אֲרֵי לֵית כַּייָ אֱלָהַנָא׃}
{And he said: ‘Against to-morrow.’ And he said: ‘Be it according to thy word; that thou mayest know that there is none like unto the \lord\space our God.}{\arabic{verse}}
\threeverse{\aliya{חמישי}}%Ex.8:7
{וְסָר֣וּ הַֽצְפַרְדְּעִ֗ים מִמְּךָ֙ וּמִבָּ֣תֶּ֔יךָ וּמֵעֲבָדֶ֖יךָ וּמֵעַמֶּ֑ךָ רַ֥ק בַּיְאֹ֖ר תִּשָּׁאַֽרְנָה׃}
{וְיִעְדּוֹן עוּרְדְּעָנַיָּא מִנָּךְ וּמִבָּתָּךְ וּמֵעַבְדָךְ וּמֵעַמָּךְ לְחוֹד דִּבְנַהְרָא יִשְׁתְּאַרוּן׃}
{And the frogs shall depart from thee, and from thy houses, and from thy servants, and from thy people; they shall remain in the river only.’}{\arabic{verse}}
\threeverse{\arabic{verse}}%Ex.8:8
{וַיֵּצֵ֥א מֹשֶׁ֛ה וְאַהֲרֹ֖ן מֵעִ֣ם פַּרְעֹ֑ה וַיִּצְעַ֤ק מֹשֶׁה֙ אֶל־יְהֹוָ֔ה עַל־דְּבַ֥ר הַֽצְפַרְדְּעִ֖ים אֲשֶׁר־שָׂ֥ם לְפַרְעֹֽה׃
\rashi{\rashiDH{ויצא. ויצעק. }מיד, שיכרתו למחר׃ 
}}
{וּנְפַק מֹשֶׁה וְאַהֲרֹן מִלְּוָת פַּרְעֹה וְצַלִּי מֹשֶׁה קֳדָם יְיָ עַל עֵיסַק עוּרְדְּעָנַיָּא דְּשַׁוִּי לְפַרְעֹה׃}
{And Moses and Aaron went out from Pharaoh; and Moses cried unto the \lord\space concerning the frogs, which He had brought upon Pharaoh.}{\arabic{verse}}
\threeverse{\arabic{verse}}%Ex.8:9
{וַיַּ֥עַשׂ יְהֹוָ֖ה כִּדְבַ֣ר מֹשֶׁ֑ה וַיָּמֻ֙תוּ֙ הַֽצְפַרְדְּעִ֔ים מִן־הַבָּתִּ֥ים מִן־הַחֲצֵרֹ֖ת וּמִן־הַשָּׂדֹֽת׃}
{וַעֲבַד יְיָ כְּפִתְגָמָא דְּמֹשֶׁה וּמִיתוּ עוּרְדְּעָנַיָּא מִן בָּתַּיָּא מִן דָּרָתָא וּמִן חַקְלָתָא׃}
{And the \lord\space did according to the word of Moses; and the frogs died out of the houses, out of the courts, and out of the fields.}{\arabic{verse}}
\threeverse{\arabic{verse}}%Ex.8:10
{וַיִּצְבְּר֥וּ אֹתָ֖ם חֳמָרִ֣ם חֳמָרִ֑ם וַתִּבְאַ֖שׁ הָאָֽרֶץ׃
\rashi{\rashiDH{חמרם חמרם. }צְבוּרִים צְבוּרִים, כתרגומו דְּגוֹרִין, גַּלִּין׃ }}
{וּכְנַשׁוּ יָתְהוֹן דְּגוֹרִין דְּגוֹרִין וּסְרִיאוּ עַל אַרְעָא׃}
{And they gathered them together in heaps; and the land stank.}{\arabic{verse}}
\threeverse{\arabic{verse}}%Ex.8:11
{וַיַּ֣רְא פַּרְעֹ֗ה כִּ֤י הָֽיְתָה֙ הָֽרְוָחָ֔ה וְהַכְבֵּד֙ אֶת־לִבּ֔וֹ וְלֹ֥א שָׁמַ֖ע אֲלֵהֶ֑ם כַּאֲשֶׁ֖ר דִּבֶּ֥ר יְהֹוָֽה׃ \setuma         
\rashi{\rashiDH{והכבד את לבו. }לשון פעול הוא, כמו הָלֹוךְ וְנָסֹועַ (בראשית יב, ט), וכן וְהַכֹּות אֶת מֹואָב (מלכים־ב ג, כד), וְשָׁאֹול לֹו בֵּאלֹהִים (שמואל־א כב, יג), הַכֵּה וּפָצֹעַ (מלכים־א כ, לז)׃ }\rashi{\rashiDH{כאשר דבר ה׳. }והיכן דבר, ולא ישמע אליכם פרעה׃ 
}}
{וַחֲזָא פַּרְעֹה אֲרֵי הֲוָת רְוַחְתָּא וְיַקַּר יָת לִבֵּיהּ וְלָא קַבֵּיל מִנְּהוֹן כְּמָא דְּמַלֵּיל יְיָ׃}
{But when Pharaoh saw that there was respite, he hardened his heart, and hearkened not unto them; as the \lord\space had spoken.}{\arabic{verse}}
\threeverse{\arabic{verse}}%Ex.8:12
{וַיֹּ֣אמֶר יְהֹוָה֮ אֶל־מֹשֶׁה֒ אֱמֹר֙ אֶֽל־אַהֲרֹ֔ן נְטֵ֣ה אֶֽת־מַטְּךָ֔ וְהַ֖ךְ אֶת־עֲפַ֣ר הָאָ֑רֶץ וְהָיָ֥ה לְכִנִּ֖ם בְּכׇל־אֶ֥רֶץ מִצְרָֽיִם׃
\rashi{\rashiDH{אמר אל אהרן. }לא היה העפר כדאי ללקות על ידי משה, לפי שהגין עליו כשהרג את המצרי ויטמנהו בחול, ולקה על ידי אהרן׃ }}
{וַאֲמַר יְיָ לְמֹשֶׁה אֵימַר לְאַהֲרֹן אֲרֵים יָת חוּטְרָךְ וּמְחִי יָת עַפְרָא דְּאַרְעָא וִיהֵי לְקַלְמְתָא בְּכָל אַרְעָא דְּמִצְרָיִם׃}
{And the \lord\space said unto Moses: ‘Say unto Aaron: Stretch out thy rod, and smite the dust of the earth, that it may become gnats throughout all the land of Egypt.’}{\arabic{verse}}
\threeverse{\arabic{verse}}%Ex.8:13
{וַיַּֽעֲשׂוּ־כֵ֗ן וַיֵּט֩ אַהֲרֹ֨ן אֶת־יָד֤וֹ בְמַטֵּ֙הוּ֙ וַיַּךְ֙ אֶת־עֲפַ֣ר הָאָ֔רֶץ וַתְּהִי֙ הַכִּנָּ֔ם בָּאָדָ֖ם וּבַבְּהֵמָ֑ה כׇּל־עֲפַ֥ר הָאָ֛רֶץ הָיָ֥ה כִנִּ֖ים בְּכׇל־אֶ֥רֶץ מִצְרָֽיִם׃
\rashi{\rashiDH{ותהי הכנם. }הָרְחִישָׁה, פדוליר״א בלע״ז׃ }}
{וַעֲבַדוּ כֵן וַאֲרֵים אַהֲרֹן יָת יְדֵיהּ בְּחוּטְרֵיהּ וּמְחָא יָת עַפְרָא דְּאַרְעָא וַהֲוָת קַלְמְתָא בַּאֲנָשָׁא וּבִבְעִירָא כָּל עַפְרָא דְּאַרְעָא הֲוָת קַלְמְתָא בְּכָל אַרְעָא דְּמִצְרָיִם׃}
{And they did so; and Aaron stretched out his hand with his rod, and smote the dust of the earth, and there were gnats upon man, and upon beast; all the dust of the earth became gnats throughout all the land of Egypt.}{\arabic{verse}}
\threeverse{\arabic{verse}}%Ex.8:14
{וַיַּעֲשׂוּ־כֵ֨ן הַחַרְטֻמִּ֧ים בְּלָטֵיהֶ֛ם לְהוֹצִ֥יא אֶת־הַכִּנִּ֖ים וְלֹ֣א יָכֹ֑לוּ וַתְּהִי֙ הַכִּנָּ֔ם בָּאָדָ֖ם וּבַבְּהֵמָֽה׃
\rashi{\rashiDH{להוציא את הכנים. }לבראותם (נ״א ולהוציאם) ממקום אחר׃ }\rashi{\rashiDH{ולא יכלו. }שאין השד שולט על בריה פחותה מכשעורה׃}}
{וַעֲבַדוּ כֵן חָרָשַׁיָּא בְּלַחֲשֵׁיהוֹן לְאַפָּקָא יָת קַלְמְתָא וְלָא יְכִילוּ וַהֲוָת קַלְמְתָא בַּאֲנָשָׁא וּבִבְעִירָא׃}
{And the magicians did so with their secret arts to bring forth gnats, but they could not; and there were gnats upon man, and upon beast.}{\arabic{verse}}
\threeverse{\arabic{verse}}%Ex.8:15
{וַיֹּאמְר֤וּ הַֽחַרְטֻמִּם֙ אֶל־פַּרְעֹ֔ה אֶצְבַּ֥ע אֱלֹהִ֖ים הִ֑וא וַיֶּחֱזַ֤ק לֵב־פַּרְעֹה֙ וְלֹֽא־שָׁמַ֣ע אֲלֵהֶ֔ם כַּאֲשֶׁ֖ר דִּבֶּ֥ר יְהֹוָֽה׃ \setuma         
\rashi{\rashiDH{אצבע אלהים היא. }מכה זו אינה על ידי כשפים, מאת המקום היא׃ }\rashi{\rashiDH{כאשר דבר ה׳. }ולא ישמע אליכם פרעה׃}}
{וַאֲמַרוּ חָרָשַׁיָּא לְפַרְעֹה מַחָא מִן קֳדָם יְיָ הִיא וְאִתַּקַּף לִבָּא דְּפַרְעֹה וְלָא קַבֵּיל מִנְּהוֹן כְּמָא דְּמַלֵּיל יְיָ׃}
{Then the magicians said unto Pharaoh: ‘This is the finger of God’; and Pharaoh’s heart was hardened, and he hearkened not unto them; as the \lord\space had spoken.}{\arabic{verse}}
\threeverse{\arabic{verse}}%Ex.8:16
{וַיֹּ֨אמֶר יְהֹוָ֜ה אֶל־מֹשֶׁ֗ה הַשְׁכֵּ֤ם בַּבֹּ֙קֶר֙ וְהִתְיַצֵּב֙ לִפְנֵ֣י פַרְעֹ֔ה הִנֵּ֖ה יוֹצֵ֣א הַמָּ֑יְמָה וְאָמַרְתָּ֣ אֵלָ֗יו כֹּ֚ה אָמַ֣ר יְהֹוָ֔ה שַׁלַּ֥ח עַמִּ֖י וְיַֽעַבְדֻֽנִי׃}
{וַאֲמַר יְיָ לְמֹשֶׁה אַקְדֵּים בְּצַפְרָא וְאִתְעַתַּד קֳדָם פַּרְעֹה הָא נָפֵיק לְמַיָּא וְתֵימַר לֵיהּ כִּדְנָן אֲמַר יְיָ שַׁלַּח עַמִּי וְיִפְלְחוּן קֳדָמָי׃}
{And the \lord\space said unto Moses: ‘Rise up early in the morning, and stand before Pharaoh; lo, he cometh forth to the water; and say unto him: Thus saith the \lord: Let My people go, that they may serve Me.}{\arabic{verse}}
\threeverse{\arabic{verse}}%Ex.8:17
{כִּ֣י אִם־אֵינְךָ֮ מְשַׁלֵּ֣חַ אֶת־עַמִּי֒ הִנְנִי֩ מַשְׁלִ֨יחַ בְּךָ֜ וּבַעֲבָדֶ֧יךָ וּֽבְעַמְּךָ֛ וּבְבָתֶּ֖יךָ אֶת־הֶעָרֹ֑ב וּמָ֨לְא֜וּ בָּתֵּ֤י מִצְרַ֙יִם֙ אֶת־הֶ֣עָרֹ֔ב וְגַ֥ם הָאֲדָמָ֖ה אֲשֶׁר־הֵ֥ם עָלֶֽיהָ׃
\rashi{\rashiDH{משליח בך. }מגרה בך, וכן וְשֶׁן בְּהֵמֹת אֲשַלַּח בָּם (דברים לב, כד), לשון שסוי אינציט״ר בלע״ז׃ 
}\rashi{\rashiDH{את הערב. }כל מיני חיות רעות ונחשים ועקרבים בערבוביא, והיו משחיתים בהם. ויש טעם בדבר באגדה בכל מכה ומכה למה זו ולמה זו, בטכסיסי מלחמות מלכים בא עליהם, כסדר מלכות, כשצרה על עיר, בתחלה מקלקל מעיינותיה, ואחר כך תוקעין עליהם ומריעין בשופרות ליראם ולבהלם, וכן הצפרדעים מקרקרים והומים וכו׳, כדאי׳ במדרש רבי תנחומא (בא ד)׃ }}
{אֲרֵי אִם לָיְתָךְ מְשַׁלַּח יָת עַמִּי הָאֲנָא מַשְׁלַח בָּךְ וּבְעַבְדָךְ וּבְעַמָּךְ וּבְבָתָּךְ יָת עָרוֹבָא וְיִתְמְלוֹן בָּתֵּי מִצְרַיִם יָת עָרוֹבָא וְאַף אַרְעָא דְּאִנּוּן עֲלַהּ׃}
{Else, if thou wilt not let My people go, behold, I will send swarms of flies upon thee, and upon thy servants, and upon thy people, and into thy houses; and the houses of the Egyptians shall be full of swarms of flies, and also the ground whereon they are.}{\arabic{verse}}
\threeverse{\arabic{verse}}%Ex.8:18
{וְהִפְלֵיתִי֩ בַיּ֨וֹם הַה֜וּא אֶת־אֶ֣רֶץ גֹּ֗שֶׁן אֲשֶׁ֤ר עַמִּי֙ עֹמֵ֣ד עָלֶ֔יהָ לְבִלְתִּ֥י הֱיֽוֹת־שָׁ֖ם עָרֹ֑ב לְמַ֣עַן תֵּדַ֔ע כִּ֛י אֲנִ֥י יְהֹוָ֖ה בְּקֶ֥רֶב הָאָֽרֶץ׃
\rashi{\rashiDH{והפליתי. }והפרשתי, וכן וְהִפְלָה ה׳ (שמות ט, ד), וכן לֹא נִפְלֵאת הִיא מִמְּךָ (דברים ל, יא), לא מובדלת ומופרשת היא ממך׃ 
}\rashi{\rashiDH{למען תדע כי אני ה׳ בקרב הארץ. }אע״פ ששכינתי בשמים, גזרתי מתקיימת בתחתונים׃ }}
{וְאַפְרֵישׁ בְּיוֹמָא הַהוּא יָת אַרְעָא דְּגֹשֶׁן דְּעַמִּי שָׁרֵי עֲלַהּ בְּדִיל דְּלָא לְמִהְוֵי תַּמָּן עָרוֹבָא בְּדִיל דְּתִדַּע אֲרֵי אֲנָא יְיָ שַׁלִּיט בְּגוֹ אַרְעָא׃}
{And I will set apart in that day the land of Goshen, in which My people dwell, that no swarms of flies shall be there; to the end that thou mayest know that I am the \lord\space in the midst of the earth.}{\arabic{verse}}
\threeverse{\aliya{ששי}}%Ex.8:19
{וְשַׂמְתִּ֣י פְדֻ֔ת בֵּ֥ין עַמִּ֖י וּבֵ֣ין עַמֶּ֑ךָ לְמָחָ֥ר יִהְיֶ֖ה הָאֹ֥ת הַזֶּֽה׃
\rashi{\rashiDH{ושמתי פדות. }שיבדיל בין עמי ובין עמך׃}}
{וַאֲשַׁוֵּי פוּרְקָן לְעַמִּי וְעַל עַמָּךְ אַיְתִי מַחָא לִמְחַר יְהֵי אָתָא הָדֵין׃}
{And I will put a division between My people and thy people—by to-morrow shall this sign be.’}{\arabic{verse}}
\threeverse{\arabic{verse}}%Ex.8:20
{וַיַּ֤עַשׂ יְהֹוָה֙ כֵּ֔ן וַיָּבֹא֙ עָרֹ֣ב כָּבֵ֔ד בֵּ֥יתָה פַרְעֹ֖ה וּבֵ֣ית עֲבָדָ֑יו וּבְכׇל־אֶ֧רֶץ מִצְרַ֛יִם תִּשָּׁחֵ֥ת הָאָ֖רֶץ מִפְּנֵ֥י הֶעָרֹֽב׃
\rashi{\rashiDH{תשחת הארץ. }נשחתה הארץ, אִתְחַבָּלַת אַרְעָא׃ }}
{וַעֲבַד יְיָ כֵּן וַאֲתָא עָרוֹבָא תַּקִּיף לְבֵית פַּרְעֹה וּלְבֵית עַבְדּוֹהִי וּבְכָל אַרְעָא דְּמִצְרַיִם אִתְחַבַּלַת אַרְעָא מִן קֳדָם עָרוֹבָא׃}
{And the \lord\space did so; and there came grievous swarms of flies into the house of Pharaoh, and into his servants’ houses; and in all the land of Egypt the land was ruined by reason of the swarms of flies.}{\arabic{verse}}
\threeverse{\arabic{verse}}%Ex.8:21
{וַיִּקְרָ֣א פַרְעֹ֔ה אֶל־מֹשֶׁ֖ה וּֽלְאַהֲרֹ֑ן וַיֹּ֗אמֶר לְכ֛וּ זִבְח֥וּ לֵאלֹֽהֵיכֶ֖ם בָּאָֽרֶץ׃
\rashi{\rashiDH{זבחו לאלהיכם בארץ. }במקומכם, ולא תלכו במדבר׃ 
}}
{וּקְרָא פַרְעֹה לְמֹשֶׁה וּלְאַהֲרֹן וַאֲמַר אִיזִילוּ דַּבַּחוּ קֳדָם אֱלָהֲכוֹן בְּאַרְעָא׃}
{And Pharaoh called for Moses and for Aaron, and said: ‘Go ye, sacrifice to your God in the land.’}{\arabic{verse}}
\threeverse{\arabic{verse}}%Ex.8:22
{וַיֹּ֣אמֶר מֹשֶׁ֗ה לֹ֤א נָכוֹן֙ לַעֲשׂ֣וֹת כֵּ֔ן כִּ֚י תּוֹעֲבַ֣ת מִצְרַ֔יִם נִזְבַּ֖ח לַיהֹוָ֣ה אֱלֹהֵ֑ינוּ הֵ֣ן נִזְבַּ֞ח אֶת־תּוֹעֲבַ֥ת מִצְרַ֛יִם לְעֵינֵיהֶ֖ם וְלֹ֥א יִסְקְלֻֽנוּ׃
\rashi{\rashiDH{תועבת מצרים. }יראת מצרים, כמו וּלְמִלְכֹּם תֹּועֲבַת בְּנֵי עַמֹּון (מלכים־ב כג, יג), ואצל ישראל קורא אותם תועבה. ועוד יש לומר בלשון אחר תועבת מצרים, דבר שנאוי הוא למצרים זביחה שאנו זובחים, שהרי יראתם אנו זובחים׃ }\rashi{\rashiDH{ולא יסקלנו. }בתמיה׃ 
}}
{וַאֲמַר מֹשֶׁה לָא תָקֵין לְמֶעֱבַד כֵּן אֲרֵי בְעִירָא דְּמִצְרָאֵי דָּחֲלִין לֵיהּ מִנֵּיהּ אֲנַחְנָא נָסְבִין לְדַבָּחָא קֳדָם יְיָ אֱלָהַנָא הָא נְדַבַּח יָת בְּעִירָא דְּמִצְרָאֵי דָּחֲלִין לֵיהּ וְאִנּוּן יְהוֹן חָזַן הֲלָא יֵימְרוּן לְמִרְגְּמַנָא׃}
{And Moses said: ‘It is not meet so to do; for we shall sacrifice the abomination of the Egyptians to the \lord\space our God; lo, if we sacrifice the abomination of the Egyptians before their eyes, will they not stone us?}{\arabic{verse}}
\threeverse{\arabic{verse}}%Ex.8:23
{דֶּ֚רֶךְ שְׁלֹ֣שֶׁת יָמִ֔ים נֵלֵ֖ךְ בַּמִּדְבָּ֑ר וְזָבַ֙חְנוּ֙ לַֽיהֹוָ֣ה אֱלֹהֵ֔ינוּ כַּאֲשֶׁ֖ר יֹאמַ֥ר אֵלֵֽינוּ׃}
{מַהְלַךְ תְּלָתָא יוֹמִין נֵיזֵיל בְּמַדְבְּרָא וּנְדַבַּח קֳדָם יְיָ אֱלָהַנָא כְּמָא דְּיֵימַר לַנָא׃}
{We will go three days’ journey into the wilderness, and sacrifice to the \lord\space our God, as He shall command us.’}{\arabic{verse}}
\threeverse{\arabic{verse}}%Ex.8:24
{וַיֹּ֣אמֶר פַּרְעֹ֗ה אָנֹכִ֞י אֲשַׁלַּ֤ח אֶתְכֶם֙ וּזְבַחְתֶּ֞ם לַיהֹוָ֤ה אֱלֹֽהֵיכֶם֙ בַּמִּדְבָּ֔ר רַ֛ק הַרְחֵ֥ק לֹא־תַרְחִ֖יקוּ לָלֶ֑כֶת הַעְתִּ֖ירוּ בַּעֲדִֽי׃}
{וַאֲמַר פַּרְעֹה אֲנָא אֲשַׁלַּח יָתְכוֹן וּתְדַבְּחוּן קֳדָם יְיָ אֱלָהֲכוֹן בְּמַדְבְּרָא לְחוֹד אַרְחָקָא לָא תְרַחֲקוּן לְמֵיזַל צַלּוֹ עֲלָי׃}
{And Pharaoh said: ‘I will let you go, that ye may sacrifice to the \lord\space your God in the wilderness; only ye shall not go very far away; entreat for me.’}{\arabic{verse}}
\threeverse{\arabic{verse}}%Ex.8:25
{וַיֹּ֣אמֶר מֹשֶׁ֗ה הִנֵּ֨ה אָנֹכִ֜י יוֹצֵ֤א מֵֽעִמָּךְ֙ וְהַעְתַּרְתִּ֣י אֶל־יְהֹוָ֔ה וְסָ֣ר הֶעָרֹ֗ב מִפַּרְעֹ֛ה מֵעֲבָדָ֥יו וּמֵעַמּ֖וֹ מָחָ֑ר רַ֗ק אַל־יֹסֵ֤ף פַּרְעֹה֙ הָתֵ֔ל לְבִלְתִּי֙ שַׁלַּ֣ח אֶת־הָעָ֔ם לִזְבֹּ֖חַ לַֽיהֹוָֽה׃
\rashi{\rashiDH{התל. }כמו להתל׃}}
{וַאֲמַר מֹשֶׁה הָא אֲנָא נָפֵיק מֵעִמָּךְ וַאֲצַלֵּי קֳדָם יְיָ וְיִעְדֵּי עָרוֹבָא מִפַּרְעֹה מֵעַבְדּוֹהִי וּמֵעַמֵּיהּ מְחַר לְחוֹד לָא יוֹסֵיף פַּרְעֹה לְשַׁקָּרָא בְּדִיל דְּלָא לְשַׁלָּחָא יָת עַמָּא לְדַבָּחָא קֳדָם יְיָ׃}
{And Moses said: ‘Behold, I go out from thee, and I will entreat the \lord\space that the swarms of flies may depart from Pharaoh, from his servants, and from his people, tomorrow; only let not Pharaoh deal deceitfully any more in not letting the people go to sacrifice to the \lord.’}{\arabic{verse}}
\threeverse{\arabic{verse}}%Ex.8:26
{וַיֵּצֵ֥א מֹשֶׁ֖ה מֵעִ֣ם פַּרְעֹ֑ה וַיֶּעְתַּ֖ר אֶל־יְהֹוָֽה׃
\rashi{\rashiDH{ויעתר אל ה׳. }נתאמץ בתפלה, וכן אם בא לומר ויעתיר, היה יכול לומר, ומשמע וירבה בתפלה, עכשיו כשהוא אומר בלשון ויפעל, משמע וירבה להתפלל׃ }}
{וּנְפַק מֹשֶׁה מִלְּוָת פַּרְעֹה וְצַלִּי קֳדָם יְיָ׃}
{And Moses went out from Pharaoh, and entreated the \lord.}{\arabic{verse}}
\threeverse{\arabic{verse}}%Ex.8:27
{וַיַּ֤עַשׂ יְהֹוָה֙ כִּדְבַ֣ר מֹשֶׁ֔ה וַיָּ֙סַר֙ הֶעָרֹ֔ב מִפַּרְעֹ֖ה מֵעֲבָדָ֣יו וּמֵעַמּ֑וֹ לֹ֥א נִשְׁאַ֖ר אֶחָֽד׃
\rashi{\rashiDH{ויסר הערוב. }ולא מתו כמו שמתו הצפרדעים, שאם מתו יהיה להם הנאה בעורות׃ }}
{וַעֲבַד יְיָ כְּפִתְגָמָא דְּמֹשֶׁה וְאַעְדִּי עָרוֹבָא מִפַּרְעֹה מֵעַבְדּוֹהִי וּמֵעַמֵּיהּ לָא אִשְׁתְּאַר חַד׃}
{And the \lord\space did according to the word of Moses; and He removed the swarms of flies from Pharaoh, from his servants, and from his people; there remained not one.}{\arabic{verse}}
\threeverse{\arabic{verse}}%Ex.8:28
{וַיַּכְבֵּ֤ד פַּרְעֹה֙ אֶת־לִבּ֔וֹ גַּ֖ם בַּפַּ֣עַם הַזֹּ֑את וְלֹ֥א שִׁלַּ֖ח אֶת־הָעָֽם׃ \petucha 
\rashi{\rashiDH{גם בפעם הזאת. }אע״פ שאמר אנכי אשלח אתכם, לא קיים הבטחתו׃ }}
{וְיַקַּר פַּרְעֹה יָת לִבֵּיהּ אַף בְּזִמְנָא הָדָא וְלָא שַׁלַּח יָת עַמָּא׃}
{And Pharaoh hardened his heart this time also, and he did not let the people go.}{\arabic{verse}}
\newperek
\threeverse{\Roman{chap}}%Ex.9:1
{וַיֹּ֤אמֶר יְהֹוָה֙ אֶל־מֹשֶׁ֔ה בֹּ֖א אֶל־פַּרְעֹ֑ה וְדִבַּרְתָּ֣ אֵלָ֗יו כֹּֽה־אָמַ֤ר יְהֹוָה֙ אֱלֹהֵ֣י הָֽעִבְרִ֔ים שַׁלַּ֥ח אֶת־עַמִּ֖י וְיַֽעַבְדֻֽנִי׃}
{וַאֲמַר יְיָ לְמֹשֶׁה עוֹל לְוָת פַּרְעֹה וּתְמַלֵּיל עִמֵּיהּ כִּדְנָן אֲמַר יְיָ אֱלָהָא דִּיהוּדָאֵי שַׁלַּח יָת עַמִּי וְיִפְלְחוּן קֳדָמָי׃}
{Then the \lord\space said unto Moses: ‘Go in unto Pharaoh, and tell him: Thus saith the \lord, the God of the Hebrews: Let My people go, that they may serve Me.}{\Roman{chap}}
\threeverse{\arabic{verse}}%Ex.9:2
{כִּ֛י אִם־מָאֵ֥ן אַתָּ֖ה לְשַׁלֵּ֑חַ וְעוֹדְךָ֖ מַחֲזִ֥יק בָּֽם׃
\rashi{\rashiDH{מחזיק בם. }אוחז בם, כמו וְהֶחֱזִיקָה בִּמְבֻשָׁיו (דברים כה, יא)׃ }}
{אֲרֵי אִם מְסָרֵיב אַתְּ לְשַׁלָּחָא וְעַד כְּעַן אַתְּ מַתְקֵיף בְּהוֹן׃}
{For if thou refuse to let them go, and wilt hold them still,}{\arabic{verse}}
\threeverse{\arabic{verse}}%Ex.9:3
{הִנֵּ֨ה יַד־יְהֹוָ֜ה הוֹיָ֗ה בְּמִקְנְךָ֙ אֲשֶׁ֣ר בַּשָּׂדֶ֔ה בַּסּוּסִ֤ים בַּֽחֲמֹרִים֙ בַּגְּמַלִּ֔ים בַּבָּקָ֖ר וּבַצֹּ֑אן דֶּ֖בֶר כָּבֵ֥ד מְאֹֽד׃
\rashi{\rashiDH{הנה יד ה׳ הויה. }לשון הוה, כי כן יאמר בלשון נקבה, על שעבר היתה, ועל העתיד תהיה, ועל העומד הווה, כמו עושה, רוצה, רועה׃ }}
{הָא מַחָא מִן קֳדָם יְיָ הָוְיָא בִּבְעִירָךְ דִּבְחַקְלָא בְּסוּסָוָתָא בִּחְמָרֵי בְּגַמְלֵי בְּתוֹרֵי וּבְעָנָא מוֹתָא סַגִּי לַחְדָּא׃}
{behold, the hand of the \lord\space is upon thy cattle which are in the field, upon the horses, upon the asses, upon the camels, upon the herds, and upon the flocks; there shall be a very grievous murrain.}{\arabic{verse}}
\threeverse{\arabic{verse}}%Ex.9:4
{וְהִפְלָ֣ה יְהֹוָ֔ה בֵּ֚ין מִקְנֵ֣ה יִשְׂרָאֵ֔ל וּבֵ֖ין מִקְנֵ֣ה מִצְרָ֑יִם וְלֹ֥א יָמ֛וּת מִכׇּל־לִבְנֵ֥י יִשְׂרָאֵ֖ל דָּבָֽר׃
\rashi{\rashiDH{והפלה. }והבדיל׃ 
}}
{וְיַפְרֵישׁ יְיָ בֵּין בְּעִירָא דְּיִשְׂרָאֵל וּבֵין בְּעִירָא דְּמִצְרָאֵי וְלָא יְמוּת מִכֹּלָא לִבְנֵי יִשְׂרָאֵל מִדָּעַם׃}
{And the \lord\space shall make a division between the cattle of Israel and the cattle of Egypt; and there shall nothing die of all that belongeth to the children of Israel.’}{\arabic{verse}}
\threeverse{\arabic{verse}}%Ex.9:5
{וַיָּ֥שֶׂם יְהֹוָ֖ה מוֹעֵ֣ד לֵאמֹ֑ר מָחָ֗ר יַעֲשֶׂ֧ה יְהֹוָ֛ה הַדָּבָ֥ר הַזֶּ֖ה בָּאָֽרֶץ׃}
{וְשַׁוִּי יְיָ זִמְנָא לְמֵימַר מְחַר יַעֲבֵיד יְיָ פִּתְגָמָא הָדֵין בְּאַרְעָא׃}
{And the \lord\space appointed a set time, saying: ‘Tomorrow the \lord\space shall do this thing in the land.’}{\arabic{verse}}
\threeverse{\arabic{verse}}%Ex.9:6
{וַיַּ֨עַשׂ יְהֹוָ֜ה אֶת־הַדָּבָ֤ר הַזֶּה֙ מִֽמׇּחֳרָ֔ת וַיָּ֕מׇת כֹּ֖ל מִקְנֵ֣ה מִצְרָ֑יִם וּמִמִּקְנֵ֥ה בְנֵֽי־יִשְׂרָאֵ֖ל לֹא־מֵ֥ת אֶחָֽד׃}
{וַעֲבַד יְיָ יָת פִּתְגָמָא הָדֵין בְּיוֹמָא דְּבָתְרוֹהִי וּמִית כָּל בְּעִירָא דְּמִצְרָאֵי וּמִבְּעִירָא דִּבְנֵי יִשְׂרָאֵל לָא מִית חַד׃}
{And the \lord\space did that thing on the morrow, and all the cattle of Egypt died; but of the cattle of the children of Israel died not one.}{\arabic{verse}}
\threeverse{\arabic{verse}}%Ex.9:7
{וַיִּשְׁלַ֣ח פַּרְעֹ֔ה וְהִנֵּ֗ה לֹא־מֵ֛ת מִמִּקְנֵ֥ה יִשְׂרָאֵ֖ל עַד־אֶחָ֑ד וַיִּכְבַּד֙ לֵ֣ב פַּרְעֹ֔ה וְלֹ֥א שִׁלַּ֖ח אֶת־הָעָֽם׃ \petucha }
{וּשְׁלַח פַּרְעֹה וְהָא לָא מִית מִבְּעִירָא דְּיִשְׂרָאֵל עַד חַד וְאִתְיַקַּר לִבָּא דְּפַרְעֹה וְלָא שַׁלַּח יָת עַמָּא׃}
{And Pharaoh sent, and, behold, there was not so much as one of the cattle of the Israelites dead. But the heart of Pharaoh was stubborn, and he did not let the people go.}{\arabic{verse}}
\threeverse{\arabic{verse}}%Ex.9:8
{וַיֹּ֣אמֶר יְהֹוָה֮ אֶל־מֹשֶׁ֣ה וְאֶֽל־אַהֲרֹן֒ קְח֤וּ לָכֶם֙ מְלֹ֣א חׇפְנֵיכֶ֔ם פִּ֖יחַ כִּבְשָׁ֑ן וּזְרָק֥וֹ מֹשֶׁ֛ה הַשָּׁמַ֖יְמָה לְעֵינֵ֥י פַרְעֹֽה׃
\rashi{\rashiDH{מלא חפניכם. }ילויינו״ש בלע״ז }\rashi{\rashiDH{פיח כבשן. }דבר הַנִּפָּח מן הגחלים עוממים הנשרפים בכבשן, ובלע״ז אולב״ש. פיח לשון הפחה, שהרוח מפיחן ומפריחן׃ }\rashi{\rashiDH{וזרקו משה. }וכל דבר הנזרק בכח, אינו נזרק אלא ביד אחת, הרי נסים הרבה, אחד שהחזיק קומצו של משה מלא חפנים שלו ושל אהרן, ואחד שהלך האבק על כל ארץ מצרים׃ }}
{וַאֲמַר יְיָ לְמֹשֶׁה וּלְאַהֲרֹן סַבוּ לְכוֹן מְלֵי חוּפְנֵיכוֹן פִּיחַ דְּאַתּוּנָא וְיִזְרְקִנֵּיהּ מֹשֶׁה לְצֵית שְׁמַיָּא לְעֵינֵי פַרְעֹה׃}
{And the \lord\space said unto Moses and unto Aaron: ‘Take to you handfuls of soot of the furnace, and let Moses throw it heavenward in the sight of Pharaoh.}{\arabic{verse}}
\threeverse{\arabic{verse}}%Ex.9:9
{וְהָיָ֣ה לְאָבָ֔ק עַ֖ל כׇּל־אֶ֣רֶץ מִצְרָ֑יִם וְהָיָ֨ה עַל־הָאָדָ֜ם וְעַל־הַבְּהֵמָ֗ה לִשְׁחִ֥ין פֹּרֵ֛חַ אֲבַעְבֻּעֹ֖ת בְּכׇל־אֶ֥רֶץ מִצְרָֽיִם׃
\rashi{\rashiDH{לשחין פרח אבעבועות. }כתרגומו לשחין סַגִּי, אֲבַעְבּוּעִין שעל ידו צומחין בהן בועות׃ }\rashi{\rashiDH{שחין. }לשון חמימות, והרבה יש בלשון משנה, שנה שחונה׃ }}
{וִיהֵי לְאַבְקָא עַל כָּל אַרְעָא דְּמִצְרָיִם וִיהֵי עַל אֲנָשָׁא וְעַל בְּעִירָא לִשְׁחִין סָגֵי אֲבַעְבּוֹעֲיָן בְּכָל אַרְעָא דְּמִצְרָיִם׃}
{And it shall become small dust over all the land of Egypt, and shall be a boil breaking forth with blains upon man and upon beast, throughout all the land of Egypt.’}{\arabic{verse}}
\threeverse{\arabic{verse}}%Ex.9:10
{וַיִּקְח֞וּ אֶת־פִּ֣יחַ הַכִּבְשָׁ֗ן וַיַּֽעַמְדוּ֙ לִפְנֵ֣י פַרְעֹ֔ה וַיִּזְרֹ֥ק אֹת֛וֹ מֹשֶׁ֖ה הַשָּׁמָ֑יְמָה וַיְהִ֗י שְׁחִין֙ אֲבַעְבֻּעֹ֔ת פֹּרֵ֕חַ בָּאָדָ֖ם וּבַבְּהֵמָֽה׃
\rashi{\rashiDH{באדם ובבהמה. }ואם תאמר מאין היו להם הבהמות, והלא כבר נאמר וַיָמָת כֹּל מִקְנֵה מִצְרָיִם, אלא לא נגזרה גזרה אלא על אותן שבשדות בלבד, שנאמר בְּמִקְנְךָ אֲשֶׁר בַּשָׂדֶה, והַיָּרֵא אֶת דְּבַר ה׳ הֵנִיס אֶת מִקְנֵהוּ אֶל הַבָּתִּים. וכן שנויה במכילתא אצל וַיִקַּח שֵׁשׁ מֵאֹות רֶכֶב בָּחוּר (שמות יד, ז)׃ 
}}
{וּנְסִיבוּ יָת פִּיחַ דְּאַתּוּנָא וְקָמוּ קֳדָם פַּרְעֹה וּזְרַק יָתֵיהּ מֹשֶׁה לְצֵית שְׁמַיָּא וַהֲוָה שְׁחִין אֲבַעְבּוֹעֲיָן סָגֵי בַּאֲנָשָׁא וּבִבְעִירָא׃}
{And they took soot of the furnace, and stood before Pharaoh; and Moses threw it up heavenward; and it became a boil breaking forth with blains upon man and upon beast.}{\arabic{verse}}
\threeverse{\arabic{verse}}%Ex.9:11
{וְלֹֽא־יָכְל֣וּ הַֽחַרְטֻמִּ֗ים לַעֲמֹ֛ד לִפְנֵ֥י מֹשֶׁ֖ה מִפְּנֵ֣י הַשְּׁחִ֑ין כִּֽי־הָיָ֣ה הַשְּׁחִ֔ין בַּֽחַרְטֻמִּ֖ם וּבְכׇל־מִצְרָֽיִם׃}
{וְלָא יְכִילוּ חָרָשַׁיָּא לִמְקָם קֳדָם מֹשֶׁה מִן קֳדָם שִׁחְנָא אֲרֵי הֲוָה שִׁחְנָא בְּחָרָשַׁיָּא וּבְכָל מִצְרָאֵי׃}
{And the magicians could not stand before Moses because of the boils; for the boils were upon the magicians, and upon all the Egyptians.}{\arabic{verse}}
\threeverse{\arabic{verse}}%Ex.9:12
{וַיְחַזֵּ֤ק יְהֹוָה֙ אֶת־לֵ֣ב פַּרְעֹ֔ה וְלֹ֥א שָׁמַ֖ע אֲלֵהֶ֑ם כַּאֲשֶׁ֛ר דִּבֶּ֥ר יְהֹוָ֖ה אֶל־מֹשֶֽׁה׃ \setuma         }
{וְתַקֵּיף יְיָ יָת לִבָּא דְּפַרְעֹה וְלָא קַבֵּיל מִנְּהוֹן כְּמָא דְּמַלֵּיל יְיָ עִם מֹשֶׁה׃}
{And the \lord\space hardened the heart of Pharaoh, and he hearkened not unto them; as the \lord\space had spoken unto Moses.}{\arabic{verse}}
\threeverse{\arabic{verse}}%Ex.9:13
{וַיֹּ֤אמֶר יְהֹוָה֙ אֶל־מֹשֶׁ֔ה הַשְׁכֵּ֣ם בַּבֹּ֔קֶר וְהִתְיַצֵּ֖ב לִפְנֵ֣י פַרְעֹ֑ה וְאָמַרְתָּ֣ אֵלָ֗יו כֹּֽה־אָמַ֤ר יְהֹוָה֙ אֱלֹהֵ֣י הָֽעִבְרִ֔ים שַׁלַּ֥ח אֶת־עַמִּ֖י וְיַֽעַבְדֻֽנִי׃}
{וַאֲמַר יְיָ לְמֹשֶׁה אַקְדֵּים בְּצַפְרָא וְאִתְעַתַּד קֳדָם פַּרְעֹה וְתֵימַר לֵיהּ כִּדְנָן אֲמַר יְיָ אֱלָהָא דִּיהוּדָאֵי שַׁלַּח יָת עַמִּי וְיִפְלְחוּן קֳדָמָי׃}
{And the \lord\space said unto Moses: ‘Rise up early in the morning, and stand before Pharaoh, and say unto him: Thus saith the \lord, the God of the Hebrews: Let My people go, that they may serve Me.}{\arabic{verse}}
\threeverse{\arabic{verse}}%Ex.9:14
{כִּ֣י \legarmeh  בַּפַּ֣עַם הַזֹּ֗את אֲנִ֨י שֹׁלֵ֜חַ אֶת־כׇּל־מַגֵּפֹתַי֙ אֶֽל־לִבְּךָ֔ וּבַעֲבָדֶ֖יךָ וּבְעַמֶּ֑ךָ בַּעֲב֣וּר תֵּדַ֔ע כִּ֛י אֵ֥ין כָּמֹ֖נִי בְּכׇל־הָאָֽרֶץ׃
\rashi{\rashiDH{את כל מגפתי. }למדנו מכאן, שמכת בכורות שקולה כנגד כל המכות׃ 
}}
{אֲרֵי בְּזִמְנָא הָדָא אֲנָא שָׁלַח יָת כָּל מַחָתַי בְּלִבָּךְ וּבְעַבְדָךְ וּבְעַמָּךְ בְּדִיל דְּתִדַּע אֲרֵי לֵית דִּכְוָתִי בְּכָל אַרְעָא׃}
{For I will this time send all My plagues upon thy person, and upon thy servants, and upon thy people; that thou mayest know that there is none like Me in all the earth.}{\arabic{verse}}
\threeverse{\arabic{verse}}%Ex.9:15
{כִּ֤י עַתָּה֙ שָׁלַ֣חְתִּי אֶת־יָדִ֔י וָאַ֥ךְ אוֹתְךָ֛ וְאֶֽת־עַמְּךָ֖ בַּדָּ֑בֶר וַתִּכָּחֵ֖ד מִן־הָאָֽרֶץ׃
\rashi{\rashiDH{כי עתה שלחתי את ידי וגו׳. }כי אלו רציתי, כשהיתה ידי במקנך שהכיתים בדבר, שלחתיה והכיתי אותך ואת עמך עם הבהמות׃ \rashiDH{ותכחד מן הארץ. }אבל בעבור זאת העמדתיך וגו׳׃}}
{אֲרֵי כְעַן קָרִיב קֳדָמַי דִּשְׁלַחִית פּוֹן יָת מַחַת גְּבוּרְתִי וּמְחֵית יָתָךְ וְיָת עַמָּךְ בְּמוֹתָא וְאִשְׁתֵּיצִיתָא מִן אַרְעָא׃}
{Surely now I had put forth My hand, and smitten thee and thy people with pestilence, and thou hadst been cut off from the earth.}{\arabic{verse}}
\threeverse{\arabic{verse}}%Ex.9:16
{וְאוּלָ֗ם בַּעֲב֥וּר זֹאת֙ הֶעֱמַדְתִּ֔יךָ בַּעֲב֖וּר הַרְאֹתְךָ֣ אֶת־כֹּחִ֑י וּלְמַ֛עַן סַפֵּ֥ר שְׁמִ֖י בְּכׇל־הָאָֽרֶץ׃}
{וּבְרַם בְּדִיל דָּא קַיֵּימְתָּךְ בְּדִיל לְאַחְזָיוּתָךְ יָת חֵילִי וּבְדִיל דִּיהוֹן מִשְׁתָּעַן גְּבוּרַת שְׁמִי בְּכָל אַרְעָא׃}
{But in very deed for this cause have I made thee to stand, to show thee My power, and that My name may be declared throughout all the earth.}{\arabic{verse}}
\threeverse{\aliya{שביעי}}%Ex.9:17
{עוֹדְךָ֖ מִסְתּוֹלֵ֣ל בְּעַמִּ֑י לְבִלְתִּ֖י שַׁלְּחָֽם׃
\rashi{\rashiDH{עודך מסתולל בעמי. }כתרגומו כְּבֵישַׁת בֵּיהּ בְּעַמִּי, והיא מגזרת מסלה דמתרגמינן אוֹרַח כְּבוּשָׁא, ובלע״ז קלקי״ר וכבר פירשתי בסוף ויהי מקץ, כל תיבה שתחלת יסודה סמ״ך והיא באה לדבר בלשון מתפעל, נותן התי״ו של שמוש באמצע אותיות של עיקר, כגון זו, וכגון וְיִסְתַּבֵּל הֶחָגָב (קהלת יב, ה), מגזרת סבל. כִּי תִּשְׂתָּרֵר עָלֵינוּ (דניאל ז, ח), מגזרת שר ונגיד. וכן מִשְׂתַּכַּל הֲוִית (דניאל ז, ח)׃ }}
{עַד כְּעַן כְּבֵישְׁתְּ לֵיהּ בְּעַמִּי בְּדִיל דְּלָא לְשַׁלָּחוּתְהוֹן׃}
{As yet exaltest thou thyself against My people, that thou wilt not let them go?}{\arabic{verse}}
\threeverse{\arabic{verse}}%Ex.9:18
{הִנְנִ֤י מַמְטִיר֙ כָּעֵ֣ת מָחָ֔ר בָּרָ֖ד כָּבֵ֣ד מְאֹ֑ד אֲשֶׁ֨ר לֹא־הָיָ֤ה כָמֹ֙הוּ֙ בְּמִצְרַ֔יִם לְמִן־הַיּ֥וֹם הִוָּסְדָ֖ה וְעַד־עָֽתָּה׃
\rashi{\rashiDH{כעת מחר. }כעת הזאת למחר, שרט לו שריטה בכותל, למחר כשתגיע חמה לכאן, ירד הברד׃ }\rashi{\rashiDH{הוסדה. }שנתיסדה. וכל תיבה שתחלת יסודה יו״ד, כגון יסד, ילד, ידע, יסר, כשהיא מתפעלת, תבא הוי״ו במקום היו״ד, כמו הוסדה, הִוָּלְדָהּ (הושע ב, ה), וַיִּוָּדַע (אסתר ב, כב), וַיִּוָּלֵד לְיֹוסֵף (בראשית מו, כ), בִּדְבָרִים לֹא יִוָּסֶר עָבֶד (משלי כט, יט)׃ }}
{הָאֲנָא מַחֵית בְּעִדָּנָא הָדֵין מְחַר בַּרְדָּא תַּקִּיף לַחְדָּא דְּלָא הֲוָה דִּכְוָתֵיהּ בְּמִצְרַיִם לְמִן יוֹמָא דְּאִשְׁתַּכְלַלַת וְעַד כְּעַן׃}
{Behold, tomorrow about this time I will cause it to rain a very grievous hail, such as hath not been in Egypt since the day it was founded even until now.}{\arabic{verse}}
\threeverse{\arabic{verse}}%Ex.9:19
{וְעַתָּ֗ה שְׁלַ֤ח הָעֵז֙ אֶֽת־מִקְנְךָ֔ וְאֵ֛ת כׇּל־אֲשֶׁ֥ר לְךָ֖ בַּשָּׂדֶ֑ה כׇּל־הָאָדָ֨ם וְהַבְּהֵמָ֜ה אֲשֶֽׁר־יִמָּצֵ֣א בַשָּׂדֶ֗ה וְלֹ֤א יֵֽאָסֵף֙ הַבַּ֔יְתָה וְיָרַ֧ד עֲלֵהֶ֛ם הַבָּרָ֖ד וָמֵֽתוּ׃
\rashi{\rashiDH{שלח העז. }כתרגומו שלח כְּנוֹשׁ, וכן ישְׁבֵי הַגֵּבִים הֵעִיזוּ (ישעיה י, לא), הָעִיזוּ בְּנֵי בִנְיָמִן (ירמיה ו, א)׃ }\rashi{\rashiDH{ולא יאסף הביתה. }לשון הכנסה היא׃}}
{וּכְעַן שְׁלַח כְּנוֹשׁ יָת בְּעִירָךְ וְיָת כָּל דְּלָךְ בְּחַקְלָא כָּל אֲנָשָׁא וּבְעִירָא דְּיִשְׁתְּכַח בְּחַקְלָא וְלָא יִתְכְּנֵישׁ לְבֵיתָא וְיֵיחוֹת עֲלֵיהוֹן בַּרְדָּא וִימוּתוּן׃}
{Now therefore send, hasten in thy cattle and all that thou hast in the field; for every man and beast that shall be found in the field, and shall not be brought home, the hail shall come down upon them, and they shall die.’}{\arabic{verse}}
\threeverse{\arabic{verse}}%Ex.9:20
{הַיָּרֵא֙ אֶת־דְּבַ֣ר יְהֹוָ֔ה מֵֽעַבְדֵ֖י פַּרְעֹ֑ה הֵנִ֛יס אֶת־עֲבָדָ֥יו וְאֶת־מִקְנֵ֖הוּ אֶל־הַבָּתִּֽים׃
\rashi{\rashiDH{הניס. }הבריח, לשון וינס׃ }}
{דְּדָחֵיל מִפִּתְגָמָא דַּייָ מֵעַבְדֵי פַרְעֹה כְּנַשׁ יָת עַבְדּוֹהִי וְיָת בִּעִירֵיהּ לְבָתַּיָּא׃}
{He that feared the word of the \lord\space among the servants of Pharaoh made his servants and his cattle flee into the houses;}{\arabic{verse}}
\threeverse{\arabic{verse}}%Ex.9:21
{וַאֲשֶׁ֥ר לֹא־שָׂ֛ם לִבּ֖וֹ אֶל־דְּבַ֣ר יְהֹוָ֑ה וַֽיַּעֲזֹ֛ב אֶת־עֲבָדָ֥יו וְאֶת־מִקְנֵ֖הוּ בַּשָּׂדֶֽה׃ \petucha }
{וּדְלָא שַׁוִּי לִבֵּיהּ לְפִתְגָמָא דַּייָ שְׁבַק יָת עַבְדּוֹהִי וְיָת בְּעִירֵיהּ בְּחַקְלָא׃}
{and he that regarded not the word of the \lord\space left his servants and his cattle in the field.}{\arabic{verse}}
\threeverse{\arabic{verse}}%Ex.9:22
{וַיֹּ֨אמֶר יְהֹוָ֜ה אֶל־מֹשֶׁ֗ה נְטֵ֤ה אֶת־יָֽדְךָ֙ עַל־הַשָּׁמַ֔יִם וִיהִ֥י בָרָ֖ד בְּכׇל־אֶ֣רֶץ מִצְרָ֑יִם עַל־הָאָדָ֣ם וְעַל־הַבְּהֵמָ֗ה וְעַ֛ל כׇּל־עֵ֥שֶׂב הַשָּׂדֶ֖ה בְּאֶ֥רֶץ מִצְרָֽיִם׃
\rashi{\rashiDH{על השמים. }לצד השמים. ומדרש אגדה, הגביהו הקב״ה למשה למעלה מן השמים׃ 
}}
{וַאֲמַר יְיָ לְמֹשֶׁה אֲרֵים יָת יְדָךְ עַל צֵית שְׁמַיָּא וִיהֵי בַרְדָּא בְּכָל אַרְעָא דְּמִצְרָיִם עַל אֲנָשָׁא וְעַל בְּעִירָא וְעַל כָּל עִסְבָּא דְּחַקְלָא בְּאַרְעָא דְּמִצְרָיִם׃}
{And the \lord\space said unto Moses: ‘Stretch forth thy hand toward heaven, that there may be hail in all the land of Egypt, upon man, and upon beast, and upon every herb of the field, throughout the land of Egypt.’}{\arabic{verse}}
\threeverse{\arabic{verse}}%Ex.9:23
{וַיֵּ֨ט מֹשֶׁ֣ה אֶת־מַטֵּ֘הוּ֮ עַל־הַשָּׁמַ֒יִם֒ וַֽיהֹוָ֗ה נָתַ֤ן קֹלֹת֙ וּבָרָ֔ד וַתִּ֥הֲלַךְ אֵ֖שׁ אָ֑רְצָה וַיַּמְטֵ֧ר יְהֹוָ֛ה בָּרָ֖ד עַל־אֶ֥רֶץ מִצְרָֽיִם׃}
{וַאֲרֵים מֹשֶׁה יָת חוּטְרֵיהּ עַל צֵית שְׁמַיָּא וַייָ יְהַב קָלִין וּבְרַד וּמְהַלְּכָא אִישָׁתָא עַל אַרְעָא וְאַמְטַר יְיָ בַּרְדָּא עַל אַרְעָא דְּמִצְרָיִם׃}
{And Moses stretched forth his rod toward heaven; and the \lord\space sent thunder and hail, and fire ran down unto the earth; and the \lord\space caused to hail upon the land of Egypt.}{\arabic{verse}}
\threeverse{\arabic{verse}}%Ex.9:24
{וַיְהִ֣י בָרָ֔ד וְאֵ֕שׁ מִתְלַקַּ֖חַת בְּת֣וֹךְ הַבָּרָ֑ד כָּבֵ֣ד מְאֹ֔ד אֲ֠שֶׁ֠ר לֹֽא־הָיָ֤ה כָמֹ֙הוּ֙ בְּכׇל־אֶ֣רֶץ מִצְרַ֔יִם מֵאָ֖ז הָיְתָ֥ה לְגֽוֹי׃
\rashi{\rashiDH{מתלקחת בתוך הברד. }נס בתוך נס, האש והברד מעורבין, והברד מים הוא, ולעשות רצון קונם עשו שלום ביניהם (שמו״ר יב, ו)׃ 
}}
{וַהֲוָה בַרְדָּא וְאִישָׁתָא מִשְׁתַּלְהֲבָא בְּגוֹ בַרְדָּא תַּקִּיף לַחְדָּא דְּלָא הֲוָה דִּכְוָתֵיהּ בְּכָל אַרְעָא דְּמִצְרַיִם מֵעִדָּן דַּהֲוָת לְעַם׃}
{So there was hail, and fire flashing up amidst the hail, very grievous, such as had not been in all the land of Egypt since it became a nation.}{\arabic{verse}}
\threeverse{\arabic{verse}}%Ex.9:25
{וַיַּ֨ךְ הַבָּרָ֜ד בְּכׇל־אֶ֣רֶץ מִצְרַ֗יִם אֵ֚ת כׇּל־אֲשֶׁ֣ר בַּשָּׂדֶ֔ה מֵאָדָ֖ם וְעַד־בְּהֵמָ֑ה וְאֵ֨ת כׇּל־עֵ֤שֶׂב הַשָּׂדֶה֙ הִכָּ֣ה הַבָּרָ֔ד וְאֶת־כׇּל־עֵ֥ץ הַשָּׂדֶ֖ה שִׁבֵּֽר׃}
{וּמְחָא בַרְדָּא בְּכָל אַרְעָא דְּמִצְרַיִם יָת כָּל דִּבְחַקְלָא מֵאֲנָשָׁא וְעַד בְּעִירָא וְיָת כָּל עִסְבָּא דְּחַקְלָא מְחָא בַרְדָּא וְיָת כָּל אִילָנֵי חַקְלָא תַּבַּר׃}
{And the hail smote throughout all the land of Egypt all that was in the field, both man and beast; and the hail smote every herb of the field, and broke every tree of the field.}{\arabic{verse}}
\threeverse{\arabic{verse}}%Ex.9:26
{רַ֚ק בְּאֶ֣רֶץ גֹּ֔שֶׁן אֲשֶׁר־שָׁ֖ם בְּנֵ֣י יִשְׂרָאֵ֑ל לֹ֥א הָיָ֖ה בָּרָֽד׃}
{לְחוֹד בְּאַרְעָא דְּגֹשֶׁן דְּתַמָּן בְּנֵי יִשְׂרָאֵל לָא הֲוָה בַרְדָּא׃}
{Only in the land of Goshen, where the children of Israel were, was there no hail.}{\arabic{verse}}
\threeverse{\arabic{verse}}%Ex.9:27
{וַיִּשְׁלַ֣ח פַּרְעֹ֗ה וַיִּקְרָא֙ לְמֹשֶׁ֣ה וּֽלְאַהֲרֹ֔ן וַיֹּ֥אמֶר אֲלֵהֶ֖ם חָטָ֣אתִי הַפָּ֑עַם יְהֹוָה֙ הַצַּדִּ֔יק וַאֲנִ֥י וְעַמִּ֖י הָרְשָׁעִֽים׃}
{וּשְׁלַח פַּרְעֹה וּקְרָא לְמֹשֶׁה וּלְאַהֲרֹן וַאֲמַר לְהוֹן חַבִית זִמְנָא הָדָא יְיָ זַכָּאָה וַאֲנָא וְעַמִּי חַיָּיבִין׃}
{And Pharaoh sent, and called for Moses and Aaron, and said unto them: ‘I have sinned this time; the \lord\space is righteous, and I and my people are wicked.}{\arabic{verse}}
\threeverse{\arabic{verse}}%Ex.9:28
{הַעְתִּ֙ירוּ֙ אֶל־יְהֹוָ֔ה וְרַ֕ב מִֽהְיֹ֛ת קֹלֹ֥ת אֱלֹהִ֖ים וּבָרָ֑ד וַאֲשַׁלְּחָ֣ה אֶתְכֶ֔ם וְלֹ֥א תֹסִפ֖וּן לַעֲמֹֽד׃
\rashi{\rashiDH{ורב. }די לו במה שהוריד כבר׃}}
{צַלּוֹ קֳדָם יְיָ וְסַגִּי קֳדָמוֹהִי רְוַח דְּלָא יְהוֹן עֲלַנָא קָלִין דִּלְוָט כְּאִלֵּין מִן קֳדָם יְיָ וּבְרַד וַאֲשַׁלַּח יָתְכוֹן וְלָא תֵיסְפוּן לְאִתְעַכָּבָא׃}
{Entreat the \lord, and let there be enough of these mighty thunderings and hail; and I will let you go, and ye shall stay no longer.’}{\arabic{verse}}
\threeverse{\arabic{verse}}%Ex.9:29
{וַיֹּ֤אמֶר אֵלָיו֙ מֹשֶׁ֔ה כְּצֵאתִי֙ אֶת־הָעִ֔יר אֶפְרֹ֥שׂ אֶת־כַּפַּ֖י אֶל־יְהֹוָ֑ה הַקֹּל֣וֹת יֶחְדָּל֗וּן וְהַבָּרָד֙ לֹ֣א יִֽהְיֶה־ע֔וֹד לְמַ֣עַן תֵּדַ֔ע כִּ֥י לַיהֹוָ֖ה הָאָֽרֶץ׃
\rashi{\rashiDH{כצאתי את העיר. }מן העיר, אבל בתוך העיר לא התפלל, לפי שהיתה מלאה גלולים (שם יב, ז)׃ }}
{וַאֲמַר לֵיהּ מֹשֶׁה כְּמִפְּקִי יָת קַרְתָּא אֶפְרוֹס יָת יְדַי בִּצְלוֹ קֳדָם יְיָ קָלַיָּא יִתְמַנְעוּן וּבַרְדָּא לָא יְהֵי עוֹד בְּדִיל דְּתִדַּע אֲרֵי דַּייָ אַרְעָא׃}
{And Moses said unto him: ‘As soon as I am gone out of the city, I will spread forth my hands unto the \lord; the thunders shall cease, neither shall there be any more hail; that thou mayest know that the earth is the \lord’s.}{\arabic{verse}}
\threeverse{\arabic{verse}}%Ex.9:30
{וְאַתָּ֖ה וַעֲבָדֶ֑יךָ יָדַ֕עְתִּי כִּ֚י טֶ֣רֶם תִּֽירְא֔וּן מִפְּנֵ֖י יְהֹוָ֥ה אֱלֹהִֽים׃
\rashi{\rashiDH{טרם תיראון. }עדיין לא תיראון. וכן כל טרם שבמקרא עדיין לא הוא, ואינו לשון קודם, כמו טֶרֶם יִשְׁכָּבוּ (בראשית יט, ד), עד לא שכיבו. טֶרֶם יִצְמָח (שם ב, ה), עד לא צמח. אף זה כן הוא, ידעתי כי עדיין אינכם יראים, ומשתהיה הרוחה תעמדו בקלקולכם׃ }}
{וְאַתְּ וְעַבְדָךְ יָדַעְנָא אֲרֵי עַד כְּעַן לָא אִתְכְּנַעְתּוּן מִן קֳדָם יְיָ אֱלֹהִים׃}
{But as for thee and thy servants, I know that ye will not yet fear the \lord\space God.’—}{\arabic{verse}}
\threeverse{\arabic{verse}}%Ex.9:31
{וְהַפִּשְׁתָּ֥ה וְהַשְּׂעֹרָ֖ה נֻכָּ֑תָה כִּ֤י הַשְּׂעֹרָה֙ אָבִ֔יב וְהַפִּשְׁתָּ֖ה גִּבְעֹֽל׃
\rashi{\rashiDH{והפשתה והשעורה נכתה. }נשברה, לשון פרעה נכה, נכאים, וכן לא נכו, ולא יתכן לפרשו לשון הכאה, שאין נו״ן במקום ה״א לפרש נכתה כמו הוכתה, נכו כמו הכו, אלא הנו״ן שורש בתיבה, והרי הוא מגזרת וְשֻׁפּוּ עַצְמֹתָיו (איוב לג, כא)׃ 
}\rashi{\rashiDH{כי השערה אביב. }כבר ביכרה ועומדת בְּקָשְׁיָהּ, ונשתברו ונפלו, וכן הפשתה גדלה כבר והוקשה לעמוד בגבעוליה׃ }\rashi{\rashiDH{השעורה אביב. }עמדה באביה, לשון בְּאִבֵּי הַנָּחַל (שיר השירים ו, יא)׃ }}
{וְכִתָּנָא וּסְעָרֵי לְקוֹ אֲרֵי סְעָרַיָּא אֲבִיב וְכִתָּנָא גַּבְעוּלִּין׃}
{And the flax and the barley were smitten; for the barley was in the ear, and the flax was in bloom.}{\arabic{verse}}
\threeverse{\arabic{verse}}%Ex.9:32
{וְהַחִטָּ֥ה וְהַכֻּסֶּ֖מֶת לֹ֣א נֻכּ֑וּ כִּ֥י אֲפִילֹ֖ת הֵֽנָּה׃
\rashi{\rashiDH{כי אפילת הנה. }מאוחרות, ועדיין היו רכות, ויכולות לעמוד בפני קשה, ואע״פ שנאמר ואת כל עשב השדה הכה הברד, יש לפרש פשוטו של מקרא בעשבים העומדים בקלחם הראויים ללקות בברד. ומדרש רבי תנחומא (וארא טז) יש מרבותינו שנחלקו על זאת, ודרשו כי אפילות, פלאי פלאות נעשו להם שלא לקו׃ 
}}
{וְחִטַּיָּא וְכוּנָתַיָּא לָא לְקַאָה אֲרֵי אַפְלָתָא אִנִּין׃}
{But the wheat and the spelt were not smitten; for they ripen late.—}{\arabic{verse}}
\threeverse{\aliya{מפטיר}}%Ex.9:33
{וַיֵּצֵ֨א מֹשֶׁ֜ה מֵעִ֤ם פַּרְעֹה֙ אֶת־הָעִ֔יר וַיִּפְרֹ֥שׂ כַּפָּ֖יו אֶל־יְהֹוָ֑ה וַֽיַּחְדְּל֤וּ הַקֹּלוֹת֙ וְהַבָּרָ֔ד וּמָטָ֖ר לֹא־נִתַּ֥ךְ אָֽרְצָה׃
\rashi{\rashiDH{לא נתך. }לא הגיע, ואף אותן שהיו באויר לא הגיעו לארץ, ודומה לו וַתִּתַּךְ עָלֵינוּ הָאָלָה וְהַשְׁבֻעָה (דניאל ט, יא) דעזרא, ותגיע עלינו. ומנחם בן סרוק חברו בחלק כְּהִתּוּך כֶּסֶף (יחזקאל כב, כב), לשון יציקת מתכת, ורואה אני את דבריו כתרגומו וַיִצֹק, וְאַתֵּיךְ. לָצֶקֶת, לְאַתָּכָא. אף זה לא נתך לארץ, לא הוצק לארץ׃ }}
{וּנְפַק מֹשֶׁה מִלְּוָת פַּרְעֹה יָת קַרְתָּא וּפְרַס יְדוֹהִי בִּצְלוֹ קֳדָם יְיָ וְאִתְמְנַעוּ קָלַיָּא וּבַרְדָּא וּמִטְרָא דַּהֲוָה נָחֵית לָא מְטָא אַרְעָא׃}
{And Moses went out of the city from Pharaoh, and spread forth his hands unto the \lord; and the thunders and hail ceased, and the rain was not poured upon the earth.}{\arabic{verse}}
\threeverse{\arabic{verse}}%Ex.9:34
{וַיַּ֣רְא פַּרְעֹ֗ה כִּֽי־חָדַ֨ל הַמָּטָ֧ר וְהַבָּרָ֛ד וְהַקֹּלֹ֖ת וַיֹּ֣סֶף לַחֲטֹ֑א וַיַּכְבֵּ֥ד לִבּ֖וֹ ה֥וּא וַעֲבָדָֽיו׃}
{וַחֲזָא פַרְעֹה אֲרֵי אִתְמְנַע מִטְרָא וּבַרְדָּא וְקָלַיָּא וְאוֹסֵיף לְמִחְטֵי וְיַקְּרֵיהּ לְלִבֵּיהּ הוּא וְעַבְדּוֹהִי׃}
{And when Pharaoh saw that the rain and the hail and the thunders were ceased, he sinned yet more, and hardened his heart, he and his servants.}{\arabic{verse}}
\threeverse{\arabic{verse}}%Ex.9:35
{וַֽיֶּחֱזַק֙ לֵ֣ב פַּרְעֹ֔ה וְלֹ֥א שִׁלַּ֖ח אֶת־בְּנֵ֣י יִשְׂרָאֵ֑ל כַּאֲשֶׁ֛ר דִּבֶּ֥ר יְהֹוָ֖ה בְּיַד־מֹשֶֽׁה׃ \petucha }
{וְאִתַּקַּף לִבָּא דְּפַרְעֹה וְלָא שַׁלַּח יָת בְּנֵי יִשְׂרָאֵל כְּמָא דְּמַלֵּיל יְיָ בִּידָא דְּמֹשֶׁה׃}
{And the heart of Pharaoh was hardened, and he did not let the children of Israel go; as the \lord\space had spoken by Moses.}{\arabic{verse}}
\newperek
\newparsha{בא}
\threeverse{\aliya{בא}}%Ex.10:1
{וַיֹּ֤אמֶר יְהֹוָה֙ אֶל־מֹשֶׁ֔ה בֹּ֖א אֶל־פַּרְעֹ֑ה כִּֽי־אֲנִ֞י הִכְבַּ֤דְתִּי אֶת־לִבּוֹ֙ וְאֶת־לֵ֣ב עֲבָדָ֔יו לְמַ֗עַן שִׁתִ֛י אֹתֹתַ֥י אֵ֖לֶּה בְּקִרְבּֽוֹ׃
\rashi{\rashiDH{ויאמר ה׳ אל משה בא אל פרעה. }והתרה בו׃}\rashi{\rashiDH{שתי. }שִׂימִי, שאשית אני׃ 
}}
{וַאֲמַר יְיָ לְמֹשֶׁה עוֹל לְוָת פַּרְעֹה אֲרֵי אֲנָא יַקַּרִית יָת לִבֵּיהּ וְיָת לִבָּא דְּעַבְדּוֹהִי בְּדִיל לְשַׁוָּאָה אָתַי אִלֵּין בֵּינֵיהוֹן׃}
{And the \lord\space said unto Moses: ‘Go in unto Pharaoh; for I have hardened his heart, and the heart of his servants, that I might show these My signs in the midst of them;}{\Roman{chap}}
\threeverse{\arabic{verse}}%Ex.10:2
{וּלְמַ֡עַן תְּסַפֵּר֩ בְּאׇזְנֵ֨י בִנְךָ֜ וּבֶן־בִּנְךָ֗ אֵ֣ת אֲשֶׁ֤ר הִתְעַלַּ֙לְתִּי֙ בְּמִצְרַ֔יִם וְאֶת־אֹתֹתַ֖י אֲשֶׁר־שַׂ֣מְתִּי בָ֑ם וִֽידַעְתֶּ֖ם כִּי־אֲנִ֥י יְהֹוָֽה׃
\rashi{\rashiDH{התעללתי. }שחקתי, כמו כִּי הִתְעַלַּלְתּ בִּי (במדבר כב, כט), הֲלֹוא כַּאֲשֶׁר הִתְעַלֵּל בָּהֶם (שמואל־א ו, י) האמור במצרים, ואינו לשון פועל ומעללים, שא״כ היה לו לכתוב עוללתי, כמו וְעֹולֵל לָמֹו כַּאֲשֶׁר עֹולַלְתָּ לִי (איכה א, כב), אֲשֶׁר עֹולַל לִי (שם יב)׃ 
}}
{וּבְדִיל דְּתִשְׁתַּעֵי קֳדָם בְּרָךְ וּבַר בְּרָךְ יָת נִסִּין דַּעֲבַדִית בְּמִצְרַיִם וְיָת אָתְוָתַי דְּשַׁוִּיתִי בְּהוֹן וְתִדְּעוּן אֲרֵי אֲנָא יְיָ׃}
{and that thou mayest tell in the ears of thy son, and of thy son’s son, what I have wrought upon Egypt, and My signs which I have done among them; that ye may know that I am the \lord.’}{\arabic{verse}}
\threeverse{\arabic{verse}}%Ex.10:3
{וַיָּבֹ֨א מֹשֶׁ֣ה וְאַהֲרֹן֮ אֶל־פַּרְעֹה֒ וַיֹּאמְר֣וּ אֵלָ֗יו כֹּֽה־אָמַ֤ר יְהֹוָה֙ אֱלֹהֵ֣י הָֽעִבְרִ֔ים עַד־מָתַ֣י מֵאַ֔נְתָּ לֵעָנֹ֖ת מִפָּנָ֑י שַׁלַּ֥ח עַמִּ֖י וְיַֽעַבְדֻֽנִי׃
\rashi{\rashiDH{לענות. }כתרגומו לְאִתְכְּנָעָא, והוא מגזרת עני, מֵאַנְתָּ להיות עני ושפל מפני׃ }}
{וְעָאל מֹשֶׁה וְאַהֲרֹן לְוָת פַּרְעֹה וַאֲמַרוּ לֵיהּ כִּדְנָן אֲמַר יְיָ אֱלָהָא דִּיהוּדָאֵי עַד אִמַּתִּי מְסָרֵיב אַתְּ לְאִתְכְּנָעָא מִן קֳדָמָי שַׁלַּח עַמִּי וְיִפְלְחוּן קֳדָמָי׃}
{And Moses and Aaron went in unto Pharaoh, and said unto him: ‘Thus saith the \lord, the God of the Hebrews: How long wilt thou refuse to humble thyself before Me? let My people go, that they may serve Me.}{\arabic{verse}}
\threeverse{\aliya{לוי}}%Ex.10:4
{כִּ֛י אִם־מָאֵ֥ן אַתָּ֖ה לְשַׁלֵּ֣חַ אֶת־עַמִּ֑י הִנְנִ֨י מֵבִ֥יא מָחָ֛ר אַרְבֶּ֖ה בִּגְבֻלֶֽךָ׃}
{אֲרֵי אִם מְסָרֵיב אַתְּ לְשַׁלָּחָא יָת עַמִּי הָאֲנָא מֵיתֵי מְחַר גּוֹבָא בִּתְחוּמָךְ׃}
{Else, if thou refuse to let My people go, behold, to-morrow will I bring locusts into thy border;}{\arabic{verse}}
\threeverse{\arabic{verse}}%Ex.10:5
{וְכִסָּה֙ אֶת־עֵ֣ין הָאָ֔רֶץ וְלֹ֥א יוּכַ֖ל לִרְאֹ֣ת אֶת־הָאָ֑רֶץ וְאָכַ֣ל \legarmeh  אֶת־יֶ֣תֶר הַפְּלֵטָ֗ה הַנִּשְׁאֶ֤רֶת לָכֶם֙ מִן־הַבָּרָ֔ד וְאָכַל֙ אֶת־כׇּל־הָעֵ֔ץ הַצֹּמֵ֥חַ לָכֶ֖ם מִן־הַשָּׂדֶֽה׃
\rashi{\rashiDH{את עין הארץ. }את מראה הארץ׃ 
}\rashi{\rashiDH{ולא יוכל וגו׳. }הרואה, לראות את הארץ, ולשון קצרה דבר׃ }}
{וְיִחְפֵי יָת עֵין שִׁמְשָׁא דְּאַרְעָא וְלָא יִכּוֹל לְמִחְזֵי יָת אַרְעָא וְיֵיכוֹל יָת שְׁאָר שֵׁיזָבְתָא דְּאִשְׁתְּאַרַת לְכוֹן מִן בַּרְדָּא וְיֵיכוֹל יָת כָּל אִילָנָא דְּאַצְמַח לְכוֹן מִן חַקְלָא׃}
{and they shall cover the face of the earth, that one shall not be able to see the earth; and they shall eat the residue of that which is escaped, which remaineth unto you from the hail, and shall eat every tree which groweth for you out of the field;}{\arabic{verse}}
\threeverse{\arabic{verse}}%Ex.10:6
{וּמָלְא֨וּ בָתֶּ֜יךָ וּבָתֵּ֣י כׇל־עֲבָדֶ֘יךָ֮ וּבָתֵּ֣י כׇל־מִצְרַ֒יִם֒ אֲשֶׁ֨ר לֹֽא־רָא֤וּ אֲבֹתֶ֙יךָ֙ וַאֲב֣וֹת אֲבֹתֶ֔יךָ מִיּ֗וֹם הֱיוֹתָם֙ עַל־הָ֣אֲדָמָ֔ה עַ֖ד הַיּ֣וֹם הַזֶּ֑ה וַיִּ֥פֶן וַיֵּצֵ֖א מֵעִ֥ם פַּרְעֹֽה׃}
{וְיִתְמְלוֹן בָּתָּךְ וּבָתֵּי כָל עַבְדָךְ וּבָתֵּי כָל מִצְרָאֵי דְּלָא חֲזוֹ אֲבָהָתָךְ וַאֲבָהָת אֲבָהָתָךְ מִיּוֹם מִהְוֵיהוֹן עַל אַרְעָא עַד יוֹמָא הָדֵין וְאִתְפְּנִי וּנְפַק מִלְּוָת פַּרְעֹה׃}
{and thy houses shall be filled, and the houses of all thy servants, and the houses of all the Egyptians; as neither thy fathers nor thy fathers’ fathers have seen, since the day that they were upon the earth unto this day.’ And he turned, and went out from Pharaoh.}{\arabic{verse}}
\threeverse{\aliya{ישראל}}%Ex.10:7
{וַיֹּאמְרוּ֩ עַבְדֵ֨י פַרְעֹ֜ה אֵלָ֗יו עַד־מָתַי֙ יִהְיֶ֨ה זֶ֥ה לָ֙נוּ֙ לְמוֹקֵ֔שׁ שַׁלַּח֙ אֶת־הָ֣אֲנָשִׁ֔ים וְיַֽעַבְד֖וּ אֶת־יְהֹוָ֣ה אֱלֹהֵיהֶ֑ם הֲטֶ֣רֶם תֵּדַ֔ע כִּ֥י אָבְדָ֖ה מִצְרָֽיִם׃
\rashi{\rashiDH{הטרם תדע. }העוד לא ידעת כי אבדה מצרים׃ 
}}
{וַאֲמַרוּ עַבְדֵי פַּרְעֹה לֵיהּ עַד אִמַּתִּי יְהֵי דֵין לַנָא לְתַקְלָא שַׁלַּח יָת גּוּבְרַיָּא וְיִפְלְחוּן קֳדָם יְיָ אֱלָהֲהוֹן הַעַד כְּעַן לָא יְדַעְתָּא אֲרֵי אֲבַדַת מִצְרָיִם׃}
{And Pharaoh’s servants said unto him: ‘How long shall this man be a snare unto us? let the men go, that they may serve the \lord\space their God, knowest thou not yet that Egypt is destroyed?’}{\arabic{verse}}
\threeverse{\arabic{verse}}%Ex.10:8
{וַיּוּשַׁ֞ב אֶת־מֹשֶׁ֤ה וְאֶֽת־אַהֲרֹן֙ אֶל־פַּרְעֹ֔ה וַיֹּ֣אמֶר אֲלֵהֶ֔ם לְכ֥וּ עִבְד֖וּ אֶת־יְהֹוָ֣ה אֱלֹהֵיכֶ֑ם מִ֥י וָמִ֖י הַהֹלְכִֽים׃
\rashi{\rashiDH{ויושב. }הושבו ע״י שליח ששלחו אחריהם, והושיבום אל פרעה׃ }}
{וְאִתָּתַב יָת מֹשֶׁה וְיָת אַהֲרֹן לְוָת פַּרְעֹה וַאֲמַר לְהוֹן אִיזִילוּ פְלַחוּ קֳדָם יְיָ אֱלָהֲכוֹן מַן וּמַן אָזְלִין׃}
{And Moses and Aaron were brought again unto Pharaoh; and he said unto them: ‘Go, serve the \lord\space your God; but who are they that shall go?’}{\arabic{verse}}
\threeverse{\arabic{verse}}%Ex.10:9
{וַיֹּ֣אמֶר מֹשֶׁ֔ה בִּנְעָרֵ֥ינוּ וּבִזְקֵנֵ֖ינוּ נֵלֵ֑ךְ בְּבָנֵ֨ינוּ וּבִבְנוֹתֵ֜נוּ בְּצֹאנֵ֤נוּ וּבִבְקָרֵ֙נוּ֙ נֵלֵ֔ךְ כִּ֥י חַג־יְהֹוָ֖ה לָֽנוּ׃}
{וַאֲמַר מֹשֶׁה בְּעוּלֵימַנָא וּבְסָבַנָא נֵיזֵיל בִּבְנַנָא וּבִבְנָתַנָא בְּעָנַנָא וּבְתוֹרַנָא נֵיזֵיל אֲרֵי חַגָּא קֳדָם יְיָ לַנָא׃}
{And Moses said: ‘We will go with our young and with our old, with our sons and with our daughters, with our flocks and with our herds we will go; for we must hold a feast unto the \lord.’}{\arabic{verse}}
\threeverse{\arabic{verse}}%Ex.10:10
{וַיֹּ֣אמֶר אֲלֵהֶ֗ם יְהִ֨י כֵ֤ן יְהֹוָה֙ עִמָּכֶ֔ם כַּאֲשֶׁ֛ר אֲשַׁלַּ֥ח אֶתְכֶ֖ם וְאֶֽת־טַפְּכֶ֑ם רְא֕וּ כִּ֥י רָעָ֖ה נֶ֥גֶד פְּנֵיכֶֽם׃
\rashi{\rashiDH{כאשר אשלח אתכם וגו׳. }אף כי אשלח גם את הצאן ואת הבקר כאשר אמרתם׃ 
}\rashi{\rashiDH{ראו כי רעה נגד פניכם. }כתרגומו. ומדרש אגדה שמעתי, כוכב אחד יש ששמו רעה, אמר להם פרעה, רואה אני בָּאִיצְטַגְנִינוּת שלי אותו כוכב עולה לקראתכם במדבר, והוא סימן דם והריגה, וכשחטאו ישראל בעגל ובקש הקב״ה להרגם, אמר משה בתפלתו, לָמָה יֹאמְרוּ מִצְרַיִם לֵאמֹר בְּרָעָה הֹוצִיאָם (שמות לב, יב), זו היא שאמר להם ראו כי רעה נגד פניכם, מיד וַיִנָחֶם ה׳ עַל הָרָעָה, והפך את הדם לדם מילה שמל יהושע אותם, וזהו שנאמר הַיֹּום גַּלֹּותִי אֶת חֶרְפַּת מִצְרַיִם מֵעֲלֵיכֶם (יהושע ה, ט), שהיו אומרים לכם דם אנו רואין עליכם במדבר׃ }}
{וַאֲמַר לְהוֹן יְהֵי כֵן מֵימְרָא דַּייָ בְּסַעְדְּכוֹן כַּד אֲשַׁלַּח יָתְכוֹן וְיָת טַפְלְכוֹן חֲזוֹ אֲרֵי בִישָׁא אַתּוּן סְבִירִין לְמֶעֱבַד לֵית קֳבֵיל אַפֵּיכוֹן לְאִסְתְּחָרָא׃}
{And he said unto them: ‘So be the \lord\space with you, as I will let you go, and your little ones; see ye that evil is before your face.}{\arabic{verse}}
\threeverse{\arabic{verse}}%Ex.10:11
{לֹ֣א כֵ֗ן לְכֽוּ־נָ֤א הַגְּבָרִים֙ וְעִבְד֣וּ אֶת־יְהֹוָ֔ה כִּ֥י אֹתָ֖הּ אַתֶּ֣ם מְבַקְשִׁ֑ים וַיְגָ֣רֶשׁ אֹתָ֔ם מֵאֵ֖ת פְּנֵ֥י פַרְעֹֽה׃ \setuma         
\rashi{\rashiDH{לא כן. }כאשר אמרתם להוליך הטף עמכם, אלא לכו הגברים ועבדו את ה׳׃ }\rashi{\rashiDH{כי אותה אתם מבקשים. }(אותה עבודה) בקשתם עד הנה, נזבחה לאלהינו, ואין דרך הטף לזבוח׃ }\rashi{\rashiDH{ויגרש אותם. }הרי זה לשון קצר, ולא פירש מי המגרש׃ }}
{לָא כֵן אִיזִילוּ כְעַן גּוּבְרַיָּא וּפְלַחוּ קֳדָם יְיָ אֲרֵי יָתַהּ אַתּוּן בָּעַן וְתָרֵיךְ יָתְהוֹן מִן קֳדָם פַּרְעֹה׃}
{Not so; go now ye that are men, and serve the \lord; for that is what ye desire.’ And they were driven out from Pharaoh’s presence.}{\arabic{verse}}
\threeverse{\aliya{שני}}%Ex.10:12
{וַיֹּ֨אמֶר יְהֹוָ֜ה אֶל־מֹשֶׁ֗ה נְטֵ֨ה יָדְךָ֜ עַל־אֶ֤רֶץ מִצְרַ֙יִם֙ בָּֽאַרְבֶּ֔ה וְיַ֖עַל עַל־אֶ֣רֶץ מִצְרָ֑יִם וְיֹאכַל֙ אֶת־כׇּל־עֵ֣שֶׂב הָאָ֔רֶץ אֵ֛ת כׇּל־אֲשֶׁ֥ר הִשְׁאִ֖יר הַבָּרָֽד׃
\rashi{\rashiDH{בארבה. }בשביל מכת הארבה׃ 
}}
{וַאֲמַר יְיָ לְמֹשֶׁה אֲרֵים יְדָךְ עַל אַרְעָא דְּמִצְרַיִם וְיֵיתֵי גּוֹבָא וְיִסַּק עַל אַרְעָא דְּמִצְרָיִם וְיֵיכוֹל יָת כָּל עִסְבָּא דְּאַרְעָא יָת כָּל דְּאַשְׁאַר בַּרְדָּא׃}
{And the \lord\space said unto Moses: ‘Stretch out thy hand over the land of Egypt for the locusts, that they may come up upon the land of Egypt, and eat every herb of the land, even all that the hail hath left.’}{\arabic{verse}}
\threeverse{\arabic{verse}}%Ex.10:13
{וַיֵּ֨ט מֹשֶׁ֣ה אֶת־מַטֵּ֘הוּ֮ עַל־אֶ֣רֶץ מִצְרַ֒יִם֒ וַֽיהֹוָ֗ה נִהַ֤ג רֽוּחַ־קָדִים֙ בָּאָ֔רֶץ כׇּל־הַיּ֥וֹם הַה֖וּא וְכׇל־הַלָּ֑יְלָה הַבֹּ֣קֶר הָיָ֔ה וְר֙וּחַ֙ הַקָּדִ֔ים נָשָׂ֖א אֶת־הָאַרְבֶּֽה׃
\rashi{\rashiDH{ורוח הקדים. }רוח מזרחית נשא את הארבה, לפי שבא כנגדו, שמצרים בדרומית מערבית היתה, כמו שמפורש במקום אחר׃ }}
{וַאֲרֵים מֹשֶׁה יָת חוּטְרֵיהּ עַל אַרְעָא דְּמִצְרַיִם וַייָ דַּבַּר רוּחַ קִדּוּמָא בְּאַרְעָא כָּל יוֹמָא הַהוּא וְכָל לֵילְיָא צַפְרָא הֲוָה וְרוּחַ קִדּוּמָא נְטַל יָת גּוֹבָא׃}
{And Moses stretched forth his rod over the land of Egypt, and the \lord\space brought an east wind upon the land all that day, and all the night; and when it was morning, the east wind brought the locusts.}{\arabic{verse}}
\threeverse{\arabic{verse}}%Ex.10:14
{וַיַּ֣עַל הָֽאַרְבֶּ֗ה עַ֚ל כׇּל־אֶ֣רֶץ מִצְרַ֔יִם וַיָּ֕נַח בְּכֹ֖ל גְּב֣וּל מִצְרָ֑יִם כָּבֵ֣ד מְאֹ֔ד לְ֠פָנָ֠יו לֹא־הָ֨יָה כֵ֤ן אַרְבֶּה֙ כָּמֹ֔הוּ וְאַחֲרָ֖יו לֹ֥א יִֽהְיֶה־כֵּֽן׃
\rashi{\rashiDH{ואחריו לא יהיה כן. }ואותו שהיה בימי יואל, שנאמר כּמֹהוּ לֹא נִהְיָה מִן הָעֹולָם (יואל ב, ב), למדנו שהיה כבד משל משה, (כי של יואל היה) ע״י מינין הרבה, שהיו יחד ארבה, ילק, חסיל, גזם, אבל של משה לא היה אלא של מין אחד, (כ״ג רא״ם יע״ש) וכמוהו לא היה ולא יהיה׃ }}
{וּסְלֵיק גּוֹבָא עַל כָּל אַרְעָא דְּמִצְרַיִם וּשְׁרָא בְּכֹל תְּחוּם מִצְרָיִם תַּקִּיף לַחְדָּא קֳדָמוֹהִי לָא הֲוָה כֵן גּוֹבָא דִּכְוָתֵיהּ וּבָתְרוֹהִי לָא יְהֵי כֵן׃}
{And the locusts went up over all the land of Egypt, and rested in all the borders of Egypt; very grievous were they; before them there were no such locusts as they, neither after them shall be such.}{\arabic{verse}}
\threeverse{\arabic{verse}}%Ex.10:15
{וַיְכַ֞ס אֶת־עֵ֣ין כׇּל־הָאָ֘רֶץ֮ וַתֶּחְשַׁ֣ךְ הָאָ֒רֶץ֒ וַיֹּ֜אכַל אֶת־כׇּל־עֵ֣שֶׂב הָאָ֗רֶץ וְאֵת֙ כׇּל־פְּרִ֣י הָעֵ֔ץ אֲשֶׁ֥ר הוֹתִ֖יר הַבָּרָ֑ד וְלֹא־נוֹתַ֨ר כׇּל־יֶ֧רֶק בָּעֵ֛ץ וּבְעֵ֥שֶׂב הַשָּׂדֶ֖ה בְּכׇל־אֶ֥רֶץ מִצְרָֽיִם׃
\rashi{\rashiDH{כל ירק. }עָלֶה ירוק, וירדור״א בלע״ז׃ 
}}
{וַחֲפָא יָת עֵין שִׁמְשָׁא דְּכָל אַרְעָא וַחֲשׁוֹכַת אַרְעָא וַאֲכַל יָת כָּל עִסְבָּא דְּאַרְעָא וְיָת כָּל פֵּירֵי אִילָנָא דְּאַשְׁאַר בַּרְדָּא וְלָא אִשְׁתְּאַר כָּל יָרוֹק בְּאִילָנָא וּבְעִסְבָּא דְּחַקְלָא בְּכָל אַרְעָא דְּמִצְרָיִם׃}
{For they covered the face of the whole earth, so that the land was darkened; and they did eat every herb of the land, and all the fruit of the trees which the hail had left; and there remained not any green thing, either tree or herb of the field, through all the land of Egypt.}{\arabic{verse}}
\threeverse{\arabic{verse}}%Ex.10:16
{וַיְמַהֵ֣ר פַּרְעֹ֔ה לִקְרֹ֖א לְמֹשֶׁ֣ה וּֽלְאַהֲרֹ֑ן וַיֹּ֗אמֶר חָטָ֛אתִי לַיהֹוָ֥ה אֱלֹֽהֵיכֶ֖ם וְלָכֶֽם׃}
{וְאוֹחִי פַרְעֹה לְמִקְרֵי לְמֹשֶׁה וּלְאַהֲרֹן וַאֲמַר חַבִית קֳדָם יְיָ אֱלָהֲכוֹן וּלְכוֹן׃}
{Then Pharaoh called for Moses and Aaron in haste; and he said: ‘I have sinned against the \lord\space your God, and against you.}{\arabic{verse}}
\threeverse{\arabic{verse}}%Ex.10:17
{וְעַתָּ֗ה שָׂ֣א נָ֤א חַטָּאתִי֙ אַ֣ךְ הַפַּ֔עַם וְהַעְתִּ֖ירוּ לַיהֹוָ֣ה אֱלֹהֵיכֶ֑ם וְיָסֵר֙ מֵֽעָלַ֔י רַ֖ק אֶת־הַמָּ֥וֶת הַזֶּֽה׃}
{וּכְעַן שְׁבוֹק כְּעַן לְחוֹבִי בְּרַם זִמְנָא הָדָא וְצַלּוֹ קֳדָם יְיָ אֱלָהֲכוֹן וְיַעְדֵּי מִנִּי לְחוֹד יָת מוֹתָא הָדֵין׃}
{Now therefore forgive, I pray thee, my sin only this once, and entreat the \lord\space your God, that He may take away from me this death only.’}{\arabic{verse}}
\threeverse{\arabic{verse}}%Ex.10:18
{וַיֵּצֵ֖א מֵעִ֣ם פַּרְעֹ֑ה וַיֶּעְתַּ֖ר אֶל־יְהֹוָֽה׃}
{וּנְפַק מִלְּוָת פַּרְעֹה וְצַלִּי קֳדָם יְיָ׃}
{And he went out from Pharaoh, and entreated the \lord.}{\arabic{verse}}
\threeverse{\arabic{verse}}%Ex.10:19
{וַיַּהֲפֹ֨ךְ יְהֹוָ֤ה רֽוּחַ־יָם֙ חָזָ֣ק מְאֹ֔ד וַיִּשָּׂא֙ אֶת־הָ֣אַרְבֶּ֔ה וַיִּתְקָעֵ֖הוּ יָ֣מָּה סּ֑וּף לֹ֤א נִשְׁאַר֙ אַרְבֶּ֣ה אֶחָ֔ד בְּכֹ֖ל גְּב֥וּל מִצְרָֽיִם׃
\rashi{\rashiDH{רוח ים. }רוח מערבי׃}\rashi{\rashiDH{ימה סוף. }אומר אני, שֶׁיַּם סוף היה מקצתו במערב כנגד כל רוח דרומית, וגם במזרח של ארץ ישראל, לפיכך רוח ים תקעו לארבה בימה סוף כנגדו, וכן מצינו לענין תחומין שהוא פונה לצד מזרח, שנאמר מִיַּם סוּף וְעַד יָם פְּלִשְׁתִּים (שמות כג, לא), ממזרח למערב, שים פלשתים במערב היה, שנאמר בפלשתים ישְׁבֵי חֶבֶל הַיָם גֹּוי כְּרֵתִים (צפניה ב, ה)׃ }\rashi{\rashiDH{לא נשאר ארבה אחד. }אף המלוחים שמלחו מהם (שמו״ר יג, ו)׃ 
}}
{וַהֲפַךְ יְיָ רוּחַ מַעְרְבָא תַּקִּיף לַחְדָּא וּנְטַל יָת גּוֹבָא וּרְמָהִי לְיַמָּא דְּסוּף לָא אִשְׁתְּאַר גּוֹבָא חַד בְּכֹל תְּחוּם מִצְרָיִם׃}
{And the \lord\space turned an exceeding strong west wind, which took up the locusts, and drove them into the Red Sea; there remained not one locust in all the border of Egypt.}{\arabic{verse}}
\threeverse{\arabic{verse}}%Ex.10:20
{וַיְחַזֵּ֥ק יְהֹוָ֖ה אֶת־לֵ֣ב פַּרְעֹ֑ה וְלֹ֥א שִׁלַּ֖ח אֶת־בְּנֵ֥י יִשְׂרָאֵֽל׃ \petucha }
{וְתַקֵּיף יְיָ יָת לִבָּא דְּפַרְעֹה וְלָא שַׁלַּח יָת בְּנֵי יִשְׂרָאֵל׃}
{But the \lord\space hardened Pharaoh’s heart, and he did not let the children of Israel go.}{\arabic{verse}}
\threeverse{\arabic{verse}}%Ex.10:21
{וַיֹּ֨אמֶר יְהֹוָ֜ה אֶל־מֹשֶׁ֗ה נְטֵ֤ה יָֽדְךָ֙ עַל־הַשָּׁמַ֔יִם וִ֥יהִי חֹ֖שֶׁךְ עַל־אֶ֣רֶץ מִצְרָ֑יִם וְיָמֵ֖שׁ חֹֽשֶׁךְ׃
\rashi{\rashiDH{וימש חשך. }ויחשיך עליהם חשך יותר מחשכו של לילה, וחשך של לילה יאמיש ויחשיך עוד׃ }\rashi{\rashiDH{וימש. }כמו ויאמש. יש לנו תיבות הרבה חסרות אל״ף, לפי שאין הברת האלף נכרת כל כך אין הכתוב מקפיד על חסרונה, כגון וְלֹא יַהֵל שָׁם עֲרָבִי (ישעי׳ יג, כ), כמו לא יאהל לא יטה אהלו. וכן וַתַּזְרֵנִי חַיִל (שמואל־ב כב, מ), כמו וַתְּאַזְּרֵנִי. ואונקלוס תרגם לשון הסרה, כמו לא ימיש בָּתַר דְּיַעְדֵי קְבֵל לֵילְיָא, כשיגיע סמוך לאור היום. אבל /אין הדבור מיושב על הוי״ו של וימש, לפי שהוא כתוב אחר ויהי חשך. ומדרש אגדה פותרו, לשון מְמַשֵּׁשׁ בַּצָּהֳרַיִם (דברים כח, כט), שהיה כפול ומכופל ועב עד שהיה בו ממש׃ 
}}
{וַאֲמַר יְיָ לְמֹשֶׁה אֲרֵים יְדָךְ עַל צֵית שְׁמַיָּא וִיהֵי חֲשׁוֹכָא עַל אַרְעָא דְּמִצְרָיִם בָּתַר דְּיִעְדֵּי קְבַל לֵילְיָא׃}
{And the \lord\space said unto Moses: ‘Stretch out thy hand toward heaven, that there may be darkness over the land of Egypt, even darkness which may be felt.’}{\arabic{verse}}
\threeverse{\arabic{verse}}%Ex.10:22
{וַיֵּ֥ט מֹשֶׁ֛ה אֶת־יָד֖וֹ עַל־הַשָּׁמָ֑יִם וַיְהִ֧י חֹֽשֶׁךְ־אֲפֵלָ֛ה בְּכׇל־אֶ֥רֶץ מִצְרַ֖יִם שְׁלֹ֥שֶׁת יָמִֽים׃
\rashi{\rashiDH{ויהי חשך אפלה שלשת ימים וגו׳. }חשך של אופל שלא ראו איש את אחיו ג׳ ימים, ועוד שלשת ימים אחרים חשך מוכפל על זה, שלא קמו איש מתחתיו, יושב אין יכול לעמוד ועומד אין יכול לישב (שמו״ר יד, ג). ולמה הביא עליהם חשך, שהיו בישראל באותו הדור רשעים, ולא היו רוצים לצאת, ומתו בשלשת ימי אפלה, כדי שלא יראו מצרים במפלתם ויאמרו אף הם לוקין כמונו. ועוד, שחפשו ישראל וראו את כליהם, וכשיצאו והיו שואלין מהן והיו אומרים אין בידינו כלום, אומר לו, אני ראיתיו בביתך ובמקום פלוני הוא (שם)׃ }\rashi{\rashiDH{שלשת ימים. }שלוש של ימים, טרציי״נא בלע״ז, וכן ז׳ ימים בכל מקום, שטיי״נא של ימים׃ }}
{וַאֲרֵים מֹשֶׁה יָת יְדֵיהּ עַל צֵית שְׁמַיָּא וַהֲוָה חֲשׁוֹךְ קְבַל בְּכָל אַרְעָא דְּמִצְרַיִם תְּלָתָא יוֹמִין׃}
{And Moses stretched forth his hand toward heaven; and there was a thick darkness in all the land of Egypt three days;}{\arabic{verse}}
\threeverse{\arabic{verse}}%Ex.10:23
{לֹֽא־רָא֞וּ אִ֣ישׁ אֶת־אָחִ֗יו וְלֹא־קָ֛מוּ אִ֥ישׁ מִתַּחְתָּ֖יו שְׁלֹ֣שֶׁת יָמִ֑ים וּֽלְכׇל־בְּנֵ֧י יִשְׂרָאֵ֛ל הָ֥יָה א֖וֹר בְּמוֹשְׁבֹתָֽם׃}
{לָא חֲזוֹ גְּבַר יָת אֲחוּהִי וְלָא קָמוּ אֲנָשׁ מִתְּחוֹתוֹהִי תְּלָתָא יוֹמִין וּלְכָל בְּנֵי יִשְׂרָאֵל הֲוָה נְהוֹרָא בְּמוֹתְבָנֵיהוֹן׃}
{they saw not one another, neither rose any from his place for three days; but all the children of Israel had light in their dwellings.}{\arabic{verse}}
\threeverse{\aliya{שלישי}}%Ex.10:24
{וַיִּקְרָ֨א פַרְעֹ֜ה אֶל־מֹשֶׁ֗ה וַיֹּ֙אמֶר֙ לְכוּ֙ עִבְד֣וּ אֶת־יְהֹוָ֔ה רַ֛ק צֹאנְכֶ֥ם וּבְקַרְכֶ֖ם יֻצָּ֑ג גַּֽם־טַפְּכֶ֖ם יֵלֵ֥ךְ עִמָּכֶֽם׃
\rashi{\rashiDH{יצג. }יהא מוצג במקומו׃ 
}}
{וּקְרָא פַרְעֹה לְמֹשֶׁה וַאֲמַר אִיזִילוּ פְלַחוּ קֳדָם יְיָ לְחוֹד עָנְכוֹן וְתוֹרֵיכוֹן שְׁבוּקוּ אַף טַפְלְכוֹן יֵיזֵיל עִמְּכוֹן׃}
{And Pharaoh called unto Moses, and said: ‘Go ye, serve the \lord; only let your flocks and your herds be stayed; let your little ones also go with you.’}{\arabic{verse}}
\threeverse{\arabic{verse}}%Ex.10:25
{וַיֹּ֣אמֶר מֹשֶׁ֔ה גַּם־אַתָּ֛ה תִּתֵּ֥ן בְּיָדֵ֖נוּ זְבָחִ֣ים וְעֹלֹ֑ת וְעָשִׂ֖ינוּ לַיהֹוָ֥ה אֱלֹהֵֽינוּ׃
\rashi{\rashiDH{גם אתה תתן. }לא דייך שמקננו ילך עמנו, אלא גם אתה תתן׃ }}
{וַאֲמַר מֹשֶׁה אַף אַתְּ תִּתֵּין בִּידַנָא נִכְסַת קוּדְשִׁין וַעֲלָוָן וְנַעֲבֵיד קֳדָם יְיָ אֱלָהַנָא׃}
{And Moses said: ‘Thou must also give into our hand sacrifices and burnt-offerings, that we may sacrifice unto the \lord\space our God.}{\arabic{verse}}
\threeverse{\arabic{verse}}%Ex.10:26
{וְגַם־מִקְנֵ֜נוּ יֵלֵ֣ךְ עִמָּ֗נוּ לֹ֤א תִשָּׁאֵר֙ פַּרְסָ֔ה כִּ֚י מִמֶּ֣נּוּ נִקַּ֔ח לַעֲבֹ֖ד אֶת־יְהֹוָ֣ה אֱלֹהֵ֑ינוּ וַאֲנַ֣חְנוּ לֹֽא־נֵדַ֗ע מַֽה־נַּעֲבֹד֙ אֶת־יְהֹוָ֔ה עַד־בֹּאֵ֖נוּ שָֽׁמָּה׃
\rashi{\rashiDH{פרסה. }פרסת רגל פלנט״א בלע״ז }\rashi{\rashiDH{לא נדע מה נעבד. }כמה תכבד העבודה, שמא ישאל יותר ממה שיש בידינו׃ }}
{וְאַף בְּעִירַנָא יֵיזֵיל עִמַּנָא לָא נַשְׁאַר מִנֵּיהּ מִדָּעַם אֲרֵי מִנֵּיהּ אֲנַחְנָא נָסְבִין לְמִפְלַח קֳדָם יְיָ אֱלָהַנָא וַאֲנַחְנָא לֵית אֲנַחְנָא יָדְעִין מָא נִפְלַח קֳדָם יְיָ עַד מֵיתַנָא לְתַמָּן׃}
{Our cattle also shall go with us; there shall not a hoof be left behind; for thereof must we take to serve the \lord\space our God; and we know not with what we must serve the \lord, until we come thither.’}{\arabic{verse}}
\threeverse{\arabic{verse}}%Ex.10:27
{וַיְחַזֵּ֥ק יְהֹוָ֖ה אֶת־לֵ֣ב פַּרְעֹ֑ה וְלֹ֥א אָבָ֖ה לְשַׁלְּחָֽם׃}
{וְתַקֵּיף יְיָ יָת לִבָּא דְּפַרְעֹה וְלָא אֲבָא לְשַׁלָּחוּתְהוֹן׃}
{But the \lord\space hardened Pharaoh’s heart, and he would not let them go.}{\arabic{verse}}
\threeverse{\arabic{verse}}%Ex.10:28
{וַיֹּֽאמֶר־ל֥וֹ פַרְעֹ֖ה לֵ֣ךְ מֵעָלָ֑י הִשָּׁ֣מֶר לְךָ֗ אַל־תֹּ֙סֶף֙ רְא֣וֹת פָּנַ֔י כִּ֗י בְּי֛וֹם רְאֹתְךָ֥ פָנַ֖י תָּמֽוּת׃}
{וַאֲמַר לֵיהּ פַּרְעֹה אִיזֵיל מֵעִלָּוָי אִסְתְּמַר לָךְ לָא תוֹסֵיף לְמִחְזֵי אַפַּי אֲרֵי בְּיוֹמָא דְּתִחְזֵי אַפַּי תְּמוּת׃}
{And Pharaoh said unto him: ‘Get thee from me, take heed to thyself, see my face no more; for in the day thou seest my face thou shalt die.’}{\arabic{verse}}
\threeverse{\arabic{verse}}%Ex.10:29
{וַיֹּ֥אמֶר מֹשֶׁ֖ה כֵּ֣ן דִּבַּ֑רְתָּ לֹא־אֹסִ֥ף ע֖וֹד רְא֥וֹת פָּנֶֽיךָ׃ \petucha 
\rashi{\rashiDH{כן דברת. }יפה דברת ובזמנו, דברת אמת שלא אוסיף עוד ראות פניך (שמו״ר יד, ד)׃ 
}}
{וַאֲמַר מֹשֶׁה יָאוּת מַלֵּילְתָּא לָא אוֹסֵיף עוֹד לְמִחְזֵי אַפָּךְ׃}
{And Moses said: ‘Thou hast spoken well; I will see thy face again no more.’}{\arabic{verse}}
\newperek
\threeverse{\Roman{chap}}%Ex.11:1
{וַיֹּ֨אמֶר יְהֹוָ֜ה אֶל־מֹשֶׁ֗ה ע֣וֹד נֶ֤גַע אֶחָד֙ אָבִ֤יא עַל־פַּרְעֹה֙ וְעַל־מִצְרַ֔יִם אַֽחֲרֵי־כֵ֕ן יְשַׁלַּ֥ח אֶתְכֶ֖ם מִזֶּ֑ה כְּשַׁ֨לְּח֔וֹ כָּלָ֕ה גָּרֵ֛שׁ יְגָרֵ֥שׁ אֶתְכֶ֖ם מִזֶּֽה׃
\rashi{\rashiDH{כלה. }גמירא, כלה כליל, כולכם ישלח׃ }}
{וַאֲמַר יְיָ לְמֹשֶׁה עוֹד מַכְתָּשׁ חַד אַיְתִי עַל פַּרְעֹה וְעַל מִצְרָאֵי בָּתַר כֵּן יְשַׁלַּח יָתְכוֹן מִכָּא כְּשַׁלָּחוּתֵיהּ גְּמֵירָא תָּרָכָא יְתָרֵיךְ יָתְכוֹן מִכָּא׃}
{And the \lord\space said unto Moses: ‘Yet one plague more will I bring upon Pharaoh, and upon Egypt; afterwards he will let you go hence; when he shall let you go, he shall surely thrust you out hence altogether.}{\Roman{chap}}
\threeverse{\arabic{verse}}%Ex.11:2
{דַּבֶּר־נָ֖א בְּאׇזְנֵ֣י הָעָ֑ם וְיִשְׁאֲל֞וּ אִ֣ישׁ \legarmeh  מֵאֵ֣ת רֵעֵ֗הוּ וְאִשָּׁה֙ מֵאֵ֣ת רְעוּתָ֔הּ כְּלֵי־כֶ֖סֶף וּכְלֵ֥י זָהָֽב׃
\rashi{\rashiDH{דבר נא. }אין נא אלא לשון בקשה, בבקשה ממך הזהירם על כך, שלא יאמר אותו צדיק אברהם, ועבדום וענו אותם קיים בהם, ואחרי כן יצאו ברכוש גדול לא קיים בהם (ברכות ט.)׃ 
}}
{מַלֵּיל כְּעַן קֳדָם עַמָּא וְיִשְׁאֲלוּן גְּבַר מִן חַבְרֵיהּ וְאִתְּתָא מִן חֲבִרְתַּהּ מָנִין דִּכְסַף וּמָנִין דִּדְהַב׃}
{Speak now in the ears of the people, and let them ask every man of his neighbour, and every woman of her neighbour, jewels of silver, and jewels of gold.’}{\arabic{verse}}
\threeverse{\arabic{verse}}%Ex.11:3
{וַיִּתֵּ֧ן יְהֹוָ֛ה אֶת־חֵ֥ן הָעָ֖ם בְּעֵינֵ֣י מִצְרָ֑יִם גַּ֣ם \legarmeh  הָאִ֣ישׁ מֹשֶׁ֗ה גָּד֤וֹל מְאֹד֙ בְּאֶ֣רֶץ מִצְרַ֔יִם בְּעֵינֵ֥י עַבְדֵֽי־פַרְעֹ֖ה וּבְעֵינֵ֥י הָעָֽם׃ \setuma         }
{וִיהַב יְיָ יָת עַמָּא לְרַחֲמִין בְּעֵינֵי מִצְרָאֵי אַף גּוּבְרָא מֹשֶׁה רַב לַחְדָּא בְּאַרְעָא דְּמִצְרַיִם בְּעֵינֵי עַבְדֵי פַּרְעֹה וּבְעֵינֵי עַמָּא׃}
{And the \lord\space gave the people favour in the sight of the Egyptians. Moreover the man Moses was very great in the land of Egypt, in the sight of Pharaoh’s servants, and in the sight of the people.}{\arabic{verse}}
\threeverse{\aliya{רביעי}}%Ex.11:4
{וַיֹּ֣אמֶר מֹשֶׁ֔ה כֹּ֖ה אָמַ֣ר יְהֹוָ֑ה כַּחֲצֹ֣ת הַלַּ֔יְלָה אֲנִ֥י יוֹצֵ֖א בְּת֥וֹךְ מִצְרָֽיִם׃
\rashi{\rashiDH{ויאמר משה כה אמר ה׳. }בעמדו לפני פרעה נאמרה לו נבואה זו, שהרי משיצא מלפניו לא הוסיף ראות פניו׃ }\rashi{\rashiDH{כחצות הלילה. }כהחלק הלילה, כחצות כמו כַּעֲלֹות (שופטים יג, כ), בַּחֲרֹות אַפָּם בָּנוּ (תהלים קכד, ג), זהו פשוטו לישבו על אופניו, שאין חצות שם דבר של חצי. ורבותינו דרשו, כמו כבחצות הלילה, ואמרו שאמר משה כחצות, דמשמע סמוך לו או לפניו או לאחריו, ולא אמר בחצות, שמא יטעו אצטגניני פרעה ויאמרו, משה בַּדָּאי הוא (ברכות ד.), אבל הקב״ה יודע עתיו ורגעיו, אמר בחצות׃ 
}}
{וַאֲמַר מֹשֶׁה כִּדְנָן אֲמַר יְיָ כְּפַלְגוּת לֵילְיָא אֲנָא מִתְגְּלֵי בְּגוֹ מִצְרָיִם׃}
{And Moses said: ‘Thus saith the \lord: About midnight will I go out into the midst of Egypt;}{\arabic{verse}}
\threeverse{\arabic{verse}}%Ex.11:5
{וּמֵ֣ת כׇּל־בְּכוֹר֮ בְּאֶ֣רֶץ מִצְרַ֒יִם֒ מִבְּכ֤וֹר פַּרְעֹה֙ הַיֹּשֵׁ֣ב עַל־כִּסְא֔וֹ עַ֚ד בְּכ֣וֹר הַשִּׁפְחָ֔ה אֲשֶׁ֖ר אַחַ֣ר הָרֵחָ֑יִם וְכֹ֖ל בְּכ֥וֹר בְּהֵמָֽה׃
\rashi{\rashiDH{עד בכור השבי. }מה לקו השבויים, כדי שלא יאמרו יראתם תבעה עלבונם, והביאה פורענות על מצרים׃ }\rashi{\rashiDH{מבכור פרעה עד בכור השפחה. }כל הפחותים מבכור פרעה וחשובים מבכור השפחה היו בכלל. ולמה לקו בני השפחות, שאף הם היו משעבדים בהם ושמחים בצרתם׃ }\rashi{\rashiDH{וכל בכור בהמה. }לפי שהיו עובדין לה, וכשהקב״ה נפרע מן האומה עובדי כוכבים, נפרע מאלהיה (מכילתא פי״ג)׃ }}
{וִימוּת כָּל בּוּכְרָא בְּאַרְעָא דְּמִצְרַיִם מִבּוּכְרָא דְּפַרְעֹה דַּעֲתִיד לְמִתַּב עַל כּוּרְסֵי מַלְכוּתֵיהּ עַד בּוּכְרָא דְּאַמְתָּא דִּבְבָּתַר רִחְיָא וְכֹל בּוּכְרָא דִּבְעִירָא׃}
{and all the first-born in the land of Egypt shall die, from the first-born of Pharaoh that sitteth upon his throne, even unto the first-born of the maid-servant that is behind the mill; and all the first-born of cattle.}{\arabic{verse}}
\threeverse{\arabic{verse}}%Ex.11:6
{וְהָ֥יְתָ֛ה צְעָקָ֥ה גְדֹלָ֖ה בְּכׇל־אֶ֣רֶץ מִצְרָ֑יִם אֲשֶׁ֤ר כָּמֹ֙הוּ֙ לֹ֣א נִהְיָ֔תָה וְכָמֹ֖הוּ לֹ֥א תֹסִֽף׃}
{וּתְהֵי צְוַחְתָּא רַבְּתָא בְּכָל אַרְעָא דְּמִצְרָיִם דִּכְוָתַהּ לָא הֲוָת וְדִכְוָתַהּ לָא תוֹסֵיף׃}
{And there shall be a great cry throughout all the land of Egypt, such as there hath been none like it, nor shall be like it any more.}{\arabic{verse}}
\threeverse{\arabic{verse}}%Ex.11:7
{וּלְכֹ֣ל \legarmeh  בְּנֵ֣י יִשְׂרָאֵ֗ל לֹ֤א יֶֽחֱרַץ־כֶּ֙לֶב֙ לְשֹׁנ֔וֹ לְמֵאִ֖ישׁ וְעַד־בְּהֵמָ֑ה לְמַ֙עַן֙ תֵּֽדְע֔וּן אֲשֶׁר֙ יַפְלֶ֣ה יְהֹוָ֔ה בֵּ֥ין מִצְרַ֖יִם וּבֵ֥ין יִשְׂרָאֵֽל׃
\rashi{\rashiDH{לא יחרץ כלב לשונו. }אומר אני שהוא לשון שנון, לא ישנן. וכן לֹא חָרַץ לִבְנֵי יִשְׁרָאֵל לְאִישׁ אֶת לְשֹׁונֹו (יהושע י, כא), לא שנן. אָז תֶּחֱרָץ (שמואל־ב ה, כד), תשתנן. לְמֹורַג חָרוּץ (ישעי׳ מא, טו), שנון. מַחְשְׁבֹות חָרוּץ (משלי כא, ה), אדם חריף ושנון. וְיַד חָרוּצִים תַּעֲשִׁיר (שם י, ד), חריפים, סוחרים שנונים׃ }\rashi{\rashiDH{אשר יפלה. }יבדיל׃}}
{וּלְכֹל בְּנֵי יִשְׂרָאֵל לָא יַנְזֵיק כַּלְבָּא בְּלִישָׁנֵיהּ לְמִבַּח לְמֵאֲנָשָׁא וְעַד בְּעִירָא בְּדִיל דְּתִדְּעוּן דְּיַפְרֵישׁ יְיָ בֵּין מִצְרָאֵי וּבֵין יִשְׂרָאֵל׃}
{But against any of the children of Israel shall not a dog whet his tongue, against man or beast; that ye may know how that the \lord\space doth put a difference between the Egyptians and Israel.}{\arabic{verse}}
\threeverse{\arabic{verse}}%Ex.11:8
{וְיָרְד֣וּ כׇל־עֲבָדֶ֩יךָ֩ אֵ֨לֶּה אֵלַ֜י וְהִשְׁתַּֽחֲווּ־לִ֣י לֵאמֹ֗ר צֵ֤א אַתָּה֙ וְכׇל־הָעָ֣ם אֲשֶׁר־בְּרַגְלֶ֔יךָ וְאַחֲרֵי־כֵ֖ן אֵצֵ֑א וַיֵּצֵ֥א מֵֽעִם־פַּרְעֹ֖ה בׇּחֳרִי־אָֽף׃ \setuma         
\rashi{\rashiDH{וירדו כל עבדיך. }חלק כבוד למלכות (זבחים קב.), שהרי בסוף ירד פרעה בעצמו אליו בלילה ואמר קוּמוּ צְאוּ מִתֹוךְ עַמִי, ולא אמר לו משה מתחלה וירדת אלי והשתחוית לי׃ }\rashi{\rashiDH{אשר ברגליך. }ההולכים אחר עצתך והלוכך׃}\rashi{\rashiDH{ואחרי כן אצא. }עם כל העם מארצך׃}\rashi{\rashiDH{ויצא מעם פרעה. }כשגמר דבריו יצא מלפניו׃}\rashi{\rashiDH{בחרי אף. }על שאמר לו אל תוסף ראות פני׃ 
}}
{וְיֵיחֲתוּן כָּל עַבְדָךְ אִלֵּין לְוָתִי וְיִבְעוֹן מִנִּי לְמֵימַר פּוֹק אַתְּ וְכָל עַמָּא דְּעִמָּךְ וּבָתַר כֵּן אֶפּוֹק וּנְפַק מִלְּוָת פַּרְעֹה בִּתְקוֹף רְגַז׃}
{And all these thy servants shall come down unto me, and bow down unto me, saying: Get thee out, and all the people that follow thee; and after that I will go out.’ And he went out from Pharaoh in hot anger.}{\arabic{verse}}
\threeverse{\arabic{verse}}%Ex.11:9
{וַיֹּ֤אמֶר יְהֹוָה֙ אֶל־מֹשֶׁ֔ה לֹא־יִשְׁמַ֥ע אֲלֵיכֶ֖ם פַּרְעֹ֑ה לְמַ֛עַן רְב֥וֹת מוֹפְתַ֖י בְּאֶ֥רֶץ מִצְרָֽיִם׃
\rashi{\rashiDH{למען רבות מופתי. }מופתי שְׁנַיִם, רבות שלשה, מכת בכורות וקריעת ים סוף ולנער את מצרים׃ }}
{וַאֲמַר יְיָ לְמֹשֶׁה לָא יְקַבֵּיל מִנְּכוֹן פַּרְעֹה בְּדִיל לְאַסְגָּאָה מוֹפְתַי בְּאַרְעָא דְּמִצְרָיִם׃}
{And the \lord\space said unto Moses: ‘Pharaoh will not hearken unto you; that My wonders may be multiplied in the land of Egypt.’}{\arabic{verse}}
\threeverse{\arabic{verse}}%Ex.11:10
{וּמֹשֶׁ֣ה וְאַהֲרֹ֗ן עָשׂ֛וּ אֶת־כׇּל־הַמֹּפְתִ֥ים הָאֵ֖לֶּה לִפְנֵ֣י פַרְעֹ֑ה וַיְחַזֵּ֤ק יְהֹוָה֙ אֶת־לֵ֣ב פַּרְעֹ֔ה וְלֹֽא־שִׁלַּ֥ח אֶת־בְּנֵֽי־יִשְׂרָאֵ֖ל מֵאַרְצֽוֹ׃ \setuma         
\rashi{\rashiDH{ומשה ואהרן עשו וגו׳. }כבר כתב לנו זאת בכל המופתים, ולא שנאה כאן אלא בשביל לסמכה לפרשה של אחריה׃ }}
{וּמֹשֶׁה וְאַהֲרֹן עֲבַדוּ יָת כָּל מוֹפְתַיָּא הָאִלֵּין קֳדָם פַּרְעֹה וְתַקֵּיף יְיָ יָת לִבָּא דְּפַרְעֹה וְלָא שַׁלַּח יָת בְּנֵי יִשְׂרָאֵל מֵאַרְעֵיהּ׃}
{And Moses and Aaron did all these wonders before Pharaoh; and the \lord\space hardened Pharaoh’s heart, and he did not let the children of Israel go out of his land.}{\arabic{verse}}
\newperek
\threeverse{\Roman{chap}}%Ex.12:1
{וַיֹּ֤אמֶר יְהֹוָה֙ אֶל־מֹשֶׁ֣ה וְאֶֽל־אַהֲרֹ֔ן בְּאֶ֥רֶץ מִצְרַ֖יִם לֵאמֹֽר׃
\rashi{\rashiDH{ויאמר ה׳ אל משה ואל אהרן. }בשביל שאהרן עשה וטרח במופתים כמשה, חלק לו כבוד זה במצוה ראשונה, שכללו עם משה בדבור׃ }\rashi{\rashiDH{בארץ מצרים. }חוץ לכרך, או אינו אלא בתוך הכרך, תלמוד לומר כצאתי את העיר וגו׳, ומה תפלה קלה לא התפלל בתוך הכרך, לפי שהיתה מלאה גילולים, דבר חמור כזה לא כל שכן (מכילתא פסחא פ״א)׃ }}
{וַאֲמַר יְיָ לְמֹשֶׁה וּלְאַהֲרֹן בְּאַרְעָא דְּמִצְרַיִם לְמֵימַר׃}
{And the \lord\space spoke unto Moses and Aaron in the land of Egypt, saying:}{\Roman{chap}}
\threeverse{\arabic{verse}}%Ex.12:2
{הַחֹ֧דֶשׁ הַזֶּ֛ה לָכֶ֖ם רֹ֣אשׁ חֳדָשִׁ֑ים רִאשׁ֥וֹן הוּא֙ לָכֶ֔ם לְחׇדְשֵׁ֖י הַשָּׁנָֽה׃
\rashi{\rashiDH{החדש הזה. }הראהו לבנה בחדושה (שמו״ר טו, כח), ואמר לו, כשהירח מתחדש יהיה לך ר״ח. ואין מקרא יוצא מידי פשוטו, על חדש ניסן אמר לו, זה יהיה ראש לסדר מנין החדשים, שיהא אייר קרוי שני, סיון שלישי (מכילתא שם)׃ }\rashi{\rashiDH{הזה. }נתקשה משה על מולד הלבנה, באיזו שעור תראה ותהיה ראויה לקדש, והראה לו באצבע את הלבנה ברקיע, ואמר לו כזה ראה וקדש. וכיצד הראהו, והלא לא היה מדבר עמו אלא ביום, שנאמר וַיְהִי בְּיֹום דִּבֶּר ה׳ (לעיל ז, כח), בְּיֹום צַוֹּתֹו (ויקרא ז, לח), מִן הַיֹום אֲשֶׁר צִוָּה ה׳ וָהָלְאָה (במדבר טו, כג), אלא סמוך לשקיעת החמה נאמרה לו פרשה זו, והראהו עם חשכה׃ }}
{יַרְחָא הָדֵין לְכוֹן רֵישׁ יַרְחַיָּא קַדְמַאי הוּא לְכוֹן לְיַרְחֵי שַׁתָּא׃}
{’This month shall be unto you the beginning of months; it shall be the first month of the year to you.}{\arabic{verse}}
\threeverse{\arabic{verse}}%Ex.12:3
{דַּבְּר֗וּ אֶֽל־כׇּל־עֲדַ֤ת יִשְׂרָאֵל֙ לֵאמֹ֔ר בֶּעָשֹׂ֖ר לַחֹ֣דֶשׁ הַזֶּ֑ה וְיִקְח֣וּ לָהֶ֗ם אִ֛ישׁ שֶׂ֥ה לְבֵית־אָבֹ֖ת שֶׂ֥ה לַבָּֽיִת׃
\rashi{\rashiDH{דברו אל כל עדת. }וכי אהרן מדבר, והלא כבר נאמר אתה תדבר, אלא חולקין כבוד זה לזה, ואומרים זה לזה למדני, והדבור יוצא מבין שניהם כאלו שניהם מדברים (מכילתא פסחא פ״ג)׃ }\rashi{\rashiDH{אל כל עדת ישראל וגו׳ בעשור לחדש. }דברו היום בראש חודש, שיקחוהו בעשור לחודש (שם)׃ 
}\rashi{\rashiDH{הזה. }פסח מצרים מקחו מבעשור, ולא פסח דורות (פסחים צו.)׃ }\rashi{\rashiDH{שה לבית אבות. }למשפחה אחת, הרי שהיו מרובין יכול שה אחד לכולן, תלמוד לומר שה לבית (מכילתא פ״ג)׃ }}
{מַלִּילוּ עִם כָּל כְּנִשְׁתָּא דְּיִשְׂרָאֵל לְמֵימַר בְּעַסְרָא לְיַרְחָא הָדֵין וְיִסְּבוּן לְהוֹן גְּבַר אִמַּר לְבֵית אַבָּא אִמְּרָא לְבֵיתָא׃}
{Speak ye unto all the congregation of Israel, saying: In the tenth day of this month they shall take to them every man a lamb, according to their fathers’ houses, a lamb for a household;}{\arabic{verse}}
\threeverse{\arabic{verse}}%Ex.12:4
{וְאִם־יִמְעַ֣ט הַבַּ֘יִת֮ מִהְי֣וֹת מִשֶּׂה֒ וְלָקַ֣ח ה֗וּא וּשְׁכֵנ֛וֹ הַקָּרֹ֥ב אֶל־בֵּית֖וֹ בְּמִכְסַ֣ת נְפָשֹׁ֑ת אִ֚ישׁ לְפִ֣י אׇכְל֔וֹ תָּכֹ֖סּוּ עַל־הַשֶּֽׂה׃
\rashi{\rashiDH{ואם ימעט הבית מהיות משה. }ואם יהיו מועטין מהיות משה אחד, שאין יכולין לאכלו ויבא לידי נותר, ולקח הוא ושכנו וגו׳, זהו משמעו לפי פשוטו. ועוד יש בו מדרש, ללמד שאחר שנמנו עליו יכולין להתמעט ולמשוך ידיהם הימנו, ולהמנות על שה אחר, אך אם באו למשוך ידיהם ולהתמעט מהיות משה, יתמעטו בעוד השה קיים, בהיותו בחיים ולא משנשחט (פסחים פט.)׃ 
}\rashi{\rashiDH{במכסת. }חשבון, וכן מִכְסַת הָעֶרְכְּך (ויקרא כז, כג)׃ }\rashi{\rashiDH{לפי אכלו. }הראוי לאכילה, פרט לחולה וזקן שאינו יכול לאכול כזית (מכילתא פ״ג)׃ }\rashi{\rashiDH{תכסו. }תִּתְמְנוּן׃}}
{וְאִם זְעֵיר בֵּיתָא מִלְּאִתְמְנָאָה עַל אִמְּרָא וְיִסַּב הוּא וְשֵׁיבָבֵיהּ דְּקָרִיב לְבֵיתֵיהּ בְּמִנְיַן נַפְשָׁתָא גְּבַר לְפוּם מֵיכְלֵיהּ תִּתְמְנוֹן עַל אִמְּרָא׃}
{and if the household be too little for a lamb, then shall he and his neighbour next unto his house take one according to the number of the souls; according to every man’s eating ye shall make your count for the lamb.}{\arabic{verse}}
\threeverse{\arabic{verse}}%Ex.12:5
{שֶׂ֥ה תָמִ֛ים זָכָ֥ר בֶּן־שָׁנָ֖ה יִהְיֶ֣ה לָכֶ֑ם מִן־הַכְּבָשִׂ֥ים וּמִן־הָעִזִּ֖ים תִּקָּֽחוּ׃
\rashi{\rashiDH{תמים. }בלא מום׃}\rashi{\rashiDH{בן שנה. }כל שנתו קרוי בן שנה, כלומר שנולד בשנה זו׃ }\rashi{\rashiDH{מן הכבשים ומן העזים. }או מזה או מזה, שאף עז קרויה שה, שנאמר וְשֵׂה עִזִּים (דברים יד, ד)׃ }}
{אִמַּר שְׁלִים דְּכַר בַּר שַׁתֵּיהּ יְהֵי לְכוֹן מִן אִמְּרַיָּא וּמִן בְּנֵי עִזַּיָּא תִּסְּבוּן׃}
{Your lamb shall be without blemish, a male of the first year; ye shall take it from the sheep, or from the goats;}{\arabic{verse}}
\threeverse{\arabic{verse}}%Ex.12:6
{וְהָיָ֤ה לָכֶם֙ לְמִשְׁמֶ֔רֶת עַ֣ד אַרְבָּעָ֥ה עָשָׂ֛ר י֖וֹם לַחֹ֣דֶשׁ הַזֶּ֑ה וְשָׁחֲט֣וּ אֹת֗וֹ כֹּ֛ל קְהַ֥ל עֲדַֽת־יִשְׂרָאֵ֖ל בֵּ֥ין הָעַרְבָּֽיִם׃
\rashi{\rashiDH{והיה לכם למשמרת. }זה לשון בקור, שטעון בקור ממום ארבעה ימים קודם שחיטה. ומפני מה הקדים לקיחתו לשחיטתו ארבעה ימים מה שלא צוה כן בפסח דורות, היה ר׳ מתיא בן חרש אומר, הרי הוא אומר וָאֶעֱבֹר עָלַיִך וָאֶרְאֵךְ וְהִנֵּה עִתֵּךְ עֵת דֹּדִים (יחזקאל טז, ח), הגיעה שבועה שנשבעתי לאברהם שאגאל את בניו, ולא היו בידם מצות להתעסק בהם כדי שיגאלו, שנאמר וְאַתְּ עֵרֹם וְעֶרְיָה (שם ז), ונתן להם שתי מצות, דם פסח ודם מילה שמלו באותו הלילה, שנאמר מִתְבֹּוסֶסֶת בְּדָמָיִך (שם ו), בשני דמים, ואומר גַּם אַתְּ בְּדַם בְּרִיתֵך שִׁלַּחְתִּי אֲסִירַיִךְ מִבֹּור אֵין מַיִם בֹּו (זכריה ט, יא), ושהיו שטופים באלילים, אמר להם מִשְכוּ וּקְחוּ לָכֶם, משכו ידיכם מאלילים וקחו לכם צאן של מצוה׃ }\rashi{\rashiDH{ושחטו אתו וגו׳. }וכי כולן שוחטין, אלא מכאן ששלוחו של אדם כמותו (מכילתא פ״ה קידושין מא׃)׃ }\rashi{\rashiDH{קהל עדת ישראל. }קהל ועדה וישראל, מכאן אמרו פסחי צבור נשחטין בשלשה כתות זו אחר זו, נכנסת כת ראשונה ננעלו דלתות העזרה וכו׳, כדאיתא בפסחים (סד.)׃ }\rashi{\rashiDH{בין הערבים. }משש שעות ולמעלה קרוי בין הערבים, שהשמש נוטה לבית מבואו לערוב, ולשון בין הערבים נראה בעיני, אותן שעות שבין עריבת היום לעריבת הלילה, עריבת היום בתחלת שבע שעות מכי ינטו צללי ערב, ועריבת הלילה בתחלת הלילה. ערב לשון נשף וחשך, כמו עָרְבָה כָּל שִׂמְחָה (ישעי׳ כד, יא)׃ }}
{וִיהֵי לְכוֹן לְמַטְּרָא עַד אַרְבְּעַת עַסְרָא יוֹמָא לְיַרְחָא הָדֵין וְיִכְּסוּן יָתֵיהּ כֹּל קְהָלָא כְּנִשְׁתָּא דְּיִשְׂרָאֵל בֵּין שִׁמְשַׁיָּא׃}
{and ye shall keep it unto the fourteenth day of the same month; and the whole assembly of the congregation of Israel shall kill it at dusk.}{\arabic{verse}}
\threeverse{\arabic{verse}}%Ex.12:7
{וְלָֽקְחוּ֙ מִן־הַדָּ֔ם וְנָ֥תְנ֛וּ עַל־שְׁתֵּ֥י הַמְּזוּזֹ֖ת וְעַל־הַמַּשְׁק֑וֹף עַ֚ל הַבָּ֣תִּ֔ים אֲשֶׁר־יֹאכְל֥וּ אֹת֖וֹ בָּהֶֽם׃
\rashi{\rashiDH{ולקחו מן הדם. }זו קבלת הדם, יכול ביד, תלמוד לומר אשר בסף׃ }\rashi{\rashiDH{המזוזות. }הם הזקופות, אחת מכאן לפתח ואחת מכאן׃ }\rashi{\rashiDH{המשקוף. }הוא העליון, שהדלת שוקף עליו כשסוגרין אותו, לינט״ל בלע״ז ולשון שקיפה, חבטה, כמו קֹול עָלֶה נִדָּף (ויקרא כו, לו), טַרְפָא דְּשָׁקִיף. חַבּוּרָה, מַשְׁקוֹפֵי׃ }\rashi{\rashiDH{על הבתים אשר יאכלו אותו בהם. }ולא על משקוף ומזוזות שבבית התבן ובבית הַבָּקָר, שאין דרין בתוכו׃ 
}}
{וְיִסְּבוּן מִן דְּמָא וְיִתְּנוּן עַל תְּרֵין סִפַּיָּא וְעַל שָׁקְפָא עַל בָּתַּיָּא דְּיֵיכְלוּן יָתֵיהּ בְּהוֹן׃}
{And they shall take of the blood, and put it on the two side-posts and on the lintel, upon the houses wherein they shall eat it.}{\arabic{verse}}
\threeverse{\arabic{verse}}%Ex.12:8
{וְאָכְל֥וּ אֶת־הַבָּשָׂ֖ר בַּלַּ֣יְלָה הַזֶּ֑ה צְלִי־אֵ֣שׁ וּמַצּ֔וֹת עַל־מְרֹרִ֖ים יֹאכְלֻֽהוּ׃
\rashi{\rashiDH{את הבשר. }ולא גידים ועצמות (מכילתא פ״ו)׃ }\rashi{\rashiDH{על מררים. }כל עשב מר נקרא מרור, וציום לאכול מרור זכר לוימררו את חייהם׃ }}
{וְיֵיכְלוּן יָת בִּסְרָא בְּלֵילְיָא הָדֵין טְוֵי נוּר וּפַטִּיר עַל מְרָרִין יֵיכְלוּנֵּיהּ׃}
{And they shall eat the flesh in that night, roast with fire, and unleavened bread; with bitter herbs they shall eat it.}{\arabic{verse}}
\threeverse{\arabic{verse}}%Ex.12:9
{אַל־תֹּאכְל֤וּ מִמֶּ֙נּוּ֙ נָ֔א וּבָשֵׁ֥ל מְבֻשָּׁ֖ל בַּמָּ֑יִם כִּ֣י אִם־צְלִי־אֵ֔שׁ רֹאשׁ֥וֹ עַל־כְּרָעָ֖יו וְעַל־קִרְבּֽוֹ׃
\rashi{\rashiDH{אל תאכלו ממנו נא. }שאינו צלוי כל צרכו קוראו נא בלשון ערבי׃}\rashi{\rashiDH{ובשל מבושל. }כל זה באזהרת לא תאכלו (פסחים מא׃)׃}\rashi{\rashiDH{במים. }מנין לשאר משקין, תלמוד לומר ובשל מבושל מכל מקום (פסחים מא.)׃ }\rashi{\rashiDH{כי אם צלי אש. }למעלה גזר עליו במצות עשה, וכאן הוסיף עליו לא תעשה, אל תאכלו ממנו כי אם צלי אש׃ }\rashi{\rashiDH{ראשו על כרעיו. }צולהו כולו כאחד עם ראשו ועם כרעיו ועם קרבו, ובני מעיו נותן לתוכו אחר הדחתן (שם עד.). ולשון על כרעיו ועל קרבו, כלשון עַל צִבְאֹתָם (שמות ו, כו), כמו בצבאותם כמות שהן, אף זה כמות שהוא, כל בשרו משלם׃ }}
{לָא תֵיכְלוּן מִנֵּיהּ כִּד חַי וְאַף לָא כִד בַּשָּׁלָא מְבוּשַּׁל בְּמַיָּא אֱלָהֵין טְוֵי נוּר רֵישֵׁיהּ עַל כְּרָעוֹהִי וְעַל גַּוֵּיהּ׃}
{Eat not of it raw, nor sodden at all with water, but roast with fire; its head with its legs and with the inwards thereof.}{\arabic{verse}}
\threeverse{\arabic{verse}}%Ex.12:10
{וְלֹא־תוֹתִ֥ירוּ מִמֶּ֖נּוּ עַד־בֹּ֑קֶר וְהַנֹּתָ֥ר מִמֶּ֛נּוּ עַד־בֹּ֖קֶר בָּאֵ֥שׁ תִּשְׂרֹֽפוּ׃
\rashi{\rashiDH{והנותר ממנו עד בקר. }מה תלמוד לומר עד בקר פעם שניה, ליתן בקר על בקר, שהבקר משמעו משעת הנץ החמה, ובא הכתוב להקדים שאסור באכילה מעלות השחר, זהו לפי משמעו. ועוד מדרש אחר, למד שאינו נשרף ביו״ט אלא ממחרת, וכך תדרשנו, והנותר ממנו בבקר ראשון, עד בקר שני תעמוד ותשרפנו׃ }}
{וְלָא תַשְׁאֲרוּן מִנֵּיהּ עַד צַפְרָא וּדְיִשְׁתְּאַר מִנֵּיהּ עַד צַפְרָא בְּנוּרָא תֵּיקְדוּן׃}
{And ye shall let nothing of it remain until the morning; but that which remaineth of it until the morning ye shall burn with fire.}{\arabic{verse}}
\threeverse{\arabic{verse}}%Ex.12:11
{וְכָ֘כָה֮ תֹּאכְל֣וּ אֹתוֹ֒ מׇתְנֵיכֶ֣ם חֲגֻרִ֔ים נַֽעֲלֵיכֶם֙ בְּרַגְלֵיכֶ֔ם וּמַקֶּלְכֶ֖ם בְּיֶדְכֶ֑ם וַאֲכַלְתֶּ֤ם אֹתוֹ֙ בְּחִפָּז֔וֹן פֶּ֥סַח ה֖וּא לַיהֹוָֽה׃
\rashi{\rashiDH{מתניכם חגורים. }מזומנים לדרך׃}\rashi{\rashiDH{בחפזון. }לשון בהלה ומהירות, כמו וַיְהִי דָוִד נֶחְפָּז לָלֶכֶת (שמואל־א כג, כו), אֲשֶׁר הִשְׁלִיכוּ אֲרָם בְּחָפְזָם (מלכים־ב ז, טו)׃ }\rashi{\rashiDH{פסח הוא לה׳. }הקרבן הוא קרוי פסח, על שם הדלוג והפסיחה שהקב״ה מדלג בתי ישראל מבין בתי מצרים, וקופץ ממצרי למצרי וישראל אמצעי נמלט, ואתם עשו כל עבודותיו לשם שמים (דבר אחר)דרך דילוג וקפיצה, זכר לשמו שקרוי פסח, וגם פסק״א לשון פסיעה׃ }}
{וּכְדֵין תֵּיכְלוּן יָתֵיהּ חַרְצֵיכוֹן יְהוֹן אֲסִירִין מְסָנֵיכוֹן בְּרַגְלֵיכוֹן וְחוּטְרֵיכוֹן בְּיַדְכוֹן וְתֵיכְלוּן יָתֵיהּ בִּבְהִילוּ פִּסְחָא הוּא קֳדָם יְיָ׃}
{And thus shall ye eat it: with your loins girded, your shoes on your feet, and your staff in your hand; and ye shall eat it in haste—it is the \lord’s passover.}{\arabic{verse}}
\threeverse{\arabic{verse}}%Ex.12:12
{וְעָבַרְתִּ֣י בְאֶֽרֶץ־מִצְרַ֘יִם֮ בַּלַּ֣יְלָה הַזֶּה֒ וְהִכֵּיתִ֤י כׇל־בְּכוֹר֙ בְּאֶ֣רֶץ מִצְרַ֔יִם מֵאָדָ֖ם וְעַד־בְּהֵמָ֑ה וּבְכׇל־אֱלֹהֵ֥י מִצְרַ֛יִם אֶֽעֱשֶׂ֥ה שְׁפָטִ֖ים אֲנִ֥י יְהֹוָֽה׃
\rashi{\rashiDH{ועברתי. }כמלך העובר ממקום למקום (מכילתא פ״ז), ובהעברה אחת וברגע אחד כולן לוקין׃ 
}\rashi{\rashiDH{כל בכור בארץ מצרים. }אף בכורות אחרים והם במצרים, ומנין אף בכורי מצרים שבמקומות אחרים, תלמוד לומר לְמַכֵּה מִצְרַיִם בִּבְכֹורֵיהֶם (תהלים קלו, י)׃ }\rashi{\rashiDH{מאדם ועד בהמה. }מי שהתחיל בעבירה תחלה ממנו מתחלת הפורענות׃}\rashi{\rashiDH{ובכל אלהי מצרים. }של עץ נרקבת, ושל מתכת נמסת ונתכת לארץ׃ }\rashi{\rashiDH{אעשה שפטים אני ה׳. }אני בעצמי, ולא על ידי שליח׃ }}
{וְאֶתְגְּלֵי בְּאַרְעָא דְּמִצְרַיִם בְּלֵילְיָא הָדֵין וְאֶקְטוֹל כָּל בּוּכְרָא בְּאַרְעָא דְּמִצְרַיִם מֵאֲנָשָׁא וְעַד בְּעִירָא וּבְכָל טָעֲוָת מִצְרָאֵי אַעֲבֵיד דִּינִין אֲנָא יְיָ׃}
{For I will go through the land of Egypt in that night, and will smite all the first-born in the land of Egypt, both man and beast; and against all the gods of Egypt I will execute judgments: I am the \lord.}{\arabic{verse}}
\threeverse{\arabic{verse}}%Ex.12:13
{וְהָיָה֩ הַדָּ֨ם לָכֶ֜ם לְאֹ֗ת עַ֤ל הַבָּתִּים֙ אֲשֶׁ֣ר אַתֶּ֣ם שָׁ֔ם וְרָאִ֙יתִי֙ אֶת־הַדָּ֔ם וּפָסַחְתִּ֖י עֲלֵכֶ֑ם וְלֹֽא־יִֽהְיֶ֨ה בָכֶ֥ם נֶ֙גֶף֙ לְמַשְׁחִ֔ית בְּהַכֹּתִ֖י בְּאֶ֥רֶץ מִצְרָֽיִם׃
\rashi{\rashiDH{והיה הדם לכם לאות. }לכם לאות ולא לאחרים לאות, מכאן שלא נתנו הדם אלא מבפנים׃ }\rashi{\rashiDH{וראיתי את הדם. }הכל גלוי לפניו, אלא אמר הקב״ה נותן אני את עיני לראות שאתם עסוקים במצותי, ופוסח אני עליכם׃ }\rashi{\rashiDH{ופסחתי. }וחמלתי, ודומה לו פסוח וְהִמְלִיט (ישעיה לא, ה). ואני אומר, כל פסיחה לשון דלוג וקפיצה, ופסחתי, מדלג היה מבתי ישראל לבתי מצרים, שהיו שרוים זה בתוך זה, וכן פֹּסְחִים עַל שְׁתֵּי הַסְּעִפִּים (מלכים־א יח, כא), וכן כל הפסחים הולכים כקופצים, וכן פסוח והמליט, מדלגו וממלטו מבין המומתים׃ }\rashi{\rashiDH{ולא יהיה בכם נגף. }אבל הווה הוא במצרים. הרי שהיה מצרי בביתו של ישראל יכול ימלט, תלמוד לומר ולא יהיה בכם נגף, אבל הווה במצרי שבבתיכם. הרי שהיה ישראל בביתו של מצרי שומע אני ילקה כמותו, תלמוד לומר ולא יהיה בכם נגף (מכילתא פ״ז)׃ }}
{וִיהֵי דֵמָא לְכוֹן לְאָת עַל בָּתַּיָּא דְּאַתּוּן תַּמָּן וְאֶחְזֵי יָת דְּמָא וַאֲחוּס עֲלֵיכוֹן וְלָא יְהֵי בְכוֹן מוֹתָא לְחַבָּלָא בְּמִקְטְלִי בְּאַרְעָא דְּמִצְרָיִם׃}
{And the blood shall be to you for a token upon the houses where ye are; and when I see the blood, I will pass over you, and there shall no plague be upon you to destroy you, when I smite the land of Egypt.}{\arabic{verse}}
\threeverse{\arabic{verse}}%Ex.12:14
{וְהָיָה֩ הַיּ֨וֹם הַזֶּ֤ה לָכֶם֙ לְזִכָּר֔וֹן וְחַגֹּתֶ֥ם אֹת֖וֹ חַ֣ג לַֽיהֹוָ֑ה לְדֹרֹ֣תֵיכֶ֔ם חֻקַּ֥ת עוֹלָ֖ם תְּחׇגֻּֽהוּ׃
\rashi{\rashiDH{לזכרון. }לדורות׃}\rashi{\rashiDH{וחגתם אותו. }יום שהוא לך לזכרון אתה חוגגו (שם), ועדיין לא שמענו אי זהו יום הזכרון, תלמוד לומר זָכֹור אֶת הַיֹּום הַזֶּה אֲשֶׁר יְצָאתֶם, למדנו שיום היציאה הוא יום של זכרון, ואיזה יום יצאו, תלמוד לומר מִמָּחֳרַת הַפֶּסַח יָצְאוּ (במדבר לג, ג), הוי אומר יום ט״ו בניסן הוא של יו״ט, שהרי ליל ט״ו אכלו את הפסח ולבקר יצאו (מכילתא שם)׃ }\rashi{\rashiDH{לדרתיכם וגו׳. }שומע אני מעוט דורות שנים, תלמוד לומר חקת עולם תחגהו (מכילתא פ״ז)׃ }}
{וִיהֵי יוֹמָא הָדֵין לְכוֹן לְדוּכְרָנָא וְתֵיחֲגוּן יָתֵיהּ חַגָּא קֳדָם יְיָ לְדָרֵיכוֹן קְיָם עָלַם תֵּיחֲגוּנֵּיהּ׃}
{And this day shall be unto you for a memorial, and ye shall keep it a feast to the \lord; throughout your generations ye shall keep it a feast by an ordinance for ever.}{\arabic{verse}}
\threeverse{\arabic{verse}}%Ex.12:15
{שִׁבְעַ֤ת יָמִים֙ מַצּ֣וֹת תֹּאכֵ֔לוּ אַ֚ךְ בַּיּ֣וֹם הָרִאשׁ֔וֹן תַּשְׁבִּ֥יתוּ שְּׂאֹ֖ר מִבָּתֵּיכֶ֑ם כִּ֣י \legarmeh  כׇּל־אֹכֵ֣ל חָמֵ֗ץ וְנִכְרְתָ֞ה הַנֶּ֤פֶשׁ הַהִוא֙ מִיִּשְׂרָאֵ֔ל מִיּ֥וֹם הָרִאשֹׁ֖ן עַד־י֥וֹם הַשְּׁבִעִֽי׃
\rashi{\rashiDH{שבעת ימים. }שטיי״נא של ימים׃ }\rashi{\rashiDH{שבעת ימים מצות תאכלו. }ובמקום אחר הוא אומר שֵׁשֶׁת יָמִים תֹּאכַל מַצֹּות (דברים טז, ח), למד על שביעי של פסח שאינו חובה לאכול מצה, ובלבד שלא יאכל חמץ, מנין אף ששה רשות, תלמוד לומר ששת ימים. זו מדה בתורה, דבר שהיה בכלל ויצא מן הכלל ללמד, לא ללמד על עצמו בלבד יצא אלא ללמד על הכלל כלו יצא, מה שביעי רשות אף ששה רשות, יכול אף לילה הראשון רשות, תלמוד לומר בערב תאכלו מצות, הכתוב קבעו חובה (פסחים קכ.)׃ }\rashi{\rashiDH{אך ביום הראשון תשביתו שאור. }מערב יום טוב, וקרוי ראשון לפי שהוא לפני השבעה, ומצינו מוקדם קרוי ראשון, כמו הֲרִאישֹׁון אָדָם תִּוָּלֵד (איוב טו, ז), הלפני אדם נולדת, או אינו אלא ראשון של שבעה, תלמוד לומר לֹא תִשְׁחַט עַל חָמֵץ, לא תשחט הפסח ועדיין חמץ קיים׃ }\rashi{\rashiDH{הנפש ההוא. }כשהיא בנפשה ובדעתה, פרט לאנוס (מכילתא פ״ח)׃ }\rashi{\rashiDH{מישראל. }שומע אני תכרת מישראל ותלך לה לעם אחר, תלמוד לומר במקום אחר, מלפני, בכל מקום שהוא רשותי׃ }}
{שִׁבְעָא יוֹמִין פַּטִּירָא תֵּיכְלוּן בְּרַם בְּיוֹמָא קַדְמָאָה תְּבַטְּלוּן חֲמִירָא מִבָּתֵּיכוֹן אֲרֵי כָל דְּיֵיכוֹל חֲמִיעַ וְיִשְׁתֵּיצֵי אֲנָשָׁא הַהוּא מִיִּשְׂרָאֵל מִיּוֹמָא קַדְמָאָה עַד יוֹמָא שְׁבִיעָאָה׃}
{Seven days shall ye eat unleavened bread; howbeit the first day ye shall put away leaven out of your houses; for whosoever eateth leavened bread from the first day until the seventh day, that soul shall be cut off from Israel.}{\arabic{verse}}
\threeverse{\arabic{verse}}%Ex.12:16
{וּבַיּ֤וֹם הָרִאשׁוֹן֙ מִקְרָא־קֹ֔דֶשׁ וּבַיּוֹם֙ הַשְּׁבִיעִ֔י מִקְרָא־קֹ֖דֶשׁ יִהְיֶ֣ה לָכֶ֑ם כׇּל־מְלָאכָה֙ לֹא־יֵעָשֶׂ֣ה בָהֶ֔ם אַ֚ךְ אֲשֶׁ֣ר יֵאָכֵ֣ל לְכׇל־נֶ֔פֶשׁ ה֥וּא לְבַדּ֖וֹ יֵעָשֶׂ֥ה לָכֶֽם׃
\rashi{\rashiDH{מקרא קדש. }מקרא שם דבר, קרא אותו קדש, לאכילה, ושתייה, וכסות (מכילתא פ״ט)׃ }\rashi{\rashiDH{לא יעשה בהם. }אפי׳ על ידי אחרים׃}\rashi{\rashiDH{הוא לבדו. }(יכול אף לעובד גלולים, תלמוד לומר הוא לבדו יעשה לכם, לכם ולא לעובד גלולים) הוא ולא מכשיריו שאפשר לעשותן מערב יום טוב׃ }\rashi{\rashiDH{לכל נפש. }אפילו לבהמה, יכול אף לנכרים, תלמוד לומר לכם (נ״א אך) (ביצה כח׃)׃ }}
{וּבְיוֹמָא קַדְמָאָה מְעָרַע קַדִּישׁ וּבְיוֹמָא שְׁבִיעָאָה מְעָרַע קַדִּישׁ יְהֵי לְכוֹן כָּל עֲבִידָא לָא יִתְעֲבֵיד בְּהוֹן בְּרַם מָא דְּמִתְאֲכִיל לְכָל נְפַשׁ הוּא בִלְחוֹדוֹהִי יִתְעֲבֵיד לְכוֹן׃}
{And in the first day there shall be to you a holy convocation, and in the seventh day a holy convocation; no manner of work shall be done in them, save that which every man must eat, that only may be done by you.}{\arabic{verse}}
\threeverse{\arabic{verse}}%Ex.12:17
{וּשְׁמַרְתֶּם֮ אֶת־הַמַּצּוֹת֒ כִּ֗י בְּעֶ֙צֶם֙ הַיּ֣וֹם הַזֶּ֔ה הוֹצֵ֥אתִי אֶת־צִבְאוֹתֵיכֶ֖ם מֵאֶ֣רֶץ מִצְרָ֑יִם וּשְׁמַרְתֶּ֞ם אֶת־הַיּ֥וֹם הַזֶּ֛ה לְדֹרֹתֵיכֶ֖ם חֻקַּ֥ת עוֹלָֽם׃
\rashi{\rashiDH{ושמרתם את המצות. }שלא יבאו לידי חמוץ, מכאן אמרו, תָּפַח, תִּלְטוֹשׁ בצונן. רבי יאשיה אומר, אל תהי קורא את הַמַּצּוֹת, אלא את הַמִּצְווֹת, כדרך שאין מחמיצין את המצות כך אין מחמיצין את המצוות, אלא אם באה לידך עשה אותה מיד׃ 
}\rashi{\rashiDH{ושמרתם את היום הזה. }ממלאכה׃}\rashi{\rashiDH{לדרתיכם חקת עולם. }לפי שלא נאמר דורות וחקת עולם על המלאכה אלא על החגיגה, לכך חזר ושנאו כאן, שלא תאמר, אזהרת כל מלאכה לא יעשה, לא לדורות נאמרה אלא לאותו הדור׃ }}
{וְתִטְּרוּן יָת פַּטִּירָא אֲרֵי בִּכְרַן יוֹמָא הָדֵין אַפֵּיקִית יָת חֵילֵיכוֹן מֵאַרְעָא דְּמִצְרָיִם וְתִטְּרוּן יָת יוֹמָא הָדֵין לְדָרֵיכוֹן קְיָם עָלַם׃}
{And ye shall observe the feast of unleavened bread; for in this selfsame day have I brought your hosts out of the land of Egypt; therefore shall ye observe this day throughout your generations by an ordinance for ever.}{\arabic{verse}}
\threeverse{\arabic{verse}}%Ex.12:18
{בָּרִאשֹׁ֡ן בְּאַרְבָּעָה֩ עָשָׂ֨ר י֤וֹם לַחֹ֙דֶשׁ֙ בָּעֶ֔רֶב תֹּאכְל֖וּ מַצֹּ֑ת עַ֠ד י֣וֹם הָאֶחָ֧ד וְעֶשְׂרִ֛ים לַחֹ֖דֶשׁ בָּעָֽרֶב׃
\rashi{\rashiDH{עד יום האחד ועשרים. }למה נאמר, והלא כבר נאמר שבעת ימים, לפי שנאמר ימים, לילות מנין, תלמוד לומר עד יום האחד וגו׳׃ 
}}
{בְּנִיסָן בְּאַרְבְּעַת עַסְרָא יוֹמָא לְיַרְחָא בְּרַמְשָׁא תֵּיכְלוּן פַּטִּירָא עַד יוֹמָא חַד וְעַסְרִין לְיַרְחָא בְּרַמְשָׁא׃}
{In the first month, on the fourteenth day of the month at even, ye shall eat unleavened bread, until the one and twentieth day of the month at even.}{\arabic{verse}}
\threeverse{\arabic{verse}}%Ex.12:19
{שִׁבְעַ֣ת יָמִ֔ים שְׂאֹ֕ר לֹ֥א יִמָּצֵ֖א בְּבָתֵּיכֶ֑ם כִּ֣י \legarmeh  כׇּל־אֹכֵ֣ל מַחְמֶ֗צֶת וְנִכְרְתָ֞ה הַנֶּ֤פֶשׁ הַהִוא֙ מֵעֲדַ֣ת יִשְׂרָאֵ֔ל בַּגֵּ֖ר וּבְאֶזְרַ֥ח הָאָֽרֶץ׃
\rashi{\rashiDH{לא ימצא בבתיכם. }מנין לגבולין, תלמוד לומר בכל גבולך. מה תלמוד לומר בבתיכם, מה ביתך ברשותך אף גבולך ברשותך, יצא חמצו של נכרי שהוא אצל ישראל ולא קבל עליו אחריות׃ }\rashi{\rashiDH{כי כל אוכל מחמצת. }לענוש כרת על השאור, והלא כבר ענש על החמץ, אלא שלא תאמר, חמץ שראוי לאכילה ענש עליו, שאור שאינו ראוי לאכילה לא יענש עליו, ואם ענש על השאור ולא ענש על החמץ, הייתי אומר, שאור שהוא מחמץ אחרים ענש עליו, חמץ שאינו מחמץ אחרים לא יענש עליו, לכך נאמרו שניהם (מכילתא פ״י)׃ }\rashi{\rashiDH{בגר ובאזרח הארץ. }לפי שהנס נעשה לישראל, הוצרך לרבות את הגרים׃ }}
{שִׁבְעָא יוֹמִין חֲמִירָא לָא יִשְׁתְּכַח בְּבָתֵּיכוֹן אֲרֵי כָל דְּיֵיכוֹל מַחְמְעָא וְיִשְׁתֵּיצֵי אֲנָשָׁא הַהוּא מִכְּנִשְׁתָּא דְּיִשְׂרָאֵל בְּגִיּוֹרַיָּא וּבְיַצִּיבַיָּא דְּאַרְעָא׃}
{Seven days shall there be no leaven found in your houses; for whosoever eateth that which is leavened, that soul shall be cut off from the congregation of Israel, whether he be a sojourner, or one that is born in the land.}{\arabic{verse}}
\threeverse{\arabic{verse}}%Ex.12:20
{כׇּל־מַחְמֶ֖צֶת לֹ֣א תֹאכֵ֑לוּ בְּכֹל֙ מוֹשְׁבֹ֣תֵיכֶ֔ם תֹּאכְל֖וּ מַצּֽוֹת׃ \petucha 
\rashi{\rashiDH{מחמצת לא תאכלו. }אזהרה על אכילת שאור׃}\rashi{\rashiDH{כל מחמצת. }להביא את תערובתו׃}\rashi{\rashiDH{בכל מושבתיכם תאכלו מצות. }זה בא ללמד שתהא ראויה ליאכל בכל מושבתיכם, פרט למעשר שני וחלות תודה (מכילתא פ״י ע״ש) (שאינה ראויה להאכל בכל מושבות אלא בירושלים)׃ }}
{כָּל מַחְמְעָא לָא תֵיכְלוּן בְּכֹל מוֹתְבָנֵיכוֹן תֵּיכְלוּן פַּטִּירָא׃}
{Ye shall eat nothing leavened; in all your habitations shall ye eat unleavened bread.’}{\arabic{verse}}
\threeverse{\aliya{חמישי}}%Ex.12:21
{וַיִּקְרָ֥א מֹשֶׁ֛ה לְכׇל־זִקְנֵ֥י יִשְׂרָאֵ֖ל וַיֹּ֣אמֶר אֲלֵהֶ֑ם מִֽשְׁכ֗וּ וּקְח֨וּ לָכֶ֥ם צֹ֛אן לְמִשְׁפְּחֹתֵיכֶ֖ם וְשַׁחֲט֥וּ הַפָּֽסַח׃
\rashi{\rashiDH{משכו. }מי שיש לו צאן ימשוך משלו׃}\rashi{\rashiDH{וקחו. }מי שאין לו יקח מן השוק׃}\rashi{\rashiDH{למשפחותיכם. }שה לבית אבות׃}}
{וּקְרָא מֹשֶׁה לְכָל סָבֵי יִשְׂרָאֵל וַאֲמַר לְהוֹן אִתְנְגִידוּ וְסַבוּ לְכוֹן מִן בְּנֵי עָנָא לְזַרְעֲיָתְכוֹן וְכוּסוּ פִסְחָא׃}
{Then Moses called for all the elders of Israel, and said unto them: ‘Draw out, and take you lambs according to your families, and kill the passover lamb.}{\arabic{verse}}
\threeverse{\arabic{verse}}%Ex.12:22
{וּלְקַחְתֶּ֞ם אֲגֻדַּ֣ת אֵז֗וֹב וּטְבַלְתֶּם֮ בַּדָּ֣ם אֲשֶׁר־בַּסַּף֒ וְהִגַּעְתֶּ֤ם אֶל־הַמַּשְׁקוֹף֙ וְאֶל־שְׁתֵּ֣י הַמְּזוּזֹ֔ת מִן־הַדָּ֖ם אֲשֶׁ֣ר בַּסָּ֑ף וְאַתֶּ֗ם לֹ֥א תֵצְא֛וּ אִ֥ישׁ מִפֶּֽתַח־בֵּית֖וֹ עַד־בֹּֽקֶר׃
\rashi{\rashiDH{אזוב. }מין ירק שיש לו גבעולין׃}\rashi{\rashiDH{אגדת אזוב. }ג׳ קלחין קרויין אגודה׃}\rashi{\rashiDH{אשר בסף. }בכלי, כמו ספות כסף׃ }\rashi{\rashiDH{מן הדם אשר בסף. }למה חזר ושנאו, שלא תאמר טבילה אחת לשלש המתנות, לכך נאמר עוד אשר בסף, שתהא כל נתינה ונתינה מן הדם אשר בסף, על כל הגעה טבילה׃ 
}\rashi{\rashiDH{ואתם לא תצאו וגו׳. }מגיד, שמאחר שנתנה רשות למשחית לחבל, אינו מבחין בין צדיק לרשע, ולילה רשות למחבלים היא, שנאמר בֹּו תִרְמֹשׂ כָּל חַיתֹו יָעַר (תהלים קד, כ)׃ }}
{וְתִסְּבוּן אֲסָרַת אֵיזוֹבָא וְתִטְבְּלוּן בִּדְמָא דִּבְמָנָא וְתַדּוֹן לְשָׁקְפָא וְלִתְרֵין סִפַּיָּא מִן דְּמָא דִּבְמָנָא וְאַתּוּן לָא תִפְּקוּן אֲנָשׁ מִתְּרַע בֵּיתֵיהּ עַד צַפְרָא׃}
{And ye shall take a bunch of hyssop, and dip it in the blood that is in the basin, and strike the lintel and the two side-posts with the blood that is in the basin; and none of you shall go out of the door of his house until the morning.}{\arabic{verse}}
\threeverse{\arabic{verse}}%Ex.12:23
{וְעָבַ֣ר יְהֹוָה֮ לִנְגֹּ֣ף אֶת־מִצְרַ֒יִם֒ וְרָאָ֤ה אֶת־הַדָּם֙ עַל־הַמַּשְׁק֔וֹף וְעַ֖ל שְׁתֵּ֣י הַמְּזוּזֹ֑ת וּפָסַ֤ח יְהֹוָה֙ עַל־הַפֶּ֔תַח וְלֹ֤א יִתֵּן֙ הַמַּשְׁחִ֔ית לָבֹ֥א אֶל־בָּתֵּיכֶ֖ם לִנְגֹּֽף׃
\rashi{\rashiDH{ופסח. }וחמל וי״ל ודלג׃ }\rashi{\rashiDH{ולא יתן המשחית. }ולא יתן לו יכולת לבא, כמו וְלֹא נְתָנֹו אֱלֹהִים לְהָרַע עִמָּדִי (בראשית לא, ז)׃ }}
{וְיִתְגְּלֵי יְיָ לְמִמְחֵי יָת מִצְרָאֵי וְיִחְזֵי יָת דְּמָא עַל שָׁקְפָא וְעַל תְּרֵין סִפַּיָּא וְיֵיחוּס יְיָ עַל תַּרְעָא וְלָא יִשְׁבּוֹק מְחַבְּלָא לְמֵיעַל לְבָתֵּיכוֹן לְמִמְחֵי׃}
{For the \lord\space will pass through to smite the Egyptians; and when He seeth the blood upon the lintel, and on the two side-posts, the \lord\space will pass over the door, and will not suffer the destroyer to come in unto your houses to smite you.}{\arabic{verse}}
\threeverse{\arabic{verse}}%Ex.12:24
{וּשְׁמַרְתֶּ֖ם אֶת־הַדָּבָ֣ר הַזֶּ֑ה לְחׇק־לְךָ֥ וּלְבָנֶ֖יךָ עַד־עוֹלָֽם׃}
{וְתִטְּרוּן יָת פִּתְגָמָא הָדֵין לִקְיָם לָךְ וְלִבְנָךְ עַד עָלְמָא׃}
{And ye shall observe this thing for an ordinance to thee and to thy sons for ever.}{\arabic{verse}}
\threeverse{\arabic{verse}}%Ex.12:25
{וְהָיָ֞ה כִּֽי־תָבֹ֣אוּ אֶל־הָאָ֗רֶץ אֲשֶׁ֨ר יִתֵּ֧ן יְהֹוָ֛ה לָכֶ֖ם כַּאֲשֶׁ֣ר דִּבֵּ֑ר וּשְׁמַרְתֶּ֖ם אֶת־הָעֲבֹדָ֥ה הַזֹּֽאת׃
\rashi{\rashiDH{והיה כי תבאו. }תלה הכתוב מצוה זו בביאתם לארץ, ולא נתחייבו במדבר אלא פסח אחד שעשו בשנה השנית על פי הדבור׃ }\rashi{\rashiDH{כאשר דבר. }והיכן דבר, וְהֵבֵאתִי אֶתְכֶם אֶל הָאָרֶץ וגו׳ (שמות ו, ח)׃ }}
{וִיהֵי אֲרֵי תֵיעֲלוּן לְאַרְעָא דְּיִתֵּין יְיָ לְכוֹן כְּמָא דְּמַלֵּיל וְתִטְּרוּן יָת פּוּלְחָנָא הָדֵין׃}
{And it shall come to pass, when ye be come to the land which the \lord\space will give you, according as He hath promised, that ye shall keep this service.}{\arabic{verse}}
\threeverse{\arabic{verse}}%Ex.12:26
{וְהָיָ֕ה כִּֽי־יֹאמְר֥וּ אֲלֵיכֶ֖ם בְּנֵיכֶ֑ם מָ֛ה הָעֲבֹדָ֥ה הַזֹּ֖את לָכֶֽם׃}
{וִיהֵי אֲרֵי יֵימְרוּן לְכוֹן בְּנֵיכוֹן מָא פוּלְחָנָא הָדֵין לְכוֹן׃}
{And it shall come to pass, when your children shall say unto you: What mean ye by this service?}{\arabic{verse}}
\threeverse{\arabic{verse}}%Ex.12:27
{וַאֲמַרְתֶּ֡ם זֶֽבַח־פֶּ֨סַח ה֜וּא לַֽיהֹוָ֗ה אֲשֶׁ֣ר פָּ֠סַ֠ח עַל־בָּתֵּ֤י בְנֵֽי־יִשְׂרָאֵל֙ בְּמִצְרַ֔יִם בְּנׇגְפּ֥וֹ אֶת־מִצְרַ֖יִם וְאֶת־בָּתֵּ֣ינוּ הִצִּ֑יל וַיִּקֹּ֥ד הָעָ֖ם וַיִּֽשְׁתַּחֲוֽוּ׃
\rashi{\rashiDH{ויקד העם. }על בשורת הגאולה, וביאת הארץ, ובשורת הבנים שיהיו להם׃ 
}}
{וְתֵימְרוּן דֵּיבַח חֲיָס הוּא קֳדָם יְיָ דְּחָס עַל בָּתֵּי בְנֵי יִשְׂרָאֵל בְּמִצְרַיִם כַּד מְחָא יָת מִצְרָאֵי וְיָת בָּתַּנָא שֵׁיזֵיב וּכְרַע עַמָּא וּסְגִידוּ׃}
{that ye shall say: It is the sacrifice of the \lord’s passover, for that He passed over the houses of the children of Israel in Egypt, when He smote the Egyptians, and delivered our houses.’ And the people bowed the head and worshipped.}{\arabic{verse}}
\threeverse{\arabic{verse}}%Ex.12:28
{וַיֵּלְכ֥וּ וַיַּֽעֲשׂ֖וּ בְּנֵ֣י יִשְׂרָאֵ֑ל כַּאֲשֶׁ֨ר צִוָּ֧ה יְהֹוָ֛ה אֶת־מֹשֶׁ֥ה וְאַהֲרֹ֖ן כֵּ֥ן עָשֽׂוּ׃ \setuma         
\rashi{\rashiDH{וילכו ויעשו בני ישראל. }וכי כבר עשו, והלא מראש חודש נאמר להם, אלא מכיון שקבלו עליהם, מעלה עליהם הכתוב כאלו עשו (מכילתא פי״ב)׃ }\rashi{\rashiDH{וילכו ויעשו. }אף ההליכה מנה הכתוב, ליתן שכר להליכה ושכר לעשייה׃ }\rashi{\rashiDH{כאשר צוה ה׳ את משה ואהרן. }להגיד שבחן של ישראל שלא הפילו דבר מכל מצות משה ואהרן, ומהו כן עשו, אף משה ואהרן כן עשו׃ }}
{וַאֲזַלוּ וַעֲבַדוּ בְנֵי יִשְׂרָאֵל כְּמָא דְּפַקֵּיד יְיָ יָת מֹשֶׁה וְאַהֲרֹן כֵּן עֲבַדוּ׃}
{And the children of Israel went and did so; as the \lord\space had commanded Moses and Aaron, so did they.}{\arabic{verse}}
\threeverse{\aliya{ששי}}%Ex.12:29
{וַיְהִ֣י \legarmeh  בַּחֲצִ֣י הַלַּ֗יְלָה וַֽיהֹוָה֮ הִכָּ֣ה כׇל־בְּכוֹר֮ בְּאֶ֣רֶץ מִצְרַ֒יִם֒ מִבְּכֹ֤ר פַּרְעֹה֙ הַיֹּשֵׁ֣ב עַל־כִּסְא֔וֹ עַ֚ד בְּכ֣וֹר הַשְּׁבִ֔י אֲשֶׁ֖ר בְּבֵ֣ית הַבּ֑וֹר וְכֹ֖ל בְּכ֥וֹר בְּהֵמָֽה׃
\rashi{\rashiDH{וה׳. }כל מקום שנאמר וה׳, הוא ובית דינו, שהוי״ו לשון תוספת הוא, כמו פלוני ופלוני׃ }\rashi{\rashiDH{הכה כל בכור. }אף של אומה אחרת והוא במצרים׃}\rashi{\rashiDH{מבכור פרעה. }אף פרעה בכור היה ונשתייר מן הבכורים, ועליו הוא אומר בַּעֲבוּר זֹאת הֶעֱמַדְתִּיךָ (שמות ט, טז)׃ }\rashi{\rashiDH{עד בכור השבי. }שהיו שמחין לאידם של ישראל, ועוד שלא יאמרו יראתנו הביאה הפורענות זו. ובכור השפחה בכלל היה, שהרי מנה מן החשוב שבכלן עד הפחות, ובכור השפחה חשוב מבכור השבי׃ }}
{וַהֲוָה בְּפַלְגוּת לֵילְיָא וַייָ קְטַל כָּל בּוּכְרָא בְּאַרְעָא דְּמִצְרַיִם מִבּוּכְרָא דְּפַרְעֹה דַּעֲתִיד לְמִתַּב עַל כּוּרְסֵי מַלְכוּתֵיהּ עַד בּוּכְרָא דְּשִׁבְיָא דִּבְבֵית אֲסִירֵי וְכֹל בּוּכְרָא דִּבְעִירָא׃}
{And it came to pass at midnight, that the \lord\space smote all the firstborn in the land of Egypt, from the first-born of Pharaoh that sat on his throne unto the first-born of the captive that was in the dungeon; and all the first-born of cattle.}{\arabic{verse}}
\threeverse{\arabic{verse}}%Ex.12:30
{וַיָּ֨קׇם פַּרְעֹ֜ה לַ֗יְלָה ה֤וּא וְכׇל־עֲבָדָיו֙ וְכׇל־מִצְרַ֔יִם וַתְּהִ֛י צְעָקָ֥ה גְדֹלָ֖ה בְּמִצְרָ֑יִם כִּֽי־אֵ֣ין בַּ֔יִת אֲשֶׁ֥ר אֵֽין־שָׁ֖ם מֵֽת׃
\rashi{\rashiDH{ויקם פרעה. }ממטתו׃}\rashi{\rashiDH{לילה. }ולא כדרך המלכים בשלש שעות ביום׃}\rashi{\rashiDH{הוא. }תחלה, ואחר כך עבדיו, מלמד שהיה הוא מחזר על בתי עבדיו ומעמידן׃ }\rashi{\rashiDH{כי אין בית אשר אין שם מת. }יש שם בכור, מת, אין שם בכור, גדול שבבית קרוי בכור, שנאמר אַף אָנִי בְּכֹור אֶתְּנֵהוּ (תהלים פט, כח). דבר אחר, מצריות מזנות תחת בעליהן ויולדות מרווקים פנויים, והיו להם בכורות הרבה, פעמים הם חמשה לאשה אחת, כל אחד בכור לאביו׃ }}
{וְקָם פַּרְעֹה בְּלֵילְיָא הוּא וְכָל עַבְדּוֹהִי וְכָל מִצְרָאֵי וַהֲוָת צְוַחְתָּא רַבְּתָא בְּמִצְרָיִם אֲרֵי לֵית בֵיתָא תַּמָּן דְּלָא הֲוָה בֵּיהּ מִיתָא׃}
{And Pharaoh rose up in the night, he, and all his servants, and all the Egyptians; and there was a great cry in Egypt; for there was not a house where there was not one dead.}{\arabic{verse}}
\threeverse{\arabic{verse}}%Ex.12:31
{וַיִּקְרָא֩ לְמֹשֶׁ֨ה וּֽלְאַהֲרֹ֜ן לַ֗יְלָה וַיֹּ֙אמֶר֙ ק֤וּמוּ צְּאוּ֙ מִתּ֣וֹךְ עַמִּ֔י גַּם־אַתֶּ֖ם גַּם־בְּנֵ֣י יִשְׂרָאֵ֑ל וּלְכ֛וּ עִבְד֥וּ אֶת־יְהֹוָ֖ה כְּדַבֶּרְכֶֽם׃
\rashi{\rashiDH{ויקרא למשה ולאהרן לילה. }מגיד שהיה מחזר על פתחי העיר וצועק, היכן משה שרוי, היכן אהרן שרוי׃ }\rashi{\rashiDH{גם אתם. }הגברים׃}\rashi{\rashiDH{גם בני ישראל. }הטף׃}\rashi{\rashiDH{ולכו עבדו וגו׳ כדברכם. }הכל כמו שאמרתם, ולא כמו שאמרתי אני, בטל לא אשלח, בטל מי ומי ההולכים, בטל רק צאנכם ובקרכם יצג׃ }}
{וּקְרָא לְמֹשֶׁה וּלְאַהֲרֹן בְּלֵילְיָא וַאֲמַר קוּמוּ פוּקוּ מִגּוֹ עַמִּי אַף אַתּוּן אַף בְּנֵי יִשְׂרָאֵל וְאִיזִילוּ פְּלַחוּ קֳדָם יְיָ כְּמָא דַּהֲוֵיתוֹן אָמְרִין׃}
{And he called for Moses and Aaron by night and said: ‘Rise up, get you forth from among my people, both ye and the children of Israel; and go, serve the \lord, as ye have said.}{\arabic{verse}}
\threeverse{\arabic{verse}}%Ex.12:32
{גַּם־צֹאנְכֶ֨ם גַּם־בְּקַרְכֶ֥ם קְח֛וּ כַּאֲשֶׁ֥ר דִּבַּרְתֶּ֖ם וָלֵ֑כוּ וּבֵֽרַכְתֶּ֖ם גַּם־אֹתִֽי׃
\rashi{\rashiDH{גם צאנכם גם בקרכם קחו. }מהו כאשר דברתם, גַּם אַתָּה תִּתֵּן בְּיָדֵינוּ זְבָחִים וְעֹלֹות (שמות י, כה)׃ }\rashi{\rashiDH{וברכתם גם אותי. }התפללו עלי שלא אמות, שאני בכור (מכילתא פי״ג)׃ }}
{אַף עָנְכוֹן אַף תּוֹרֵיכוֹן דְּבַרוּ כְּמָא דְּמַלֵּילְתּוּן וְאִיזִילוּ וְצַלּוֹ אַף עֲלָי׃}
{Take both your flocks and your herds, as ye have said, and be gone; and bless me also.’}{\arabic{verse}}
\threeverse{\arabic{verse}}%Ex.12:33
{וַתֶּחֱזַ֤ק מִצְרַ֙יִם֙ עַל־הָעָ֔ם לְמַהֵ֖ר לְשַׁלְּחָ֣ם מִן־הָאָ֑רֶץ כִּ֥י אָמְר֖וּ כֻּלָּ֥נוּ מֵתִֽים׃
\rashi{\rashiDH{כלנו מתים. }אמרו, לא כגזרת משה הוא, שהרי אמר ומת כל בכור, וכאן אף הפשוטים מתים, ה׳ או י׳ בבית אחד (שם)׃ }}
{וּתְקִיפוּ מִצְרָאֵי עַל עַמָּא לְאוֹחָאָה לְשַׁלָּחוּתְהוֹן מִן אַרְעָא אֲרֵי אֲמַרוּ כּוּלַּנָא מָיְתִין׃}
{And the Egyptians were urgent upon the people, to send them out of the land in haste; for they said: ‘We are all dead men.’}{\arabic{verse}}
\threeverse{\arabic{verse}}%Ex.12:34
{וַיִּשָּׂ֥א הָעָ֛ם אֶת־בְּצֵק֖וֹ טֶ֣רֶם יֶחְמָ֑ץ מִשְׁאֲרֹתָ֛ם צְרֻרֹ֥ת בְּשִׂמְלֹתָ֖ם עַל־שִׁכְמָֽם׃
\rashi{\rashiDH{טרם יחמץ. }המצריים לא הניחום לשהות כדי חימוץ׃}\rashi{\rashiDH{משארתם. }שירי מצה ומרור (שם)׃}\rashi{\rashiDH{על שכמם. }אע״פ שבהמות הרבה הוליכו עמהם, מחבבים היו את המצות (שם)׃ }}
{וּנְטַל עַמָּא יָת לֵישְׁהוֹן עַד לָא חֲמַע מוֹתַר אָצְוָתְהוֹן צְרִיר בִּלְבוּשֵׁיהוֹן עַל כַּתְפֵיהוֹן׃}
{And the people took their dough before it was leavened, their kneading-troughs being bound up in their clothes upon their shoulders.}{\arabic{verse}}
\threeverse{\arabic{verse}}%Ex.12:35
{וּבְנֵי־יִשְׂרָאֵ֥ל עָשׂ֖וּ כִּדְבַ֣ר מֹשֶׁ֑ה וַֽיִּשְׁאֲלוּ֙ מִמִּצְרַ֔יִם כְּלֵי־כֶ֛סֶף וּכְלֵ֥י זָהָ֖ב וּשְׂמָלֹֽת׃
\rashi{\rashiDH{כדבר משה. }שאמר להם במצרים, וְיִשְׁאֲלוּ אִישׁ מֵאֵת רֵעֵהוּ (שמות יא, ב)׃ }\rashi{\rashiDH{ושמלת. }אף הן היו חשובות להם מן הכסף ומן הזהב, והמאוחר בפסוק חשוב (מכילתא שם)׃ }}
{וּבְנֵי יִשְׂרָאֵל עֲבַדוּ כְּפִתְגָמָא דְּמֹשֶׁה וּשְׁאִילוּ מִמִּצְרַיִם מָנִין דִּכְסַף וּמָאנִין דִּדְהַב וּלְבוּשִׁין׃}
{And the children of Israel did according to the word of Moses; and they asked of the Egyptians jewels of silver, and jewels of gold, and raiment.}{\arabic{verse}}
\threeverse{\arabic{verse}}%Ex.12:36
{וַֽיהֹוָ֞ה נָתַ֨ן אֶת־חֵ֥ן הָעָ֛ם בְּעֵינֵ֥י מִצְרַ֖יִם וַיַּשְׁאִל֑וּם וַֽיְנַצְּל֖וּ אֶת־מִצְרָֽיִם׃ \petucha 
\rashi{\rashiDH{וישאלום. }אף מה שלא היו שואלים מהם היו נותנים להם, אתה אומר אחד טול שנים ולך׃ }\rashi{\rashiDH{וינצלו. }ורוקינו׃}}
{וַייָ יְהַב יָת עַמָּא לְרַחֲמִין בְּעֵינֵי מִצְרָאֵי וְאַשְׁאִילוּנוּן וְרוֹקִינוּ יָת מִצְרָיִם׃}
{And the \lord\space gave the people favour in the sight of the Egyptians, so that they let them have what they asked. And they despoiled the Egyptians.}{\arabic{verse}}
\threeverse{\arabic{verse}}%Ex.12:37
{וַיִּסְע֧וּ בְנֵֽי־יִשְׂרָאֵ֛ל מֵרַעְמְסֵ֖ס סֻכֹּ֑תָה כְּשֵׁשׁ־מֵא֨וֹת אֶ֧לֶף רַגְלִ֛י הַגְּבָרִ֖ים לְבַ֥ד מִטָּֽף׃
\rashi{\rashiDH{מרעמסס סכתה. }ק״ך מיל היו, ובאו שם לפי שעה, שנאמר וָאֶשָׂא אֶתְכֶם עַל כַּנְפֵי נְשָׁרִים (שמות יט, ד)׃ }\rashi{\rashiDH{הגברים׃ }מבן עשרים שנה ומעלה׃}}
{וּנְטַלוּ בְנֵי יִשְׂרָאֵל מֵרַעְמְסֵס לְסֻכּוֹת כְּשֵׁית מְאָה אַלְפִין גּוּבְרָא רִגְלָאָה בָּר מִטַּפְלָא׃}
{And the children of Israel journeyed from Rameses to Succoth, about six hundred thousand men on foot, beside children.}{\arabic{verse}}
\threeverse{\arabic{verse}}%Ex.12:38
{וְגַם־עֵ֥רֶב רַ֖ב עָלָ֣ה אִתָּ֑ם וְצֹ֣אן וּבָקָ֔ר מִקְנֶ֖ה כָּבֵ֥ד מְאֹֽד׃
\rashi{\rashiDH{ערב רב. }תערובות אומות של גרים׃}}
{וְאַף נוּכְרָאִין סַגִּיאִין סְלִיקוּ עִמְּהוֹן וְעָנָא וְתוֹרֵי בְּעִירָא סַגִּי לַחְדָּא׃}
{And a mixed multitude went up also with them; and flocks, and herds, even very much cattle.}{\arabic{verse}}
\threeverse{\arabic{verse}}%Ex.12:39
{וַיֹּאפ֨וּ אֶת־הַבָּצֵ֜ק אֲשֶׁ֨ר הוֹצִ֧יאוּ מִמִּצְרַ֛יִם עֻגֹ֥ת מַצּ֖וֹת כִּ֣י לֹ֣א חָמֵ֑ץ כִּֽי־גֹרְשׁ֣וּ מִמִּצְרַ֗יִם וְלֹ֤א יָֽכְלוּ֙ לְהִתְמַהְמֵ֔הַּ וְגַם־צֵדָ֖ה לֹא־עָשׂ֥וּ לָהֶֽם׃
\rashi{\rashiDH{עגות מצות. }חררה של מצה. בצק שלא החמיץ קרוי מצה׃}\rashi{\rashiDH{וגם צדה לא עשו להם. }לדרך. מגיד שבחן של ישראל, שלא אמרו האיך נצא למדבר בלא צדה, אלא האמינו והלכו (מכילתא פי״ד), הוא שמפורש בקבלה זָכַרְתִּי לָךְ חֶסֶד נְעוּרַיִךְ אַהֲבַת כְּלוּלֹתָיִךְ לֶכְתֵּךְ אַחֲרַי בַּמִדְבָּר בְּאֶרֶץ לֹא זְרוּעָה (ירמיה ב, ב), ומה שכר מפורש אחריו קֹדֶש יִשְׂרָאֵל לַה׳ וגו׳׃ 
}}
{וַאֲפוֹ יָת לֵישָׁא דְּאַפִּיקוּ מִמִּצְרַיִם גְּרִיצָן פַּטִּירָן אֲרֵי לָא חֲמַע אֲרֵי אִתָּרַכוּ מִמִּצְרַיִם וְלָא יְכִילוּ לְאִתְעַכָּבָא וְאַף זְוָדִין לָא עֲבַדוּ לְהוֹן׃}
{And they baked unleavened cakes of the dough which they brought forth out of Egypt, for it was not leavened; because they were thrust out of Egypt, and could not tarry, neither had they prepared for themselves any victual.}{\arabic{verse}}
\threeverse{\arabic{verse}}%Ex.12:40
{וּמוֹשַׁב֙ בְּנֵ֣י יִשְׂרָאֵ֔ל אֲשֶׁ֥ר יָשְׁב֖וּ בְּמִצְרָ֑יִם שְׁלֹשִׁ֣ים שָׁנָ֔ה וְאַרְבַּ֥ע מֵא֖וֹת שָׁנָֽה׃
\rashi{\rashiDH{אשר ישבו במצרים. }אחר שאר הישיבות שישבו גרים בארץ לא להם׃}\rashi{\rashiDH{שלשים שנה וארבע מאות שנה. }בין הכל, משנולד יצחק עד עכשיו היו ארבע מאות שנה, משהיה לו זרע לאברהם נתקיים כי גר יהיה זרעך, ושלשים שנה היו משנגזרה גזירת בין הבתרים עד שנולד יצחק. ואי אפשר לומר בארץ מצרים לבדה, שהרי קהת מן הבאים עם יעקב היה, צא וחשוב כל שנותיו וכל שנות עמרם בנו ושמנים של משה, לא תמצאם כל כך, ועל כרחך הרבה שנים היו לקהת עד שלא ירד למצרים, והרבה משנות עמרם נבלעים בשנות קהת, והרבה משמונים של משה נבלעים בשנות עמרם, הרי שלא תמצא ארבע מאות לביאת מצרים, והוזקקת לומר על כרחך שאף שאר הישיבות נקראו גרות, אפילו בחברון, כענין שנאמר אֲשֶׁר גָּר שָׁם אַבְרָהָם וְיִצְחָק (בראשית לה, כז), ואומר אֵת אֶרֶץ מְגֻרֵיהֶם אֲשֶׁר גָרוּ בָהּ (שמות ו, ד), לפיכך אתה צריך לומר כי גר יהיה זרעך, משהיה לו זרע, וכשתמנה ארבע מאות שנה משנולד יצחק, תמצא מביאתן למצרים עד יציאתן ר״י שנה, וזה אחד מן הדברים ששינו לתלמי המלך׃ }}
{וּמוֹתַב בְּנֵי יִשְׂרָאֵל דִּיתִיבוּ בְּמִצְרָיִם אַרְבַּע מְאָה וּתְלָתִין שְׁנִין׃}
{Now the time that the children of Israel dwelt in Egypt was four hundred and thirty years.}{\arabic{verse}}
\threeverse{\arabic{verse}}%Ex.12:41
{וַיְהִ֗י מִקֵּץ֙ שְׁלֹשִׁ֣ים שָׁנָ֔ה וְאַרְבַּ֥ע מֵא֖וֹת שָׁנָ֑ה וַיְהִ֗י בְּעֶ֙צֶם֙ הַיּ֣וֹם הַזֶּ֔ה יָ֥צְא֛וּ כׇּל־צִבְא֥וֹת יְהֹוָ֖ה מֵאֶ֥רֶץ מִצְרָֽיִם׃
\rashi{\rashiDH{ויהי מקץ שלשים שנה וגו׳ ויהי בעצם היום הזה. }מגיד, שכיון שהגיע הקץ, לא עכבן המקום כהרף עין, בט״ו בניסן באו מלאכי השרת אצל אברהם לבשרו, בט״ו בניסן נולד יצחק, ובט״ו בניסן נגזרה גזירת בין הבתרים (מכילתא פי״ד)׃ 
}}
{וַהֲוָה מִסּוֹף אַרְבַּע מְאָה וּתְלָתִין שְׁנִין וַהֲוָה בִּכְרַן יוֹמָא הָדֵין נְפַקוּ כָּל חֵילַיָּא דַּייָ מֵאַרְעָא דְּמִצְרָיִם׃}
{And it came to pass at the end of four hundred and thirty years, even the selfsame day it came to pass, that all the host of the \lord\space went out from the land of Egypt.}{\arabic{verse}}
\threeverse{\arabic{verse}}%Ex.12:42
{לֵ֣יל שִׁמֻּרִ֥ים הוּא֙ לַֽיהֹוָ֔ה לְהוֹצִיאָ֖ם מֵאֶ֣רֶץ מִצְרָ֑יִם הֽוּא־הַלַּ֤יְלָה הַזֶּה֙ לַֽיהֹוָ֔ה שִׁמֻּרִ֛ים לְכׇל־בְּנֵ֥י יִשְׂרָאֵ֖ל לְדֹרֹתָֽם׃ \petucha 
\rashi{\rashiDH{ליל שמרים. }שהיה הקב״ה שומר ומצפה לו לקיים הבטחתו להוציאם מארץ מצרים׃ }\rashi{\rashiDH{הוא הלילה הזה לה׳. }הוא הלילה שאמר לאברהם בלילה הזה אני גואל את בניך׃}\rashi{\rashiDH{שמרים לכל בני ישראל לדרתם. }מְשֻׁמָּר ובא מן המזיקין, כענין שנאמר וְלֹא יִתֵּן הַמַשְׁחִית וגו׳׃ }}
{לֵילֵי נְטִיר הוּא קֳדָם יְיָ לְאַפָּקוּתְהוֹן מֵאַרְעָא דְּמִצְרָיִם הוּא לֵילְיָא הָדֵין קֳדָם יְיָ נְטִיר לְכָל בְּנֵי יִשְׂרָאֵל לְדָרֵיהוֹן׃}
{It was a night of watching unto the \lord\space for bringing them out from the land of Egypt; this same night is a night of watching unto the \lord\space for all the children of Israel throughout their generations.}{\arabic{verse}}
\threeverse{\arabic{verse}}%Ex.12:43
{וַיֹּ֤אמֶר יְהֹוָה֙ אֶל־מֹשֶׁ֣ה וְאַהֲרֹ֔ן זֹ֖את חֻקַּ֣ת הַפָּ֑סַח כׇּל־בֶּן־נֵכָ֖ר לֹא־יֹ֥אכַל בּֽוֹ׃
\rashi{\rashiDH{זאת חקת הפסח. }בי״ד בניסן נאמרה להם פרשה זו׃ }\rashi{\rashiDH{כל בן נכר. }שנתנכרו מעשיו לאביו שבשמים (פסחים צו.), ואחד נכרי ואחד ישראל מומר במשמע (מכילתא פט״ו)׃ }}
{וַאֲמַר יְיָ לְמֹשֶׁה וְאַהֲרֹן דָּא גְּזֵירַת פִּסְחָא כָּל בַּר יִשְׂרָאֵל דְּיִשְׁתַּמַּד לָא יֵיכוֹל בֵּיהּ׃}
{And the \lord\space said unto Moses and Aaron: ‘This is the ordinance of the passover: there shall no alien eat thereof;}{\arabic{verse}}
\threeverse{\arabic{verse}}%Ex.12:44
{וְכׇל־עֶ֥בֶד אִ֖ישׁ מִקְנַת־כָּ֑סֶף וּמַלְתָּ֣ה אֹת֔וֹ אָ֖ז יֹ֥אכַל בּֽוֹ׃
\rashi{\rashiDH{ומלתה אותו אז יאכל בו. }רבו, מגיד שמילת עבדיו מעכבתו מלאכול בפסח (יבמות ע׃), דברי רבי יהושע. רבי אליעזר אומר, אין מילת עבדיו מעכבתו מלאכול בפסח, א״כ מה תלמוד לומר אז יאכל בו, העבד׃ }}
{וְכָל עֶבֶד גְּבַר זְבִין כַּסְפָּא וְתִגְזַר יָתֵיהּ בְּכֵין יֵיכוֹל בֵּיהּ׃}
{but every man’s servant that is bought for money, when thou hast circumcised him, then shall he eat thereof.}{\arabic{verse}}
\threeverse{\arabic{verse}}%Ex.12:45
{תּוֹשָׁ֥ב וְשָׂכִ֖יר לֹא־יֹ֥אכַל בּֽוֹ׃
\rashi{\rashiDH{תושב. }זה גר תושב׃}\rashi{\rashiDH{ושכיר. }זה הנכרי, ומה תלמוד לומר, והלא ערלים הם ונאמר וְכָל עָרֵל לֹא יֹאכַל בֹּו, אלא כגון ערבי מהול וגבעוני מהול והוא תושב או שכיר׃ }}
{תּוֹתָבָא וַאֲגִירָא לָא יֵיכוֹל בֵּיהּ׃}
{A sojourner and a hired servant shall not eat thereof.}{\arabic{verse}}
\threeverse{\arabic{verse}}%Ex.12:46
{בְּבַ֤יִת אֶחָד֙ יֵאָכֵ֔ל לֹא־תוֹצִ֧יא מִן־הַבַּ֛יִת מִן־הַבָּשָׂ֖ר ח֑וּצָה וְעֶ֖צֶם לֹ֥א תִשְׁבְּרוּ־בֽוֹ׃
\rashi{\rashiDH{בבית אחד יאכל. }בחבורה אחת, שלא יעשו הנמנין עליו שתי חבורות ויחלקוהו, אתה אומר בחבורה אחת או אינו אלא בבית אחד כמשמעו, וללמד שאם התחילו והיו אוכלים בחצר וירדו גשמים שלא יכנסו לבית, תלמוד לומר על הבתים אשר יאכלו אותו בהם, מכאן שהאוכל, אוכל בשני מקומות (מכילתא פט״ו)׃ }\rashi{\rashiDH{לא תוציא מן הבית. }מן החבורה׃}\rashi{\rashiDH{ועצם לא תשברו בו. }הראוי לאכילה, כגון שיש עליו כזית בשר יש בו משום שבירת עצם, אין עליו כזית בשר או מוח, אין בו משום שבירת עצם׃ }}
{בַּחֲבוּרָא חֲדָא יִתְאֲכִיל לָא תַפְּקוּן מִן בֵּיתָא מִן בִּשְׂרָא לְבָרָא וְגַרְמָא לָא תִתְבְּרוּן בֵּיהּ׃}
{In one house shall it be eaten; thou shalt not carry forth aught of the flesh abroad out of the house; neither shall ye break a bone thereof.}{\arabic{verse}}
\threeverse{\arabic{verse}}%Ex.12:47
{כׇּל־עֲדַ֥ת יִשְׂרָאֵ֖ל יַעֲשׂ֥וּ אֹתֽוֹ׃
\rashi{\rashiDH{כל עדת ישראל יעשו אותו. }למה נאמר, לפי שהוא אומר בפסח מצרים שה לבית אבות, שנמנו עליו למשפחות, יכול אף פסח דורות כן, תלמוד לומר כל עדת ישראל יעשו אותו׃ }}
{כָּל כְּנִשְׁתָּא דְּיִשְׂרָאֵל יַעְבְּדוּן יָתֵיהּ׃}
{All the congregation of Israel shall keep it.}{\arabic{verse}}
\threeverse{\arabic{verse}}%Ex.12:48
{וְכִֽי־יָג֨וּר אִתְּךָ֜ גֵּ֗ר וְעָ֣שָׂה פֶ֘סַח֮ לַיהֹוָה֒ הִמּ֧וֹל ל֣וֹ כׇל־זָכָ֗ר וְאָז֙ יִקְרַ֣ב לַעֲשֹׂת֔וֹ וְהָיָ֖ה כְּאֶזְרַ֣ח הָאָ֑רֶץ וְכׇל־עָרֵ֖ל לֹֽא־יֹ֥אכַל בּֽוֹ׃
\rashi{\rashiDH{ועשה פסח. }יכול כל המתגייר יעשה פסח מיד, תלמוד לומר והיה כאזרח הארץ, מה אזרח בארבעה עשר אף גר בארבעה עשר׃ }\rashi{\rashiDH{וכל ערל לא יאכל בו. }להביא את שמתו אחיו מחמת מילה, שאינו מומר לערלות ואינו נלמד מבן נכר לא יאכל בו׃ }}
{וַאֲרֵי יִתְגַיַּיר עִמְּכוֹן גִּיּוֹרָא וְיַעֲבֵיד פִּסְחָא קֳדָם יְיָ מִגְזַר לֵיהּ כָּל דְּכוּרָא וּבְכֵן יִקְרַב לְמִעְבְּדֵיהּ וִיהֵי כְּיַצִּיבֵי אַרְעָא וְכָל עַרְלָא לָא יֵיכוֹל בֵּיהּ׃}
{And when a stranger shall sojourn with thee, and will keep the passover to the \lord, let all his males be circumcised, and then let him come near and keep it; and he shall be as one that is born in the land; but no uncircumcised person shall eat thereof.}{\arabic{verse}}
\threeverse{\arabic{verse}}%Ex.12:49
{תּוֹרָ֣ה אַחַ֔ת יִהְיֶ֖ה לָֽאֶזְרָ֑ח וְלַגֵּ֖ר הַגָּ֥ר בְּתוֹכְכֶֽם׃
\rashi{\rashiDH{תורה אחת וגו׳. }להשוות גר לאזרח אף לשאר מצות שבתורה (מכילתא שם)׃}}
{אוֹרָיְתָא חֲדָא תְּהֵי לְיַצִּיבַיָּא וּלְגִיּוֹרַיָּא דְּיִתְגַייְּרוּן בֵּינֵיכוֹן׃}
{One law shall be to him that is homeborn, and unto the stranger that sojourneth among you.’}{\arabic{verse}}
\threeverse{\arabic{verse}}%Ex.12:50
{וַיַּֽעֲשׂ֖וּ כׇּל־בְּנֵ֣י יִשְׂרָאֵ֑ל כַּאֲשֶׁ֨ר צִוָּ֧ה יְהֹוָ֛ה אֶת־מֹשֶׁ֥ה וְאֶֽת־אַהֲרֹ֖ן כֵּ֥ן עָשֽׂוּ׃ \setuma         }
{וַעֲבַדוּ כָל בְּנֵי יִשְׂרָאֵל כְּמָא דְּפַקֵּיד יְיָ יָת מֹשֶׁה וְיָת אַהֲרֹן כֵּן עֲבַדוּ׃}
{Thus did all the children of Israel; as the \lord\space commanded Moses and Aaron, so did they.}{\arabic{verse}}
\threeverse{\arabic{verse}}%Ex.12:51
{וַיְהִ֕י בְּעֶ֖צֶם הַיּ֣וֹם הַזֶּ֑ה הוֹצִ֨יא יְהֹוָ֜ה אֶת־בְּנֵ֧י יִשְׂרָאֵ֛ל מֵאֶ֥רֶץ מִצְרַ֖יִם עַל־צִבְאֹתָֽם׃ \petucha }
{וַהֲוָה בִּכְרַן יוֹמָא הָדֵין אַפֵּיק יְיָ יָת בְּנֵי יִשְׂרָאֵל מֵאַרְעָא דְּמִצְרַיִם עַל חֵילֵיהוֹן׃}
{And it came to pass the selfsame day that the \lord\space did bring the children of Israel out of the land of Egypt by their hosts.}{\arabic{verse}}
\newperek
\threeverse{\aliya{שביעי}}%Ex.13:1
{וַיְדַבֵּ֥ר יְהֹוָ֖ה אֶל־מֹשֶׁ֥ה לֵּאמֹֽר׃}
{וּמַלֵּיל יְיָ עִם מֹשֶׁה לְמֵימַר׃}
{And the \lord\space spoke unto Moses, saying:}{\Roman{chap}}
\threeverse{\arabic{verse}}%Ex.13:2
{קַדֶּשׁ־לִ֨י כׇל־בְּכ֜וֹר פֶּ֤טֶר כׇּל־רֶ֙חֶם֙ בִּבְנֵ֣י יִשְׂרָאֵ֔ל בָּאָדָ֖ם וּבַבְּהֵמָ֑ה לִ֖י הֽוּא׃
\rashi{\rashiDH{פטר כל רחם. }שפתח את הרחם תחלה, כמו פֹּוטֵר מַיִם רֵאשִׁית מָדֹון (משלי יז, יד). וכן יַפְטִירוּ בְשָׂפָה (תהלים כב, ח), יפתחו שפתים׃ }\rashi{\rashiDH{לי הוא. }לעצמי קניתים, ע״י שהכיתי בכורי מצרים׃ }}
{אַקְדֵּישׁ קֳדָמַי כָּל בּוּכְרָא פָּתַח כָּל וַלְדָּא בִּבְנֵי יִשְׂרָאֵל בַּאֲנָשָׁא וּבִבְעִירָא דִּילִי הוּא׃}
{’Sanctify unto Me all the first-born, whatsoever opens the womb among the children of Israel, both of man and of beast, it is Mine.’}{\arabic{verse}}
\threeverse{\arabic{verse}}%Ex.13:3
{וַיֹּ֨אמֶר מֹשֶׁ֜ה אֶל־הָעָ֗ם זָכ֞וֹר אֶת־הַיּ֤וֹם הַזֶּה֙ אֲשֶׁ֨ר יְצָאתֶ֤ם מִמִּצְרַ֙יִם֙ מִבֵּ֣ית עֲבָדִ֔ים כִּ֚י בְּחֹ֣זֶק יָ֔ד הוֹצִ֧יא יְהֹוָ֛ה אֶתְכֶ֖ם מִזֶּ֑ה וְלֹ֥א יֵאָכֵ֖ל חָמֵֽץ׃
\rashi{\rashiDH{זכור את היום הזה. }למד, שמזכירין יציאת מצרים בכל יום׃ }}
{וַאֲמַר מֹשֶׁה לְעַמָּא הֲווֹ דְּכִירִין יָת יוֹמָא הָדֵין דִּנְפַקְתּוּן מִמִּצְרַיִם מִבֵּית עַבְדּוּתָא אֲרֵי בִּתְקוֹף יַד אַפֵּיק יְיָ יָתְכוֹן מִכָּא וְלָא יִתְאֲכִיל חֲמִיעַ׃}
{And Moses said unto the people: ‘Remember this day, in which ye came out from Egypt, out of the house of bondage; for by strength of hand the \lord\space brought you out from this place; there shall no leavened bread be eaten.}{\arabic{verse}}
\threeverse{\arabic{verse}}%Ex.13:4
{הַיּ֖וֹם אַתֶּ֣ם יֹצְאִ֑ים בְּחֹ֖דֶשׁ הָאָבִֽיב׃
\rashi{\rashiDH{בחדש האביב. }וכי לא היינו יודעין באיזה חדש יצאו, אלא כך אמר להם, ראו חסד שגמלכם, שהוציא אתכם בחדש שהוא כָּשֵׁר לצאת, לא חמה, ולא צנה, ולא גשמים, וכן הוא אומר מֹוצִיא אֲסִירִים בַּכֹּושָׁרֹות (תהלים סח, ז), חֹדֶשׁ שהוא כשר לצאת׃ 
}}
{יוֹמָא דֵין אַתּוּן נָפְקִין בְּיַרְחָא דַּאֲבִיבָא׃}
{This day ye go forth in the month Abib.}{\arabic{verse}}
\threeverse{\arabic{verse}}%Ex.13:5
{וְהָיָ֣ה כִֽי־יְבִיאֲךָ֣ יְהֹוָ֡ה אֶל־אֶ֣רֶץ הַֽ֠כְּנַעֲנִ֠י וְהַחִתִּ֨י וְהָאֱמֹרִ֜י וְהַחִוִּ֣י וְהַיְבוּסִ֗י אֲשֶׁ֨ר נִשְׁבַּ֤ע לַאֲבֹתֶ֙יךָ֙ לָ֣תֶת לָ֔ךְ אֶ֛רֶץ זָבַ֥ת חָלָ֖ב וּדְבָ֑שׁ וְעָבַדְתָּ֛ אֶת־הָעֲבֹדָ֥ה הַזֹּ֖את בַּחֹ֥דֶשׁ הַזֶּֽה׃
\rashi{\rashiDH{אל ארץ הכנעני וגו׳. }אע״פ שלא מנה אלא חמשה עממין, כל שבעה גוים במשמע (תנחומא בא יב), שכולן בכלל כנעני הם, ואחת ממשפחת כנען היתה שלא נקרא לה שם, אלא כנעני׃ }\rashi{\rashiDH{נשבע לאבתיך וגו׳. }באברהם הוא אומר, בַּיֹּום הַהוּא כָּרַת ה׳ אֶת אַבְרָם וגו׳ (בראשית טו, יח), וביצחק הוא אומר גּוּר בָּאָרֶץ הַזֹּאת וגו׳ (שם כו, ג), וביעקב הוא אומר הָאַרֶץ אֲשֶׁר אַתָּה שֹׁכֵב עָלֶיהָ וגו׳ (שם כח, יג)׃ }\rashi{\rashiDH{זבת חלב ודבש. }חלב זב מן העזים, והדבש זב מן התמרים ומן התאנים (רש״י מגילה ו.)׃ }\rashi{\rashiDH{את העבודה הזאת. }של פסח (פסחים צו.), והלא כבר נאמר למעלה והיה כי תבאו אל הארץ וגו׳, ולמה חזר ושנאה, בשביל דבר שנתחדש בה, בפרשה ראשונה נאמר וְהָיָה כִּי יֹאמְרוּ אֲלֵיכֶם בְּנֵיכֶם מָה הָעֲבֹדָה הַזֹּאת לָכֶם (שמות יב, כו), בבן רשע הכתוב מדבר שהוציא את עצמו מן הכלל, וכאן והגדת לבנך בבן שאינו יודע לשאול, והכתוב מלמדך שתפתח לו אתה בדברי אגדה המושכין את הלב׃ 
}}
{וִיהֵי אֲרֵי יַעֵילִנָּךְ יְיָ לַאֲרַע כְּנַעֲנָאֵי וְחִתָּאֵי וֶאֱמוֹרָאֵי וְחִוָּאֵי וִיבוּסָאֵי דְּקַיֵּים לַאֲבָהָתָךְ לְמִתַּן לָךְ אֲרַע עָבְדָא חֲלָב וּדְבַשׁ וְתִפְלַח יָת פּוּלְחָנָא הָדָא בְּיַרְחָא הָדֵין׃}
{And it shall be when the \lord\space shall bring thee into the land of the Canaanite, and the Hittite, and the Amorite, and the Hivite, and the Jebusite, which He swore unto thy fathers to give thee, a land flowing with milk and honey, that thou shalt keep this service in this month.}{\arabic{verse}}
\threeverse{\arabic{verse}}%Ex.13:6
{שִׁבְעַ֥ת יָמִ֖ים תֹּאכַ֣ל מַצֹּ֑ת וּבַיּוֹם֙ הַשְּׁבִיעִ֔י חַ֖ג לַיהֹוָֽה׃}
{שִׁבְעָא יוֹמִין תֵּיכוֹל פַּטִּירָא וּבְיוֹמָא שְׁבִיעָאָה חַגָּא קֳדָם יְיָ׃}
{Seven days thou shalt eat unleavened bread, and in the seventh day shall be a feast to the \lord.}{\arabic{verse}}
\threeverse{\arabic{verse}}%Ex.13:7
{מַצּוֹת֙ יֵֽאָכֵ֔ל אֵ֖ת שִׁבְעַ֣ת הַיָּמִ֑ים וְלֹֽא־יֵרָאֶ֨ה לְךָ֜ חָמֵ֗ץ וְלֹֽא־יֵרָאֶ֥ה לְךָ֛ שְׂאֹ֖ר בְּכׇל־גְּבֻלֶֽךָ׃}
{פַּטִּירָא יִתְאֲכִיל יָת שִׁבְעָא יוֹמִין וְלָא יִתַּחְזֵי לָךְ חֲמִיעַ וְלָא יִתַּחְזֵי לָךְ חֲמִיר בְּכָל תְּחוּמָךְ׃}
{Unleavened bread shall be eaten throughout the seven days; and there shall no leavened bread be seen with thee, neither shall there be leaven seen with thee, in all thy borders.}{\arabic{verse}}
\threeverse{\arabic{verse}}%Ex.13:8
{וְהִגַּדְתָּ֣ לְבִנְךָ֔ בַּיּ֥וֹם הַה֖וּא לֵאמֹ֑ר בַּעֲב֣וּר זֶ֗ה עָשָׂ֤ה יְהֹוָה֙ לִ֔י בְּצֵאתִ֖י מִמִּצְרָֽיִם׃
\rashi{\rashiDH{בעבור זה. }בעבור שאקיים מצותיו, כגון פסח מצה ומרור הללו׃ }\rashi{\rashiDH{עשה ה׳ לי. }רמז תשובה לבן רשע לומר, עשה ה׳ לי ולא לך, שאלו היית שם לא היית כדאי ליגאל (מכילתא פי״ז)׃ }}
{וּתְחַוֵּי לִבְרָךְ בְּיוֹמָא הַהוּא לְמֵימַר בְּדִיל דָּא עֲבַד יְיָ לִי בְּמִפְּקִי מִמִּצְרָיִם׃}
{And thou shalt tell thy son in that day, saying: It is because of that which the \lord\space did for me when I came forth out of Egypt.}{\arabic{verse}}
\threeverse{\arabic{verse}}%Ex.13:9
{וְהָיָה֩ לְךָ֨ לְא֜וֹת עַל־יָדְךָ֗ וּלְזִכָּרוֹן֙ בֵּ֣ין עֵינֶ֔יךָ לְמַ֗עַן תִּהְיֶ֛ה תּוֹרַ֥ת יְהֹוָ֖ה בְּפִ֑יךָ כִּ֚י בְּיָ֣ד חֲזָקָ֔ה הוֹצִֽאֲךָ֥ יְהֹוָ֖ה מִמִּצְרָֽיִם׃
\rashi{\rashiDH{והיה לך לאות. }יציאת מצרים תהיה לך לאות׃}\rashi{\rashiDH{על ידך ולזכרון בין עיניך. }רוצה לומר, שתכתוב פרשיות הללו ותקשרם בראש ובזרוע׃ }\rashi{\rashiDH{על ידך. }יד שמאל, לפיכך ידכה מלא בפרשה שנייה (פסוק טז), לדרוש בה יד שהיא כהה (מכילתא שם  מנחות לז.)׃ }}
{וִיהֵי לָךְ לְאָת עַל יְדָךְ וּלְדֻכְרָן בֵּין עֵינָךְ בְּדִיל דִּתְהֵי אוֹרָיְתָא דַּייָ בְּפוּמָּךְ אֲרֵי בְּיַד תַּקִּיפָא אַפְּקָךְ יְיָ מִמִּצְרָיִם׃}
{And it shall be for a sign unto thee upon thy hand, and for a memorial between thine eyes, that the law of the \lord\space may be in thy mouth; for with a strong hand hath the \lord\space brought thee out of Egypt.}{\arabic{verse}}
\threeverse{\arabic{verse}}%Ex.13:10
{וְשָׁמַרְתָּ֛ אֶת־הַחֻקָּ֥ה הַזֹּ֖את לְמוֹעֲדָ֑הּ מִיָּמִ֖ים יָמִֽימָה׃ \petucha 
\rashi{\rashiDH{מימים ימימה. }משנה לשנה (שם לו׃)׃}}
{וְתִטַּר יָת קְיָמָא הָדֵין בְּזִמְנֵיהּ מִזְּמָן לִזְמָן׃}
{Thou shalt therefore keep this ordinance in its season from year to year.}{\arabic{verse}}
\threeverse{\arabic{verse}}%Ex.13:11
{וְהָיָ֞ה כִּֽי־יְבִאֲךָ֤ יְהֹוָה֙ אֶל־אֶ֣רֶץ הַֽכְּנַעֲנִ֔י כַּאֲשֶׁ֛ר נִשְׁבַּ֥ע לְךָ֖ וְלַֽאֲבֹתֶ֑יךָ וּנְתָנָ֖הּ לָֽךְ׃
\rashi{\rashiDH{והיה כי יבאך. }יש מרבותינו שלמדו מכאן, שלא קדשו בכורות הנולדים במדבר, והאומר קדשו מפרש ביאה זו, אם תקיימוהו במדבר, תזכו ליכנס לארץ ותקיימוהו שם (מכילתא שם)׃ }\rashi{\rashiDH{נשבע לך. }והיכן נשבע לך, וְהֵבֵאתִי אֶתְכֶם אֶל הָאָרֶץ אֲשֶׁר נָשָׂאתִי וגו׳ (שמות ו, ח)׃ }\rashi{\rashiDH{ונתנה לך. }תהא בעיניך כאילו נתנה לך בו ביום, ואל תהי בעיניך כירושת אבות (מכילתא פי״ח)׃ }}
{וִיהֵי אֲרֵי יַעֵילִנָּךְ יְיָ לַאֲרַע כְּנַעֲנָאֵי כְּמָא דְּקַיֵּים לָךְ וְלַאֲבָהָתָךְ וְיִתְּנַהּ לָךְ׃}
{And it shall be when the \lord\space shall bring thee into the land of the Canaanite, as He swore unto thee and to thy fathers, and shall give it thee,}{\arabic{verse}}
\threeverse{\arabic{verse}}%Ex.13:12
{וְהַעֲבַרְתָּ֥ כׇל־פֶּֽטֶר־רֶ֖חֶם לַֽיהֹוָ֑ה וְכׇל־פֶּ֣טֶר \legarmeh  שֶׁ֣גֶר בְּהֵמָ֗ה אֲשֶׁ֨ר יִהְיֶ֥ה לְךָ֛ הַזְּכָרִ֖ים לַיהֹוָֽה׃
\rashi{\rashiDH{והעברת. }אין והעברת אלא לשון הפרשה, וכן הוא אומר וְהַעֲבַרְתֶּם אֶת נַחֲלָתֹו לְבִתֹּו (במדבר כז, ח)׃ }\rashi{\rashiDH{שגר בהמה. }נֵפֶל, ששגרתו אמו ושלחתו בלא עתו, ולמדך הכתוב שהוא קדוש בבכורה לפטור את הבא אחריו, ואף שאינו נפל קרוי שגר, כמו שְׁגַר אֲלָפֶיךָ, אבל זה לא בא אלא ללמד על הנפל, שהרי כבר כתב כל פטר רחם, ואם תאמר אף בכור בהמה טמאה במשמע, בא ופירש במקום אחר בִּבְקָרְךָ וּבְצֹאנְךָ. לשון אחר יש לפרש והעברת כל פטר רחם, בבכור אדם הכתוב מדבר׃ }}
{וְתַעְבַּר כָּל פָּתַח וַלְדָּא קֳדָם יְיָ וְכָל פָּתַח וְלַד בְּעִירָא דִּיהוֹן לָךְ דִּכְרִין תַּקְדֵּישׁ קֳדָם יְיָ׃}
{that thou shalt set apart unto the \lord\space all that openeth the womb; every firstling that is a male, which thou hast coming of a beast, shall be the \lord’s.}{\arabic{verse}}
\threeverse{\arabic{verse}}%Ex.13:13
{וְכׇל־פֶּ֤טֶר חֲמֹר֙ תִּפְדֶּ֣ה בְשֶׂ֔ה וְאִם־לֹ֥א תִפְדֶּ֖ה וַעֲרַפְתּ֑וֹ וְכֹ֨ל בְּכ֥וֹר אָדָ֛ם בְּבָנֶ֖יךָ תִּפְדֶּֽה׃
\rashi{\rashiDH{פטר חמור. }ולא פטר שאר בהמה טמאה, גזרת הכתוב היא, לפי שנמשלו בכורי מצרים לחמורים, ועוד שסייעו את ישראל ביציאתן ממצרים, (שאין לך אחד מישראל שלא נטל הרבה חמורים) טעונים מכספם ומזהבם של מצרים׃ }\rashi{\rashiDH{תפדה בשה. }נותן שה לכהן, ופטר חמור מותר בהנאה והשה חולין ביד כהן׃ }\rashi{\rashiDH{וערפתו. }עורפו בקופיץ מאחוריו והורגו, הוא הפסיד ממונו של כהן לפיכך יפסיד ממונו (מכילתא פי״ח)׃ }\rashi{\rashiDH{וכל בכור אדם בבניך תפדה. }חמש סלעים פדיונו, קצוב במקום אחר׃ }}
{וְכָל בּוּכְרָא דִּחְמָרָא תִּפְרוּק בְּאִמְּרָא וְאִם לָא תִפְרוּק וְתִקְפֵיהּ וְכֹל בּוּכְרָא דַּאֲנָשָׁא בִּבְנָךְ תִּפְרוּק׃}
{And every firstling of an ass thou shalt redeem with a lamb; and if thou wilt not redeem it, then thou shalt break its neck; and all the first-born of man among thy sons shalt thou redeem.}{\arabic{verse}}
\threeverse{\aliya{מפטיר}}%Ex.13:14
{וְהָיָ֞ה כִּֽי־יִשְׁאָלְךָ֥ בִנְךָ֛ מָחָ֖ר לֵאמֹ֣ר מַה־זֹּ֑את וְאָמַרְתָּ֣ אֵלָ֔יו בְּחֹ֣זֶק יָ֗ד הוֹצִיאָ֧נוּ יְהֹוָ֛ה מִמִּצְרַ֖יִם מִבֵּ֥ית עֲבָדִֽים׃
\rashi{\rashiDH{כי ישאלך בנך מחר. }יש מחר שהיא עכשיו ויש מחר שהוא לאחר זמן, כגון זה, וכגון מָחָר יֹאמְרוּ בְנֵיכֶם לְבָנֵינוּ (יהושע כב, כז), דבני גד ובני ראובן׃ }\rashi{\rashiDH{מה זאת. }זה תינוק טפש שאינו יודע להעמיק שאלתו, וסותם ושואל מה זאת, ובמקום אחר הוא אומר מָה הָעֵדֹת וְהַחֻקִּים וְהַמִּשְׁפָּטִים וגו׳ (דברים ו, כ), הרי זאת שאלת בן חכם. דברה תורה כנגד ארבעה בנים, תם, רשע, ושאינו יודע לשאול, והשואל דרך חכמה׃ }}
{וִיהֵי אֲרֵי יִשְׁאֲלִנָּךְ בְּרָךְ מְחַר לְמֵימַר מָא דָא וְתֵימַר לֵיהּ בִּתְקוֹף יַד אַפְּקַנָא יְיָ מִמִּצְרַיִם מִבֵּית עַבְדּוּתָא׃}
{And it shall be when thy son asketh thee in time to come, saying: What is this? that thou shalt say unto him: By strength of hand the \lord\space brought us out from Egypt, from the house of bondage;}{\arabic{verse}}
\threeverse{\arabic{verse}}%Ex.13:15
{וַיְהִ֗י כִּֽי־הִקְשָׁ֣ה פַרְעֹה֮ לְשַׁלְּחֵ֒נוּ֒ וַיַּהֲרֹ֨ג יְהֹוָ֤ה כׇּל־בְּכוֹר֙ בְּאֶ֣רֶץ מִצְרַ֔יִם מִבְּכֹ֥ר אָדָ֖ם וְעַד־בְּכ֣וֹר בְּהֵמָ֑ה עַל־כֵּן֩ אֲנִ֨י זֹבֵ֜חַ לַֽיהֹוָ֗ה כׇּל־פֶּ֤טֶר רֶ֙חֶם֙ הַזְּכָרִ֔ים וְכׇל־בְּכ֥וֹר בָּנַ֖י אֶפְדֶּֽה׃}
{וַהֲוָה כַּד אַקְשִׁי פַרְעֹה לְשַׁלָּחוּתַנָא וּקְטַל יְיָ כָּל בּוּכְרָא בְּאַרְעָא דְּמִצְרַיִם מִבּוּכְרָא דַּאֲנָשָׁא וְעַד בּוּכְרָא דִּבְעִירָא עַל כֵּן אֲנָא דָּבַח קֳדָם יְיָ כָּל פָּתַח וְלַד דִּכְרַיָּא וְכָל בּוּכְרָא דִּבְנַי אֶפְרוּק׃}
{and it came to pass, when Pharaoh would hardly let us go that the \lord\space slew all the firstborn in the land of Egypt, both the first-born of man, and the first-born of beast; therefore I sacrifice to the \lord\space all that openeth the womb, being males; but all the first-born of my sons I redeem.}{\arabic{verse}}
\threeverse{\arabic{verse}}%Ex.13:16
{וְהָיָ֤ה לְאוֹת֙ עַל־יָ֣דְכָ֔ה וּלְטוֹטָפֹ֖ת בֵּ֣ין עֵינֶ֑יךָ כִּ֚י בְּחֹ֣זֶק יָ֔ד הוֹצִיאָ֥נוּ יְהֹוָ֖ה מִמִּצְרָֽיִם׃ \setuma         
\rashi{\rashiDH{ולטוטפות בין עיניך. }תפילין, ועל שם שהם ארבעה בתים קרוין טטפת, טט בְּכַתְּפֵי שתים, פת באפריקי שתים (סנהדרין ד׃). ומנחם חברו עם וְהַטֵּף אֶל דָּרֹום (יחזקאל כא, ב), אַל תַּטִיפוּ (מיכה ב, ו), לשון דבור, כמו ולזכרון בין עיניך האמורה בפרשה ראשונה, שהרואה אותם קשורים בין העינים, יזכור הנס וידבר בו׃ 
}}
{וִיהֵי לְאָת עַל יְדָךְ וְלִתְפִלִּין בֵּין עֵינָךְ אֲרֵי בִּתְקוֹף יַד אַפְּקַנָא יְיָ מִמִּצְרָיִם׃}
{And it shall be for a sign upon thy hand, and for frontlets between your eyes; for by strength of hand the \lord\space brought us forth out of Egypt.’}{\arabic{verse}}
\newparsha{בשלח}
\threeverse{\aliya{בשלח}}%Ex.13:17
{וַיְהִ֗י בְּשַׁלַּ֣ח פַּרְעֹה֮ אֶת־הָעָם֒ וְלֹא־נָחָ֣ם אֱלֹהִ֗ים דֶּ֚רֶךְ אֶ֣רֶץ פְּלִשְׁתִּ֔ים כִּ֥י קָר֖וֹב ה֑וּא כִּ֣י \legarmeh  אָמַ֣ר אֱלֹהִ֗ים פֶּֽן־יִנָּחֵ֥ם הָעָ֛ם בִּרְאֹתָ֥ם מִלְחָמָ֖ה וְשָׁ֥בוּ מִצְרָֽיְמָה׃
\rashi{\rashiDH{ויהי בשלח פרעה וגו׳ ולא נחם. }ולא נהגם, כמו לֵךְ נְחֵה אֶת הָעָם (שמות לב, לד), בְּהִתְהַלֶּכְךָ תַּנְחֶה אֹתָךְ (משלי ו, כב)׃ }\rashi{\rashiDH{כי קרוב הוא. }ונוח לשוב באותו הדרך למצרים. ומדרש אגדה יש הרבה׃}\rashi{\rashiDH{בראותם  מלחמה. }כגון מלחמת וַיֵּרֶד הָעֲמָלֵקִי וְהַכְּנַעֲנִי וגו׳ (במדבר יד, מב), אם הלכו דרך ישר היו חוזרים, ומה אם כשהקיפם דרך מעוקם אמרו נִתְּנָה רֹאשׁ וְנָשׁוּבָה מִצְרָיְמָה, אם הוליכם בפשוטה על אחת כמה וכמה (מכילתא פסחא פי״ח). (לפי סדר הכתוב נראה הרשימות מהופכים, ועיין ברא״ם ובג״א ובמ״י ישוב נכון ע״ז)׃ }\rashi{\rashiDH{פן ינחם. }יחשבו מחשבה על שיצאו, ויתנו לב לשוב׃ }}
{וַהֲוָה כַּד שַׁלַּח פַּרְעֹה יָת עַמָּא וְלָא דַּבַּרִנּוּן יְיָ אוֹרַח אֲרַע פְּלִשְׁתָּאֵי אֲרֵי קָרִיבָא הִיא אֲרֵי אֲמַר יְיָ דִּלְמָא יְזוּעוּן עַמָּא בְּמִחְזֵיהוֹן קְרָבָא וִיתוּבוּן לְמִצְרָיִם׃}
{And it came to pass, when Pharaoh had let the people go, that God led them not by the way of the land of the Philistines, although that was near; for God said: 'Lest the people regret when they see war, and they return to Egypt.’}{\arabic{verse}}
\threeverse{\arabic{verse}}%Ex.13:18
{וַיַּסֵּ֨ב אֱלֹהִ֧ים \pasek  אֶת־הָעָ֛ם דֶּ֥רֶךְ הַמִּדְבָּ֖ר יַם־ס֑וּף וַחֲמֻשִׁ֛ים עָל֥וּ בְנֵי־יִשְׂרָאֵ֖ל מֵאֶ֥רֶץ מִצְרָֽיִם׃
\rashi{\rashiDH{ויסב. }הסיבם מן הדרך הפשוטה לדרך העקומה׃}\rashi{\rashiDH{ים סוף. }כמו לים סוף. וסוף הוא לשון אגם שֶׁגְּדֵלִים בו קנים, כמו וַתָּשֶׂם בַּסּוּף (שמות ב, ג), קָנֶה וָסוּף קָמֵלוּ (ישעיה יט, ו)׃ }\rashi{\rashiDH{וחמשים. }אין חמושים אלא מזויינים, (לפי שהסיבתן במדבר גרם להם שעלו חמושים, שאילו היה דרך ישוב לא היו מחומשים להם כל מה שצריכין, אלא כאדם שעובר ממקום למקום ובדעתו לקנות שם מה שיצטרך, אבל כשהוא פורש למדבר צריך לזמן כל הצורך. וכתוב זה לא נכתב כי אם לשבר את האוזן, שלא תאמר, במלחמת עמלק ובמלחמת סיחון ועוג ומדין מהיכן היו להם כלי זיין שהכו ישראל בחרב. ברש״י ישן) וכן הוא אומר, וְאַתֶּם תַּעַבְרוּ חֲמֻשִׁים (יהושע א, יד), וכן תרגם אונקלוס מְזָרְזִין, כמו וַיֶָּרק אֶת חֲנִיכָיו (בראשית יד, יד) וזריז. דבר אחר חמושים מחומשים, אחד מחמשה יצאו, וארבעה חלקים מתו בשלשת ימי אפילה׃ }}
{וְאַסְחַר יְיָ יָת עַמָּא אוֹרַח מַדְבְּרָא לְיַמָּא דְּסוּף וּמְזָרְזִין סְלִיקוּ בְנֵי יִשְׂרָאֵל מֵאַרְעָא דְּמִצְרָיִם׃}
{But God led the people about, by the way of the wilderness by the Red Sea; and the children of Israel went up armed out of the land of Egypt.}{\arabic{verse}}
\threeverse{\arabic{verse}}%Ex.13:19
{וַיִּקַּ֥ח מֹשֶׁ֛ה אֶת־עַצְמ֥וֹת יוֹסֵ֖ף עִמּ֑וֹ כִּי֩ הַשְׁבֵּ֨עַ הִשְׁבִּ֜יעַ אֶת־בְּנֵ֤י יִשְׂרָאֵל֙ לֵאמֹ֔ר פָּקֹ֨ד יִפְקֹ֤ד אֱלֹהִים֙ אֶתְכֶ֔ם וְהַעֲלִיתֶ֧ם אֶת־עַצְמֹתַ֛י מִזֶּ֖ה אִתְּכֶֽם׃
\rashi{\rashiDH{השבע השביע. }השביעם שישביעו לבניהם, למה לא השביע בניו שישאוהו לארץ כנען מיד, כמו שהשביע יעקב, אמר יוסף, אני שליט הייתי במצרים והיה ספוק בידי לעשות, אבל בָּנַי לא יניחום מצרים לעשות, לכך השביעם לכשיגאלו ויצאו משם, שישאוהו (מכילתא פי״ח)׃ 
}\rashi{\rashiDH{והעליתם את עצמותי מזה אתכם. }לאחיו השביע כן, למדנו שאף עצמות כל השבטים העלו עמהם, שנאמר אתכם׃ }}
{וַאַסֵּיק מֹשֶׁה יָת גַּרְמֵי יוֹסֵף עִמֵּיהּ אֲרֵי אוֹמָאָה אוֹמִי יָת בְּנֵי יִשְׂרָאֵל לְמֵימַר מִדְכָר דְּכִיר יְיָ יָתְכוֹן וְתַסְּקוּן יָת גַּרְמַי מִכָּא עִמְּכוֹן׃}
{And Moses took the bones of Joseph with him; for he had surely sworn the children of Israel, saying: ‘God will surely remember you; and ye shall carry up my bones away hence with you.’}{\arabic{verse}}
\threeverse{\arabic{verse}}%Ex.13:20
{וַיִּסְע֖וּ מִסֻּכֹּ֑ת וַיַּחֲנ֣וּ בְאֵתָ֔ם בִּקְצֵ֖ה הַמִּדְבָּֽר׃
\rashi{\rashiDH{ויסעו מסכות. }ביום השני, שהרי בראשון באו מרעמסס לסכות׃ }}
{וּנְטַלוּ מִסּוּכּוֹת וּשְׁרוֹ בְאֵיתָם בִּסְטַר מַדְבְּרָא׃}
{And they took their journey from Succoth, and encamped in Etham, in the edge of the wilderness.}{\arabic{verse}}
\threeverse{\arabic{verse}}%Ex.13:21
{וַֽיהֹוָ֡ה הֹלֵךְ֩ לִפְנֵיהֶ֨ם יוֹמָ֜ם בְּעַמּ֤וּד עָנָן֙ לַנְחֹתָ֣ם הַדֶּ֔רֶךְ וְלַ֛יְלָה בְּעַמּ֥וּד אֵ֖שׁ לְהָאִ֣יר לָהֶ֑ם לָלֶ֖כֶת יוֹמָ֥ם וָלָֽיְלָה׃
\rashi{\rashiDH{לנחתם הדרך. }נקוד פתח, שהוא כמו להנחותם, כמו לַרְאֹתְכֶם בַּדֶּרֶךְ אֲשֶׁר תֵּלְכוּ בָהּ (דברים א, לג), שהוא כמו להראותכם, אף כאן להנחותם ע״י שליח, ומי הוא השליח, עמוד הענן, והקב״ה בכבודו מוליכו לפניהם, ומכל מקום את עמוד הענן הכין להנחותם על ידו, שהרי על ידי עמוד הענן הם הולכים. עמוד הענן אינו לאורה אלא להורותם הדרך׃ }}
{וַייָ מְדַבַּר קֳדָמֵיהוֹן בִּימָמָא בְּעַמּוּדָא דַּעֲנָנָא לְדַבָּרוּתְהוֹן בְּאוֹרְחָא וּבְלֵילְיָא בְּעַמּוּדָא דְּאִישָׁתָא לְאַנְהָרָא לְהוֹן לְמֵיזַל בִּימָמָא וּבְלֵילְיָא׃}
{And the \lord\space went before them by day in a pillar of cloud, to lead them the way; and by night in a pillar of fire, to give them light; that they might go by day and by night:}{\arabic{verse}}
\threeverse{\arabic{verse}}%Ex.13:22
{לֹֽא־יָמִ֞ישׁ עַמּ֤וּד הֶֽעָנָן֙ יוֹמָ֔ם וְעַמּ֥וּד הָאֵ֖שׁ לָ֑יְלָה לִפְנֵ֖י הָעָֽם׃ \petucha 
\rashi{\rashiDH{לא ימיש. }הקב״ה את עמוד הענן יומם ועמוד האש לילה, מגיד שעמוד הענן משלים לעמוד האש ועמוד האש משלים לעמוד הענן, שעד שלא ישקע זה עולה זה (שבת כג׃)׃ }}
{לָא עָדֵי עַמּוּדָא דַּעֲנָנָא בִּימָמָא וְאַף לָא עַמּוּדָא דְּאִישָׁתָא בְּלֵילְיָא מִן קֳדָם עַמָּא׃}
{the pillar of cloud by day, and the pillar of fire by night, departed not from before the people.}{\arabic{verse}}
\newperek
\threeverse{\aliya{לוי}}%Ex.14:1
{וַיְדַבֵּ֥ר יְהֹוָ֖ה אֶל־מֹשֶׁ֥ה לֵּאמֹֽר׃}
{וּמַלֵּיל יְיָ עִם מֹשֶׁה לְמֵימַר׃}
{And the \lord\space spoke unto Moses, saying:}{\Roman{chap}}
\threeverse{\arabic{verse}}%Ex.14:2
{דַּבֵּר֮ אֶל־בְּנֵ֣י יִשְׂרָאֵל֒ וְיָשֻׁ֗בוּ וְיַחֲנוּ֙ לִפְנֵי֙ פִּ֣י הַחִירֹ֔ת בֵּ֥ין מִגְדֹּ֖ל וּבֵ֣ין הַיָּ֑ם לִפְנֵי֙ בַּ֣עַל צְפֹ֔ן נִכְח֥וֹ תַחֲנ֖וּ עַל־הַיָּֽם׃
\rashi{\rashiDH{וישבו. }לאחוריהם, לצד מצרים היו מקרבין כל יום השלישי, כדי להטעות את פרעה, שיאמר תועים הם בדרך, כמו שנאמר ואמר פרעה לבני ישראל וגו׳׃ }\rashi{\rashiDH{ויחנו לפני פי החירות. }הוא פיתום, ועכשיו נקרא פי החירות, על שם שנעשו שם בני חורין (מכילתא בשלח פ״א), והם שני סלעים גבוהים זקופים, והגיא שביניהם קרוי פי הסלעים׃ }\rashi{\rashiDH{לפני בעל צפון. }הוא נשאר מכל אלהי מצרים, כדי להטעותן, שיאמרו קשה יראתן (שם), ועליו פירש איוב מַשְׂגִּיא לַגֹּויִם וַיְאַבְּדֵם (איוב יב, כג)׃ }}
{מַלֵּיל עִם בְּנֵי יִשְׂרָאֵל וִיתוּבוּן וְיִשְׁרוֹן קֳדָם פּוֹם חִירָתָא בֵּין מִגְדּוֹל וּבֵין יַמָּא קֳדָם בְּעֵיל צְפוֹן לְקִבְלֵיהּ תִּשְׁרוֹן עַל יַמָּא׃}
{’Speak unto the children of Israel, that they turn back and encamp before Pi-hahiroth, between Migdol and the sea, before Baal-zephon, over against it shall ye encamp by the sea.}{\arabic{verse}}
\threeverse{\arabic{verse}}%Ex.14:3
{וְאָמַ֤ר פַּרְעֹה֙ לִבְנֵ֣י יִשְׂרָאֵ֔ל נְבֻכִ֥ים הֵ֖ם בָּאָ֑רֶץ סָגַ֥ר עֲלֵיהֶ֖ם הַמִּדְבָּֽר׃
\rashi{\rashiDH{ואמר פרעה. }כשישמע שהם שָׁבִים לאחוריהם׃}\rashi{\rashiDH{לבני ישראל. }על בני ישראל. וכן ה׳ יִלָּחֵם לָכֶם, עליכם. אִמְרִי לִי אָחִי הוּא (בראשית כ, יג), אמרי עלי׃ }\rashi{\rashiDH{נבכים הם. }כלואים ומשוקעים, ובלע״ז שיר״יר כמו בִּעֵמֶק הַבָּכָא (תהלים פד, ז), מִבְּכִי נְהָרֹות (איוב כח, יא), נִבְכֵי יָם (שם לח, טז). נבכים הם, כלואים הם במדבר, שאינן יודעין לצאת ממנו ולהיכן ילכו׃ }}
{וְיֵימַר פַּרְעֹה עַל בְּנֵי יִשְׂרָאֵל מְעוּרְבְּלִין אִנּוּן בְּאַרְעָא אֲחַד עֲלֵיהוֹן מַדְבְּרָא׃}
{And Pharaoh will say of the children of Israel: They are entangled in the land, the wilderness hath shut them in.}{\arabic{verse}}
\threeverse{\arabic{verse}}%Ex.14:4
{וְחִזַּקְתִּ֣י אֶת־לֵב־פַּרְעֹה֮ וְרָדַ֣ף אַחֲרֵיהֶם֒ וְאִכָּבְדָ֤ה בְּפַרְעֹה֙ וּבְכׇל־חֵיל֔וֹ וְיָדְע֥וּ מִצְרַ֖יִם כִּֽי־אֲנִ֣י יְהֹוָ֑ה וַיַּֽעֲשׂוּ־כֵֽן׃
\rashi{\rashiDH{ואכבדה בפרעה. }כשהקב״ה מתנקם ברשעים שמו מתגדל ומתכבד, וכן הוא אומר וְנִשְׁפַּטְתִּי אִתֹּו וגו׳, ואחר כך וְהִתְגַדִּלְתִּי וְהִתְקַדִּשְׁתִּי וִנֹודַעְתִּי וגו׳ (יחזקאל לח, כבכג), ואומר שָׁמָּה שִׁבַּר רִשְׁפֵי קָשֶׁת, ואחר כך נֹודָע בִּיהוּדָה אֱלֹהִים (תהלים עו, דב), ואומר נֹודַע ה׳ מִשְׁפָּט עָשָׂה (שם ט, יז)׃ }\rashi{\rashiDH{בפרעה ובכל חילו. }הוא התחיל בעבירה וממנו התחילה הפורענות (מכילתא פ״א)׃ 
}\rashi{\rashiDH{ויעשו כן. }להגיד שבחן ששמעו לקול משה, ולא אמרו היאך נתקרב אל רודפינו, אנו צריכים לברוח, אלא אמרו אין לנו אלא דברי בן עמרם (שם)׃ }}
{וַאֲתַקֵּיף יָת לִבָּא דְּפַרְעֹה וְיִרְדּוֹף בָּתְרֵיהוֹן וְאֶתְיַקַּר בְּפַרְעֹה וּבְכָל מַשְׁרְיָתֵיהּ וְיִדְּעוּן מִצְרָאֵי אֲרֵי אֲנָא יְיָ וַעֲבַדוּ כֵן׃}
{And I will harden Pharaoh’s heart, and he shall follow after them; and I will get Me honour upon Pharaoh, and upon all his host; and the Egyptians shall know that I am the \lord.’ And they did so.}{\arabic{verse}}
\threeverse{\aliya{ישראל}}%Ex.14:5
{וַיֻּגַּד֙ לְמֶ֣לֶךְ מִצְרַ֔יִם כִּ֥י בָרַ֖ח הָעָ֑ם וַ֠יֵּהָפֵ֠ךְ לְבַ֨ב פַּרְעֹ֤ה וַעֲבָדָיו֙ אֶל־הָעָ֔ם וַיֹּֽאמְרוּ֙ מַה־זֹּ֣את עָשִׂ֔ינוּ כִּֽי־שִׁלַּ֥חְנוּ אֶת־יִשְׂרָאֵ֖ל מֵעָבְדֵֽנוּ׃
\rashi{\rashiDH{ויוגד למלך מצרים. }אִיקְטוּרִיןשלח עמהם (מכילתא פ״א), וכיון שהגיעו לשלשת ימים שקבעו לילך ולשוב, וראו שאינן חוזרין למצרים, באו והגידו לפרעה ביום הרביעי, ובחמישי ובששי רדפו אחריהם, וליל שביעי ירדו לים, בשחרית אמרו שירה, והוא יום שביעי של פסח, לכך אנו קורין השירה ביום השביעי׃ }\rashi{\rashiDH{ויהפך. }נהפך ממה שהיה, שהרי אמר להם קוּמוּ צְּאוּ מִתֹּוךְ עַמִּי (שמות יב, לא), ונהפך לבב עבדיו, שהרי לשעבר היו אומרים לו עַד מָתַי יִהְיֶה זֶה לָנוּ לְמֹוקֵשׁ, ועכשיו נהפכו לרדוף אחריהם בשביל ממונם שהשאילום׃ }\rashi{\rashiDH{מעבדנו. }מעבוד אותנו׃}}
{וְאִתְחַוַּא לְמַלְכָּא דְּמִצְרַיִם אֲרֵי אֲזַל עַמָּא וְאִתְהֲפֵיךְ לִבָּא דְּפַרְעֹה וְעַבְדּוֹהִי בְּעַמָּא וַאֲמַרוּ מָא דָּא עֲבַדְנָא אֲרֵי שַׁלַּחְנָא יָת יִשְׂרָאֵל מִפּוּלְחָנַנָא׃}
{And it was told the king of Egypt that the people were fled; and the heart of Pharaoh and of his servants was turned towards the people, and they said: ‘What is this we have done, that we have let Israel go from serving us?}{\arabic{verse}}
\threeverse{\arabic{verse}}%Ex.14:6
{וַיֶּאְסֹ֖ר אֶת־רִכְבּ֑וֹ וְאֶת־עַמּ֖וֹ לָקַ֥ח עִמּֽוֹ׃
\rashi{\rashiDH{ויאסר את רכבו. }הוא בעצמו (מכילתא פ״א)׃ }\rashi{\rashiDH{ואת עמו לקח עמו. }מְשָׁכָם בדברים, לקינו ונטלו ממוננו ושלחנום, בואו עמי, ואני לא אתנהג עמכם כשאר מלכים, דרך שאר מלכים עבדיו קודמין לו במלחמה, ואני אקדים לפניכם, שנאמר וּפַרְעֹה הִקְרִיב, הקריב עצמו מיהר לפני חיילותיו, דרך שאר מלכים ליטול ביזה בראש כמו שיבחר, אני אשוה עמכם בחלק, שנאמר אֲחַלֵּק שָׁלָל׃ }}
{וְטַקֵּיס יָת רְתִכֵּיהּ וְיָת עַמֵּיהּ דְּבַר עִמֵּיהּ׃}
{And he made ready his chariots, and took his people with him.}{\arabic{verse}}
\threeverse{\arabic{verse}}%Ex.14:7
{וַיִּקַּ֗ח שֵׁשׁ־מֵא֥וֹת רֶ֙כֶב֙ בָּח֔וּר וְכֹ֖ל רֶ֣כֶב מִצְרָ֑יִם וְשָׁלִשִׁ֖ם עַל־כֻּלּֽוֹ׃
\rashi{\rashiDH{בחור. }נבחרים, בחור לשון יחיד, כל רכב ורכב שבמנין זה היה בחור׃ }\rashi{\rashiDH{וכל רכב מצרים. }ועמהם כל שאר הרכב, ומהיכן היו הבהמות הללו, אם תאמר מִשֶׁל מצרים, הרי נאמר וַיָּמָת כֹל מִקְנֵה מִצְרָיִם (שמות ט, י), ואם תאמר מִשֶׁל ישראל, והלא נאמר וְגַם מִקְנֵנוּ יֵלֵךְ עִמָּנוּ (שם י, כו), מִשֶׁל מי היו, מהירא את דבר ה׳, מכאן היה רבי שמעון אומר, כשר שבמצרים הרוג, טוב שבנחשים רצוץ את מוחו (מכילתא פ״א)׃ }\rashi{\rashiDH{ושלשים על כלו. }שרי צבאות כתרגומו׃ 
}}
{וּדְבַר שֵׁית מְאָה רְתִכִּין בְּחִירָן וְכֹל רְתִכֵּי מִצְרָאֵי וְגִבָּרִין מְמֻנַּן עַל כּוּלְּהוֹן׃}
{And he took six hundred chosen chariots, and all the chariots of Egypt, and captains over all of them.}{\arabic{verse}}
\threeverse{\arabic{verse}}%Ex.14:8
{וַיְחַזֵּ֣ק יְהֹוָ֗ה אֶת־לֵ֤ב פַּרְעֹה֙ מֶ֣לֶךְ מִצְרַ֔יִם וַיִּרְדֹּ֕ף אַחֲרֵ֖י בְּנֵ֣י יִשְׂרָאֵ֑ל וּבְנֵ֣י יִשְׂרָאֵ֔ל יֹצְאִ֖ים בְּיָ֥ד רָמָֽה׃
\rashi{\rashiDH{ויחזק ה׳ את לב פרעה. }שהיה תולה אם לרדוף אם לאו, וחזק את לבו לרדוף׃ }\rashi{\rashiDH{ביד רמה. }בגבורה גבוהה ומפורסמת (מכילתא פ״א)׃ }}
{וְתַקֵּיף יְיָ יָת לִבָּא דְּפַרְעֹה מַלְכָּא דְּמִצְרַיִם וּרְדַף בָּתַר בְּנֵי יִשְׂרָאֵל וּבְנֵי יִשְׂרָאֵל נָפְקִין בְּרֵישׁ גְּלֵי׃}
{And the \lord\space hardened the heart of Pharaoh king of Egypt, and he pursued after the children of Israel; for the children of Israel went out with a high hand.}{\arabic{verse}}
\threeverse{\aliya{שני}}%Ex.14:9
{וַיִּרְדְּפ֨וּ מִצְרַ֜יִם אַחֲרֵיהֶ֗ם וַיַּשִּׂ֤יגוּ אוֹתָם֙ חֹנִ֣ים עַל־הַיָּ֔ם כׇּל־סוּס֙ רֶ֣כֶב פַּרְעֹ֔ה וּפָרָשָׁ֖יו וְחֵיל֑וֹ עַל־פִּי֙ הַֽחִירֹ֔ת לִפְנֵ֖י בַּ֥עַל צְפֹֽן׃}
{וּרְדַפוּ מִצְרָאֵי בָּתְרֵיהוֹן וְאַדְבִּיקוּ יָתְהוֹן כַּד שְׁרַן עַל יַמָּא כָּל סוּסָוָת רְתִכֵּי פַרְעֹה וּפָרָשׁוֹהִי וּמַשְׁרְיָתֵיהּ עַל פֹּם חִירָתָא דִּקְדָם בְּעֵיל צְפוֹן׃}
{And the Egyptians pursued after them, all the horses and chariots of Pharaoh, and his horsemen, and his army, and overtook them encamping by the sea, beside Pi-hahiroth, in front of Baal-zephon.}{\arabic{verse}}
\threeverse{\arabic{verse}}%Ex.14:10
{וּפַרְעֹ֖ה הִקְרִ֑יב וַיִּשְׂאוּ֩ בְנֵֽי־יִשְׂרָאֵ֨ל אֶת־עֵינֵיהֶ֜ם וְהִנֵּ֥ה מִצְרַ֣יִם \legarmeh  נֹסֵ֣עַ אַחֲרֵיהֶ֗ם וַיִּֽירְאוּ֙ מְאֹ֔ד וַיִּצְעֲק֥וּ בְנֵֽי־יִשְׂרָאֵ֖ל אֶל־יְהֹוָֽה׃
\rashi{\rashiDH{ופרעה הקריב. }היה לו לכתוב ופרעה קרב, מהו הקריב, הקריב עצמו ונתאמץ לקדם לפניהם, כמו שהתנה עמהם׃ }\rashi{\rashiDH{נסע אחריהם. }בלב אחד כאיש אחד. דבר אחר והנה מצרים נוסע אחריהם, ראו שר של מצרים נוסע מן השמים לעזור למצרים (תנחומא בשלח יג)׃ }\rashi{\rashiDH{ויצעקו. }תפשו אומנות אבותם (מכילתא פ״ב). באברהם הוא אומר, אֶל הַמָּקֹום אֲשֶׁר עָמַד שָׁם (בראשית יט, כז). ביצחק, לָשׂוּחַ בַּשָׂדֶה (שם כד, סג). ביעקב, וַיִּפְגַּע בַּמָּקֹום (שם כח, יא)׃ }}
{וּפַרְעֹה קְרֵיב וּזְקַפוּ בְּנֵי יִשְׂרָאֵל יָת עֵינֵיהוֹן וְהָא מִצְרָאֵי נָטְלִין בָּתְרֵיהוֹן וּדְחִילוּ לַחְדָּא וּזְעִיקוּ בְנֵי יִשְׂרָאֵל קֳדָם יְיָ׃}
{And when Pharaoh drew nigh, the children of Israel lifted up their eyes, and, behold, the Egyptians were marching after them; and they were sore afraid; and the children of Israel cried out unto the \lord.}{\arabic{verse}}
\threeverse{\arabic{verse}}%Ex.14:11
{וַיֹּאמְרוּ֮ אֶל־מֹשֶׁה֒ הֲֽמִבְּלִ֤י אֵין־קְבָרִים֙ בְּמִצְרַ֔יִם לְקַחְתָּ֖נוּ לָמ֣וּת בַּמִּדְבָּ֑ר מַה־זֹּאת֙ עָשִׂ֣יתָ לָּ֔נוּ לְהוֹצִיאָ֖נוּ מִמִּצְרָֽיִם׃
\rashi{\rashiDH{המבלי אין קברים. }וכי מחמת חסרון קברים, שאין קברים במצרים ליקבר שם, לקחתנו משם. שיפו״ר פלינצס״א דינו״ן פושי״ש׃ 
}}
{וַאֲמַרוּ לְמֹשֶׁה הֲמִדְּלֵית קַבְרִין בְּמִצְרַיִם דְּבַרְתַּנָא לִמְמָת בְּמַדְבְּרָא מָא דָא עֲבַדְתְּ לַנָא לְאַפָּקוּתַנָא מִמִּצְרָיִם׃}
{And they said unto Moses: ‘Because there were no graves in Egypt, hast thou taken us away to die in the wilderness? wherefore hast thou dealt thus with us, to bring us forth out of Egypt?}{\arabic{verse}}
\threeverse{\arabic{verse}}%Ex.14:12
{הֲלֹא־זֶ֣ה הַדָּבָ֗ר אֲשֶׁר֩ דִּבַּ֨רְנוּ אֵלֶ֤יךָ בְמִצְרַ֙יִם֙ לֵאמֹ֔ר חֲדַ֥ל מִמֶּ֖נּוּ וְנַֽעַבְדָ֣ה אֶת־מִצְרָ֑יִם כִּ֣י ט֥וֹב לָ֙נוּ֙ עֲבֹ֣ד אֶת־מִצְרַ֔יִם מִמֻּתֵ֖נוּ בַּמִּדְבָּֽר׃
\rashi{\rashiDH{אשר דברנו אליך במצרים. }והיכן דברו, יֵרֶא ה׳ עֲלֵיכֶם וְיִשְׁפֹּוט (שמות ה, כא  מכילתא פ״ב)׃ }\rashi{\rashiDH{ממותנו. }מאשר נמות, ואם היה נקוד מלאפו״ם, (ר״ל חול״ם, כנודע לבעלי דקדוק שקראו חול״ם מלאפו״ם, ועיין לקמן פרשת יתרו ברש״י פסוק פן יפרוץ) היה נבאר ממיתתנו, עכשיו שנקוד בשורק, נבאר מאשר נמות. וכן מִי יִתֵּן מוּתֵנוּ, שנמות. וכן מִי יִתֵּן מוּתִי (שמואל־ב יט, יט) דאבשלום, שאמות. כמו לְיֹום קוּמִי לָעַד (צפניה ג, ח), עַד שׁוּבִי בְשָׁלֹום (דברי הימים־ב יח, כו) שאקום שאשוב׃ }}
{הֲלָא דֵּין פִּתְגָמָא דְּמַלֵּילְנָא עִמָּךְ בְּמִצְרַיִם לְמֵימַר שְׁבוֹק מִנַּנָא וְנִפְלַח יָת מִצְרָאֵי אֲרֵי טָב לַנָא דְּנִפְלַח יָת מִצְרָאֵי מִדִּנְמוּת בְּמַדְבְּרָא׃}
{Is not this the word that we spoke unto thee in Egypt, saying: Let us alone, that we may serve the Egyptians? For it were better for us to serve the Egyptians, than that we should die in the wilderness.’}{\arabic{verse}}
\threeverse{\arabic{verse}}%Ex.14:13
{וַיֹּ֨אמֶר מֹשֶׁ֣ה אֶל־הָעָם֮ אַל־תִּירָ֒אוּ֒ הִֽתְיַצְּב֗וּ וּרְאוּ֙ אֶת־יְשׁוּעַ֣ת יְהֹוָ֔ה אֲשֶׁר־יַעֲשֶׂ֥ה לָכֶ֖ם הַיּ֑וֹם כִּ֗י אֲשֶׁ֨ר רְאִיתֶ֤ם אֶת־מִצְרַ֙יִם֙ הַיּ֔וֹם לֹ֥א תֹסִ֛פוּ לִרְאֹתָ֥ם ע֖וֹד עַד־עוֹלָֽם׃
\rashi{\rashiDH{כי אשר ראיתם את מצרים וגו׳. }מה שראיתם אותם אינו אלא היום, היום הוא שראיתם אותם ולא תוסיפו עוד׃ 
}}
{וַאֲמַר מֹשֶׁה לְעַמָּא לָא תִדְחֲלוּן אִתְעַתַּדוּ וַחֲזוֹ יָת פּוּרְקָנָא דַּייָ דְּיַעֲבֵיד לְכוֹן יוֹמָא דֵין אֲרֵי דַּחֲזֵיתוֹן יָת מִצְרָאֵי יוֹמָא דֵין לָא תֵיסְפוּן לְמִחְזֵיהוֹן עוֹד עַד עָלְמָא׃}
{And Moses said unto the people: ‘Fear ye not, stand still, and see the salvation of the \lord, which He will work for you to-day; for whereas ye have seen the Egyptians to-day, ye shall see them again no more for ever.}{\arabic{verse}}
\threeverse{\arabic{verse}}%Ex.14:14
{יְהֹוָ֖ה יִלָּחֵ֣ם לָכֶ֑ם וְאַתֶּ֖ם תַּחֲרִשֽׁוּן׃ \petucha 
\rashi{\rashiDH{ה׳ ילחם לכם. }בשבילכם, וכן כִּי ה׳ נִלְחָם לָהֶם, וכן אִם לָאֵל תְּרִיבוּן (איוב יג, ח), וכן וַאֲשֶׁר דִּבֶּר לִי (בראשית כד, ז), וכן ה ַאַתֶּם תְּרִיבוּן לַבַּעַל (שופטים ו, לא)׃ }}
{יְיָ יְגִיחַ לְכוֹן קְרָב וְאַתּוּן תִּשְׁתְּקוּן׃}
{The \lord\space will fight for you, and ye shall hold your peace.’}{\arabic{verse}}
\threeverse{\aliya{שלישי}}%Ex.14:15
{וַיֹּ֤אמֶר יְהֹוָה֙ אֶל־מֹשֶׁ֔ה מַה־תִּצְעַ֖ק אֵלָ֑י דַּבֵּ֥ר אֶל־בְּנֵי־יִשְׂרָאֵ֖ל וְיִסָּֽעוּ׃
\rashi{\rashiDH{מה תצעק אלי. }למדנו, שהיה משה עומד ומתפלל, אמר לו הקב״ה, לא עת עתה להאריך בתפלה, שישראל נתונין בצרה. דבר אחר מה תצעק אלי, עלי הדבר תלוי ולא עליך, כמ״ש להלן, עַל בָּנַי וְעַל פֹּעַל יָדַי תְּצַוֻּנִי (ישעיה מה, יא)׃ }\rashi{\rashiDH{דבר אל בני ישראל ויסעו. }אין להם אלא ליסע, שאין הים עומד בפניהם, כדאי זכות אבותיהם, והם, והאמונה שהאמינו בי ויצאו, לקרוע להם הים (מכילתא פ״ג)׃ }}
{וַאֲמַר יְיָ לְמֹשֶׁה קַבֵּילִית צְלוֹתָךְ מַלֵּיל עִם בְּנֵי יִשְׂרָאֵל וְיִטְּלוּן׃}
{And the \lord\space said unto Moses: ‘Wherefore criest thou unto Me? speak unto the children of Israel, that they go forward.}{\arabic{verse}}
\threeverse{\arabic{verse}}%Ex.14:16
{וְאַתָּ֞ה הָרֵ֣ם אֶֽת־מַטְּךָ֗ וּנְטֵ֧ה אֶת־יָדְךָ֛ עַל־הַיָּ֖ם וּבְקָעֵ֑הוּ וְיָבֹ֧אוּ בְנֵֽי־יִשְׂרָאֵ֛ל בְּת֥וֹךְ הַיָּ֖ם בַּיַּבָּשָֽׁה׃}
{וְאַתְּ טוֹל יָת חוּטְרָךְ וַאֲרֵים יָת יְדָךְ עַל יַמָּא וּבַזַּעְהִי וְיֵיעֲלוּן בְּנֵי יִשְׂרָאֵל בְּגוֹ יַמָּא בְּיַבֶּשְׁתָּא׃}
{And lift thou up thy rod, and stretch out thy hand over the sea, and divide it; and the children of Israel shall go into the midst of the sea on dry ground.}{\arabic{verse}}
\threeverse{\arabic{verse}}%Ex.14:17
{וַאֲנִ֗י הִנְנִ֤י מְחַזֵּק֙ אֶת־לֵ֣ב מִצְרַ֔יִם וְיָבֹ֖אוּ אַחֲרֵיהֶ֑ם וְאִכָּבְדָ֤ה בְּפַרְעֹה֙ וּבְכׇל־חֵיל֔וֹ בְּרִכְבּ֖וֹ וּבְפָרָשָֽׁיו׃}
{וַאֲנָא הָאֲנָא מְתַקֵּיף יָת לִבָּא דְּמִצְרָאֵי וְיֵיעֲלוּן בָּתְרֵיהוֹן וְאֶתְיַקַּר בְּפַרְעֹה וּבְכָל מַשְׁרְיָתֵיהּ בִּרְתִכּוֹהִי וּבְפָרָשׁוֹהִי׃}
{And I, behold, I will harden the hearts of the Egyptians, and they shall go in after them; and I will get Me honour upon Pharaoh, and upon all his host, upon his chariots, and upon his horsemen.}{\arabic{verse}}
\threeverse{\arabic{verse}}%Ex.14:18
{וְיָדְע֥וּ מִצְרַ֖יִם כִּי־אֲנִ֣י יְהֹוָ֑ה בְּהִכָּבְדִ֣י בְּפַרְעֹ֔ה בְּרִכְבּ֖וֹ וּבְפָרָשָֽׁיו׃}
{וְיִדְּעוּן מִצְרָאֵי אֲרֵי אֲנָא יְיָ בְּאִתְיַקָּרוּתִי בְּפַרְעֹה בִּרְתִכּוֹהִי וּבְפָרָשׁוֹהִי׃}
{And the Egyptians shall know that I am the \lord, when I have gotten Me honour upon Pharaoh, upon his chariots, and upon his horsemen.’}{\arabic{verse}}
\threeverse{\arabic{verse}}%Ex.14:19
{וַיִּסַּ֞ע מַלְאַ֣ךְ הָאֱלֹהִ֗ים הַהֹלֵךְ֙ לִפְנֵי֙ מַחֲנֵ֣ה יִשְׂרָאֵ֔ל וַיֵּ֖לֶךְ מֵאַחֲרֵיהֶ֑ם וַיִּסַּ֞ע עַמּ֤וּד הֶֽעָנָן֙ מִפְּנֵיהֶ֔ם וַיַּֽעֲמֹ֖ד מֵאַחֲרֵיהֶֽם׃
\rashi{\rashiDH{וילך מאחריהם. }להבדיל בין מחנה מצרים ובין מחנה ישראל, ולקבל חצים וּבְלִיסְטְרָאוֹת של מצרים. בכל מקום הוא אומר מלאך ה׳, וכאן מלאך האלהים, אין אלהים בכל מקום אלא דין, מלמד שהיו ישראל נתונין בדין באותה שעה, אם להנצל אם להאבד עם מצרים׃ }\rashi{\rashiDH{ויסע עמוד הענן. }כשחשיכה, והשלים עמוד הענן את המחנה לעמוד האש, לא נסתלק הענן כמו שהיה רגיל להסתלק ערבית לגמרי, אלא נסע והלך לו מאחריהם, להחשיך למצרים׃ }}
{וּנְטַל מַלְאֲכָא דַּייָ דִּמְדַבַּר קֳדָם מַשְׁרִיתָא דְּיִשְׂרָאֵל וַאֲתָא מִבָּתְרֵיהוֹן וּנְטַל עַמּוּדָא דַּעֲנָנָא מִן קֳדָמֵיהוֹן וּשְׁרָא מִבָּתְרֵיהוֹן׃}
{And the angel of God, who went before the camp of Israel, removed and went behind them; and the pillar of cloud removed from before them, and stood behind them;}{\arabic{verse}}
\threeverse{\arabic{verse}}%Ex.14:20
{וַיָּבֹ֞א בֵּ֣ין \legarmeh  מַחֲנֵ֣ה מִצְרַ֗יִם וּבֵין֙ מַחֲנֵ֣ה יִשְׂרָאֵ֔ל וַיְהִ֤י הֶֽעָנָן֙ וְהַחֹ֔שֶׁךְ וַיָּ֖אֶר אֶת־הַלָּ֑יְלָה וְלֹא־קָרַ֥ב זֶ֛ה אֶל־זֶ֖ה כׇּל־הַלָּֽיְלָה׃
\rashi{\rashiDH{ויבא בין מחנה מצרים. }משל למהלך בדרך ובנו מהלך לפניו, באו לסטים לשבותו, נטלו מלפניו ונתנו לאחריו, בא זאב מאחריו, נתנו לפניו, באו לסטים לפניו וזאבים מאחריו, נתנו על זרועו ונלחם בהם. כך וְאָנֹכִי תִּרְגַּלְתִּי לְאֶפְרַיִם קָחָם עַל זְרֹועֹתָיו (הושע יא, ג)׃ }\rashi{\rashiDH{ויהי הענן והחשך. }למצרים׃}\rashi{\rashiDH{ויאר. }עמוד האש את הלילה לישראל, והלך לפניהם כדרכו ללכת כל הלילה, והחשך של ערפל לצד מצרים׃ }\rashi{\rashiDH{ולא קרב זה אל זה. }מחנה אל מחנה (מכילתא פ״ד)׃ }}
{וְעָאל בֵּין מַשְׁרִיתָא דְּמִצְרָאֵי וּבֵין מַשְׁרִיתָא דְּיִשְׂרָאֵל וַהֲוָה עֲנָנָא וְקַבְלָא לְמִצְרָאֵי וּלְיִשְׂרָאֵל נָהַר כָּל לֵילְיָא וְלָא אִתְקָרַבוּ דֵין לְוָת דֵּין כָּל לֵילְיָא׃}
{and it came between the camp of Egypt and the camp of Israel; and there was the cloud and the darkness here, yet gave it light by night there; and the one came not near the other all the night.}{\arabic{verse}}
\threeverse{\arabic{verse}}%Ex.14:21
{וַיֵּ֨ט מֹשֶׁ֣ה אֶת־יָדוֹ֮ עַל־הַיָּם֒ וַיּ֣וֹלֶךְ יְהֹוָ֣ה \pasek  אֶת־הַ֠יָּ֠ם בְּר֨וּחַ קָדִ֤ים עַזָּה֙ כׇּל־הַלַּ֔יְלָה וַיָּ֥שֶׂם אֶת־הַיָּ֖ם לֶחָרָבָ֑ה וַיִּבָּקְע֖וּ הַמָּֽיִם׃
\rashi{\rashiDH{ברוח קדים עזה. }ברוח קדים שהיא עזה שברוחות, הוא הרוח שהקב״ה נפרע בה מן הרשעים, שנאמר כְּרוּחַ קָדִים אֲפִיצֵם (ירמי׳ יח, יז), יָבֹא קָדִים רוּחַ ה׳ (הושע יג, טו), רוּחַ הַקָּדִים שְׁבָרֵךְ בְּלְב יַמִּים (יחזקאל כז, כו), הָגָה בְּרוּחֹו הַקָּשָׁה בְּיֹום קָדִים (ישעיה כז, ח)׃ }\rashi{\rashiDH{ויבקעו המים. }כל מים שבעולם (מכילתא פ״ד)׃ }}
{וַאֲרֵים מֹשֶׁה יָת יְדֵיהּ עַל יַמָּא וְדַבַּר יְיָ יָת יַמָּא בְּרוּחַ קִדּוּמָא תַּקִּיף כָּל לֵילְיָא וְשַׁוִּי יָת יַמָּא לְיַבֶּשְׁתָּא וְאִתְבְּזַעוּ מַיָּא׃}
{And Moses stretched out his hand over the sea; and the \lord\space caused the sea to go back by a strong east wind all the night, and made the sea dry land, and the waters were divided.}{\arabic{verse}}
\threeverse{\arabic{verse}}%Ex.14:22
{וַיָּבֹ֧אוּ בְנֵֽי־יִשְׂרָאֵ֛ל בְּת֥וֹךְ הַיָּ֖ם בַּיַּבָּשָׁ֑ה וְהַמַּ֤יִם לָהֶם֙ חוֹמָ֔ה מִֽימִינָ֖ם וּמִשְּׂמֹאלָֽם׃}
{וְעָאלוּ בְנֵי יִשְׂרָאֵל בְּגוֹ יַמָּא בְּיַבֶּשְׁתָּא וּמַיָּא לְהוֹן שׁוּרִין מִיַּמִּינְהוֹן וּמִשְּׂמָאלְהוֹן׃}
{And the children of Israel went into the midst of the sea upon the dry ground; and the waters were a wall unto them on their right hand, and on their left.}{\arabic{verse}}
\threeverse{\arabic{verse}}%Ex.14:23
{וַיִּרְדְּפ֤וּ מִצְרַ֙יִם֙ וַיָּבֹ֣אוּ אַחֲרֵיהֶ֔ם כֹּ֚ל ס֣וּס פַּרְעֹ֔ה רִכְבּ֖וֹ וּפָרָשָׁ֑יו אֶל־תּ֖וֹךְ הַיָּֽם׃
\rashi{\rashiDH{כל סוס פרעה. }וכי סוס אחד היה, אלא מגיד שאין כולם חשובין לפני המקום אלא כסוס אחד׃ }}
{וּרְדַפוּ מִצְרָאֵי וְעָאלוּ בָתְרֵיהוֹן כֹּל סוּסָוָת פַּרְעֹה רְתִכּוֹהִי וּפָרָשׁוֹהִי לְגוֹ יַמָּא׃}
{And the Egyptians pursued, and went in after them into the midst of the sea, all Pharaoh’s horses, his chariots, and his horsemen.}{\arabic{verse}}
\threeverse{\arabic{verse}}%Ex.14:24
{וַֽיְהִי֙ בְּאַשְׁמֹ֣רֶת הַבֹּ֔קֶר וַיַּשְׁקֵ֤ף יְהֹוָה֙ אֶל־מַחֲנֵ֣ה מִצְרַ֔יִם בְּעַמּ֥וּד אֵ֖שׁ וְעָנָ֑ן וַיָּ֕הׇם אֵ֖ת מַחֲנֵ֥ה מִצְרָֽיִם׃
\rashi{\rashiDH{באשמרת הבוקר. }שלשת חלקי הלילה קרוין אשמורת, ואותה שלפני הבקר קורא אשמורת הבוקר (ברכות ג.). ואומר אני, לפי שהלילה חלוק למשמרות שיר של מלאכי השרת, כת אחר כת לשלשה חלקים, לכך קרוי אשמורת, וזהו שתרגם אונקלוס מַטְרַת׃ }\rashi{\rashiDH{וישקף. }ויבט, כלומר פנה אליהם להשחיתם. ותרגומו וְאִסְתְּכֵי, אף הוא לשון הבטה, כמו שְׂדֵה צֹפִים (במדבר כג, יד), לַחֲקַל סְכוּתָה׃ }\rashi{\rashiDH{בעמוד אש וענן. }עמוד ענן יורד ועושה אותו כטיט, ועמוד אש מרתיחו, וטלפי סוסיהם משתמטות (מכילתא פ״ה)׃ }\rashi{\rashiDH{ויהם. }לשון מהומה, אשטורד״יטון בלע״זערבבם, נטל סִגְנָיוֹת שלהם. ושנינו בפרקי ר׳ אליעזר בנו של ר׳ יוסי הגלילי, כל מקום שנאמר בו מהומה, הרעשת קול הוא, וזה אב לכלן, וַיַּרְעֵם ה׳ בְּקֹול גָּדֹול וגו׳ עַל פְּלִשְׁתִּים וַיְהֻמֵּם (שמואל־א ז, י)׃ 
}}
{וַהֲוָה בְּמַטְּרַת צַפְרָא וְאִסְתַּכִי יְיָ לְמַשְׁרִיתָא דְּמִצְרָאֵי בְּעַמּוּדָא דְּאִישָׁתָא וַעֲנָנָא וְשַׁגֵּישׁ יָת מַשְׁרִיתָא דְּמִצְרָאֵי׃}
{And it came to pass in the morning watch, that the \lord\space looked forth upon the host of the Egyptians through the pillar of fire and of cloud, and discomfited the host of the Egyptians.}{\arabic{verse}}
\threeverse{\arabic{verse}}%Ex.14:25
{וַיָּ֗סַר אֵ֚ת אֹפַ֣ן מַרְכְּבֹתָ֔יו וַֽיְנַהֲגֵ֖הוּ בִּכְבֵדֻ֑ת וַיֹּ֣אמֶר מִצְרַ֗יִם אָנ֙וּסָה֙ מִפְּנֵ֣י יִשְׂרָאֵ֔ל כִּ֣י יְהֹוָ֔ה נִלְחָ֥ם לָהֶ֖ם בְּמִצְרָֽיִם׃ \petucha 
\rashi{\rashiDH{ויסר את אופן מרכבותיו. }מכח האש נשרפו הגלגלים, והמרכבות נגררות, והיושבים בהם נעים ואבריהן מתפרקין׃ }\rashi{\rashiDH{וינהגהו בכבדות. }בהנהגה שהיא כבדה וקשה להם, במדה שמדדו וַיַּכְבֵּד לִבֹּו הוּא וַעֲבָדָיו (שמות ט, לד), אף כאן וינהגהו בכבדות׃ }\rashi{\rashiDH{נלחם להם במצרים. }במצריים. דבר אחר במצרים, בארץ מצרים, שכשם שאלו לוקים על הים, כך לוקים אותם שנשארו במצרים׃ }}
{וְאַעְדִּי יָת גִּלְגְּלֵי רְתִכֵּיהוֹן וּמְדַבְּרִין לְהוֹן בִּתְקוֹף וַאֲמַרוּ מִצְרָאֵי נִעְרוֹק מִן קֳדָם יִשְׂרָאֵל אֲרֵי דָא הִיא גְּבוּרְתָא דַּייָ דַּעֲבַד לְהוֹן קְרָבִין בְּמִצְרָיִם׃}
{And He took off their chariot wheels, and made them to drive heavily; so that the Egyptians said: ‘Let us flee from the face of Israel; for the \lord\space fighteth for them against the Egyptians.’}{\arabic{verse}}
\threeverse{\aliya{רביעי}}%Ex.14:26
{וַיֹּ֤אמֶר יְהֹוָה֙ אֶל־מֹשֶׁ֔ה נְטֵ֥ה אֶת־יָדְךָ֖ עַל־הַיָּ֑ם וְיָשֻׁ֤בוּ הַמַּ֙יִם֙ עַל־מִצְרַ֔יִם עַל־רִכְבּ֖וֹ וְעַל־פָּרָשָֽׁיו׃
\rashi{\rashiDH{וישובו המים. }שזקופים ועומדים כחומה, ישובו למקומם ויכסו על מצרים׃ 
}}
{וַאֲמַר יְיָ לְמֹשֶׁה אֲרֵים יָת יְדָךְ עַל יַמָּא וִיתוּבוּן מַיָּא עַל מִצְרָאֵי עַל רְתִכֵּיהוֹן וְעַל פָּרָשֵׁיהוֹן׃}
{And the \lord\space said unto Moses: ‘Stretch out thy hand over the sea, that the waters may come back upon the Egyptians, upon their chariots, and upon their horsemen.’}{\arabic{verse}}
\threeverse{\arabic{verse}}%Ex.14:27
{וַיֵּט֩ מֹשֶׁ֨ה אֶת־יָד֜וֹ עַל־הַיָּ֗ם וַיָּ֨שׇׁב הַיָּ֜ם לִפְנ֥וֹת בֹּ֙קֶר֙ לְאֵ֣יתָנ֔וֹ וּמִצְרַ֖יִם נָסִ֣ים לִקְרָאת֑וֹ וַיְנַעֵ֧ר יְהֹוָ֛ה אֶת־מִצְרַ֖יִם בְּת֥וֹךְ הַיָּֽם׃
\rashi{\rashiDH{לפנות בקר. }לעת שהבוקר פונה לבא׃}\rashi{\rashiDH{לאיתנו. }לתקפו הראשון׃}\rashi{\rashiDH{נסים לקראתו. }שהיו מהוממים ומטורפים ורצין לקראת המים׃}\rashi{\rashiDH{וינער ה׳. }כאדם שמנער את הקדירה והופך העליון למטה והתחתון למעלה, כך היו עולין ויורדין ומשתברין בים, ונתן הקב״ה בהם חיות לקבל היסורין׃ }\rashi{\rashiDH{וינער. }וְשַׁנִּיק, והוא לשון טרוף בלשון ארמי. והרבה יש במדרש אגדה׃ }}
{וַאֲרֵים מֹשֶׁה יָת יְדֵיהּ עַל יַמָּא וְתָב יַמָּא לְעִדָּן צַפְרָא לְתוּקְפֵיהּ וּמִצְרָאֵי עָרְקִין לְקַדָּמוּתֵיהּ וְשַׁנֵּיק יְיָ יָת מִצְרָאֵי בְּגוֹ יַמָּא׃}
{And Moses stretched forth his hand over the sea, and the sea returned to its strength when the morning appeared; and the Egyptians fled against it; and the \lord\space overthrew the Egyptians in the midst of the sea.}{\arabic{verse}}
\threeverse{\arabic{verse}}%Ex.14:28
{וַיָּשֻׁ֣בוּ הַמַּ֗יִם וַיְכַסּ֤וּ אֶת־הָרֶ֙כֶב֙ וְאֶת־הַפָּ֣רָשִׁ֔ים לְכֹל֙ חֵ֣יל פַּרְעֹ֔ה הַבָּאִ֥ים אַחֲרֵיהֶ֖ם בַּיָּ֑ם לֹֽא־נִשְׁאַ֥ר בָּהֶ֖ם עַד־אֶחָֽד׃
\rashi{\rashiDH{ויכסו את הרכב וגו׳ לכל חיל פרעה. }כך דרך המקראות לכתוב למ״ד יתירה, כמו לְכָל כֵּלָיו תַּעֲשֶׂה נְחשֶׁת (שמות כז, ג), וכן לְכֹל כְּלֵי הַמִּשְׁכָּן, בְּכֹל עֲבֹדָתֹו (שם יט), וְיתֵדֹתָם וּמֵיתְרֵיהֶם לְכָל כְּלֵיהֶם, ואינה אלא תקון לשון׃ }}
{וְתָבוּ מַיָּא וַחֲפוֹ יָת רְתִכַּיָּא וְיָת פָּרָשַׁיָּא לְכֹל מַשְׁרְיָת פַּרְעֹה דְּאַלָע בָּתְרֵיהוֹן בְּיַמָּא לָא אִשְׁתְּאַר בְּהוֹן עַד חַד׃}
{And the waters returned, and covered the chariots, and the horsemen, even all the host of Pharaoh that went in after them into the sea; there remained not so much as one of them.}{\arabic{verse}}
\threeverse{\arabic{verse}}%Ex.14:29
{וּבְנֵ֧י יִשְׂרָאֵ֛ל הָלְכ֥וּ בַיַּבָּשָׁ֖ה בְּת֣וֹךְ הַיָּ֑ם וְהַמַּ֤יִם לָהֶם֙ חֹמָ֔ה מִֽימִינָ֖ם וּמִשְּׂמֹאלָֽם׃}
{וּבְנֵי יִשְׂרָאֵל הַלִּיכוּ בְּיַבֶּשְׁתָּא בְּגוֹ יַמָּא וּמַיָּא לְהוֹן שׁוּרִין מִיַּמִּינְהוֹן וּמִסְּמָאלְהוֹן׃}
{But the children of Israel walked upon dry land in the midst of the sea; and the waters were a wall unto them on their right hand, and on their left.}{\arabic{verse}}
\threeverse{\arabic{verse}}%Ex.14:30
{וַיּ֨וֹשַׁע יְהֹוָ֜ה בַּיּ֥וֹם הַה֛וּא אֶת־יִשְׂרָאֵ֖ל מִיַּ֣ד מִצְרָ֑יִם וַיַּ֤רְא יִשְׂרָאֵל֙ אֶת־מִצְרַ֔יִם מֵ֖ת עַל־שְׂפַ֥ת הַיָּֽם׃
\rashi{\rashiDH{וירא ישראל את מצרים מת. }שפלטן הים על שפתו, כדי שלא יאמרו ישראל, כשם שאנו עולים מצד זה, כך הם עולין מצד אחר רחוק ממנו, וירדפו אחרינו׃ }}
{וּפְרַק יְיָ בְּיוֹמָא הַהוּא יָת יִשְׂרָאֵל מִיְּדָא דְּמִצְרָאֵי וַחֲזָא יִשְׂרָאֵל יָת מִצְרָאֵי מָיְתִין עַל כֵּיף יַמָּא׃}
{Thus the \lord\space saved Israel that day out of the hand of the Egyptians; and Israel saw the Egyptians dead upon the sea-shore.}{\arabic{verse}}
\threeverse{\arabic{verse}}%Ex.14:31
{וַיַּ֨רְא יִשְׂרָאֵ֜ל אֶת־הַיָּ֣ד הַגְּדֹלָ֗ה אֲשֶׁ֨ר עָשָׂ֤ה יְהֹוָה֙ בְּמִצְרַ֔יִם וַיִּֽירְא֥וּ הָעָ֖ם אֶת־יְהֹוָ֑ה וַיַּֽאֲמִ֙ינוּ֙ בַּֽיהֹוָ֔ה וּבְמֹשֶׁ֖ה עַבְדּֽוֹ׃ \petucha 
\rashi{\rashiDH{את היד הגדולה. }את הגבורה הגדולה שעשתה ידו של הקב״ה. והרבה לשונות נופלין על לשון יד, וכולן לשון יד ממש הן, והמפרשו יתקן הלשון אחר ענין הדבור׃ 
}}
{וַחֲזָא יִשְׂרָאֵל יָת גְּבוּרַת יְדָא רַבְּתָא דַּעֲבַד יְיָ בְּמִצְרָאֵי וּדְחִילוּ עַמָּא מִן קֳדָם יְיָ וְהֵימִינוּ בְּמֵימְרָא דַּייָ וּבִנְבִיאוּת מֹשֶׁה עַבְדֵּיהּ׃}
{And Israel saw the great work which the \lord\space did upon the Egyptians, and the people feared the \lord; and they believed in the \lord, and in His servant Moses.}{\arabic{verse}}
\newperek
\engnote{For brevity, all notes of breaks in the Song of the Sea are omitted. For the proper formatting for this passage, see page \pageref{shirathayam}.}
\threeverse{\Roman{chap}}%Ex.15:1
{אָ֣ז יָשִֽׁיר־מֹשֶׁה֩ וּבְנֵ֨י יִשְׂרָאֵ֜ל אֶת־הַשִּׁירָ֤ה הַזֹּאת֙ לַֽיהֹוָ֔ה וַיֹּאמְר֖וּ לֵאמֹ֑ר           אָשִׁ֤ירָה לַֽיהֹוָה֙ כִּֽי־גָאֹ֣ה גָּאָ֔ה           ס֥וּס וְרֹכְב֖וֹ רָמָ֥ה בַיָּֽם׃          
\rashi{\rashiDH{אז ישיר משה. }אז כשראה הנס, עלה בלבו שישיר שירה. וכן אָז יְדַבֵּר יְהֹושֻׁעַ (יהושע י, יב). וכן וּבַיִת יַעֲשֶׂה לְבַת פַּרְעֹה (מלכים־א ז, ח), חשב בלבו שיעשה לה. אף כאן ישיר, אמר לו לבו שישיר, וכן עשה, ויאמרו לאמר אשירה לה׳. וכן ביהושע כשראה הנס, אמר לו לבו שידבר, וכן עשה, וַיֹּאמֶר לְעֵינֵי יִשְׂרָאֵל. וכן שירת הבאר, שפתח בה אָז יָשִׁיר יִשְׂרָאֵל (במדבר כא, יז), פירש אחריו עֲלִי בְאֵר עֱנוּ לָהּ. אָז יִבְנֶה שְׁלֹמֹה בָּמָה (מלכים־א יא, ז), פירשו בו חכמי ישראל שבקש לבנות ולא בנה, למדנו שהיו״ד על שם המחשבה נאמרה, זהו ליישב פשוטו. אבל מדרשו אמרו רז״ל, מכאן רמז לתחיית המתים מן התורה, וכן בכלן, חוץ משל שלמה שפירשוהו בקש לבנות ולא בנה. ואין לומר וליישב לשון הזה כשאר דברים הנכתבים בלשון עתיד והן מיד, כגון כָּכָה יַעֲשֶׂה אִיֹּוב (איוב א, ה), עַל פִּי ה׳ יַחֲנוּ (במדבר ט, כג), וְיֵשׁ אֲשֶׁר יִהְיֶה הֶעָנָן, לפי שהן דבר ההווה תמיד, ונופל בו בין לשון עתיד ובין לשון עבר, אבל זה שלא היה אלא לשעה, אינו יכול לישבו בלשון הזה׃ }\rashi{\rashiDH{כי גאה גאה. }כתרגומו. (דבר אחר, בא הכפל לומר שעשה דבר שאי אפשר לבשר ודם לעשות, כשהוא נלחם בחבירו ומתגבר עליו, מפילו מן הסוס, וכאן סוס ורוכבו רמה בים, וכל שאי אפשר לעשות על ידי זולתו נופל בו לשון גאות, כמו כי גֵּאוּת עָשָׂה (ישעיה יב, ה), וכן כל השירה תמצא כפולה, עזי וזמרת יה ויהי לי לישועה, ה׳ איש מלחמה ה׳ שמו, וכן כולם. ברש״י ישן). דבר אחר כי גאה גאה, על כל השירות וכל מה שאקלס בו, עוד יש בו תוספת, ולא כמדת בשר ודם, שמקלסין אותו ואין בו׃ }\rashi{\rashiDH{סוס ורכבו. }שניהם קשורים זה בזה, והמים מעלין אותם לרום ומורידין אותם לעומק ואינן נפרדין׃ 
}\rashi{\rashiDH{רמה. }השליך, וכן וּרְמִיו לְגֹוא אַתּוּן נוּרָא (דניאל ג, כא). ומדרש אגדה, כתוב אחד אומר רמה, וכתוב אחד אומר ירה, מלמד שהיו עולין לרום ויורדין לתהום, כמו מִי יָרָה אֶבֶן פִּנָּתָהּ (איוב לח, ו), מלמעלה למטה׃ }}
{בְּכֵן שַׁבַּח מֹשֶׁה וּבְנֵי יִשְׂרָאֵל יָת תּוּשְׁבַּחְתָּא הָדָא קֳדָם יְיָ וַאֲמַרוּ לְמֵימַר נְשַׁבַּח וְנוֹדֵי קֳדָם יְיָ אֲרֵי אִתְגְּאִי עַל גֵּיוְתָנַיָּא וְגֵיאוּתָא דִּילֵיהּ הִיא סוּסְיָא וְרָכְבֵיהּ רְמָא בְיַמָּא׃}
{Then sang Moses and the children of Israel this song unto the \lord, and spoke, saying: I will sing unto the \lord, for He is highly exalted; The horse and his rider hath He thrown into the sea.}{\Roman{chap}}
\threeverse{\arabic{verse}}%Ex.15:2
{עָזִּ֤י וְזִמְרָת֙ יָ֔הּ וַֽיְהִי־לִ֖י לִֽישׁוּעָ֑ה           זֶ֤ה אֵלִי֙ וְאַנְוֵ֔הוּ           אֱלֹהֵ֥י אָבִ֖י וַאֲרֹמְמֶֽנְהוּ׃          
\rashi{\rashiDH{עזי וזמרת יה. }אונקלוס תרגם תּוֹקְפִי וְתֻשְׁבַּחְתִּי, עזי כמו עזי בשור״ק, וזמרת כמו וזמרתי, ואני תמה על לשון המקרא, שאין לך כמוהו בנקודתו במקרא, אלא בשלשה מקומות שהוא סמוך אצל וזמרת, וכל שאר מקומות נקוד שור״ק, ה׳ עֻזִּי וּמָעֻזִּי (ירמיה טז, יט), עֻזֹּו אֵלֶיךָ אֶשְׁמֹרָה (תהלים נט, י), וכן כל תיבה בת שתי אותיות הנקודה מלאפו״ם, כשהיא מארכת באות שלישית ואין השניה (בשו״א) בחטף, הראשונה נקודה בשור״ק, כגון עז עזי, רוק רוקי, חק חקי, על עולו, יסור עולו, כל כלו, ושלישים על כלו. ואלו שלשה עזי וזמרת, של כאן ושל ישעיה ושל תהלים, נקודה בחטף קמ״ץ, ועוד אין באחד מהם כתוב וזמרתי, אלא וזמרת, וכלם סמוך להם ויהי לי לישועה. לכך אני אומר ליישב לשון המקרא, שאין עזי כמו עוזי, ולא וזמרת כמו וזמרתי, אלא עזי שם דבר הוא, כמו הַיֹּשְבִי בַּשָּׁמָיִם (שם קכג, א), שֹׁכְנִי בְחַגְוִי סֶלַע (עובדיה א, ג), שֹׁכְנִי סְנֶה (דברים לג, טז). וזהו השבח, עזי וזמרת יה, הוא היה לי לישועה, וזמרת דבוק הוא לתיבת ה׳, כמו לְעֶזְרַת ה׳ (שופטים ה, כג), בְּעֶבְרַת ה׳ (ישעיה ט, יח), עַל דִּבְרַת בְּנֵי הָאָדָם (קהלת ג, יח). ולשון וזמרת, לשון לֹא תִזְמֹור (ויקרא כה, ד), זְמִיר עָרִיצִים (ישעיה כה, ה), לשון כסוח וכריתה, עוזו ונקמתו של אלהינו היה לנו לישועה. ואל תתמה על לשון ויהי, שלא נאמר היה, שיש לנו מקראות מדברים בלשון זה, וזה דוגמתו, אֶת קִירֹות הַבַּיִת סָבִיב לַהֵיכָל וְלַדְּבִיר וַיַעֲשׂ צְלָעֹות סָבִיב (מלכים־א ו, ה), היה לו לומר עשה צלעות סביב. וכן וּבְנֵי יִשְׂרָאֵל הַיֹשְבִים בְּעָרֵי יְהוּדָה וַיִמְלֹךְ עֲלֵיהֶם רְחַבְעָם (דברי הימים־ב י, יז), היה לו לומר מלך עליהם רחבעם. מִבִּלְתִּי יְכֹלֶת ה׳ וגו׳ וַישְׁחָטֵם (במדבר יד, טז), היה לו לומר שחטם. וְהָאֲנָשִׁים אֲשֶׁר שָׁלַח משֶׁה וגו׳ וַיָמֻתוּ (שם לולז), מתו היה לו לומר. וַאֲשֶׁר לֹא שָׂם לִבֹּו אֶל דְּבַר ה׳ וַיַּעֲזֹב (שמות ט, כא), היה לו לומר עזב׃ }\rashi{\rashiDH{זה אלי. }בכבודו נגלה עליהם והיו מראין אותו באצבע, ראתה שפחה על הים מה שלא ראו נביאים׃ }\rashi{\rashiDH{ואנוהו. }אונקלוס תרגם לשון נוה, נָוֶה שַׁאֲנָן (ישעיה לג, כ), לִנְוֵה צֹאן (שם סה, י). דבר אחר ואנוהו, לשון נוי, אספר נויו ושבחו לבאי עולם, כגון מַה דֹּודֵךְ מִדֹּוד, דֹּודִי צַח וְאָדֹום (שיר השירים ה, טי), וכל הענין׃ }\rashi{\rashiDH{אלהי אבי. }הוא זה, וארוממנהו. אלהי אבי, לא אני תחלת הקדושה, אלא מוחזקת ועומדת לי הקדושה, ואלהותו עלי מימי אבותי׃ }}
{תּוּקְפִי וְתוּשְׁבַּחְתִּי דְּחִילָא יְיָ אֲמַר בְּמֵימְרֵיהּ וַהֲוָה לִי לְפָרִיק דֵּין אֱלָהִי וְאֶבְנֵי לֵיהּ מַקְדַּשׁ אֱלָהָא דַּאֲבָהָתִי וְאֶפְלַח קֳדָמוֹהִי׃}
{The \lord\space is my strength and song, And He is become my salvation; This is my God, and I will glorify Him; My father’s God, and I will exalt Him.}{\arabic{verse}}
\threeverse{\arabic{verse}}%Ex.15:3
{יְהֹוָ֖ה אִ֣ישׁ מִלְחָמָ֑ה יְהֹוָ֖ה שְׁמֽוֹ׃          
\rashi{\rashiDH{ה׳ איש מלחמה. }בעל מלחמה, כמו אִישׁ נָעֳמִי (רות א, ג), וכל איש ואישך מתורגמין בעל, וכן וְחָזַקְתָּ וְהָיִיתָ לְאִישׁ (מלכים־א ב, ב), לגבור׃ }\rashi{\rashiDH{ה׳ שמו. }מלחמותיו לא בכלי זיין, אלא בשמו הוא נלחם, כמו שאמר דוד וְאָנֹכִי בָּא אֵלֶיךָ בְּשֵׁם ה׳ צְבָאֹות (שמואל־א יז, מה). דבר אחר ה׳ שמו, אף בשעה שהוא נלחם ונוקם מאויביו, אוחז הוא במדתו לרחם על ברואיו ולזון את כל באי עולם, ולא כמדת מלכי אדמה, כשהוא עוסק במלחמה פונה עצמו מכל עסקים, ואין בו כח לעשות זו וזו׃ }}
{יְיָ מָארֵי נִצְחָן קְרָבַיָּא יְיָ שְׁמֵיהּ׃}
{The \lord\space is a man of war, The \lord\space is His name.}{\arabic{verse}}
\threeverse{\arabic{verse}}%Ex.15:4
{מַרְכְּבֹ֥ת פַּרְעֹ֛ה וְחֵיל֖וֹ יָרָ֣ה בַיָּ֑ם           וּמִבְחַ֥ר שָֽׁלִשָׁ֖יו טֻבְּע֥וּ בְיַם־סֽוּף׃          
\rashi{\rashiDH{ירה בים. }שַׁדִּי בְיַמָּא, שדי לשון ירייה. וכן הוא אומר אֹו יָרֹה יִיָּרֶה (שמות יט, יג), או אִשְׁתְּדָאָה אִישְׁתְּדֵי, והתי״ו משתמש באלו במקום התפעל׃ }\rashi{\rashiDH{ומבחר. }שם דבר, כמו מרכב, משכב, מקרא קדש׃ }\rashi{\rashiDH{טבעו. }אין טביעה אלא במקום טיט, כמו טָבַעְתִּי בִּיוֵן מְצוּלָה (תהלים סט, ג), וַיִּטְבַּע יִרְמְיָהוּ בַּטִּיט (ירמיה לח, ו.  מכילתא פ״ד). מלמד שנעשה הים טיט, לגמול להם כמדתם ששעבדו את ישראל בחומר ובלבנים׃ }}
{רְתִכֵּי פַרְעֹה וּמַשְׁרְיָתֵיהּ שְׁדִי בְיַמָּא וּמִבְחַר גִּבָּרוֹהִי אִטְּבַעוּ בְיַמָּא דְּסוּף׃}
{Pharaoh’s chariots and his host hath He cast into the sea, And his chosen captains are sunk in the Red Sea.}{\arabic{verse}}
\threeverse{\arabic{verse}}%Ex.15:5
{תְּהֹמֹ֖ת יְכַסְיֻ֑מוּ יָרְד֥וּ בִמְצוֹלֹ֖ת כְּמוֹ־אָֽבֶן׃          
\rashi{\rashiDH{יכסימו. }כמו יכסום, והיו״ד האמצעית יתירה בו, ודרך מקראות בכך, כמו וּבְקָרְךָ וְצֹאנְךָ יִרְבְּיֻן (דברים ח, יד), יִרְוְיֻן מִדֶּשֶׁן בֵּיתֶך (תהלים לו, ט), והיו״ד ראשונה שמשמעה לשון עתיד, כך פרשוהו, טבעו בים סוף כדי שיחזרו המים ויכסו אותן. יכסיומו, אין דומה לו במקרא בנקודתו, ודרכו להיות בנקודתו יכסיומו במלא״פום (גם כאן מוכח להיות חולם כמ״ש)׃ }\rashi{\rashiDH{כמו אבן. }ובמקום אחר צללו כעופרת, ובמקום אחר יאכלמו כקש, הרשעים כקש, הולכים ומטורפין עולין ויורדין. בינונים כאבן, והכשרים כעופרת, שנחו מיד׃ }}
{תְּהוֹמַיָּא חֲפוֹ עֲלֵיהוֹן נְחַתוּ לְעוּמְקַיָּא כְּאַבְנָא׃}
{The deeps cover them— They went down into the depths like a stone.}{\arabic{verse}}
\threeverse{\arabic{verse}}%Ex.15:6
{יְמִֽינְךָ֣ יְהֹוָ֔ה נֶאְדָּרִ֖י בַּכֹּ֑חַ           יְמִֽינְךָ֥ יְהֹוָ֖ה תִּרְעַ֥ץ אוֹיֵֽב׃          
\rashi{\rashiDH{ימינך. ימינך. }שני פעמים, כשישראל עושין רצונו של מקום השמאל נעשית ימין׃ }\rashi{\rashiDH{ימינך ה׳ נאדרי בכח. }להציל את ישראל, וימינך השנית תרעץ אויב. ולי נראה, אותה ימין עצמה תרעץ אויב, מה שאי אפשר לאדם לעשות שתי מלאכות ביד אחת. ופשוטו של מקרא, ימינך הנאדרת בכח מה מלאכתה, ימינך, היא תרעץ אויב, וכמה מקראות דוגמתו, כִּי הִנֵּה אֹיְבֶיך ה׳ כִּי הִנְּה אֹיְבֶיךָ יֹאבֵדוּ (תהלים צב, י), עַד מָתַי רְשָׁעִים ה׳ עַד מָתַי רְשָׁעִים יַעֲלֹזוּ (שם צד, ג), נָשְׂאוּ נְהָרֹות ה׳ נָשְׂאוּ נְהָרֹות קֹולָם (שם צג, ג), לֹא לָנוּ ה׳ ל ֹא לָנוּ (שם קטו, א), אֶעֱנֶה נְאֻם ה׳ אֶעֱנֶה אֶת הַשָּׁמָּיִם (הושע ב, כג), אָנֹכִי לַה׳ אָנֹכִי אָשִׁירָה (שופטים ה, ג), לוּלֵי ה׳ וגו׳ לוּלֵי ה׳ שֶׁהָיָה לָנוּ בְּקוּם עָלֵינוּ אָדָם (תהלים קכד, אב), עוּרִי עוּרִי דְּבֹורָה עוּרִי עוּרִי דַּבְּרִי שִׁיר (שופטים ה, יב), תִּרְמְסֶנָּה רָגֶל רַגְלֵי עָנִי (ישעיה כו, ו), וְנָתַן אַרְצָם לְנַחֲלָה נַחֲלָה לְיִשְׂרָאֵל עַבְדֹּו (תהלים קלו, כאכב)׃ }\rashi{\rashiDH{נאדרי. }היו״ד יתירה, כמו רַבָּתִי עָם, שָׂרָתִי בַּמְּדִינֹות (איכה א, א), גְּנֻבְתִי יֹום (בראשית לא, לט)׃ }\rashi{\rashiDH{תרעץ אויב. }תמיד היא רועצת ומשברת האויב, ודומה לו וַיִּרְעֲצוּ וַיְרֹצְצוּ את בְּנֵי יִשְׂרָאֵל, בשופטים (י, ח). (דבר אחר, ימינך הנאדרת בכח היא משברת ומלקה אויב)׃ }}
{יַמִּינָךְ יְיָ אַדִּירָא בְּחֵילָא יַמִּינָךְ יְיָ תְּבַרַת סָנְאָה׃}
{Thy right hand, O \lord, glorious in power, Thy right hand, O \lord, dasheth in pieces the enemy.}{\arabic{verse}}
\threeverse{\arabic{verse}}%Ex.15:7
{וּבְרֹ֥ב גְּאוֹנְךָ֖ תַּהֲרֹ֣ס קָמֶ֑יךָ           תְּשַׁלַּח֙ חֲרֹ֣נְךָ֔ יֹאכְלֵ֖מוֹ כַּקַּֽשׁ׃          
\rashi{\rashiDH{וברב גאונך. }אם היד בלבד רועצת האויב, כשהוא מרימה ברוב גאונו אז יהרוס קמיו, ואם ברוב גאונו לבד אויביו נהרסים, ק״ו כששלח בם חרון אף יאכלמו׃ }\rashi{\rashiDH{תהרס. }תמיד אתה הורס קָמֶיךָ הקמים נגדך, ומי הם הקמים כנגדו, אלו הקמים על ישראל, וכן הוא אומר, כִּי הִנֵּה אֹויְבֶיךָ יֶהֶמָיוּן (תהלים פג, ג), ומה היא ההמיה, עַל עַמְּךָ יַעֲרִימוּ סֹוד (שם שם, ד), ועל זה קורא אותם אויביו של מקום׃ }}
{וּבִסְגֵּי תוּקְפָךְ תַּבַּרְתָּנוּן לִדְקָמוּ עַל עַמָּךְ שַׁלַּחְתְּ רוּגְזָךְ שֵׁיצֵינוּן כְּנוּרָא לְקַשָּׁא׃}
{And in the greatness of Thine excellency Thou overthrowest them that rise up against Thee; Thou sendest forth Thy wrath, it consumeth them as stubble.}{\arabic{verse}}
\threeverse{\arabic{verse}}%Ex.15:8
{וּבְר֤וּחַ אַפֶּ֙יךָ֙ נֶ֣עֶרְמוּ מַ֔יִם           נִצְּב֥וּ כְמוֹ־נֵ֖ד נֹזְלִ֑ים           קָֽפְא֥וּ תְהֹמֹ֖ת בְּלֶב־יָֽם׃          
\rashi{\rashiDH{וברוח אפיך. }היוצא משני נחירים של אף, דִּבֵּר הכתוב כביכול בשכינה דוגמת מלך בשר ודם, כדי להשמיע אוזן הבריות כפי ההוה, שיוכלו להבין דבר. כשאדם כועס יוצא רוח מנחיריו, וכן עָלָה עָשָׁן בְּאַפֹּו (תהלים יח, ט), וכן וּמְרוּחַ אַפֹּו יִכְלוּ (איוב ד, ט), וזהו שאמר לְמַעַן שְׁמִי אַאֲרִיךְ אַפִּי (ישעיה מח, ט), כשזעפו נחה נשימתו ארוכה, וכשהוא כועס נשימתו קצרה. וּתְהִלָּתִי אֶחֱטָם לָךְ (שם), ולמען תהלתי אשים חטם באפי, לסתום נחירי בפני האף והרוח שלא יצאו. לך, בשבילך. אחטם, כמו נאקה בחטם, במסכת שבת (נא׃), כך נראה בעיני. וכל אף וחרון שבמקרא אני אומר כן, חרה אף, כמו וְעַצְמִי חָרָה מִנִּי חֹרֶב (איוב ל, ל), לשון שרפה ומוקד, שהנחירים מתחממים ונחרים בעת הקצף וחרון, מגזרת חרה, כמו רצון מגזרת רצה, וכן חמה לשון חמימות, על כן הוא אומר וַחֲמָתֹו בָּעֲרָה בֹו (אסתר א, יג), ובנוח החמה אומר, נתקררה דעתו׃ }\rashi{\rashiDH{נערמו מים. }אונקלוס תרגם לשון ערמימות, ולשון צחות המקרא כמו עֲרֵמַת חִטִּים (שיר השירים ז, ג), ונצבו כמו נד יוכיח׃ \rashiDH{נערמו מים. }ממוקד רוח שיצא מאפך יבשו המים, והם נעשו כמין גלים וכריות של ערימה שהם גבוהים׃ }\rashi{\rashiDH{כמו נד. }כתרגומו כְשׁוּר, כחומה׃ }\rashi{\rashiDH{נד. }לשון צָבוּר וכנוס, כמו נֵד קָצִיר בְּיֹום נַחֲלָה (ישעיה יז, יא), כֹּנֵס כַּנֵּד (תהלים לג, ו), לא כתב כונס כנאד אלא כנד, ואילו היה כנד כמו כנאד, וכונס לשון הכנסה, היה לו לכתוב מכניס כבנאד מי הים, אלא כונס לשון אוסף וצובר הוא, וכן קָמוּ נֵד אֶחָד (יהושע ג, טז), וְיַעַמְדוּ נֵד אֶחָד (שם יג), ואין לשון קימה ועמידה בנאדות אלא בחומות וצבורים, ולא מצינו נאד נקוד אלא במלאפו״ם (חול״ם), כמו שִׂימָה דִמְעָתִי בְּנֹאדֶךָ (תהלים נו, ט), אֶת נֹאוד הֶחָלָב (שופטים ד, כ)׃ }\rashi{\rashiDH{קפאו. }כמו וְכַגְּבִינָה תַּקְפִּיאֵנִי (איוב י, ו), שהוקשו ונעשו כאבנים, והמים זורקים את המצרים על האבן בכח ונלחמים בם בכל מיני קושי׃ 
}\rashi{\rashiDH{בלב ים. }בחוזק הים, ודרך המקראות לדבר כן, עד לֵב הַשָּׁמַיִם (דברים ד, יא), בְּלֵב הָאֵלָה (שמואל־ב יח, יד), לשון עקרו ותקפו של דבר׃ }}
{וּבְמֵימַר פּוּמָּךְ חֲכִימוּ מַיָּא קָמוּ כְשׁוּר אָזְלַיָּא קְפוֹ תְהוֹמֵי בְּלִבָּא דְּיַמָּא׃}
{And with the blast of Thy nostrils the waters were piled up— The floods stood upright as a heap; The deeps were congealed in the heart of the sea.}{\arabic{verse}}
\threeverse{\arabic{verse}}%Ex.15:9
{אָמַ֥ר אוֹיֵ֛ב אֶרְדֹּ֥ף אַשִּׂ֖יג           אֲחַלֵּ֣ק שָׁלָ֑ל תִּמְלָאֵ֣מוֹ נַפְשִׁ֔י           אָרִ֣יק חַרְבִּ֔י תּוֹרִישֵׁ֖מוֹ יָדִֽי׃          
\rashi{\rashiDH{אמר אויב. }לעמו, כְּשֶׁפִּתָּם בדברים ארדוף ואשיגם ואחלק שלל עם שרי ועבדי׃ }\rashi{\rashiDH{תמלאמו. }תמלא מהם \rashiDH{נפשי. }רוחי ורצוני, ואל תתמה על תיבה המדברת בשתים, תמלאמו תמלא מהם, יש הרבה כלשון הזה, כִּי אֶרֶץ הַנֶּגֶב נְתַתָּנִי (שופטים א, טו), כמו נתת לי. וְלֹא יָכְלוּ דַּבְּרֹו לְשָׁלֹם (בראשית לז, ד), כמו דבר עמו. בָּנַי יְצָאֻנִי (ירמיה י, כ), כמו יצאו ממני. מִסְפַּר צְעָדַי אַגִּידֶנוּ (איוב לא, לז), כמו אגיד לו. אף כאן תמלאמו, תמלא נפשי מהם׃ }\rashi{\rashiDH{אריק חרבי. }אשלוף, ועל שם שהוא מריק את התער בשליפתו ונשאר ריק, נופל בו לשון הרקה, כמו מְרִיקִים שַׂקֵּיהֶם (בראשית מב, לה), וְכֵליו יָרִיקוּ (ירמיה מח, יב). ואל תאמר, אין לשון ריקות נופל על היוצא, אלא על התיק ועל השק ועל הכלי שיצא ממנה, אבל לא על החרב ועל היין, ולדחוק ולפרש אריק חרבי כלשון וַיָּרֶק אֶת חֲנִיכָיו (בראשית יד, יד), אזדיין בחרבי, מצינו הלשון מוסב אף על היוצא, שֶׁמֶן תּוּרַק (שיר השירים א, ג), וְלֹא הוּרַק מִכְּלִי אֶל כֶּלִי (ירמיה מח, יא). לא הורק הכלי אין כתיב כאן, אלא לא הורק היין מכלי אל כלי, מצינו הלשון מוסב על היין, וכן וְהֵרִיקוּ חַרְבֹותָם עַל יְפִי חָכְמָתֶךָ (יחזקאל כח, ז), דחירם׃ }\rashi{\rashiDH{תורישמו. }לשון רישות ודלות, כמו מֹורִישׁ וּמַעֲשִׁיר (שמואל־א ב, ז)׃ }}
{דַּהֲוָה אָמַר סָנְאָה אֶרְדּוֹף אַדְבֵּיק אֲפַלֵּיג בִּזְּתָא תִּסְבַּע מִנְּהוֹן נַפְשִׁי אֶשְׁלוֹף חַרְבִּי תְּשֵׁיצֵינוּן יְדִי׃}
{The enemy said: ‘I will pursue, I will overtake, I will divide the spoil; My lust shall be satisfied upon them; I will draw my sword, my hand shall destroy them.’}{\arabic{verse}}
\threeverse{\arabic{verse}}%Ex.15:10
{נָשַׁ֥פְתָּ בְרוּחֲךָ֖ כִּסָּ֣מוֹ יָ֑ם           צָֽלְלוּ֙ כַּֽעוֹפֶ֔רֶת בְּמַ֖יִם אַדִּירִֽים׃          
\rashi{\rashiDH{נשפת. }לשון הפחה, וכן וְגַם נָשַׁף בָּהֶם (ישעיה מ, כד)׃ }\rashi{\rashiDH{צללו. }שקעו, עמקו לשון מצולה׃ }\rashi{\rashiDH{כעופרת.} אבר, פלו״ם בלע״ז׃ }}
{אֲמַרְתְּ בְּמֵימְרָךְ חֲפָא עֲלֵיהוֹן יַמָּא אִשְׁתְּקַעוּ כַּאֲבָרָא בְּמַיִין תַּקִּיפִין׃}
{Thou didst blow with Thy wind, the sea covered them; They sank as lead in the mighty waters.}{\arabic{verse}}
\threeverse{\arabic{verse}}%Ex.15:11
{מִֽי־כָמֹ֤כָה בָּֽאֵלִם֙ יְהֹוָ֔ה           מִ֥י כָּמֹ֖כָה נֶאְדָּ֣ר בַּקֹּ֑דֶשׁ           נוֹרָ֥א תְהִלֹּ֖ת עֹ֥שֵׂה פֶֽלֶא׃          
\rashi{\rashiDH{באלים. }בחזקים, כמו וְאֶת אֵילֵי הָאָרֶץ לָקָח (יחזקאל יז, יג), אֱיָלוּתִי לְעֶזְרָתִי חוּשָׁה (תהלים כב, כ)׃ 
}\rashi{\rashiDH{נורא תהלת. }יראוי מלהגיד תהלותיו פן ימעטו, כמ״ש לְךָ דֻּמִיָּה תְהִלָּה (שם סה, ב)׃ }}
{לֵית בָּר מִנָּךְ אַתְּ הוּא אֱלָהָא יְיָ לֵית אֱלָהּ אֵלָא אַתְּ אַדִּיר בְּקוּדְשָׁא דְּחִיל תּוּשְׁבְּחָן עָבֵיד פְּרִישָׁן׃}
{Who is like unto Thee, O \lord, among the mighty? Who is like unto Thee, glorious in holiness, Fearful in praises, doing wonders?}{\arabic{verse}}
\threeverse{\arabic{verse}}%Ex.15:12
{נָטִ֙יתָ֙ יְמִ֣ינְךָ֔ תִּבְלָעֵ֖מוֹ אָֽרֶץ׃          
\rashi{\rashiDH{נטית ימינך. }כשהקב״ה נוטה ידו, הרשעים כָּלִים ונופלים, לפי שהכל נתון בידו ונופלים בהטייתה, וכן הוא אומר, וה׳ יַטֶּה יָדֹו וְכָשַׁל עֹוזֵר וְנָפַל עָזֻר (ישעיה לא, ג), משל לכלי זכוכית הנתונים בידי אדם, מטה ידו מעט והן נופלין ומשתברין׃ }\rashi{\rashiDH{תבלעמו ארץ. }מכאן שזכו לקבורה, בשכר שאמרו ה׳ הצדיק׃ }}
{אֲרֵימְתְּ יַמִּינָךְ בְּלַעַתְנוּן אַרְעָא׃}
{Thou stretchedst out Thy right hand— The earth swallowed them.}{\arabic{verse}}
\threeverse{\arabic{verse}}%Ex.15:13
{נָחִ֥יתָ בְחַסְדְּךָ֖ עַם־ז֣וּ גָּאָ֑לְתָּ           נֵהַ֥לְתָּ בְעׇזְּךָ֖ אֶל־נְוֵ֥ה קׇדְשֶֽׁךָ׃          
\rashi{\rashiDH{נהלת. }לשון מנהל. ואונקלוס תרגם לשון נושא וסובל, ולא דקדק לפרש אחר לשון העברית׃ 
}}
{דַּבַּרְהִי בְּטָבְוָתָךְ לְעַמָּא דְּנָן דִּפְרַקְתָּא דַּבַּרְהִי בְּתוּקְפָךְ לְדֵירָא דְּקוּדְשָׁךְ׃}
{Thou in Thy love hast led the people that Thou hast redeemed; Thou hast guided them in Thy strength to Thy holy habitation.}{\arabic{verse}}
\threeverse{\arabic{verse}}%Ex.15:14
{שָֽׁמְע֥וּ עַמִּ֖ים יִרְגָּז֑וּן           חִ֣יל אָחַ֔ז יֹשְׁבֵ֖י פְּלָֽשֶׁת׃          
\rashi{ירגזון. מתרגזין׃}\rashi{\rashiDH{ישבי פלשת. }מפני שהרגו את בני אפרים, שמיהרו את הקץ ויצאו בחזקה, כמפורש בדברי הימים, והרגום אנשי גת׃ }}
{שְׁמַעוּ עַמְמַיָּא וְזָעוּ דַּחְלָא אֲחַדַתְנוּן לְדַהֲווֹ יָתְבִין בִּפְלָשֶׁת׃}
{The peoples have heard, they tremble; Pangs have taken hold on the inhabitants of Philistia.}{\arabic{verse}}
\threeverse{\arabic{verse}}%Ex.15:15
{אָ֤ז נִבְהֲלוּ֙ אַלּוּפֵ֣י אֱד֔וֹם           אֵילֵ֣י מוֹאָ֔ב יֹֽאחֲזֵ֖מוֹ רָ֑עַד           נָמֹ֕גוּ כֹּ֖ל יֹשְׁבֵ֥י כְנָֽעַן׃          
\rashi{\rashiDH{אלופי אדום אילי מואב. }והלא לא היה להם לירא כלום, שהרי לא עליהם הולכים, אלא מפני אנינות (מכילתא שירה פ״ט), שהיו מתאוננים ומצטערים על כבודם של ישראל׃ }\rashi{\rashiDH{נמוגו. }נמסו, כמו בִּרְבִיבִים תְּמֹגְגֶנָּה (תהלים סה, יא). אמרו, עלינו הם באים, לכלותינו ולירש את ארצנו׃ }}
{בְּכֵן אִתְבְּהִילוּ רַבְרְבֵי אֱדוֹם תַּקִּיפֵי מוֹאָב אֲחַדִנּוּן רְתֵיתָא אִתְּבַרוּ כֹּל דַּהֲווֹ יָתְבִין בִּכְנָעַן׃}
{Then were the chiefs of Edom affrighted; The mighty men of Moab, trembling taketh hold upon them; All the inhabitants of Canaan are melted away.}{\arabic{verse}}
\threeverse{\arabic{verse}}%Ex.15:16
{תִּפֹּ֨ל עֲלֵיהֶ֤ם אֵימָ֙תָה֙ וָפַ֔חַד           בִּגְדֹ֥ל זְרוֹעֲךָ֖ יִדְּמ֣וּ כָּאָ֑בֶן           עַד־יַעֲבֹ֤ר עַמְּךָ֙ יְהֹוָ֔ה           עַֽד־יַעֲבֹ֖ר עַם־ז֥וּ קָנִֽיתָ׃          
\rashi{\rashiDH{תפול עליהם אימתה. }על הרחוקים׃ 
}\rashi{\rashiDH{ופחד. }על הקרובים, כענין שנאמר כִּי שָׁמַעְנוּ אֵת אֳשֶׁר הֹובִישׁ וגו׳ (יהושע ב, י.  מכילתא שירה פ״ט)׃ }\rashi{\rashiDH{עד יעבור. עד יעבור. }כתרגומו׃}\rashi{\rashiDH{קנית. }חבבת משאר אומות, כחפץ הקנוי בדמים יקרים שחביב על האדם׃ }}
{תִּפּוֹל עֲלֵיהוֹן אֵימְתָא וְדַחְלְתָא בִּסְגֵּי תוּקְפָךְ יִשְׁתְּקוּן כְּאַבְנָא עַד דְּיִעְבַּר עַמָּךְ יְיָ יָת אַרְנוֹנָא עַד דְּיִעְבַּר עַמָּא דְּנָן דִּפְרַקְתָּא יָת יַרְדְּנָא׃}
{Terror and dread falleth upon them; By the greatness of Thine arm they are as still as a stone; Till Thy people pass over, O \lord, Till the people pass over that Thou hast gotten.}{\arabic{verse}}
\threeverse{\arabic{verse}}%Ex.15:17
{תְּבִאֵ֗מוֹ וְתִטָּעֵ֙מוֹ֙ בְּהַ֣ר נַחֲלָֽתְךָ֔           מָכ֧וֹן לְשִׁבְתְּךָ֛ פָּעַ֖לְתָּ יְהֹוָ֑ה           מִקְּדָ֕שׁ אֲדֹנָ֖י כּוֹנְנ֥וּ יָדֶֽיךָ׃          
\rashi{\rashiDH{תביאמו. }נתנבא משה שלא יכנס לארץ, לכך לא נאמר תביאנו, (נראה שלא יכנסו לארץ וכו׳, והכי איתא בהדיא פרק יש נוחלין (בבא בתרא דף קי״ט׃) ובמכילתא (שם פ״י), הבנים יכנסו ולא האבות, אף שלא נגזרה גזירת מרגלים עדיין, מכל מקום ניבא ולא ידע מה ניבא. מהרש״ל)׃ }\rashi{\rashiDH{מכון לשבתך. }מקדש של מטה מכוון כנגד כסא של מעלה אשר פעלת׃}\rashi{\rashiDH{מקדש. }הטעם עליו זקף גדול, להפרידו מתיבת השם שלאחריו, המקדש אשר כוננו ידיך ה׳. חביב בית המקדש, שהעולם נברא ביד אחת, שנאמר אַף יָדִי יָסְדָה אֶרֶץ (ישעיה מח, יג), ומקדש בשתי ידים, ואימתי יבנה בשתי ידים, בזמן שה׳ ימלוך לעולם ועד, לעתיד לבא שכל המלוכה שלו׃ }}
{תַּעֵילִנּוּן וְתַשְׁרֵינוּן בְּטוּרָא דְּאַחְסָנְתָךְ אֲתַר לְבֵית שְׁכִינְתָךְ אַתְקֵינְתָּא יְיָ מַקְדְּשָׁא יְיָ אַתְקְנָהִי יְדָךְ׃}
{Thou bringest them in, and plantest them in the mountain of Thine inheritance, The place, O \lord, which Thou hast made for Thee to dwell in, The sanctuary, O Lord, which Thy hands have established.}{\arabic{verse}}
\threeverse{\arabic{verse}}%Ex.15:18
{יְהֹוָ֥ה \pasek  יִמְלֹ֖ךְ לְעֹלָ֥ם וָעֶֽד׃          
\rashi{\rashiDH{לעולם ועד. }לשון עולמית הוא, והוי״ו בו יסוד, לפיכך הוא פתוחה, אבל וְאָנֹכִי הַיֹּודֵעַ וָעֵד (ירמיה כט, כג), שהוי״ו בו שמוש, קמוצה היא׃ 
}}
{יְיָ מַלְכוּתֵיהּ לְעָלְמָא וּלְעָלְמֵי עָלְמַיָּא׃}
{The \lord\space shall reign for ever and ever.}{\arabic{verse}}
\threeverse{\arabic{verse}}%Ex.15:19
{כִּ֣י בָא֩ ס֨וּס פַּרְעֹ֜ה בְּרִכְבּ֤וֹ וּבְפָרָשָׁיו֙ בַּיָּ֔ם           וַיָּ֧שֶׁב יְהֹוָ֛ה עֲלֵהֶ֖ם אֶת־מֵ֣י הַיָּ֑ם           וּבְנֵ֧י יִשְׂרָאֵ֛ל הָלְכ֥וּ בַיַּבָּשָׁ֖ה בְּת֥וֹךְ הַיָּֽם׃ \petucha 
\rashi{\rashiDH{כי בא סוס פרעה. }כאשר בא׃}}
{אֲרֵי עָאלוּ סוּסָוָת פַּרְעֹה בִּרְתִכּוֹהִי וּבְפָרָשׁוֹהִי בְּיַמָּא וַאֲתֵיב יְיָ עֲלֵיהוֹן יָת מֵי יַמָּא וּבְנֵי יִשְׂרָאֵל הַלִּיכוּ בְּיַבֶּשְׁתָּא בְּגוֹ יַמָּא׃}
{For the horses of Pharaoh went in with his chariots and with his horsemen into the sea, and the \lord\space brought back the waters of the sea upon them; but the children of Israel walked on dry land in the midst of the sea.}{\arabic{verse}}
\threeverse{\arabic{verse}}%Ex.15:20
{וַתִּקַּח֩ מִרְיָ֨ם הַנְּבִיאָ֜ה אֲח֧וֹת אַהֲרֹ֛ן אֶת־הַתֹּ֖ף בְּיָדָ֑הּ וַתֵּצֶ֤אןָ כׇֽל־הַנָּשִׁים֙ אַחֲרֶ֔יהָ בְּתֻפִּ֖ים וּבִמְחֹלֹֽת׃
\rashi{\rashiDH{ותקח מרים הנביאה. }היכן נתנבאה, כשהיתה אחות אהרן קודם שנולד משה, אמרה עתידה אמי שתלד בן וכו׳, כדאיתא בסוטה (יב׃). דבר אחר אחות אהרן, לפי שמסר נפשו עליה כשנצטרעה, נקראת על שמו׃ 
}\rashi{\rashiDH{את התף. }כלי של מיני זמר׃}\rashi{\rashiDH{בתופים ובמחולות. }מובטחות היו צדקניות שבדור שהקב״ה עושה להם נסים, והוציאו תופים ממצרים׃ }}
{וּנְסֵיבַת מִרְיָם נְבִיאֲתָא אֲחָתֵיהּ דְּאַהֲרֹן יָת תּוּפָּא בִּידַהּ וּנְפַקָא כָל נְשַׁיָּא בָּתְרַהָא בְּתוּפִּין וּבְחִנְגִין׃}
{And Miriam the prophetess, the sister of Aaron, took a timbrel in her hand; and all the women went out after her with timbrels and with dances.}{\arabic{verse}}
\threeverse{\arabic{verse}}%Ex.15:21
{וַתַּ֥עַן לָהֶ֖ם מִרְיָ֑ם שִׁ֤ירוּ לַֽיהֹוָה֙ כִּֽי־גָאֹ֣ה גָּאָ֔ה ס֥וּס וְרֹכְב֖וֹ רָמָ֥ה בַיָּֽם׃ \setuma         
\rashi{\rashiDH{ותען להם מרים. }משה אמר שירה לאנשים, הוא אומר והם עונין אחריו, ומרים אמרה שירה לנשים (מכילתא פ״י)׃ }}
{וּמְעַנְיָא לְהוֹן מִרְיָם שַׁבַּחוּ וְאוֹדוֹ קֳדָם יְיָ אֲרֵי אִתְגְּאִי עַל גֵּיוְתָנַיָּא וְגֵיאוּתָא דִּילֵיהּ הִיא סוּסְיָא וְרָכְבֵיהּ רְמָא בְיַמָּא׃}
{And Miriam sang unto them: Sing ye to the \lord, for He is highly exalted: The horse and his rider hath He thrown into the sea.}{\arabic{verse}}
\threeverse{\arabic{verse}}%Ex.15:22
{וַיַּסַּ֨ע מֹשֶׁ֤ה אֶת־יִשְׂרָאֵל֙ מִיַּם־ס֔וּף וַיֵּצְא֖וּ אֶל־מִדְבַּר־שׁ֑וּר וַיֵּלְכ֧וּ שְׁלֹֽשֶׁת־יָמִ֛ים בַּמִּדְבָּ֖ר וְלֹא־מָ֥צְאוּ מָֽיִם׃
\rashi{\rashiDH{ויסע משה. }הסיען בעל כרחם, שעטרו מצרים את סוסיהם בתכשיטי זהב וכסף ואבנים טובות, והיו ישראל מוצאין אותם בים, וגדולה היתה ביזת הים מביזת מצרים, שנאמר תֹּורֵי זָהָב נַעֲשֶׂה לָךְ עִם נְקֻדֹּות הַכָּסֶף (שיר השירים א, יא.  מכילתא פסחא סוף פי״ג), לפיכך הוצרך להסיען בעל כרחם׃ 
}}
{וְאַטֵּיל מֹשֶׁה יָת יִשְׂרָאֵל מִיַּמָּא דְּסוּף וּנְפַקוּ לְמַדְבְּרָא דְּחַגְרָא וַאֲזַלוּ תְּלָתָא יוֹמִין בְּמַדְבְּרָא וְלָא אַשְׁכַּחוּ מַיָּא׃}
{And Moses led Israel onward from the Red Sea, and they went out into the wilderness of Shur; and they went three days in the wilderness, and found no water.}{\arabic{verse}}
\threeverse{\arabic{verse}}%Ex.15:23
{וַיָּבֹ֣אוּ מָרָ֔תָה וְלֹ֣א יָֽכְל֗וּ לִשְׁתֹּ֥ת מַ֙יִם֙ מִמָּרָ֔ה כִּ֥י מָרִ֖ים הֵ֑ם עַל־כֵּ֥ן קָרָֽא־שְׁמָ֖הּ מָרָֽה׃
\rashi{\rashiDH{ויבאו מרתה. }כמו למרה. ה״א בסוף תיבה במקום למ״ד בתחלתה, והתי״ו היא במקום ה״א הנשרשת בתיבת מרה, ובסמיכתה כשהיא נדבקת לה״א שהוא מוסיף במקום הלמ״ד, תהפך הה״א של שרש לתי״ו, וכן כל ה״א שהיא שרש בתיבה תתהפך לתי״ו בסמיכתה. כמו חֵמָה אֵין לִי (ישעיה כז, ד), וַחֲמָתֹו בָּעֲרָה בֹו (אסתר א, יב), הרי ה״א של שורש נהפכת לתי״ו מפני שנסמכת אל הוא״ו הנוספת. וכן עבד ואמה, הִנֵּה אֲמָתִי בִלְהָה (בראשית ל, ג). לְנֶפֶשׁ חַיָּה (שם ב, ז), וְזִהֲמַתֹּו חַיָּתֹו לָחֶם (איוב לג, כ). בֵּין הָרָמָה (שופטים ד, ה), וּתְשֻׁבָתֹו הָרָמָתָה (שמואל־א ז, יז)׃ }}
{וַאֲתוֹ לְמָרָה וְלָא יְכִילוּ לְמִשְׁתֵּי מַיָּא מִמָּרָה אֲרֵי מָרִירִין אִנּוּן עַל כֵּן קְרָא שְׁמַהּ מָרָה׃}
{And when they came to Marah, they could not drink of the waters of Marah, for they were bitter. Therefore the name of it was called Marah.}{\arabic{verse}}
\threeverse{\arabic{verse}}%Ex.15:24
{וַיִּלֹּ֧נוּ הָעָ֛ם עַל־מֹשֶׁ֥ה לֵּאמֹ֖ר מַה־נִּשְׁתֶּֽה׃
\rashi{\rashiDH{וילנו. }לשון נפעל הוא, וכן התרגום לשון נפעל הוא, וְאִתְרָעָמוּ, וכן דרך לשון תלונה להסב הדבור אל האדם, מתלונן, מתרועם, ולא אמר לונן, רועם, וכן יאמר הלועז דקומפ״ל ישנ״ק שי״י מוסב הדבור אליו באמרו שי״י׃ }}
{וְאִתְרָעַמוּ עַמָּא עַל מֹשֶׁה לְמֵימַר מָא נִשְׁתֵּי׃}
{And the people murmured against Moses, saying: ‘What shall we drink?’}{\arabic{verse}}
\threeverse{\arabic{verse}}%Ex.15:25
{וַיִּצְעַ֣ק אֶל־יְהֹוָ֗ה וַיּוֹרֵ֤הוּ יְהֹוָה֙ עֵ֔ץ וַיַּשְׁלֵךְ֙ אֶל־הַמַּ֔יִם וַֽיִּמְתְּק֖וּ הַמָּ֑יִם שָׁ֣ם שָׂ֥ם ל֛וֹ חֹ֥ק וּמִשְׁפָּ֖ט וְשָׁ֥ם נִסָּֽהוּ׃
\rashi{\rashiDH{שם שם לו. }במרה נתן להם מקצת פרשיות של תורה שיתעסקו בהם, שבת, ופרה אדומה, ודינין (סנהדרין נו׃)׃ }\rashi{\rashiDH{ושם נסהו. }לעם, וראה קשי ערפן, שלא נמלכו במשה בלשון יפה, בקש עלינו רחמים שיהיו לנו מים לשתות, אלא נתלוננו׃ }}
{וְצַלִּי קֳדָם יְיָ וְאַלְּפֵיהּ יְיָ אָעָא וּרְמָא לְמַיָּא וּבְסִימוּ מַיָּא תַּמָּן גְּזַר לֵיהּ קְיָם וְדִין וְתַמָּן נַסְּיֵיהּ׃}
{And he cried unto the \lord; and the \lord\space showed him a tree, and he cast it into the waters, and the waters were made sweet. There He made for them a statute and an ordinance, and there He proved them;}{\arabic{verse}}
\threeverse{\arabic{verse}}%Ex.15:26
{וַיֹּ֩אמֶר֩ אִם־שָׁמ֨וֹעַ תִּשְׁמַ֜ע לְק֣וֹל \legarmeh  יְהֹוָ֣ה אֱלֹהֶ֗יךָ וְהַיָּשָׁ֤ר בְּעֵינָיו֙ תַּעֲשֶׂ֔ה וְהַֽאֲזַנְתָּ֙ לְמִצְוֺתָ֔יו וְשָׁמַרְתָּ֖ כׇּל־חֻקָּ֑יו כׇּֽל־הַמַּחֲלָ֞ה אֲשֶׁר־שַׂ֤מְתִּי בְמִצְרַ֙יִם֙ לֹא־אָשִׂ֣ים עָלֶ֔יךָ כִּ֛י אֲנִ֥י יְהֹוָ֖ה רֹפְאֶֽךָ׃ \setuma         
\rashi{\rashiDH{אם שמוע תשמע. }זו קבלה שיקבלו עליהם׃}\rashi{\rashiDH{תעשה. }היא עשייה׃}\rashi{\rashiDH{והאזנת. }תטה אזנים לדקדק בהם׃}\rashi{\rashiDH{כל חקיו. }דברים שאינן אלא גזירת מלך בלא שום טעם, ויצר הרע מקנטר עליהם, מה איסור באלו, למה נאסרו, כגון לבישת כלאים ואכילת חזיר ופרה אדומה וכיוצא בהם׃ }\rashi{\rashiDH{לא אשים עליך. }ואם אשים, הרי הוא כלא הושמה, כי אני ה׳ רופאך (מכילתא ויסע פ״א), זהו מדרשו. ולפי פשוטו כי אני ה׳ רופאך, ומלמדך תורה ומצות למען תנצל מהם, כרופא הזה האומר לאדם אל תאכל דברים שמחזירים אותך לידי חולי, וזהו איזון מצות, וכן הוא אומר רִפְאוּת תְּהִי לְשָׁרֶךָ (משלי ג, ח)׃ }}
{וַאֲמַר אִם קַבָּלָא תְקַבֵּיל לְמֵימְרָא דַּייָ אֱלָהָךְ וּדְכָשַׁר קֳדָמוֹהִי תַּעֲבֵיד וּתְצִית לְפִקּוֹדוֹהִי וְתִטַּר כָּל קְיָמוֹהִי כָּל מַרְעִין דְּשַׁוִּיתִי בְּמִצְרַיִם לָא אֲשַׁוֵּינוּן עֲלָךְ אֲרֵי אֲנָא יְיָ אָסָךְ׃}
{and He said: ‘If thou wilt diligently hearken to the voice of the \lord\space thy God, and wilt do that which is right in His eyes, and wilt give ear to His commandments, and keep all His statutes, I will put none of the diseases upon thee, which I have put upon the Egyptians; for I am the \lord\space that healeth thee.’}{\arabic{verse}}
\threeverse{\aliya{חמישי}}%Ex.15:27
{וַיָּבֹ֣אוּ אֵילִ֔מָה וְשָׁ֗ם שְׁתֵּ֥ים עֶשְׂרֵ֛ה עֵינֹ֥ת מַ֖יִם וְשִׁבְעִ֣ים תְּמָרִ֑ים וַיַּחֲנוּ־שָׁ֖ם עַל־הַמָּֽיִם׃
\rashi{\rashiDH{שתים עשרה עינת מים. }כנגד י״ב שבטים נזדמנו להם׃}\rashi{\rashiDH{וע׳ תמרים.} כנגד שבעים זקנים׃}}
{וַאֲתוֹ לְאֵילִים וְתַמָּן תְּרֵי עֲסַר מַבּוּעִין דְּמַיִין וְשִׁבְעִין דִּקְלִין וּשְׁרוֹ תַּמָּן עַל מַיָּא׃}
{And they came to Elim, where were twelve springs of water, and three score and ten palm-trees; and they encamped there by the waters.}{\arabic{verse}}
\newperek
\threeverse{\Roman{chap}}%Ex.16:1
{וַיִּסְעוּ֙ מֵֽאֵילִ֔ם וַיָּבֹ֜אוּ כׇּל־עֲדַ֤ת בְּנֵֽי־יִשְׂרָאֵל֙ אֶל־מִדְבַּר־סִ֔ין אֲשֶׁ֥ר בֵּין־אֵילִ֖ם וּבֵ֣ין סִינָ֑י בַּחֲמִשָּׁ֨ה עָשָׂ֥ר יוֹם֙ לַחֹ֣דֶשׁ הַשֵּׁנִ֔י לְצֵאתָ֖ם מֵאֶ֥רֶץ מִצְרָֽיִם׃
\rashi{\rashiDH{בחמשה עשר יום. }נתפרש היום של חנייה זו, לפי שבו ביום כלתה החררה שהוציאו ממצרים והוצרכו למן, למדנו, שאכלו משירי הבצק (או משירי המצה) ששים ואחת סעודות, וירד להם מן בט״ז באייר, ויום א׳ בשבת היה, כדאיתא במסכת שבת (פז׃)׃ }}
{וּנְטַלוּ מֵאֵילִים וַאֲתוֹ כָּל כְּנִשְׁתָּא דִּבְנֵי יִשְׂרָאֵל לְמַדְבְּרָא דְּסִין דְּבֵין אֵילִים וּבֵין סִינָי בַּחֲמֵישְׁתְּ עַסְרָא יוֹמָא לְיַרְחָא תִּנְיָנָא לְמִפַּקְהוֹן מֵאַרְעָא דְּמִצְרָיִם׃}
{And they took their journey from Elim, and all the congregation of the children of Israel came unto the wilderness of Sin, which is between Elim and Sinai, on the fifteenth day of the second month after their departing out of the land of Egypt.}{\Roman{chap}}
\threeverse{\arabic{verse}}%Ex.16:2
{\qk{וַיִּלּ֜וֹנוּ}{וילינו} כׇּל־עֲדַ֧ת בְּנֵי־יִשְׂרָאֵ֛ל עַל־מֹשֶׁ֥ה וְעַֽל־אַהֲרֹ֖ן בַּמִּדְבָּֽר׃
\rashi{\rashiDH{וילונו. }לפי שכלה הלחם׃ 
}}
{וְאִתְרָעַמוּ כָּל כְּנִשְׁתָּא דִּבְנֵי יִשְׂרָאֵל עַל מֹשֶׁה וְעַל אַהֲרֹן בְּמַדְבְּרָא׃}
{And the whole congregation of the children of Israel murmured against Moses and against Aaron in the wilderness;}{\arabic{verse}}
\threeverse{\arabic{verse}}%Ex.16:3
{וַיֹּאמְר֨וּ אֲלֵהֶ֜ם בְּנֵ֣י יִשְׂרָאֵ֗ל מִֽי־יִתֵּ֨ן מוּתֵ֤נוּ בְיַד־יְהֹוָה֙ בְּאֶ֣רֶץ מִצְרַ֔יִם בְּשִׁבְתֵּ֙נוּ֙ עַל־סִ֣יר הַבָּשָׂ֔ר בְּאׇכְלֵ֥נוּ לֶ֖חֶם לָשֹׂ֑בַע כִּֽי־הוֹצֵאתֶ֤ם אֹתָ֙נוּ֙ אֶל־הַמִּדְבָּ֣ר הַזֶּ֔ה לְהָמִ֛ית אֶת־כׇּל־הַקָּהָ֥ל הַזֶּ֖ה בָּרָעָֽב׃ \setuma         
\rashi{\rashiDH{מי יתן מותנו. }שנמות, ואינו שם דבר כמו מותנו (בחולם), אלא כמו עשותנו, חנותנו, שובנו, לעשות אנחנו, לחנות אנחנו, למות אנחנו. ותרגומו לְוַי דְמִיתְנָא, לו מתנו, הלואי והיינו מתים׃ }}
{וַאֲמַרוּ לְהוֹן בְּנֵי יִשְׂרָאֵל לְוֵי דְּמֵיתְנָא קֳדָם יְיָ בְּאַרְעָא דְּמִצְרַיִם כַּד הֲוֵינָא יָתְבִין עַל דּוּדֵי בִשְׂרָא כַּד הֲוֵינָא אָכְלִין לַחְמָא וְסָבְעִין אֲרֵי אַפֵּיקְתּוּן יָתַנָא לְמַדְבְּרָא הָדֵין לְקַטָּלָא יָת כָּל קְהָלָא הָדֵין בְּכַפְנָא׃}
{and the children of Israel said unto them: ‘Would that we had died by the hand of the \lord\space in the land of Egypt, when we sat by the flesh-pots, when we did eat bread to the full; for ye have brought us forth into this wilderness, to kill this whole assembly with hunger.’}{\arabic{verse}}
\threeverse{\arabic{verse}}%Ex.16:4
{וַיֹּ֤אמֶר יְהֹוָה֙ אֶל־מֹשֶׁ֔ה הִנְנִ֨י מַמְטִ֥יר לָכֶ֛ם לֶ֖חֶם מִן־הַשָּׁמָ֑יִם וְיָצָ֨א הָעָ֤ם וְלָֽקְטוּ֙ דְּבַר־י֣וֹם בְּיוֹמ֔וֹ לְמַ֧עַן אֲנַסֶּ֛נּוּ הֲיֵלֵ֥ךְ בְּתוֹרָתִ֖י אִם־לֹֽא׃
\rashi{\rashiDH{דבר יום ביומו. }צורך אכילת יום ילקטו ביומו, ולא ילקטו היום לצורך מחר (מכילתא ויסע פ״ב)׃ }\rashi{\rashiDH{למען אנסנו הילך בתורתי. }אם ישמרו מצות התלויות בו, שלא יותירו ממנו, ולא יצאו בשבת ללקוט׃ }}
{וַאֲמַר יְיָ לְמֹשֶׁה הָאֲנָא מַחֵית לְכוֹן לַחְמָא מִן שְׁמַיָּא וְיִפְּקוּן עַמָּא וְיִלְקְטוּן פִּתְגָם יוֹם בְּיוֹמֵיהּ בְּדִיל דַּאֲנַסֵּינוּן הַיְהָכוּן בְּאוֹרָיְתִי אִם לָא׃}
{Then said the \lord\space unto Moses: ‘Behold, I will cause to rain bread from heaven for you; and the people shall go out and gather a day’s portion every day, that I may prove them, whether they will walk in My law, or not.}{\arabic{verse}}
\threeverse{\arabic{verse}}%Ex.16:5
{וְהָיָה֙ בַּיּ֣וֹם הַשִּׁשִּׁ֔י וְהֵכִ֖ינוּ אֵ֣ת אֲשֶׁר־יָבִ֑יאוּ וְהָיָ֣ה מִשְׁנֶ֔ה עַ֥ל אֲשֶֽׁר־יִלְקְט֖וּ י֥וֹם \pasek  יֽוֹם׃
\rashi{\rashiDH{והיה משנה. }ליום ולמחרת׃}\rashi{\rashiDH{משנה. }על שהיו רגילים ללקוט יום יום של שאר ימות השבוע. ואומר אני אשר יביאו והיה משנה, לאחר שיביאו ימצאו משנה במדידה, על אשר ילקטו וימדו יום יום, וזהו לָקְטוּ לֶחֶם מִשְׁנֶה, בלקיטתו היה נמצא לחם משנה, וזהו עַל כֵּן הוּא נֹתֵן לָכֶם בַּיֹּום הַשִּׁשִּׁי לֶחֶם יֹומָיִם, נותן לכם ברכה (פויש״ן) בבית, למלאות העומר פעמים ללחם יומים׃ 
}}
{וִיהֵי בְּיוֹמָא שְׁתִיתָאָה וִיתַקְּנוּן יָת דְּיַיְתוֹן וִיהֵי עַל חַד תְּרֵין עַל דְּיִלְקְטוּן יוֹם יוֹם׃}
{And it shall come to pass on the sixth day that they shall prepare that which they bring in, and it shall be twice as much as they gather daily.’}{\arabic{verse}}
\threeverse{\arabic{verse}}%Ex.16:6
{וַיֹּ֤אמֶר מֹשֶׁה֙ וְאַהֲרֹ֔ן אֶֽל־כׇּל־בְּנֵ֖י יִשְׂרָאֵ֑ל עֶ֕רֶב וִֽידַעְתֶּ֕ם כִּ֧י יְהֹוָ֛ה הוֹצִ֥יא אֶתְכֶ֖ם מֵאֶ֥רֶץ מִצְרָֽיִם׃
\rashi{\rashiDH{ערב. }כמו לערב׃}\rashi{\rashiDH{וידעתם כי ה׳ הוציא אתכם מארץ מצרים. }לפי שאמרתם לנו כי הוצאתם אותנו, תדעו כי לא אנחנו המוציאים, אלא ה׳ הוציא אתכם שיגיז לכם את השליו׃ }}
{וַאֲמַר מֹשֶׁה וְאַהֲרֹן לְכָל בְּנֵי יִשְׂרָאֵל בְּרַמְשָׁא וְתִדְּעוּן אֲרֵי יְיָ אַפֵּיק יָתְכוֹן מֵאַרְעָא דְּמִצְרָיִם׃}
{And Moses and Aaron said unto all the children of Israel: ‘At even, then ye shall know that the \lord\space hath brought you out from the land of Egypt;}{\arabic{verse}}
\threeverse{\arabic{verse}}%Ex.16:7
{וּבֹ֗קֶר וּרְאִיתֶם֙ אֶת־כְּב֣וֹד יְהֹוָ֔ה בְּשׇׁמְע֥וֹ אֶת־תְּלֻנֹּתֵיכֶ֖ם עַל־יְהֹוָ֑ה וְנַ֣חְנוּ מָ֔ה כִּ֥י \qk{תַלִּ֖ינוּ}{תלונו} עָלֵֽינוּ׃
\rashi{\rashiDH{ובקר וראיתם. }לא על הכבוד שנאמר וְהִנֵּה כְּבֹוד ה׳ נִרְאָה בֶּעָנָן נאמר, אלא כך אמר להם, ערב וידעתם כי היכולת בידו ליתן תאותכם, ובשר יתן, אך לא בפנים מאירות יתננה לכם, כי שלא כהוגן שאלתם אותו, ומכרס מלאה, והלחם ששאלתם לצורך בירידתו, לבקר תראו את כבוד אור פניו, שיורידוהו לכם דרך חבה בבקר, שיש שעות להכינו, וטל מלמעלה וטל מלמטה כמונח בקופסא׃ 
}\rashi{\rashiDH{תלנותיכם על ה׳. }כמו אשר על ה׳׃}\rashi{\rashiDH{ונחנו מה. }מה אנחנו חשובין׃}\rashi{\rashiDH{כי תלינו עלינו. }שתרעימו עלינו את הכל, את בניכם ונשיכם ובנותיכם וערב רב. ועל כרחי אני זקוק לפרש תלינו בלשון תפעילו, מפני דגשותו וקרייתו, שאילו היה רפה, הייתי מפרשו בלשון תפעילו, כמו וַיָּלֶן הָעָם עַל משֶׁה (שמות יז, ג), או אם היה דגוש ואין בו יו״ד ונקרא תלונו, הייתי מפרשו לשון תתלוננו, עכשיו הוא משמע תלינו את אחרים, כמו במרגלים וַיַּלִּינוּ עָלָיו אֶת כָּל הָעֵדָה (במדבר יד, ב)׃ }}
{וּבְצַפְרָא וְתִחְזוֹן יָת יְקָרָא דַּייָ בְּדִשְׁמִיעָן קֳדָמוֹהִי תּוּרְעֲמָתְכוֹן עַל יְיָ וְנַחְנָא מָא אֲרֵי מִתְרָעֲמִתּוּן עֲלַנָא׃}
{and in the morning, then ye shall see the glory of the \lord; for that He hath heard your murmurings against the \lord; and what are we, that ye murmur against us?’}{\arabic{verse}}
\threeverse{\arabic{verse}}%Ex.16:8
{וַיֹּ֣אמֶר מֹשֶׁ֗ה בְּתֵ֣ת יְהֹוָה֩ לָכֶ֨ם בָּעֶ֜רֶב בָּשָׂ֣ר לֶאֱכֹ֗ל וְלֶ֤חֶם בַּבֹּ֙קֶר֙ לִשְׂבֹּ֔עַ בִּשְׁמֹ֤עַ יְהֹוָה֙ אֶת־תְּלֻנֹּ֣תֵיכֶ֔ם אֲשֶׁר־אַתֶּ֥ם מַלִּינִ֖ם עָלָ֑יו וְנַ֣חְנוּ מָ֔ה לֹא־עָלֵ֥ינוּ תְלֻנֹּתֵיכֶ֖ם כִּ֥י עַל־יְהֹוָֽה׃
\rashi{\rashiDH{בשר לאכול. }ולא לשובע, למדה תורה דרך ארץ שאין אוכלין בשר לשובע. ומה ראה להוריד לחם בבקר ובשר בערב, לפי שהלחם שאלו כהוגן, שאי אפשר לו לאדם בלא לחם, אבל בשר שאלו שלא כהוגן, שהרבה בהמות היו להם, ועוד שהיה אפשר להם בלא בשר, לפיכך נתן להם בשעת טורח שלא כהוגן׃ 
}\rashi{\rashiDH{אשר אתם מלינים עליו. }את האחרים, השומעים אתכם מתלוננים׃ }}
{וַאֲמַר מֹשֶׁה בִּדְיִתֵּין יְיָ לְכוֹן בְּרַמְשָׁא בִּסְרָא לְמֵיכַל וְלַחְמָא בְּצַפְרָא לְמִסְבַּע בְּדִשְׁמִיעָן קֳדָם יְיָ תּוּרְעֲמָתְכוֹן דְּאַתּוּן מִתְרָעֲמִין עֲלוֹהִי וְנַחְנָא מָא לָא עֲלַנָא תּוּרְעֲמָתְכוֹן אֱלָהֵין עַל מֵימְרָא דַּייָ׃}
{And Moses said: ‘This shall be, when the \lord\space shall give you in the evening flesh to eat, and in the morning bread to the full; for that the \lord\space heareth your murmurings which ye murmur against Him; and what are we? your murmurings are not against us, but against the \lord.’}{\arabic{verse}}
\threeverse{\arabic{verse}}%Ex.16:9
{וַיֹּ֤אמֶר מֹשֶׁה֙ אֶֽל־אַהֲרֹ֔ן אֱמֹ֗ר אֶֽל־כׇּל־עֲדַת֙ בְּנֵ֣י יִשְׂרָאֵ֔ל קִרְב֖וּ לִפְנֵ֣י יְהֹוָ֑ה כִּ֣י שָׁמַ֔ע אֵ֖ת תְּלֻנֹּתֵיכֶֽם׃
\rashi{\rashiDH{קרבו. }למקום שהענן ירד׃}}
{וַאֲמַר מֹשֶׁה לְאַהֲרֹן אֵימַר לְכָל כְּנִשְׁתָּא דִּבְנֵי יִשְׂרָאֵל קְרוּבוּ קֳדָם יְיָ אֲרֵי שְׁמִיעָן קֳדָמוֹהִי תּוּרְעֲמָתְכוֹן׃}
{And Moses said unto Aaron: ‘Say unto all the congregation of the children of Israel: Come near before the \lord; for He hath heard your murmurings.’}{\arabic{verse}}
\threeverse{\arabic{verse}}%Ex.16:10
{וַיְהִ֗י כְּדַבֵּ֤ר אַהֲרֹן֙ אֶל־כׇּל־עֲדַ֣ת בְּנֵֽי־יִשְׂרָאֵ֔ל וַיִּפְנ֖וּ אֶל־הַמִּדְבָּ֑ר וְהִנֵּה֙ כְּב֣וֹד יְהֹוָ֔ה נִרְאָ֖ה בֶּעָנָֽן׃ \petucha }
{וַהֲוָה כַּד מַלֵּיל אַהֲרֹן עִם כָּל כְּנִשְׁתָּא דִּבְנֵי יִשְׂרָאֵל וְאִתְפְּנִיאוּ לְמַדְבְּרָא וְהָא יְקָרָא דַּייָ אִתְגְּלִי בַּעֲנָנָא׃}
{And it came to pass, as Aaron spoke unto the whole congregation of the children of Israel, that they looked toward the wilderness, and, behold, the glory of the \lord\space appeared in the cloud.}{\arabic{verse}}
\threeverse{\aliya{ששי}}%Ex.16:11
{וַיְדַבֵּ֥ר יְהֹוָ֖ה אֶל־מֹשֶׁ֥ה לֵּאמֹֽר׃}
{וּמַלֵּיל יְיָ עִם מֹשֶׁה לְמֵימַר׃}
{And the \lord\space spoke unto Moses, saying:}{\arabic{verse}}
\threeverse{\arabic{verse}}%Ex.16:12
{שָׁמַ֗עְתִּי אֶת־תְּלוּנֹּת֮ בְּנֵ֣י יִשְׂרָאֵל֒ דַּבֵּ֨ר אֲלֵהֶ֜ם לֵאמֹ֗ר בֵּ֤ין הָֽעַרְבַּ֙יִם֙ תֹּאכְל֣וּ בָשָׂ֔ר וּבַבֹּ֖קֶר תִּשְׂבְּעוּ־לָ֑חֶם וִֽידַעְתֶּ֕ם כִּ֛י אֲנִ֥י יְהֹוָ֖ה אֱלֹהֵיכֶֽם׃}
{שְׁמִיעַ קֳדָמַי יָת תּוּרְעֲמָת בְּנֵי יִשְׂרָאֵל מַלֵּיל עִמְּהוֹן לְמֵימַר בֵּין שִׁמְשַׁיָּא תֵּיכְלוּן בִּסְרָא וּבְצַפְרָא תִּסְבְּעוּן לַחְמָא וְתִדְּעוּן אֲרֵי אֲנָא יְיָ אֱלָהֲכוֹן׃}
{’I have heard the murmurings of the children of Israel. Speak unto them, saying: At dusk ye shall eat flesh, and in the morning ye shall be filled with bread; and ye shall know that I am the \lord\space your God.’}{\arabic{verse}}
\threeverse{\arabic{verse}}%Ex.16:13
{וַיְהִ֣י בָעֶ֔רֶב וַתַּ֣עַל הַשְּׂלָ֔ו וַתְּכַ֖ס אֶת־הַֽמַּחֲנֶ֑ה וּבַבֹּ֗קֶר הָֽיְתָה֙ שִׁכְבַ֣ת הַטַּ֔ל סָבִ֖יב לַֽמַּחֲנֶֽה׃
\rashi{\rashiDH{השליו. }מין עוף, ושמן מאד (יומא עה.׃)׃ }\rashi{\rashiDH{היתה שכבת הטל. }הטל שוכב על המן, ובמקום אחר הוא אומר וּבְרֶדֶת הַטַּל וגו׳ (במדבר יא, ט), הטל יורד על הארץ, והמן יורד עליו, וחוזר ויורד טל עליו, והרי הוא כמונח בקופסא (יומא עה׃  מכילתא ויסע פ״ג)׃ }}
{וַהֲוָה בְרַמְשָׁא וּסְלֵיקַת שְׂלָיו וַחֲפָת יָת מַשְׁרִיתָא וּבְצַפְרָא הֲוָת נָחֲתַת טַלָּא סְחוֹר סְחוֹר לְמַשְׁרִיתָא׃}
{And it came to pass at even, that the quails came up, and covered the camp; and in the morning there was a layer of dew round about the camp.}{\arabic{verse}}
\threeverse{\arabic{verse}}%Ex.16:14
{וַתַּ֖עַל שִׁכְבַ֣ת הַטָּ֑ל וְהִנֵּ֞ה עַל־פְּנֵ֤י הַמִּדְבָּר֙ דַּ֣ק מְחֻסְפָּ֔ס דַּ֥ק כַּכְּפֹ֖ר עַל־הָאָֽרֶץ׃
\rashi{\rashiDH{ותעל שכבת הטל וגו׳. }כשהחמה זורחת, עולה הטל שעל המן לקראת החמה כדרך טל עולה לקראת החמה, אף אם תמלא שפופרת של ביצה טל, ותסתום את פיה ותניחה בחמה, היא עולה מאליה באויר. ורבותינו דרשו, שהטל עולה מן הארץ באויר, וכעלות שכבת הטל נתגלה המן, וראו והנה על פני המדבר וגו׳׃ 
}\rashi{\rashiDH{דק. }דבר דק׃}\rashi{\rashiDH{מחוספס. }מגולה, ואין דומה לו במקרא, ויש לפרש מחוספס, לשון חפיסה וּדְלוּסְקְמָא שבלשון משנה, כשנתגלה משכבת הטל, ראו שהיה דבר דק מחוספס בתוכו בין שתי שכבות הטל. ואונקלוס תרגם מקלף, לשון מחשוף הלבן׃ }\rashi{\rashiDH{ככפור. }כפור גליד״א בלע״ז דַּעְדַק כגיר, כאבני גיר, והוא מין צבע שחור, כדאמרינן גבי כסוי הדם, הגיר וְהַזַּרְנִיךְ. דַּעְדַק כְּגִיר כִּגְלִידָא עַל אַרְעָא, דק היה כגיר ושוכב מוגלד כקרח על הארץ, וכן פירושו דק ככפור, שטוח קלוש ומחובר כגליד. דק טינב״ש בלע״ז שהיה מגליד גלד דק מלמעלה, וכגיר שתרגם אונקלוס, תוספת הוא על לשון העברית, ואין לו תיבה בפסוק׃ }}
{וּסְלֵיקַת נָחֲתַת טַלָּא וְהָא עַל אַפֵּי מַדְבְּרָא דַּעְדַּק מְקֻלַּף דַּעְדַּק דְּגִיר כִּגְלִידָא עַל אַרְעָא׃}
{And when the layer of dew was gone up, behold upon the face of the wilderness a fine, scale-like thing, fine as the hoar-frost on the ground.}{\arabic{verse}}
\threeverse{\arabic{verse}}%Ex.16:15
{וַיִּרְא֣וּ בְנֵֽי־יִשְׂרָאֵ֗ל וַיֹּ֨אמְר֜וּ אִ֤ישׁ אֶל־אָחִיו֙ מָ֣ן ה֔וּא כִּ֛י לֹ֥א יָדְע֖וּ מַה־ה֑וּא וַיֹּ֤אמֶר מֹשֶׁה֙ אֲלֵהֶ֔ם ה֣וּא הַלֶּ֔חֶם אֲשֶׁ֨ר נָתַ֧ן יְהֹוָ֛ה לָכֶ֖ם לְאׇכְלָֽה׃
\rashi{\rashiDH{מן הוא. }הכנת מזון הוא, כמו וַיְמַן לָהֶם הַמֶּלֶךְ (דניאל א, ה)׃ }\rashi{\rashiDH{כי לא ידעו מה הוא. }שיקראוהו בשמו׃}}
{וַחֲזוֹ בְנֵי יִשְׂרָאֵל וַאֲמַרוּ גְּבַר לְאַחוּהִי מַנָּא הוּא אֲרֵי לָא יָדְעִין מָא הוּא וַאֲמַר מֹשֶׁה לְהוֹן הוּא לַחְמָא דִּיהַב יְיָ לְכוֹן לְמֵיכַל׃}
{And when the children of Israel saw it, they said one to another: a‘What is it?’—for they knew not what it was. And Moses said unto them: ‘It is the bread which the \lord\space hath given you to eat.}{\arabic{verse}}
\threeverse{\arabic{verse}}%Ex.16:16
{זֶ֤ה הַדָּבָר֙ אֲשֶׁ֣ר צִוָּ֣ה יְהֹוָ֔ה לִקְט֣וּ מִמֶּ֔נּוּ אִ֖ישׁ לְפִ֣י אׇכְל֑וֹ עֹ֣מֶר לַגֻּלְגֹּ֗לֶת מִסְפַּר֙ נַפְשֹׁ֣תֵיכֶ֔ם אִ֛ישׁ לַאֲשֶׁ֥ר בְּאׇהֳל֖וֹ תִּקָּֽחוּ׃
\rashi{\rashiDH{עומר. }שם מדה׃}\rashi{\rashiDH{מספר נפשותיכם. }כפי מנין נפשות שיש לאיש באהלו, תקחו עומר לכל גולגולת׃ }}
{דֵּין פִּתְגָמָא דְּפַקֵּיד יְיָ לְקוּטוּ מִנֵּיהּ גְּבַר לְפוֹם מֵיכְלֵיהּ עוּמְרָא לְגוּלְגּוּלְתָּא מִנְיַן נַפְשָׁתְכוֹן גְּבַר לְדִבְמַשְׁכְּנֵיהּ תִּסְּבוּן׃}
{This is the thing which the \lord\space hath commanded: Gather ye of it every man according to his eating; an omer a head, according to the number of your persons, shall ye take it, every man for them that are in his tent.’}{\arabic{verse}}
\threeverse{\arabic{verse}}%Ex.16:17
{וַיַּעֲשׂוּ־כֵ֖ן בְּנֵ֣י יִשְׂרָאֵ֑ל וַֽיִּלְקְט֔וּ הַמַּרְבֶּ֖ה וְהַמַּמְעִֽיט׃
\rashi{\rashiDH{המרבה והממעיט. }יש שלקטו הרבה ויש שלקטו מעט, וכשבאו לביתם, ומדדו בעומר איש איש מה שלקטו, ומצאו שהמרבה ללקוט לא העדיף על עומר לגולגולת אשר באהלו, והממעיט ללקוט לא מצא חסר מעומר לגולגולת, וזהו נס גדול שנעשה בו׃ 
}}
{וַעֲבַדוּ כֵּן בְּנֵי יִשְׂרָאֵל וּלְקַטוּ דְּאַסְגִּי וּדְאַזְעַר׃}
{And the children of Israel did so, and gathered some more, some less.}{\arabic{verse}}
\threeverse{\arabic{verse}}%Ex.16:18
{וַיָּמֹ֣דּוּ בָעֹ֔מֶר וְלֹ֤א הֶעְדִּיף֙ הַמַּרְבֶּ֔ה וְהַמַּמְעִ֖יט לֹ֣א הֶחְסִ֑יר אִ֥ישׁ לְפִֽי־אׇכְל֖וֹ לָקָֽטוּ׃}
{וְכָלוּ בְעוֹמְרָא וְלָא אוֹתַר דְּאַסְגִּי וּדְאַזְעַר לָא חֲסַר גְּבַר לְפוֹם מֵיכְלֵיהּ לְקַטוּ׃}
{And when they did mete it with an omer, he that gathered much had nothing over, and he that gathered little had no lack; they gathered every man according to his eating.}{\arabic{verse}}
\threeverse{\arabic{verse}}%Ex.16:19
{וַיֹּ֥אמֶר מֹשֶׁ֖ה אֲלֵהֶ֑ם אִ֕ישׁ אַל־יוֹתֵ֥ר מִמֶּ֖נּוּ עַד־בֹּֽקֶר׃}
{וַאֲמַר מֹשֶׁה לְהוֹן אֱנָשׁ לָא יַשְׁאַר מִנֵּיהּ עַד צַפְרָא׃}
{And Moses said unto them: ‘Let no man leave of it till the morning.’}{\arabic{verse}}
\threeverse{\arabic{verse}}%Ex.16:20
{וְלֹא־שָׁמְע֣וּ אֶל־מֹשֶׁ֗ה וַיּוֹתִ֨רוּ אֲנָשִׁ֤ים מִמֶּ֙נּוּ֙ עַד־בֹּ֔קֶר וַיָּ֥רֻם תּוֹלָעִ֖ים וַיִּבְאַ֑שׁ וַיִּקְצֹ֥ף עֲלֵהֶ֖ם מֹשֶֽׁה׃
\rashi{\rashiDH{ויותירו אנשים. }דתן ואבירם (שמו״ר כה, י)׃ }\rashi{\rashiDH{וירם תולעים. }לשון רמה׃}\rashi{\rashiDH{ויבאש. }הרי זה מקרא הפוך, שבתחלה הבאיש ולבסוף התליע, כענין שנאמר וְלֹּא הִבְאִישׁ וְרִמָּה לֹא הָיְתָה בֹּו, וכן דרך כל המתליעים׃ }}
{וְלָא קַבִּילוּ מִן מֹשֶׁה וְאַשְׁאַרוּ גֻּבְרַיָּא מִנֵּיהּ עַד צַפְרָא וּרְחֵישׁ רִחְשָׁא וּסְרִי וּרְגֵיז עֲלֵיהוֹן מֹשֶׁה׃}
{Notwithstanding they hearkened not unto Moses; but some of them left of it until the morning, and it bred worms, and rotted; and Moses was wroth with them.}{\arabic{verse}}
\threeverse{\arabic{verse}}%Ex.16:21
{וַיִּלְקְט֤וּ אֹתוֹ֙ בַּבֹּ֣קֶר בַּבֹּ֔קֶר אִ֖ישׁ כְּפִ֣י אׇכְל֑וֹ וְחַ֥ם הַשֶּׁ֖מֶשׁ וְנָמָֽס׃
\rashi{\rashiDH{וחם השמש ונמס. }הנשאר בשדה נמוח ונעשה נחלים, ושותין ממנו אילים וצבאים, ואומות העולם צדין מהם וטועמים בהם טעם מן (מכילתא ויסע פ״ד), ויודעים מה שבחן של ישראל. ונמס, פשר, לשון פושרים, ע״י השמש מתחמם ומפשיר׃ }\rashi{\rashiDH{ונמס. }דישטנ״פריר (צו גיין צו שמעלצען), ודוגמתו בסנהדרין בסוף ד׳ מיתות (סז׃)׃ }}
{וּלְקַטוּ יָתֵיהּ בִּצְפַר בִּצְפַר גְּבַר כְּפוֹם מֵיכְלֵיהּ וּמָא דְּמִשְׁתְּאַר מִנֵּיהּ עַל אַפֵּי חַקְלָא כַּד חֲמָא עֲלוֹהִי שִׁמְשָׁא פָּשַׁר׃}
{And they gathered it morning by morning, every man according to his eating; and as the sun waxed hot, it melted.}{\arabic{verse}}
\threeverse{\arabic{verse}}%Ex.16:22
{וַיְהִ֣י \legarmeh  בַּיּ֣וֹם הַשִּׁשִּׁ֗י לָֽקְט֥וּ לֶ֙חֶם֙ מִשְׁנֶ֔ה שְׁנֵ֥י הָעֹ֖מֶר לָאֶחָ֑ד וַיָּבֹ֙אוּ֙ כׇּל־נְשִׂיאֵ֣י הָֽעֵדָ֔ה וַיַּגִּ֖ידוּ לְמֹשֶֽׁה׃
\rashi{\rashiDH{לקטו לחם משנה. }כשמדדו את לקיטתם באהליהם, מצאו כפלים שני העומר לאחד. ומדרש אגדה, לחם משונה, אותו היום נשתנה לשבח בריחו וטעמו (מכילתא ויסע פ״ב), (שאם להגיד ששנים היו והלא כתיב שני העומר לאחד, אלא משונה בטעם וריח)׃ }\rashi{\rashiDH{ויגידו למשה. }שאלוהו מה היום מיומים, ומכאן יש ללמוד שעדיין לא הגיד להם משה פרשת שבת, שנצטוה לומר להם וְהָיָה בַּיֹּום הַשִּׁשִּׁי וְהֵכִינוּ וגו׳, עד ששאלו את זאת אמר להם הוא אשר דבר ה׳ שנצטויתי לומר לכם, ולכך ענשו הכתוב, שאמר לו עַד אָנָה מֵאַנְתֶּם, ולא הוציאו מן הכלל׃ }}
{וַהֲוָה בְּיוֹמָא שְׁתִיתָאָה לְקַטוּ לַחְמָא עַל חַד תְּרֵין תְּרֵין עוּמְרִין לְחַד וַאֲתוֹ כָּל רַבְרְבֵי כְּנִשְׁתָּא וְחַוִּיאוּ לְמֹשֶׁה׃}
{And it came to pass that on the sixth day they gathered twice as much bread, two omers for each one; and all the rulers of the congregation came and told Moses.}{\arabic{verse}}
\threeverse{\arabic{verse}}%Ex.16:23
{וַיֹּ֣אמֶר אֲלֵהֶ֗ם ה֚וּא אֲשֶׁ֣ר דִּבֶּ֣ר יְהֹוָ֔ה שַׁבָּת֧וֹן שַׁבַּת־קֹ֛דֶשׁ לַֽיהֹוָ֖ה מָחָ֑ר אֵ֣ת אֲשֶׁר־תֹּאפ֞וּ אֵפ֗וּ וְאֵ֤ת אֲשֶֽׁר־תְּבַשְּׁלוּ֙ בַּשֵּׁ֔לוּ וְאֵת֙ כׇּל־הָ֣עֹדֵ֔ף הַנִּ֧יחוּ לָכֶ֛ם לְמִשְׁמֶ֖רֶת עַד־הַבֹּֽקֶר׃
\rashi{\rashiDH{את אשר תאפו אפו. }מה שאתם רוצים לאפות בתנור, אֱפוּ היום הכל לשני ימים, ומה שאתם צריכים לבשל ממנו במים, בשלו היום. לשון אפייה נופל בלחם ולשון בישול בתבשיל׃ }\rashi{\rashiDH{למשמרת. }לגניזה׃}}
{וַאֲמַר לְהוֹן הוּא דְּמַלֵּיל יְיָ שַׁבָּא שַׁבָּתָא קוּדְשָׁא קֳדָם יְיָ מְחַר יָת דְּאַתּוּן עֲתִידִין לְמֵיפָא אֵיפוֹ וְיָת דְּאַתּוּן עֲתִידִין לְבַשָּׁלָא בַּשִּׁילוּ וְיָת כָּל מוֹתָרָא אַצְנַעוּ לְכוֹן לְמַטְּרָא עַד צַפְרָא׃}
{And he said unto them: ‘This is that which the \lord\space hath spoken: To-morrow is a solemn rest, a holy sabbath unto the \lord. Bake that which ye will bake, and seethe that which ye will seethe; and all that remaineth over lay up for you to be kept until the morning.’}{\arabic{verse}}
\threeverse{\arabic{verse}}%Ex.16:24
{וַיַּנִּ֤יחוּ אֹתוֹ֙ עַד־הַבֹּ֔קֶר כַּאֲשֶׁ֖ר צִוָּ֣ה מֹשֶׁ֑ה וְלֹ֣א הִבְאִ֔ישׁ וְרִמָּ֖ה לֹא־הָ֥יְתָה בּֽוֹ׃}
{וְאַצְנַעוּ יָתֵיהּ עַד צַפְרָא כְּמָא דְּפַקֵּיד מֹשֶׁה וְלָא סְרִי וְרִחְשָׁא לָא הֲוָת בֵּיהּ׃}
{And they laid it up till the morning, as Moses bade; and it did not rot, neither was there any worm therein.}{\arabic{verse}}
\threeverse{\arabic{verse}}%Ex.16:25
{וַיֹּ֤אמֶר מֹשֶׁה֙ אִכְלֻ֣הוּ הַיּ֔וֹם כִּֽי־שַׁבָּ֥ת הַיּ֖וֹם לַיהֹוָ֑ה הַיּ֕וֹם לֹ֥א תִמְצָאֻ֖הוּ בַּשָּׂדֶֽה׃
\rashi{\rashiDH{ויאמר משה אכלהו היום וגו׳. }שחרית שהיו רגילין לצאת וללקוט, באו לשאול אם נצא אם לאו, אמר להם את שבידכם אכלו. לערב חזרו לפניו ושאלוהו מהו לצאת, אמר להם שבת היום, ראה אותם דואגים שמא פסק המן ולא ירד עוד, אמר להם היום לא תמצאוהו, מה תלמוד לומר היום, היום לא תמצאוהו אבל מחר תמצאוהו (מכילתא ויסע פ״ד)׃ }}
{וַאֲמַר מֹשֶׁה אִכְלוּהִי יוֹמָא דֵין אֲרֵי שַׁבְּתָא יוֹמָא דֵין קֳדָם יְיָ יוֹמָא דֵין לָא תַּשְׁכְּחוּנֵּיהּ בְּחַקְלָא׃}
{And Moses said: ‘Eat that to-day; for to-day is a sabbath unto the \lord; to-day ye shall not find it in the field.}{\arabic{verse}}
\threeverse{\arabic{verse}}%Ex.16:26
{שֵׁ֥שֶׁת יָמִ֖ים תִּלְקְטֻ֑הוּ וּבַיּ֧וֹם הַשְּׁבִיעִ֛י שַׁבָּ֖ת לֹ֥א יִֽהְיֶה־בּֽוֹ׃
\rashi{\rashiDH{וביום השביעי שבת. }שבת הוא, המן לא יהיה בו. ולא בא הכתוב אלא לרבות יום הכפורים וימים טובים (מכילתא שם)׃ 
}}
{שִׁתָּא יוֹמִין תִּלְקְטוּנֵּיהּ וּבְיוֹמָא שְׁבִיעָאָה שַׁבְּתָא לָא יְהֵי בֵיהּ׃}
{Six days ye shall gather it; but on the seventh day is the sabbath, in it there shall be none.’}{\arabic{verse}}
\threeverse{\arabic{verse}}%Ex.16:27
{וַֽיְהִי֙ בַּיּ֣וֹם הַשְּׁבִיעִ֔י יָצְא֥וּ מִן־הָעָ֖ם לִלְקֹ֑ט וְלֹ֖א מָצָֽאוּ׃ \setuma         }
{וַהֲוָה בְּיוֹמָא שְׁבִיעָאָה נְפַקוּ מִן עַמָּא לְמִלְקַט וְלָא אַשְׁכַּחוּ׃}
{And it came to pass on the seventh day, that there went out some of the people to gather, and they found none.}{\arabic{verse}}
\threeverse{\arabic{verse}}%Ex.16:28
{וַיֹּ֥אמֶר יְהֹוָ֖ה אֶל־מֹשֶׁ֑ה עַד־אָ֙נָה֙ מֵֽאַנְתֶּ֔ם לִשְׁמֹ֥ר מִצְוֺתַ֖י וְתוֹרֹתָֽי׃
\rashi{\rashiDH{עד אנה מאנתם. }משל הדיוט הוא, בַּהֲדֵי הוּצָא לָקֵי כַּרְבָּא (ב״ק צב.), ע״י הרשעים מתגנין הכשרין׃ }}
{וַאֲמַר יְיָ לְמֹשֶׁה עַד אִמַּתִּי אַתּוּן מְסָרְבִין לְמִטַּר פִּקּוֹדַי וְאוֹרָיְתָי׃}
{And the \lord\space said unto Moses: ‘How long refuse ye to keep My commandments and My laws?}{\arabic{verse}}
\threeverse{\arabic{verse}}%Ex.16:29
{רְא֗וּ כִּֽי־יְהֹוָה֮ נָתַ֣ן לָכֶ֣ם הַשַּׁבָּת֒ עַל־כֵּ֠ן ה֣וּא נֹתֵ֥ן לָכֶ֛ם בַּיּ֥וֹם הַשִּׁשִּׁ֖י לֶ֣חֶם יוֹמָ֑יִם שְׁב֣וּ \legarmeh  אִ֣ישׁ תַּחְתָּ֗יו אַל־יֵ֥צֵא אִ֛ישׁ מִמְּקֹמ֖וֹ בַּיּ֥וֹם הַשְּׁבִיעִֽי׃
\rashi{\rashiDH{ראו. }בעיניכם כי ה׳ בכבודו מזהיר אתכם על השבת, שהרי נס נעשה בכל ערב שבת, לתת לכם לחם יומים׃ 
}\rashi{\rashiDH{שבו איש תחתיו. }מכאן סמכו חכמים ד׳ אמות ליוצא חוץ לתחום, ג׳ לגופו, וא׳ לפישוט ידים ורגלים׃ }\rashi{\rashiDH{אל יצא איש ממקומו. }אלו אלפים אמה של תחום שבת, ולא במפורש, שאין תחומין אלא מדברי סופרים, ועיקרו של מקרא על לוקטי המן נאמר׃ 
}}
{חֲזוֹ אֲרֵי יְיָ יְהַב לְכוֹן שַׁבְּתָא עַל כֵּן הוּא יָהֵיב לְכוֹן בְּיוֹמָא שְׁתִיתָאָה לְחֵים תְּרֵין יוֹמִין תִּיבוּ אֱנָשׁ תְּחוֹתוֹהִי לָא יִפּוֹק אֱנָשׁ מֵאַתְרֵיהּ בְּיוֹמָא שְׁבִיעָאָה׃}
{See that the \lord\space hath given you the sabbath; therefore He giveth you on the sixth day the bread of two days; abide ye every man in his place, let no man go out of his place on the seventh day.’}{\arabic{verse}}
\threeverse{\arabic{verse}}%Ex.16:30
{וַיִּשְׁבְּת֥וּ הָעָ֖ם בַּיּ֥וֹם הַשְּׁבִעִֽי׃}
{וּשְׁבַתוּ עַמָּא בְּיוֹמָא שְׁבִיעָאָה׃}
{So the people rested on the seventh day.}{\arabic{verse}}
\threeverse{\arabic{verse}}%Ex.16:31
{וַיִּקְרְא֧וּ בֵֽית־יִשְׂרָאֵ֛ל אֶת־שְׁמ֖וֹ מָ֑ן וְה֗וּא כְּזֶ֤רַע גַּד֙ לָבָ֔ן וְטַעְמ֖וֹ כְּצַפִּיחִ֥ת בִּדְבָֽשׁ׃
\rashi{\rashiDH{והוא כזרע גד לבן. }עשב ששמו קוליינד״רי (קאריאדער) וזרע שלו עגול ואינו לבן, והמן היה לבן, ואינו נמשל לזרע גד אלא לענין העגול כזרע גד היה, והוא לבן׃ }\rashi{\rashiDH{כצפיחת. }בצק שמטגנין אותו בדבש, וקורין לו אַסְקְרִיטוּן בלשון משנה, והוא תרגום של אונקלוס׃ }}
{וּקְרוֹ בֵית יִשְׂרָאֵל יָת שְׁמֵיהּ מַנָּא וְהוּא כְּבַר זְרַע גִּדָּא חִיוָר וְטַעְמֵיהּ כְּאִסְקְרִיטָוָן בִּדְבַשׁ׃}
{And the house of Israel called the name thereof Manna; and it was like coriander seed, white; and the taste of it was like wafers made with honey.}{\arabic{verse}}
\threeverse{\arabic{verse}}%Ex.16:32
{וַיֹּ֣אמֶר מֹשֶׁ֗ה זֶ֤ה הַדָּבָר֙ אֲשֶׁ֣ר צִוָּ֣ה יְהֹוָ֔ה מְלֹ֤א הָעֹ֙מֶר֙ מִמֶּ֔נּוּ לְמִשְׁמֶ֖רֶת לְדֹרֹתֵיכֶ֑ם לְמַ֣עַן \legarmeh  יִרְא֣וּ אֶת־הַלֶּ֗חֶם אֲשֶׁ֨ר הֶאֱכַ֤לְתִּי אֶתְכֶם֙ בַּמִּדְבָּ֔ר בְּהוֹצִיאִ֥י אֶתְכֶ֖ם מֵאֶ֥רֶץ מִצְרָֽיִם׃
\rashi{\rashiDH{למשמרת. }לגניזה׃}\rashi{\rashiDH{לדורותיכם. }בימי ירמיהו. כשהיה ירמיהו מוכיחם למה אין אתם עוסקים בתורה, והם אומרים נניח מלאכתנו ונעסוק בתורה מהיכן נתפרנס, הוציא להם צנצנת המן, אמר להם אַתֶּם רְאוּ דְבַר ה׳ (ירמיהו ב, לא), שמעו לא נאמר, אלא ראו, בזה נתפרנסו אבותיכם, הרבה שלוחין יש לו למקום להכין מזון ליראיו׃ 
}}
{וַאֲמַר מֹשֶׁה דֵּין פִּתְגָמָא דְּפַקֵּיד יְיָ מְלֵי עוּמְרָא מִנֵּיהּ לְמַטְּרָא לְדָרֵיכוֹן בְּדִיל דְּיִחְזוֹן יָת לַחְמָא דְּאוֹכֵילִית יָתְכוֹן בְּמַדְבְּרָא בְּאַפָּקוּתִי יָתְכוֹן מֵאַרְעָא דְּמִצְרָיִם׃}
{And Moses said: ‘This is the thing which the \lord\space hath commanded: Let an omerful of it be kept throughout your generations; that they may see the bread wherewith I fed you in the wilderness, when I brought you forth from the land of Egypt.’}{\arabic{verse}}
\threeverse{\arabic{verse}}%Ex.16:33
{וַיֹּ֨אמֶר מֹשֶׁ֜ה אֶֽל־אַהֲרֹ֗ן קַ֚ח צִנְצֶ֣נֶת אַחַ֔ת וְתֶן־שָׁ֥מָּה מְלֹֽא־הָעֹ֖מֶר מָ֑ן וְהַנַּ֤ח אֹתוֹ֙ לִפְנֵ֣י יְהֹוָ֔ה לְמִשְׁמֶ֖רֶת לְדֹרֹתֵיכֶֽם׃
\rashi{\rashiDH{צנצנת. }צלוחית של חרס כתרגומו׃}\rashi{\rashiDH{והנח אותו לפני ד׳. }לפני הארון, ולא נאמר מקרא זה עד שנבנה אהל מועד, אלא שנכתב כאן בפרשת המן׃ }}
{וַאֲמַר מֹשֶׁה לְאַהֲרֹן סַב צְלוֹחִית חֲדָא וְהַב תַּמָּן מְלֵי עוּמְרָא מַנָּא וְאַצְנַע יָתֵיהּ קֳדָם יְיָ לְמַטְּרָא לְדָרֵיכוֹן׃}
{And Moses said unto Aaron: ‘Take a jar, and put an omerful of manna therein, and lay it up before the \lord, to be kept throughout your generations.’}{\arabic{verse}}
\threeverse{\arabic{verse}}%Ex.16:34
{כַּאֲשֶׁ֛ר צִוָּ֥ה יְהֹוָ֖ה אֶל־מֹשֶׁ֑ה וַיַּנִּיחֵ֧הוּ אַהֲרֹ֛ן לִפְנֵ֥י הָעֵדֻ֖ת לְמִשְׁמָֽרֶת׃}
{כְּמָא דְּפַקֵּיד יְיָ לְמֹשֶׁה וְאַצְנְעֵיהּ אַהֲרֹן קֳדָם סָהֲדוּתָא לְמַטְּרָא׃}
{As the \lord\space commanded Moses, so Aaron laid it up before the Testimony, to be kept.}{\arabic{verse}}
\threeverse{\arabic{verse}}%Ex.16:35
{וּבְנֵ֣י יִשְׂרָאֵ֗ל אָֽכְל֤וּ אֶת־הַמָּן֙ אַרְבָּעִ֣ים שָׁנָ֔ה עַד־בֹּאָ֖ם אֶל־אֶ֣רֶץ נוֹשָׁ֑בֶת אֶת־הַמָּן֙ אָֽכְל֔וּ עַד־בֹּאָ֕ם אֶל־קְצֵ֖ה אֶ֥רֶץ כְּנָֽעַן׃
\rashi{\rashiDH{ארבעים שנה. }והלא חסר ל׳ יום, שהרי בט״ו באייר ירד להם המן תחלה, ובט״ו בניסן פסק, שנאמר וַיִּשְׁבֹּות הַמָּן מִמָּחֳרָת (יהושע ה, יב), אלא מגיד שהעוגות שהוציאו ישראל ממצרים טעמו בהם טעם מן׃ 
}\rashi{\rashiDH{אל ארץ נושבת. }לאחר שעברו את הירדן (קידושין לח.). (ס״א, שאותה שבעבר הירדן מיושבת וטובה, שנאמר אֶעְבְּרָה נָּא וְאֶרְאֶה אֶת הָאָרֶץ הַטֹּובָה אֲשֶׁר בְּעֵבֶר הַיַרְדְּן (דברים ג, כה), ותרגום של נושבת יָתְבָתָא, ר״ל מיושבת. רש״י ישן)׃ }\rashi{\rashiDH{אל קצה ארץ כנען. }בתחלת הגבול, קודם שעברו את הירדן והוא ערבות מואב, נמצאו מכחישין זה את זה, אלא בערבות מואב כשמת משה בז׳ באדר פסק המן מלירד, ונסתפקו ממן שלקטו בו ביום, עד שהקריבו העומר בששה עשר בניסן, שנאמר וַיֹאכְלוּ מֵעֲבוּר הָאָרֶץ מִמָּחֳרַת הַפֶּסַח (יהושע ה, יא)׃ }}
{וּבְנֵי יִשְׂרָאֵל אֲכַלוּ יָת מַנָּא אַרְבְּעִין שְׁנִין עַד דְּעָאלוּ לַאֲרַע יָתֵיבְתָּא יָת מַנָּא אֲכַלוּ עַד דַּאֲתוֹ לִסְיָפֵי אַרְעָא דִּכְנָעַן׃}
{And the children of Israel did eat the manna forty years, until they came to a land inhabited; they did eat the manna, until they came unto the borders of the land of Canaan.}{\arabic{verse}}
\threeverse{\arabic{verse}}%Ex.16:36
{וְהָעֹ֕מֶר עֲשִׂרִ֥ית הָאֵיפָ֖ה הֽוּא׃ \petucha 
\rashi{\rashiDH{עשירית האיפה. }האיפה שלש סאין, והסאה ו׳ קבין, והקב ד׳ לוגין, והלוג ששה ביצים, נמצא עשירית האיפה מ״ג ביצים וחומש ביצה, והוא שיעור לחלה ולמנחות׃ 
}}
{וְעוּמְרָא חַד מִן עַשְׂרָא בִּתְלָת סְאִין הוּא׃}
{Now an omer is the tenth part of an ephah.}{\arabic{verse}}
\newperek
\threeverse{\aliya{שביעי}}%Ex.17:1
{וַ֠יִּסְע֠וּ כׇּל־עֲדַ֨ת בְּנֵֽי־יִשְׂרָאֵ֧ל מִמִּדְבַּר־סִ֛ין לְמַסְעֵיהֶ֖ם עַל־פִּ֣י יְהֹוָ֑ה וַֽיַּחֲנוּ֙ בִּרְפִידִ֔ים וְאֵ֥ין מַ֖יִם לִשְׁתֹּ֥ת הָעָֽם׃}
{וּנְטַלוּ כָּל כְּנִשְׁתָּא דִּבְנֵי יִשְׂרָאֵל מִמַּדְבְּרָא דְּסִין לְמַטְּלָנֵיהוֹן עַל מֵימְרָא דַּייָ וּשְׁרוֹ בִּרְפִידִים וְלֵית מַיָּא לְמִשְׁתֵּי עַמָּא׃}
{And all the congregation of the children of Israel journeyed from the wilderness of Sin, by their stages, according to the commandment of the \lord, and encamped in Rephidim; and there was no water for the people to drink.}{\Roman{chap}}
\threeverse{\arabic{verse}}%Ex.17:2
{וַיָּ֤רֶב הָעָם֙ עִם־מֹשֶׁ֔ה וַיֹּ֣אמְר֔וּ תְּנוּ־לָ֥נוּ מַ֖יִם וְנִשְׁתֶּ֑ה וַיֹּ֤אמֶר לָהֶם֙ מֹשֶׁ֔ה מַה־תְּרִיבוּן֙ עִמָּדִ֔י מַה־תְּנַסּ֖וּן אֶת־יְהֹוָֽה׃
\rashi{\rashiDH{מה תנסון. }לומר היוכל לתת מים בארץ ציה׃}}
{וּנְצָא עַמָּא עִם מֹשֶׁה וַאֲמַרוּ הַבוּ לַנָא מַיָּא וְנִשְׁתֵּי וַאֲמַר לְהוֹן מֹשֶׁה מָא נָצַן אַתּוּן עִמִּי מָא מְנַסַּן אַתּוּן קֳדָם יְיָ׃}
{Wherefore the people strove with Moses, and said: ‘Give us water that we may drink.’ And Moses said unto them: ‘Why strive ye with me? wherefore do ye try the \lord?’}{\arabic{verse}}
\threeverse{\arabic{verse}}%Ex.17:3
{וַיִּצְמָ֨א שָׁ֤ם הָעָם֙ לַמַּ֔יִם וַיָּ֥לֶן הָעָ֖ם עַל־מֹשֶׁ֑ה וַיֹּ֗אמֶר לָ֤מָּה זֶּה֙ הֶעֱלִיתָ֣נוּ מִמִּצְרַ֔יִם לְהָמִ֥ית אֹתִ֛י וְאֶת־בָּנַ֥י וְאֶת־מִקְנַ֖י בַּצָּמָֽא׃}
{וּצְהִי תַּמָּן עַמָּא לְמַיָּא וְאִתְרָעַם עַמָּא עַל מֹשֶׁה וַאֲמַרוּ לְמָא דְנָן אַסֵּיקְתַּנָא מִמִּצְרַיִם לְקַטָּלָא יָתִי וְיָת בְּנַי וְיָת בְּעִירַי בְּצָהוּתָא׃}
{And the people thirsted there for water; and the people murmured against Moses, and said: ‘Wherefore hast thou brought us up out of Egypt, to kill us and our children and our cattle with thirst?’}{\arabic{verse}}
\threeverse{\arabic{verse}}%Ex.17:4
{וַיִּצְעַ֤ק מֹשֶׁה֙ אֶל־יְהֹוָ֣ה לֵאמֹ֔ר מָ֥ה אֶעֱשֶׂ֖ה לָעָ֣ם הַזֶּ֑ה ע֥וֹד מְעַ֖ט וּסְקָלֻֽנִי׃
\rashi{\rashiDH{עוד מעט. }אם אמתין, עוד מעט וסקלוני׃ }}
{וְצַלִּי מֹשֶׁה קֳדָם יְיָ לְמֵימַר מָא אַעֲבֵיד לְעַמָּא הָדֵין עוֹד זְעֵיר פּוֹן וְרַגְמוּנִי׃}
{And Moses cried unto the \lord, saying: ‘What shall I do unto this people? they are almost ready to stone me.’}{\arabic{verse}}
\threeverse{\arabic{verse}}%Ex.17:5
{וַיֹּ֨אמֶר יְהֹוָ֜ה אֶל־מֹשֶׁ֗ה עֲבֹר֙ לִפְנֵ֣י הָעָ֔ם וְקַ֥ח אִתְּךָ֖ מִזִּקְנֵ֣י יִשְׂרָאֵ֑ל וּמַטְּךָ֗ אֲשֶׁ֨ר הִכִּ֤יתָ בּוֹ֙ אֶת־הַיְאֹ֔ר קַ֥ח בְּיָדְךָ֖ וְהָלָֽכְתָּ׃
\rashi{\rashiDH{עבור לפני העם. }וראה אם יסקלוך, למה הוצאת לעז על בני׃ }\rashi{\rashiDH{וקח אתך מזקני ישראל. }לעדות, שיראו שעל ידך המים יוצאים מן הצור, ולא יאמרו מעינות היו שם מימי קדם׃ 
}\rashi{\rashiDH{ומטך אשר הכית בו את היאור. }מה תלמוד לומר אשר הכית בו את היאור, אלא שהיו ישראל אומרים על המטה, שאינו מוכן אלא לפורענות, בו לקה פרעה, ומצרים כמה מכות, במצרים ועל הים, לכך נאמר אשר הכית בו את היאור, יראו עתה שאף לטובה הוא מוכן׃ }}
{וַאֲמַר יְיָ לְמֹשֶׁה עֲבַר קֳדָם עַמָּא וְסַב עִמָּךְ מִסָּבֵי יִשְׂרָאֵל וְחוּטְרָךְ דִּמְחֵיתָא בֵיהּ יָת נַהְרָא סַב בִּידָךְ וְתֵיזֵיל׃}
{And the \lord\space said unto Moses: ‘Pass on before the people, and take with thee of the elders of Israel; and thy rod, wherewith thou smotest the river, take in thy hand, and go.}{\arabic{verse}}
\threeverse{\arabic{verse}}%Ex.17:6
{הִנְנִ֣י עֹמֵד֩ לְפָנֶ֨יךָ שָּׁ֥ם \pasek  עַֽל־הַצּוּר֮ בְּחֹרֵב֒ וְהִכִּ֣יתָ בַצּ֗וּר וְיָצְא֥וּ מִמֶּ֛נּוּ מַ֖יִם וְשָׁתָ֣ה הָעָ֑ם וַיַּ֤עַשׂ כֵּן֙ מֹשֶׁ֔ה לְעֵינֵ֖י זִקְנֵ֥י יִשְׂרָאֵֽל׃
\rashi{\rashiDH{והכית בצור. }על הצור לא נאמר, אלא בצור, מכאן שהמטה היה של מין דבר חזק ושמו סְנַפִּירִינוֹן, והצור נבקע מפניו. }}
{הָאֲנָא קָאֵים קֳדָמָךְ תַּמָּן עַל טִנָּרָא בְּחוֹרֵב וְתִמְחֵי בְּטִנָּרָא וְיִפְּקוּן מִנֵּיהּ מַיָּא וְיִשְׁתֵּי עַמָּא וַעֲבַד כֵּן מֹשֶׁה לְעֵינֵי סָבֵי יִשְׂרָאֵל׃}
{Behold, I will stand before thee there upon the rock in Horeb; and thou shalt smite the rock, and there shall come water out of it, that the people may drink.’ And Moses did so in the sight of the elders of Israel.}{\arabic{verse}}
\threeverse{\arabic{verse}}%Ex.17:7
{וַיִּקְרָא֙ שֵׁ֣ם הַמָּק֔וֹם מַסָּ֖ה וּמְרִיבָ֑ה עַל־רִ֣יב \legarmeh  בְּנֵ֣י יִשְׂרָאֵ֗ל וְעַ֨ל נַסֹּתָ֤ם אֶת־יְהֹוָה֙ לֵאמֹ֔ר הֲיֵ֧שׁ יְהֹוָ֛ה בְּקִרְבֵּ֖נוּ אִם־אָֽיִן׃ \petucha }
{וּקְרָא שְׁמֵיהּ דְּאַתְרָא נִסֵּיתָא וּמַצּוּתָא עַל דִּנְצוֹ בְּנֵי יִשְׂרָאֵל וְעַל דְּנַסִּיאוּ קֳדָם יְיָ לְמֵימַר הַאִית שְׁכִינְתָא דַּייָ בֵּינַנָא אִם לָא׃}
{And the name of the place was called Massah, and Meribah, because of the striving of the children of Israel, and because they tried the \lord, saying: ‘Is the \lord\space among us, or not?’}{\arabic{verse}}
\threeverse{\arabic{verse}}%Ex.17:8
{וַיָּבֹ֖א עֲמָלֵ֑ק וַיִּלָּ֥חֶם עִם־יִשְׂרָאֵ֖ל בִּרְפִידִֽם׃
\rashi{\rashiDH{ויבא עמלק וגו׳. }סמך פרשה זו למקרא זה לומר, תמיד אני ביניכם ומזומן לכל צרכיכם, ואתם אומרים היש ה׳ בקרבנו אם אין, חייכם שהכלב בא ונושך אתכם, ואתם צועקים לי ותדעו היכן אני. משל לאדם שהרכיב בנו על כתפו ויצא לדרך, היה אותו הבן רואה חפץ ואומר, אבא טול חפץ זה ותן לי, והוא נותן לו, וכן שנייה, וכן שלישית, פגעו באדם אחד, אמר לו אותו הבן ראית את אבא, אמר לו אביו, אינך יודע היכן אני, השליכו מעליו ובא הכלב ונשכו׃ 
}}
{וַאֲתָא עֲמָלֵק וַאֲגִיחַ קְרָבָא עִם יִשְׂרָאֵל בִּרְפִידִים׃}
{Then came Amalek, and fought with Israel in Rephidim.}{\arabic{verse}}
\threeverse{\arabic{verse}}%Ex.17:9
{וַיֹּ֨אמֶר מֹשֶׁ֤ה אֶל־יְהוֹשֻׁ֙עַ֙ בְּחַר־לָ֣נוּ אֲנָשִׁ֔ים וְצֵ֖א הִלָּחֵ֣ם בַּעֲמָלֵ֑ק מָחָ֗ר אָנֹכִ֤י נִצָּב֙ עַל־רֹ֣אשׁ הַגִּבְעָ֔ה וּמַטֵּ֥ה הָאֱלֹהִ֖ים בְּיָדִֽי׃
\rashi{\rashiDH{בחר לנו. }לי ולך, השוהו לו, מכאן אמרו, יהי כבוד תלמידך חביב עליך כשלך, וכבוד חברך כמורא רבך מנין, שנאמר וַיֹּאמֶר אַהֲרֹן אֶל משֶׁה בִּי אֲדֹנִי (במדבר יב, יא), והלא אהרן גדול מאחיו היה, ועושה את חברו כרבו. ומורא רבך כמורא שמים מנין, שנאמר אֲדֹנִי משֶׁה כְּלָאֵם (שם יא, כח), כלם מן העולם, חייבין הם כלייה, המורדים בך כאילו מרדו בהקב״ה׃ }\rashi{\rashiDH{וצא הלחם. }צא מן הענן והלחם בו (מכילתא עמלק פ״א)׃ }\rashi{\rashiDH{מחר. }בעת המלחמה, אנכי נצב׃ }\rashi{\rashiDH{בחר לנו אנשים. }גבורים ויראי חטא, שתהא זכותן מסייעתן. דבר אחר בחר לנו אנשים, שיודעין לבטל כשפים, לפי שבני עמלק מכשפים היו׃ 
}}
{וַאֲמַר מֹשֶׁה לִיהוֹשֻעַ בְּחַר לַנָא גּוּבְרִין וּפוֹק אֲגִיחַ קְרָבָא בַּעֲמָלֵק מְחַר אֲנָא קָאֵים עַל רֵישׁ רָמְתָא וְחוּטְרָא דְּאִתְעֲבִידוּ בֵיהּ נִסִּין מִן קֳדָם יְיָ בִּידִי׃}
{And Moses said unto Joshua: ‘Choose us out men, and go out, fight with Amalek; tomorrow I will stand on the top of the hill with the rod of God in my hand.’}{\arabic{verse}}
\threeverse{\arabic{verse}}%Ex.17:10
{וַיַּ֣עַשׂ יְהוֹשֻׁ֗עַ כַּאֲשֶׁ֤ר אָֽמַר־לוֹ֙ מֹשֶׁ֔ה לְהִלָּחֵ֖ם בַּעֲמָלֵ֑ק וּמֹשֶׁה֙ אַהֲרֹ֣ן וְח֔וּר עָל֖וּ רֹ֥אשׁ הַגִּבְעָֽה׃
\rashi{\rashiDH{ומשה אהרן וחור. }מכאן לתענית שצריכים שלשה לעבור לפני התיבה, שבתענית היו שרוים׃ }\rashi{\rashiDH{חור. }בנה של מרים היה, וכלב בעלה׃ }}
{וַעֲבַד יְהוֹשֻעַ כְּמָא דַּאֲמַר לֵיהּ מֹשֶׁה לְאָגָחָא קְרָבָא בַּעֲמָלֵק וּמֹשֶׁה אַהֲרֹן וְחוּר סְלִיקוּ לְרֵישׁ רָמְתָא׃}
{So Joshua did as Moses had said to him, and fought with Amalek; and Moses, Aaron, and Hur went up to the top of the hill.}{\arabic{verse}}
\threeverse{\arabic{verse}}%Ex.17:11
{וְהָיָ֗ה כַּאֲשֶׁ֨ר יָרִ֥ים מֹשֶׁ֛ה יָד֖וֹ וְגָבַ֣ר יִשְׂרָאֵ֑ל וְכַאֲשֶׁ֥ר יָנִ֛יחַ יָד֖וֹ וְגָבַ֥ר עֲמָלֵֽק׃
\rashi{\rashiDH{כאשר ירים משה ידו. }וכי ידיו של משה נוצחות היו המלחמה וכו׳, כדאיתא בר״ה (כט.)׃ }}
{וְהָוֵי כַּד מָרֵים מֹשֶׁה יְדוֹהִי מִתְגַּבְּרִין דְּבֵית יִשְׂרָאֵל וְכַד מַנַּח יְדוֹהִי מִתְגַּבְּרִין דְּבֵית עֲמָלֵק׃}
{And it came to pass, when Moses held up his hand, that Israel prevailed; and when he let down his hand, Amalek prevailed.}{\arabic{verse}}
\threeverse{\arabic{verse}}%Ex.17:12
{וִידֵ֤י מֹשֶׁה֙ כְּבֵדִ֔ים וַיִּקְחוּ־אֶ֛בֶן וַיָּשִׂ֥ימוּ תַחְתָּ֖יו וַיֵּ֣שֶׁב עָלֶ֑יהָ וְאַהֲרֹ֨ן וְח֜וּר תָּֽמְכ֣וּ בְיָדָ֗יו מִזֶּ֤ה אֶחָד֙ וּמִזֶּ֣ה אֶחָ֔ד וַיְהִ֥י יָדָ֛יו אֱמוּנָ֖ה עַד־בֹּ֥א הַשָּֽׁמֶשׁ׃
\rashi{\rashiDH{וידי משה כבדים. }בשביל שנתעצל במצוה ומנה אחר תחתיו, נתייקרו ידיו׃ }\rashi{\rashiDH{ויקחו. }אהרן וחור׃}\rashi{\rashiDH{אבן וישימו תחתיו. }ולא ישב לו על כר וכסת, אמר, ישראל שרויין בצער, אף אני אהיה עמהם בצער׃ 
}\rashi{\rashiDH{ויהי ידיו אמונה. }ויהי משה ידיו באמונה, פרושות השמים בתפלה נאמנה ונכונה׃ }\rashi{\rashiDH{עד בא השמש. }שהיו עמלקים מחשבין את השעות באיצטרו״לוגיאה, באיזו שעה הם נוצחים, והעמיד להם משה חמה וערבב את השעות׃ }}
{וִידֵי מֹשֶׁה יְקַרָא וּנְסִיבוּ אַבְנָא וְשַׁוִּיאוּ תְּחוֹתוֹהִי וִיתֵיב עֲלַהּ וְאַהֲרֹן וְחוּר סְעִידִין בִּידוֹהִי מִכָּא חַד וּמִכָּא חַד וַהֲוַאָה יְדוֹהִי פְּרִיסָן בִּצְלוֹ עַד דְּעָאל שִׁמְשָׁא׃}
{But Moses’ hands were heavy; and they took a stone, and put it under him, and he sat thereon; and Aaron and Hur stayed up his hands, the one on the one side, and the other on the other side; and his hands were steady until the going down of the sun.}{\arabic{verse}}
\threeverse{\arabic{verse}}%Ex.17:13
{וַיַּחֲלֹ֧שׁ יְהוֹשֻׁ֛עַ אֶת־עֲמָלֵ֥ק וְאֶת־עַמּ֖וֹ לְפִי־חָֽרֶב׃ \petucha 
\rashi{\rashiDH{ויחלש יהושע. }חתך ראשי גבוריו (תנחומא בשלח כ״ח) ולא השאיר אלא חלשים שבהם, ולא הרגם כולם, מכאן אנו למדים, שעשו על פי הדבור של שכינה׃ }}
{וְתַבַּר יְהוֹשֻעַ יָת עֲמָלֵק וְיָת עַמֵּיהּ לְפִתְגָם דְּחָרֶב׃}
{And Joshua discomfited Amalek and his people with the edge of the sword.}{\arabic{verse}}
\threeverse{\aliya{מפטיר}}%Ex.17:14
{וַיֹּ֨אמֶר יְהֹוָ֜ה אֶל־מֹשֶׁ֗ה כְּתֹ֨ב זֹ֤את זִכָּרוֹן֙ בַּסֵּ֔פֶר וְשִׂ֖ים בְּאׇזְנֵ֣י יְהוֹשֻׁ֑עַ כִּֽי־מָחֹ֤ה אֶמְחֶה֙ אֶת־זֵ֣כֶר עֲמָלֵ֔ק מִתַּ֖חַת הַשָּׁמָֽיִם׃
\rashi{\rashiDH{כתב זאת זכרון. }שבא עמלק להזדווג לישראל קודם לכל האומות (מכילתא עמלק פ״ב)׃ 
}\rashi{\rashiDH{ושים באזני יהושע. }המכניס את ישראל לארץ, שיצוה את ישראל לשלם לו את גמולו, כאן נרמז לו למשה שיהושע מכניס את ישראל לארץ׃ }\rashi{\rashiDH{כי מחה אמחה. }לכך אני מזהירך כן, כי חפץ אני למחותו׃ }}
{וַאֲמַר יְיָ לְמֹשֶׁה כְּתוֹב דָּא דּוּכְרָנָא בְּסִפְרָא וְשַׁו קֳדָם יְהוֹשֻעַ אֲרֵי מִמְחָא אֶמְחֵי יָת דּוּכְרָנֵיהּ דַּעֲמָלֵק מִתְּחוֹת שְׁמַיָּא׃}
{And the \lord\space said unto Moses: ‘Write this for a memorial in the book, and rehearse it in the ears of Joshua: for I will utterly blot out the remembrance of Amalek from under heaven.’}{\arabic{verse}}
\threeverse{\arabic{verse}}%Ex.17:15
{וַיִּ֥בֶן מֹשֶׁ֖ה מִזְבֵּ֑חַ וַיִּקְרָ֥א שְׁמ֖וֹ יְהֹוָ֥ה \pasek  נִסִּֽי׃
\rashi{\rashiDH{ויקרא שמו. }של מזבח׃}\rashi{\rashiDH{ה׳ נסי. }הקב״ה עשה לנו כאן נס גדול, לא שהמזבח קרוי ה׳, אלא המזכיר שמו של מזבח, זוכר את הנס שעשה המקום, ה׳ הוא נס שלנו׃ }}
{וּבְנָא מֹשֶׁה מַדְבְּחָא וּפְלַח עֲלוֹהִי קֳדָם יְיָ דַּעֲבַד לֵיהּ נִסִּין׃}
{And Moses built an altar, and called the name of it Adonai-nissi.}{\arabic{verse}}
\threeverse{\arabic{verse}}%Ex.17:16
{וַיֹּ֗אמֶר כִּֽי־יָד֙ עַל־כֵּ֣ס יָ֔הּ\note{בכתר ארם צובה היה כתוב כֵּ֣סְיָ֔הּ בתיבה אחת} מִלְחָמָ֥ה לַיהֹוָ֖ה בַּֽעֲמָלֵ֑ק מִדֹּ֖ר דֹּֽר׃ \petucha 
\rashi{\rashiDH{ויאמר. }משה׃}\rashi{\rashiDH{כי יד על כס יה. }ידו של הקב״ה הורמה לישבע בכסאו, להיות לו מלחמה ואיבה בעמלק עולמית, ומהו כס ולא נאמר כסא, ואף השם נחלק לחציו, נשבע הקב״ה, שאין שמו שלם ואין כסאו שלם עד שימחה שמו של עמלק כולו, וכשימחה שמו, יהיה השם שלם והכסא שלם, שנאמר הָאֹויֵב תַּמּוּ חֳרָבֹות לָנֶצַח (תהלים ט, ז), זהו עמלק שכתוב בו וְעֶבְרָתֹו שְׁמָרָה נֶצַח (עמוס א, יא), וְעָרִים נָתַשְׁתָּ אָבַד זִכְרָם הֵמָּה (תהלים שם), מהו אומר אחריו, וַה׳ לְעֹולָם יֵשֵׁב, הרי השם שלם, כֹּונֵן לַמִּשְׁפָּט כִּסְאֹו, הרי הכסא שלם׃ 
}}
{וַאֲמַר בִּשְׁבוּעָה אֲמִירָא דָּא מִן קֳדָם דְּחִילָא דִּשְׁכִינְתֵיהּ עַל כּוּרְסֵי יְקָרָא דַּעֲתִיד דְּיִתָּגַח קְרָבָא קֳדָם יְיָ בִּדְבֵית עֲמָלֵק לְשֵׁיצָיוּתְהוֹן מִדָּרֵי עָלְמָא׃}
{And he said: ‘The hand upon the throne of the \lord: the \lord\space will have war with Amalek from generation to generation.’}{\arabic{verse}}
\newperek
\newparsha{יתרו}
\threeverse{\aliya{יתרו}}%Ex.18:1
{וַיִּשְׁמַ֞ע יִתְר֨וֹ כֹהֵ֤ן מִדְיָן֙ חֹתֵ֣ן מֹשֶׁ֔ה אֵת֩ כׇּל־אֲשֶׁ֨ר עָשָׂ֤ה אֱלֹהִים֙ לְמֹשֶׁ֔ה וּלְיִשְׂרָאֵ֖ל עַמּ֑וֹ כִּֽי־הוֹצִ֧יא יְהֹוָ֛ה אֶת־יִשְׂרָאֵ֖ל מִמִּצְרָֽיִם׃
\rashi{\rashiDH{וישמע יתרו. }מה שמיעה שמע ובא, קריעת ים סוף ומלחמת עמלק (זבחים קטז.)׃ }\rashi{\rashiDH{יתרו. }שבע שמות נקראו לו, רעואל, יתר, יתרו, חובב, חבר, קיני, פוטיאל, (מכילתא יתרו פ״א). תר, ע״ש שֶׁיֶּתֶר פרשה אחת בתורה, ואתה תחזה. יתרו, לכשנתגייר וקיים המצות, הוסיפו לו אות אחת על שמו. חובב, שחבב את התורה, וחובב הוא יתרו שנאמר מִבְּנֵי חֹבָב חֹתֵן משֶׁה (שופטים ד, יא). ויש אומרים רעואל אביו של יתרו היה, ומהו אומר וַתָּבֹאנָה אֶל רְעוּאֵל אֲבִיהֶן (שמות ב, יח), שהתינוקות קורין לאבי אביהן אבא. בספרי (בהעלותך עח)׃ }\rashi{\rashiDH{חותן משה. }כאן היה יתרו מתכבד במשה, אני חותן המלך, ולשעבר היה משה תולה הגדולה בחמיו, שנאמר וַיָשָׁב אֶל יֶתֶר חֹתְנֹו (מכילתא שם)׃ 
}\rashi{\rashiDH{למשה ולישראל. }שָׁקוּל משה כנגד כל ישראל׃}\rashi{\rashiDH{את כל אשר עשה. }להם בירידת המן, ובבאר, ובעמלק׃ }\rashi{\rashiDH{כי הוציא ה׳ וגו׳. }זו גדולה על כולם (מכילתא שם)׃}}
{וּשְׁמַע יִתְרוֹ רַבָּא דְּמִדְיָן חֲמוּהִי דְּמֹשֶׁה יָת כָּל דַּעֲבַד יְיָ לְמֹשֶׁה וּלְיִשְׂרָאֵל עַמֵּיהּ אֲרֵי אַפֵּיק יְיָ יָת יִשְׂרָאֵל מִמִּצְרָיִם׃}
{Now Jethro, the priest of Midian, Moses’ father-in-law, heard of all that God had done for Moses, and for Israel His people, how that the \lord\space had brought Israel out of Egypt.}{\Roman{chap}}
\threeverse{\arabic{verse}}%Ex.18:2
{וַיִּקַּ֗ח יִתְרוֹ֙ חֹתֵ֣ן מֹשֶׁ֔ה אֶת־צִפֹּרָ֖ה אֵ֣שֶׁת מֹשֶׁ֑ה אַחַ֖ר שִׁלּוּחֶֽיהָ׃
\rashi{\rashiDH{אחר שלוחיה. }כשאמר לו הקב״ה במדין, לֵךְ שֻב מִצְרָים, וַיִקַּח מֹשֶה אֶת אִשְתֹּו וְאֶת בָּנָיו וגו׳ (שמות ד, יטכ), ויצא אהרן לקראתו ויפגשהו בהר האלהים, אמר לו מי הם הללו, אמר לו זו היא אשתי שנשאתי במדין ואלו בני, אמר לו והיכן אתה מוליכן, אמר לו למצרים, אמר לו על הראשונים אנו מצטערים ואתה בא להוסיף עליהם, אמר לה לכי אל בית אביך, נטלה שני בניה והלכה לה׃ }}
{וּדְבַר יִתְרוֹ חֲמוּהִי דְּמֹשֶׁה יָת צִפֹּרָה אִתַּת מֹשֶׁה בָּתַר דְּשַׁלְּחַהּ׃}
{And Jethro, Moses’ father-in-law, took Zipporah, Moses’ wife, after he had sent her away,}{\arabic{verse}}
\threeverse{\arabic{verse}}%Ex.18:3
{וְאֵ֖ת שְׁנֵ֣י בָנֶ֑יהָ אֲשֶׁ֨ר שֵׁ֤ם הָֽאֶחָד֙ גֵּֽרְשֹׁ֔ם כִּ֣י אָמַ֔ר גֵּ֣ר הָיִ֔יתִי בְּאֶ֖רֶץ נׇכְרִיָּֽה׃}
{וְיָת תְּרֵין בְּנַהָא דְּשׁוֹם חַד גֵּרְשׁוֹם אֲרֵי אֲמַר דַּיָּיר הֲוֵיתִי בַּאֲרַע נוּכְרָאָה׃}
{and her two sons; of whom the name of the one was Gershom; for he said: ‘I have been a stranger in a strange land’;}{\arabic{verse}}
\threeverse{\arabic{verse}}%Ex.18:4
{וְשֵׁ֥ם הָאֶחָ֖ד אֱלִיעֶ֑זֶר כִּֽי־אֱלֹהֵ֤י אָבִי֙ בְּעֶזְרִ֔י וַיַּצִּלֵ֖נִי מֵחֶ֥רֶב פַּרְעֹֽה׃
\rashi{\rashiDH{ויצלני מחרב פרעה. }כשגילו דתן ואבירם על דבר המצרי, ובקש להרוג את משה, נעשה צוארו כעמוד של שיש (שמו״ר א, לו)׃ }}
{וְשׁוֹם חַד אֱלִיעֶזֶר אֲרֵי אֱלָהֵיהּ דְּאַבָּא הֲוָה בְּסַעֲדִי וְשֵׁיזְבַנִי מֵחַרְבָּא דְּפַרְעֹה׃}
{and the name of the other was Eliezer: ‘for the God of my father was my help, and delivered me from the sword of Pharaoh.’}{\arabic{verse}}
\threeverse{\aliya{לוי}}%Ex.18:5
{וַיָּבֹ֞א יִתְר֨וֹ חֹתֵ֥ן מֹשֶׁ֛ה וּבָנָ֥יו וְאִשְׁתּ֖וֹ אֶל־מֹשֶׁ֑ה אֶל־הַמִּדְבָּ֗ר אֲשֶׁר־ה֛וּא חֹנֶ֥ה שָׁ֖ם הַ֥ר הָאֱלֹהִֽים׃
\rashi{\rashiDH{אל המדבר. }אף אנו יודעים שבמדבר היה, אלא בשבחו של יתרו דִּבֶּר הכתוב, שהיה יושב בכבודו של עולם, ונדבו לבו לצאת אל המדבר מקום תהו, לשמוע דברי תורה׃ 
}}
{וַאֲתָא יִתְרוֹ חֲמוּהִי דְּמֹשֶׁה וּבְנוֹהִי וְאִתְּתֵיהּ לְוָת מֹשֶׁה לְמַדְבְּרָא דְּהוּא שָׁרֵי תַּמָּן לְטוּרָא דְּאִתְגְּלִי עֲלוֹהִי יְקָרָא דַּייָ׃}
{And Jethro, Moses’ father-in-law, came with his sons and his wife unto Moses into the wilderness where he was encamped, at the mount of God;}{\arabic{verse}}
\threeverse{\arabic{verse}}%Ex.18:6
{וַיֹּ֙אמֶר֙ אֶל־מֹשֶׁ֔ה אֲנִ֛י חֹתֶנְךָ֥ יִתְר֖וֹ בָּ֣א אֵלֶ֑יךָ וְאִ֨שְׁתְּךָ֔ וּשְׁנֵ֥י בָנֶ֖יהָ עִמָּֽהּ׃
\rashi{\rashiDH{ויאמר אל משה. }ע״י שליח׃ }\rashi{\rashiDH{אני חתנך יתרו וגו׳. }אם אין אתה יוצא בגיני צא בגין אשתך, ואם אין אתה יוצא בגין אשתך צא בגין שני בניה (מכילתא שם)׃ }}
{וַאֲמַר לְמֹשֶׁה אֲנָא חֲמוּךְ יִתְרוֹ אָתֵי לְוָתָךְ וְאִתְּתָךְ וּתְרֵין בְּנַהָא עִמַּהּ׃}
{and he said unto Moses: ‘I thy father-in-law Jethro am coming unto thee, and thy wife, and her two sons with her.’}{\arabic{verse}}
\threeverse{\arabic{verse}}%Ex.18:7
{וַיֵּצֵ֨א מֹשֶׁ֜ה לִקְרַ֣את חֹֽתְנ֗וֹ וַיִּשְׁתַּ֙חוּ֙ וַיִּשַּׁק־ל֔וֹ וַיִּשְׁאֲל֥וּ אִישׁ־לְרֵעֵ֖הוּ לְשָׁל֑וֹם וַיָּבֹ֖אוּ הָאֹֽהֱלָה׃
\rashi{\rashiDH{ויצא משה. }כבוד גדול נתכבד יתרו באותה שעה, כיון שיצא משה, יצא אהרן נדב ואביהוא, ומי הוא שראה את אלו יוצאין ולא יצא׃ }\rashi{\rashiDH{וישתחו וישק לו. }איני יודע מי השתחוה למי, כשהוא אומר איש לרעהו, מי הקרוי איש, זה משה, שנאמר וְהָאִישׁ משֶׁה (מכילתא שם)׃ }}
{וּנְפַק מֹשֶׁה לְקַדָּמוּת חֲמוּהִי וּסְגֵיד וְנַשֵּׁיק לֵיהּ וּשְׁאִילוּ גְּבַר לְחַבְרֵיהּ לִשְׁלָם וְעָאלוּ לְמַשְׁכְּנָא׃}
{And Moses went out to meet his father-in-law, and bowed down and kissed him; and they asked each other of their welfare; and they came into the tent.}{\arabic{verse}}
\threeverse{\arabic{verse}}%Ex.18:8
{וַיְסַפֵּ֤ר מֹשֶׁה֙ לְחֹ֣תְנ֔וֹ אֵת֩ כׇּל־אֲשֶׁ֨ר עָשָׂ֤ה יְהֹוָה֙ לְפַרְעֹ֣ה וּלְמִצְרַ֔יִם עַ֖ל אוֹדֹ֣ת יִשְׂרָאֵ֑ל אֵ֤ת כׇּל־הַתְּלָאָה֙ אֲשֶׁ֣ר מְצָאָ֣תַם בַּדֶּ֔רֶךְ וַיַּצִּלֵ֖ם יְהֹוָֽה׃
\rashi{\rashiDH{ויספר משה לחותנו. }למשוך את לבו לקרבו לתורה (מכילתא שם)׃}\rashi{\rashiDH{את כל התלאה. }שעל הים ושל עמלק (מכילתא שם)׃}\rashi{\rashiDH{התלאה. }למ״ד אל״ף מן היסוד של תיבה, והתי״ו הוא תיקון ויסוד הנופל ממנו לפרקים, וכן תרומה, תנופה, תקומה, תנואה׃ }}
{וְאִשְׁתַּעִי מֹשֶׁה לַחֲמוּהִי יָת כָּל דַּעֲבַד יְיָ לְפַרְעֹה וּלְמִצְרָאֵי עַל עֵיסַק יִשְׂרָאֵל יָת כָּל עָקְתָא דְּאַשְׁכַּחַתְנוּן בְּאוֹרְחָא וְשֵׁיזֵיבִנּוּן יְיָ׃}
{And Moses told his father-in-law all that the \lord\space had done unto Pharaoh and to the Egyptians for Israel’s sake, all the travail that had come upon them by the way, and how the \lord\space delivered them.}{\arabic{verse}}
\threeverse{\aliya{ישראל}}%Ex.18:9
{וַיִּ֣חַדְּ יִתְר֔וֹ עַ֚ל כׇּל־הַטּוֹבָ֔ה אֲשֶׁר־עָשָׂ֥ה יְהֹוָ֖ה לְיִשְׂרָאֵ֑ל אֲשֶׁ֥ר הִצִּיל֖וֹ מִיַּ֥ד מִצְרָֽיִם׃
\rashi{\rashiDH{ויחד יתרו. }וישמח יתרו, זהו פשוטו. ומדרש אגדה, נעשה בשרו חדודין חדודין, מיצר על אבוד מצרים, היינו דאמרי אינשי, גיורא עד עשרה דרי לא תבזי ארמאה באפיה (סנהדרין צד.)׃ }\rashi{\rashiDH{על כל הטובה. }טובת המן והבאר והתורה (מכילתא שם), ועל כולן אשר הצילו מיד מצרים, עד עכשיו לא היה עבד יכול לברוח ממצרים, שהיתה הארץ מסוגרת, ואלו יצאו ששים רבוא (מכילתא שם)׃ }}
{וַחְדִּי יִתְרוֹ עַל כָּל טָבְתָא דַּעֲבַד יְיָ לְיִשְׂרָאֵל דְּשֵׁיזֵיבִנּוּן מִיְּדָא דְּמִצְרָאֵי׃}
{And Jethro rejoiced for all the goodness which the \lord\space had done to Israel, in that He had delivered them out of the hand of the Egyptians.}{\arabic{verse}}
\threeverse{\arabic{verse}}%Ex.18:10
{וַיֹּ֘אמֶר֮ יִתְרוֹ֒ בָּר֣וּךְ יְהֹוָ֔ה אֲשֶׁ֨ר הִצִּ֥יל אֶתְכֶ֛ם מִיַּ֥ד מִצְרַ֖יִם וּמִיַּ֣ד פַּרְעֹ֑ה אֲשֶׁ֤ר הִצִּיל֙ אֶת־הָעָ֔ם מִתַּ֖חַת יַד־מִצְרָֽיִם׃
\rashi{\rashiDH{אשר הציל אתכם מיד מצרים. }אומה קשה׃}\rashi{\rashiDH{ומיד פרעה. }מלך קשה׃}\rashi{\rashiDH{מתחת יד מצרים. }כתרגומו לשון רדוי ומרות, היד שהיו מכבידים עליהם, היא העבודה׃ }}
{וַאֲמַר יִתְרוֹ בְּרִיךְ יְיָ דְּשֵׁיזֵיב יָתְכוֹן מִיְּדָא דְּמִצְרָאֵי וּמִיְּדָא דְּפַרְעֹה דְּשֵׁיזֵיב יָת עַמָּא מִתְּחוֹת מַרְוַת מִצְרָאֵי׃}
{And Jethro said: ‘Blessed be the \lord, who hath delivered you out of the hand of the Egyptians, and out of the hand of Pharaoh; who hath delivered the people from under the hand of the Egyptians.}{\arabic{verse}}
\threeverse{\arabic{verse}}%Ex.18:11
{עַתָּ֣ה יָדַ֔עְתִּי כִּֽי־גָד֥וֹל יְהֹוָ֖ה מִכׇּל־הָאֱלֹהִ֑ים כִּ֣י בַדָּבָ֔ר אֲשֶׁ֥ר זָד֖וּ עֲלֵיהֶֽם׃
\rashi{\rashiDH{עתה ידעתי. }מכירו הייתי לשעבר, ועכשיו ביותר (מכילתא שם)׃ }\rashi{\rashiDH{מכל האלהים. }מלמד שהיה מכיר בכל עבודת אלילים שבעולם, שלא הניח עבודת אלילים שלא עבדה (מכילתא שם)׃ }\rashi{\rashiDH{כי בדבר אשר זדו עליהם. }כתרגומו, במים דִּמּוּ לאבדם והם נאבדו במים׃ }\rashi{\rashiDH{אשר זדו. }אשר הרשיעו. ורבותינו דרשוהו (סוטה יא.) לשון ויזד יעקב נזיד (בראשית כה, כט), בקדרה אשר בשלו בה נתבשלו׃ }}
{כְּעַן יָדַעְנָא אֲרֵי רָב יְיָ וְלֵית אֱלָהּ בָּר מִנֵּיהּ אֲרֵי בְּפִתְגָמָא דְּחַשִּׁיבוּ מִצְרָאֵי לִמְדָּן יָת יִשְׂרָאֵל בֵּיהּ דָּנִינּוּן׃}
{Now I know that the \lord\space is greater than all gods; yea, for that they dealt proudly against them.’}{\arabic{verse}}
\threeverse{\arabic{verse}}%Ex.18:12
{וַיִּקַּ֞ח יִתְר֨וֹ חֹתֵ֥ן מֹשֶׁ֛ה עֹלָ֥ה וּזְבָחִ֖ים לֵֽאלֹהִ֑ים וַיָּבֹ֨א אַהֲרֹ֜ן וְכֹ֣ל \legarmeh  זִקְנֵ֣י יִשְׂרָאֵ֗ל לֶאֱכׇל־לֶ֛חֶם עִם־חֹתֵ֥ן מֹשֶׁ֖ה לִפְנֵ֥י הָאֱלֹהִֽים׃
\rashi{\rashiDH{עולה. }כמשמעה, שהיא כולה כליל׃ }\rashi{\rashiDH{זבחים. }שלמים׃}\rashi{\rashiDH{ויבא אהרן וגו׳. }ומשה היכן הלך, והלא הוא שיצא לקראתו וגרם לו את כל הכבוד, אלא שהיה עומד ומשמש לפניהם׃ 
}\rashi{\rashiDH{לפני האלהים. }מכאן שהנהנה מסעודה שתלמידי חכמים מסובין בה, כאילו נהנה מזיו השכינה (ברכות סד.)׃ }}
{וְקָרֵיב יִתְרוֹ חֲמוּהִי דְּמֹשֶׁה עֲלָוָן וְנִכְסַת קוּדְשִׁין קֳדָם יְיָ וַאֲתָא אַהֲרֹן וְכֹל סָבֵי יִשְׂרָאֵל לְמֵיכַל לַחְמָא עִם חֲמוּהִי דְּמֹשֶׁה קֳדָם יְיָ׃}
{And Jethro, Moses’ father-in-law, took a burnt-offering and sacrifices for God; and Aaron came, and all the elders of Israel, to eat bread with Moses’ father-in-law before God.}{\arabic{verse}}
\threeverse{\aliya{שני}}%Ex.18:13
{וַיְהִי֙ מִֽמׇּחֳרָ֔ת וַיֵּ֥שֶׁב מֹשֶׁ֖ה לִשְׁפֹּ֣ט אֶת־הָעָ֑ם וַיַּעֲמֹ֤ד הָעָם֙ עַל־מֹשֶׁ֔ה מִן־הַבֹּ֖קֶר עַד־הָעָֽרֶב׃
\rashi{\rashiDH{ויהי ממחרת. }מוצאי יום הכפורים היה, כך שנינו בספרי, ומהו ממחרת, למחרת רדתו מן ההר. ועל כרחך אי אפשר לומר אלא ממחרת יום הכפורים, שהרי קודם מתן תורה אי אפשר לומר והודעתי את חוקי וגו׳, ומשנתנה תורה עד יום הכפורים לא ישב משה לשפוט את העם, שהרי בי״ז בתמוז ירד ושבר את הלוחות, ולמחר עלה בהשכמה ושהה שמונים יום וירד ביום הכפורים. ואין פרשה זו כתובה כסדר, שלא נאמר ויהי ממחרת עד שנה שנייה, אף לדברי האומר יתרו קודם מתן תורה בא, שילוחו אל ארצו לא היה אלא עד שנה שנייה, שהרי נאמר כאן וישלח משה את חותנו, ומצינו במסע הדגלים שאמר לו משה נֹסְעִים אֲנַחְנוּ אֶל הַמָּקֹום וגו׳ אַל נָא תַּעֲזֹב אֹתָנוּ (במדבר י, לא), ואם זה קודם מתן תורה, מששלחו והלך היכן מצינו שחזר. ואם תאמר שם לא נאמר יתרו אלא חובב ובנו של יתרו היה, הוא חובב הוא יתרו, שהרי כתיב מִבְּנֵי חֹבָב חֹתֵן משֶׁה (שופטים ד, יא)׃ }\rashi{\rashiDH{וישב משה וגו׳ ויעמוד העם. }יושב כמלך וכולן עומדים, והוקשה הדבר ליתרו שהיה מזלזל בכבודן של ישראל, והוכיחו על כך, שנאמר מדוע אתה יושב לבדך וכלם נצבים׃ }\rashi{\rashiDH{מן הבקר עד הערב. }אפשר לומר כן, אלא כל דיין שדן דין אמת לאמיתו אפילו שעה אחת, מעלה עליו הכתוב כאילו עוסק בתורה כל היום, וכאילו נעשה שותף להקב״ה במעשה בראשית, שנאמר בו וַיְהִי עֶרֶב וגו׳ (שבת י.)׃ 
}}
{וַהֲוָה בְּיוֹמָא דְּבָתְרוֹהִי וִיתֵיב מֹשֶׁה לִמְדָּן יָת עַמָּא וְקָם עַמָּא עִלָּווֹהִי דְּמֹשֶׁה מִן צַפְרָא עַד רַמְשָׁא׃}
{And it came to pass on the morrow, that Moses sat to judge the people; and the people stood about Moses from the morning unto the evening.}{\arabic{verse}}
\threeverse{\arabic{verse}}%Ex.18:14
{וַיַּרְא֙ חֹתֵ֣ן מֹשֶׁ֔ה אֵ֛ת כׇּל־אֲשֶׁר־ה֥וּא עֹשֶׂ֖ה לָעָ֑ם וַיֹּ֗אמֶר מָֽה־הַדָּבָ֤ר הַזֶּה֙ אֲשֶׁ֨ר אַתָּ֤ה עֹשֶׂה֙ לָעָ֔ם מַדּ֗וּעַ אַתָּ֤ה יוֹשֵׁב֙ לְבַדֶּ֔ךָ וְכׇל־הָעָ֛ם נִצָּ֥ב עָלֶ֖יךָ מִן־בֹּ֥קֶר עַד־עָֽרֶב׃}
{וַחֲזָא חֲמוּהִי דְּמֹשֶׁה יָת כָּל דְּהוּא עָבֵיד לְעַמָּא וַאֲמַר מָא פִתְגָמָא הָדֵין דְּאַתְּ עָבֵיד לְעַמָּא מָדֵין אַתְּ יָתֵיב בִּלְחוֹדָךְ וְכָל עַמָּא קָיְמִין עִלָּוָךְ מִן צַפְרָא עַד רַמְשָׁא׃}
{And when Moses’ father-in-law saw all that he did to the people, he said: ‘What is this thing that thou doest to the people? why sittest thou thyself alone, and all the people stand about thee from morning unto even?’}{\arabic{verse}}
\threeverse{\arabic{verse}}%Ex.18:15
{וַיֹּ֥אמֶר מֹשֶׁ֖ה לְחֹתְנ֑וֹ כִּֽי־יָבֹ֥א אֵלַ֛י הָעָ֖ם לִדְרֹ֥שׁ אֱלֹהִֽים׃
\rashi{\rashiDH{כי יבא. }כי בא, לשון הווה׃ }\rashi{\rashiDH{לדרש אלהים. }כתרגומו לְמִתְבַּע אוּלְפַן, לשאול תלמוד מפי הגבורה׃ }}
{וַאֲמַר מֹשֶׁה לַחֲמוּהִי אֲרֵי אָתַן לְוָתִי עַמָּא לְמִתְבַּע אוּלְפָן מִן קֳדָם יְיָ׃}
{And Moses said unto his father-in-law: ‘Because the people come unto me to inquire of God;}{\arabic{verse}}
\threeverse{\arabic{verse}}%Ex.18:16
{כִּֽי־יִהְיֶ֨ה לָהֶ֤ם דָּבָר֙ בָּ֣א אֵלַ֔י וְשָׁ֣פַטְתִּ֔י בֵּ֥ין אִ֖ישׁ וּבֵ֣ין רֵעֵ֑הוּ וְהוֹדַעְתִּ֛י אֶת־חֻקֵּ֥י הָאֱלֹהִ֖ים וְאֶת־תּוֹרֹתָֽיו׃
\rashi{\rashiDH{כי יהיה להם דבר בא. }מי שהיה לו הדבר בא אלי׃ 
}}
{כַּד הָוֵי לְהוֹן דִּינָא אָתַן לְוָתִי וְדָאֵינְנָא בֵּין גּוּבְרָא וּבֵין חַבְרֵיהּ וּמְהוֹדַעְנָא לְהוֹן יָת קְיָמַיָּא דַּייָ וְיָת אוֹרָיָתֵיהּ׃}
{when they have a matter, it cometh unto me; and I judge between a man and his neighbour, and I make them know the statutes of God, and His laws.’}{\arabic{verse}}
\threeverse{\arabic{verse}}%Ex.18:17
{וַיֹּ֛אמֶר חֹתֵ֥ן מֹשֶׁ֖ה אֵלָ֑יו לֹא־טוֹב֙ הַדָּבָ֔ר אֲשֶׁ֥ר אַתָּ֖ה עֹשֶֽׂה׃
\rashi{\rashiDH{ויאמר חתן משה. }דרך כבוד קוראו הכתוב חותנו של מלך׃}}
{וַאֲמַר חֲמוּהִי דְּמֹשֶׁה לֵיהּ לָא תָקֵין פִּתְגָמָא דְּאַתְּ עָבֵיד׃}
{And Moses’ father-in-law said unto him: ‘The thing that thou doest is not good.}{\arabic{verse}}
\threeverse{\arabic{verse}}%Ex.18:18
{נָבֹ֣ל תִּבֹּ֔ל גַּם־אַתָּ֕ה גַּם־הָעָ֥ם הַזֶּ֖ה אֲשֶׁ֣ר עִמָּ֑ךְ כִּֽי־כָבֵ֤ד מִמְּךָ֙ הַדָּבָ֔ר לֹא־תוּכַ֥ל עֲשֹׂ֖הוּ לְבַדֶּֽךָ׃
\rashi{\rashiDH{נבל תבול. }כתרגומו. ולשונו לשון כמישה פלייש״טרא, כמו וְהֶעָלֶה נָבֵל (ירמיה ח, יג), כִּנְבֹל עָלֶה מִגֶּפֶן וגו׳ (ישעיה לד, ד), שהוא כמוש ע״י חמה וע״י קרח, וכחו תש ונלאה׃ }\rashi{\rashiDH{גם אתה. }לרבות אהרן וחור וע׳ זקנים׃}\rashi{\rashiDH{כי כבד ממך. }כובדו רב יותר מכחך׃}}
{מִלְאָה תִלְאֵי אַף אַתְּ אַף עַמָּא הָדֵין דְּעִמָּךְ אֲרֵי יַקִּיר מִנָּךְ פִּתְגָמָא לָא תִכּוֹל לְמִעְבְּדֵיהּ בִּלְחוֹדָךְ׃}
{Thou wilt surely wear away, both thou, and this people that is with thee; for the thing is too heavy for thee; thou art not able to perform it thyself alone.}{\arabic{verse}}
\threeverse{\arabic{verse}}%Ex.18:19
{עַתָּ֞ה שְׁמַ֤ע בְּקֹלִי֙ אִיעָ֣צְךָ֔ וִיהִ֥י אֱלֹהִ֖ים עִמָּ֑ךְ הֱיֵ֧ה אַתָּ֣ה לָעָ֗ם מ֚וּל הָֽאֱלֹהִ֔ים וְהֵבֵאתָ֥ אַתָּ֛ה אֶת־הַדְּבָרִ֖ים אֶל־הָאֱלֹהִֽים׃
\rashi{\rashiDH{איעצך ויהי אלהים עמך. }בעצה, אמר לו צא המלך בגבורה (מכילתא פ״ב)׃ 
}\rashi{\rashiDH{היה אתה לעם מול האלהים. }שליח ומליץ בינותם למקום, ושואל משפטים מאתו׃ }\rashi{\rashiDH{הדברים. }דברי ריבותם׃}}
{כְּעַן קַבֵּיל מִנִּי אַמְלְכִנָּךְ וִיהֵי מֵימְרָא דַּייָ בְּסַעֲדָךְ הֱוִי אַתְּ לְעַמָּא תָּבַע אוּלְפָן מִן קֳדָם יְיָ וּתְהֵי מֵיתֵי אַתְּ יָת פִּתְגָמַיָּא לִקְדָם יְיָ׃}
{Hearken now unto my voice, I will give thee counsel, and God be with thee: be thou for the people before God, and bring thou the causes unto God.}{\arabic{verse}}
\threeverse{\arabic{verse}}%Ex.18:20
{וְהִזְהַרְתָּ֣ה אֶתְהֶ֔ם אֶת־הַחֻקִּ֖ים וְאֶת־הַתּוֹרֹ֑ת וְהוֹדַעְתָּ֣ לָהֶ֗ם אֶת־הַדֶּ֙רֶךְ֙ יֵ֣לְכוּ בָ֔הּ וְאֶת־הַֽמַּעֲשֶׂ֖ה אֲשֶׁ֥ר יַעֲשֽׂוּן׃}
{וְתַזְהַר יָתְהוֹן יָת קְיָמַיָּא וְיָת אוֹרָיָתָא וּתְהוֹדַע לְהוֹן יָת אוֹרְחָא דִּיהָכוּן בַּהּ וְיָת עוּבָדָא דְּיַעְבְּדוּן׃}
{And thou shalt teach them the statutes and the laws, and shalt show them the way wherein they must walk, and the work that they must do.}{\arabic{verse}}
\threeverse{\arabic{verse}}%Ex.18:21
{וְאַתָּ֣ה תֶחֱזֶ֣ה מִכׇּל־הָ֠עָ֠ם אַנְשֵׁי־חַ֜יִל יִרְאֵ֧י אֱלֹהִ֛ים אַנְשֵׁ֥י אֱמֶ֖ת שֹׂ֣נְאֵי בָ֑צַע וְשַׂמְתָּ֣ עֲלֵהֶ֗ם שָׂרֵ֤י אֲלָפִים֙ שָׂרֵ֣י מֵא֔וֹת שָׂרֵ֥י חֲמִשִּׁ֖ים וְשָׂרֵ֥י עֲשָׂרֹֽת׃
\rashi{\rashiDH{ואתה תחזה. }ברוח הקדש שעליך׃}\rashi{\rashiDH{אנשי חיל. }עשירים, שאין צריכין להחניף ולהכיר פנים׃ }\rashi{\rashiDH{אנשי אמת. }אלו בעלי הבטחה, שהם כדאי לסמוך על דבריהם, שע״י כן יהיו דבריהם נשמעין׃ }\rashi{\rashiDH{שנאי בצע. }ששונאין את ממונם בדין, כההיא דאמרינן, כל דיינא דמפקין ממונא מיניה בדינא, לאו דיינא הוא (בבא בתרא נח׃)׃ }\rashi{\rashiDH{שרי אלפים. }הם היו שש מאות שרים לשש מאות אלף (סנהדרין יח.)׃}\rashi{\rashiDH{שרי מאות. }ששת אלפים היו׃}\rashi{\rashiDH{שרי חמשים. }י״ב אלף׃ }\rashi{\rashiDH{שרי עשרות. }ששים אלף. (מה שפירש״י על כל השרים כמה היו. והוא לכאורה ללא צורך, והנה באמת תיקן בזה ותירץ קושיא בפסוק, דקחשיב מלמעלה למטה, ר״ל המספר מרובה קודם מספר המועט, ולא הל״ל אלא מתחלה שרי עשרות בראשונה, ואח״כ בהדרגה כולם, ולפי פירושו שהזכיר ופרט סכום מנין השרים, צא וחשוב, וכשתדקדק במנינם ולגבייהו אתי שפיר, מספר המועט תחלה ואח״כ בהדרגה, כן נ״ל נכון ודו״ק)׃ 
}}
{וְאַתְּ תִּחְזֵי מִכָּל עַמָּא גּוּבְרִין דְּחֵילָא דָּחֲלַיָּא דַּייָ גּוּבְרִין דִּקְשׁוֹט דְּסָנַן לְקַבָּלָא מָמוֹן וּתְמַנֵּי עֲלֵיהוֹן רַבָּנֵי אַלְפֵי רַבָּנֵי מָאוָתָא רַבָּנֵי חַמְשִׁין וְרַבָּנֵי עֲשׂוֹרְיָיתָא׃}
{Moreover thou shalt provide out of all the people able men, such as fear God, men of truth, hating unjust gain; and place such over them, to be rulers of thousands, rulers of hundreds, rulers of fifties, and rulers of tens.}{\arabic{verse}}
\threeverse{\arabic{verse}}%Ex.18:22
{וְשָׁפְט֣וּ אֶת־הָעָם֮ בְּכׇל־עֵת֒ וְהָיָ֞ה כׇּל־הַדָּבָ֤ר הַגָּדֹל֙ יָבִ֣יאוּ אֵלֶ֔יךָ וְכׇל־הַדָּבָ֥ר הַקָּטֹ֖ן יִשְׁפְּטוּ־הֵ֑ם וְהָקֵל֙ מֵֽעָלֶ֔יךָ וְנָשְׂא֖וּ אִתָּֽךְ׃
\rashi{\rashiDH{ושפטו. }וִידוּנוּן, לשון צווי׃ }\rashi{\rashiDH{והקל מעליך. }דבר זה להקל מעליך. והקל, כמו וְהַכְבֵּד אֶת לִבֹּו (שמות ח, יא), וְהַכֹּות אֶת מֹואָב (מלכים־ב ג, כד), לשון הווה׃ }}
{וִידִינוּן יָת עַמָּא בְּכָל עִדָּן וִיהֵי כָּל פִּתְגָם רַב יַיְתוֹן לְוָתָךְ וְכָל פִּתְגָם זְעֵיר יְדִינוּן אִנּוּן וְיֵיקְלוּן מִנָּךְ וִיסוֹבְרוּן עִמָּךְ׃}
{And let them judge the people at all seasons; and it shall be, that every great matter they shall bring unto thee, but every small matter they shall judge themselves; so shall they make it easier for thee and bear the burden with thee.}{\arabic{verse}}
\threeverse{\arabic{verse}}%Ex.18:23
{אִ֣ם אֶת־הַדָּבָ֤ר הַזֶּה֙ תַּעֲשֶׂ֔ה וְצִוְּךָ֣ אֱלֹהִ֔ים וְיָֽכׇלְתָּ֖ עֲמֹ֑ד וְגַם֙ כׇּל־הָעָ֣ם הַזֶּ֔ה עַל־מְקֹמ֖וֹ יָבֹ֥א בְשָׁלֽוֹם׃
\rashi{\rashiDH{וצוך אלהים ויכלת עמוד. }המלך בגבורה, אם יצוה אותך לעשות כך תוכל עמוד, ואם יעכב על ידך לא תוכל לעמוד (מכילתא פ״ב)׃ }\rashi{\rashiDH{וגם כל העם הזה. }אהרן נדב ואביהוא, ושבעים זקנים הנלוים עתה עמך (מכילתא שם)׃ }}
{אִם יָת פִּתְגָמָא הָדֵין תַּעֲבֵיד וִיפַקְּדִנָּךְ יְיָ וְתִכּוֹל לִמְקָם וְאַף כָּל עַמָּא הָדֵין עַל אַתְרֵיהּ יְהָךְ בִּשְׁלָם׃}
{If thou shalt do this thing, and God command thee so, then thou shalt be able to endure, and all this people also shall go to their place in peace.’}{\arabic{verse}}
\threeverse{\aliya{שלישי}}%Ex.18:24
{וַיִּשְׁמַ֥ע מֹשֶׁ֖ה לְק֣וֹל חֹתְנ֑וֹ וַיַּ֕עַשׂ כֹּ֖ל אֲשֶׁ֥ר אָמָֽר׃}
{וְקַבֵּיל מֹשֶׁה לְמֵימַר חֲמוּהִי וַעֲבַד כֹּל דַּאֲמַר׃}
{So Moses hearkened to the voice of his father-in-law, and did all that he had said.}{\arabic{verse}}
\threeverse{\arabic{verse}}%Ex.18:25
{וַיִּבְחַ֨ר מֹשֶׁ֤ה אַנְשֵׁי־חַ֙יִל֙ מִכׇּל־יִשְׂרָאֵ֔ל וַיִּתֵּ֥ן אֹתָ֛ם רָאשִׁ֖ים עַל־הָעָ֑ם שָׂרֵ֤י אֲלָפִים֙ שָׂרֵ֣י מֵא֔וֹת שָׂרֵ֥י חֲמִשִּׁ֖ים וְשָׂרֵ֥י עֲשָׂרֹֽת׃}
{וּבְחַר מֹשֶׁה גּוּבְרִין דְּחֵילָא מִכָּל יִשְׂרָאֵל וּמַנִּי יָתְהוֹן רֵישִׁין עַל עַמָּא רַבָּנֵי אַלְפֵי רַבָּנֵי מָאוָתָא רַבָּנֵי חַמְשִׁין וְרַבָּנֵי עֲשׂוֹרְיָיתָא׃}
{And Moses chose able men out of all Israel, and made them heads over the people, rulers of thousands, rulers of hundreds, rulers of fifties, and rulers of tens.}{\arabic{verse}}
\threeverse{\arabic{verse}}%Ex.18:26
{וְשָׁפְט֥וּ אֶת־הָעָ֖ם בְּכׇל־עֵ֑ת אֶת־הַדָּבָ֤ר הַקָּשֶׁה֙ יְבִיא֣וּן אֶל־מֹשֶׁ֔ה וְכׇל־הַדָּבָ֥ר הַקָּטֹ֖ן יִשְׁפּוּט֥וּ הֵֽם׃
\rashi{\rashiDH{ושפטו. }וְדָיְינוּן יָת עַמָּא׃}\rashi{\rashiDH{יביאון. }מָיְיתִין׃}\rashi{\rashiDH{ישפוטו הם. }כמו ישפטו (בחולם) וכן לֹא תַעֲבוּרִי (רות ב, ח), כמו לא תעברי. ותרגומו דָּיְינִין אִינוּן. מקראות העליונים היו לשון צווי, לכך מתורגמין וִידוּנוּן, יֵיתוּן, יְדוּנוּן, ומקראות הללו לשון עשייה׃ }}
{וְדָיְנִין יָת עַמָּא בְּכָל עִדָּן יָת פִּתְגָם קְשֵׁי מֵיתַן לְוָת מֹשֶׁה וְכָל פִּתְגָם זְעֵיר דָּיְנִין אִנּוּן׃}
{And they judged the people at all seasons: the hard causes they brought unto Moses, but every small matter they judged themselves.}{\arabic{verse}}
\threeverse{\arabic{verse}}%Ex.18:27
{וַיְשַׁלַּ֥ח מֹשֶׁ֖ה אֶת־חֹתְנ֑וֹ וַיֵּ֥לֶךְ ל֖וֹ אֶל־אַרְצֽוֹ׃ \petucha 
\rashi{\rashiDH{וילך לו אל ארצו. }לגייר בני משפחתו (מכילתא פ״ב)׃ 
}}
{וְשַׁלַּח מֹשֶׁה יָת חֲמוּהִי וַאֲזַל לֵיהּ לְאַרְעֵיהּ׃}
{And Moses let his father-in-law depart; and he went his way into his own land.}{\arabic{verse}}
\newperek
\threeverse{\aliya{רביעי}}%Ex.19:1
{בַּחֹ֙דֶשׁ֙ הַשְּׁלִישִׁ֔י לְצֵ֥את בְּנֵי־יִשְׂרָאֵ֖ל מֵאֶ֣רֶץ מִצְרָ֑יִם בַּיּ֣וֹם הַזֶּ֔ה בָּ֖אוּ מִדְבַּ֥ר סִינָֽי׃
\rashi{\rashiDH{ביום הזה. }בראש חדש (שבת פו׃). לא היה צריך לכתוב אלא ביום ההוא, מהו ביום הזה, שיהיו דברי תורה חדשים עליך כאילו היום ניתנו׃ }}
{בְּיַרְחָא תְּלִיתָאָה לְמִפַּק בְּנֵי יִשְׂרָאֵל מֵאַרְעָא דְּמִצְרָיִם בְּיוֹמָא הָדֵין אֲתוֹ לְמַדְבְּרָא דְּסִינָי׃}
{In the third month after the children of Israel were gone forth out of the land of Egypt, the same day came they into the wilderness of Sinai.}{\Roman{chap}}
\threeverse{\arabic{verse}}%Ex.19:2
{וַיִּסְע֣וּ מֵרְפִידִ֗ים וַיָּבֹ֙אוּ֙ מִדְבַּ֣ר סִינַ֔י וַֽיַּחֲנ֖וּ בַּמִּדְבָּ֑ר וַיִּֽחַן־שָׁ֥ם יִשְׂרָאֵ֖ל נֶ֥גֶד הָהָֽר׃
\rashi{\rashiDH{ויסעו מרפידים. }למה הוצרך לחזור ולפרש מהיכן נסעו, והלא כבר כתב שברפידים היו חונים, בידוע שמשם נסעו, אלא להקיש נסיעתן מרפידים לביאתן למדבר סיני, מה ביאתן למדבר סיני בתשובה, אף נסיעתן מרפידים בתשובה (מכילתא בחדש פ״א)׃ }\rashi{\rashiDH{ויחן שם ישראל. }כאיש אחד בלב אחד, אבל שאר כל החניות בתרעומות ובמחלוקת (מכילתא שם)׃ }\rashi{\rashiDH{נגד ההר. }למזרחו, וכל מקום שאתה מוצא נגד, פנים למזרח (מכילתא שם)׃ }}
{וּנְטַלוּ מֵרְפִידִים וַאֲתוֹ לְמַדְבְּרָא דְּסִינַי וּשְׁרוֹ בְּמַדְבְּרָא וּשְׁרָא תַּמָּן יִשְׂרָאֵל לָקֳבֵיל טוּרָא׃}
{And when they were departed from Rephidim, and were come to the wilderness of Sinai, they encamped in the wilderness; and there Israel encamped before the mount.}{\arabic{verse}}
\threeverse{\arabic{verse}}%Ex.19:3
{וּמֹשֶׁ֥ה עָלָ֖ה אֶל־הָאֱלֹהִ֑ים וַיִּקְרָ֨א אֵלָ֤יו יְהֹוָה֙ מִן־הָהָ֣ר לֵאמֹ֔ר כֹּ֤ה תֹאמַר֙ לְבֵ֣ית יַעֲקֹ֔ב וְתַגֵּ֖יד לִבְנֵ֥י יִשְׂרָאֵֽל׃
\rashi{\rashiDH{ומשה עלה. }ביום השני, וכל עליותיו בהשכמה היו, שנאמר וַיַּשְׁכֵּם משֶׁה בַבֹּקֶר (שמות לד, ד)׃ 
}\rashi{\rashiDH{כה תאמר. }בלשון הזה וכסדר הזה׃}\rashi{\rashiDH{לבית יעקב. }אלו הנשים, תאמר להם בלשון רכה׃ }\rashi{\rashiDH{ותגיד לבני ישראל. }עונשין ודקדוקין פירש לזכרים, דברים הקשין כגידין (שבת פז.  מכילתא בחדש פ״ב)׃ }}
{וּמֹשֶׁה סְלֵיק לִקְדָם יְיָ וּקְרָא לֵיהּ יְיָ מִן טוּרָא לְמֵימַר כְּדֵין תֵּימַר לְבֵית יַעֲקֹב וּתְחַוֵּי לִבְנֵי יִשְׂרָאֵל׃}
{And Moses went up unto God, and the \lord\space called unto him out of the mountain, saying: ‘Thus shalt thou say to the house of Jacob, and tell the children of Israel:}{\arabic{verse}}
\threeverse{\arabic{verse}}%Ex.19:4
{אַתֶּ֣ם רְאִיתֶ֔ם אֲשֶׁ֥ר עָשִׂ֖יתִי לְמִצְרָ֑יִם וָאֶשָּׂ֤א אֶתְכֶם֙ עַל־כַּנְפֵ֣י נְשָׁרִ֔ים וָאָבִ֥א אֶתְכֶ֖ם אֵלָֽי׃
\rashi{\rashiDH{אתם ראיתם. }לא מסורת היא בידכם, ולא בדברים אני משגר לכם, לא בעדים אני מעיד עליכם, אלא אתם ראיתם אשר עשיתי למצרים, על כמה עבירות היו חייבין לי קודם שנזדווגו לכם, ולא נפרעתי מהם אלא על ידכם׃ }\rashi{\rashiDH{ואשא אתכם. }זה יום שבאו ישראל לרעמסס, שהיו ישראל מפוזרין בכל ארץ גושן, ולשעה קלה כשבאו ליסע ולצאת, נקבצו כלם לרעמסס (מכילתא פ״ב). ואונקלוס תרגם ואשא, וְאַטְלִית יָתְכוֹן, כמו ואסיע אתכם, תיקן את הדבור דרך כבוד למעלה׃ }\rashi{\rashiDH{על כנפי נשרים. }כנשר הנושא גוזליו על כנפיו, שכל שאר העופות נותנים את בניהם בין רגליהם, לפי שמתיראין מעוף אחר שפורח על גביהם, אבל הנשר הזה אינו מתירא אלא מן האדם שמא יזרוק בו חץ, לפי שאין עוף פורח על גביו, לכך נותנו על כנפיו אומר מוטב יכנס החץ בי ולא בבני, אף אני עשיתי כן, וַיִּסַּע מַלְאַךְ הָאֳלֹהִים וגו׳ וַיָּבֹא בֵּין מַחֲנֵה מִצְרַיִם וגו׳ (שמות יד, יטכ), והיו מצרים זורקים חצים ואבני בְּלִיסְטְרָאוֹת, והענן מקבלם׃ }\rashi{\rashiDH{ואבא אתכם אלי. }כתרגומו׃}}
{אַתּוּן חֲזֵיתוֹן דַּעֲבַדִית לְמִצְרָאֵי וְנַטֵּילִית יָתְכוֹן כִּד עַל גַּדְפֵּי נִשְׁרִין וְקָרֵיבִית יָתְכוֹן לְפוּלְחָנִי׃}
{Ye have seen what I did unto the Egyptians, and how I bore you on eagles’ wings, and brought you unto Myself.}{\arabic{verse}}
\threeverse{\arabic{verse}}%Ex.19:5
{וְעַתָּ֗ה אִם־שָׁמ֤וֹעַ תִּשְׁמְעוּ֙ בְּקֹלִ֔י וּשְׁמַרְתֶּ֖ם אֶת־בְּרִיתִ֑י וִהְיִ֨יתֶם לִ֤י סְגֻלָּה֙ מִכׇּל־הָ֣עַמִּ֔ים כִּי־לִ֖י כׇּל־הָאָֽרֶץ׃
\rashi{\rashiDH{ועתה. }אם עתה תקבלו עליכם, יערב לכם מכאן ואילך, שכל התחלות קשות (מכילתא פ״ב)׃ }\rashi{\rashiDH{ושמרתם את בריתי. }שאכרות עמכם על שמירת התורה׃}\rashi{\rashiDH{סגלה. }אוצר חביב, כמו וּסְגֻלַּת מְלָכִים (קהלת ב, ח), כלי יקר ואבנים טובות שהמלכים גונזים אותם, כך אתם תהיו לי סגולה משאר אומות, ולא תאמרו אתם לבדכם שלי ואין לי אחרים עמכם, ומה יש לי עוד שתהא חבתכם נכרת, כי לי כל הארץ, והם בעיני ולפני לכלום׃ }}
{וּכְעַן אִם קַבָּלָא תְקַבְּלוּן לְמֵימְרִי וְתִטְּרוּן יָת קְיָמִי וּתְהוֹן קֳדָמַי חַבִּיבִין מִכָּל עַמְמַיָּא אֲרֵי דִּילִי כָל אַרְעָא׃}
{Now therefore, if ye will hearken unto My voice indeed, and keep My covenant, then ye shall be Mine own treasure from among all peoples; for all the earth is Mine;}{\arabic{verse}}
\threeverse{\arabic{verse}}%Ex.19:6
{וְאַתֶּ֧ם תִּהְיוּ־לִ֛י מַמְלֶ֥כֶת כֹּהֲנִ֖ים וְג֣וֹי קָד֑וֹשׁ אֵ֚לֶּה הַדְּבָרִ֔ים אֲשֶׁ֥ר תְּדַבֵּ֖ר אֶל־בְּנֵ֥י יִשְׂרָאֵֽל׃
\rashi{\rashiDH{ואתם תהיו לי ממלכת כהנים. }שרים, כְּמָה דְאַתְּ אָמַר, וּבְנֵי דָוִד כֹּהֲנִים הָיוּ (שמואל־ב ח, יח)׃ }\rashi{\rashiDH{אלה הדברים. }לא פחות ולא יותר׃}}
{וְאַתּוּן תְּהוֹן קֳדָמַי מַלְכִין כָּהֲנִין וְעַם קַדִּישׁ אִלֵּין פִּתְגָמַיָּא דִּתְמַלֵּיל עִם בְּנֵי יִשְׂרָאֵל׃}
{and ye shall be unto Me a kingdom of priests, and a holy nation. These are the words which thou shalt speak unto the children of Israel.’}{\arabic{verse}}
\threeverse{\aliya{חמישי}}%Ex.19:7
{וַיָּבֹ֣א מֹשֶׁ֔ה וַיִּקְרָ֖א לְזִקְנֵ֣י הָעָ֑ם וַיָּ֣שֶׂם לִפְנֵיהֶ֗ם אֵ֚ת כׇּל־הַדְּבָרִ֣ים הָאֵ֔לֶּה אֲשֶׁ֥ר צִוָּ֖הוּ יְהֹוָֽה׃}
{וַאֲתָא מֹשֶׁה וּקְרָא לְסָבֵי עַמָּא וְסַדַּר קֳדָמֵיהוֹן יָת כָּל פִּתְגָמַיָּא הָאִלֵּין דְּפַקְּדֵיהּ יְיָ׃}
{And Moses came and called for the elders of the people, and set before them all these words which the \lord\space commanded him.}{\arabic{verse}}
\threeverse{\arabic{verse}}%Ex.19:8
{וַיַּעֲנ֨וּ כׇל־הָעָ֤ם יַחְדָּו֙ וַיֹּ֣אמְר֔וּ כֹּ֛ל אֲשֶׁר־דִּבֶּ֥ר יְהֹוָ֖ה נַעֲשֶׂ֑ה וַיָּ֧שֶׁב מֹשֶׁ֛ה אֶת־דִּבְרֵ֥י הָעָ֖ם אֶל־יְהֹוָֽה׃
\rashi{\rashiDH{וישב משה את דברי העם וגו׳. }ביום המחרת שהוא יום שלישי, שהרי בהשכמה עלה. וכי צריך היה משה להשיב, אלא בא הכתוב ללמדך דרך ארץ ממשה, שלא אמר הואיל ויודע מי ששלחני, איני צריך להשיב׃ }}
{וַאֲתִיבוּ כָל עַמָּא כַּחְדָּא וַאֲמַרוּ כֹּל דְּמַלֵּיל יְיָ נַעֲבֵיד וַאֲתֵיב מֹשֶׁה יָת פִּתְגָמֵי עַמָּא לִקְדָם יְיָ׃}
{And all the people answered together, and said: ‘All that the \lord\space hath spoken we will do.’ And Moses reported the words of the people unto the \lord.}{\arabic{verse}}
\threeverse{\arabic{verse}}%Ex.19:9
{וַיֹּ֨אמֶר יְהֹוָ֜ה אֶל־מֹשֶׁ֗ה הִנֵּ֨ה אָנֹכִ֜י בָּ֣א אֵלֶ֘יךָ֮ בְּעַ֣ב הֶֽעָנָן֒ בַּעֲב֞וּר יִשְׁמַ֤ע הָעָם֙ בְּדַבְּרִ֣י עִמָּ֔ךְ וְגַם־בְּךָ֖ יַאֲמִ֣ינוּ לְעוֹלָ֑ם וַיַּגֵּ֥ד מֹשֶׁ֛ה אֶת־דִּבְרֵ֥י הָעָ֖ם אֶל־יְהֹוָֽה׃
\rashi{\rashiDH{בעב הענן. }במעבה הענן, וזהו ערפל׃ }\rashi{\rashiDH{וגם בך. }גם בנביאים הבאים אחריך׃}\rashi{\rashiDH{ויגד משה וגו׳. }ביום המחרת שהוא רביעי לחדש׃}\rashi{\rashiDH{את דברי העם וגו׳. }תשובה על דבר זה שמעתי מהם, שרצונם לשמוע ממך, אינו דומה השומע מפי שליח לשומע מפי המלך, רצוננו לראות את מלכנו׃ 
}}
{וַאֲמַר יְיָ לְמֹשֶׁה הָא אֲנָא מִתְגְּלֵי לָךְ בְּעֵיבָא דַּעֲנָנָא בְּדִיל דְּיִשְׁמַע עַמָּא בְּמַלָּלוּתִי עִמָּךְ וְאַף בָּךְ יְהֵימְנוּן לְעָלַם וְחַוִּי מֹשֶׁה יָת פִּתְגָמֵי עַמָּא לִקְדָם יְיָ׃}
{And the \lord\space said unto Moses: ‘Lo, I come unto thee in a thick cloud, that the people may hear when I speak with thee, and may also believe thee for ever.’ And Moses told the words of the people unto the \lord.}{\arabic{verse}}
\threeverse{\arabic{verse}}%Ex.19:10
{וַיֹּ֨אמֶר יְהֹוָ֤ה אֶל־מֹשֶׁה֙ לֵ֣ךְ אֶל־הָעָ֔ם וְקִדַּשְׁתָּ֥ם הַיּ֖וֹם וּמָחָ֑ר וְכִבְּס֖וּ שִׂמְלֹתָֽם׃
\rashi{\rashiDH{ויאמר ה׳ אל משה. }אם כן שמזקיקין לְדַבֵּר עמם, לך אל העם׃ }\rashi{\rashiDH{וקדשתם. }וזימנתם, שיכינו עצמם היום ומחר׃ }}
{וַאֲמַר יְיָ לְמֹשֶׁה אִיזֵיל לְוָת עַמָּא וּתְזָמֵינִנּוּן יוֹמָא דֵין וּמְחַר וִיחַוְּרוּן לְבוּשֵׁיהוֹן׃}
{And the \lord\space said unto Moses: ‘Go unto the people, and sanctify them to-day and to-morrow, and let them wash their garments,}{\arabic{verse}}
\threeverse{\arabic{verse}}%Ex.19:11
{וְהָי֥וּ נְכֹנִ֖ים לַיּ֣וֹם הַשְּׁלִישִׁ֑י כִּ֣י \legarmeh  בַּיּ֣וֹם הַשְּׁלִשִׁ֗י יֵרֵ֧ד יְהֹוָ֛ה לְעֵינֵ֥י כׇל־הָעָ֖ם עַל־הַ֥ר סִינָֽי׃
\rashi{\rashiDH{והיו נכונים. }מובדלים מאשה (מכילתא פ״ג)׃ }\rashi{\rashiDH{ליום השלישי. }שהוא ששה בחדש, ובחמישי בנה משה את המזבח תחת ההר ושתים עשרה מצבה (מכילתא שם), כל הענין האמור בפרשת ואלה המשפטים, ואין מוקדם ומאוחר בתורה׃ }\rashi{\rashiDH{לעיני כל העם. }מלמד, שלא היה בהם סומא, שנתרפאו כולם (מכילתא שם)׃ }}
{וִיהוֹן זְמִינִין לְיוֹמָא תְּלִיתָאָה אֲרֵי בְּיוֹמָא תְּלִיתָאָה יִתְגְּלֵי יְיָ לְעֵינֵי כָל עַמָּא עַל טוּרָא דְּסִינָי׃}
{and be ready against the third day; for the third day the \lord\space will come down in the sight of all the people upon mount Sinai.}{\arabic{verse}}
\threeverse{\arabic{verse}}%Ex.19:12
{וְהִגְבַּלְתָּ֤ אֶת־הָעָם֙ סָבִ֣יב לֵאמֹ֔ר הִשָּׁמְר֥וּ לָכֶ֛ם עֲל֥וֹת בָּהָ֖ר וּנְגֹ֣עַ בְּקָצֵ֑הוּ כׇּל־הַנֹּגֵ֥עַ בָּהָ֖ר מ֥וֹת יוּמָֽת׃
\rashi{\rashiDH{והגבלת. }קבע להם תחומין לסימן, שלא יקרבו מן הגבול והלאה׃ }\rashi{\rashiDH{לאמר. }הגבול אומר להם השמרו מעלות מכאן והלאה, ואתה תזהירם על כך׃ 
}\rashi{\rashiDH{ונגע בקצהו. }אפילו בקצהו׃}}
{וּתְתַּחֵים יָת עַמָּא סְחוֹר סְחוֹר לְמֵימַר אִסְתְּמַרוּ לְכוֹן מִלְּמִסַּק בְּטוּרָא וּלְמִקְרַב בְּסוֹפֵיהּ כָּל דְּיִקְרַב בְּטוּרָא אִתְקְטָלָא יִתְקְטִיל׃}
{And thou shalt set bounds unto the people round about, saying: Take heed to yourselves, that ye go not up into the mount, or touch the border of it; whosoever toucheth the mount shall be surely put to death;}{\arabic{verse}}
\threeverse{\arabic{verse}}%Ex.19:13
{לֹא־תִגַּ֨ע בּ֜וֹ יָ֗ד כִּֽי־סָק֤וֹל יִסָּקֵל֙ אוֹ־יָרֹ֣ה יִיָּרֶ֔ה אִם־בְּהֵמָ֥ה אִם־אִ֖ישׁ לֹ֣א יִחְיֶ֑ה בִּמְשֹׁךְ֙ הַיֹּבֵ֔ל הֵ֖מָּה יַעֲל֥וּ בָהָֽר׃
\rashi{\rashiDH{ירה יירה. }מכאן לנסקלין שהם נדחין למטה (סנהדרין מה.) מבית הסקילה שהיה גבוה שתי קומות׃}\rashi{\rashiDH{יירה. }יושלך למטה לארץ, כמו יָרָה בַיָּם (שמות טו, ד)׃ }\rashi{\rashiDH{במשך היובל. }כשימשוך היובל קול ארוך, הוא סימן סלוק שכינה והפסקת הקול, וכיון שנסתלק הם רשאין לעלות׃ }\rashi{\rashiDH{היובל. }הוא שופר של איל, שכן בערביא קורין לְדִכְרָא יוֹבָלָא. ושופר של אילו של יצחק היה׃ }}
{לָא תִקְרַב בֵּיהּ יַד אֲרֵי אִתְרְגָמָא יִתְרְגֵים אוֹ אִשְׁתְּדָאָה יִשְׁתְּדֵי אִם בְּעִירָא אִם אֲנָשָׁא לָא יִתְקַיַּים בְּמֵיגַד שׁוֹפָרָא אִנּוּן מוּרְשַׁן לְמִסַּק בְּטוּרָא׃}
{no hand shall touch him, but he shall surely be stoned, or shot through; whether it be beast or man, it shall not live; when the ram’s horn soundeth long, they shall come up to the mount.’}{\arabic{verse}}
\threeverse{\arabic{verse}}%Ex.19:14
{וַיֵּ֧רֶד מֹשֶׁ֛ה מִן־הָהָ֖ר אֶל־הָעָ֑ם וַיְקַדֵּשׁ֙ אֶת־הָעָ֔ם וַֽיְכַבְּס֖וּ שִׂמְלֹתָֽם׃
\rashi{\rashiDH{מן ההר אל העם. }מלמד שלא היה משה פונה לעסקיו, אלא מן ההר אל העם׃ }}
{וּנְחַת מֹשֶׁה מִן טוּרָא לְוָת עַמָּא וְזָמֵין יָת עַמָּא וְחַוַּרוּ לְבוּשֵׁיהוֹן׃}
{And Moses went down from the mount unto the people, and sanctified the people; and they washed their garments.}{\arabic{verse}}
\threeverse{\arabic{verse}}%Ex.19:15
{וַיֹּ֙אמֶר֙ אֶל־הָעָ֔ם הֱי֥וּ נְכֹנִ֖ים לִשְׁלֹ֣שֶׁת יָמִ֑ים אַֽל־תִּגְּשׁ֖וּ אֶל־אִשָּֽׁה׃
\rashi{\rashiDH{היו נכונים לשלשת ימים. }לסוף שלשת ימים, הוא יום רביעי, שהוסיף משה יום אחד מדעתו, כדברי רבי יוסי (שבת פז.), ולדברי האומר בששה בחדש ניתנו עשרת הדברות, לא הוסיף משה כלום, ולשלשת ימים, כמו ליום השלישי׃ 
}\rashi{\rashiDH{אל תגשו אל אשה. }כל שלשת ימים הללו, כדי שיהיו הנשים טובלות ליום השלישי ותהיינה טהורות לקבל תורה, שאם ישמש תוך ג׳ ימים, שמא תפלוט האשה שכבת זרע לאחר טבילתה ותחזור ותטמא, אבל מששהתה שלשה ימים כבר הזרע מסריח ואינו ראוי להזריע, וטהור מלטמא את הפולטת׃ }}
{וַאֲמַר לְעַמָּא הֲווֹ זְמִינִין לִתְלָתָא יוֹמִין לָא תִקְרְבוּן לְצַד אִתְּתָא׃}
{And he said unto the people: ‘Be ready against the third day; come not near a woman.’}{\arabic{verse}}
\threeverse{\arabic{verse}}%Ex.19:16
{וַיְהִי֩ בַיּ֨וֹם הַשְּׁלִישִׁ֜י בִּֽהְיֹ֣ת הַבֹּ֗קֶר וַיְהִי֩ קֹלֹ֨ת וּבְרָקִ֜ים וְעָנָ֤ן כָּבֵד֙ עַל־הָהָ֔ר וְקֹ֥ל שֹׁפָ֖ר חָזָ֣ק מְאֹ֑ד וַיֶּחֱרַ֥ד כׇּל־הָעָ֖ם אֲשֶׁ֥ר בַּֽמַּחֲנֶֽה׃
\rashi{\rashiDH{בהיות הבקר. }מלמד שהקדים על ידם, מה שאין דרך בשר ודם לעשות כן שיהא הרב ממתין לתלמיד, וכן מצינו קוּם צֵא אֶל הַבִּקְעָה וגו׳ (יחזקאל ג, כב), וָאָקוּם וָאֵצֵא אֶל הַבִּקְעָה וְהִנֵּה שָׁם כְּבֹוד ה׳ עֹמֵד (שם כג)׃ }}
{וַהֲוָה בְּיוֹמָא תְּלִיתָאָה בְּמִהְוֵי צַפְרָא וַהֲווֹ קָלִין וּבַרְקִין וַעֲנָנָא תַּקִּיף עַל טוּרָא וְקָל שׁוֹפָרָא תַּקִּיף לַחְדָּא וְזָע כָּל עַמָּא דִּבְמַשְׁרִיתָא׃}
{And it came to pass on the third day, when it was morning, that there were thunders and lightnings and a thick cloud upon the mount, and the voice of a horn exceeding loud; and all the people that were in the camp trembled.}{\arabic{verse}}
\threeverse{\arabic{verse}}%Ex.19:17
{וַיּוֹצֵ֨א מֹשֶׁ֧ה אֶת־הָעָ֛ם לִקְרַ֥את הָֽאֱלֹהִ֖ים מִן־הַֽמַּחֲנֶ֑ה וַיִּֽתְיַצְּב֖וּ בְּתַחְתִּ֥ית הָהָֽר׃
\rashi{\rashiDH{לקראת האלהים. }מגיד שהשכינה יצאה לקראתם כחתן היוצא לקראת כלה, וזה שנאמר ה׳ מִסִּינַי בָּא (דברים לג, ב.  מכילתא פ״ג), ולא נאמר לסיני בא׃ 
}\rashi{\rashiDH{בתחתית ההר. }לפי פשוטו ברגלי ההר. ומדרשו, שנתלש ההר ממקומו ונכפה עליהם כגיגית (שבת פח.)׃ }}
{וְאַפֵּיק מֹשֶׁה יָת עַמָּא לְקַדָּמוּת מֵימְרָא דַּייָ מִן מַשְׁרִיתָא וְאִתְעַתַּדוּ בְּשִׁפּוֹלֵי טוּרָא׃}
{And Moses brought forth the people out of the camp to meet God; and they stood at the nether part of the mount.}{\arabic{verse}}
\threeverse{\arabic{verse}}%Ex.19:18
{וְהַ֤ר סִינַי֙ עָשַׁ֣ן כֻּלּ֔וֹ מִ֠פְּנֵ֠י אֲשֶׁ֨ר יָרַ֥ד עָלָ֛יו יְהֹוָ֖ה בָּאֵ֑שׁ וַיַּ֤עַל עֲשָׁנוֹ֙ כְּעֶ֣שֶׁן הַכִּבְשָׁ֔ן וַיֶּחֱרַ֥ד כׇּל־הָהָ֖ר מְאֹֽד׃
\rashi{\rashiDH{עשן כלו. }אין עשן זה שם דבר, שהרי נקוד השי״ן פת״ח, אלא לשון פעל, כמו אמר, שמר, שמע, לכך תרגומו תָּנַן כֻּלֵּיהּ ולא תרגם תְּנָנָא, וכל עשן שבמקרא נקודים קמ״ץ, מפני שהם שם דבר׃ }\rashi{\rashiDH{הכבשן. }של סיד, יכול ככבשן זה ולא יותר, תלמוד לומר בוער באש עד לב השמים, ומה תלמוד לומר כבשן, לְשַׂבֵּר את האוזן מה שהיא יכולה לשמוע, נותן לבריות סימן הניכר להם. כיוצא בו כְּאַרְיֵּה יִשְׁאָג (הושע יא, י), וכי מי נתן כח בארי אלא הוא, והכתוב מושלו כאריה, אלא אנו מכנין ומדמין אותו לבריותיו, כדי לְשַׂבֵּר את האוזן מה שיכולה לשמוע. כיוצא בו וְקֹולֹו כְּקֹול מַיִם רַבִּים (יחזקאל מג, ב), וכי מי נתן קול למים והלא הוא, ואתה מכנה אותו לדמותו לבריותיו כדי לְשַׂבֵּר את האוזן׃ }}
{וְטוּרָא דְּסִינַי תָּנַן כּוּלֵּיהּ מִן קֳדָם דְּאִתְגְּלִי עֲלוֹהִי יְיָ בְּאִישָׁתָא וּסְלֵיק תַּנְנֵיהּ כְּתַנְנָא דְּאַתּוּנָא וְזָע כָּל טוּרָא לַחְדָּא׃}
{Now mount Sinai was altogether on smoke, because the \lord\space descended upon it in fire; and the smoke thereof ascended as the smoke of a furnace, and the whole mount quaked greatly.}{\arabic{verse}}
\threeverse{\arabic{verse}}%Ex.19:19
{וַיְהִי֙ ק֣וֹל הַשֹּׁפָ֔ר הוֹלֵ֖ךְ וְחָזֵ֣ק מְאֹ֑ד מֹשֶׁ֣ה יְדַבֵּ֔ר וְהָאֱלֹהִ֖ים יַעֲנֶ֥נּוּ בְקֽוֹל׃
\rashi{\rashiDH{הולך וחזק מאד. }מנהג הדיוט כל זמן שהוא מאריך לתקוע קולו מחליש וכוהה, אבל כאן הולך וחזק מאד, ולמה כך, מתחלה לְשַׂבֵּר אזניהם מה שיכולין לשמוע׃ }\rashi{\rashiDH{משה ידבר. }כשהיה משה מדבר ומשמיע הדברות לישראל, שהרי לא שמעו מפי הגבורה אלא אנכי ולא יהיה לך, והקדוש ברוך הוא מסייעו לתת בו כח להיות קולו מגביר ונשמע׃ }\rashi{\rashiDH{יעננו בקול. }יעננו על דבר הקול, כמו אֲשֶׁר יַעֲנֶה בָאֵשׁ (מלכים־א יח, כד), על דבר האש להורידו׃ }}
{וַהֲוָה קָל שׁוֹפָרָא אָזֵיל וְתָקֵיף לַחְדָּא מֹשֶׁה מְמַלֵּיל וּמִן קֳדָם יְיָ מִתְעֲנֵי לֵיהּ בְּקָל׃}
{And when the voice of the horn waxed louder and louder, Moses spoke, and God answered him by a voice.}{\arabic{verse}}
\threeverse{\aliya{ששי}}%Ex.19:20
{וַיֵּ֧רֶד יְהֹוָ֛ה עַל־הַ֥ר סִינַ֖י אֶל־רֹ֣אשׁ הָהָ֑ר וַיִּקְרָ֨א יְהֹוָ֧ה לְמֹשֶׁ֛ה אֶל־רֹ֥אשׁ הָהָ֖ר וַיַּ֥עַל מֹשֶֽׁה׃
\rashi{\rashiDH{וירד ה׳ על הר סיני. }יכול ירד עליו ממש, תלמוד לומר כִּי מִן הַשָּׁמַיִם דִּבַּרְתִּי עִמָּכֶם (שמות כ, יט), למד שֶׁהִרְכִּין שמים עליונים ותחתונים, והציען על גבי ההר כמצע על המטה, וירד כסא הכבוד עליהם (מכילתא פ״ד)׃ }}
{וְאִתְגְּלִי יְיָ עַל טוּרָא דְּסִינַי לְרֵישׁ טוּרָא וּקְרָא יְיָ לְמֹשֶׁה לְרֵישׁ טוּרָא וּסְלֵיק מֹשֶׁה׃}
{And the \lord\space came down upon mount Sinai, to the top of the mount; and the \lord\space called Moses to the top of the mount; and Moses went up.}{\arabic{verse}}
\threeverse{\arabic{verse}}%Ex.19:21
{וַיֹּ֤אמֶר יְהֹוָה֙ אֶל־מֹשֶׁ֔ה רֵ֖ד הָעֵ֣ד בָּעָ֑ם פֶּן־יֶהֶרְס֤וּ אֶל־יְהֹוָה֙ לִרְא֔וֹת וְנָפַ֥ל מִמֶּ֖נּוּ רָֽב׃
\rashi{\rashiDH{העד בעם. }התרה בהם שלא לעלות בהר׃}\rashi{\rashiDH{פן יהרסו וגו׳. }שלא יהרסו את מצבם, על ידי שֶׁתַּאֲוָתָם אל ה׳ לראות, ויקרבו לצד ההר׃ }\rashi{\rashiDH{ונפל ממנו רב. }כל מה שיפול מהם, ואפילו הוא יחידי חשוב לפני רב (מכילתא שם)׃ }\rashi{\rashiDH{יהרסו. }כל הריסה מפרדת אסיפת הבנין, אף הנפרדין ממצב אנשים הורסים את המצב׃ }}
{וַאֲמַר יְיָ לְמֹשֶׁה חוֹת אַסְהֵיד בְּעַמָּא דִּלְמָא יְפַגְּרוּן קֳדָם יְיָ לְמִחְזֵי וְיִפּוֹל מִנְּהוֹן סַגִּי׃}
{And the \lord\space said unto Moses: ‘Go down, charge the people, lest they break through unto the \lord\space to gaze, and many of them perish.}{\arabic{verse}}
\threeverse{\arabic{verse}}%Ex.19:22
{וְגַ֧ם הַכֹּהֲנִ֛ים הַנִּגָּשִׁ֥ים אֶל־יְהֹוָ֖ה יִתְקַדָּ֑שׁוּ פֶּן־יִפְרֹ֥ץ בָּהֶ֖ם יְהֹוָֽה׃
\rashi{\rashiDH{וגם הכהנים. }אף הבכורות שהעבודה בהם (זבחים קטו׃)׃}\rashi{\rashiDH{הנגשים אל ה׳. }להקריב קרבנות, אף הם אל יסמכו על חשיבותם לעלות׃ 
}\rashi{\rashiDH{יתקדשו. }יהיו מזומנים להתיצב על עמדן׃}\rashi{\rashiDH{פן יפרץ. }לשון פרצה, יהרוג בהם ויעשה בהם פרצה׃ }}
{וְאַף כָּהֲנַיָּא דְּקָרִיבִין לְשַׁמָּשָׁא קֳדָם יְיָ יִתְקַדְּשׁוּן דִּלְמָא יִקְטוֹל בְּהוֹן יְיָ׃}
{And let the priests also, that come near to the \lord, sanctify themselves, lest the \lord\space break forth upon them.’}{\arabic{verse}}
\threeverse{\arabic{verse}}%Ex.19:23
{וַיֹּ֤אמֶר מֹשֶׁה֙ אֶל־יְהֹוָ֔ה לֹא־יוּכַ֣ל הָעָ֔ם לַעֲלֹ֖ת אֶל־הַ֣ר סִינָ֑י כִּֽי־אַתָּ֞ה הַעֵדֹ֤תָה בָּ֙נוּ֙ לֵאמֹ֔ר הַגְבֵּ֥ל אֶת־הָהָ֖ר וְקִדַּשְׁתּֽוֹ׃
\rashi{\rashiDH{לא יוכל העם. }איני צריך להעיד בהם, שהרי מותרין ועומדין הם היום שלשת ימים, ולא יוכלו לעלות, שאין להם רשות׃ }}
{וַאֲמַר מֹשֶׁה קֳדָם יְיָ לָא יִכּוֹל עַמָּא לְמִסַּק לְטוּרָא דְּסִינָי אֲרֵי אַתְּ אַסְהֵידְתְּ בַּנָא לְמֵימַר תַּחֵים יָת טוּרָא וְקַדֵּישְׁהִי׃}
{And Moses said unto the \lord: ‘The people cannot come up to mount Sinai; for thou didst charge us, saying: Set bounds about the mount, and sanctify it.’}{\arabic{verse}}
\threeverse{\arabic{verse}}%Ex.19:24
{וַיֹּ֨אמֶר אֵלָ֤יו יְהֹוָה֙ לֶךְ־רֵ֔ד וְעָלִ֥יתָ אַתָּ֖ה וְאַהֲרֹ֣ן עִמָּ֑ךְ וְהַכֹּהֲנִ֣ים וְהָעָ֗ם אַל־יֶֽהֶרְס֛וּ לַעֲלֹ֥ת אֶל־יְהֹוָ֖ה פֶּן־יִפְרׇץ־בָּֽם׃
\rashi{\rashiDH{לך רד. }והעד בהם שנית, שמזרזין את האדם קודם מעשה, וחוזרין ומזרזין אותו בשעת מעשה (מכילתא שם)׃ }\rashi{\rashiDH{ועלית אתה ואהרן עמך והכהנים. }יכול אף הם עמך, תלמוד לומר ועלית אתה, אמור מעתה, אתה מחיצה לעצמך, ואהרן מחיצה לעצמו, והכהנים מחיצה לעצמם, משה נגש יותר מאהרן, ואהרן יותר מן הכהנים, והעם כל עיקר אל יהרסו את מצבם לעלות אל ה׳׃ }\rashi{\rashiDH{פן יפרץ בם. }אף על פי שהוא נקוד חטף קמ״ץ, אינו זז מגזרתו, כך דרך כל תיבה שנקודתה מלאפו״ם, כשהיא באה במקף, משתנה הנקוד לחטף קמ״ץ׃ }}
{וַאֲמַר לֵיהּ יְיָ אִיזֵיל חוֹת וְתִסַּק אַתְּ וְאַהֲרֹן עִמָּךְ וְכָהֲנַיָּא וְעַמָּא לָא יְפַגְּרוּן לְמִסַּק לִקְדָם יְיָ דִּלְמָא יִקְטוֹל בְּהוֹן׃}
{And the \lord\space said unto him: ‘Go, get thee down, and thou shalt come up, thou, and Aaron with thee; but let not the priests and the people break through to come up unto the \lord, lest He break forth upon them.’}{\arabic{verse}}
\threeverse{\arabic{verse}}%Ex.19:25
{וַיֵּ֥רֶד מֹשֶׁ֖ה אֶל־הָעָ֑ם וַיֹּ֖אמֶר אֲלֵהֶֽם׃ \setuma         
\rashi{\rashiDH{ויאמר אליהם. }התראה זו׃}}
{וּנְחַת מֹשֶׁה לְוָת עַמָּא וַאֲמַר לְהוֹן׃}
{So Moses went down unto the people, and told them.}{\arabic{verse}}
\newperek
\threeverse{\Roman{chap}}%Ex.20:1
{וַיְדַבֵּ֣ר אֱלֹהִ֔ים אֵ֛ת כׇּל־הַדְּבָרִ֥ים הָאֵ֖לֶּה לֵאמֹֽר׃ \setuma         
\rashi{\rashiDH{וידבר אלהים. }אין אלהים אלא דיין, וכן הוא אומר אֱלֹהִים לֹא תְקַלֵּל (שמות כב, כז) ותרגומו דַּיָּינָא, לפי שיש פרשיות בתורה שאם עשאן אדם מקבל שכר ואם לאו אינו מקבל עליהם פורעניות, יכול אף עשרת הדברות כן, תלמוד לומר וידבר אלהים, דיין להפרע׃ }\rashi{\rashiDH{את כל הדברים האלה. }מלמד שאמר הקדוש ברוך הוא עשרת הדברות בדבור אחד, מה שאי אפשר לאדם לומר כן, אם כן מה תלמוד לומר עוד אנכי ולא יהיה לך, שחזר ופירש על כל דבור ודבור בפני עצמו. (קושיית אם כן כו׳, פירוש, לפי זה שמוכח מאת כל הדברים האלה, שגם שאר עשרת הדברות כולם אמר הקב״ה בדבור אחד, אם כן מה תלמוד לומר אנכי וגו׳, רצונו לומר מה בא להודיענו במה שפרט לשון ב׳ דברות אלו מהשאר, מדהוציאם בלשון זה שהמשמעות דוקא אלו שנים בפרט מפי הגבורה יצאו, והלא לפי זה כולם כמוהם, הקב״ה בכבודו ובעצמו דיברם. ותירץ שחזר ופירש וכו׳, ורצונו לומר, באותה החזרה, לא החזיר ללמד על הכלל יצא, אלא על הפרט אלו שתי דברות ראשונות ביחוד ודו״ק)׃ }\rashi{\rashiDH{לאמר. }מלמד שהיו עונין על הן הן ועל לאו לאו (מכילתא שם)׃}}
{וּמַלֵּיל יְיָ יָת כָּל פִּתְגָמַיָּא הָאִלֵּין לְמֵימַר׃}
{And God spoke all these words, saying:}{\Roman{chap}}
\engnote{The Ten Commandments are presented here in Ta'am Ta\d{h}ton. For the Ten Commandments in Ta'am Elyon, see page \pageref{elyon}.}
\threeverse{\arabic{verse}}%Ex.20:2
{אָֽנֹכִ֖י יְהֹוָ֣ה אֱלֹהֶ֑יךָ אֲשֶׁ֧ר הוֹצֵאתִ֛יךָ מֵאֶ֥רֶץ מִצְרַ֖יִם מִבֵּ֥ית עֲבָדִֽים׃
\rashi{\rashiDH{אשר הוצאתיך מארץ מצרים. }כדאי היא ההוצאה, שתהיו משועבדים לי. דבר אחר, לפי שנגלה בים כגבור מלחמה ונגלה כאן כזקן מלא רחמים, שנאמר וְתַחַת רַגְלָיו כְּמַעֲשֵׂה לִבְנַת הַסַּפִּיר (שמות כד, י), זו היתה לפניו בשעת השעבוד, וכעצם השמים משנגאלו, הואיל ואני משתנה במראות, אל תאמרו שתי רשויות הן (מכילתא פ״ה), אנכי הוא אשר הוצאתיך ממצרים ועל הים. דבר אחר, לפי שהיו שומעין קולות הרבה, שנאמר את הקולות, קולות באין מד׳ רוחות ומן השמים ומן הארץ, אל תאמרו רשויות הרבה הן. ולמה אמר לשון יחיד אלהיך, ליתן פתחון פה למשה ללמד סניגוריא במעשה העגל, וזה הוא שאמר לָמָה ה׳ יֶחֱרֶה אַפְּךָ בְּעַמֶּךָ (שמות לב, יא), לא להם צוית לא יהיה לכם אלהים אחרים, אלא לי לבדי׃ }\rashi{\rashiDH{מבית עבדים. }מבית פרעה שהייתם עבדים לו, או אינו אומר אלא מבית עבדים שהיו עבדים לעבדים, תלמוד לומר ויפדך מבית עבדים מיד פרעה מלך מצרים, אמור מעתה, עבדים למלך היו, ולא עבדים לעבדים׃ }}
{אֲנָא יְיָ אֱלָהָךְ דְּאַפֵּיקְתָּךְ מֵאַרְעָא דְּמִצְרַיִם מִבֵּית עַבְדּוּתָא:}
{I am the \lord\space thy God, who brought thee out of the land of Egypt, out of the house of bondage.}{\arabic{verse}}
\threeverse{\arabic{verse}}%Ex.20:3
{לֹֽא־יִהְיֶ֥ה לְךָ֛ אֱלֹהִ֥ים אֲחֵרִ֖ים עַל־פָּנָֽי׃
\rashi{\rashiDH{לא יהיה לך. }למה נאמר, לפי שנאמר לא תעשה לך, אין לי אלא שלא יעשה, העשוי כבר מנין שלא יקיים, תלמוד לומר לא יהיה לך (מכילתא פ״ו)׃ }\rashi{\rashiDH{אלהים אחרים. }שאינן אלהות, אלא אחרים עשאום אלהים עליהם (מכילתא שם). ולא יתכן לפרש אלהים אחרים זולתי, שגנאי הוא כלפי מעלה לקרותם אלהות אצלו. דבר אחר אלהים אחרים, שהם אחרים לעובדיהם, צועקים אליהם ואינן עונים אותם, ודומה כאילו הוא אחר שאינו מכירו מעולם׃ }\rashi{\rashiDH{על פני. }כל זמן שאני קיים, שלא תאמר לא נצטוו על עבודת אלילים אלא אותו הדור (מכילתא שם)׃ }}
{לָא יִהְוֵי לָךְ אֱלָהּ אָחֳרָן בָּר מִנִּי׃}
{Thou shalt have no other gods before Me.}{\arabic{verse}}
\threeverse{\arabic{verse}}%Ex.20:4
{לֹֽא־תַעֲשֶׂ֨ה לְךָ֥ פֶ֙סֶל֙ וְכׇל־תְּמוּנָ֔ה אֲשֶׁ֤ר בַּשָּׁמַ֙יִם֙ מִמַּ֔עַל וַֽאֲשֶׁ֥ר בָּאָ֖רֶץ מִתָּ֑חַת וַאֲשֶׁ֥ר בַּמַּ֖יִם מִתַּ֥חַת לָאָֽרֶץ׃
\rashi{\rashiDH{פסל. }על שם שנפסל׃}\rashi{\rashiDH{וכל תמונה. }תמונת כל דבר אשר בשמים׃ 
}}
{לָא תַעֲבֵיד לָךְ צֵילַם וְכָל דְּמוּ דְּבִשְׁמַיָּא מִלְּעֵילָא וְדִבְאַרְעָא מִלְּרַע וְדִבְמַיָּא מִלְּרַע לְאַרְעָא׃}
{Thou shalt not make unto thee a graven image, nor any manner of likeness, of any thing that is in heaven above, or that is in the earth beneath, or that is in the water under the earth;}{\arabic{verse}}
\threeverse{\arabic{verse}}%Ex.20:5
{לֹֽא־תִשְׁתַּחֲוֶ֥ה לָהֶ֖ם וְלֹ֣א תׇעׇבְדֵ֑ם כִּ֣י אָֽנֹכִ֞י יְהֹוָ֤ה אֱלֹהֶ֙יךָ֙ אֵ֣ל קַנָּ֔א פֹּ֠קֵ֠ד עֲוֺ֨ן אָבֹ֧ת עַל־בָּנִ֛ים עַל־שִׁלֵּשִׁ֥ים וְעַל־רִבֵּעִ֖ים לְשֹׂנְאָֽי׃
\rashi{\rashiDH{אל קנא. }מקנא להפרע, ואינו עובר על מדתו למחול על עון עבודת אלילים. כל לשון קנא אנפרי״מנט בלע״ז (אייפערנד) נותן לב ליפרע׃ }\rashi{\rashiDH{לשנאי. }כתרגומו, כשאוחזין מעשה אבותיהם בידיהם (סנהדרין כז׃)׃ 
}}
{לָא תִסְגּוֹד לְהוֹן וְלָא תִפְלְחִנִּין אֲרֵי אֲנָא יְיָ אֱלָהָךְ אֵל קַנָּא מַסְעַר חוֹבֵי אֲבָהָן עַל בְּנִין מָרָדִין עַל דָּר תְּלִיתַאי וְעַל דָּר רְבִיעַאי לְסָנְאָי כַּד מַשְׁלְמִין בְּנַיָּא לְמִחְטֵי בָתַר אֲבָהָתְהוֹן׃}
{thou shalt not bow down unto them, nor serve them; for I the \lord\space thy God am a jealous God, visiting the iniquity of the fathers upon the children unto the third and fourth generation of them that hate Me;}{\arabic{verse}}
\threeverse{\arabic{verse}}%Ex.20:6
{וְעֹ֥שֶׂה חֶ֖סֶד לַאֲלָפִ֑ים לְאֹהֲבַ֖י וּלְשֹׁמְרֵ֥י מִצְוֺתָֽי׃\setuma
\rashi{\rashiDH{נוצר חסד. }שאדם עושה, לשלם שכר עד לאלפים דור, נמצאת מדה טובה יתירה על מדת פורעניות אחת על חמש מאות, שזו לארבעה דורות, וזו לאלפים (תוספתא סוטה ד, א)׃ }}
{וְעָבֵיד טֵיבוּ לְאַלְפֵי דָרִין לְרָחֲמַי וּלְנָטְרֵי פִקּוֹדָי׃}
{and showing mercy unto the thousandth generation of them that love Me and keep My commandments.}{\arabic{verse}}
\threeverse{\arabic{verse}}%Ex.20:7
{לֹ֥א תִשָּׂ֛א אֶת־שֵֽׁם־יְהֹוָ֥ה אֱלֹהֶ֖יךָ לַשָּׁ֑וְא כִּ֣י לֹ֤א יְנַקֶּה֙ יְהֹוָ֔ה אֵ֛ת אֲשֶׁר־יִשָּׂ֥א אֶת־שְׁמ֖וֹ לַשָּֽׁוְא׃\petucha
\rashi{\rashiDH{לשוא. }(השני לשון שקר, כתרגומו) כְּמָה דְּתֵימַר אי זהו שבועת שוא, נשבע לשנות את הידוע, על עמוד של אבן שהוא של זהב, (הראשון לשון מגן, כתרגומו) זה הנשבע לחנם ולהבל על של עץ עץ, ועל אבן אבן (שבועות כט.)׃ }}
{לָא תֵימֵי בִּשְׁמָא דַּייָ אֱלָהָךְ לְמַגָּנָא אֲרֵי לָא יְזַכֵּי יְיָ יָת דְּיֵימֵי בִשְׁמֵיהּ לְשִׁקְרָא׃}
{Thou shalt not take the name of the \lord\space thy God in vain; for the \lord\space will not hold him guiltless that taketh His name in vain.}{\arabic{verse}}
\threeverse{\arabic{verse}}%Ex.20:8
{זָכ֛וֹר אֶת־י֥וֹם הַשַּׁבָּ֖ת לְקַדְּשֽׁוֹ׃
\rashi{\rashiDH{זכור. }זכור ושמור בדבור אחד נאמרו, וכן מְחַלֳלֶיה מֹות יוּמָת (שמות לא, יד) וּבְיֹום הַשַּׁבָּת שְׁנֵי כְבָשִׂים (במדבר כח, ט), וכן לֹא תִלְבַּשׁ שַׁעַטְנֵז גְּדִלִים תַּעֲשֶׂה לָךְ (דברים כב, יאיב), וכן עֶרְוַת אֵשֶׁת אָחִיךָ (ויקרא יח, טז) יְבָמָהּ יָבֹא עָלֶיהָ (דברים כה, ה), הוא שנאמר אַחַת דִּבֶּר אֱלֹהִים שְׁתַּיִם זוּ שָׁמָעְתִּי (תהלים סב, יב). זכור לשון פעול הוא, כמו אָכֹול וְשָׁתֹו (ישעיה כב, יג), הָלֹוךְ וּבָכֹה (שמואל־ב ג, טז), וכן פתרונו תנו לב לזכור תמיד את יום השבת, שאם נזדמן לך חפץ יפה, תהא מזמינו לשבת (ביצה טז)׃ }}
{הֱוִי דְּכִיר יָת יוֹמָא דְּשַׁבְּתָא לְקַדָּשׁוּתֵיהּ׃}
{Remember the sabbath day, to keep it holy.}{\arabic{verse}}
\threeverse{\arabic{verse}}%Ex.20:9
{שֵׁ֤שֶׁת יָמִים֙ תַּֽעֲבֹ֔ד וְעָשִׂ֖יתָ כׇּל־מְלַאכְתֶּֽךָ׃
\rashi{\rashiDH{ועשית כל מלאכתך. }כשתבא שבת, יהא בעיניך כאילו כל מלאכתך עשויה, שלא תהרהר אחר מלאכה (מכילתא פ״ז)׃}}
{שִׁתָּא יוֹמִין תִּפְלַח וְתַעֲבֵיד כָּל עֲבִידְתָךְ׃}
{Six days shalt thou labour, and do all thy work;}{\arabic{verse}}
\threeverse{\arabic{verse}}%Ex.20:10
{וְיוֹם֙ הַשְּׁבִיעִ֔י שַׁבָּ֖ת לַיהֹוָ֣ה אֱלֹהֶ֑יךָ לֹֽא־תַעֲשֶׂ֨ה כׇל־מְלָאכָ֜ה אַתָּ֣ה ׀ וּבִנְךָ֣ וּבִתֶּ֗ךָ עַבְדְּךָ֤ וַאֲמָֽתְךָ֙ וּבְהֶמְתֶּ֔ךָ וְגֵרְךָ֖ אֲשֶׁ֥ר בִּשְׁעָרֶֽיךָ׃
\rashi{אתה ובנך ובתך. אלו הקטנים, או אינו אלא גדולים, אמרת, הרי כבר מוזהרין הם, אלא לא בא אלא להזהיר גדולים על שביתת הקטנים, וזה ששנינו (שבת קכא.)קטן שבא לכבות, אין שומעים לו, מפני ששביתתו עליך׃ }}
{וְיוֹמָא שְׁבִיעָאָה שַׁבְּתָא קֳדָם יְיָ אֱלָהָךְ לָא תַעֲבֵיד כָּל עֲבִידָא אַתְּ וּבְרָךְ וּבְרַתָּךְ עַבְדָּךְ וְאַמְתָּךְ וּבְעִירָךְ וְגִיּוֹרָךְ דִּבְקִרְוָךְ׃}
{but the seventh day is a sabbath unto the \lord\space thy God, in it thou shalt not do any manner of work, thou, nor thy son, nor thy daughter, nor thy man-servant, nor thy maid-servant, nor thy cattle, nor thy stranger that is within thy gates;}{\arabic{verse}}
\threeverse{\arabic{verse}}%Ex.20:11
{כִּ֣י שֵֽׁשֶׁת־יָמִים֩ עָשָׂ֨ה יְהֹוָ֜ה אֶת־הַשָּׁמַ֣יִם וְאֶת־הָאָ֗רֶץ אֶת־הַיָּם֙ וְאֶת־כׇּל־אֲשֶׁר־בָּ֔ם וַיָּ֖נַח בַּיּ֣וֹם הַשְּׁבִיעִ֑י עַל־כֵּ֗ן בֵּרַ֧ךְ יְהֹוָ֛ה אֶת־י֥וֹם הַשַּׁבָּ֖ת וַֽיְקַדְּשֵֽׁהוּ׃\setuma
\rashi{\rashiDH{וינח ביום השביעי. }כביכול הכתיב בעצמו מנוחה, ללמד הימנו קל וחומר לאדם שמלאכתו בעמל וביגיעה שיהא נוח בשבת׃ 
}\rashi{\rashiDH{ברך. ויקדשהו. }ברכו במן לכופלו בששי לחם משנה, וקדשו במן שלא היה יורד בו׃ }}
{אֲרֵי שִׁתָּא יוֹמִין עֲבַד יְיָ יָת שְׁמַיָּא וְיָת אַרְעָא יָת יַמָּא וְיָת כָּל דִּבְהוֹן וְנָח בְּיוֹמָא שְׁבִיעָאָה עַל כֵּן בָּרֵיךְ יְיָ יָת יוֹמָא דְּשַׁבְּתָא וְקַדְּשֵׁיהּ׃}
{for in six days the \lord\space made heaven and earth, the sea, and all that in them is, and rested on the seventh day; wherefore the \lord\space blessed the sabbath day, and hallowed it.}{\arabic{verse}}
\threeverse{\arabic{verse}}%Ex.20:12
{כַּבֵּ֥ד אֶת־אָבִ֖יךָ וְאֶת־אִמֶּ֑ךָ לְמַ֙עַן֙ יַאֲרִכ֣וּן יָמֶ֔יךָ עַ֚ל הָאֲדָמָ֔ה אֲשֶׁר־יְהֹוָ֥ה אֱלֹהֶ֖יךָ נֹתֵ֥ן לָֽךְ׃\setuma
\rashi{\rashiDH{למען יאריכון ימיך. }אם תכבד יאריכון ימיך, ואם לאו יקצרון, שדברי תורה נוטריקון הם נדרשים, מכלל הן לאו ומכלל לאו הן (מכילתא פ״ח)׃ }}
{יַקַּר יָת אֲבוּךְ וְיָת אִמָּךְ בְּדִיל דְּיֵירְכוּן יוֹמָךְ עַל אַרְעָא דַּייָ אֱלָהָךְ יָהֵיב לָךְ׃}
{Honour thy father and thy mother, that thy days may be long upon the land which the \lord\space thy God giveth thee.}{\arabic{verse}}
\threeverse{\arabic{verse}}%Ex.20:13
{לֹ֥א תִרְצָ֖ח \setuma לֹ֣א תִנְאָ֑ף \setuma לֹ֣א תִגְנֹ֔ב לֹֽא־תַעֲנֶ֥ה בְרֵעֲךָ֖ עֵ֥ד שָֽׁקֶר׃ \setuma
\rashi{\rashiDH{לא תנאף. }אין ניאוף אלא באשת איש, שנאמר מֹות יוּמַת הַנֹאֵף וְהַנֹּאָפֶת (ויקרא כ, ו), ואומר הָאִשָּׁה הַמְנָאָפֶת תַּחַת אִישָׁהּ תִּקַּח אֶת זָרִים (יחזקאל טז, לב)׃ }\rashi{\rashiDH{לא תגנוב. }בגונב נפשות הכתוב מדבר, לא תגנובו בגונב ממון, או אינו אלא זה בגונב ממון ולהלן בגונב נפשות, אמרת, דבר הלמד מענינו, מה לא תרצח לא תנאף מדבר בדבר שחייבין עליהם מיתת בית דין, אף לא תגנוב דבר שחייב עליו מיתת בית דין (סנהדרין פו.)׃ }}
{לָא תִקְטוּל נְפַשׁ לָא תְגוּף לָא תִגְנוּב לָא תַסְהֵיד בְּחַבְרָךְ סָהֲדוּתָא דְּשִׁקְרָא׃}
{Thou shalt not murder. Thou shalt not commit adultery. Thou shalt not steal. Thou shalt not bear false witness against thy neighbour.}{\arabic{verse}}
\threeverse{\arabic{verse}}%Ex.20:14
{לֹ֥א תַחְמֹ֖ד בֵּ֣ית רֵעֶ֑ךָ לֹֽא־תַחְמֹ֞ד אֵ֣שֶׁת רֵעֶ֗ךָ וְעַבְדּ֤וֹ וַאֲמָתוֹ֙ וְשׁוֹר֣וֹ וַחֲמֹר֔וֹ וְכֹ֖ל אֲשֶׁ֥ר לְרֵעֶֽךָ׃\petucha}
{לָא תַחְמֵיד בֵּית חַבְרָךְ לָא תַחְמֵיד אִתַּת חַבְרָךְ וְעַבְדֵּיהּ וְאַמְתֵּיהּ וְתוֹרֵיהּ וּחְמָרֵיהּ וְכֹל דִּלְחַבְרָךְ׃}
{Thou shalt not covet thy neighbour’s house; thou shalt not covet thy neighbour’s wife, nor his man-servant, nor his maid-servant, nor his ox, nor his ass, nor any thing that is thy neighbour’s.}{\arabic{verse}}
\threeverse{\aliya{שביעי}}%Ex.20:15
{וְכׇל־הָעָם֩ רֹאִ֨ים אֶת־הַקּוֹלֹ֜ת וְאֶת־הַלַּפִּידִ֗ם וְאֵת֙ ק֣וֹל הַשֹּׁפָ֔ר וְאֶת־הָהָ֖ר עָשֵׁ֑ן וַיַּ֤רְא הָעָם֙ וַיָּנֻ֔עוּ וַיַּֽעַמְד֖וּ מֵֽרָחֹֽק׃
\rashi{\rashiDH{וכל העם רואים. }מלמד שלא היה בהם אחד סומא, ומנין שלא היה בהם אלם, תלמוד לומר ויענו כל העם, ומנין שלא היה בהם חרש, תלמוד לומר נעשה ונשמע (מכילתא פ״ט)׃ }\rashi{\rashiDH{רואים את הקולות. }רואין את הנשמע, שאי אפשר לראות במקום אחר (שם)׃ }\rashi{\rashiDH{את הקולות. }היוצאין מפי הגבורה׃}\rashi{\rashiDH{וינעו. }אין נוע אלא זיע (שם)׃}\rashi{\rashiDH{ויעמדו מרחוק. }היו נרתעין לאחוריהם שנים עשר מיל, כאורך מחניהם, ומלאכי השרת באין ומסייעין אותן להחזירם, שנאמר מַלְכֵי צְבָאֹות יִדֹּדוּן יִדֹּדוּן (תהלים סח, יג.  מכילתא שם)׃ 
}}
{וְכָל עַמָּא חָזַן יָת קָלַיָּא וְיָת בָּעוֹרַיָּא וְיָת קָל שׁוֹפָרָא וְיָת טוּרָא דְּתָנַן וַחֲזָא עַמָּא וְזָעוּ וְקָמוּ מֵרַחִיק׃}
{And all the people perceived the thunderings, and the lightnings, and the voice of the horn, and the mountain smoking; and when the people saw it, they trembled, and stood afar off.}{\arabic{verse}}
\threeverse{\arabic{verse}}%Ex.20:16
{וַיֹּֽאמְרוּ֙ אֶל־מֹשֶׁ֔ה דַּבֵּר־אַתָּ֥ה עִמָּ֖נוּ וְנִשְׁמָ֑עָה וְאַל־יְדַבֵּ֥ר עִמָּ֛נוּ אֱלֹהִ֖ים פֶּן־נָמֽוּת׃}
{וַאֲמַרוּ לְמֹשֶׁה מַלֵּיל אַתְּ עִמַּנָא וּנְקַבֵּיל וְלָא יִתְמַלַּל עִמַּנָא מִן קֳדָם יְיָ דִּלְמָא נְמוּת׃}
{And they said unto Moses: ‘Speak thou with us, and we will hear; but let not God speak with us, lest we die.’}{\arabic{verse}}
\threeverse{\arabic{verse}}%Ex.20:17
{וַיֹּ֨אמֶר מֹשֶׁ֣ה אֶל־הָעָם֮ אַל־תִּירָ֒אוּ֒ כִּ֗י לְבַֽעֲבוּר֙ נַסּ֣וֹת אֶתְכֶ֔ם בָּ֖א הָאֱלֹהִ֑ים וּבַעֲב֗וּר תִּהְיֶ֧ה יִרְאָת֛וֹ עַל־פְּנֵיכֶ֖ם לְבִלְתִּ֥י תֶחֱטָֽאוּ׃
\rashi{\rashiDH{לבעבור נסות אתכם. }לגדל אתכם בעולם, שיצא לכם שם באומות שהוא בכבודו נגלה עליכם׃ }\rashi{\rashiDH{נסות. }לשון הרמה וגדולה, כמו הָרִימוּ נֵס (ישעיה סב, י), אָרִים נִסִּי (שם מט, כב), וְכַנֵּס עַל הַגִּבְעָה (שם ל, יז), שהוא זקוף׃ }\rashi{\rashiDH{ובעבור תהיה יראתו. }על ידי שראיתם אותו יָראוּי וּמְאֻיָּם, תדעו כי אין זולתו, ותיראו מפניו׃ }}
{וַאֲמַר מֹשֶׁה לְעַמָּא לָא תִדְחֲלוּן אֲרֵי בְּדִיל לְנַסָּאָה יָתְכוֹן אִתְגְּלִי לְכוֹן יְקָרָא דַּייָ וּבְדִיל דִּתְהֵי דַּחְלְתֵיהּ עַל אַפֵּיכוֹן בְּדִיל דְּלָא תְחוּבוּן׃}
{And Moses said unto the people: ‘Fear not; for God is come to prove you, and that His fear may be before you, that ye sin not.’}{\arabic{verse}}
\threeverse{\arabic{verse}}%Ex.20:18
{וַיַּעֲמֹ֥ד הָעָ֖ם מֵרָחֹ֑ק וּמֹשֶׁה֙ נִגַּ֣שׁ אֶל־הָֽעֲרָפֶ֔ל אֲשֶׁר־שָׁ֖ם הָאֱלֹהִֽים׃ \setuma         
\rashi{\rashiDH{נגש אל הערפל. }לפנים משלש מחיצות, חושך, ענן, וערפל, שנאמר וְהָהָר בֹּעֵר בָּאֵשׁ עַד לֵב הַשָּׁמַיִם חשֶׁךְ עָנָן וְעֲרָפֶל (דברים ד, יא). ערפל הוא עב הענן, שאמר לו הנה אָנֹכִי בָּא אֵלֶיךָ בְּעַב הֶעָנָן (שמות יט, ט)׃ }}
{וְקָם עַמָּא מֵרַחִיק וּמֹשֶׁה קְרֵיב לְצַד אֲמִטְּתָא דְּתַמָּן יְקָרָא דַּייָ׃}
{And the people stood afar off; but Moses drew near unto the thick darkness where God was.}{\arabic{verse}}
\threeverse{\aliya{מפטיר}}%Ex.20:19
{וַיֹּ֤אמֶר יְהֹוָה֙ אֶל־מֹשֶׁ֔ה כֹּ֥ה תֹאמַ֖ר אֶל־בְּנֵ֣י יִשְׂרָאֵ֑ל אַתֶּ֣ם רְאִיתֶ֔ם כִּ֚י מִן־הַשָּׁמַ֔יִם דִּבַּ֖רְתִּי עִמָּכֶֽם׃
\rashi{\rashiDH{כה תאמר. }בלשון הזה׃}\rashi{\rashiDH{אתם ראיתם. }יש הפרש בין מה שאדם רואה למה שאחרים משיחין לו, שמה שאחרים משיחין לו פעמים שלבו חלוק מלהאמין׃ }\rashi{\rashiDH{כי מן השמים דברתי. }וכתוב אחר אומר, וַיֵּרֶד ה׳ עַל הַר סִינַי, בא הכתוב השלישי והכריע ביניהם, מִן הַשָּׁמַיִם הִשְׁמִיעֲךָ אֶת קֹלֹו לְיַסְּרֶךָ וְעַל הָאָרֶץ הֶרְאֲךָ אֶת אִשֹּׁו הַגְּדֹולָה (דברים ד, לו), כבודו בשמים, ואשו וגבורתו על הארץ. דבר אחר, הִרְכִּין השמים ושמי השמים, והציען על ההר, וכן הוא אומר וַיֵּט שָׁמַיִם וַיֵּרַד (תהלים יח, ו.  מכילתא פ״ט)׃ }}
{וַאֲמַר יְיָ לְמֹשֶׁה כִּדְנָן תֵּימַר לִבְנֵי יִשְׂרָאֵל אַתּוּן חֲזֵיתוֹן אֲרֵי מִן שְׁמַיָּא מַלֵּילִית עִמְּכוֹן׃}
{And the \lord\space said unto Moses: Thus thou shalt say unto the children of Israel: Ye yourselves have seen that I have talked with you from heaven.}{\arabic{verse}}
\threeverse{\arabic{verse}}%Ex.20:20
{לֹ֥א תַעֲשׂ֖וּן אִתִּ֑י אֱלֹ֤הֵי כֶ֙סֶף֙ וֵאלֹהֵ֣י זָהָ֔ב לֹ֥א תַעֲשׂ֖וּ לָכֶֽם׃
\rashi{\rashiDH{לא תעשון אתי. }לא תעשון דמות שמשי המשמשים לפני במרום (מכילתא פ״י)׃ }\rashi{\rashiDH{אלהי כסף. }בא להזהיר על הכרובים, שאתה עושה לעמוד אתי, שלא יהיו של כסף, שאם שניתם לעשותם של כסף, הרי הן לפני כאלהות׃ }\rashi{\rashiDH{ואלהי זהב. }בא להזהיר שלא יוסיף על ב׳, שאם עשית ד׳, הרי הן לפני כאלהי זהב׃ }\rashi{\rashiDH{לא תעשו לכם. }לא תאמר, הריני עושה כרובים בבתי כנסיות ובבתי מדרשות כדרך שאני עושה בבית עולמים, לכך נאמר לא תעשו לכם׃ }}
{לָא תַעְבְּדוּן קֳדָמָי דַּחְלָן דִּכְסַף וְדַחְלָן דִּדְהַב לָא תַעְבְּדוּן לְכוֹן׃}
{Ye shall not make with Me—gods of silver, or gods of gold, ye shall not make unto you.}{\arabic{verse}}
\threeverse{\arabic{verse}}%Ex.20:21
{מִזְבַּ֣ח אֲדָמָה֮ תַּעֲשֶׂה־לִּי֒ וְזָבַחְתָּ֣ עָלָ֗יו אֶת־עֹלֹתֶ֙יךָ֙ וְאֶת־שְׁלָמֶ֔יךָ אֶת־צֹֽאנְךָ֖ וְאֶת־בְּקָרֶ֑ךָ בְּכׇל־הַמָּקוֹם֙ אֲשֶׁ֣ר אַזְכִּ֣יר אֶת־שְׁמִ֔י אָב֥וֹא אֵלֶ֖יךָ וּבֵרַכְתִּֽיךָ׃
\rashi{\rashiDH{מזבח אדמה. }מחובר באדמה, שלא יבננו על גבי עמודים או על גבי כיפין (נ״א בסיס) (מכילתא פי״א). בר אחר, שהיה ממלא את חלל מזבח הנחושת אדמה בשעת חנייתן (מכילתא שם)׃ }\rashi{\rashiDH{תעשה לי. }שתהא תחלת עשייתו לשמי׃}\rashi{\rashiDH{וזבחת עליו. }אצלו, כמו וְעָלָיו מַטֵּה מְנַשֶּׁה (במדבר ב, כ), ו אינו אלא עליו ממש, תלמוד לומר הַבָּשָׂר וְהַדָּם עַל מִזְבַּח ה׳ אֶלֹהֶיךָ (דברים יב, כז), ואין שחיטה בראש המזבח (מכילתא פי״א  זבחים נח.)׃ }\rashi{\rashiDH{את עולתיך ואת שלמיך. }אשר מצאנך ומבקרך. את צאנך ואת בקרך. ירוש לאת עולתיך ואת שלמיך׃}\rashi{\rashiDH{בכל המקום אשר אזכיר את שמי. }אשר אתן לך רשות להזכיר שם המפורש שלי, שם אבוא אליך וברכתיך, אשרה שכינתי עליך, מכאן אתה למד, שלא ניתן רשות להזכיר שם המפורש אלא במקום שהשכינה באה שם, וזהו בית הבחירה, שם ניתן רשות לכהנים להזכיר שם המפורש בנשיאת כפים לברך את העם׃ }}
{מַדְבַּח אֲדַמְתָּא תַּעֲבֵיד קֳדָמַי וּתְהֵי דָּבַח עֲלוֹהִי יָת עֲלָוָתָךְ וְיָת נִכְסַת קוּדְשָׁךְ מִן עָנָךְ וּמִן תּוֹרָךְ בְּכָל אֲתַר דְּאַשְׁרֵי שְׁכִינְתִי לְתַמָּן אֶשְׁלַח בִּרְכְתִי לָךְ וַאֲבָרְכִנָּךְ׃}
{An altar of earth thou shalt make unto Me, and shalt sacrifice thereon thy burnt-offerings, and thy peace-offerings, thy sheep, and thine oxen; in every place where I cause My name to be mentioned I will come unto thee and bless thee.}{\arabic{verse}}
\threeverse{\arabic{verse}}%Ex.20:22
{וְאִם־מִזְבַּ֤ח אֲבָנִים֙ תַּֽעֲשֶׂה־לִּ֔י לֹֽא־תִבְנֶ֥ה אֶתְהֶ֖ן גָּזִ֑ית כִּ֧י חַרְבְּךָ֛ הֵנַ֥פְתָּ עָלֶ֖יהָ וַתְּחַֽלְלֶֽהָ׃
\rashi{\rashiDH{ואם מזבח אבנים. }רבי ישמעאל אומר, כל אם ואם שבתורה רשות, חוץ מג׳, ואם מזבח אבנים תעשה לי, הרי אם זה משמש בלשון כאשר, וכאשר תעשה לי מזבח אבנים לא תבנה אתהן גזית, שהרי חובה עליך לבנות מזבח אבנים, שנאמר אֲבָנִים שְׁלֵמֹות תִּבְנֶה (דברים כז, ו). וכן אִם כֶּסֶף תַּלְוֶה (שמות כב, כד), חובה הוא, שנאמר וְהַעֲבֵט תַּעֲבִיטֶנּוּ (דברים טו, ח), ואף זה משמש בלשון כאשר. וכן וְאִם תַּקְרִיב מִנְחַת בִּכּוּרִים (ויקרא ב, יד), זו מנחת העומר שהיא חובה (מכילתא פי״א), ועל כרחך אין אם הללו תלוין, אלא ודאין, ובלשון כאשר הם משמשים׃ }\rashi{\rashiDH{גזית. }לשון גזיזה, שפוסלן ומכתתן בברזל׃ }\rashi{\rashiDH{כי חרבך הנפת עליה. }הרי כי זה משמש בלשון פן, שהוא דילמא, פן תניף חרבך עליה׃ 
}\rashi{\rashiDH{ותחללה. }הא למדת, שאם הנפת עליה ברזל חללת שהמזבח נברא להאריך ימיו של אדם, והברזל נברא לקצר ימיו של אדם, אין זה בדין שיונף המקצר על המאריך (מדות פ״ג מ״ד). ועוד, שהמזבח מטיל שלום בין ישראל לאביהם שבשמים, לפיכך לא יבא עליו כורת ומחבל, והרי דברים קל וחומר, ומה אבנים שאינם רואות ולא שומעות ולא מדברות, על ידי שמטילות שלום אמרה תורה לא תניף עליהם ברזל, המטיל שלום בין איש לאשתו, בין משפחה למשפחה, בין אדם לחבירו, על אחת כמה וכמה שלא תבואהו פורענות׃ }}
{וְאִם מַדְבַּח אַבְנִין תַּעֲבֵיד קֳדָמַי לָא תִבְנֵי יָתְהוֹן פְּסִילָן לָא תְרִים חַרְבָּךְ עֲלַהּ וְתַחֲלִינַּהּ׃}
{And if thou make Me an altar of stone, thou shalt not build it of hewn stones; for if thou lift up thy tool upon it, thou hast profaned it.}{\arabic{verse}}
\threeverse{\arabic{verse}}%Ex.20:23
{וְלֹֽא־תַעֲלֶ֥ה בְמַעֲלֹ֖ת עַֽל־מִזְבְּחִ֑י אֲשֶׁ֛ר לֹֽא־תִגָּלֶ֥ה עֶרְוָתְךָ֖ עָלָֽיו׃ \petucha 
\rashi{\rashiDH{ולא תעלה במעלות. }כשאתה בונה כבש למזבח, לא תעשהו מעלות מעלות, אשקנו״ש בלע״ז (שטופען שטאפלען) אלא חלוק יהא ומשופע׃ }\rashi{\rashiDH{אשר לא תגלה ערותך. }שעל ידי המעלות אתה צריך להרחיב פסיעותיך, ואף על פי שאינו גלוי ערוה ממש, שהרי כתיב ועשה להם מכנסי בד, מכל מקום הרחבת הפסיעות קרוב לגלוי ערוה הוא, ואתה נוהג בהם מנהג בזיון, והרי דברים קל וחומר, ומה אבנים הללו שאין בהם דעת להקפיד על בזיונן, אמרה תורה הואיל ויש בהם צורך לא תנהג בהם מנהג בזיון, חבירך שהוא בדמות יוצרך, ומקפיד על בזיונו, על אחת כמה וכמה׃ 
}}
{וְלָא תִסַּק בְּדַרְגִּין עַל מַדְבְּחִי דְּלָא תִתְגַּלֵּי עַרְיְתָךְ עֲלוֹהִי׃}
{Neither shalt thou go up by steps unto Mine altar, that thy nakedness be not uncovered thereon.}{\arabic{verse}}
\newperek
\newparsha{משפטים}
\threeverse{\aliya{משפטים}}%Ex.21:1
{וְאֵ֙לֶּה֙ הַמִּשְׁפָּטִ֔ים אֲשֶׁ֥ר תָּשִׂ֖ים לִפְנֵיהֶֽם׃
\rashi{\rashiDH{ואלה המשפטים. }כל מקום שנאמר אלה, פסל את הראשונים, ואלה, מוסיף על הראשונים (שמו״ר ל, ב), מה הראשונים מסיני אף אלו מסיני. ולמה נסמכה פרשת דינין לפרשת מזבח, לומר לך שתשים סנהדרין אצל המקדש (ס״א המזבח)׃ 
}\rashi{\rashiDH{אשר תשים לפניהם. }אמר לו הקב״ה למשה, לא תעלה על דעתך לומר, אשנה להם הפרק וההלכה ב׳ או ג׳ פעמים, עד שתהא סדורה בפיהם כמשנתה, ואיני מטריח עצמי להבינם טעמי הדבר ופירושו, לכך נאמר אשר תשים לפניהם, כשלחן הערוך ומוכן לאכול לפני האדם׃ }\rashi{\rashiDH{לפניהם. }ולא לפני עובדי אלילים (גיטין פח׃), ואפילו ידעת בדין אחד שהם דנין אותו כדיני ישראל, אל תביאהו בערכאות שלהם, שהמביא דיני ישראל לפני ארמים, מחלל את השם ומיקר שם האלילים להשביחם (ס״א להחשיבם), שנאמר כי לא כצורנו צורם ואויבינו פלילים (דברים לב, לא), כשאויבינו פלילים זהו עדות לעלוי יראתם׃ }}
{וְאִלֵּין דִּינַיָּא דְּתַסְדַּר קֳדָמֵיהוֹן׃}
{Now these are the ordinances which thou shalt set before them.}{\Roman{chap}}
\threeverse{\arabic{verse}}%Ex.21:2
{כִּ֤י תִקְנֶה֙ עֶ֣בֶד עִבְרִ֔י שֵׁ֥שׁ שָׁנִ֖ים יַעֲבֹ֑ד וּבַ֨שְּׁבִעִ֔ת יֵצֵ֥א לַֽחׇפְשִׁ֖י חִנָּֽם׃
\rashi{\rashiDH{כי תקנה עבד עברי. }עבד שהוא עברי, או אינו אלא עבדו של עברי, עבד כנעני שלקחתו מישראל, ועליו הוא אומר שש שנים יעבוד, ומה אני מקיים וְהִתְנַחַלְתֶּם אֹתָם (ויקרא כה, מו), בלקוח מן הכנענים, אבל בלקוח מישראל יצא בשש, תלמוד לומר כי ימכר לך אחיך העברי (דברים טו, יב), לא אמרתי אלא באחיך׃ }\rashi{\rashiDH{כי תקנה. }מיד בית דין שמכרוהו בגנבתו, כמו שנאמר אִם אֵין לֹו וְנִמְכַּר בִּגְנֵבָתֹו (שמות כב, א), או אינו אלא במוכר עצמו מפני דחקו, אבל מכרוהו בית דין לא יצא בשש, כשהוא אומר וְכִי יָמוּךְ אָחִיךָ עִמָךְ וְנִמְכּר לָךְ (ויקרא כה, לט), הרי מוכר עצמו מפני דוחקו אמור, ומה אני מקיים כי תקנה, בנמכר בבית דין׃ }\rashi{\rashiDH{לחפשי. }לחירות׃}}
{אֲרֵי תִזְבּוֹן עַבְדָּא בַר יִשְׂרָאֵל שֵׁית שְׁנִין יִפְלַח וּבִשְׁבִיעֵיתָא יִפּוֹק לְבַר חוֹרִין מַגָּן׃}
{If thou buy a Hebrew servant, six years he shall serve; and in the seventh he shall go out free for nothing.}{\arabic{verse}}
\threeverse{\arabic{verse}}%Ex.21:3
{אִם־בְּגַפּ֥וֹ יָבֹ֖א בְּגַפּ֣וֹ יֵצֵ֑א אִם־בַּ֤עַל אִשָּׁה֙ ה֔וּא וְיָצְאָ֥ה אִשְׁתּ֖וֹ עִמּֽוֹ׃
\rashi{\rashiDH{אם בגפו יבא. }שלא היה נשוי אשה, כתרגומו אם בלחודוהי. ולשון בגפו, בכנפו, שלא בא אלא כמות שהוא, יחידי בתוך לבושו, בכנף בגדו׃ }\rashi{\rashiDH{בגפו יצא. }מגיד, שאם לא היה נשוי מתחלה, אין רבו מוסר לו שפחה כנענית להוליד ממנה עבדים (קידושין כ.)׃ }\rashi{\rashiDH{אם בעל אשה הוא. }ישראלית (מכילתא פ״ב)׃ }\rashi{\rashiDH{ויצאה אשתו עמו. }וכי מי הכניסה שתצא, אלא מגיד הכתוב, שהקונה עבד עברי חייב במזונות אשתו ובניו (קידושין כב.)׃ }}
{אִם בִּלְחוֹדוֹהִי יֵיעוֹל בִּלְחוֹדוֹהִי יִפּוֹק אִם בְּעֵיל אִתְּתָא הוּא וְתִפּוֹק אִתְּתֵיהּ עִמֵּיהּ׃}
{If he come in by himself, he shall go out by himself; if he be married, then his wife shall go out with him.}{\arabic{verse}}
\threeverse{\arabic{verse}}%Ex.21:4
{אִם־אֲדֹנָיו֙ יִתֶּן־ל֣וֹ אִשָּׁ֔ה וְיָלְדָה־ל֥וֹ בָנִ֖ים א֣וֹ בָנ֑וֹת הָאִשָּׁ֣ה וִילָדֶ֗יהָ תִּהְיֶה֙ לַֽאדֹנֶ֔יהָ וְה֖וּא יֵצֵ֥א בְגַפּֽוֹ׃
\rashi{\rashiDH{אם אדניו יתן לו אשה. }מכאן, שהרשות ביד רבו למסור לו שפחה כנענית להוליד ממנה עבדים. או אינו אלא בישראלית, תלמוד לומר האשה וילדיה תהיה לאדוניה, הא אינו מדבר אלא בכנענית, שהרי העבריה אף היא יוצאה בשש, ואפילו לפני שש אם הביאה סימנין יוצאה, שנאמר אָחִיךָ הָעִבְרִי אֹו הָעִבְרִיָה (דברים טו, יב) מלמד שאף העבריה יוצאה בשש׃ }}
{אִם רִבּוֹנֵיהּ יִתֵּין לֵיהּ אִתְּתָא וּתְלִיד לֵיהּ בְּנִין אוֹ בְנָן אִתְּתָא וּבְנַהָא תְּהֵי לְרִבּוֹנַהּ וְהוּא יִפּוֹק בִּלְחוֹדוֹהִי׃}
{If his master give him a wife, and she bear him sons or daughters; the wife and her children shall be her master’s, and he shall go out by himself.}{\arabic{verse}}
\threeverse{\arabic{verse}}%Ex.21:5
{וְאִם־אָמֹ֤ר יֹאמַר֙ הָעֶ֔בֶד אָהַ֙בְתִּי֙ אֶת־אֲדֹנִ֔י אֶת־אִשְׁתִּ֖י וְאֶת־בָּנָ֑י לֹ֥א אֵצֵ֖א חׇפְשִֽׁי׃
\rashi{\rashiDH{את אשתי. }השפחה׃ 
}}
{וְאִם מֵימָר יֵימַר עַבְדָּא רָחֵימְנָא יָת רִבּוֹנִי יָת אִתְּתִי וְיָת בְּנָי לָא אֶפּוֹק בַּר חוֹרִין׃}
{But if the servant shall plainly say: I love my master, my wife, and my children; I will not go out free;}{\arabic{verse}}
\threeverse{\arabic{verse}}%Ex.21:6
{וְהִגִּישׁ֤וֹ אֲדֹנָיו֙ אֶל־הָ֣אֱלֹהִ֔ים וְהִגִּישׁוֹ֙ אֶל־הַדֶּ֔לֶת א֖וֹ אֶל־הַמְּזוּזָ֑ה וְרָצַ֨ע אֲדֹנָ֤יו אֶת־אׇזְנוֹ֙ בַּמַּרְצֵ֔עַ וַעֲבָד֖וֹ לְעֹלָֽם׃ \setuma         
\rashi{\rashiDH{אל האלהים. }לבית דין, צריך שימלך במוכריו שמכרוהו לו (מכילתא פ״ב)׃ }\rashi{\rashiDH{אל הדלת או אל המזוזה. }יכול שתהא המזוזה כשרה לרצוע עליה, תלמוד לומר וְנָתַתָּה בְאָזְנֹו וּבַדֶּלֶת (דברים טו, יז), בדלת ולא במזוזה, הא מה תלמוד לומר או אל המזוזה, הקיש דלת למזוזה, מה מזוזה מעומד אף דלת מעומד (קידושין כב׃)׃ }\rashi{\rashiDH{ורצע אדוניו את אזנו במרצע. }הימנית, או אינו אלא של שמאל, תלמוד לומר אזן אזן לגזירה שוה, נאמר כאן ורצע אדוניו את אזנו, ונאמר במצורע תְּנוּך אֹזֶן הַמִּטַּהֵר הַיְמָנִית (ויקרא יד, יד), מה להלן הימנית אף כאן הימנית. ומה ראה אזן להרצע מכל שאר אברים שבגוף, אמר ר׳ יוחנן בן זכאי, אזן זאת ששמעה על הר סיני לא תגנוב, והלך וגנב, תרצע (מכילתא פ״ב), ואם מוכר עצמו, אזן ששמעה על הר סיני כי לי בני ישראל עבדים, והלך וקנה אדון לעצמו, תרצע. רבי שמעון היה דורש מקרא זה כמין חומר, (ר״ל קשר צרור מבושם שתולין בצואר לתכשיט) מה נשתנו דלת ומזוזה מכל כלים שבבית, אמר הקב״ה, דלת ומזוזה שהיו עדים במצרים כשפסחתי על המשקוף ועל שתי המזוזות, ואמרתי כי לי בני ישראל עבדים, עבדי הם, ולא עבדים לעבדים, והלך זה וקנה אדון לעצמו, ירצע בפניהם׃ }\rashi{\rashiDH{ועבדו לעולם. }עד היובל, או אינו אלא לעולם כמשמעו, תלמוד לומר וְאִישׁ אֶל מִשְׁפַּחְתֹּו תָּשֻׁבוּ (ויקרא כה, י), מגיד שחמשים שנה קרוים עולם, ולא שיהא עובדו כל חמשים שנה, אלא עובדו עד היובל, בין סמוך בין מופלג׃ }}
{וִיקָרְבִנֵּיהּ רִבּוֹנֵיהּ לִקְדָם דַּיָּינַיָּא וִיקָרְבִנֵּיהּ לְוָת דַּשָּׁא אוֹ דִּלְוָת מְזוּזְתָא וְיַרְצַע רִבּוֹנֵיהּ יָת אוּדְנֵיהּ בְּמַרְצְעָא וִיהֵי לֵיהּ עֶבֶד פָּלַח לְעָלַם׃}
{then his master shall bring him unto God, and shall bring him to the door, or unto the door-post; and his master shall bore his ear through with an awl; and he shall serve him for ever.}{\arabic{verse}}
\threeverse{\aliya{לוי}}%Ex.21:7
{וְכִֽי־יִמְכֹּ֥ר אִ֛ישׁ אֶת־בִּתּ֖וֹ לְאָמָ֑ה לֹ֥א תֵצֵ֖א כְּצֵ֥את הָעֲבָדִֽים׃
\rashi{\rashiDH{וכי ימכר איש את בתו לאמה. }בקטנה הכתוב מדבר (מכילתא פ״ג), יכול אפילו הביאה סימנים, אמרת קל וחומר, ומה מכורה קודם לכן יוצאה בסימנין, כמו שכתוב ויצאה חנם אין כסף, שאנו דורשים אותו לסימני נערות, שאינה מכורה אינו דין שלא תמכר (ערכין כט׃)׃ }\rashi{\rashiDH{לא תצא כצאת העבדים. }כיציאת עבדים כנענים שיוצאים בשן ועין, אבל זו לא תצא בשן ועין, אלא עובדת שש, או עד היובל, או עד שתביא סימנין, וכל הקודם קודם לחירותה, ונותן לה דמי עינה או דמי שינה, או אינו אלא לא תצא כצאת העבדים בשש וביובל, תלמוד לומר כי ימכר לך אחיך העברי או העבריה, מקיש עבריה לעברי לכל יציאותיו, מה עברי יוצא בשש וביובל, אף עבריה יוצאה בשש וביובל, ומהו לא תצא כצאת העבדים, לא תצא בראשי איברים כעבדים כנענים, יכול העברי יוצא בראשי איברים, תלמוד לומר העברי או העבריה, מקיש עברי לעבריה, מה העבריה אינה יוצאה בראשי איברים, אף הוא אינו יוצא בראשי איברים׃ }}
{וַאֲרֵי יְזַבֵּין גְּבַר יָת בְּרַתֵּיהּ לְאַמְהוּ לָא תִפּוֹק כְּמַפְּקָנוּת עַבְדַיָּא׃}
{And if a man sell his daughter to be a maid-servant, she shall not go out as the men-servants do.}{\arabic{verse}}
\threeverse{\arabic{verse}}%Ex.21:8
{אִם־רָעָ֞ה בְּעֵינֵ֧י אֲדֹנֶ֛יהָ אֲשֶׁר־\qk{ל֥וֹ}{לא} יְעָדָ֖הּ וְהֶפְדָּ֑הּ לְעַ֥ם נׇכְרִ֛י לֹא־יִמְשֹׁ֥ל לְמׇכְרָ֖הּ בְּבִגְדוֹ־בָֽהּ׃
\rashi{\rashiDH{אם רעה בעיני אדניה. }שלא נשאה חן בעיניו לכנסה (מכילתא פ״ג)׃ }\rashi{\rashiDH{אשר לא יעדה. }שהיה לו ליעדה ולהכניסה לו לאשה, וכסף קנייתה הוא כסף קדושיה. וכאן רמז לך הכתוב שמצוה ביעוד, ורמז לך שאינה צריכה קדושין אחרים׃ }\rashi{\rashiDH{והפדה. }יתן לה מקום להפדות ולצאת, שאף הוא מסייע בפדיונה, ומה הוא מקום שנותן לה, שמגרע מפדיונה במספר השנים שעשתה אצלו כאילו היא שכורה אצלו, כיצד, הרי שקנאה במנה ועשתה אצלו ב׳ שנים, אומרים לו, יודע היית שעתידה לצאת לסוף שש שנה, נמצא שקנית עבודת כל שנה ושנה בששית המנה, ועשתה אצלך ב׳ שנים, הרי שלישית המנה, טול שני שלישיות המנה ותצא מאצלך׃ }\rashi{\rashiDH{לעם נכרי לא ימשל למכרה.} שאינו רשאי למכרה לאחר, לא האדון ולא האב (קידושין יח׃)׃ }\rashi{\rashiDH{בבגדו בה. }אם בא לבגוד בה, שלא לקיים בה מצות יעוד, וכן אביה, מאחר שבגד בה ומכרה לזה׃ }}
{אִם בִּישָׁת בְּעֵינֵי רִבּוֹנַהּ דִּיקַיְּימַהּ לֵיהּ וְיִפְרְקִנַּהּ לִגְבַר אָחֳרָן לֵית לֵיהּ רְשׁוּ לְזַבּוֹנַהּ בְּמִשְׁלְטֵיהּ בַּהּ׃}
{If she please not her master, who hath espoused her to himself, then shall he let her be redeemed; to sell her unto a foreign people he shall have no power, seeing he hath dealt deceitfully with her.}{\arabic{verse}}
\threeverse{\arabic{verse}}%Ex.21:9
{וְאִם־לִבְנ֖וֹ יִֽיעָדֶ֑נָּה כְּמִשְׁפַּ֥ט הַבָּנ֖וֹת יַעֲשֶׂה־לָּֽהּ׃
\rashi{\rashiDH{ואם לבנו ייעדנה. }האדון, מלמד שאף בנו קם תחתיו ליעדה אם ירצה אביו, ואינו צריך לקדושין אחרים, אלא אומר לה, הרי את מיועדת לי בכסף שקיבל אביך בדמיך׃ }\rashi{\rashiDH{כמשפט הבנות. }שאר כסות ועונה׃ 
}}
{וְאִם לִבְרֵיהּ יְקַיְּימִנַּהּ כְּהִלְכָּת בְּנָת יִשְׂרָאֵל יַעֲבֵיד לַהּ׃}
{And if he espouse her unto his son, he shall deal with her after the manner of daughters.}{\arabic{verse}}
\threeverse{\arabic{verse}}%Ex.21:10
{אִם־אַחֶ֖רֶת יִֽקַּֽח־ל֑וֹ שְׁאֵרָ֛הּ כְּסוּתָ֥הּ וְעֹנָתָ֖הּ לֹ֥א יִגְרָֽע׃
\rashi{\rashiDH{אם אחרת יקח לו. }עליה׃}\rashi{\rashiDH{שארה כסותה ועונתה לא יגרע. }מן האמה שייעד לו כבר׃}\rashi{\rashiDH{שארה. }מזונות (כתובות מז׃)׃}\rashi{\rashiDH{כסותה. }כמשמעו׃}\rashi{\rashiDH{ענתה. }תשמיש׃}}
{אִם אוּחְרָנְתָא יִסַּב לֵיהּ זִיּוּנַהּ כְּסוּתַהּ וְעָנְתַהּ לָא יִמְנַע׃}
{If he take him another wife, her food, her raiment, and her conjugal rights, shall he not diminish.}{\arabic{verse}}
\threeverse{\arabic{verse}}%Ex.21:11
{וְאִ֨ם־שְׁלׇשׁ־אֵ֔לֶּה לֹ֥א יַעֲשֶׂ֖ה לָ֑הּ וְיָצְאָ֥ה חִנָּ֖ם אֵ֥ין כָּֽסֶף׃ \setuma         
\rashi{\rashiDH{ואם שלש אלה לא יעשה לה. }אם אחת משלש אלה לא יעשה לה, ומה הן השלש, ייעדנה לו, או לבנו, או יגרע מפדיונה ותצא, וזה לא יעדה לא לו, ולא לבנו, והיא לא היה בידה לפדות את עצמה׃ }\rashi{\rashiDH{ויצאה חנם. }ריבה לה יציאה לזו יותר ממה שריבה לעבדים, ומה היא היציאה, ללמדך שתצא בסימנין, ותשהה עמו עוד עד שתביא סימנין, ואם הגיעו שש שנים קודם סימנין, כבר למדנו שתצא, שנאמר הָעִבְרִי אֹו הָעִבְרִיָּה וַעֲבָדְךָ שֵׁשׁ שָׁנִים (דברים טו, יב), ומהו האמור כאן ויצאה חנם, שאם קדמו סימנים לשש שנים תצא בהן (מכילתא פ״ג), או אינו אומר שתצא אלא בבגרות (קידושין ד.), תלמוד לומר אין כסף, לרבות יציאת בגרות, ואם לא נאמרו שניהם, הייתי אומר ויצאה חנם זו בגרות, לכך נאמרו שניהם שלא ליתן פתחון פה לבעל הדין לחלוק׃ }}
{וְאִם תְּלָת אִלֵּין לָא יַעֲבֵיד לַהּ וְתִפּוֹק מַגָּן דְּלָא כְסַף׃}
{And if he do not these three unto her, then shall she go out for nothing, without money.}{\arabic{verse}}
\threeverse{\aliya{ישראל}}%Ex.21:12
{מַכֵּ֥ה אִ֛ישׁ וָמֵ֖ת מ֥וֹת יוּמָֽת׃
\rashi{\rashiDH{מכה איש ומת. }כמה כתובים נאמרו בפרשת רוצחין, ומה שבידי לפרש למה באו כולם, אפרש׃ }\rashi{\rashiDH{מכה איש ומת. }למה נאמר, לפי שנאמר וְאִישׁ כִּי יַכֶּה כָּל נֶפֶשׁ אָדָם מֹות יוּמָת (ויקרא כד, יז), שומע אני הכאה בלא מיתה, תלמוד לומר מכה איש ומת, אינו חייב אלא בהכאה של מיתה (סנהדרין פד׃). ואם נאמר מכה איש ולא נאמר ואיש כי יכה, הייתי אומר אינו חייב עד שיכה איש, הכה את האשה ואת הקטן מנין, תלמוד לומר כי יכה כל נפש אדם, אפילו קטן ואפילו אשה. ועוד, אילו נאמר מכה איש, שומע אני אפילו קטן שהכה והרג יהא חייב, תלמוד לומר ואיש כי יכה (שם), ולא קטן שהכה. ועוד, כי יכה כל נפש אדם אפילו נפלים במשמע, תלמוד לומר מכה איש, שאינו חייב עד שיכה בן קיימא, הראוי להיות איש (מכילתא פ״ד)׃ }}
{דְּיִמְחֵי לַאֲנָשׁ וְיִקְטְלִנֵּיהּ אִתְקְטָלָא יִתְקְטִיל׃}
{He that smiteth a man, so that he dieth, shall surely be put to death.}{\arabic{verse}}
\threeverse{\arabic{verse}}%Ex.21:13
{וַאֲשֶׁר֙ לֹ֣א צָדָ֔ה וְהָאֱלֹהִ֖ים אִנָּ֣ה לְיָד֑וֹ וְשַׂמְתִּ֤י לְךָ֙ מָק֔וֹם אֲשֶׁ֥ר יָנ֖וּס שָֽׁמָּה׃ \setuma         
\rashi{\rashiDH{ואשר לא צדה. }לא ארב לו ולא נתכוין. צדה לשון ארב, וכן הוא אומר וְאַתָּה צֹדֶה אֶת נַפְשִׁי לְקַחְתָּהּ (שמואל־א כד, יא). ולא יתכן לומר צדה לשון הצד ציד, שצידת חיות אין נופל ה״א בפועל שלה, ושם דבר בה ציד, וזה שם דבר בו צדייה ופועל שלו צודה, וזהו פועל שלו צד. ואומר אני פתרונו כתרגומו ודלא כמן ליה. ומנחם חברו בחלק צד ציד, ואין אני מודה לו, ואם יש לחברו באחת ממחלוקת של צד, נחברנו בחלק עַל צַד תִּנָשֵׂאוּ (ישעיה סו, יב), צִדָּה אֹורֶה (שמואל־א כ, כ), וּמִלִּין לְצַד עִלָּאָה יְמַלִּל (דניאל ז, כה), אף כאן אשר לא צדה, לא צדד למצוא לו שום צד מיתה, ואף זה יש להרהר עליו, מכל מקום לשון אורב הוא׃ }\rashi{\rashiDH{והאלהים אנה לידו. }זמן לידו, לשון לֹא תְאֻנֶה אֵלֶיךָ רָעָה (תהלים צא, י), לֹא יְאֻנֶּה לַצַּדִּיק כָּל אָוֶן (משלי יב, כא), מִתְאַנֵּה הוּא לִי (מלכים־ב ה, ז), מזדמן למצוא לי עלה׃ }\rashi{\rashiDH{והאלהים אנה לידו. }ולמה תצא זאת מלפניו, הוא שאמר דוד, כַּאֲשֶׁר יֹאמַר מְשַׁל הַקַּדְמֹנִי מֵרְשָׁעִים יֵצֵא רָשַׁע (שמואל־א כד, יג), ומשל הקדמוני היא התורה, שהיא משל הקב״ה שהוא קדמונו של עולם, והיכן אמרה תורה מרשעים יצא רשע, והאלהים אנה לידו, במה הכתוב מדבר, בשני בני אדם, אחד הרג שוגג ואחד הרג מזיד, ולא היו עדים בדבר שיעידו, זה לא נהרג וזה לא גלה, והקב״ה מזמנן לפונדק אחד, זה שהרג במזיד יושב תחת הסולם, וזה שהרג שוגג עולה בסולם ונופל על זה שהרג במזיד והורגו, ועדים מעידים עליו ומחייבים אותו לגלות, נמצא זה שהרג בשוגג גולה, וזה שהרג במזיד נהרג׃ }\rashi{\rashiDH{ושמתי לך מקום. }אף במדבר שינוס שמה. ואי זה מקום קולטו, זה מחנה לויה (מכות יב׃)׃ }}
{וּדְלָא כְמַן לֵיהּ וּמִן קֳדָם יְיָ אִתְמְסַר לִידֵיהּ וַאֲשַׁוֵּי לָךְ אֲתַר דְּיִעְרוֹק לְתַמָּן׃}
{And if a man lie not in wait, but God cause it to come to hand; then I will appoint thee a place whither he may flee.}{\arabic{verse}}
\threeverse{\arabic{verse}}%Ex.21:14
{וְכִֽי־יָזִ֥ד אִ֛ישׁ עַל־רֵעֵ֖הוּ לְהׇרְג֣וֹ בְעׇרְמָ֑ה מֵעִ֣ם מִזְבְּחִ֔י תִּקָּחֶ֖נּוּ לָמֽוּת׃ \setuma         
\rashi{\rashiDH{וכי יזיד. }למה נאמר, לפי שנאמר מכה איש וגו׳, שומע אני אפילו רופא שהמית, ושליח בית דין שהמית במלקות ארבעים, והאב המכה את בנו, והרב הרודה את תלמידו, והשוגג, תלמוד לומר וכי יזיד ולא שוגג, להרגו בערמה ולא שליח בית דין והרופא והרודה בנו ותלמידו, שאף על פי שהם מזידין, אין מערימין׃ }\rashi{\rashiDH{מעם מזבחי. }אם היה כהן ורוצה לעבוד עבודה, תקחנו למות (סנהדרין לה׃, יומא פה.)׃ }}
{וַאֲרֵי יַרְשַׁע גְּבַר עַל חַבְרֵיהּ לְמִקְטְלֵיהּ בִּנְכִילוּ מִן מַדְבְּחִי תִּדְבְּרִנֵּיהּ לְמִקְטַל׃}
{And if a man come presumptuously upon his neighbour, to slay him with guile; thou shalt take him from Mine altar, that he may die.}{\arabic{verse}}
\threeverse{\arabic{verse}}%Ex.21:15
{וּמַכֵּ֥ה אָבִ֛יו וְאִמּ֖וֹ מ֥וֹת יוּמָֽת׃ \setuma         
\rashi{\rashiDH{ומכה אביו ואמו. }לפי שלמדנו על החובל בחבירו שהוא בתשלומין ולא במיתה, הוצרך לומר על החובל באביו שהוא במיתה. ואינו חייב אלא בהכאה שיש בה חבורה (סנהדרין פד׃)׃ 
}\rashi{\rashiDH{אביו ואמו. }או זה או זה׃}\rashi{\rashiDH{מות יומת. }בחנק׃}}
{וּדְיִמְחֵי לַאֲבוּהִי וּלְאִמֵּיהּ אִתְקְטָלָא יִתְקְטִיל׃}
{And he that smiteth his father, or his mother, shall be surely put to death. .}{\arabic{verse}}
\threeverse{\arabic{verse}}%Ex.21:16
{וְגֹנֵ֨ב אִ֧ישׁ וּמְכָר֛וֹ וְנִמְצָ֥א בְיָד֖וֹ מ֥וֹת יוּמָֽת׃ \setuma         
\rashi{\rashiDH{וגנב איש ומכרו. }למה נאמר, לפי שנאמר כִּי יִמָּצֵא אִישׁ גֹּנֵב נֶפֶשׁ מֵאֶחָיו (דברים כד, ז), אין לי אלא איש שגנב נפש, אשה או טומטום או אנדרוגינוס שגנבו מנין, תלמוד לומר וגונב איש ומכרו. ולפי שנאמר כאן וגונב איש, אין לי אלא גונב איש, גונב אשה מנין, תלמוד לומר וגונב נפש (שם), לכך הוצרכו שניהם, מה שחסר זה גלה זה (מכילתא פ״ה)׃ 
}\rashi{\rashiDH{ונמצא בידו. }שראוהו עדים שגנבו ומכרו, ונמצא בידו כבר קודם מכירה (סנהדרין פה׃)׃ }\rashi{\rashiDH{מות יומת. }בחנק. כל מיתה האמורה בתורה סתם, חנק היא. והפסיק הענין וכתב וגונב איש בין מכה אביו ואמו למקלל אביו ואמו, ונראה לי היינו פלוגתא, דמר סבר מקשינן הכאה לקללה, ומר סבר לא מקשינן (שם)׃ }}
{וּדְיִגְנוֹב נַפְשָׁא מִבְּנֵי יִשְׂרָאֵל וִיזַבְּנִנֵּיהּ וְיִשְׁתְּכַח בִּידֵיהּ אִתְקְטָלָא יִתְקְטִיל׃}
{And he that stealeth a man, and selleth him, or if he be found in his hand, he shall surely be put to death.}{\arabic{verse}}
\threeverse{\arabic{verse}}%Ex.21:17
{וּמְקַלֵּ֥ל אָבִ֛יו וְאִמּ֖וֹ מ֥וֹת יוּמָֽת׃ \setuma         
\rashi{\rashiDH{ומקלל אביו ואמו. }למה נאמר, לפי שהוא אומר אִישׁ אִישׁ אֲשֶׁר יִקַלֵּל אֶת אָבִיו (ויקרא כ, ט), אין לי אלא איש שקלל את אביו, אשה שקללה את אביה מנין, תלמוד לומר ומקלל אביו ואמו, סתם, בין איש ובין אשה, אם כן למה נאמר איש אשר יקלל, להוציא את הקטן׃ }\rashi{\rashiDH{מות יומת. }בסקילה. וכל מקום שנאמר דמיו בו, בסקילה, ובנין אב לכולם, בָּאֶבֶן יִרְגְּמוּ אֹתָם דְּמֵיהֶם בָּם (שם כ, כז), ובמקלל אביו ואמו נאמר דָּמָיו בֹּו (ת״כ פ׳ קדושים  קידושין ל׃)׃ }}
{וְדִילוּט לַאֲבוּהִי וְאִמֵּיהּ אִתְקְטָלָא יִתְקְטִיל׃}
{And he that curseth his father or his mother, shall surely be put to death.}{\arabic{verse}}
\threeverse{\arabic{verse}}%Ex.21:18
{וְכִֽי־יְרִיבֻ֣ן אֲנָשִׁ֔ים וְהִכָּה־אִישׁ֙ אֶת־רֵעֵ֔הוּ בְּאֶ֖בֶן א֣וֹ בְאֶגְרֹ֑ף וְלֹ֥א יָמ֖וּת וְנָפַ֥ל לְמִשְׁכָּֽב׃
\rashi{\rashiDH{וכי יריבון אנשים. }למה נאמר, לפי שנאמר עין תחת עין, לא למדנו אלא דמי איבריו, אבל שבת ורפוי לא למדנו, לכך נאמרה פרשה זו (מכילתא פ״ו)׃ }\rashi{\rashiDH{ונפל למשכב. }כתרגומו ויפל לבוטלן, לחולי שמבטלו ממלאכתו׃ }}
{וַאֲרֵי יִנְצוֹן גּוּבְרִין וְיִמְחֵי גְּבַר יָת חַבְרֵיהּ בְּאַבְנָא אוֹ בְּכוּרְמֵיזָא וְלָא יְמוּת וְיִפּוֹל לְבוּטְלָן׃}
{And if men contend, and one smite the other with a stone, or with his fist, and he die not, but keep his bed;}{\arabic{verse}}
\threeverse{\arabic{verse}}%Ex.21:19
{אִם־יָק֞וּם וְהִתְהַלֵּ֥ךְ בַּח֛וּץ עַל־מִשְׁעַנְתּ֖וֹ וְנִקָּ֣ה הַמַּכֶּ֑ה רַ֥ק שִׁבְתּ֛וֹ יִתֵּ֖ן וְרַפֹּ֥א יְרַפֵּֽא׃ \setuma         
\rashi{\rashiDH{על משענתו. }על בוריו וכחו (מכילתא שם)׃}\rashi{\rashiDH{ונקה המכה. }וכי תעלה על דעתך שיהרג זה שלא הרג, אלא ללמדך כאן, שחובשים אותו עד שנראה אם יתרפא זה, וכן משמעו, כשקם זה והלך על משענתו, אז ונקה המכה, אבל עד שלא יקום זה, לא נקה המכה׃ }\rashi{\rashiDH{רק שבתו. }בטול מלאכתו מחמת החולי, אם קטע ידו או רגלו, רואין בטול מלאכתו מחמת החולי, כאילו הוא שומר קשואין, שהרי אף לאחר החולי אינו ראוי למלאכת יד ורגל, והוא כבר נתן לו מחמת נזקו דמי ידו ורגלו, שנאמר יד תחת יד רגל תחת רגל׃ }\rashi{\rashiDH{ורפא ירפא. }כתרגומו, ישלם שכר הרופא (בבא קמא פה׃)׃ }}
{אִם יְקוּם וִיהַלֵּיךְ בְּבָרָא עַל בָּרְיֵיהּ וִיהֵי זָכָא מָחְיָא לְחוֹד בּוּטְלָנֵיהּ יִתֵּין וַאֲגַר אַסְיָא יְשַׁלֵּים׃}
{if he rise again, and walk abroad upon his staff, then shall he that smote him be quit; only he shall pay for the loss of his time, and shall cause him to be thoroughly healed.}{\arabic{verse}}
\threeverse{\aliya{שני}}%Ex.21:20
{וְכִֽי־יַכֶּה֩ אִ֨ישׁ אֶת־עַבְדּ֜וֹ א֤וֹ אֶת־אֲמָתוֹ֙ בַּשֵּׁ֔בֶט וּמֵ֖ת תַּ֣חַת יָד֑וֹ נָקֹ֖ם יִנָּקֵֽם׃
\rashi{\rashiDH{וכי יכה איש את עבדו או את אמתו. }בעבד כנעני הכתוב מדבר, או אינו אלא בעברי, תלמוד לומר כי כספו הוא, מה כספו קנוי לו עולמית, אף עבד הקנוי לו עולמית, והרי היה בכלל מכה איש ומת, אלא בא הכתוב והוציאו מן הכלל, להיות נדון בדין יום או יומים, שאם לא מת תחת ידו ושהה מעת לעת פטור׃ }\rashi{\rashiDH{בשבט. }כשיש בו כדי להמית הכתוב מדבר, או אינו אפילו אין בו כדי להמית, תלמוד לומר בישראל, וְאִם בְּאֶבֶן יָד אֲשֶׁר יָמוּת בָּה (במדבר לה, יז) (או בכלי עץ יד אשר ימות בו. גירסת רא״ם) הכהו, והלא דברים קל וחומר, מה ישראל חמור אין חייב עליו אלא אם כן הכהו בדבר שיש בו כדי להמית, ועל אבר שהוא כדי למות בהכאה זו, עבד הקל לא כל שכן׃ }\rashi{\rashiDH{נקם ינקם. }מיתת סייף (סנהדרין נב׃  מכילתא פ״ז), וכן הוא אומר חֶרֶב נֹקֶמֶת נְקַם בְּרִית (ויקרא כו, כה)׃ }}
{וַאֲרֵי יִמְחֵי גְּבַר יָת עַבְדֵּיהּ אוֹ יָת אַמְתֵּיהּ בְּשׁוּלְטָן וִימוּת תְּחוֹת יְדֵיהּ אִתְּדָנָא יִתְּדָן׃}
{And if a man smite his bondman, or his bondwoman, with a rod, and he die under his hand, he shall surely be punished.}{\arabic{verse}}
\threeverse{\arabic{verse}}%Ex.21:21
{אַ֥ךְ אִם־י֛וֹם א֥וֹ יוֹמַ֖יִם יַעֲמֹ֑ד לֹ֣א יֻקַּ֔ם כִּ֥י כַסְפּ֖וֹ הֽוּא׃ \setuma         
\rashi{\rashiDH{אך אם יום או יומים יעמוד לא יוקם. }אם על יום אחד הוא פטור על יומים לא כל שכן, אלא יום שהוא כיומים, ואיזה, זה מעת לעת (מכילתא פ״ז)׃ }\rashi{\rashiDH{לא יוקם כי כספו הוא. }הא אחר שהכהו, אף על פי ששהה מעת לעת קודם שמת, חייב׃ }}
{בְּרַם אִם יוֹמָא אוֹ תְּרֵין יוֹמִין יִתְקַיַּים לָא יִתְּדָן אֲרֵי כַסְפֵּיהּ הוּא׃}
{Notwithstanding if he continue a day or two, he shall not be punished; for he is his money.}{\arabic{verse}}
\threeverse{\arabic{verse}}%Ex.21:22
{וְכִֽי־יִנָּצ֣וּ אֲנָשִׁ֗ים וְנָ֨גְפ֜וּ אִשָּׁ֤ה הָרָה֙ וְיָצְא֣וּ יְלָדֶ֔יהָ וְלֹ֥א יִהְיֶ֖ה אָס֑וֹן עָנ֣וֹשׁ יֵעָנֵ֗שׁ כַּֽאֲשֶׁ֨ר יָשִׁ֤ית עָלָיו֙ בַּ֣עַל הָֽאִשָּׁ֔ה וְנָתַ֖ן בִּפְלִלִֽים׃
\rashi{\rashiDH{וכי ינצו אנשים. }זה עם זה, ונתכוון להכות את חבירו, והכה את האשה (מכילתא פ״ח)׃ }\rashi{\rashiDH{ונגפו. }אין נגיפה אלא לשון דחיפה והכאה, כמו פֶּן תִּגֹּף בָּאֶבֶן רַגְלֶךָ (תהלים צא, יב), וּבְטֶרֶם יִתְנַגְּפוּ רַגְלֵיכֶם (ירמיה יג, טז), וּלְאֶבֶן נֶגֶף (ישעיה ח, יד)׃ }\rashi{\rashiDH{ולא יהיה אסון }באשה׃}\rashi{\rashiDH{ענוש יענש. }לשלם דמי ולדות לבעל, שמין אותה כמה היתה ראויה להמכר בשוק, להעלות בדמיה בשביל הריונה׃ }\rashi{\rashiDH{ענש יענש. }יגבו ממון ממנו, כמו וְעָנְשׁוּ אֹתֹו מֵאָה כֶּסֶף (דברים כב, יט)׃ }\rashi{\rashiDH{כאשר ישית עליו וגו׳. }כשיתבענו הבעל בבית דין להשית עליו עונש על כך׃}\rashi{\rashiDH{ונתן. }המכה דמי ולדות׃}\rashi{\rashiDH{בפללים. }על פי הדיינים׃}}
{וַאֲרֵי יִנְצוֹן גּוּבְרִין וְיִמְחוֹן אִתְּתָא מְעַדְּיָא וְיִפְּקוּן וַלְדַּהָא וְלָא יְהֵי מוֹתָא אִתְגְּבָאָה יִתְגְּבֵי כְּמָא דִּישַׁוֵּי עֲלוֹהִי בַּעְלַהּ דְּאִתְּתָא וְיִתֵּין מִמֵּימַר דַּיָּינַיָּא׃}
{And if men strive together, and hurt a woman with child, so that her fruit depart, and yet no harm follow, he shall be surely fined, according as the woman’s husband shall lay upon him; and he shall pay as the judges determine.}{\arabic{verse}}
\threeverse{\arabic{verse}}%Ex.21:23
{וְאִם־אָס֖וֹן יִהְיֶ֑ה וְנָתַתָּ֥ה נֶ֖פֶשׁ תַּ֥חַת נָֽפֶשׁ׃
\rashi{\rashiDH{ואם אסון יהיה. }באשה׃}\rashi{\rashiDH{ונתת נפש תחת נפש. }רבותינו חולקין בדבר (סנהדרין עט.), יש אומרים נפש ממש, ויש אומרים ממון אבל לא נפש ממש, שהמתכוין להרוג את זה והרג את זה פטור ממיתה, ומשלם ליורשיו דמיו כמו שהיה נמכר בשוק׃ }}
{וְאִם מוֹתָא יְהֵי וְתִתֵּין נַפְשָׁא חֲלָף נַפְשָׁא׃}
{But if any harm follow, then thou shalt give life for life,}{\arabic{verse}}
\threeverse{\arabic{verse}}%Ex.21:24
{עַ֚יִן תַּ֣חַת עַ֔יִן שֵׁ֖ן תַּ֣חַת שֵׁ֑ן יָ֚ד תַּ֣חַת יָ֔ד רֶ֖גֶל תַּ֥חַת רָֽגֶל׃
\rashi{\rashiDH{עין תחת עין. }סימא עין חבירו, נותן לו דמי עינו כמה שפחתו דמיו למכור בשוק, וכן כולם, ולא נטילת אבר ממש, כמו שדרשו רבותינו בפרק החובל (בבא קמא פד.)׃ }}
{עֵינָא חֲלָף עֵינָא שִׁנָּא חֲלָף שִׁנָּא יְדָא חֲלָף יְדָא רִגְלָא חֲלָף רִגְלָא׃}
{eye for eye, tooth for tooth, hand for hand, foot for foot,}{\arabic{verse}}
\threeverse{\arabic{verse}}%Ex.21:25
{כְּוִיָּה֙ תַּ֣חַת כְּוִיָּ֔ה פֶּ֖צַע תַּ֣חַת פָּ֑צַע חַבּוּרָ֕ה תַּ֖חַת חַבּוּרָֽה׃ \setuma         
\rashi{\rashiDH{כויה תחת כויה. }מכות אש. ועד עכשיו דבר בחבלה שיש בה פחת דמים, ועכשיו בשאין בה פחת דמים אלא צער, כגון כוואו בשפוד על צפרניו, אומדים כמה אדם כיוצא בזה רוצה ליטול להיות מצטער כך׃ }\rashi{\rashiDH{פצע. }היא מכה המוציאה דם, שפצע את בשרו, נפר״דור בלע״ז (אפענע וואונדע) הכל לפי מה שהוא, אם יש בו פחת דמים נותן נזק, ואם נפל למשכב נותן שבת ורפוי ובשת וצער. ומקרא זה יתר הוא, ובהחובל דרשוהו רבותינו לחייב על הצער אפילו במקום נזק, שאף על פי שנותן לו דמי ידו, אין פוטרין אותו מן הצער, לומר, הואיל וקנה ידו, יש עליו לחתכה בכל מה שירצה, אלא אומרים יש לו לחתכה בסם שאינו מצטער כל כך, וזה חתכה בברזל וצערו׃ }\rashi{\rashiDH{חבורה. }היא מכה שהדם נצרר בה ואינו יוצא, אלא שמאדים הבשר כנגדו, ולשון חבורה טק״א בלע״ז (פפלעקקן) כמו וְנָמֵר חֲבְַרבֻּרֹתָיו (ירמיה יג, כג), ותרגומו משקופי, לשון חבטה, בטדור״א בלע״ז (שלאג) וכן שְׁדֻפֹות קָדִים, שקיפן קידום, חבוטות ברוח. וכן וְעַל הַמַשׁקֹוף (שמות יב, ז), על שם שהדלת נוקש עליו׃ 
}}
{כְּוַאָה חֲלָף כְּוַאָה פִּדְעָא חֲלָף פִּדְעָא מַשְׁקוֹפִי חֲלָף מַשְׁקוֹפִי׃}
{burning for burning, wound for wound, stripe for stripe.}{\arabic{verse}}
\threeverse{\arabic{verse}}%Ex.21:26
{וְכִֽי־יַכֶּ֨ה אִ֜ישׁ אֶת־עֵ֥ין עַבְדּ֛וֹ אֽוֹ־אֶת־עֵ֥ין אֲמָת֖וֹ וְשִֽׁחֲתָ֑הּ לַֽחָפְשִׁ֥י יְשַׁלְּחֶ֖נּוּ תַּ֥חַת עֵינֽוֹ׃
\rashi{\rashiDH{את עין עבדו. }כנעני, אבל עברי אינו יוצא בשן ועין, כמו שאמרנו אצל לא תצא כצאת העבדים׃ }\rashi{\rashiDH{תחת עינו. }וכן בכ״ד ראשי אברים, אצבעות הידים והרגלים, ושתי אזנים, והחוטם, וראש הגויה שהוא גיד האמה. ולמה נאמר שן ועין, שאם נאמר עין ולא נאמר שן, הייתי אומר, מה עין שנברא עמו אף כל שנברא עמו, והרי שן לא נברא עמו. ואם נאמר שן ולא נאמר עין, הייתי אומר, אפילו שן תינוק שיש לה חליפין, לכך נאמר עין (מכילתא פ״ט)׃ 
}}
{וַאֲרֵי יִמְחֵי גְּבַר יָת עֵינָא דְּעַבְדֵּיהּ אוֹ יָת עֵינָא דְּאַמְתֵּיהּ וִיחַבְּלִנַּהּ לְבַר חוֹרִין יִפְטְרִנֵּיהּ חֲלָף עֵינֵיהּ׃}
{And if a man smite the eye of his bondman, or the eye of his bondwoman, and destroy it, he shall let him go free for his eye’s sake.}{\arabic{verse}}
\threeverse{\arabic{verse}}%Ex.21:27
{וְאִם־שֵׁ֥ן עַבְדּ֛וֹ אֽוֹ־שֵׁ֥ן אֲמָת֖וֹ יַפִּ֑יל לַֽחׇפְשִׁ֥י יְשַׁלְּחֶ֖נּוּ תַּ֥חַת שִׁנּֽוֹ׃ \petucha }
{וְאִם שִׁנָּא דְּעַבְדֵּיהּ אוֹ שִׁנָּא דְּאַמְתֵּיהּ יַפִּיל לְבַר חוֹרִין יִפְטְרִנֵּיהּ חֲלָף שִׁנֵּיהּ׃}
{And if he smite out his bondman’s tooth, or his bondwoman’s tooth, he shall let him go free for his tooth’s sake.}{\arabic{verse}}
\threeverse{\arabic{verse}}%Ex.21:28
{וְכִֽי־יִגַּ֨ח שׁ֥וֹר אֶת־אִ֛ישׁ א֥וֹ אֶת־אִשָּׁ֖ה וָמֵ֑ת סָק֨וֹל יִסָּקֵ֜ל הַשּׁ֗וֹר וְלֹ֤א יֵאָכֵל֙ אֶת־בְּשָׂר֔וֹ וּבַ֥עַל הַשּׁ֖וֹר נָקִֽי׃
\rashi{\rashiDH{וכי יגח שור. }אחד שור ואחד כל בהמה וחיה ועוף, אלא שדבר הכתוב בהווה (בבא קמא נד׃)׃ }\rashi{\rashiDH{ולא יאכל את בשרו. }ממשמע שנאמר סקול יסקל השור, איני יודע שהוא נבלה, ונבלה אסורה באכילה, אלא מה תלמוד לומר ולא יאכל את בשרו, שאפילו שחטו לאחר שנגמר דינו אסור באכילה, בהנאה מנין, תלמוד לומר ובעל השור נקי, כאדם האומר לחברו יצא פלוני נקי מנכסיו ואין לו בהם הנאה של כלום, זהו מדרשו (פסחים כב׃, בבא קמא מא). ופשוטו כמשמעו, לפי שנאמר במועד, וגם בעליו יומת, הוצרך לומר בתם ובעל השור נקי׃ }}
{וַאֲרֵי יִגַּח תּוֹרָא יָת גּוּבְרָא אוֹ יָת אִתְּתָא וִימוּת אִתְרְגָמָא יִתְרְגֵים תּוֹרָא וְלָא יִתְאֲכִיל יָת בִּסְרֵיהּ וּמָרֵיהּ דְּתוֹרָא יְהֵי זָכָא׃}
{And if an ox gore a man or a woman, that they die, the ox shall be surely stoned, and its flesh shall not be eaten; but the owner of the ox shall be quit.}{\arabic{verse}}
\threeverse{\arabic{verse}}%Ex.21:29
{וְאִ֡ם שׁוֹר֩ נַגָּ֨ח ה֜וּא מִתְּמֹ֣ל שִׁלְשֹׁ֗ם וְהוּעַ֤ד בִּבְעָלָיו֙ וְלֹ֣א יִשְׁמְרֶ֔נּוּ וְהֵמִ֥ית אִ֖ישׁ א֣וֹ אִשָּׁ֑ה הַשּׁוֹר֙ יִסָּקֵ֔ל וְגַם־בְּעָלָ֖יו יוּמָֽת׃
\rashi{\rashiDH{מתמל שלשום. }הרי שלש נגיחות (מכילתא פ״י)׃ 
}\rashi{\rashiDH{והועד בבעליו. }לשון התראה בעדים, כמו הָעֵד הֵעִד בָּנוּ הָאִישׁ (בראשית מג, ג)׃ }\rashi{\rashiDH{והמית איש וגו׳. }לפי שנאמר כי יגח, אין לי אלא שהמיתו בנגיחה, המיתו בנשיכה, דחיפה, ובעיטה, מניין, תלמוד לומר והמית׃ }\rashi{\rashiDH{וגם בעליו יומת. }בידי שמים, יכול בידי אדם, תלמוד לומר מֹות יוּמַת הַמַּכֶּה רֹצֵחַ הוּא (במדבר לה, טז), על רציחתו אתה הורגו, ואי אתה הורגו על רציחת שורו (סנהדרין טו׃)׃ }}
{וְאִם תּוֹר נַגָּח הוּא מֵאֶתְמָלִי וּמִדְּקַמּוֹהִי וְאִתַּסְהַד בְּמָרֵיהּ וְלָא נַטְרֵיהּ וְיִקְטוֹל גְּבַר אוֹ אִתָּא תּוֹרָא יִתְרְגֵים וְאַף מָרֵיהּ יִתְקְטִיל׃}
{But if the ox was wont to gore in time past, and warning hath been given to its owner, and he hath not kept it in, but it hath killed a man or a woman; the ox shall be stoned, and its owner also shall be put to death.}{\arabic{verse}}
\threeverse{\arabic{verse}}%Ex.21:30
{אִם־כֹּ֖פֶר יוּשַׁ֣ת עָלָ֑יו וְנָתַן֙ פִּדְיֹ֣ן נַפְשׁ֔וֹ כְּכֹ֥ל אֲשֶׁר־יוּשַׁ֖ת עָלָֽיו׃
\rashi{\rashiDH{אם כופר יושת עליו. }אם זה אינו תלוי, והרי הוא כמו אם כסף תלוה, לשון אשר, זה משפטו, שישיתו עליו בית דין כופר׃ }\rashi{\rashiDH{ונתן פדיון נפשו. }דמי ניזק, דברי רבי ישמעאל. רבי עקיבא אומר, דמי מזיק (בבא קמא כז.)׃ }}
{אִם מָמוֹן יְשַׁוּוֹן עֲלוֹהִי וְיִתֵּין פּוּרְקַן נַפְשֵׁיהּ כְּכֹל דִּישַׁוּוֹן עֲלוֹהִי׃}
{If there be laid on him a ransom, then he shall give for the redemption of his life whatsoever is laid upon him.}{\arabic{verse}}
\threeverse{\arabic{verse}}%Ex.21:31
{אוֹ־בֵ֥ן יִגָּ֖ח אוֹ־בַ֣ת יִגָּ֑ח כַּמִּשְׁפָּ֥ט הַזֶּ֖ה יֵעָ֥שֶׂה לּֽוֹ׃
\rashi{\rashiDH{או בן יגח. }בן שהוא קטן׃}\rashi{\rashiDH{או בת. }שהיא קטנה, לפי שנאמר והמית איש או אשה, יכול אינו חייב אלא על הגדולים, תלמוד לומר או בן יגח וגו׳, לחייב על הקטנים כגדולים (מכילתא נזיקין פי״א)׃ }}
{אוֹ לְבַר יִשְׂרָאֵל יִגַּח תּוֹרָא אוֹ לְבַת יִשְׂרָאֵל יִגַּח כְּדִינָא הָדֵין יִתְעֲבֵיד לֵיהּ׃}
{Whether it have gored a son, or have gored a daughter, according to this judgment shall it be done unto him.}{\arabic{verse}}
\threeverse{\arabic{verse}}%Ex.21:32
{אִם־עֶ֛בֶד יִגַּ֥ח הַשּׁ֖וֹר א֣וֹ אָמָ֑ה כֶּ֣סֶף \legarmeh  שְׁלֹשִׁ֣ים שְׁקָלִ֗ים יִתֵּן֙ לַֽאדֹנָ֔יו וְהַשּׁ֖וֹר יִסָּקֵֽל׃ \setuma         
\rashi{\rashiDH{אם עבד או אמה. }כנעניים (מכילתא שם)׃}\rashi{\rashiDH{שלשים שקלים יתן. }גזירת הכתוב הוא, בין שהוא שוה אלף זוז, בין שאינו שוה אלא דינר. והשקל משקלו ד׳ זהובים, שהם חצי אונקיא למשקל הישר של קלוני״א׃ }}
{אִם לְעַבְדָּא יִגַּח תּוֹרָא אוֹ לְאַמְתָּא כַּסְפָּא תְּלָתִין סִלְעִין יִתֵּין לְרִבּוֹנֵיהּ וְתוֹרָא יִתְרְגֵים׃}
{If the ox gore a bondman or a bondwoman, he shall give unto their master thirty shekels of silver, and the ox shall be stoned.}{\arabic{verse}}
\threeverse{\arabic{verse}}%Ex.21:33
{וְכִֽי־יִפְתַּ֨ח אִ֜ישׁ בּ֗וֹר א֠וֹ כִּֽי־יִכְרֶ֥ה אִ֛ישׁ בֹּ֖ר וְלֹ֣א יְכַסֶּ֑נּוּ וְנָֽפַל־שָׁ֥מָּה שּׁ֖וֹר א֥וֹ חֲמֽוֹר׃
\rashi{\rashiDH{וכי יפתח איש בור. }שהיה מכוסה וגלהו׃}\rashi{\rashiDH{או כי יכרה. }למה נאמר, אם על הפתיחה חייב על הכרייה לא כל שכן, אלא להביא כורה אחר כורה שהוא חייב (בבא קמא נא.)׃ 
}\rashi{\rashiDH{ולא יכסנו. }הא אם כסהו פטור, ובחופר ברשות הרבים דבר הכתוב (שם נ.)׃ }\rashi{\rashiDH{שור או חמור. }הוא הדין לכל בהמה וחיה, שבכל מקום שנאמר שור וחמור, אנו למדין אותו שור שור משבת, שנאמר למען ינוח שורך וחמורך, מה להלן כל בהמה וחיה כשור, שהרי נאמר במקום אחר וכל בהמתך, אף כאן כל בהמה וחיה כשור, ולא נאמר שור וחמור אלא שור ולא אדם חמור ולא כלים (שם נג׃)׃ }}
{וַאֲרֵי יִפְתַּח גְּבַר גּוּב אוֹ אֲרֵי יִכְרֵי גְּבַר גּוּב וְלָא יְכַסֵּינֵיהּ וְיִפּוֹל תַּמָּן תּוֹרָא אוֹ חֲמָרָא׃}
{And if a man shall open a pit, or if a man shall dig a pit and not cover it, and an ox or an ass fall therein,}{\arabic{verse}}
\threeverse{\arabic{verse}}%Ex.21:34
{בַּ֤עַל הַבּוֹר֙ יְשַׁלֵּ֔ם כֶּ֖סֶף יָשִׁ֣יב לִבְעָלָ֑יו וְהַמֵּ֖ת יִֽהְיֶה־לּֽוֹ׃ \setuma         
\rashi{\rashiDH{בעל הבור. }בעל התקלה, אף על פי שאין הבור שלו, שעשאו ברשות הרבים, עשאו הכתוב בעליו להתחייב עליו בנזקיו׃ }\rashi{\rashiDH{כסף ישיב לבעליו. }ישיב, לרבות שוה כסף ואפילו סובין (שם ז.), (והא דכתיב מיטב שדהו וכתב רש״י שהניזקין מן העידית, תירץ בגמ׳ בפרק קמא דבבא קמא רב הונא בריה דרב יהושע, דזהו אם בא לגבות קרקע, אבל מטלטלי כל מילי מיטב הוא, דאי לא מזדבן הכא מזדבן הכא)׃ }\rashi{\rashiDH{והמת יהיה לו. }לניזק, שמין את הנבלה ונוטלה בדמים, ומשלם לו המזיק עליה תשלומי נזקו (מכילתא פי״א  בבא קמא י׃)׃ 
}}
{מָרֵיהּ דְּגוּבָּא יְשַׁלֵּים כַּסְפָּא יָתִיב לְמָרוֹהִי וּמִיתָא יְהֵי דִּילֵיהּ׃}
{the owner of the pit shall make it good; he shall give money unto the owner of them, and the dead beast shall be his.}{\arabic{verse}}
\threeverse{\arabic{verse}}%Ex.21:35
{וְכִֽי־יִגֹּ֧ף שֽׁוֹר־אִ֛ישׁ אֶת־שׁ֥וֹר רֵעֵ֖הוּ וָמֵ֑ת וּמָ֨כְר֜וּ אֶת־הַשּׁ֤וֹר הַחַי֙ וְחָצ֣וּ אֶת־כַּסְפּ֔וֹ וְגַ֥ם אֶת־הַמֵּ֖ת יֶֽחֱצֽוּן׃
\rashi{\rashiDH{וכי יגוף. }ידחוף, בין בקרניו, בין בגופו, בין ברגליו, בין שנשכו בשיניו, כולן בכלל נגיפה הם, שאין נגיפה אלא לשון מכה׃ }\rashi{\rashiDH{שור איש. }שור של איש׃}\rashi{\rashiDH{ומכרו את השור וגו׳. }בשוים הכתוב מדבר, שור שוה מאתים שהמית שור שוה מאתים, בין שהנבלה שוה הרבה בין שהיא שוה מעט, כשנוטל זה חצי החי וחצי המת וזה חצי החי וחצי המת, נמצא כל אחד מפסיד חצי נזק שהזיקה המיתה, למדנו, שהתם משלם חצי נזק, שמן השוין אתה למד לשאינן שוין, כי דין התם לשלם חצי נזק לא פחות ולא יותר. או יכול אף בשאינן שוין בדמיהן כשהן חיים אמר הכתוב וחצו את שניהם, אם אמרת כן, פעמים שהמזיק משתכר הרבה, כשהנבלה שוה לימכר לעובד כוכבים הרבה יותר מדמי שור המזיק, ואי אפשר שיאמר הכתוב שיהא המזיק נשכר, או פעמים שהניזק נוטל הרבה יותר מדמי נזק שלם, שחצי דמי שור המזיק שוה יותר מכל דמי שור הניזק, ואם אמרת כן הרי תם חמור ממועד. על כרחך לא דבר הכתוב אלא בשוין, ולמדך שהתם משלם חצי נזק, ומן השוין תלמוד לשאינן שוין, שהמשתלם חצי נזקו, שמין לו את הנבלה, ומה שפחתו דמיו בשביל המיתה נוטל חצי הפחת והולך. ולמה אמר הכתוב בלשון הזה ולא אמר ישלם חציו, ללמד שאין התם משלם אלא מגופו, ואם נגח ומת אח״כ, אין הניזק נוטל אלא הנבלה, ואם אינה מגעת לחצי נזקו יפסיד. או שור שוה מנה שנגח שור שוה חמש מאות זוז, אינו נוטל אלא את השור, שלא נתחייב התם לחייב את בעליו לשלם מן העליה (בבא קמא טז׃)׃ }}
{וַאֲרֵי יִגּוֹף תּוֹר דִּגְבַר יָת תּוֹרָא דְּחַבְרֵיהּ וִימוּת וִיזַבְּנוּן יָת תּוֹרָא חַיָא וְיִפְלְגוּן יָת כַּסְפֵּיהּ וְאַף יָת דְּמֵי מִיתָא יִפְלְגוּן׃}
{And if one man’s ox hurt another’s, so that it dieth; then they shall sell the live ox, and divide the price of it; and the dead also they shall divide.}{\arabic{verse}}
\threeverse{\arabic{verse}}%Ex.21:36
{א֣וֹ נוֹדַ֗ע כִּ֠י שׁ֣וֹר נַגָּ֥ח הוּא֙ מִתְּמ֣וֹל שִׁלְשֹׁ֔ם וְלֹ֥א יִשְׁמְרֶ֖נּוּ בְּעָלָ֑יו שַׁלֵּ֨ם יְשַׁלֵּ֥ם שׁוֹר֙ תַּ֣חַת הַשּׁ֔וֹר וְהַמֵּ֖ת יִֽהְיֶה־לּֽוֹ׃ \setuma         
\rashi{\rashiDH{או נודע. }או לא היה תם, אלא נודע כי שור נגח הוא, היום ומתמול שלשום, הרי שלש נגיחות׃ }\rashi{\rashiDH{שלם ישלם שור. }נזק שלם׃ 
}\rashi{\rashiDH{והמת יהיה לו. }לניזק, ועליו ישלים המזיק עד שישתלם ניזק כל נזקו׃ }}
{אוֹ אִתְיְדַע אֲרֵי תּוֹר נַגָּח הוּא מֵאֶתְמָלִי וּמִדְּקַמּוֹהִי וְלָא נַטְרֵיהּ מָרֵיהּ שַׁלָּמָא יְשַׁלֵּים תּוֹרָא חֲלָף תּוֹרָא וּמִיתָא יְהֵי דִּילֵיהּ׃}
{Or if it be known that the ox was wont to gore in time past, and its owner hath not kept it in; he shall surely pay ox for ox, and the dead beast shall be his own.}{\arabic{verse}}
\threeverse{\arabic{verse}}%Ex.21:37
{כִּ֤י יִגְנֹֽב־אִישׁ֙ שׁ֣וֹר אוֹ־שֶׂ֔ה וּטְבָח֖וֹ א֣וֹ מְכָר֑וֹ חֲמִשָּׁ֣ה בָקָ֗ר יְשַׁלֵּם֙ תַּ֣חַת הַשּׁ֔וֹר וְאַרְבַּע־צֹ֖אן תַּ֥חַת הַשֶּֽׂה׃
\rashi{\rashiDH{חמשה בקר וגו׳. }אמר ר׳ יוחנן בן זכאי, חס המקום על כבודן של בריות, שור שהולך ברגליו, ולא נתבזה בו הגנב לנושאו על כתפו, משלם חמשה, שה שנושאו על כתפו, משלם ארבעה, הואיל ונתבזה בו. אמר רבי מאיר, בא וראה כמה גדולה כחה של מלאכה, שור שבטלו ממלאכתו, חמשה, שה שלא בטלו ממלאכתו, ארבעה׃ }\rashi{\rashiDH{תחת השור תחת השה. }שנאן הכתוב לומר, שאין מדת תשלומי ד׳ וה׳ נוהגת אלא בשור ושה בלבד (שם סז׃)׃ }}
{אֲרֵי יִגְנוֹב גְּבַר תּוֹר אוֹ אִמַּר וְיִכְּסִנֵּיהּ אוֹ יְזַבְּנִנֵּיהּ חַמְשָׁא תוֹרִין יְשַׁלֵּים חֲלָף תּוֹרָא וְאַרְבַּע עָנָא חֲלָף אִמְּרָא׃}
{If a man steal an ox, or a sheep, and kill it, or sell it, he shall pay five oxen for an ox, and four sheep for a sheep.}{\arabic{verse}}
\newperek
\threeverse{\Roman{chap}}%Ex.22:1
{אִם־בַּמַּחְתֶּ֛רֶת יִמָּצֵ֥א הַגַּנָּ֖ב וְהֻכָּ֣ה וָמֵ֑ת אֵ֥ין ל֖וֹ דָּמִֽים׃
\rashi{\rashiDH{אם במחתרת. }כשהיה חותר את הבית׃}\rashi{\rashiDH{אין לו דמים. }אין זו רציחה, הרי הוא כמת מעיקרו. כאן למדתך תורה, אם בא להרגך השכם להרגו, וזה להרגך בא, שהרי יודע הוא שאין אדם מעמיד עצמו ורואה שנוטלין ממונו בפניו ושותק, לפיכך, על מנת כן בא, שאם יעמוד בעל הממון כנגדו, יהרגנו (סנהדרין עב.)׃ }}
{אִם בְּמַחְתַּרְתָּא יִשְׁתְּכַח גַּנָּבָא וְיִתְמְחֵי וִימוּת לֵית לֵיהּ דַּם׃}
{If a thief be found breaking in, and be smitten so that he dieth, there shall be no bloodguiltiness for him.}{\Roman{chap}}
\threeverse{\arabic{verse}}%Ex.22:2
{אִם־זָרְחָ֥ה הַשֶּׁ֛מֶשׁ עָלָ֖יו דָּמִ֣ים ל֑וֹ שַׁלֵּ֣ם יְשַׁלֵּ֔ם אִם־אֵ֣ין ל֔וֹ וְנִמְכַּ֖ר בִּגְנֵבָתֽוֹ׃
\rashi{\rashiDH{אם זרחה השמש עליו. }אין זה אלא כמין משל, אם ברור לך הדבר שיש לו שלום עמך, כשמש הזה שהוא שלום בעולם, כך פשוט לך שאינו בא להרוג אפילו יעמוד בעל הממון כנגדו, כגון אב החותר לגנוב ממון הבן, בידוע שרחמי האב על הבן ואינו בא על עסקי נפשות (מכילתא פי״ג,  סנהדרין עב.)׃ }\rashi{\rashiDH{דמים לו. }כחי הוא חשוב, ורציחה היא אם יהרגנו בעל הבית׃ }\rashi{\rashiDH{שלם ישלם. }הגנב ממון שגנב, ואינו חייב מיתה. ואונקלוס שתרגם אם עינא דסהדיא נפלת עלוהי, לקח לו שטה אחרת, לומר שאם מצאוהו עדים קודם שבא בעל הבית, וכשבא בעל הבית נגדו התרו בו שלא יהרגהו, דמים לו, חייב עליו אם הרגו, שמאחר שיש רואים לו, אין הגנב הזה בא על עסקי נפשות, ולא יהרוג את בעל הממון׃ }}
{אִם עֵינָא דְּסָהֲדַיָּא נְפַלַת עֲלוֹהִי דְּמָא לֵיהּ שַׁלָּמָא יְשַׁלֵּים אִם לֵית לֵיהּ וְיִזְדַּבַּן בִּגְנוּבְתֵּיהּ׃}
{If the sun be risen upon him, there shall be bloodguiltiness for him—he shall make restitution; if he have nothing, then he shall be sold for his theft.}{\arabic{verse}}
\threeverse{\arabic{verse}}%Ex.22:3
{אִֽם־הִמָּצֵא֩ תִמָּצֵ֨א בְיָד֜וֹ הַגְּנֵבָ֗ה מִשּׁ֧וֹר עַד־חֲמ֛וֹר עַד־שֶׂ֖ה חַיִּ֑ים שְׁנַ֖יִם יְשַׁלֵּֽם׃ \setuma         
\rashi{\rashiDH{אם המצא תמצא בידו. }ברשותו, שלא טבח ולא מכר׃ }\rashi{\rashiDH{משור עד חמור. }כל דבר בכלל תשלומי כפל, בין שיש בו רוח חיים בין שאין בו רוח חיים, שהרי נאמר במקום אחר עַל שֶׂה עַל שַׁלְמָה עַל כָּל אֲבֵדָה וגו׳ יְשַׁלֵּם שְׁנַיִם לְרֵעֵהוּ׃ }\rashi{\rashiDH{חיים שנים ישלם. }ולא ישלם לו מתים, אלא חיים או דמי חיים׃ }}
{אִם אִשְׁתְּכָחָא תִשְׁתְּכַח בִּידֵיהּ גְּנוּבְתָּא מִתּוֹר עַד חֲמָר עַד אִמַּר אִנּוּן חַיִּין עַל חַד תְּרֵין יְשַׁלֵּים׃}
{If the theft be found in his hand alive, whether it be ox, or ass, or sheep, he shall pay double.}{\arabic{verse}}
\threeverse{\aliya{שלישי}}%Ex.22:4
{כִּ֤י יַבְעֶר־אִישׁ֙ שָׂדֶ֣ה אוֹ־כֶ֔רֶם וְשִׁלַּח֙ אֶת־בְּעִירֹ֔ה וּבִעֵ֖ר בִּשְׂדֵ֣ה אַחֵ֑ר מֵיטַ֥ב שָׂדֵ֛הוּ וּמֵיטַ֥ב כַּרְמ֖וֹ יְשַׁלֵּֽם׃ \setuma         
\rashi{\rashiDH{כי יבער. את בעירה. ובער. }כולם לשון בהמה, כמו אֲנַחְנוּ וּבְעִירנוּ (במדבר כ, ד)׃ }\rashi{\rashiDH{כי יבער. }יוליך בהמותיו בשדה וכרם של חבירו, ויזיק אותו באחת משתי אלו, או בשלוח בעירה, או בביעור, ופירשו רבותינו (בבא קמא ב׃), ושלח הוא נזקי מדרך כף רגל, ובער הוא נזקי השן האוכלת ומבערת׃ }\rashi{\rashiDH{בשדה אחר. }בשדה של איש אחר׃}\rashi{\rashiDH{מיטב שדהו ישלם. }שמין את הנזק, ואם בא לשלם לו קרקע דמי נזקו, ישלם לו ממיטב שדותיו, אם היה נזקו סלע, יתן לו שוה סלע מעידית שיש לו. למדך הכתוב, שהנזקין שמין להם בעידית (שם ו׃)׃ }}
{אֲרֵי יוֹכֵיל גְּבַר חֲקַל אוֹ כְרַם וִישַׁלַּח יָת בְּעִירֵיהּ וְיֵיכוֹל בַּחֲקַל אָחֳרָן שְׁפַר חַקְלֵיהּ וּשְׁפַר כַּרְמֵיהּ יְשַׁלֵּים׃}
{If a man cause a field or vineyard to be eaten, and shall let his beast loose, and it feed in another man’s field; of the best of his own field, and of the best of his own vineyard, shall he make restitution.}{\arabic{verse}}
\threeverse{\arabic{verse}}%Ex.22:5
{כִּֽי־תֵצֵ֨א אֵ֜שׁ וּמָצְאָ֤ה קֹצִים֙ וְנֶאֱכַ֣ל גָּדִ֔ישׁ א֥וֹ הַקָּמָ֖ה א֣וֹ הַשָּׂדֶ֑ה שַׁלֵּ֣ם יְשַׁלֵּ֔ם הַמַּבְעִ֖ר אֶת־הַבְּעֵרָֽה׃ \setuma         
\rashi{\rashiDH{כי תצא אש. }אפילו מעצמה׃}\rashi{\rashiDH{ומצאה קוצים. }קרדו״נש בלע״ז׃ 
}\rashi{\rashiDH{ונאכל גדיש. }שליחכה בקוצים, עד שהגיעה לגדיש או לקמה המחוברת בקרקע׃ }\rashi{\rashiDH{או השדה. }שליחכה את נירו, וצריך לניר אותה פעם שניה׃ }\rashi{\rashiDH{שלם ישלם המבעיר. }אף על פי שהדליק בתוך שלו, והיא יצאה מעצמה על ידי קוצים שמצאה, חייב לשלם, לפי שלא שמר את גחלתו שלא תצא ותזיק׃ }}
{אֲרֵי יִתַּפַּק נוּר וְיַשְׁכַּח כּוּבִּין וְיֵיכוֹל גְּדִישִׁין אוֹ קָמָא אוֹ חֲקַל שַׁלָּמָא יְשַׁלֵּים דְּאַדְלֵיק יָת דְּלֵיקְתָא׃}
{If fire break out, and catch in thorns, so that the shocks of corn, or the standing corn, or the field are consumed; he that kindled the fire shall surely make restitution.}{\arabic{verse}}
\threeverse{\arabic{verse}}%Ex.22:6
{כִּֽי־יִתֵּן֩ אִ֨ישׁ אֶל־רֵעֵ֜הוּ כֶּ֤סֶף אֽוֹ־כֵלִים֙ לִשְׁמֹ֔ר וְגֻנַּ֖ב מִבֵּ֣ית הָאִ֑ישׁ אִם־יִמָּצֵ֥א הַגַּנָּ֖ב יְשַׁלֵּ֥ם שְׁנָֽיִם׃
\rashi{\rashiDH{וגנב מבית האיש. }לפי דבריו (שם סג׃)׃}\rashi{\rashiDH{אם ימצא הגנב. }ישלם הגנב שנים לבעלים׃}}
{אֲרֵי יִתֵּין גְּבַר לְחַבְרֵיהּ כְּסַף אוֹ מָנִין לְמִטַּר וְיִתְגַּנְבוּן מִבֵּית גּוּבְרָא אִם יִשְׁתְּכַח גַּנָּבָא יְשַׁלֵּים עַל חַד תְּרֵין׃}
{If a man deliver unto his neighbour money or stuff to keep, and it be stolen out of the man’s house; if the thief be found, he shall pay double.}{\arabic{verse}}
\threeverse{\arabic{verse}}%Ex.22:7
{אִם־לֹ֤א יִמָּצֵא֙ הַגַּנָּ֔ב וְנִקְרַ֥ב בַּֽעַל־הַבַּ֖יִת אֶל־הָֽאֱלֹהִ֑ים אִם־לֹ֥א שָׁלַ֛ח יָד֖וֹ בִּמְלֶ֥אכֶת רֵעֵֽהוּ׃
\rashi{\rashiDH{אם לא ימצא הגנב. }ובא השומר הזה שהוא בעל הבית׃ 
}\rashi{\rashiDH{ונקרב. }אל הדיינין, לדון עם זה, ולישבע לו שלא שלח ידו בשלו׃ }}
{אִם לָא יִשְׁתְּכַח גַּנָּבָא וְיִתְקָרַב מָרֵיהּ דְּבֵיתָא לִקְדָם דַּיָּינַיָּא אִם לָא אוֹשֵׁיט יְדֵיהּ בְּמָא דִּמְסַר לֵיהּ חַבְרֵיהּ׃}
{If the thief be not found, then the master of the house shall come near unto God, to see whether he have not put his hand unto his neighbour’s goods.}{\arabic{verse}}
\threeverse{\arabic{verse}}%Ex.22:8
{עַֽל־כׇּל־דְּבַר־פֶּ֡שַׁע עַל־שׁ֡וֹר עַל־חֲ֠מ֠וֹר עַל־שֶׂ֨ה עַל־שַׂלְמָ֜ה עַל־כׇּל־אֲבֵדָ֗ה אֲשֶׁ֤ר יֹאמַר֙ כִּי־ה֣וּא זֶ֔ה עַ֚ד הָֽאֱלֹהִ֔ים יָבֹ֖א דְּבַר־שְׁנֵיהֶ֑ם אֲשֶׁ֤ר יַרְשִׁיעֻן֙ אֱלֹהִ֔ים יְשַׁלֵּ֥ם שְׁנַ֖יִם לְרֵעֵֽהוּ׃ \setuma         
\rashi{\rashiDH{על כל דבר פשע. }שימצא שקרן בשבועתו, שיעידו עדים שהוא עצמו גנבו, וירשיעוהו אלהים על פי העדים׃ }\rashi{\rashiDH{ישלם שנים לרעהו. }למדך הכתוב, שהטוען בפקדון לומר נגנב הימני, ונמצא שהוא עצמו גנבו, משלם תשלומי כפל, ואימתי, בזמן שנשבע ואחר כך באו עדים. שכך דרשו רבותינו, ונקרב בעל הבית אל האלהים, קריבה זו שבועה היא, אתה אומר לשבועה או אינו אלא לדין, שכיון שבא לדין וכפר לומר נגנבה, מיד יתחייב בכפל אם באו עדים שהוא בידו, נאמר כאן שליחות יד, ונאמר למטה שליחות יד שְׁבֻעַת ה׳ תִּהְיֶה בֵּין שְׁנֵיהֶם אִם לֹא שָׁלַח יָדֹו, מה להלן שבועה אף כאן שבועה׃ }\rashi{\rashiDH{אשר יאמר כי הוא זה. }לפי פשוטו, אשר יאמר העד כי הוא זה שנשבעת עליו הרי הוא אצלך, עד הדיינין יבא דבר שניהם ויחקרו את העדות, ואם כשרים הם וירשיעוהו לשומר זה, ישלם שנים, ואם ירשיעו את העדים שנמצאו זוממין, ישלמו הם שנים לשומר. ורבותינו ז״ל דרשו, כי הוא זה, ללמד שאין מחייבין אותו שבועה אלא אם כן הודה במקצת, לומר כך וכך אני חייב לך, והמותר נגנב ממני (שם קו׃)׃}}
{עַל כָּל פִּתְגָם דְּחוֹב עַל תּוֹר עַל חֲמָר עַל אִמַּר עַל כְּסוּ עַל כָּל אֲבֵידְתָא דְּיֵימַר אֲרֵי הוּא דֵין לִקְדָם דַּיָּינַיָּא יֵיעוֹל דִּין תַּרְוֵיהוֹן דִּיחַיְּבוּן דַּיָּינַיָּא יְשַׁלֵּים עַל חַד תְּרֵין לְחַבְרֵיהּ׃}
{For every matter of trespass, whether it be for ox, for ass, for sheep, for raiment, or for any manner of lost thing, whereof one saith: 'This is it,' the cause of both parties shall come before God; he whom God shall condemn shall pay double unto his neighbour.}{\arabic{verse}}
\threeverse{\arabic{verse}}%Ex.22:9
{כִּֽי־יִתֵּן֩ אִ֨ישׁ אֶל־רֵעֵ֜הוּ חֲמ֨וֹר אוֹ־שׁ֥וֹר אוֹ־שֶׂ֛ה וְכׇל־בְּהֵמָ֖ה לִשְׁמֹ֑ר וּמֵ֛ת אוֹ־נִשְׁבַּ֥ר אוֹ־נִשְׁבָּ֖ה אֵ֥ין רֹאֶֽה׃
\rashi{\rashiDH{כי יתן איש אל רעהו חמור או שור. פרשה ראשונה נאמרה בשומר חנם (בבא מציעא צד׃) לפיכך פטר בו את הגנבה, כמו שכתוב וגונב מבית האיש אם לא ימצא הגנב ונקרב בעל הבית, לשבועה, למדת שפוטר עצמו בשבועה זו, ופרשה זו אמורה בשומר שכר, לפיכך אינו פטור אם נגנבה, כמו שכתוב אם גנוב יגנב מעמו ישלם, אבל על האונס, כמו מת מעצמו, או נשבר, או נשבה בחזקה על ידי לסטים. }ואין רואה. יעיד בדבר׃ }}
{אֲרֵי יִתֵּין גְּבַר לְחַבְרֵיהּ חֲמָר אוֹ תוֹר אוֹ אִמַּר וְכָל בְּעִירָא לְמִטַּר וּמִית אוֹ אִתְּבַר אוֹ אִשְׁתְּבִי לֵית דְּחָזֵי׃}
{If a man deliver unto his neighbour an ass, or an ox, or a sheep, or any beast, to keep, and it die, or be hurt, or driven away, no man seeing it;}{\arabic{verse}}
\threeverse{\arabic{verse}}%Ex.22:10
{שְׁבֻעַ֣ת יְהֹוָ֗ה תִּהְיֶה֙ בֵּ֣ין שְׁנֵיהֶ֔ם אִם־לֹ֥א שָׁלַ֛ח יָד֖וֹ בִּמְלֶ֣אכֶת רֵעֵ֑הוּ וְלָקַ֥ח בְּעָלָ֖יו וְלֹ֥א יְשַׁלֵּֽם׃
\rashi{\rashiDH{שבעת ה׳ תהיה. }ישבע שכן הוא כדבריו, והוא לא שלח בה יד להשתמש בה לעצמו, שאם שלח בה יד ואחר כך נאנסה, חייב באונסים׃ }\rashi{\rashiDH{ולקח בעליו. }השבועה׃}\rashi{\rashiDH{ולא ישלם. }לו השומר כלום׃}}
{מוֹמָתָא דַּייָ תְּהֵי בֵּין תַּרְוֵיהוֹן אִם לָא אוֹשֵׁיט יְדֵיהּ בְּמָא דִּמְסַר לֵיהּ חַבְרֵיהּ וִיקַבֵּיל מָרֵיהּ מִנֵּיהּ מוֹמָתָא וְלָא יְשַׁלֵּים׃}
{the oath of the \lord\space shall be between them both, to see whether he have not put his hand unto his neighbour’s goods; and the owner thereof shall accept it, and he shall not make restitution.}{\arabic{verse}}
\threeverse{\arabic{verse}}%Ex.22:11
{וְאִם־גָּנֹ֥ב יִגָּנֵ֖ב מֵעִמּ֑וֹ יְשַׁלֵּ֖ם לִבְעָלָֽיו׃}
{וְאִם אִתְגְּנָבָא יִתְגְּנֵיב מֵעִמֵּיהּ יְשַׁלֵּים לְמָרוֹהִי׃}
{But if it be stolen from him, he shall make restitution unto the owner thereof.}{\arabic{verse}}
\threeverse{\arabic{verse}}%Ex.22:12
{אִם־טָרֹ֥ף יִטָּרֵ֖ף יְבִאֵ֣הוּ עֵ֑ד הַטְּרֵפָ֖ה לֹ֥א יְשַׁלֵּֽם׃ \petucha 
\rashi{\rashiDH{אם טרף יטרף. }על ידי חיה רעה׃ 
}\rashi{\rashiDH{יבאהו עד. }יביא עדים שנטרפה באונס ופטור׃}\rashi{\rashiDH{הטרפה לא ישלם. }אינו אומר טרפה לא ישלם, אלא הטרפה, יש טרפה שהוא משלם ויש טרפה שאינו משלם, טרפת חתול ושועל ונמיה משלם, טרפת זאב ארי ודוב ונחש אינו משלם, ומי לחשך לדון כן, שהרי כתיב ומת או נשבר או נשבה, מה מיתה שאין יכול להציל, אף שבר ושביה שאין יכול להציל׃ }}
{אִם אִתְּבָרָא יִתְּבַר יַיְתֵי סָהֲדִין דִּתְבִיר לָא יְשַׁלֵּים׃}
{If it be torn in pieces, let him bring it for witness; he shall not make good that which was torn.}{\arabic{verse}}
\threeverse{\arabic{verse}}%Ex.22:13
{וְכִֽי־יִשְׁאַ֥ל אִ֛ישׁ מֵעִ֥ם רֵעֵ֖הוּ וְנִשְׁבַּ֣ר אוֹ־מֵ֑ת בְּעָלָ֥יו אֵין־עִמּ֖וֹ שַׁלֵּ֥ם יְשַׁלֵּֽם׃
\rashi{\rashiDH{וכי ישאל. }בא ללמדך על השואל שחייב באונסין׃ 
}\rashi{\rashiDH{בעליו אין עמו. }אם בעליו של שור אינו עם השואל במלאכתו (בבא מציעא שם)׃}}
{וַאֲרֵי יִשְׁאַל גְּבַר מִן חַבְרֵיהּ וְיִתְּבַר אוֹ יְמוּת מָרֵיהּ לֵית עִמֵּיהּ שַׁלָּמָא יְשַׁלֵּים׃}
{And if a man borrow aught of his neighbour, and it be hurt, or die, the owner thereof not being with it, he shall surely make restitution.}{\arabic{verse}}
\threeverse{\arabic{verse}}%Ex.22:14
{אִם־בְּעָלָ֥יו עִמּ֖וֹ לֹ֣א יְשַׁלֵּ֑ם אִם־שָׂכִ֣יר ה֔וּא בָּ֖א בִּשְׂכָרֽוֹ׃ \setuma         
\rashi{\rashiDH{אם בעליו עמו. }בין שהוא באותה מלאכה בין שהוא במלאכה אחרת, היה עמו בשעת שאלה, אינו צריך להיות עמו בשעת שבורה ומתה (שם צה׃)׃ }\rashi{\rashiDH{אם שכיר הוא. }אם השור אינו שאול אלא שכור, בא בשכרו ליד השוכר הזה ולא בשאלה, ואין כל הנאה שלו, שהרי על ידי שכרו נשתמש, ואין לו משפט שואל להתחייב באונסין. ולא פירש מה דינו אם כשומר חנם או כשומר שכר, לפיכך נחלקו בו חכמי ישראל, שוכר כיצד משלם, רבי מאיר אומר כשומר חנם, רבי יהודה אומר כשומר שכר׃ }}
{אִם מָרֵיהּ עִמֵּיהּ לָא יְשַׁלֵּים אִם אֲגִירָא הוּא עָאל בְּאַגְרֵיהּ׃}
{If the owner thereof be with it, he shall not make it good; if it be a hireling, he loseth his hire.}{\arabic{verse}}
\threeverse{\arabic{verse}}%Ex.22:15
{וְכִֽי־יְפַתֶּ֣ה אִ֗ישׁ בְּתוּלָ֛ה אֲשֶׁ֥ר לֹא־אֹרָ֖שָׂה וְשָׁכַ֣ב עִמָּ֑הּ מָהֹ֛ר יִמְהָרֶ֥נָּה לּ֖וֹ לְאִשָּֽׁה׃
\rashi{\rashiDH{וכי יפתה. }מדבר על לבה עד ששומעת לו, וכן תרגומו וארי ישדל. שדול בלשון ארמי כפתוי בלשון עברי׃ }\rashi{\rashiDH{מהר ימהרנה. }יפסוק לה מוהר כמשפט איש לאשתו, שכותב לה כתובה וישאנה׃ }}
{וַאֲרֵי יְשַׁדֵּיל גְּבַר בְּתוּלְתָא דְּלָא מְאָרְסָא וְיִשְׁכּוֹב עִמַּהּ קַיָּימָא יְקַיְּימִנַּהּ לֵיהּ לְאִתּוּ׃}
{And if a man entice a virgin that is not betrothed, and lie with her, he shall surely pay a dowry for her to be his wife.}{\arabic{verse}}
\threeverse{\arabic{verse}}%Ex.22:16
{אִם־מָאֵ֧ן יְמָאֵ֛ן אָבִ֖יהָ לְתִתָּ֣הּ ל֑וֹ כֶּ֣סֶף יִשְׁקֹ֔ל כְּמֹ֖הַר הַבְּתוּלֹֽת׃ \setuma         
\rashi{\rashiDH{כמהר הבתולות. }שהוא קצוב חמשים כסף אצל התופס את הבתולה ושוכב עמה באונס, שנאמר וְנָתַן הָאִיש הַשֹׁכֵב עִמָּהּ לַאֲבִי הַנַעֲרָה חֲמִשִׁים כָּסֶף (דברים כב, כט)׃ }}
{אִם מִצְבָּא לָא יִצְבֵּי אֲבוּהָא לְמִתְּנַהּ לֵיהּ כַּסְפָּא יִתְקוּל כְּמוּהְרֵי בְּתוּלָתָא׃}
{If her father utterly refuse to give her unto him, he shall pay money according to the dowry of virgins.}{\arabic{verse}}
\threeverse{\arabic{verse}}%Ex.22:17
{מְכַשֵּׁפָ֖ה לֹ֥א תְחַיֶּֽה׃
\rashi{\rashiDH{מכשפה לא תחיה. }אלא תומת בבית דין, ואחד זכרים ואחד נקבות, אלא שדבר הכתוב בהווה, שהנשים מצויות מכשפות (סנהדרין סז׃)׃ }}
{חָרָשָׁא לָא תַחֵי׃}
{Thou shalt not suffer a sorceress to live.}{\arabic{verse}}
\threeverse{\arabic{verse}}%Ex.22:18
{כׇּל־שֹׁכֵ֥ב עִם־בְּהֵמָ֖ה מ֥וֹת יוּמָֽת׃ \setuma         
\rashi{(ס״א \rashiDH{כל שוכב עם בהמה מות יומת. }בסקילה, רובע כנרבעת, שכתוב בהן דמיהם בם׃) }}
{כָּל דְּיִשְׁכּוֹב עִם בְּעִירָא אִתְקְטָלָא יִתְקְטִיל׃}
{Whosoever lieth with a beast shall surely be put to death.}{\arabic{verse}}
\threeverse{\arabic{verse}}%Ex.22:19
{זֹבֵ֥חַ לָאֱלֹהִ֖ים יׇֽחֳרָ֑ם בִּלְתִּ֥י לַיהֹוָ֖ה לְבַדּֽוֹ׃
\rashi{\rashiDH{לאלהים. }לעבודת גילולים. אילו היה נקוד לאלהים (הלמ״ד בציר״י), היה צריך לפרש ולכתוב אחרים, עכשיו שאמר לאלהים, אין צריך לפרש אחרים, שכל למ״ד ובי״ת וה״א המשמשות בראש התיבה, אם נקודה בחטף, כגון למלך, למדבר, לעיר, צריך לפרש לאיזה מלך, לאיזה מדבר, לאיזה עיר, וכן למלכים, ולרגלים, בחיר״ק, צריך לפרש לאיזה, ואם אינו מפרש, כל מלכים במשמע, וכן לאלהים כל אלהים במשמע, אפילו קודש, אבל כשהיא נקודה פתח, כמו למלך, למדבר, לעיר, (פת״ח וקמ״ץ ענין אחד בענין זה, וגם יש לומר בדרך אחר כמ״ש בדקדוקי רש״י יעויין בו) נודע באיזה מלך מדבר, וכן לעיר נודע באיזה עיר מדבר, וכן לאלהים לאותן שהוזהרתם עליהם במקום אחר. כיוצא בו אֵין כָּמֹוךָ בָּאֱלֹהִים (תהלים פו, ח), לפי שלא פירש, הוצרך לינקד פת״ח׃ }\rashi{\rashiDH{יחרם. }יומת. ולמה נאמר יחרם, והלא כבר נאמרה בו מיתה במקום אחר וְהֹוצֵאתָ אֶת הָאִישׁ הַהוּא אֹו אֶת הָאִשָּׁה הַהִיא וגו׳ (דברים יז, ה), אלא לפי שלא פירש על איזו עבודה חייב מיתה, שלא תאמר כל עבודות במיתה, בא ופירש לך כאן זובח לאלהים יחרם, לומר לך, מה זביחה עבודה הנעשית בפנים לשמים, אף אני מרבה המקטיר והמנסך שהם עבודות בפנים, וחייבין עליהם לכל עבודת אלילים, בין שדרכה לעבדה בכך בין שאין דרכה לעבדה בכך, אבל שאר עבודות, כגון המכבד והמרבץ והמגפף והמנשק, אינו במיתה אלא באזהרה׃ }}
{דִּידַבַּח לְטָעֲוָת עַמְמַיָּא יִתְקְטִיל אֱלָהֵין לִשְׁמָא דַּייָ בִּלְחוֹדוֹהִי׃}
{He that sacrificeth unto the gods, save unto the \lord\space only, shall be utterly destroyed.}{\arabic{verse}}
\threeverse{\arabic{verse}}%Ex.22:20
{וְגֵ֥ר לֹא־תוֹנֶ֖ה וְלֹ֣א תִלְחָצֶ֑נּוּ כִּֽי־גֵרִ֥ים הֱיִיתֶ֖ם בְּאֶ֥רֶץ מִצְרָֽיִם׃
\rashi{\rashiDH{וגר לא תונה. }אונאת דברים, קונטרליאר״ר בלע״ז (העהנען) כמו וְהַאֲכַלְתִּי אֶת מֹונַיִךְ אֶת בְּשָׂרם (ישעיה מט, כו)׃ }\rashi{\rashiDH{ולא תלחצנו. }בגזילת ממון׃}\rashi{\rashiDH{כי גרים הייתם. }אם הוניתו, אף הוא יכול להונותך, ולומר לך אף אתה מגרים באת, מום שבך אל תאמר לחברך. כל לשון גר, אדם שלא נולד באותה מדינה, אלא בא ממדינה אחרת לגור שם׃ }}
{וּלְגִיּוֹרָא לָא תּוֹנוֹן וְלָא תָעִיקוּן אֲרֵי דַּיָּירִין הֲוֵיתוֹן בְּאַרְעָא דְּמִצְרָיִם׃}
{And a stranger shalt thou not wrong, neither shalt thou oppress him; for ye were strangers in the land of Egypt.}{\arabic{verse}}
\threeverse{\arabic{verse}}%Ex.22:21
{כׇּל־אַלְמָנָ֥ה וְיָת֖וֹם לֹ֥א תְעַנּֽוּן׃
\rashi{\rashiDH{כל אלמנה ויתום לא תענון. }הוא הדין לכל אדם, אלא שדבר הכתוב בהווה, לפי שהם תשושי כח ודבר מצוי לענותם׃ }}
{כָּל אַרְמְלָא וְיִיתַם לָא תְעַנּוֹן׃}
{Ye shall not afflict any widow, or fatherless child.}{\arabic{verse}}
\threeverse{\arabic{verse}}%Ex.22:22
{אִם־עַנֵּ֥ה תְעַנֶּ֖ה אֹת֑וֹ כִּ֣י אִם־צָעֹ֤ק יִצְעַק֙ אֵלַ֔י שָׁמֹ֥עַ אֶשְׁמַ֖ע צַעֲקָתֽוֹ׃
\rashi{\rashiDH{אם ענה תענה אתו. }הרי זה מקרא קצר, גזם ולא פירש ענשו, (והא דכתיב והיו נשיכם וגו׳, זהו אם צעוק יצעק, אבל באם לא יצעק לא פירש, וק״ל) כמו לָכֵן כָּל הֹרֵג קַיִן (בראשית ד, טו), גזם ולא פירש ענשו, אף כאן אם ענה תענה אותו, לשון גזום, כלומר סופך ליטול את שלך, למה, כי אם צעק יצעק אלי וגו׳׃ }}
{אִם עַנָּאָה תְעַנֵּי יָתֵיהּ אֲרֵי אִם מִקְבָּל יִקְבַּל קֳדָמַי קַבָּלָא אֲקַבֵּיל קְבִילְתֵיהּ׃}
{If thou afflict them in any wise—for if they cry at all unto Me, I will surely hear their cry—}{\arabic{verse}}
\threeverse{\arabic{verse}}%Ex.22:23
{וְחָרָ֣ה אַפִּ֔י וְהָרַגְתִּ֥י אֶתְכֶ֖ם בֶּחָ֑רֶב וְהָי֤וּ נְשֵׁיכֶם֙ אַלְמָנ֔וֹת וּבְנֵיכֶ֖ם יְתֹמִֽים׃ \petucha 
\rashi{\rashiDH{והיו נשיכם אלמנות. }ממשמע שנאמר והרגתי אתכם, איני יודע שנשיכם אלמנות ובניכם יתומים, אלא הרי זו קללה אחרת, שיהיו הנשים צרורות כאלמנות חיות, שלא יהיו עדים למיתת בעליהן ותהיינה אסורות להנשא, והבנים יהיו יתומים, שלא יניחום בית דין לירד לנכסי אביהם, לפי שאין יודעים אם מתו אם נשבו׃ }}
{וְיִתְקַף רוּגְזִי וְאֶקְטוֹל יָתְכוֹן בְּחַרְבָּא וְיִהְוְיָן נְשֵׁיכוֹן אַרְמְלָן וּבְנֵיכוֹן יַתְמִין׃}
{My wrath shall wax hot, and I will kill you with the sword; and your wives shall be widows, and your children fatherless.}{\arabic{verse}}
\threeverse{\arabic{verse}}%Ex.22:24
{אִם־כֶּ֣סֶף \legarmeh  תַּלְוֶ֣ה אֶת־עַמִּ֗י אֶת־הֶֽעָנִי֙ עִמָּ֔ךְ לֹא־תִהְיֶ֥ה ל֖וֹ כְּנֹשֶׁ֑ה לֹֽא־תְשִׂימ֥וּן עָלָ֖יו נֶֽשֶׁךְ׃
\rashi{\rashiDH{אם כסף תלוה את עמי. }רבי ישמעאל אומר, כל אם ואם שבתורה רשות, חוץ מג׳, וזה אחד מהן׃ }\rashi{\rashiDH{את עמי. }עמי ונכרי, עמי קודם. עני ועשיר, עני קודם. עניי עירך ועניי עיר אחרת, עניי עירך קודמין (בבא מציעא עא.). וזה משמעו, אם כסף תלוה, את עמי תלוהו קודם לעובד גילולים, ולאיזה מעמי, את העני, ולאיזה עני, לאותו שעמך. (ד״א את העני, שלא תנהוג בו מנהג בזיון בהלואה שהוא עמי. }\rashi{\rashiDH{את העני עמך, }הוי מסתכל בעצמך כאילו אתה עני)׃ }\rashi{\rashiDH{לא תהיה לו כנשה. }לא תתבענו בחזקה, אם אתה יודע שאין לו, אל תהי דומה עליו כאילו הלויתו, אלא כאילו לא הלויתו, כלומר, לא תכלימהו׃ }\rashi{\rashiDH{נשך. }רבית, שהוא כנשיכת נחש, שנחש נושך חבורה קטנה ברגלו ואינו מרגיש, ופתאום הוא מבצבץ ונופח עד קדקדו, כך רבית, אינו מרגיש ואינו ניכר עד שהרבית עולה ומחסרו ממון הרבה׃ }}
{אִם כַּסְפָּא תוֹזֵיף בְּעַמִּי לְעַנְיָא דְּעִמָּךְ לָא תְהֵי לֵיהּ כְּרָשְׁיָא לָא תְשַׁוּוֹן עֲלוֹהִי חִיבוּלְיָא׃}
{If thou lend money to any of My people, even to the poor with thee, thou shalt not be to him as a creditor; neither shall ye lay upon him interest.}{\arabic{verse}}
\threeverse{\arabic{verse}}%Ex.22:25
{אִם־חָבֹ֥ל תַּחְבֹּ֖ל שַׂלְמַ֣ת רֵעֶ֑ךָ עַד־בֹּ֥א הַשֶּׁ֖מֶשׁ תְּשִׁיבֶ֥נּוּ לֽוֹ׃
\rashi{\rashiDH{אם חבול תחבל. }כל לשון חבלה אינו משכון בשעת הלואה, אלא שממשכנין את הלוה כשמגיע הזמן ואינו פורע. (חבול תחבול כפל לך בחבלה עד כמה פעמים, אמר הקב״ה, כמה אתה חייב לי, והרי נפשך עולה אצלי כל אמש ואמש ונותנת דין וחשבון ומתחייבת לפני, ואני מחזירה לך, אף אתה טול והשב טול והשב)׃ }\rashi{\rashiDH{עד בא השמש תשיבנו לו. }כל היום תשיבנו לו עד בא השמש, וכבוא השמש תחזור ותטלנו עד שיבא בקר של מחר, ובכסות יום הכתוב מדבר שאין צריך לה בלילה (מכילתא פי״ט)׃ 
}}
{אִם מִשְׁכּוֹנָא תִסַּב כְּסוּתָא דְּחַבְרָךְ עַד מֵיעַל שִׁמְשָׁא תָּתִיבִנֵּיהּ לֵיהּ׃}
{If thou at all take thy neighbour’s garment to pledge, thou shalt restore it unto him by that the sun goeth down;}{\arabic{verse}}
\threeverse{\arabic{verse}}%Ex.22:26
{כִּ֣י הִ֤וא כְסוּתֹה֙ לְבַדָּ֔הּ הִ֥וא שִׂמְלָת֖וֹ לְעֹר֑וֹ בַּמֶּ֣ה יִשְׁכָּ֔ב וְהָיָה֙ כִּֽי־יִצְעַ֣ק אֵלַ֔י וְשָׁמַעְתִּ֖י כִּֽי־חַנּ֥וּן אָֽנִי׃ \setuma         
\rashi{\rashiDH{כי הוא כסותה. }זו טלית׃}\rashi{\rashiDH{שמלתו. }זו חלוק׃}\rashi{\rashiDH{במה ישכב. }לרבות את המצע׃}}
{אֲרֵי הִיא כְסוּתֵיהּ בִּלְחוֹדַהּ הִיא תּוּתְבֵּיהּ לְמַשְׁכֵּיהּ בְּמָא יִשְׁכּוֹב וִיהֵי אֲרֵי יִקְבַּל קֳדָמַי וַאֲקַבֵּיל קְבִילְתֵיהּ אֲרֵי חַנָּנָא אֲנָא׃}
{for that is his only covering, it is his garment for his skin; wherein shall he sleep? and it shall come to pass, when he crieth unto Me, that I will hear; for I am gracious.}{\arabic{verse}}
\threeverse{\aliya{רביעי}}%Ex.22:27
{אֱלֹהִ֖ים לֹ֣א תְקַלֵּ֑ל וְנָשִׂ֥יא בְעַמְּךָ֖ לֹ֥א תָאֹֽר׃
\rashi{\rashiDH{אלהים לא תקלל. }הרי זו אזהרה לברכת השם, ואזהרה לקללת דיין (סנהדרין סו.)׃ 
}}
{דַּיָּינָא לָא תַקִיל וְרַבָּא בְּעַמָּךְ לָא תְלוּט׃}
{Thou shalt not revile God, nor curse a ruler of thy people.}{\arabic{verse}}
\threeverse{\arabic{verse}}%Ex.22:28
{מְלֵאָתְךָ֥ וְדִמְעֲךָ֖ לֹ֣א תְאַחֵ֑ר בְּכ֥וֹר בָּנֶ֖יךָ תִּתֶּן־לִֽי׃
\rashi{\rashiDH{מלאתך. }חובה המוטלת עליך כשתתמלא תבואתך להתבשל, והם בכורים׃ }\rashi{\rashiDH{ודמעך. }התרומה, ואיני יודע מהו לשון דמע׃ }\rashi{\rashiDH{לא תאחר. }לא תשנה סדר הפרשתן, לאחר את המוקדם ולהקדים את המאוחר, שלא יקדים תרומה לבכורים, ומעשר לתרומה׃ }\rashi{\rashiDH{בכור בניך תתן לי. }לפדותו בחמש סלעים מן הכהן, והלא כבר צוה עליו במקום אחר, אלא כדי לסמוך לו כן תעשה לשורך, מה בכור אדם לאחר ל׳ יום פודהו, שנאמר וּפְדוּיָיו מִבֶּן חֹדֶשׁ תִּפְדֶּה (במדבר יח, טז), אף בכור בהמה דקה מטפל בו ל׳ יום, ואחר כך נותנו לכהן׃ }}
{בִּכּוּרָךְ וְדִמְעָךְ לָא תְאַחַר בּוּכְרָא דִּבְנָךְ תַּפְרֵישׁ קֳדָמָי׃}
{Thou shalt not delay to offer of the fulness of thy harvest, and of the outflow of thy presses. The first-born of thy sons shalt thou give unto Me.}{\arabic{verse}}
\threeverse{\arabic{verse}}%Ex.22:29
{כֵּֽן־תַּעֲשֶׂ֥ה לְשֹׁרְךָ֖ לְצֹאנֶ֑ךָ שִׁבְעַ֤ת יָמִים֙ יִהְיֶ֣ה עִם־אִמּ֔וֹ בַּיּ֥וֹם הַשְּׁמִינִ֖י תִּתְּנוֹ־לִֽי׃
\rashi{\rashiDH{שבעת ימים יהיה עם אמו. }זו אזהרה לכהן, שאם בא למהר את קרבנו, לא ימהר קודם שמונה, לפי שהוא מחוסר זמן׃ }\rashi{\rashiDH{ביום השמיני תתנו לי. }יכול יהא חובה לבו ביום, נאמר כאן שמיני ונאמר להלן וּמִיֹום הַשְּׁמִינִי וָהָלְאָה יֵרָצֶה (ויקרא כב, כז), מה שמיני האמור להלן להכשיר משמיני ולהלן, אף שמיני האמור כאן להכשיר משמיני ולהלן (מכילתא פי״ט), וכן משמעו, וביום השמיני אתה רשאי ליתנו לי׃ }}
{כֵּן תַּעֲבֵיד לְתוֹרָךְ לְעָנָךְ שִׁבְעָא יוֹמִין יְהֵי עִם אִמֵּיהּ בְּיוֹמָא תְּמִינָאָה תַּפְרְשִׁנֵּיהּ קֳדָמָי׃}
{Likewise shalt thou do with thine oxen, and with thy sheep; seven days it shall be with its dam; on the eighth day thou shalt give it Me.}{\arabic{verse}}
\threeverse{\arabic{verse}}%Ex.22:30
{וְאַנְשֵׁי־קֹ֖דֶשׁ תִּהְי֣וּן לִ֑י וּבָשָׂ֨ר בַּשָּׂדֶ֤ה טְרֵפָה֙ לֹ֣א תֹאכֵ֔לוּ לַכֶּ֖לֶב תַּשְׁלִכ֥וּן אֹתֽוֹ׃ \setuma         
\rashi{\rashiDH{ואנשי קודש תהיון לי. }אם אתם קדושים ופרושים משקוצי נבלות וטרפות, הרי אתם שלי, ואם לאו אינכם שלי׃ 
}\rashi{\rashiDH{ובשר בשדה טרפה. }אף בבית כן, אלא שדבר הכתוב בהווה (מכילתא פ״כ), מקום שדרך בהמות ליטרף, וכן כִּי בַשָׂדֶה מְצָאָהּ (דברים כב, כז), וכן אֲשֶׁר לֹא יִהְיֶה טָהֹור מִקְּרֵה לָיְלָה (שם כג, יא), הוא הדין למקרה יום, אלא שדבר הכתוב בהווה. (ואונקלוס תרגם) ובשר דתליש מן חיוא חיתא, בשר שנתלש על ידי טרפת זאב או ארי (או) מן חיה כשרה או מבהמה כשרה בחייה׃ }\rashi{\rashiDH{לכלב תשליכון אתו. }אף הוא כו׳ או אינו אלא כלב כמשמעו בנבלה או מכור לנכרי קל וחומר לטרפה שמותרת בכל הנאות א״כ מה תלוד לומר לכלב למדך הכתוב שאין הקב״ה מקפח שכר כל בריה, שנאמר וּלְכֹל בְּנֵי יִשִׂרָאֵל לֹא יֶחֱרַץ כֶּלֶב לְשֹׁנֹו (שמות יא, ז), אמר הקב״ה תנו לו שכרו (מכילתא פ״כ)׃ }}
{וַאֲנָשִׁין קַדִּישִׁין תְּהוֹן קֳדָמָי וּבְשַׁר תְּלִישׁ מִן חֵיוָא חַיָא לָא תֵיכְלוּן לְכַלְבָּא תִּרְמוֹן יָתֵיהּ׃}
{And ye shall be holy men unto Me; therefore ye shall not eat any flesh that is torn of beasts in the field; ye shall cast it to the dogs.}{\arabic{verse}}
\newperek
\threeverse{\Roman{chap}}%Ex.23:1
{לֹ֥א תִשָּׂ֖א שֵׁ֣מַע שָׁ֑וְא אַל־תָּ֤שֶׁת יָֽדְךָ֙ עִם־רָשָׁ֔ע לִהְיֹ֖ת עֵ֥ד חָמָֽס׃
\rashi{\rashiDH{לא תשא שמע שוא. }כתרגומו לא תקבל שמע דשקר, אזהרה למקבל לשון הרע, ולדיין שלא ישמע דברי בעל דין עד שיבא בעל דין חבירו׃ 
}\rashi{\rashiDH{אל תשת ידך עם רשע. }הטוען את חבירו תביעת שקר, שהבטיחהו להיות לו עד חמס׃ }}
{לָא תְקַבֵּיל שֵׁימַע דִּשְׁקַר לָא תְשַׁוֵּי יְדָךְ עִם חַיָּיבָא לְמִהְוֵי לֵיהּ סָהִיד שַׁקָּר׃}
{Thou shalt not utter a false report; put not thy hand with the wicked to be an unrighteous witness.}{\Roman{chap}}
\threeverse{\arabic{verse}}%Ex.23:2
{לֹֽא־תִהְיֶ֥ה אַחֲרֵֽי־רַבִּ֖ים לְרָעֹ֑ת וְלֹא־תַעֲנֶ֣ה עַל־רִ֗ב לִנְטֹ֛ת אַחֲרֵ֥י רַבִּ֖ים לְהַטֹּֽת׃
\rashi{\rashiDH{לא תהיה אחרי רבים לרעות. }יש במקרא זה מדרשי חכמי ישראל, אבל אין לשון המקרא מיושב בהן על אופניו. מכאן דרשו שאין מטין לחובה בהכרעת דיין אחד (סנהדרין ב.), וסוף המקרא דרשו, אחרי רבים להטות, שאם יש שנים מחייבין יותר על המזכין, הטה הדין על פיהם לחובה, ובדיני נפשות הכתוב מדבר. ואמצע המקרא דרשו, ולא תענה על ריב, על רב, שאין חולקין על מופלא שבבית דין, לפיכך מתחילין בדיני נפשות מן הצד, לקטנים שבהן שואלין תחלה שיאמרו את דעתם. ולפי דברי רבותינו כך פתרון המקרא׃ \rashiDH{לא תהיה אחרי רבים לרעת. }לחייב מיתה בשביל דיין אחד שירבו מחייבין על המזכין. \rashiDH{ולא תענה על רב. }לנטות מדבריו, ולפי שהוא חסר יו״ד דרשו בו כן. \rashiDH{אחרי רבים להטת. }ויש רבים שאתה נוטה אחריהם, ואימתי, בזמן שהן שנים המכריעין במחייבין יותר מן המזכין. וממשמע שנאמר לא תהיה אחרי רבים לרעות, שומע אני אבל היה עמהם לטובה, מכאן אמרו דיני נפשות מטין על פי אחד לזכות ועל פי שנים לחובה. ואונקלוס תרגם לא תתמנע מלאלפא מה דבעינך על דינא, ולשון העברי לפי התרגום כך הוא נדרש׃ \rashiDH{לא תענה על ריב לנטות}, אם ישאלך דבר למשפט, לא תענה לנטות לצד אחד ולסלק עצמך מן הריב, אלא הוי דן אותו לאמיתו. }\rashi{ואני אומר לישבו על אופניו כפשוטו וכך פתרונו. \rashiDH{לא תהיה אחרי רבים לרעות. }אם ראית רשעים מטין משפט, לא תאמר הואיל ורבים הם הנני נוטה אחריהם׃}\rashi{\rashiDH{ולא תענה על ריב לנטות וגו׳. }ואם ישאלך הנדון על אותו המשפט, אל תעננו על הריב דבר הנוטה אחרי אותן רבים להטות את המשפט מאמיתו, אלא אמור את המשפט כאשר הוא, וקולר יהא תלוי בצואר הרבים׃ }}
{לָא תְהֵי בָּתַר סַגִּיאֵי לְאַבְאָשָׁא וְלָא תִתְמְנַע מִלְּאַלָּפָא מָא דִּבְעֵינָךְ עַל דִּינָא בָּתַר סַגִּיאֵי שַׁלֵּים דִּינָא׃}
{Thou shalt not follow a multitude to do evil; neither shalt thou bear witness in a cause to turn aside after a multitude to pervert justice;}{\arabic{verse}}
\threeverse{\arabic{verse}}%Ex.23:3
{וְדָ֕ל לֹ֥א תֶהְדַּ֖ר בְּרִיבֽוֹ׃ \setuma         
\rashi{\rashiDH{לא תהדר. }לא תחלוק לו כבוד לזכותו בדין ולומר, דל הוא אזכנו ואכבדנו׃ }}
{וְעַל מִסְכֵּינָא לָא תְרַחֵים בְּדִינֵיהּ׃}
{neither shalt thou favour a poor man in his cause.}{\arabic{verse}}
\threeverse{\arabic{verse}}%Ex.23:4
{כִּ֣י תִפְגַּ֞ע שׁ֧וֹר אֹֽיִבְךָ֛ א֥וֹ חֲמֹר֖וֹ תֹּעֶ֑ה הָשֵׁ֥ב תְּשִׁיבֶ֖נּוּ לֽוֹ׃ \setuma         }
{אֲרֵי תִפְגַּע תּוֹרָא דְּשָׁנְאָךְ אוֹ חֲמָרֵיהּ דְּטָעֵי אָתָבָא תָּתִיבִנֵּיהּ לֵיהּ׃}
{If thou meet thine enemy’s ox or his ass going astray, thou shalt surely bring it back to him again.}{\arabic{verse}}
\threeverse{\arabic{verse}}%Ex.23:5
{כִּֽי־תִרְאֶ֞ה חֲמ֣וֹר שֹׂנַאֲךָ֗ רֹבֵץ֙ תַּ֣חַת מַשָּׂא֔וֹ וְחָדַלְתָּ֖ מֵעֲזֹ֣ב ל֑וֹ עָזֹ֥ב תַּעֲזֹ֖ב עִמּֽוֹ׃ \setuma         
\rashi{\rashiDH{כי תראה חמור שונאך וגו׳. }הרי כי משמש בלשון דלמא, שהוא מד׳ לשונות של שמושי כי, וכה פתרונו, שמא תראה חמורו רובץ תחת משאו׃ \rashiDH{וחדלת מעזוב לו.} בתמיה׃}\rashi{\rashiDH{עזב תעזב עמו. }עזיבה זו לשון עזרה, וכן עָצוּר וְעָזוּב (מלכים־א יד, י), וכן וַיַּעַזְבוּ יְרוּשָׁלַיִם עַד הַחֹומָה (נחמיה ג, ח), מלאוה עפר לעזוב ולסייע את חוזק החומה. כיוצא בו, כִּי תֹאמַר בִּלְבָבְךָ רַבִּים הַגּ ֹויִם הָאֵלֶּה מִמֶּנִּי וגו׳ (דברים ז, יז), שמא תאמר כן, בתמיה, לֹא תִירָא מֵהֶם. ומדרשו כך דרשו רבותינו, כי תראה וחדלת, פעמים שאתה חודל ופעמים שאתה עוזר, הא כיצד, זקן ואינו לפי כבודו, וחדלת, או בהמת עובד כוכבים ומשאו של ישראל (בבא מציעא לב׃), וחדלת׃ }\rashi{\rashiDH{עזב תעזב עמו. }לפרק המשא, מלמשקל ליה, מליטול משאוי ממנו׃ }}
{אֲרֵי תִחְזֵי חֲמָרָא דְּסָנְאָךְ רְבִיעַ תְּחוֹת טוּעְנֵיהּ וְתִתְמְנַע מִלְּמִשְׁקַל לֵיהּ מִשְׁבָּק תִּשְׁבּוֹק מָא דִּבְלִבָּךְ עֲלוֹהִי וּתְפָרֵיק עִמֵּיהּ׃}
{If thou see the ass of him that hateth thee lying under its burden, thou shalt forbear to pass by him; thou shalt surely release it with him.}{\arabic{verse}}
\threeverse{\aliya{חמישי}}%Ex.23:6
{לֹ֥א תַטֶּ֛ה מִשְׁפַּ֥ט אֶבְיֹנְךָ֖ בְּרִיבֽוֹ׃
\rashi{\rashiDH{אבינך. }לשון אובה, שהוא מדולדל ותאב לכל טובה (שם קיא׃)׃ }}
{לָא תַצְלֵי דִּין מִסְכֵּינָךְ בְּדִינֵיהּ׃}
{Thou shalt not wrest the judgment of thy poor in his cause.}{\arabic{verse}}
\threeverse{\arabic{verse}}%Ex.23:7
{מִדְּבַר־שֶׁ֖קֶר תִּרְחָ֑ק וְנָקִ֤י וְצַדִּיק֙ אַֽל־תַּהֲרֹ֔ג כִּ֥י לֹא־אַצְדִּ֖יק רָשָֽׁע׃
\rashi{\rashiDH{ונקי וצדיק אל תהרג. }מנין ליוצא מבית דין חייב, ואמר אחד יש לי ללמד עליו זכות, שמחזירין אותו, תלמוד לומר ונקי אל תהרוג, ואף על פי שאינו צדיק, שלא נצטדק בבית דין, מכל מקום נקי הוא מדין מיתה, שהרי יש לך לזכותו. ומנין ליוצא מבית דין זכאי, ואמר אחד יש לי ללמד עליו חובה שאין מחזירין אותו לבית דין, תלמוד לומר וצדיק אל תהרג, וזה צדיק הוא, שנצטדק בבית דין׃ }\rashi{\rashiDH{כי לא אצדיק רשע. }אין עליך להחזירו, כי אני לא אצדיקנו בדינו אם יצא מידך זכאי, יש לי שלוחים הרבה להמיתו במיתה שנתחייב בה׃ 
}}
{מִפִּתְגָמָא דְּשִׁקְרָא הֱוִי רַחִיק וְדִזְכֵּי וְדִנְפַק דְּכֵי מִן דִּינָא לָא תִקְטוּל אֲרֵי לָא אֲזַכֵּי חַיָּיבָא׃}
{Keep thee far from a false matter; and the innocent and righteous slay thou not; for I will not justify the wicked.}{\arabic{verse}}
\threeverse{\arabic{verse}}%Ex.23:8
{וְשֹׁ֖חַד לֹ֣א תִקָּ֑ח כִּ֤י הַשֹּׁ֙חַד֙ יְעַוֵּ֣ר פִּקְחִ֔ים וִֽיסַלֵּ֖ף דִּבְרֵ֥י צַדִּיקִֽים׃
\rashi{\rashiDH{ושחד לא תקח. }אפילו לשפוט אמת, וכל שכן כדי להטות את הדין, שהרי כדי להטות את הדין נאמר כבר לא תטה משפט׃ }\rashi{\rashiDH{יעור פקחים. }ואפילו חכם בתורה ונוטל שוחד, סוף שתטרף דעתו עליו, וישתכח תלמודו, ויכהה מאור עיניו (מכילתא פ״כ)׃ }\rashi{\rashiDH{ויסלף. }כתרגומו ומקלקל׃}\rashi{\rashiDH{דברי צדיקים. }דברים המצודקים, משפטי אמת, וכן תרגומו פתגמין תריצין, ישרים׃ }}
{וְשׁוּחְדָּא לָא תְקַבֵּיל אֲרֵי שׁוּחְדָּא מְעַוַּר עֵינֵי חַכִּימִין וּמְקַלְקֵיל פִּתְגָמִין תְּרִיצִין׃}
{And thou shalt take no gift; for a gift blindeth them that have sight, and perverteth the words of the righteous.}{\arabic{verse}}
\threeverse{\arabic{verse}}%Ex.23:9
{וְגֵ֖ר לֹ֣א תִלְחָ֑ץ וְאַתֶּ֗ם יְדַעְתֶּם֙ אֶת־נֶ֣פֶשׁ הַגֵּ֔ר כִּֽי־גֵרִ֥ים הֱיִיתֶ֖ם בְּאֶ֥רֶץ מִצְרָֽיִם׃
\rashi{\rashiDH{וגר לא תלחץ. }בהרבה מקומות הזהירה תורה על הגר, מפני שסורו רע (בבא מציעא נט׃)׃ }\rashi{\rashiDH{את נפש הגר. }כמה קשה לו כשלוחצים אותו׃ 
}}
{וּלְגִיּוֹרָא לָא תָעִיקוּן וְאַתּוּן יְדַעְתּוּן יָת נַפְשָׁא דְּגִיּוֹרָא אֲרֵי דַּיָּירִין הֲוֵיתוֹן בְּאַרְעָא דְּמִצְרָיִם׃}
{And a stranger shalt thou not oppress; for ye know the heart of a stranger, seeing ye were strangers in the land of Egypt.}{\arabic{verse}}
\threeverse{\arabic{verse}}%Ex.23:10
{וְשֵׁ֥שׁ שָׁנִ֖ים תִּזְרַ֣ע אֶת־אַרְצֶ֑ךָ וְאָסַפְתָּ֖ אֶת־תְּבוּאָתָֽהּ׃
\rashi{\rashiDH{ואספת את תבואתה. }לשון הכנסה לבית, כמו וַאֲסַפְתֹּו אֶל תֹּוךְ בֵּיתֶךָ (דברים כב, ב)׃ }}
{וְשֵׁית שְׁנִין תִּזְרַע יָת אַרְעָךְ וְתִכְנוֹשׁ יָת עֲלַלְתַּהּ׃}
{And six years thou shalt sow thy land, and gather in the increase thereof;}{\arabic{verse}}
\threeverse{\arabic{verse}}%Ex.23:11
{וְהַשְּׁבִיעִ֞ת תִּשְׁמְטֶ֣נָּה וּנְטַשְׁתָּ֗הּ וְאָֽכְלוּ֙ אֶבְיֹנֵ֣י עַמֶּ֔ךָ וְיִתְרָ֕ם תֹּאכַ֖ל חַיַּ֣ת הַשָּׂדֶ֑ה כֵּֽן־תַּעֲשֶׂ֥ה לְכַרְמְךָ֖ לְזֵיתֶֽךָ׃
\rashi{\rashiDH{תשמטנה. }מעבודה׃}\rashi{\rashiDH{ונטשתה. }מאכילה אחר זמן הביעור. דבר אחר תשמטנה, מעבודה גמורה, כגון חרישה וזריעה. ונטשתה, מלזבל ומלקשקש׃ }\rashi{\rashiDH{ויתרם תאכל חית השדה. }להקיש מאכל אביון למאכל חיה, מה חיה אוכלת בלא מעשר, אף אביונים אוכלים בלא מעשר, מכאן אמרו אין מעשר בשביעית (מכילתא פ״כ)׃ }\rashi{\rashiDH{כן תעשה לכרמך. }ותחלת המקרא מדבר בשדה הלבן, כמו שאמר למעלה הימנו תזרע את ארצך׃ }}
{וּשְׁבִיעֵיתָא תִּשְׁמְטִנַּהּ וְתִרְטְשִׁנַּהּ וְיֵיכְלוּן מִסְכֵּינֵי עַמָּךְ וּשְׁאָרְהוֹן תֵּיכוֹל חַיַּת בָּרָא כֵּן תַּעֲבֵיד לְכַרְמָךְ לְזֵיתָךְ׃}
{but the seventh year thou shalt let it rest and lie fallow, that the poor of thy people may eat; and what they leave the beast of the field shall eat. In like manner thou shalt deal with thy vineyard, and with thy oliveyard.}{\arabic{verse}}
\threeverse{\arabic{verse}}%Ex.23:12
{שֵׁ֤שֶׁת יָמִים֙ תַּעֲשֶׂ֣ה מַעֲשֶׂ֔יךָ וּבַיּ֥וֹם הַשְּׁבִיעִ֖י תִּשְׁבֹּ֑ת לְמַ֣עַן יָנ֗וּחַ שֽׁוֹרְךָ֙ וַחֲמֹרֶ֔ךָ וְיִנָּפֵ֥שׁ בֶּן־אֲמָתְךָ֖ וְהַגֵּֽר׃
\rashi{\rashiDH{וביום השביעי תשבת. }אף בשנה השביעית לא תעקר שבת בראשית ממקומה (שם), שלא תאמר, הואיל וכל השנה קרויה שבת, לא תנהג בה שבת בראשית׃ }\rashi{\rashiDH{למען ינוח שורך וחמורך. }תן לו נייח, להתיר שיהא תולש ואוכל עשבים מן הקרקע, או אינו אלא יחבשנו בתוך הבית, אמרת, אין זה נייח אלא צער׃ }\rashi{\rashiDH{בן אמתך. }בעבד הערל הכתוב מדבר (שם)׃}\rashi{\rashiDH{והגר. }זה גר תושב׃ 
}}
{שִׁתָּא יוֹמִין תַּעֲבֵיד עוּבָדָךְ וּבְיוֹמָא שְׁבִיעָאָה תְּנוּחַ בְּדִיל דִּינוּחַ תּוֹרָךְ וּחְמָרָךְ וְיִשְׁקוֹט בַּר אַמְתָּךְ וְגִיּוֹרָא׃}
{Six days thou shalt do thy work, but on the seventh day thou shalt rest; that thine ox and thine ass may have rest, and the son of thy handmaid, and the stranger, may be refreshed.}{\arabic{verse}}
\threeverse{\arabic{verse}}%Ex.23:13
{וּבְכֹ֛ל אֲשֶׁר־אָמַ֥רְתִּי אֲלֵיכֶ֖ם תִּשָּׁמֵ֑רוּ וְשֵׁ֨ם אֱלֹהִ֤ים אֲחֵרִים֙ לֹ֣א תַזְכִּ֔ירוּ לֹ֥א יִשָּׁמַ֖ע עַל־פִּֽיךָ׃
\rashi{\rashiDH{ובכל אשר אמרתי אליכם תשמרו. }לעשות כל מצות עשה באזהרה, שכל שמירה שבתורה אזהרה היא במקום לאו (יל״ש שנה, בשם המכילתא)׃ }\rashi{\rashiDH{לא תזכירו. }שלא יאמר לו, שמור לי בצד עבודת אלילים פלונית (סנהדרין סג׃), או תעמוד עמי ביום עבודת אלילים פלונית. דבר אחר ובכל אשר אמרתי אליכם תשמרו ושם אלהים אחרים לא תזכירו, ללמדך, ששקולה עבודת אלילים כנגד כל המצות כולם, והנזהר בה כשומר את כולן׃ }\rashi{\rashiDH{לא ישמע. }מן הנכרי׃ \rashiDH{על פיך. }שלא תעשה שותפות עם עובדי כוכבים, וישבע לך בעבודת אלילים שלו, נמצאת שאתה גורם שיזכיר על ידך׃ }}
{וּבְכֹל דַּאֲמַרִית לְכוֹן תִּסְתַּמְרוּן וְשׁוֹם טָעֲוָת עַמְמַיָּא לָא תִדְכְרוּן לָא יִשְׁתְּמַע עַל פּוּמְּכוֹן׃}
{And in all things that I have said unto you take ye heed; and make no mention of the name of other gods, neither let it be heard out of thy mouth. .}{\arabic{verse}}
\threeverse{\arabic{verse}}%Ex.23:14
{שָׁלֹ֣שׁ רְגָלִ֔ים תָּחֹ֥ג לִ֖י בַּשָּׁנָֽה׃
\rashi{\rashiDH{רגלים. }פעמים, וכן כִּי הִכִּיתָנִי זֶה שׁלשׁ רְגָלִים (במדבר כב, כח)׃ 
}}
{תְּלָת זִמְנִין תֵּיחֲגוּן קֳדָמַי בְּשַׁתָּא׃}
{Three times thou shalt keep a feast unto Me in the year.}{\arabic{verse}}
\threeverse{\arabic{verse}}%Ex.23:15
{אֶת־חַ֣ג הַמַּצּוֹת֮ תִּשְׁמֹר֒ שִׁבְעַ֣ת יָמִים֩ תֹּאכַ֨ל מַצּ֜וֹת כַּֽאֲשֶׁ֣ר צִוִּיתִ֗ךָ לְמוֹעֵד֙ חֹ֣דֶשׁ הָֽאָבִ֔יב כִּי־ב֖וֹ יָצָ֣אתָ מִמִּצְרָ֑יִם וְלֹא־יֵרָא֥וּ פָנַ֖י רֵיקָֽם׃
\rashi{\rashiDH{חדש האביב. }שהתבואה מתמלאת בו באביה. אביב לשון אב, בכור וראשון לבשל פירות׃ }\rashi{\rashiDH{ולא יראו פני ריקם. }כשתבאו לראות פני ברגלים, הביאו לי עולות (חגיגה ז.)׃ }}
{יָת חַגָּא דְּפַטִּירַיָּא תִטַּר שִׁבְעָא יוֹמִין תֵּיכוֹל פַּטִּירָא כְּמָא דְּפַקֵּידְתָּךְ לִזְמַן יַרְחָא דַּאֲבִיבָא אֲרֵי בֵיהּ נְפַקְתָּא מִמִּצְרָיִם וְלָא יִתַּחְזוֹן קֳדָמַי רֵיקָנִין׃}
{The feast of unleavened bread shalt thou keep; seven days thou shalt eat unleavened bread, as I commanded thee, at the time appointed in the month Abib—for in it thou camest out from Egypt; and none shall appear before Me empty;}{\arabic{verse}}
\threeverse{\arabic{verse}}%Ex.23:16
{וְחַ֤ג הַקָּצִיר֙ בִּכּוּרֵ֣י מַעֲשֶׂ֔יךָ אֲשֶׁ֥ר תִּזְרַ֖ע בַּשָּׂדֶ֑ה וְחַ֤ג הָֽאָסִף֙ בְּצֵ֣את הַשָּׁנָ֔ה בְּאׇסְפְּךָ֥ אֶֽת־מַעֲשֶׂ֖יךָ מִן־הַשָּׂדֶֽה׃
\rashi{\rashiDH{וחג הקציר. }הוא חג שבועות׃}\rashi{\rashiDH{בכורי מעשיך. }שהוא זמן הבאת בכורים, ששתי הלחם הבאין בעצרת, היו מתירין החדש למנחות ולהביא בכורים למקדש, שנאמר וּבְיֹום הַבִּכּוּרִים וגו׳ (במדבר כח, כו)׃ }\rashi{\rashiDH{וחג האסיף. }הוא חג הסוכות׃}\rashi{\rashiDH{באספך את מעשיך. }שכל ימות החמה התבואה מתייבשת בשדות, ובחג אוספים אותה אל הבית מפני הגשמים׃ }}
{וְחַגָּא דִחְצָדָא בִּכּוּרֵי עוּבָדָךְ דְּתִזְרַע בְּחַקְלָא וְחַגָּא דִּכְנָשָׁא בְּמִפְּקַהּ דְּשַׁתָּא בְּמִכְנְשָׁךְ יָת עוּבָדָךְ מִן חַקְלָא׃}
{and the feast of harvest, the first-fruits of thy labours, which thou sowest in the field; and the feast of ingathering, at the end of the year, when thou gatherest in thy labours out of the field.}{\arabic{verse}}
\threeverse{\arabic{verse}}%Ex.23:17
{שָׁלֹ֥שׁ פְּעָמִ֖ים בַּשָּׁנָ֑ה יֵרָאֶה֙ כׇּל־זְכ֣וּרְךָ֔ אֶל־פְּנֵ֖י הָאָדֹ֥ן \pasek  יְהֹוָֽה׃
\rashi{\rashiDH{שלש פעמים וגו׳. }לפי שהענין מדבר בשביעית, הוצרך לומר שלא יתעקרו שלש רגלים ממקומן (מכילתא פ״כ)׃ }\rashi{\rashiDH{כל זכורך. }הזכרים שבך׃}}
{תְּלָת זִמְנִין בְּשַׁתָּא יִתַּחְזוֹן כָּל דְּכוּרָךְ קֳדָם רִבּוֹן עָלְמָא יְיָ׃}
{Three times in the year all thy males shall appear before the Lord \textsc{God}.}{\arabic{verse}}
\threeverse{\arabic{verse}}%Ex.23:18
{לֹֽא־תִזְבַּ֥ח עַל־חָמֵ֖ץ דַּם־זִבְחִ֑י וְלֹֽא־יָלִ֥ין חֵֽלֶב־חַגִּ֖י עַד־בֹּֽקֶר׃
\rashi{\rashiDH{לא תזבח על חמץ וגו׳. }לא תשחט את הפסח בי״ד בניסן עד שתבער החמץ (מכילתא שם  פסחים סג)׃ }\rashi{\rashiDH{ולא ילין חלב חגי וגו׳. }חוץ למזבח׃}\rashi{\rashiDH{עד בקר. }יכול אף על המערכה יפסל בלינה, תלמוד לומר עַל מֹוקְדָה עַל הַמִּזְבֵּחַ כָּל הַלַּיְלָה (ויקרא ו, ב)׃ }\rashi{\rashiDH{ולא ילין. }אין לינה אלא בעמוד השחר, שנאמר עד בקר, אבל כל הלילה יכול להעלותו מן הרצפה למזבח׃ }}
{לָא תִכּוֹס עַל חֲמִיעַ דַּם פִּסְחִי וְלָא יְבִיתוּן בָּר מִמַּדְבְּחָא תַּרְבֵּי נִכְסַת חַגָּא עַד צַפְרָא׃}
{Thou shalt not offer the blood of My sacrifice with leavened bread; neither shall the fat of My feast remain all night until the morning.}{\arabic{verse}}
\threeverse{\arabic{verse}}%Ex.23:19
{רֵאשִׁ֗ית בִּכּוּרֵי֙ אַדְמָ֣תְךָ֔ תָּבִ֕יא בֵּ֖ית יְהֹוָ֣ה אֱלֹהֶ֑יךָ לֹֽא־תְבַשֵּׁ֥ל גְּדִ֖י בַּחֲלֵ֥ב אִמּֽוֹ׃ \petucha 
\rashi{\rashiDH{ראשית בכורי אדמתך. }אף השביעית חייבת בבכורים, לכך נאמר אף כאן בכורי אדמתך. כיצד, אדם נכנס לתוך שדהו, רואה תאנה שבכרה, כורך עליה גמי לסימן ומקדישה. ואין בכורים אלא משבעת המינין האמורין במקרא אֶרֶץ חִטָּה וּשׂעֹרָה וגו׳ (דברים ח, ח)׃ }\rashi{\rashiDH{לא תבשל גדי. }אף עגל וכבש בכלל גדי, שאין גדי אלא לשון ולד רך, ממה שאתה מוצא בכמה מקומות בתורה שכתוב גדי והוצרך לפרש אחריו עזים, כגון אָנֹכִי אֲשַׁלַּח גְּדִי עִזִּים (בראשית לח, יז), אֶת גְּדִי הָעִזִים (שם כ), שְׁנֵי גְּדָיֵי עִזִּים (שם כז, ט). ללמדך שכל מקום שנאמר גדי סתם, אף עגל וכבש במשמע. ובג׳ מקומות נכתב בתורה, אחד לאיסור אכילה, ואחד לאיסור הנאה, ואחד לאיסור בשול (חולין קטו׃)׃ }}
{רֵישׁ בִּכּוּרֵי אַרְעָךְ תַּיְתֵי לְבֵית מַקְדְּשָׁא דַּייָ אֱלָהָךְ לָא תֵיכְלוּן בְּשַׂר בַּחֲלַב׃}
{The choicest first-fruits of thy land thou shalt bring into the house of the \lord\space thy God. Thou shalt not seethe a kid in its mother’s milk.}{\arabic{verse}}
\threeverse{\aliya{ששי}}%Ex.23:20
{הִנֵּ֨ה אָנֹכִ֜י שֹׁלֵ֤חַ מַלְאָךְ֙ לְפָנֶ֔יךָ לִשְׁמָרְךָ֖ בַּדָּ֑רֶךְ וְלַהֲבִ֣יאֲךָ֔ אֶל־הַמָּק֖וֹם אֲשֶׁ֥ר הֲכִנֹֽתִי׃
\rashi{\rashiDH{הנה אנכי שולח מלאך. }כאן נתבשרו שעתידין לחטוא ושכינה אומרת להם כִּי לֹא אֶעֱלֶה בְּקִרְבְּךָ (שמות לג, ג)׃ 
}\rashi{\rashiDH{אשר הכנותי. }אשר זמנתי לתת לכם, זהו פשוטו. ומדרשו, אל המקום אשר הכינותי כבר, מקומי ניכר כנגדו, וזה אחד מן המקראות שאומרים שבית המקדש של מעלה, מכוון כנגד בית המקדש של מטה׃ }}
{הָא אֲנָא שָׁלַח מַלְאֲכָא קֳדָמָךְ לְמִטְּרָךְ בְּאוֹרְחָא וּלְאַעָלוּתָךְ לְאַתְרָא דְּאַתְקֵינִית׃}
{Behold, I send an angel before thee, to keep thee by the way, and to bring thee into the place which I have prepared.}{\arabic{verse}}
\threeverse{\arabic{verse}}%Ex.23:21
{הִשָּׁ֧מֶר מִפָּנָ֛יו וּשְׁמַ֥ע בְּקֹל֖וֹ אַל־תַּמֵּ֣ר בּ֑וֹ כִּ֣י לֹ֤א יִשָּׂא֙ לְפִשְׁעֲכֶ֔ם כִּ֥י שְׁמִ֖י בְּקִרְבּֽוֹ׃
\rashi{\rashiDH{אל תמר בו. }לשון המראה, כמו אֲשֶׁר יַמְרֶה אֶת פִּיךָ (יהושע א, יח)׃ }\rashi{\rashiDH{כי לא ישא לפשעכם. }אינו מלומד בכך, שהוא מן הכת שאין חוטאין, ועוד, שהוא שליח, ואינו עושה אלא שליחותו׃ }\rashi{\rashiDH{כי שמי בקרבו. }מחובר לראש המקרא, השמר מפניו כי שמי משותף בו. ורבותינו אמרו, זה מטטרו״ן, ששמו כשם רבו, מטטרו״ן בגימטריא שדי׃ }}
{אִסְתְּמַר מִן קֳדָמוֹהִי וְקַבֵּיל לְמֵימְרֵיהּ לָא תְסָרֵיב לְקִבְלֵיהּ אֲרֵי לָא יִשְׁבּוֹק לְחוֹבֵיכוֹן אֲרֵי בִּשְׁמִי מֵימְרֵיהּ׃}
{Take heed of him, and hearken unto his voice; be not rebellious against him; for he will not pardon your transgression; for My name is in him.}{\arabic{verse}}
\threeverse{\arabic{verse}}%Ex.23:22
{כִּ֣י אִם־שָׁמ֤וֹעַ תִּשְׁמַע֙ בְּקֹל֔וֹ וְעָשִׂ֕יתָ כֹּ֖ל אֲשֶׁ֣ר אֲדַבֵּ֑ר וְאָֽיַבְתִּי֙ אֶת־אֹ֣יְבֶ֔יךָ וְצַרְתִּ֖י אֶת־צֹרְרֶֽיךָ׃
\rashi{\rashiDH{וצרתי. }כתרגומו ואעיק׃}}
{אֲרֵי אִם קַבָּלָא תְקַבֵּיל לְמֵימְרֵיהּ וְתַעֲבֵיד כֹּל דַּאֲמַלֵּיל וְאֶסְנֵי יָת סָנְאָךְ וְאַעֵיק לְדִמְעִיקִין לָךְ׃}
{But if thou shalt indeed hearken unto his voice, and do all that I speak; then I will be an enemy unto thine enemies, and an adversary unto thine adversaries.}{\arabic{verse}}
\threeverse{\arabic{verse}}%Ex.23:23
{כִּֽי־יֵלֵ֣ךְ מַלְאָכִי֮ לְפָנֶ֒יךָ֒ וֶהֱבִֽיאֲךָ֗ אֶל־הָֽאֱמֹרִי֙ וְהַ֣חִתִּ֔י וְהַפְּרִזִּי֙ וְהַֽכְּנַעֲנִ֔י הַחִוִּ֖י וְהַיְבוּסִ֑י וְהִכְחַדְתִּֽיו׃}
{אֲרֵי יְהָךְ מַלְאֲכִי קֳדָמָךְ וְיַעֵילִנָּךְ לְוָת אֱמוֹרָאֵי וְחִתָּאֵי וּפְרִזָּאֵי וּכְנַעֲנָאֵי חִוָּאֵי וִיבוּסָאֵי וַאֲשֵׁיצֵינוּן׃}
{For Mine angel shall go before thee, and bring thee in unto the Amorite, and the Hittite, and the Perizzite, and the Canaanite, the Hivite, and the Jebusite; and I will cut them off.}{\arabic{verse}}
\threeverse{\arabic{verse}}%Ex.23:24
{לֹֽא־תִשְׁתַּחֲוֶ֤ה לֵאלֹֽהֵיהֶם֙ וְלֹ֣א תָֽעׇבְדֵ֔ם וְלֹ֥א תַעֲשֶׂ֖ה כְּמַֽעֲשֵׂיהֶ֑ם כִּ֤י הָרֵס֙ תְּהָ֣רְסֵ֔ם וְשַׁבֵּ֥ר תְּשַׁבֵּ֖ר מַצֵּבֹתֵיהֶֽם׃
\rashi{\rashiDH{הרס תהרסם. }לאותם אלהות׃}\rashi{\rashiDH{מצבותיהם. }אבנים שהם מציבין להשתחוות להם׃}}
{לָא תִסְגּוֹד לְטָעֲוָתְהוֹן וְלָא תִפְלְחִנִּין וְלָא תַעֲבֵיד כְּעוּבָדֵיהוֹן אֲרֵי פַגָּרָא תְפַגְּרִנּוּן וְתַבָּרָא תְּתַבַּר קָמָתְהוֹן׃}
{Thou shalt not bow down to their gods, nor serve them, nor do after their doings; but thou shalt utterly overthrow them, and break in pieces their pillars.}{\arabic{verse}}
\threeverse{\arabic{verse}}%Ex.23:25
{וַעֲבַדְתֶּ֗ם אֵ֚ת יְהֹוָ֣ה אֱלֹֽהֵיכֶ֔ם וּבֵרַ֥ךְ אֶֽת־לַחְמְךָ֖ וְאֶת־מֵימֶ֑יךָ וַהֲסִרֹתִ֥י מַחֲלָ֖ה מִקִּרְבֶּֽךָ׃ \setuma         }
{וְתִפְלְחוּן קֳדָם יְיָ אֱלָהֲכוֹן וִיבָרֵיךְ יָת מֵיכְלָךְ וְיָת מִשְׁתָּךְ וְאַעְדֵּי מַרְעִין בִּישִׁין מִבֵּינָךְ׃}
{And ye shall serve the \lord\space your God, and He will bless thy bread, and thy water; and I will take sickness away from the midst of thee.}{\arabic{verse}}
\threeverse{\aliya{שביעי}}%Ex.23:26
{לֹ֥א תִהְיֶ֛ה מְשַׁכֵּלָ֥ה וַעֲקָרָ֖ה בְּאַרְצֶ֑ךָ אֶת־מִסְפַּ֥ר יָמֶ֖יךָ אֲמַלֵּֽא׃
\rashi{\rashiDH{לא תהיה משכלה. }אם תעשה רצוני׃}\rashi{\rashiDH{משכלה. }מפלת נפלים או קוברת את בניה, קרויה משכלה׃ }}
{לָא תְהֵי תָּכְלָא וְעַקְרָא בְּאַרְעָךְ יָת מִנְיַן יוֹמָךְ אַשְׁלֵים׃}
{None shall miscarry, nor be barren, in thy land; the number of thy days I will fulfil.}{\arabic{verse}}
\threeverse{\arabic{verse}}%Ex.23:27
{אֶת־אֵֽימָתִי֙ אֲשַׁלַּ֣ח לְפָנֶ֔יךָ וְהַמֹּתִי֙ אֶת־כׇּל־הָעָ֔ם אֲשֶׁ֥ר תָּבֹ֖א בָּהֶ֑ם וְנָתַתִּ֧י אֶת־כׇּל־אֹיְבֶ֛יךָ אֵלֶ֖יךָ עֹֽרֶף׃
\rashi{\rashiDH{והמותי. }כמו והממתי, ותרגומו ואשגש. וכן כל תיבה שפועל שלה בכפל אות אחרונה, כשתהפוך לדבר בלשון פעלתי, יש מקומות שנוטל אות הכפולה ומדגיש את האות ונוקדו במלאפו״ם, כגון והמותי, מגזרת וְהָמַם גִּלְגַּל עֶגְלָתֹו (ישעיה כח, כח). וְסַבֹּותִי, מגזרת וְסָבַב בֵּית אֵל (שמואל־א ז, טז). דַּלֹותִי, מגזרת דָּלְלוּ וְחָרְבוּ (ישעיה יט, ו). עַל כַּפַּיִם חַקֹתִיךְ (שם מט, טז), מגזרת חִקְקֵי לֵב (שופטים ה, טו). אֶת מִי רַצֹּותִי (שמואל־א יב, ג), מגזרת רִצַּץ עָזַב דַּלִּים (איוב כ, יט). והמתרגם והמותי, ואקטל, טועה הוא, שאלו מגזרת מיתה היתה, אין ה״א שלה בפת״ח, ולא מ״ם שלה מודגשת, ולא נקודה מלאפו״ם. אלא וְהֵמַתִּי (בצירי)כגון וְהֵמַתָּה אֶת הָעָם הַזֶּה (במדבר יד, טו), והתי״ו מודגשת לפי שתבא במקום ב׳ תוי״ן, האחת נשרשת לפי שאין מיתה בלא תי״ו, והאחרת משמשת כמו אמרתי, חטאתי, עשיתי. וכן ונתתי, התי״ו מודגשת, שהיא באה במקום שתים, לפי שהיה צריך שלשה תוי״ן, שתים ליסוד כמו בְּיֹום תֵּת ה׳ (יהושע י, יב), מַתַּת אֱלֹהִים הִיא (קהלת ג, יג), והשלישית לשמוש׃ }\rashi{\rashiDH{עורף. }שינוסו מלפניך ויהפכו לך ערפם׃}}
{יָת אֵימְתִי אֲשַׁלַּח קֳדָמָךְ וַאֲשַׁגֵּישׁ יָת כָּל עַמָּא דְּאַתְּ אָתֵי לְאָגָחָא בְּהוֹן קְרָב וְאֶמְסַר יָת כָּל בַּעֲלֵי דְּבָבָךְ קֳדָמָךְ מַחְזְרֵי קְדָל׃}
{I will send My terror before thee, and will discomfit all the people to whom thou shalt come, and I will make all thine enemies turn their backs unto thee.}{\arabic{verse}}
\threeverse{\arabic{verse}}%Ex.23:28
{וְשָׁלַחְתִּ֥י אֶת־הַצִּרְעָ֖ה לְפָנֶ֑יךָ וְגֵרְשָׁ֗ה אֶת־הַחִוִּ֧י אֶת־הַֽכְּנַעֲנִ֛י וְאֶת־הַחִתִּ֖י מִלְּפָנֶֽיךָ׃
\rashi{\rashiDH{הצרעה. }מין שרץ העוף, והיתה מכה אותם בעיניהם, ומטילה בהם ארס והם מתים. והצרעה לא עברה את הירדן (סוטה לו.)׃ \rashiDH{והחתי והכנעני. }הם ארץ סיחון ועוג, לפיכך מכל ז׳ אומות לא מנה כאן אלא אלו. וחוי, אף על פי שהוא מעבר הירדן והלאה, שנו רבותינו במסכת סוטה (שם), על שפת הירדן עמדה וזרקה בהם מרה׃ }}
{וְאֶשְׁלַח יָת עָרָעִיתָא קֳדָמָךְ וּתְתָרֵיךְ יָת חִוָּאֵי יָת כְּנַעֲנָאֵי וְיָת חִתָּאֵי מִן קֳדָמָךְ׃}
{And I will send the hornet before thee, which shall drive out the Hivite, the Canaanite, and the Hittite, from before thee.}{\arabic{verse}}
\threeverse{\arabic{verse}}%Ex.23:29
{לֹ֧א אֲגָרְשֶׁ֛נּוּ מִפָּנֶ֖יךָ בְּשָׁנָ֣ה אֶחָ֑ת פֶּן־תִּהְיֶ֤ה הָאָ֙רֶץ֙ שְׁמָמָ֔ה וְרַבָּ֥ה עָלֶ֖יךָ חַיַּ֥ת הַשָּׂדֶֽה׃
\rashi{\rashiDH{שממה. }ריקנית מבני אדם, לפי שאתם מעט ואין בכם כדי למלאות אותה׃ }\rashi{\rashiDH{ורבה עליך. }ותרבה עליך׃}}
{לָא אֲתָרֵיכִנּוּן מִן קֳדָמָךְ בְּשַׁתָּא חֲדָא דִּלְמָא תְהֵי אַרְעָא צָדְיָא וְתִסְגֵּי עֲלָךְ חַיַּת בָּרָא׃}
{I will not drive them out from before thee in one year, lest the land become desolate, and the beasts of the field multiply against thee.}{\arabic{verse}}
\threeverse{\arabic{verse}}%Ex.23:30
{מְעַ֥ט מְעַ֛ט אֲגָרְשֶׁ֖נּוּ מִפָּנֶ֑יךָ עַ֚ד אֲשֶׁ֣ר תִּפְרֶ֔ה וְנָחַלְתָּ֖ אֶת־הָאָֽרֶץ׃
\rashi{\rashiDH{עד אשר תפרה. }תרבה, לשון פרי, כמו פרו ורבו׃ }}
{זְעֵיר זְעֵיר אֲתָרֵיכִנּוּן מִן קֳדָמָךְ עַד דְּתִסְגֵּי וְתַחְסֵין יָת אַרְעָא׃}
{By little and little I will drive them out from before thee, until thou be increased, and inherit the land.}{\arabic{verse}}
\threeverse{\arabic{verse}}%Ex.23:31
{וְשַׁתִּ֣י אֶת־גְּבֻלְךָ֗ מִיַּם־סוּף֙ וְעַד־יָ֣ם פְּלִשְׁתִּ֔ים וּמִמִּדְבָּ֖ר עַד־הַנָּהָ֑ר כִּ֣י \legarmeh  אֶתֵּ֣ן בְּיֶדְכֶ֗ם אֵ֚ת יֹשְׁבֵ֣י הָאָ֔רֶץ וְגֵרַשְׁתָּ֖מוֹ מִפָּנֶֽיךָ׃
\rashi{\rashiDH{ושתי. }לשון השתה, והתי״ו מודגשת מפני שבאה תחת שתים, שאין שיתה בלא תי״ו, והאחת לשמוש׃ }\rashi{\rashiDH{עד הנהר. }פרת׃}\rashi{\rashiDH{וגרשתמו. }ותגרשם׃}}
{וַאֲשַׁוֵּי יָת תְּחוּמָךְ מִיַּמָּא דְּסוּף וְעַד יַמָּא דִּפְלִשְׁתָּאֵי וּמִמַּדְבְּרָא עַד פְּרָת אֲרֵי אֶמְסַר בְּיַדְכוֹן יָת יָתְבֵי אַרְעָא וּתְתָרֵיכִנּוּן מִן קֳדָמָךְ׃}
{And I will set thy border from the Red Sea even unto the sea of the Philistines, and from the wilderness unto athe River; for I will deliver the inhabitants of the land into your hand; and thou shalt drive them out before thee.}{\arabic{verse}}
\threeverse{\arabic{verse}}%Ex.23:32
{לֹֽא־תִכְרֹ֥ת לָהֶ֛ם וְלֵאלֹֽהֵיהֶ֖ם בְּרִֽית׃}
{לָא תִגְזַר לְהוֹן וּלְטָעֲוָתְהוֹן קְיָם׃}
{Thou shalt make no covenant with them, nor with their gods.}{\arabic{verse}}
\threeverse{\arabic{verse}}%Ex.23:33
{לֹ֤א יֵשְׁבוּ֙ בְּאַרְצְךָ֔ פֶּן־יַחֲטִ֥יאוּ אֹתְךָ֖ לִ֑י כִּ֤י תַעֲבֹד֙ אֶת־אֱלֹ֣הֵיהֶ֔ם כִּֽי־יִהְיֶ֥ה לְךָ֖ לְמוֹקֵֽשׁ׃ \petucha 
\rashi{\rashiDH{כי תעבד וגו׳. }הרי אלו כי משמשין במקום אשר, וכן בכמה מקומות, וזהו לשון אי, שהוא אחד מד׳ לשונות שהכי משמש, וגם מצינו בהרבה מקומות אם משמש בלשון אשר, כמו וְאִם תַּקְרִיב מִנְחַת בִּכּוּרִים (ויקרא ב, יד), שהיא חובה׃ }}
{לָא יִתְּבוּן בְּאַרְעָךְ דִּלְמָא יְחַיְּיבוּן יָתָךְ קֳדָמָי אֲרֵי תִפְלַח יָת טָעֲוָתְהוֹן אֲרֵי יְהוֹן לָךְ לְתַקְלָא׃}
{They shall not dwell in thy land—lest they make thee sin against Me, for thou wilt serve their gods—for they will be a snare unto thee.}{\arabic{verse}}
\newperek
\threeverse{\Roman{chap}}%Ex.24:1
{וְאֶל־מֹשֶׁ֨ה אָמַ֜ר עֲלֵ֣ה אֶל־יְהֹוָ֗ה אַתָּה֙ וְאַהֲרֹן֙ נָדָ֣ב וַאֲבִיה֔וּא וְשִׁבְעִ֖ים מִזִּקְנֵ֣י יִשְׂרָאֵ֑ל וְהִשְׁתַּחֲוִיתֶ֖ם מֵרָחֹֽק׃
\rashi{\rashiDH{ואל משה אמר. }פרשה זו נאמרה קודם עשרת הדברות, ובד׳ בסיון נאמרה לו עלה׃ 
}}
{וּלְמֹשֶׁה אֲמַר סַק לִקְדָם יְיָ אַתְּ וְאַהֲרֹן נָדָב וַאֲבִיהוּא וְשִׁבְעִין מִסָּבֵי יִשְׂרָאֵל וְתִסְגְּדוּן מֵרַחִיק׃}
{And unto Moses He said: ‘Come up unto the \lord, thou, and Aaron, Nadab, and Abihu, and seventy of the elders of Israel; and worship ye afar off;}{\Roman{chap}}
\threeverse{\arabic{verse}}%Ex.24:2
{וְנִגַּ֨שׁ מֹשֶׁ֤ה לְבַדּוֹ֙ אֶל־יְהֹוָ֔ה וְהֵ֖ם לֹ֣א יִגָּ֑שׁוּ וְהָעָ֕ם לֹ֥א יַעֲל֖וּ עִמּֽוֹ׃
\rashi{\rashiDH{ונגש משה לבדו. }אל הערפל׃}}
{וְיִתְקָרַב מֹשֶׁה בִּלְחוֹדוֹהִי לִקְדָם יְיָ וְאִנּוּן לָא יִתְקָרְבוּן וְעַמָּא לָא יִסְּקוּן עִמֵּיהּ׃}
{and Moses alone shall come near unto the \lord; but they shall not come near; neither shall the people go up with him.’}{\arabic{verse}}
\threeverse{\arabic{verse}}%Ex.24:3
{וַיָּבֹ֣א מֹשֶׁ֗ה וַיְסַפֵּ֤ר לָעָם֙ אֵ֚ת כׇּל־דִּבְרֵ֣י יְהֹוָ֔ה וְאֵ֖ת כׇּל־הַמִּשְׁפָּטִ֑ים וַיַּ֨עַן כׇּל־הָעָ֜ם ק֤וֹל אֶחָד֙ וַיֹּ֣אמְר֔וּ כׇּל־הַדְּבָרִ֛ים אֲשֶׁר־דִּבֶּ֥ר יְהֹוָ֖ה נַעֲשֶֽׂה׃
\rashi{\rashiDH{ויבא משה ויספר לעם. }בו ביום׃}\rashi{\rashiDH{את כל דברי ה׳. }מצות, פרישה, והגבלה׃ }\rashi{\rashiDH{ואת כל המשפטים. }ז׳ מצות שנצטוו בני נח, ושבת, וכבוד אב ואם, ופרה אדומה, ודינין, שניתנו להם במרה (מכילתא בחדש פ״ג)׃ }}
{וַאֲתָא מֹשֶׁה וְאִשְׁתַּעִי לְעַמָּא יָת כָּל פִּתְגָמַיָּא דַּייָ וְיָת כָּל דִּינַיָּא וַאֲתֵיב כָּל עַמָּא קָלָא חַד וַאֲמַרוּ כָּל פִּתְגָמַיָּא דְּמַלֵּיל יְיָ נַעֲבֵיד׃}
{And Moses came and told the people all the words of the \lord, and all the ordinances; and all the people answered with one voice, and said: ‘All the words which the Lord hath spoken will we do.’}{\arabic{verse}}
\threeverse{\arabic{verse}}%Ex.24:4
{וַיִּכְתֹּ֣ב מֹשֶׁ֗ה אֵ֚ת כׇּל־דִּבְרֵ֣י יְהֹוָ֔ה וַיַּשְׁכֵּ֣ם בַּבֹּ֔קֶר וַיִּ֥בֶן מִזְבֵּ֖חַ תַּ֣חַת הָהָ֑ר וּשְׁתֵּ֤ים עֶשְׂרֵה֙ מַצֵּבָ֔ה לִשְׁנֵ֥ים עָשָׂ֖ר שִׁבְטֵ֥י יִשְׂרָאֵֽל׃
\rashi{\rashiDH{ויכתוב משה. }מבראשית ועד מתן תורה, וכתב מצות שנצטוו במרה (שם)׃ }\rashi{\rashiDH{וישכם בבקר. }בחמשה בסיון׃}}
{וּכְתַב מֹשֶׁה יָת כָּל פִּתְגָמַיָּא דַּייָ וְאַקְדֵּים בְּצַפְרָא וּבְנָא מַדְבְּחָא בְּשִׁפּוֹלֵי טוּרָא וְתַרְתַּא עֶשְׂרֵי קָמָא לִתְרֵי עֲשַׂר שִׁבְטַיָּא דְּיִשְׂרָאֵל׃}
{And Moses wrote all the words of the \lord, and rose up early in the morning, and builded an altar under the mount, and twelve pillars, according to the twelve tribes of Israel.}{\arabic{verse}}
\threeverse{\arabic{verse}}%Ex.24:5
{וַיִּשְׁלַ֗ח אֶֽת־נַעֲרֵי֙ בְּנֵ֣י יִשְׂרָאֵ֔ל וַיַּֽעֲל֖וּ עֹלֹ֑ת וַֽיִּזְבְּח֞וּ זְבָחִ֧ים שְׁלָמִ֛ים לַיהֹוָ֖ה פָּרִֽים׃
\rashi{\rashiDH{את נערי. }הבכורות׃}}
{וּשְׁלַח יָת בְּכוֹרֵי בְנֵי יִשְׂרָאֵל וְאַסִּיקוּ עֲלָוָן וְנַכִּיסוּ נִכְסַת קוּדְשִׁין קֳדָם יְיָ תּוֹרִין׃}
{And he sent the young men of the children of Israel, who offered burnt-offerings, and sacrificed peace-offerings of oxen unto the \lord.}{\arabic{verse}}
\threeverse{\arabic{verse}}%Ex.24:6
{וַיִּקַּ֤ח מֹשֶׁה֙ חֲצִ֣י הַדָּ֔ם וַיָּ֖שֶׂם בָּאַגָּנֹ֑ת וַחֲצִ֣י הַדָּ֔ם זָרַ֖ק עַל־הַמִּזְבֵּֽחַ׃
\rashi{\rashiDH{ויקח משה חצי הדם. }מי חלקו, מלאך בא וחלקו׃ 
}\rashi{\rashiDH{באגנות. }שתי אגנות, אחד לחצי דם עולה, ואחד לחצי דם שלמים, להזות אותם על העם. ומכאן למדו רבותינו, שנכנסו אבותינו לברית במילה וטבילה והזאת דמים, שאין הזאה בלא טבילה׃ }}
{וּנְסֵיב מֹשֶׁה פַּלְגוּת דְּמָא וְשַׁוִּי בְּמִזְרְקַיָּא וּפַלְגוּת דְּמָא זְרַק עַל מַדְבְּחָא׃}
{And Moses took half of the blood, and put it in basins; and half of the blood he dashed against the altar.}{\arabic{verse}}
\threeverse{\arabic{verse}}%Ex.24:7
{וַיִּקַּח֙ סֵ֣פֶר הַבְּרִ֔ית וַיִּקְרָ֖א בְּאׇזְנֵ֣י הָעָ֑ם וַיֹּ֣אמְר֔וּ כֹּ֛ל אֲשֶׁר־דִּבֶּ֥ר יְהֹוָ֖ה נַעֲשֶׂ֥ה וְנִשְׁמָֽע׃
\rashi{\rashiDH{ספר הברית. }מבראשית ועד מתן תורה, ומצות שנצטוו במרה׃ }}
{וּנְסֵיב סִפְרָא דִּקְיָמָא וּקְרָא קֳדָם עַמָּא וַאֲמַרוּ כֹּל דְּמַלֵּיל יְיָ נַעֲבֵיד וּנְקַבֵּיל׃}
{And he took the book of the covenant, and read in the hearing of the people; and they said: ‘All that the \lord\space hath spoken will we do, and obey.’}{\arabic{verse}}
\threeverse{\arabic{verse}}%Ex.24:8
{וַיִּקַּ֤ח מֹשֶׁה֙ אֶת־הַדָּ֔ם וַיִּזְרֹ֖ק עַל־הָעָ֑ם וַיֹּ֗אמֶר הִנֵּ֤ה דַֽם־הַבְּרִית֙ אֲשֶׁ֨ר כָּרַ֤ת יְהֹוָה֙ עִמָּכֶ֔ם עַ֥ל כׇּל־הַדְּבָרִ֖ים הָאֵֽלֶּה׃
\rashi{\rashiDH{ויזרוק. }ענין הזאה, ותרגומו וזרק על מדבחא לכפרא על עמא׃ 
}}
{וּנְסֵיב מֹשֶׁה יָת דְּמָא וּזְרַק עַל מַדְבְּחָא לְכַפָּרָא עַל עַמָּא וַאֲמַר הָא דַם קְיָמָא דִּגְזַר יְיָ עִמְּכוֹן עַל כָּל פִּתְגָמַיָּא הָאִלֵּין׃}
{And Moses took the blood, and sprinkled it on the people, and said: ‘Behold the blood of the covenant, which the \lord\space hath made with you in agreement with all these words.’}{\arabic{verse}}
\threeverse{\arabic{verse}}%Ex.24:9
{וַיַּ֥עַל מֹשֶׁ֖ה וְאַהֲרֹ֑ן נָדָב֙ וַאֲבִיה֔וּא וְשִׁבְעִ֖ים מִזִּקְנֵ֥י יִשְׂרָאֵֽל׃}
{וּסְלֵיק מֹשֶׁה וְאַהֲרֹן נָדָב וַאֲבִיהוּא וְשִׁבְעִין מִסָּבֵי יִשְׂרָאֵל׃}
{Then went up Moses, and Aaron, Nadab, and Abihu, and seventy of the elders of Israel;}{\arabic{verse}}
\threeverse{\arabic{verse}}%Ex.24:10
{וַיִּרְא֕וּ אֵ֖ת אֱלֹהֵ֣י יִשְׂרָאֵ֑ל וְתַ֣חַת רַגְלָ֗יו כְּמַעֲשֵׂה֙ לִבְנַ֣ת הַסַּפִּ֔יר וּכְעֶ֥צֶם הַשָּׁמַ֖יִם לָטֹֽהַר׃
\rashi{\rashiDH{ויראו את אלהי ישראל. }נסתכלו והציצו ונתחייבו מיתה, אלא שלא רצה הקב״ה לערבב שמחת התורה, והמתין לנדב ואביהוא עד יום חנוכת המשכן, ולזקנים עד וַיְהִי הָעָם כְּמִתְאֹנֲנִים וגו׳ וַתִּבְעַר בָּם אֵש ה׳ וַתֹּאכַל בִּקְצֵה הַמַּחֲנֶה, בקצינים שבמחנה׃ }\rashi{\rashiDH{כמעשה לבנת הספיר. }היא היתה לפניו בשעת השעבוד, לזכור צרתן של ישראל שהיו משועבדים במעשה לבנים׃ }\rashi{\rashiDH{וכעצם השמים לטהר. }משנגאלו היה אור וחדוה לפניו׃}\rashi{\rashiDH{וכעצם. }כתרגומו לשון מראה׃}\rashi{\rashiDH{לטהר. }לשון ברור וצלול׃ 
}}
{וַחֲזוֹ יָת יְקָר אֱלָהָא דְּיִשְׂרָאֵל וּתְחוֹת כּוּרְסֵי יְקָרֵיהּ כְּעוֹבָד אֶבֶן טָבָא וּכְמִחְזֵי שְׁמַיָּא לְבָרִירוּ׃}
{and they saw the God of Israel; and there was under His feet the like of a paved work of sapphire stone, and the like of the very heaven for clearness.}{\arabic{verse}}
\threeverse{\arabic{verse}}%Ex.24:11
{וְאֶל־אֲצִילֵי֙ בְּנֵ֣י יִשְׂרָאֵ֔ל לֹ֥א שָׁלַ֖ח יָד֑וֹ וַֽיֶּחֱזוּ֙ אֶת־הָ֣אֱלֹהִ֔ים וַיֹּאכְל֖וּ וַיִּשְׁתּֽוּ׃ \setuma         
\rashi{\rashiDH{ואל אצילי. }הם נדב ואביהוא והזקנים׃}\rashi{\rashiDH{לא שלח ידו. }מכלל שהיו ראויים להשתלח בהם יד׃}\rashi{\rashiDH{ויחזו את האלהים. }היו מסתכלין בו בלב גס, מתוך אכילה ושתייה, כך מדרש תנחומא (בהעלותך ט״ו). ואונקלוס לא תרגם כן, אצילי, לשון גדולים, כמו וּמֵאֲצִילֶיהָ קְרָאתִיךָ (ישעיה מא, ט), וַיָּאצֶל מִן הָרוּחַ (במדבר יא, כה), שֵׁשׁ אַמֹּות אַצִּילָה (יחזקאל מא, ח)׃ }}
{וּלְרַבְרְבֵי בְּנֵי יִשְׂרָאֵל לָא הֲוָה נִזְקָא וַחֲזוֹ יָת יְקָרָא דַּייָ וַהֲווֹ חָדַן בְּקוּרְבָּנֵיהוֹן דְּאִתְקַבַּלוּ כְּאִלּוּ אָכְלִין וְשָׁתַן׃}
{And upon the nobles of the children of Israel He laid not His hand; and they beheld God, and did eat and drink.}{\arabic{verse}}
\threeverse{\arabic{verse}}%Ex.24:12
{וַיֹּ֨אמֶר יְהֹוָ֜ה אֶל־מֹשֶׁ֗ה עֲלֵ֥ה אֵלַ֛י הָהָ֖רָה וֶהְיֵה־שָׁ֑ם וְאֶתְּנָ֨ה לְךָ֜ אֶת־לֻחֹ֣ת הָאֶ֗בֶן וְהַתּוֹרָה֙ וְהַמִּצְוָ֔ה אֲשֶׁ֥ר כָּתַ֖בְתִּי לְהוֹרֹתָֽם׃
\rashi{\rashiDH{ויאמר ה׳ אל משה. }לאחר מתן תורה׃}\rashi{\rashiDH{עלה אלי ההרה והיה שם. }מ׳ יום׃}\rashi{\rashiDH{את לחת האבן והתורה והמצוה אשר כתבתי להורותם. }כל שש מאות ושלש עשרה מצות בכלל עשרת הדברות הן. ורבינו סעדיה פירש באזהרות שיסד, לכל דבור ודבור מצות התלויות בו׃ }}
{וַאֲמַר יְיָ לְמֹשֶׁה סַק לִקְדָמַי לְטוּרָא וִהְוִי תַמָּן וְאֶתֵּין לָךְ יָת לוּחֵי אַבְנָא וְאוֹרָיְתָא וְתַפְקֵידְתָּא דִּכְתַבִית לְאַלּוֹפֵיהוֹן׃}
{And the \lord\space said unto Moses: ‘Come up to Me into the mount and be there; and I will give thee the tables of stone, and the law and the commandment, which I have written, that thou mayest teach them.’}{\arabic{verse}}
\threeverse{\arabic{verse}}%Ex.24:13
{וַיָּ֣קׇם מֹשֶׁ֔ה וִיהוֹשֻׁ֖עַ מְשָׁרְת֑וֹ וַיַּ֥עַל מֹשֶׁ֖ה אֶל־הַ֥ר הָאֱלֹהִֽים׃
\rashi{\rashiDH{ויקם משה ויהושע משרתו. }לא ידעתי מה טיבו של יהושע כאן, ואומר אני שהיה התלמיד מלוה לרב עד מקום הגבלת תחומי ההר, שאינו רשאי לילך משם והלאה, ומשם ויעל משה לבדו אל הר האלהים, ויהושע נטה שם אהלו ונתעכב שם כל מ׳ יום, שכן מצינו כשירד משה, וישמע יהושע את קול העם ברעה, למדנו שלא היה יהושע עמהם׃ }}
{וְקָם מֹשֶׁה וִיהוֹשֻעַ מְשׁוּמְשָׁנֵיהּ וּסְלֵיק מֹשֶׁה לְטוּרָא דְּאִתְגְּלִי עֲלוֹהִי יְקָרָא דַּייָ׃}
{And Moses rose up, and Joshua his minister; and Moses went up into the mount of God.}{\arabic{verse}}
\threeverse{\arabic{verse}}%Ex.24:14
{וְאֶל־הַזְּקֵנִ֤ים אָמַר֙ שְׁבוּ־לָ֣נוּ בָזֶ֔ה עַ֥ד אֲשֶׁר־נָשׁ֖וּב אֲלֵיכֶ֑ם וְהִנֵּ֨ה אַהֲרֹ֤ן וְחוּר֙ עִמָּכֶ֔ם מִי־בַ֥עַל דְּבָרִ֖ים יִגַּ֥שׁ אֲלֵהֶֽם׃
\rashi{\rashiDH{ואל הזקנים אמר. }בצאתו מן המחנה׃}\rashi{\rashiDH{שבו לנו בזה. }והתעכבו כאן עם שאר העם במחנה, להיות נכונים לשפוט לכל איש ריבו׃ }\rashi{\rashiDH{חור. }בנה של מרים היה, ואביו כלב בן יפנה, שנאמר וַיִּקַּח לֹו כָלֵב אֶת אֶפְרָת וַתֵּלֶד לֹו אֶת חוּר (דברי הימים־א ב, יט), אפרת זו מרים, כדאיתא בסוטה (יא׃)׃ }\rashi{\rashiDH{מי בעל דברים. }מי שיש לו דין׃}}
{וּלְסָבַיָּא אֲמַר אוֹרִיכוּ לַנָא הָכָא עַד דִּנְתוּב לְוָתְכוֹן וְהָא אַהֲרֹן וְחוּר עִמְּכוֹן מַן דְּאִית לֵיהּ דִּינָא יִתְקָרַב לִקְדָמֵיהוֹן׃}
{And unto the elders he said: ‘Tarry ye here for us, until we come back unto you; and, behold, Aaron and Hur are with you; whosoever hath a cause, let him come near unto them.’}{\arabic{verse}}
\threeverse{\aliya{מפטיר}}%Ex.24:15
{וַיַּ֥עַל מֹשֶׁ֖ה אֶל־הָהָ֑ר וַיְכַ֥ס הֶעָנָ֖ן אֶת־הָהָֽר׃}
{וּסְלֵיק מֹשֶׁה לְטוּרָא וַחֲפָא עֲנָנָא יָת טוּרָא׃}
{And Moses went up into the mount, and the cloud covered the mount.}{\arabic{verse}}
\threeverse{\arabic{verse}}%Ex.24:16
{וַיִּשְׁכֹּ֤ן כְּבוֹד־יְהֹוָה֙ עַל־הַ֣ר סִינַ֔י וַיְכַסֵּ֥הוּ הֶעָנָ֖ן שֵׁ֣שֶׁת יָמִ֑ים וַיִּקְרָ֧א אֶל־מֹשֶׁ֛ה בַּיּ֥וֹם הַשְּׁבִיעִ֖י מִתּ֥וֹךְ הֶעָנָֽן׃
\rashi{\rashiDH{ויכסהו הענן. }רבותינו חולקים בדבר (יומא ד.׃), יש מהם אומרים אלו ששה ימים שמראש חדש (עד עצרת יום מתן תורה. רש״י ישן)׃ }\rashi{\rashiDH{ויכסהו הענן. }להר׃}\rashi{\rashiDH{ויקרא אל משה ביום השביעי. }לומר עשרת הדברות, ומשה וכל בני ישראל עומדים, אלא שחלק הכתוב כבוד למשה. ויש אומרים, ויכסהו הענן למשה ו׳ ימים, לאחר עשרת הדברות, והם היו בתחלת מ׳ יום שעלה משה לקבל הלוחות, ולמדך שכל הנכנס למחנה שכינה טעון פרישה ששה ימים (שם)׃ }}
{וּשְׁרָא יְקָרָא דַּייָ עַל טוּרָא דְּסִינַי וַחֲפָהִי עֲנָנָא שִׁתָּא יוֹמִין וּקְרָא לְמֹשֶׁה בְּיוֹמָא שְׁבִיעָאָה מִגּוֹ עֲנָנָא׃}
{And the glory of the \lord\space abode upon mount Sinai, and the cloud covered it six days; and the seventh day He called unto Moses out of the midst of the cloud.}{\arabic{verse}}
\threeverse{\arabic{verse}}%Ex.24:17
{וּמַרְאֵה֙ כְּב֣וֹד יְהֹוָ֔ה כְּאֵ֥שׁ אֹכֶ֖לֶת בְּרֹ֣אשׁ הָהָ֑ר לְעֵינֵ֖י בְּנֵ֥י יִשְׂרָאֵֽל׃}
{וְחֵיזוּ יְקָרָא דַּייָ כְּאִישָׁא אָכְלָא בְּרֵישׁ טוּרָא לְעֵינֵי בְּנֵי יִשְׂרָאֵל׃}
{And the appearance of the glory of the \lord\space was like devouring fire on the top of the mount in the eyes of the children of Israel.}{\arabic{verse}}
\threeverse{\arabic{verse}}%Ex.24:18
{וַיָּבֹ֥א מֹשֶׁ֛ה בְּת֥וֹךְ הֶעָנָ֖ן וַיַּ֣עַל אֶל־הָהָ֑ר וַיְהִ֤י מֹשֶׁה֙ בָּהָ֔ר אַרְבָּעִ֣ים י֔וֹם וְאַרְבָּעִ֖ים לָֽיְלָה׃ \petucha 
\rashi{\rashiDH{בתוך הענן. }ענן זה כמין עשן הוא, ועשה לו הקב״ה למשה שביל (נ״א חופה)בתוכו׃ }}
{וְעָאל מֹשֶׁה בְּגוֹ עֲנָנָא וּסְלֵיק לְטוּרָא וַהֲוָה מֹשֶׁה בְּטוּרָא אַרְבְּעִין יְמָמִין וְאַרְבְּעִין לֵילָוָן׃}
{And Moses entered into the midst of the cloud, and went up into the mount; and Moses was in the mount forty days and forty nights.}{\arabic{verse}}
\newperek
\newparsha{תרומה}
\threeverse{\aliya{תרומה}}%Ex.25:1
{וַיְדַבֵּ֥ר יְהֹוָ֖ה אֶל־מֹשֶׁ֥ה לֵּאמֹֽר׃}
{וּמַלֵּיל יְיָ עִם מֹשֶׁה לְמֵימַר׃}
{And the \lord\space spoke unto Moses, saying:}{\Roman{chap}}
\threeverse{\arabic{verse}}%Ex.25:2
{דַּבֵּר֙ אֶל־בְּנֵ֣י יִשְׂרָאֵ֔ל וְיִקְחוּ־לִ֖י תְּרוּמָ֑ה מֵאֵ֤ת כׇּל־אִישׁ֙ אֲשֶׁ֣ר יִדְּבֶ֣נּוּ לִבּ֔וֹ תִּקְח֖וּ אֶת־תְּרוּמָתִֽי׃
\rashi{\rashiDH{ויקחו לי תרומה. }לי, לשמי׃ }\rashi{\rashiDH{תרומה. }הפרשה, יפרישו לי מממונם נדבה׃ }\rashi{\rashiDH{ידבנו לבו. }לשון נדבה, והוא לשון רצון טוב, פיישנ״ט בלע״ז (געשענק) }\rashi{\rashiDH{תקחו את תרומתי. }אמרו רבותינו, ג׳ תרומות אמורות כאן, אחת תרומת בקע לגלגלת שנעשו מהם האדנים, כמו שמפורש באלה פקודי, ואחת תרומת המזבח בקע לגלגלת, לקופות, לקנות מהן קרבנות צבור, ואחת תרומת המשכן, נדבת כל אחד ואחד שהתנדבו. י״ג דברים האמורים בענין, כולם הוצרכו למלאכת המשכן או לבגדי כהונה כשתדקדק בהם׃ 
}}
{מַלֵּיל עִם בְּנֵי יִשְׂרָאֵל וְיַפְרְשׁוּן קֳדָמַי אַפְרָשׁוּתָא מִן כָּל גְּבַר דְּיִתְרְעֵי לִבֵּיהּ תִּסְּבוּן יָת אַפְרָשׁוּתִי׃}
{’Speak unto the children of Israel, that they take for Me an offering; of every man whose heart maketh him willing ye shall take My offering.}{\arabic{verse}}
\threeverse{\arabic{verse}}%Ex.25:3
{וְזֹאת֙ הַתְּרוּמָ֔ה אֲשֶׁ֥ר תִּקְח֖וּ מֵאִתָּ֑ם זָהָ֥ב וָכֶ֖סֶף וּנְחֹֽשֶׁת׃
\rashi{\rashiDH{זהב וכסף ונחשת וגו׳. }כלם באו בנדבה איש איש מה שנדבו לבו, חוץ מן הכסף שבא בשוה, מחצית השקל לכל אחד. ולא מצינו בכל מלאכת המשכן שהוצרך שם כסף יותר, שנאמר וְכֶסֶף פְּקוּדֵי הָעֵדָה וגו׳ בֶּקַע לַגֻּלְגֹלֶת וגו׳ (שמות לח, כוכז), ושאר הכסף הבא שם בנדבה, עשאוה לכלי שרת׃ }}
{וְדָא אַפְרָשׁוּתָא דְּתִסְּבְוּן מִנְּהוֹן דַּהְבָּא וְכַסְפָּא וּנְחָשָׁא׃}
{And this is the offering which ye shall take of them: gold, and silver, and brass;}{\arabic{verse}}
\threeverse{\arabic{verse}}%Ex.25:4
{וּתְכֵ֧לֶת וְאַרְגָּמָ֛ן וְתוֹלַ֥עַת שָׁנִ֖י וְשֵׁ֥שׁ וְעִזִּֽים׃
\rashi{\rashiDH{ותכלת. }צמר צבוע בדם חלזון (מנחות מד.), וצבעו ירוק׃ }\rashi{\rashiDH{וארגמן. }צמר צבוע ממין צבע ששמו ארגמן׃}\rashi{\rashiDH{ושש. }הוא פשתן (יבמות ד׃)׃ 
}\rashi{\rashiDH{ועזים. }נוצה של עזים, לכך תרגם אונקלוס וּמֵעַזֵּי, הבא מן העזים, ולא עזים עצמן, שתרגום של עזים עִזַיָא׃ }}
{וְתַכְלָא וְאַרְגְּוָנָא וּצְבַע זְהוֹרִי וּבוּץ וּמַעְזֵי׃}
{and blue, and purple, and scarlet, and fine linen, and goats’ hair;}{\arabic{verse}}
\threeverse{\arabic{verse}}%Ex.25:5
{וְעֹרֹ֨ת אֵילִ֧ם מְאׇדָּמִ֛ים וְעֹרֹ֥ת תְּחָשִׁ֖ים וַעֲצֵ֥י שִׁטִּֽים׃
\rashi{\rashiDH{מאדמים. }צבועות היו אדום לאחר עבודן׃}\rashi{\rashiDH{תחשים. }מין חיה, ולא היתה אלא לשעה (שבת כח׃), והרבה גוונים היו לה, לכך מתרגם סַסְגוֹנָא, ֶׁשָּׂשׂ ומתפאר בגוונין שלו (שם)׃ }\rashi{\rashiDH{ועצי שטים. }ומאין היו להם במדבר, פירש רבי תנחומא (תרומה ט), יעקב אבינו צפה ברוח הקדש שעתידין ישראל לבנות משכן במדבר, והביא ארזים למצרים ונטעם, וצוה לבניו ליטלם עמהם כשיצאו ממצרים׃ }}
{וּמַשְׁכֵּי דְּדִכְרֵי מְסוּמְּקֵי וּמַשְׁכֵּי סָסְגוֹנָא וְאָעֵי שִׁטִּין׃}
{and rams’ skins dyed red, and sealskins, and acacia-wood;}{\arabic{verse}}
\threeverse{\aliya{לוי}}%Ex.25:6
{שֶׁ֖מֶן לַמָּאֹ֑ר בְּשָׂמִים֙ לְשֶׁ֣מֶן הַמִּשְׁחָ֔ה וְלִקְטֹ֖רֶת הַסַּמִּֽים׃
\rashi{\rashiDH{שמן למאור. }שמן זית זך להעלות נר תמיד׃}\rashi{\rashiDH{בשמים לשמן המשחה. }שנעשה למשוח כלי המשכן והמשכן לקדשו, והוצרכו לו בשמים, כמו שמפורש בכי תשא׃ }\rashi{\rashiDH{ולקטורת הסמים. }שהיו מקטירין בכל ערב ובקר, כמו שמפורש בואתה תצוה. ולשון קטרת, העלאת קיטור ותמרות עשן׃ }}
{מִשְׁחָא לְאַנְהָרוּתָא בּוּסְמַיָּא לִמְשַׁח רְבוּתָא וְלִקְטֹרֶת בּוּסְמַיָּא׃}
{oil for the light, spices for the anointing oil, and for the sweet incense;}{\arabic{verse}}
\threeverse{\arabic{verse}}%Ex.25:7
{אַבְנֵי־שֹׁ֕הַם וְאַבְנֵ֖י מִלֻּאִ֑ים לָאֵפֹ֖ד וְלַחֹֽשֶׁן׃
\rashi{\rashiDH{אבני שהם. }שתים הוצרכו שם, לצורך האפוד האמור בואתה תצוה׃ }\rashi{\rashiDH{מלאים. }על שם שעושין להם בזהב מושב כמין גומא, ונותנין האבן שם למלאות הגומא, קרויים אבני מלואים, ומקום המושב קרוי משבצות׃ }\rashi{\rashiDH{לאפוד ולחושן. }אבני השהם לאפוד ואבני המלואים לחשן. וחשן ואפוד מפורשים בואתה תצוה, והם מיני תכשיט׃ }}
{אַבְנֵי בוּרְלָא וְאַבְנֵי אַשְׁלָמוּתָא לְשַׁקָּעָא בְּאֵיפוֹדָא וּבְחוּשְׁנָא׃}
{onyx stones, and stones to be set, for the ephod, and for the breastplate.}{\arabic{verse}}
\threeverse{\arabic{verse}}%Ex.25:8
{וְעָ֥שׂוּ לִ֖י מִקְדָּ֑שׁ וְשָׁכַנְתִּ֖י בְּתוֹכָֽם׃
\rashi{\rashiDH{ועשו לי מקדש. }ועשו לשמי בית קדושה׃}}
{וְיַעְבְּדוּן קֳדָמַי מַקְדַּשׁ וְאַשְׁרֵי שְׁכִינְתִי בֵּינֵיהוֹן׃}
{And let them make Me a sanctuary, that I may dwell among them.}{\arabic{verse}}
\threeverse{\arabic{verse}}%Ex.25:9
{כְּכֹ֗ל אֲשֶׁ֤ר אֲנִי֙ מַרְאֶ֣ה אוֹתְךָ֔ אֵ֚ת תַּבְנִ֣ית הַמִּשְׁכָּ֔ן וְאֵ֖ת תַּבְנִ֣ית כׇּל־כֵּלָ֑יו וְכֵ֖ן תַּעֲשֽׂוּ׃ \setuma         
\rashi{\rashiDH{ככל אשר אני מראה אותך. }כאן, את תבנית המשכן (מנחות כט.). המקרא הזה מחובר למקרא שלמעלה הימנו, ועשו לי מקדש ככל אשר אני מראה אותך׃ }\rashi{\rashiDH{וכן תעשו. }לדורות (סנהדרין טז׃, שבועות יד׃), אם יאבד אחד מן הכלים או כשתעשו לי כלי בית עולמים, כגון שולחנות ומנורות וכיורות ומכונות שעשה שלמה, כתבנית אלו תעשו אותם. ואם לא היה המקרא מחובר למקרא שלמעלה הימנו, לא היה לו לכתוב וכן תעשו, אלא כן תעשו, והיה מדבר על עשיית אהל מועד וכליו׃ }}
{כְּכֹל דַּאֲנָא מַחְזֵי יָתָךְ יָת דְּמוּת מַשְׁכְּנָא וְיָת דְּמוּת כָּל מָנוֹהִי וְכֵן תַּעְבְּדוּן׃}
{According to all that I show thee, the pattern of the tabernacle, and the pattern of all the furniture thereof, even so shall ye make it.}{\arabic{verse}}
\threeverse{\aliya{ישראל}}%Ex.25:10
{וְעָשׂ֥וּ אֲר֖וֹן עֲצֵ֣י שִׁטִּ֑ים אַמָּתַ֨יִם וָחֵ֜צִי אָרְכּ֗וֹ וְאַמָּ֤ה וָחֵ֙צִי֙ רׇחְבּ֔וֹ וְאַמָּ֥ה וָחֵ֖צִי קֹמָתֽוֹ׃
\rashi{\rashiDH{ועשו ארון. }כמין ארונות שעושים בלא רגלים, עשוים כמין ארגז שקורין אישקרי״ן (שריין שראנק) יושב על שוליו׃ }}
{וְיַעְבְּדוּן אֲרוֹנָא דְּאָעֵי שִׁטִּין תַּרְתֵּין אַמִּין וּפַלְגָּא אוּרְכֵּיהּ וְאַמְּתָא וּפַלְגָּא פוּתְיֵיהּ וְאַמְּתָא וּפַלְגָּא רוּמֵיהּ׃}
{And they shall make an ark of acacia-wood: two cubits and a half shall be the length thereof, and a cubit and a half the breadth thereof, and a cubit and a half the height thereof.}{\arabic{verse}}
\threeverse{\arabic{verse}}%Ex.25:11
{וְצִפִּיתָ֤ אֹתוֹ֙ זָהָ֣ב טָה֔וֹר מִבַּ֥יִת וּמִח֖וּץ תְּצַפֶּ֑נּוּ וְעָשִׂ֧יתָ עָלָ֛יו זֵ֥ר זָהָ֖ב סָבִֽיב׃
\rashi{\rashiDH{מבית ומחוץ תצפנו. }שלשה ארונות עשה בצלאל, ב׳ של זהב וא׳ של עץ (יומא עב׃), וד׳ כתלים ושולים לכל אחד, ופתוחים מלמעלה, נתן של עץ בתוך של זהב, ושל זהב בתוך של עץ, וְחִפָּה שפתו העליונה בזהב, נמצא מצופה מבית ומחוץ׃ }\rashi{\rashiDH{זר זהב. }כמין כתר מוקף לו סביב, למעלה משפתו, שעשה הארון החיצון גבוה מן הפנימי, עד שעלה למול עובי הכפורת ולמעלה הימנו משהו, וכשהכפורת שוכב על עובי הכתלים, עולה הזר למעלה מכל עובי הכפורת כל שהוא, והוא סימן לכתר תורה׃ }}
{וְתִחְפֵי יָתֵיהּ דְּהַב דְּכֵי מִגָּיו וּמִבַּרָא תִּחְפֵינֵיהּ וְתַעֲבֵיד עֲלוֹהִי זִיר דִּדְהַב סְחוֹר סְחוֹר׃}
{And thou shalt overlay it with pure gold, within and without shalt thou overlay it, and shalt make upon it a crown of gold round about.}{\arabic{verse}}
\threeverse{\arabic{verse}}%Ex.25:12
{וְיָצַ֣קְתָּ לּ֗וֹ אַרְבַּע֙ טַבְּעֹ֣ת זָהָ֔ב וְנָ֣תַתָּ֔ה עַ֖ל אַרְבַּ֣ע פַּעֲמֹתָ֑יו וּשְׁתֵּ֣י טַבָּעֹ֗ת עַל־צַלְעוֹ֙ הָֽאֶחָ֔ת וּשְׁתֵּי֙ טַבָּעֹ֔ת עַל־צַלְע֖וֹ הַשֵּׁנִֽית׃
\rashi{\rashiDH{ויצקת. }לשון התכה כתרגומו׃}\rashi{\rashiDH{פעמותיו. }כתרגומו זִוְיָתֵיהּ. ובזויות העליונות סמוך לכפורת היו נתונים, שתים מכאן ושתים מכאן לרחבו של ארון, והבדים נתונים בהם, וארכו של ארון מפסיק בין הבדים אמתים וחצי בין בד לבד, שיהיו שני בני אדם הנושאין את הארון מהלכין ביניהם, וכן מפורש במנחות (צח׃) בפרק שתי הלחם׃ }\rashi{\rashiDH{ושתי טבעות על צלעו האחת. }הן הן ד׳ טבעות שבתחלת המקרא, ופירש לך היכן היו, והוי״ו זו יתירה היא, ופתרונו כמו שתי טבעות, ויש לך לישבה כן, ושתי מן הטבעות האלו על צלעו האחת׃ }\rashi{\rashiDH{צלעו. }צדו׃}}
{וְתַתֵּיךְ לֵיהּ אַרְבַּע עִזְקָן דִּדְהַב וְתִתֵּין עַל אַרְבַּע זָוְיָתֵיהּ וְתַרְתֵּין עִזְקָן עַל סִטְרֵיהּ חַד וְתַרְתֵּין עִזְקָן עַל סִטְרֵיהּ תִּנְיָנָא׃}
{And thou shalt cast four rings of gold for it, and put them in the four feet thereof; and two rings shall be on the one side of it, and two rings on the other side of it.}{\arabic{verse}}
\threeverse{\arabic{verse}}%Ex.25:13
{וְעָשִׂ֥יתָ בַדֵּ֖י עֲצֵ֣י שִׁטִּ֑ים וְצִפִּיתָ֥ אֹתָ֖ם זָהָֽב׃
\rashi{\rashiDH{בדי. }מוטות׃ 
}}
{וְתַעֲבֵיד אֲרִיחֵי דְּאָעֵי שִׁטִּין וְתִחְפֵי יָתְהוֹן דַּהְבָּא׃}
{And thou shalt make staves of acacia-wood, and overlay them with gold.}{\arabic{verse}}
\threeverse{\arabic{verse}}%Ex.25:14
{וְהֵֽבֵאתָ֤ אֶת־הַבַּדִּים֙ בַּטַּבָּעֹ֔ת עַ֖ל צַלְעֹ֣ת הָאָרֹ֑ן לָשֵׂ֥את אֶת־הָאָרֹ֖ן בָּהֶֽם׃}
{וְתַעֵיל יָת אֲרִיחַיָּא בְּעִזְקָתָא עַל סִטְרֵי אֲרוֹנָא לְמִטַּל יָת אֲרוֹנָא בְּהוֹן׃}
{And thou shalt put the staves into the rings on the sides of the ark, wherewith to bear the ark.}{\arabic{verse}}
\threeverse{\arabic{verse}}%Ex.25:15
{בְּטַבְּעֹת֙ הָאָרֹ֔ן יִהְי֖וּ הַבַּדִּ֑ים לֹ֥א יָסֻ֖רוּ מִמֶּֽנּוּ׃
\rashi{\rashiDH{לא יסורו ממנו} לעולם (יומא עב.)׃}}
{בְּעִזְקָת אֲרוֹנָא יְהוֹן אֲרִיחַיָּא לָא יִעְדּוֹן מִנֵּיהּ׃}
{The staves shall be in the rings of the ark; they shall not be taken from it.}{\arabic{verse}}
\threeverse{\arabic{verse}}%Ex.25:16
{וְנָתַתָּ֖ אֶל־הָאָרֹ֑ן אֵ֚ת הָעֵדֻ֔ת אֲשֶׁ֥ר אֶתֵּ֖ן אֵלֶֽיךָ׃
\rashi{\rashiDH{ונתת אל הארן. }כמו בארון׃}\rashi{\rashiDH{העדת. }התורה, שהיא לעדות ביני וביניכם שצויתי אתכם מצות הכתובות בה׃ }}
{וְתִתֵּין בַּאֲרוֹנָא יָת סָהֲדוּתָא דְּאֶתֵּין לָךְ׃}
{And thou shalt put into the ark the testimony which I shall give thee.}{\arabic{verse}}
\threeverse{\aliya{שני}}%Ex.25:17
{וְעָשִׂ֥יתָ כַפֹּ֖רֶת זָהָ֣ב טָה֑וֹר אַמָּתַ֤יִם וָחֵ֙צִי֙ אׇרְכָּ֔הּ וְאַמָּ֥ה וָחֵ֖צִי רׇחְבָּֽהּ׃
\rashi{\rashiDH{כפורת. }כסוי על הארון, שהיה פתוח מלמעלה, ומניחו עליו כמין דף׃ }\rashi{\rashiDH{אמתים וחצי ארכה. }כארכו של ארון, ורחבה כרחבו של ארון, ומונחת על עובי הכתלים ארבעתם, ואף על פי שלא נתן שיעור לעוביה, פירשו רבותינו שהיה עוביה טפח (סוכה ה.)׃ }}
{וְתַעֲבֵיד כָּפוּרְתָּא דִּדְהַב דְּכֵי תַּרְתֵּין אַמִּין וּפַלְגָּא אוּרְכַּהּ וְאַמְּתָא וּפַלְגָּא פוּתְיַהּ׃}
{And thou shalt make an ark-cover of pure gold: two cubits and a half shall be the length thereof, and a cubit and a half the breadth thereof.}{\arabic{verse}}
\threeverse{\arabic{verse}}%Ex.25:18
{וְעָשִׂ֛יתָ שְׁנַ֥יִם כְּרֻבִ֖ים זָהָ֑ב מִקְשָׁה֙ תַּעֲשֶׂ֣ה אֹתָ֔ם מִשְּׁנֵ֖י קְצ֥וֹת הַכַּפֹּֽרֶת׃
\rashi{\rashiDH{כרבים. }דמות פרצוף תינוק להם׃}\rashi{\rashiDH{מקשה תעשה. }שלא תעשם בפני עצמם ותחברם בראשי הכפורת לאחר עשייתם, כמעשה צורפים שקורין שולדירי״ץ, אלא הטיל זהב הרבה בתחלת עשיית הכפורת, והכה בפטיש ובקורנס באמצע, וראשין בולטין למעלה, וצייר הכרובים בבליטת קצותיו׃ }\rashi{\rashiDH{מקשה. }בטדי״ץ בלע״ז כמו דָּא לְדָא נָקְשָׁן (דניאל ה, ו)׃ }\rashi{\rashiDH{קצות הכפורת. }ראשי הכפורת׃}}
{וְתַעֲבֵיד תְּרֵין כְּרוּבִין דִּדְהַב נְגִיד תַּעֲבֵיד יָתְהוֹן מִתְּרֵין סִטְרֵי כָּפוּרְתָּא׃}
{And thou shalt make two cherubim of gold; of beaten work shalt thou make them, at the two ends of the ark-cover.}{\arabic{verse}}
\threeverse{\arabic{verse}}%Ex.25:19
{וַ֠עֲשֵׂ֠ה כְּר֨וּב אֶחָ֤ד מִקָּצָה֙ מִזֶּ֔ה וּכְרוּב־אֶחָ֥ד מִקָּצָ֖ה מִזֶּ֑ה מִן־הַכַּפֹּ֛רֶת תַּעֲשׂ֥וּ אֶת־הַכְּרֻבִ֖ים עַל־שְׁנֵ֥י קְצוֹתָֽיו׃
\rashi{\rashiDH{ועשה כרוב אחד מקצה. }שלא תאמר שנים כרובים לכל קצה וקצה, לכך הוצרך לפרש כרוב אחד מקצה מזה׃ 
}\rashi{\rashiDH{מן הכפורת. }עצמה תעשה את הכרובים, זהו פירושו של מקשה תעשה אותם, שלא תעשם בפני עצמם ותחברם לכפרת׃ }}
{וַעֲבֵיד כְּרוּבָא חַד מִסִּטְרָא מִכָּא וּכְרוּבָא חַד מִסִּטְרָא מִכָּא מִן כָּפוּרְתָּא תַּעְבְּדוּן יָת כְּרוּבַיָּא עַל תְּרֵין סִטְרוֹהִי׃}
{And make one cherub at the one end, and one cherub at the other end; of one piece with the ark-cover shall ye make the cherubim of the two ends thereof.}{\arabic{verse}}
\threeverse{\arabic{verse}}%Ex.25:20
{וְהָי֣וּ הַכְּרֻבִים֩ פֹּרְשֵׂ֨י כְנָפַ֜יִם לְמַ֗עְלָה סֹכְכִ֤ים בְּכַנְפֵיהֶם֙ עַל־הַכַּפֹּ֔רֶת וּפְנֵיהֶ֖ם אִ֣ישׁ אֶל־אָחִ֑יו אֶ֨ל־הַכַּפֹּ֔רֶת יִהְי֖וּ פְּנֵ֥י הַכְּרֻבִֽים׃
\rashi{\rashiDH{פורשי כנפים. }שלא תעשה כנפיהם שוכבים, אלא פרושים וגבוהים למעלה אצל ראשיהם, שיהא י׳ טפחים בחלל שבין הכנפים לכפורת, כדאיתא בסוכה (ה׃)׃ 
}}
{וִיהוֹן כְּרוּבַיָּא פְרִיסִין גַּדְפֵּיהוֹן לְעֵילָא מְטַלַּן בְּגַדְפֵּיהוֹן עַל כָּפוּרְתָּא וְאַפֵּיהוֹן חַד לָקֳבֵיל חַד לָקֳבֵיל כָּפוּרְתָּא יְהוֹן אַפֵּי כְרוּבַיָּא׃}
{And the cherubim shall spread out their wings on high, screening the ark-cover with their wings, with their faces one to another; toward the ark-cover shall the faces of the cherubim be.}{\arabic{verse}}
\threeverse{\arabic{verse}}%Ex.25:21
{וְנָתַתָּ֧ אֶת־הַכַּפֹּ֛רֶת עַל־הָאָרֹ֖ן מִלְמָ֑עְלָה וְאֶל־הָ֣אָרֹ֔ן תִּתֵּן֙ אֶת־הָ֣עֵדֻ֔ת אֲשֶׁ֥ר אֶתֵּ֖ן אֵלֶֽיךָ׃
\rashi{\rashiDH{ואל הארון תתן את העדת. }לא ידעתי למה נכפל, שהרי כבר נאמר ונתת אל הארון את העדות, ויש לומר, שבא ללמד שבעודו ארון לבדו בלא כפורת, יתן תחלה העדות לתוכו, ואחר כך יתן את הכפורת עליו, וכן מצינו כשהקים את המשכן, נאמר וַיִּתֵּן אֶת הָעֵדֻת אֶל הָאָרֹן (שמות מ, כ), ואחר כך וַיִּתֵּן אֶת הַכַּפֹּרֶת עַל הָאָרֹן מִלְמָעְלָה׃ }}
{וְתִתֵּין יָת כָּפוּרְתָּא עַל אֲרוֹנָא מִלְּעֵילָא וּבַאֲרוֹנָא תִּתֵּין יָת סָהֲדוּתָא דְּאֶתֵּין לָךְ׃}
{And thou shalt put the ark-cover above upon the ark; and in the ark thou shalt put the testimony that I shall give thee.}{\arabic{verse}}
\threeverse{\arabic{verse}}%Ex.25:22
{וְנוֹעַדְתִּ֣י לְךָ֮ שָׁם֒ וְדִבַּרְתִּ֨י אִתְּךָ֜ מֵעַ֣ל הַכַּפֹּ֗רֶת מִבֵּין֙ שְׁנֵ֣י הַכְּרֻבִ֔ים אֲשֶׁ֖ר עַל־אֲר֣וֹן הָעֵדֻ֑ת אֵ֣ת כׇּל־אֲשֶׁ֧ר אֲצַוֶּ֛ה אוֹתְךָ֖ אֶל־בְּנֵ֥י יִשְׂרָאֵֽל׃ \petucha 
\rashi{\rashiDH{ונועדתי. }כשאקבע מועד לך לדבר עמך, אותו מקום אקבע למועד, שאבא שם לדבר אליך׃ }\rashi{\rashiDH{ודברתי אתך מעל הכפורת. }ובמקום אחר הוא אומר, וַיְדַבֵּר ה׳ אֵלָיו מֵאֹהֶל מֹועֵד לֵאמֹר (ויקרא א, א), זה המשכן מחוץ לפרכת, נמצאו שני כתובים מכחישים זה את זה, בא הכתוב השלישי והכריע ביניהם, וּבְבֹא מֹשֶה אֶל אֹהֶל מֹועֵד וַיִּשְׁמַע אֶת הַקֹּול מִדַּבֵּר אֵלָיו מֵעַל הַכַּפֹּרֶת וגו׳ (במדבר ז, פט), משה היה נכנס למשכן, וכיון שבא בתוך הפתח, קול יורד מן השמים לבין הכרובים, ומשם יוצא ונשמע למשה באהל מועד׃ }\rashi{\rashiDH{ואת כל אשר אצוה אותך אל בני ישראל. }הרי וי״ו זו יתירה וטפלה, וכמוהו הרבה במקרא, וכה תפתר, ואת אשר אדבר עמך שם את כל אשר אצוה אותך, אל בני ישראל הוא׃ }}
{וַאֲזָמֵין מֵימְרִי לָךְ תַּמָּן וַאֲמַלֵּיל עִמָּךְ מֵעִלָּוֵי כָּפוּרְתָּא מִבֵּין תְּרֵין כְּרוּבַיָּא דְּעַל אֲרוֹנָא דְּסָהֲדוּתָא יָת כָּל דַּאֲפַקֵּיד יָתָךְ לְוָת בְּנֵי יִשְׂרָאֵל׃}
{And there I will meet with thee, and I will speak with thee from above the ark-cover, from between the two cherubim which are upon the ark of the testimony, of all things which I will give thee in commandment unto the children of Israel.}{\arabic{verse}}
\threeverse{\arabic{verse}}%Ex.25:23
{וְעָשִׂ֥יתָ שֻׁלְחָ֖ן עֲצֵ֣י שִׁטִּ֑ים אַמָּתַ֤יִם אׇרְכּוֹ֙ וְאַמָּ֣ה רׇחְבּ֔וֹ וְאַמָּ֥ה וָחֵ֖צִי קֹמָתֽוֹ׃
\rashi{\rashiDH{קומתו. }גובה רגליו עם עובי השלחן (פסחים קט׃  ובתוס׳ שם ד״ה אמתא)׃ }}
{וְתַעֲבֵיד פָּתוּרָא דְּאָעֵי שִׁטִּין תַּרְתֵּין אַמִּין אוּרְכֵּיהּ וְאַמְּתָא פוּתְיֵיהּ וְאַמְּתָא וּפַלְגָּא רוּמֵיהּ׃}
{And thou shalt make a table of acacia-wood: two cubits shall be the length thereof, and a cubit the breadth thereof, and a cubit and a half the height thereof.}{\arabic{verse}}
\threeverse{\arabic{verse}}%Ex.25:24
{וְצִפִּיתָ֥ אֹת֖וֹ זָהָ֣ב טָה֑וֹר וְעָשִׂ֥יתָ לּ֛וֹ זֵ֥ר זָהָ֖ב סָבִֽיב׃
\rashi{\rashiDH{זר זהב. }סימן לכתר מלכות, שהשולחן שם עושר וגדולה, כמו שאומרים שלחן מלכים׃ }}
{וְתִחְפֵי יָתֵיהּ דְּהַב דְּכֵי וְתַעֲבֵיד לֵיהּ זִיר דִּדְהַב סְחוֹר סְחוֹר׃}
{And thou shalt overlay it with pure gold, and make thereto a crown of gold round about.}{\arabic{verse}}
\threeverse{\arabic{verse}}%Ex.25:25
{וְעָשִׂ֨יתָ לּ֥וֹ מִסְגֶּ֛רֶת טֹ֖פַח סָבִ֑יב וְעָשִׂ֧יתָ זֵר־זָהָ֛ב לְמִסְגַּרְתּ֖וֹ סָבִֽיב׃
\rashi{\rashiDH{מסגרת. }כתרגומו גְּדַנְפָא, ונחלקו חכמי ישראל בדבר, יש אומרים למעלה היתה סביב לשולחן, כמו לבזבזין שבשפת שולחן שרים, ויש אומרים למטה היתה תקועה, מרגל לרגל בארבע רוחות השולחן, ודף השולחן שוכב על אותה מסגרת׃ }\rashi{\rashiDH{ועשית זר זהב למסגרתו. }הוא זר האמור למעלה, ופירש לך כאן שעל המסגרת היתה׃ }}
{וְתַעֲבֵיד לֵיהּ גְּדָנְפָא רוּמֵיהּ פּוּשְׁכָּא סְחוֹר סְחוֹר וְתַעֲבֵיד זִיר דִּדְהַב לִגְדָנְפֵיהּ סְחוֹר סְחוֹר׃}
{And thou shalt make unto it a border of a handbreadth round about, and thou shalt make a golden crown to the border thereof round about.}{\arabic{verse}}
\threeverse{\arabic{verse}}%Ex.25:26
{וְעָשִׂ֣יתָ לּ֔וֹ אַרְבַּ֖ע טַבְּעֹ֣ת זָהָ֑ב וְנָתַתָּ֙ אֶת־הַטַּבָּעֹ֔ת עַ֚ל אַרְבַּ֣ע הַפֵּאֹ֔ת אֲשֶׁ֖ר לְאַרְבַּ֥ע רַגְלָֽיו׃}
{וְתַעֲבֵיד לֵיהּ אַרְבַּע עִזְקָן דִּדְהַב וְתִתֵּין יָת עִזְקָתָא עַל אַרְבַּע זָוְיָתָא דִּלְאַרְבַּע רַגְלוֹהִי׃}
{And thou shalt make for it four rings of gold, and put the rings in the four corners that are on the four feet thereof.}{\arabic{verse}}
\threeverse{\arabic{verse}}%Ex.25:27
{לְעֻמַּת֙ הַמִּסְגֶּ֔רֶת תִּהְיֶ֖יןָ הַטַּבָּעֹ֑ת לְבָתִּ֣ים לְבַדִּ֔ים לָשֵׂ֖את אֶת־הַשֻּׁלְחָֽן׃
\rashi{\rashiDH{לעמת המסגרת תהיין הטבעות. }ברגלים תקועות כנגד ראשי המסגרת׃}\rashi{\rashiDH{לבתים לבדים. }אותן טבעות יהיו בתים להכניס בהן הבדים׃}\rashi{\rashiDH{לבתים. }לצורך בתים׃  \rashiDH{לבדים. }כתרגומו לְאַתְרָא לַאֲרִיחַיָא׃}}
{לָקֳבֵיל גְּדָנְפָא יִהְוְיָן עִזְקָתָא אַתְרָא לַאֲרִיחַיָּא לְמִטַּל יָת פָּתוּרָא׃}
{Close by the border shall the rings be, for places for the staves to bear the table.}{\arabic{verse}}
\threeverse{\arabic{verse}}%Ex.25:28
{וְעָשִׂ֤יתָ אֶת־הַבַּדִּים֙ עֲצֵ֣י שִׁטִּ֔ים וְצִפִּיתָ֥ אֹתָ֖ם זָהָ֑ב וְנִשָּׂא־בָ֖ם אֶת־הַשֻּׁלְחָֽן׃
\rashi{\rashiDH{ונשא בם. }לשון נפעל, יהיה נשא בם את השלחן׃ }}
{וְתַעֲבֵיד יָת אֲרִיחַיָּא דְּאָעֵי שִׁטִּין וְתִחְפֵי יָתְהוֹן דַּהְבָּא וִיהוֹן נָטְלִין בְּהוֹן יָת פָּתוּרָא׃}
{And thou shalt make the staves of acacia-wood, and overlay them with gold, that the table may be borne with them.}{\arabic{verse}}
\threeverse{\arabic{verse}}%Ex.25:29
{וְעָשִׂ֨יתָ קְּעָרֹתָ֜יו וְכַפֹּתָ֗יו וּקְשׂוֹתָיו֙ וּמְנַקִּיֹּתָ֔יו אֲשֶׁ֥ר יֻסַּ֖ךְ בָּהֵ֑ן זָהָ֥ב טָה֖וֹר תַּעֲשֶׂ֥ה אֹתָֽם׃
\rashi{\rashiDH{ועשית קערותיו וכפותיו. }קערותיו זה הדפוס, שהיה עשוי כדפוס הלחם, והלחם היה עשוי כמין תיבה פרוצה משתי רוחותיה, שולים לו למטה, וקופל מכאן ומכאן כלפי מעלה כמין כותלים, ולכך קרוי לחם הפנים, שיש לו פנים רואין לכאן ולכאן, לצדי הבית מזה ומזה, נותן ארכו לרחבו של שולחן, וכתליו זקופים כנגד שפת השולחן. והיה עשוי לו דפוס זהב ודפוס ברזל, בשל ברזל הוא נאפה, וכשמוציאו מן התנור נותנו בשל זהב עד למחר בשבת שמסדרו על השולחן, ואותו דפוס קרוי קערה׃ }\rashi{\rashiDH{וכפותיו. }הן בזיכין שנותנין בהם לבונה, ושתים היו לשני קומצי לבונה שנותנין על שתי המערכות, שנאמר וְנָתַתָּ עַל הַמַּעֲרֶכֶת לְבֹנָה זַכָּה (ויקרא כד, ז)׃ }\rashi{\rashiDH{וקשותיו. }הן כמין חצאי קנים חלולים הנסדקין לארכן, דוגמתן עשה של זהב, ומסדר ג׳ על ראש כל לחם, שישב לחם האחד על גבי אותן הקנים, ומבדילין בין לחם ללחם, כדי שתכנס הרוח ביניהם ולא יתעפשו, ובלשון ערבי כל דבר חלול קרוי קסו״א׃ }\rashi{\rashiDH{ומנקיותיו. }תרגומו וּמְכִילָתֵיהּ, הן סניפים, כמין יתדות זהב עומדין בארץ, וגבוהים עד למעלה מן השלחן הרבה כנגד גובה מערכת הלחם, ומפוצלים ששה (הרא״ם גורם חמשה) פצולים זה למעלה מזה, וראשי הקנים שבין לחם ללחם סמוכין על אותן פצולין, כדי שלא יכבד משא הלחם העליונים על התחתונים וישברו, ולשון מְכִילָתֵיהּ, סובלותיו, כמו נִלְאֵיתִי הָכִיל (ירמיה ו, יא). אבל לשון מנקיות איני יודע איך נופל על סניפין, ויש מחכמי ישראל אומרים (מנחות צז.), קשותיו אלו סניפין, שמקשין אותו ומחזיקים אותו שלא ישבר. ומנקיותיו. לו הקנים שמנקין אותו שלא יתעפש, אבל אונקלוס שתרגם מְכִילָתֵיהּ, היה שונה כדברי האומר מנקיות הן סניפין׃ }\rashi{\rashiDH{אשר יסך בהן. }אשר יכוסה בהן, ועל קשותיו הוא אומר אשר יוסך, שהיו עליו כמין סכך וכסוי, וכן במקום אחר הוא אומר וְאֵת קְשׂוֹת הַנָּסֶךְ (במדבר ד, ז), וזה וזה, יוסך והנסך, לשון סכך וכסוי הם׃ }}
{וְתַעֲבֵיד מְגִסּוֹהִי וּבָזִכּוֹהִי וְקָסְוָתֵיהּ וּמְכִילָתֵיהּ דְּיִתְנַסַּךְ בְּהוֹן דִּדְהַב דְּכֵי תַּעֲבֵיד יָתְהוֹן׃}
{And thou shalt make the dishes thereof, and the pans thereof, and the jars thereof, and the bowls thereof, wherewith to pour out; of pure gold shalt thou make them.}{\arabic{verse}}
\threeverse{\arabic{verse}}%Ex.25:30
{וְנָתַתָּ֧ עַֽל־הַשֻּׁלְחָ֛ן לֶ֥חֶם פָּנִ֖ים לְפָנַ֥י תָּמִֽיד׃ \petucha 
\rashi{\rashiDH{לחם פנים. }שיש לו פנים, כמו שפירשתי, ומנין הלחם וסדר מערכותיו, מפורשים באמור אל הכהנים׃ }}
{וְתִתֵּין עַל פָּתוּרָא לְחֵים אַפַּיָּא קֳדָמַי תְּדִירָא׃}
{And thou shalt set upon the table showbread before Me always.}{\arabic{verse}}
\threeverse{\aliya{שלישי}}%Ex.25:31
{וְעָשִׂ֥יתָ מְנֹרַ֖ת זָהָ֣ב טָה֑וֹר מִקְשָׁ֞ה תֵּעָשֶׂ֤ה\note{בספרי ספרד ואשכנז תֵּיעָשֶׂ֤ה} הַמְּנוֹרָה֙ יְרֵכָ֣הּ וְקָנָ֔הּ גְּבִיעֶ֛יהָ כַּפְתֹּרֶ֥יהָ וּפְרָחֶ֖יהָ מִמֶּ֥נָּה יִהְיֽוּ׃
\rashi{\rashiDH{מקשה תיעשה המנורה. }שלא יעשנה חוליות, ולא יעשה קניה ונרותיה איברים איברים, ואחר כך ידביקם כדרך הצורפים שקורין שולדיר״ץ, אלא כולה באה מחתיכה אחת, ומקיש בקורנס וחותך בכלי האומנות, ומפריד הקנים אילך ואילך׃ }\rashi{\rashiDH{מקשה. }תרגומו נְגִיד, לשון המשכה, שממשיך את האיברים מן העשת לכאן ולכאן בהקשת הקורנס, ולשון מקשה מכת קורנס, בטדי״ץ בלע״ז כמו דָּא לְדָא נָקְשָׁן (דניאל ה, ו)׃ }\rashi{\rashiDH{תיעשה המנורה. }מאליה (תנחומא בהעלותך ג), לפי שהיה משה מתקשה בה (מנחות כט.), אמר לו הקב״ה, השלך את הככר לאור והיא נעשית מאליה, לכך לא נכתב תעשה (תנחומא שם)׃ }\rashi{\rashiDH{ירכה. }הוא הרגל של מטה העשוי כמין תיבה, ושלשה רגלים יוצאין הימנה ולמטה׃ }\rashi{\rashiDH{וקנה. }הקנה האמצעי שלה העולה באמצע הירך, זקוף כלפי מעלה, ועליו נר האמצעי עשוי כמין בזך, לצוק השמן לתוכו ולתת הפתילה׃ }\rashi{\rashiDH{גביעיה. }הן כמין כוסות שעושין מזכוכית, ארוכים וקצרים, וקורין להם מדירנ״ס, ואלו עשויין מזהב, ובולטין ויוצאין מכל קנה וקנה כמנין שנתן בהם הכתוב, ולא היו בה אלא לנוי׃ }\rashi{\rashiDH{כפתריה. }כמין תפוחים היו, עגולין סביב, בולטין סביבות הקנה האמצעי, כדרך שעושין למנורות שלפני השרים, וקורין להם פימל״ש, ומנין שלהם כתוב בפרשה כמה כפתורים בולטין ממנה וכמה חלק שבין כפתור לכפתור׃ }\rashi{\rashiDH{ופרחיה. }ציורין עשוין בה כמין פרחין׃}\rashi{\rashiDH{ממנה יהיו. }הכל מקשה יוצא מתוך חתיכת העשת, ולא יעשם לבדם וידביקם׃ }}
{וְתַעֲבֵיד מְנָרְתָא דִּדְהַב דְּכֵי נְגִיד תִּתְעֲבֵיד מְנָרְתָא שִׁדַּהּ וּקְנַהּ כַּלִּידַהָא חַזּוּרַהָא וְשׁוֹשַׁנַּהָא מִנַּהּ יְהוֹן׃}
{And thou shalt make a candlestick of pure gold: of beaten work shall the candlestick be made, even its base, and its shaft; its cups, its knops, and its flowers, shall be of one piece with it.}{\arabic{verse}}
\threeverse{\arabic{verse}}%Ex.25:32
{וְשִׁשָּׁ֣ה קָנִ֔ים יֹצְאִ֖ים מִצִּדֶּ֑יהָ שְׁלֹשָׁ֣ה \legarmeh  קְנֵ֣י מְנֹרָ֗ה מִצִּדָּהּ֙ הָאֶחָ֔ד וּשְׁלֹשָׁה֙ קְנֵ֣י מְנֹרָ֔ה מִצִּדָּ֖הּ הַשֵּׁנִֽי׃
\rashi{\rashiDH{יוצאים מצדיה. }לכאן ולכאן, באלכסון נמשכים ועולין עד כנגד גובהה של מנורה, שהוא קנה האמצעי, ויוצאין מתוך קנה האמצעי זה למעלה מזה, התחתון ארוך, ושל מעלה קצר הימנו, והעליון קצר הימנו, לפי שהיה גובה ראשיהן שוה לגובהו של קנה האמצעי השביעי שממנו יוצאים הששה קנים׃ 
}}
{וְשִׁתָּא קְנִין נָפְקִין מִסִּטְרַהָא תְּלָתָא קְנֵי מְנָרְתָא מִסִּטְרַהּ חַד וּתְלָתָא קְנֵי מְנָרְתָא מִסִּטְרַהּ תִּנְיָנָא׃}
{And there shall be six branches going out of the sides thereof: three branches of the candlestick out of the one side thereof, and three branches of the candle-stick out of the other side thereof;}{\arabic{verse}}
\threeverse{\arabic{verse}}%Ex.25:33
{שְׁלֹשָׁ֣ה גְ֠בִעִ֠ים מְֽשֻׁקָּדִ֞ים בַּקָּנֶ֣ה הָאֶחָד֮ כַּפְתֹּ֣ר וָפֶ֒רַח֒ וּשְׁלֹשָׁ֣ה גְבִעִ֗ים מְשֻׁקָּדִ֛ים בַּקָּנֶ֥ה הָאֶחָ֖ד כַּפְתֹּ֣ר וָפָ֑רַח כֵּ֚ן לְשֵׁ֣שֶׁת הַקָּנִ֔ים הַיֹּצְאִ֖ים מִן־הַמְּנֹרָֽה׃
\rashi{\rashiDH{משקדים. }כתרגומו, מצויירים היו, כדרך שעושין לכלי כסף וזהב שקורין ניאל״ר׃ }\rashi{\rashiDH{ושלשה גבעים. }בולטין מכל קנה וקנה׃}\rashi{\rashiDH{כפתור ופרח. }היה לכל קנה וקנה׃}}
{תְּלָתָא כַלִּידִין מְצָיְרִין בְּקַנְיָא חַד חַזּוּר וְשׁוֹשָׁן וּתְלָתָא כַלִּידִין מְצָיְרִין בְּקַנְיָא חַד חַזּוּר וְשׁוֹשָׁן כֵּן לְשִׁתָּא קְנִין דְּנָפְקִין מִן מְנָרְתָא׃}
{three cups made like almond-blossoms in one branch, a knop and a flower; and three cups made like almond-blossoms in the other branch, a knop and a flower; so for the six branches going out of the candlestick.}{\arabic{verse}}
\threeverse{\arabic{verse}}%Ex.25:34
{וּבַמְּנֹרָ֖ה אַרְבָּעָ֣ה גְבִעִ֑ים מְשֻׁ֨קָּדִ֔ים כַּפְתֹּרֶ֖יהָ וּפְרָחֶֽיהָ׃
\rashi{\rashiDH{ובמנרה ארבעה גבעים. }בגופה של מנורה היו ארבעה גביעים, אחד בולט בה למטה מן הקנים, והג׳ למעלה מן יציאת הקנים היוצאין מצדיה׃ }\rashi{\rashiDH{משקדים כפתוריה ופרחיה. }זה אחד מחמשה מקראות שאין להם הכרע (יומא נב׃), אין ידוע אם גביעים משוקדים, או משוקדים כפתוריה ופרחיה׃ }}
{וּבִמְנָרְתָא אַרְבְּעָא כַלִּידִין מְצָיְרִין חַזּוּרַהָא וְשׁוֹשַׁנַּהָא׃}
{And in the candlestick four cups made like almond-blossoms, the knops thereof, and the flowers thereof.}{\arabic{verse}}
\threeverse{\arabic{verse}}%Ex.25:35
{וְכַפְתֹּ֡ר תַּ֩חַת֩ שְׁנֵ֨י הַקָּנִ֜ים מִמֶּ֗נָּה וְכַפְתֹּר֙ תַּ֣חַת שְׁנֵ֤י הַקָּנִים֙ מִמֶּ֔נָּה וְכַפְתֹּ֕ר תַּחַת־שְׁנֵ֥י הַקָּנִ֖ים מִמֶּ֑נָּה לְשֵׁ֙שֶׁת֙ הַקָּנִ֔ים הַיֹּצְאִ֖ים מִן־הַמְּנֹרָֽה׃
\rashi{\rashiDH{וכפתור תחת שני הקנים. }מתוך הכפתור היו הקנים נמשכים משני צדיה אילך ואילך. כך שנינו במלאכת המשכן (מנחות כח׃), גובהה של מנורה י״ח טפחים, הרגלים והפרח ג׳ טפחים, הוא הפרח האמור בירך, שנאמר עַד יְרֵכָהּ עַד פִּרְחָהּ (במדבר ח, ד), וטפחיים חלק, וטפח שבו גביע מהארבעה גביעים, וכפתור ופרח משני כפתורים ושני פרחים האמורים במנורה עצמה, שנאמר משוקדים כפתוריה ופרחיה, למדנו שהיו בקנה שני כפתורים ושני פרחים לבד מן הג׳ כפתורים שהקנים נמשכין מתוכן, שנאמר וכפתור תחת שני הקנים וגו׳, וטפחיים חלק, וטפח כפתור ושני קנים יוצאים ממנו אילך ואילך נמשכים ועולים כנגד גובהה של מנורה, טפח חלק, וטפח כפתור ושני קנים יוצאים ממנו, וטפח חלק, וטפח כפתור ושני קנים יוצאים ממנו ונמשכים ועולין כנגד גובהה של מנורה, וטפחיים חלק, נשתיירו שם ג׳ טפחים, שבהם ג׳ גביעים וכפתור ופרח, נמצאו גביעים כ״ב, י״ח לששה קנים ג׳ לכל אחד ואחד, וארבעה בגופה של מנורה הרי כ״ב, ואחד עשר כפתורים, ו׳ בששת הקנים, וג׳ בגופה של מנורה שהקנים יוצאים מהם, ושנים עוד נאמרו במנורה, שנאמר משוקדים כפתוריה, ומיעוט כפתורים שנים, האחד למטה אצל הירך, והאחד בג׳ טפחים העליונים עם ג׳ הגביעים, ותשעה פרחים היו לה, ו׳ לששת הקנים, שנאמר בקנה האחד כפתור ופרח, וג׳ למנורה, שנאמר משוקדים כפתוריה ופרחיה, ומיעוט פרחים שנים, ואחד האמור בפרשת בהעלותך, עד ירכה עד פרחה. ואם תדקדק במשנה זו הכתובה למעלה, תמצאם כמנינם איש איש במקומו׃ 
}}
{וְחַזּוּר תְּחוֹת תְּרֵין קְנִין דְּמִנַּהּ וְחַזּוּר תְּחוֹת תְּרֵין קְנִין דְּמִנַּהּ וְחַזּוּר תְּחוֹת תְּרֵין קְנִין דְּמִנַּהּ לְשִׁתָּא קְנִין דְּנָפְקִין מִן מְנָרְתָא׃}
{And a knop under two branches of one piece with it, and a knop under two branches of one piece with it, and a knop under two branches of one piece with it, for the six branches going out of the candlestick.}{\arabic{verse}}
\threeverse{\arabic{verse}}%Ex.25:36
{כַּפְתֹּרֵיהֶ֥ם וּקְנֹתָ֖ם מִמֶּ֣נָּה יִהְי֑וּ כֻּלָּ֛הּ מִקְשָׁ֥ה אַחַ֖ת זָהָ֥ב טָהֽוֹר׃}
{חַזּוּרֵיהוֹן וּקְנֵיהוֹן מִנַּהּ יְהוֹן כּוּלַּהּ נְגִידָא חֲדָא דִּדְהַב דְּכֵי׃}
{Their knops and their branches shall be of one piece with it; the whole of it one beaten work of pure gold.}{\arabic{verse}}
\threeverse{\arabic{verse}}%Ex.25:37
{וְעָשִׂ֥יתָ אֶת־נֵרֹתֶ֖יהָ שִׁבְעָ֑ה וְהֶֽעֱלָה֙ אֶת־נֵ֣רֹתֶ֔יהָ וְהֵאִ֖יר עַל־עֵ֥בֶר פָּנֶֽיהָ׃
\rashi{\rashiDH{את נרותיה. }כמין בזיכין שנותנין בתוכן השמן והפתילות׃}\rashi{\rashiDH{והאיר על עבר פניה. }עשה פי ששת הנרות שבראשי הקנים היוצאים מצדיה, מוסבים כלפי האמצעי, כדי שיהיו הנרות כשתדליקם מאירים על עבר פניה, מוסב אורם אל צד פני הקנה האמצעי שהוא גוף המנורה׃ }}
{וְתַעֲבֵיד יָת בּוֹצִינַהָא שִׁבְעָא וְתַדְלֵיק יָת בּוֹצִינַהָא וִיהוֹן מְנָהֲרִין לָקֳבֵיל אַפַּהָא׃}
{And thou shalt make the lamps thereof, seven; and they shall light the lamps thereof, to give light over against it.}{\arabic{verse}}
\threeverse{\arabic{verse}}%Ex.25:38
{וּמַלְקָחֶ֥יהָ וּמַחְתֹּתֶ֖יהָ זָהָ֥ב טָהֽוֹר׃
\rashi{\rashiDH{ומלקחיה. }הם הצבתים העשויין ליקח בהם הפתילה מתוך השמן, לישבן ולמושכן בפי הנרות, ועל שם שלוקחים בהם קרויים מלקחים. וְצִבְיְתָהָא שתרגם אונקלוס, לשון צבת, טוליי״ש בלע״ז }\rashi{\rashiDH{ומחתותיה. }הם כמין בזיכין קטנים, שחותה בהן את האפר שבנר בבקר בבקר, כשהוא מטיב את הנרות מאפר הפתילות שדלקו הלילה וכבו, ולשון מחתה פויישד״א בלע״ז כמו לַחְתֹּות אֵשׁ מִיָּקוּד (ישעיה ל, יד)׃ }}
{וְצֵיבְתַהָא וּמַחְתְּיָתַהָא דִּדְהַב דְּכֵי׃}
{And the tongs thereof, and the snuffdishes thereof, shall be of pure gold.}{\arabic{verse}}
\threeverse{\arabic{verse}}%Ex.25:39
{כִּכָּ֛ר זָהָ֥ב טָה֖וֹר יַעֲשֶׂ֣ה אֹתָ֑הּ אֵ֥ת כׇּל־הַכֵּלִ֖ים הָאֵֽלֶּה׃
\rashi{\rashiDH{ככר זהב טהור. }שלא יהיה משקלה עם כל כליה אלא ככר, לא פחות ולא יותר, והככר של חול ששים מנה, ושל קדש היה כפול, ק״ך מנה, והמנה הוא ליטרא ששוקלין בה כסף למשקל קולוני״א, והם ק׳ זהובים, כ״ה סלעים, והסלע ארבעה זהובים׃ }}
{כַּכְּרָא דְּדַהְבָּא דָּכְיָא יַעֲבֵיד יָתַהּ יָת כָּל מָנַיָּא הָאִלֵּין׃}
{Of a talent of pure gold shall it be made, with all these vessels.}{\arabic{verse}}
\threeverse{\arabic{verse}}%Ex.25:40
{וּרְאֵ֖ה וַעֲשֵׂ֑ה בְּתַ֨בְנִיתָ֔ם אֲשֶׁר־אַתָּ֥ה מׇרְאֶ֖ה בָּהָֽר׃ \setuma         
\rashi{\rashiDH{וראה ועשה. }ראה כאן בהר תבנית שאני מראה אותך. מגיד שנתקשה משה במעשה המנורה, עד שהראה לו הקב״ה מנורה של אש׃ 
}\rashi{\rashiDH{אשר אתה מראה. }כתרגומו דְּאַתְּ מִתְחֲזֵי בְּטוּרָא, אילו היה נקוד מראה בפת״ח, היה פתרונו אתה מראה לאחרים, עכשיו שנקוד חטף קמץ, פתרונו דאת מתחזי, שאחרים מראים לך (שהנקוד מפריד בין עושה לנעשה)׃ }}
{וַחְזִי וַעֲבֵיד בִּדְמוּתְהוֹן דְּאַתְּ מִתַּחְזֵי בְּטוּרָא׃}
{And see that thou make them after their pattern, which is being shown thee in the mount.}{\arabic{verse}}
\newperek
\threeverse{\Roman{chap}}%Ex.26:1
{וְאֶת־הַמִּשְׁכָּ֥ן תַּעֲשֶׂ֖ה עֶ֣שֶׂר יְרִיעֹ֑ת שֵׁ֣שׁ מׇשְׁזָ֗ר וּתְכֵ֤לֶת וְאַרְגָּמָן֙ וְתֹלַ֣עַת שָׁנִ֔י כְּרֻבִ֛ים מַעֲשֵׂ֥ה חֹשֵׁ֖ב תַּעֲשֶׂ֥ה אֹתָֽם׃
\rashi{\rashiDH{ואת המשכן תעשה עשר יריעות. }להיות לו לגג, ולמחיצות מחוץ לקרשים, שהיריעות תלויות מאחוריהן לכסותן׃ }\rashi{\rashiDH{שש משזר ותכלת וארגמן ותולעת שני. }הרי ארבע מינין יחד בכל חוט וחוט, אחד של פשתים, וג׳ של צמר, וכל חוט וחוט כפול ו׳, הרי ד׳ מינין כשהן שזורין יחד כ״ד כפלים לחוט (ברייתא דמלאכת המשכן)׃ }\rashi{\rashiDH{כרובים מעשה חשב. }כרובים היו מצויירין בהם באריגתן, ולא ברקימה שהוא מעשה מחט, אלא באריגה בשני כותלים, פרצוף אחד מכאן ופרצוף אחד מכאן, ארי מצד זה ונשר מצד זה, כמו שאורגין חגורות של משי שקורין בלע״ז פיישיש״א }}
{וְיָת מַשְׁכְּנָא תַּעֲבֵיד עֲשַׂר יְרִיעָן דְּבוּץ שְׁזִיר וְתַכְלָא וְאַרְגְּוָנָא וּצְבַע זְהוֹרִי צוּרַת כְּרוּבִין עוֹבָד אוּמָּן תַּעֲבֵיד יָתְהוֹן׃}
{Moreover thou shalt make the tabernacle with ten curtains: of fine twined linen, and blue, and purple, and scarlet, with cherubim the work of the skilful workman shalt thou make them.}{\Roman{chap}}
\threeverse{\arabic{verse}}%Ex.26:2
{אֹ֣רֶךְ \legarmeh  הַיְרִיעָ֣ה הָֽאַחַ֗ת שְׁמֹנֶ֤ה וְעֶשְׂרִים֙ בָּֽאַמָּ֔ה וְרֹ֙חַב֙ אַרְבַּ֣ע בָּאַמָּ֔ה הַיְרִיעָ֖ה הָאֶחָ֑ת מִדָּ֥ה אַחַ֖ת לְכׇל־הַיְרִיעֹֽת׃}
{אוּרְכָּא דִּירִיעֲתָא חֲדָא עֶשְׂרִין וְתַמְנֵי אַמִּין וּפוּתְיָא אַרְבַּע אַמִּין דִּירִיעֲתָא חֲדָא מִשְׁחֲתָא חֲדָא לְכָל יְרִיעָתָא׃}
{The length of each curtain shall be eight and twenty cubits, and the breadth of each curtain four cubits; all the curtains shall have one measure.}{\arabic{verse}}
\threeverse{\arabic{verse}}%Ex.26:3
{חֲמֵ֣שׁ הַיְרִיעֹ֗ת תִּֽהְיֶ֙יןָ֙ חֹֽבְרֹ֔ת אִשָּׁ֖ה אֶל־אֲחֹתָ֑הּ וְחָמֵ֤שׁ יְרִיעֹת֙ חֹֽבְרֹ֔ת אִשָּׁ֖ה אֶל־אֲחֹתָֽהּ׃
\rashi{\rashiDH{תהיין חוברות. }תופרן במחט זו בצד זו, חמש לבד וחמש לבד׃ }\rashi{\rashiDH{אשה אל אחותה. }כך דרך המקרא לדבר בדבר שהוא לשון נקבה, ובדבר שהוא לשון זכר אומר איש אל אחיו כמו שנאמר בכרובים, וּפְנֵיהֶם אִיש אֶל אָחִיו (שמות כה, כ)׃ }}
{חֲמֵישׁ יְרִיעָן יִהְוְיָן מְלָפְפָן חֲדָא עִם חֲדָא וַחֲמֵישׁ יְרִיעָן מְלָפְפָן חֲדָא עִם חֲדָא׃}
{Five curtains shall be coupled together one to another; and the other five curtains shall be coupled one to another.}{\arabic{verse}}
\threeverse{\arabic{verse}}%Ex.26:4
{וְעָשִׂ֜יתָ לֻֽלְאֹ֣ת תְּכֵ֗לֶת עַ֣ל שְׂפַ֤ת הַיְרִיעָה֙ הָאֶחָ֔ת מִקָּצָ֖ה בַּחֹבָ֑רֶת וְכֵ֤ן תַּעֲשֶׂה֙ בִּשְׂפַ֣ת הַיְרִיעָ֔ה הַקִּ֣יצוֹנָ֔ה בַּמַּחְבֶּ֖רֶת הַשֵּׁנִֽית׃
\rashi{\rashiDH{לולאות. }לצו״לש בלע״ז וכן תרגם אונקלוס עֲנוּבִין, לשון עניבה׃ }\rashi{\rashiDH{מקצה בחוברת. }באותה יריעה שבסוף החבור. קבוצת חמשת היריעות קרויה חוברת׃}\rashi{\rashiDH{וכן תעשה בשפת היריעה הקיצונה במחברת השנית. }באותה יריעה שהיא קיצונה, לשון קצה, כלומר לסוף החוברת׃ }}
{וְתַעֲבֵיד עֲנוּבִּין דְּתַכְלָא עַל סִפְתָּא דִּירִיעֲתָא חֲדָא מִסִּטְרָא בֵּית לוֹפֵי וְכֵן תַּעֲבֵיד בְּסִפְתָּא דִּירִיעֲתָא בְּסִטְרָא בֵּית לוֹפֵי תִּנְיָנָא׃}
{And thou shalt make loops of blue upon the edge of the one curtain that is outmost in the first set; and likewise shalt thou make in the edge of the curtain that is outmost in the second set.}{\arabic{verse}}
\threeverse{\arabic{verse}}%Ex.26:5
{חֲמִשִּׁ֣ים לֻֽלָאֹ֗ת תַּעֲשֶׂה֮ בַּיְרִיעָ֣ה הָאֶחָת֒ וַחֲמִשִּׁ֣ים לֻֽלָאֹ֗ת תַּעֲשֶׂה֙ בִּקְצֵ֣ה הַיְרִיעָ֔ה אֲשֶׁ֖ר בַּמַּחְבֶּ֣רֶת הַשֵּׁנִ֑ית מַקְבִּילֹת֙ הַלֻּ֣לָאֹ֔ת אִשָּׁ֖ה אֶל־אֲחֹתָֽהּ׃
\rashi{\rashiDH{מקבילות הלולאות אשה אל אחותה. }שמור שתעשה הלולאות במדה אחת, מכוונות הבדלתן זו מזו, וכמדתן ביריעה זו כן יהא בחברתה, כשתפרוש חוברת אצל חוברת יהיו הלולאות של יריעה זו מכוונות כנגד לולאות של זו, וזהו לשון מקבילות, זו כנגד זו, תרגומו של כנגד, לקבל. היריעות ארכן כ״ח ורחבן ארבע, וכשחבר חמש יריעות יחד נמצא רחבן כ׳, וכן החוברת השנית. והמשכן ארכו שלשים מן המזרח למערב, שנאמר עשרים קרשים לפאת נגבה תימנה, וכן לצפון, וכל קרש אמה וחצי האמה, הרי שלשים מן המזרח למערב. רוחב המשכן מן הצפון לדרום עשר אמות, שנאמר ולירכתי המשכן ימה וגו׳, ושני קרשים למקצעות הרי עשר, ובמקומם אפרשם למקראות הללו. נותן היריעות ארכן לרחבו של משכן, עשר אמות אמצעיות לגג חלל רוחב המשכן, ואמה מכאן ואמה מכאן לעובי ראשי הקרשים שעוביים אמה, נשתיירו ט״ז אמה, ח׳ לצפון וח׳ לדרום מכסות קומת הקרשים שגבהן עשר, נמצאו שתי אמות התחתונות מגולות. רחבן של יריעות ארבעים אמה כשהן מחוברות, עשרים אמה לחוברת, שלשים מהן לגג חלל המשכן לארכו, ואמה כנגד עובי ראשי הקרשים שבמערב, ואמה לכסות עובי העמודים שבמזרח, שלא היו קרשים במזרח אלא ד׳ (ברש״י ישן חמשה) עמודים, שהמסך פרוש ותלוי בווין שבהן כמין וילון, נשתיירו ח׳ אמות התלויין על אחורי המשכן שבמערב, ושתי אמות התחתונות מגולות. זו מצאתי בברייתא דמסכת מדות, אבל במסכת שבת (פרק הזורק צח׃), אין היריעות מכסות את עמודי המזרח, וט׳ אמות תלויות אחורי המשכן, והכתוב בפרשה זו מסייענו, ונתת את הפרוכת תחת הקרסים, ואם כדברי הברייתא הזאת, נמצאת פרוכת משוכה מן הקרסים ולמערב אמה׃ 
}}
{חַמְשִׁין עֲנוּבִּין תַּעֲבֵיד בִּירִיעֲתָא חֲדָא וְחַמְשִׁין עֲנוּבִּין תַּעֲבֵיד בְּסִטְרָא דִּירִיעֲתָא דְּבֵית לוֹפֵי תִּנְיָנָא מַכְוְנָן עֲנוּבַּיָּא חֲדָא לָקֳבֵיל חֲדָא׃}
{Fifty loops shalt thou make in the one curtain, and fifty loops shalt thou make in the edge of the curtain that is in the second set; the loops shall be opposite one to another.}{\arabic{verse}}
\threeverse{\arabic{verse}}%Ex.26:6
{וְעָשִׂ֕יתָ חֲמִשִּׁ֖ים קַרְסֵ֣י זָהָ֑ב וְחִבַּרְתָּ֨ אֶת־הַיְרִיעֹ֜ת אִשָּׁ֤ה אֶל־אֲחֹתָהּ֙ בַּקְּרָסִ֔ים וְהָיָ֥ה הַמִּשְׁכָּ֖ן אֶחָֽד׃
\rashi{\rashiDH{קרסי זהב. }פירמיל״ש בלע״ז ומכניסין ראשן אחד בלולאות שבחוברת זו, וראשן אחד בלולאות שבחוברת זו, ומחברן בהן׃ }}
{וְתַעֲבֵיד חַמְשִׁין פּוּרְפִין דִּדְהַב וּתְלָפֵיף יָת יְרִיעָתָא חֲדָא עִם חֲדָא בְּפוּרְפַיָּא וִיהֵי מַשְׁכְּנָא חַד׃}
{And thou shalt make fifty clasps of gold, and couple the curtains one to another with the clasps, that the tabernacle may be one whole.}{\arabic{verse}}
\threeverse{\arabic{verse}}%Ex.26:7
{וְעָשִׂ֙יתָ֙ יְרִיעֹ֣ת עִזִּ֔ים לְאֹ֖הֶל עַל־הַמִּשְׁכָּ֑ן עַשְׁתֵּי־עֶשְׂרֵ֥ה יְרִיעֹ֖ת תַּעֲשֶׂ֥ה אֹתָֽם׃
\rashi{\rashiDH{יריעות עזים. }מנוצה של עזים׃}\rashi{\rashiDH{לאהל על המשכן. }לפרוש אותן על היריעות התחתונות׃}}
{וְתַעֲבֵיד יְרִיעָן דְּמַעְזֵי לְפָרָסָא עַל מַשְׁכְּנָא חֲדָא עֶשְׂרֵי יְרִיעָן תַּעֲבֵיד יָתְהוֹן׃}
{And thou shalt make curtains of goats’ hair for a tent over the tabernacle; eleven curtains shalt thou make them.}{\arabic{verse}}
\threeverse{\arabic{verse}}%Ex.26:8
{אֹ֣רֶךְ \legarmeh  הַיְרִיעָ֣ה הָֽאַחַ֗ת שְׁלֹשִׁים֙ בָּֽאַמָּ֔ה וְרֹ֙חַב֙ אַרְבַּ֣ע בָּאַמָּ֔ה הַיְרִיעָ֖ה הָאֶחָ֑ת מִדָּ֣ה אַחַ֔ת לְעַשְׁתֵּ֥י עֶשְׂרֵ֖ה יְרִיעֹֽת׃
\rashi{\rashiDH{שלשים באמה. }שכשנותן ארכן לרוחב המשכן כמו שנתן את הראשונות, נמצאו אלו עודפות אמה מכאן ואמה מכאן, לכסות אחת מהשתי אמות שנשארו מגולות מן הקרשים, והאמה התחתונה של קרש שאין היריעה מכסה אותו, היא האמה התחובה בנקב האדן, שהאדנים גבהן אמה׃ 
}}
{אוּרְכָּא דִּירִיעֲתָא חֲדָא תְּלָתִין אַמִּין וּפוּתְיָא אַרְבַּע אַמִּין דִּירִיעֲתָא חֲדָא מִשְׁחֲתָא חֲדָא לַחֲדָא עֶשְׂרֵי יְרִיעָן׃}
{The length of each curtain shall be thirty cubits, and the breadth of each curtain four cubits; the eleven curtains shall have one measure.}{\arabic{verse}}
\threeverse{\arabic{verse}}%Ex.26:9
{וְחִבַּרְתָּ֞ אֶת־חֲמֵ֤שׁ הַיְרִיעֹת֙ לְבָ֔ד וְאֶת־שֵׁ֥שׁ הַיְרִיעֹ֖ת לְבָ֑ד וְכָפַלְתָּ֙ אֶת־הַיְרִיעָ֣ה הַשִּׁשִּׁ֔ית אֶל־מ֖וּל פְּנֵ֥י הָאֹֽהֶל׃
\rashi{\rashiDH{וכפלת את היריעה הששית. }העודפת באלו העליונות יותר מן התחתונות׃}\rashi{\rashiDH{אל מול פני האהל. }חצי רחבה היה תלוי, וכפול על המסך שבמזרח כנגד הפתח, דומה לכלה צנועה המכוסה בצעיף על פניה׃ }}
{וּתְלָפֵיף יָת חֲמֵישׁ יְרִיעָן לְחוֹד וְיָת שֵׁית יְרִיעָן לְחוֹד וְתֵיעוֹף יָת יְרִיעֲתָא שְׁתִיתֵיתָא לָקֳבֵיל אַפֵּי מַשְׁכְּנָא׃}
{And thou shalt couple five curtains by themselves, and six curtains by themselves, and shalt double over the sixth curtain in the forefront of the tent.}{\arabic{verse}}
\threeverse{\arabic{verse}}%Ex.26:10
{וְעָשִׂ֜יתָ חֲמִשִּׁ֣ים לֻֽלָאֹ֗ת עַ֣ל שְׂפַ֤ת הַיְרִיעָה֙ הָֽאֶחָ֔ת הַקִּיצֹנָ֖ה בַּחֹבָ֑רֶת וַחֲמִשִּׁ֣ים לֻֽלָאֹ֗ת עַ֚ל שְׂפַ֣ת הַיְרִיעָ֔ה הַחֹבֶ֖רֶת הַשֵּׁנִֽית׃}
{וְתַעֲבֵיד חַמְשִׁין עֲנוּבִּין עַל סִפְתָּא דִּירִיעֲתָא חֲדָא בְּסִטְרָא בֵּית לוֹפֵי וְחַמְשִׁין עֲנוּבִּין עַל סִפְתָּא דִּירִיעֲתָא דְּבֵית לוֹפֵי תִּנְיָנָא׃}
{And thou shalt make fifty loops on the edge of the one curtain that is outmost in the first set, and fifty loops upon the edge of the curtain which is outmost in the second set.}{\arabic{verse}}
\threeverse{\arabic{verse}}%Ex.26:11
{וְעָשִׂ֛יתָ קַרְסֵ֥י נְחֹ֖שֶׁת חֲמִשִּׁ֑ים וְהֵבֵאתָ֤ אֶת־הַקְּרָסִים֙ בַּלֻּ֣לָאֹ֔ת וְחִבַּרְתָּ֥ אֶת־הָאֹ֖הֶל וְהָיָ֥ה אֶחָֽד׃}
{וְתַעֲבֵיד פּוּרְפִין דִּנְחָשׁ חַמְשִׁין וְתַעֵיל יָת פּוּרְפַיָּא בַּעֲנוּבַּיָּא וּתְלָפֵיף יָת מַשְׁכְּנָא וִיהֵי חַד׃}
{And thou shalt make fifty clasps of brass, and put the clasps into the loops, and couple the tent together, that it may be one.}{\arabic{verse}}
\threeverse{\arabic{verse}}%Ex.26:12
{וְסֶ֙רַח֙ הָעֹדֵ֔ף בִּירִיעֹ֖ת הָאֹ֑הֶל חֲצִ֤י הַיְרִיעָה֙ הָעֹדֶ֔פֶת תִּסְרַ֕ח עַ֖ל אֲחֹרֵ֥י הַמִּשְׁכָּֽן׃
\rashi{\rashiDH{וסרח העודף ביריעות האהל. }על יריעות המשכן. יריעות האהל הן העליונות של עזים, שקרוים אהל, כמו שנאמר בהן לאהל על המשכן, וכל אהל האמור בהן אינו אלא לשון גג, שמאהילות ומסככות על התחתונות, והן היו עודפות על התחתונות חצי היריעה למערב, שהחצי של יריעה אחת עשרה היתירה, היה נכפל אל מול פני האהל, נשארו שתי אמות רוחב חציה, עודף על רוחב התחתונות׃ }\rashi{\rashiDH{תסרח על אחורי המשכן. }לכסות ב׳ אמות שהיו מגולות בקרשים׃}\rashi{\rashiDH{אחורי המשכן. }הוא צד מערבי, לפי שהפתח במזרח שהן פניו, וצפון ודרום קרויין צדדין, לימין ולשמאל׃ }}
{וְסֻרְחָא דְּיַתִּיר בִּירִיעָת מַשְׁכְּנָא פַּלְגוּת יְרִיעֲתָא דְּיָתְרָא תִּסְרַח עַל אֲחוֹרֵי מַשְׁכְּנָא׃}
{And as for the overhanging part that remaineth of the curtains of the tent, the half curtain that remaineth over shall hang over the back of the tabernacle.}{\arabic{verse}}
\threeverse{\arabic{verse}}%Ex.26:13
{וְהָאַמָּ֨ה מִזֶּ֜ה וְהָאַמָּ֤ה מִזֶּה֙ בָּעֹדֵ֔ף בְּאֹ֖רֶךְ יְרִיעֹ֣ת הָאֹ֑הֶל יִהְיֶ֨ה סָר֜וּחַ עַל־צִדֵּ֧י הַמִּשְׁכָּ֛ן מִזֶּ֥ה וּמִזֶּ֖ה לְכַסֹּתֽוֹ׃
\rashi{\rashiDH{והאמה מזה והאמה מזה. }לצפון ולדרום׃}\rashi{\rashiDH{בעדף באורך יריעות האהל. }שהן עודפות על אורך יריעות המשכן שתי אמות׃}\rashi{\rashiDH{יהיה סרוח על צדי המשכן. }לצפון ולדרום, כמו שפירשתי למעלה. למדה תורה דרך ארץ, שיהא אדם חס על היפה׃ }}
{וְאַמְּתָא מִכָּא וְאַמְּתָא מִכָּא בִּדְיַתִּיר בְּאֹרֶךְ יְרִיעָת מַשְׁכְּנָא יְהֵי סְרִיחַ עַל סִטְרֵי מַשְׁכְּנָא מִכָּא וּמִכָּא לְכַסָּיוּתֵיהּ׃}
{And the cubit on the one side, and the cubit on the other side, of that which remaineth over in the length of the curtains of the tent, shall hang over the sides of the tabernacle on this side and on that side, to cover it.}{\arabic{verse}}
\threeverse{\arabic{verse}}%Ex.26:14
{וְעָשִׂ֤יתָ מִכְסֶה֙ לָאֹ֔הֶל עֹרֹ֥ת אֵילִ֖ם מְאׇדָּמִ֑ים וּמִכְסֵ֛ה עֹרֹ֥ת תְּחָשִׁ֖ים מִלְמָֽעְלָה׃ \petucha 
\rashi{\rashiDH{מכסה לאהל. }לאותו גג של יריעות עזים, עשה עוד מכסה אחד של עורות אילים מאדמים, ועוד למעלה ממנו מכסה עורות תחשים, ואותן מכסאות לא היו מכסין אלא את הגג, ארכן ל׳ ורחבן י׳, אלו דברי רבי נחמיה. ולדברי רבי יהודה, מכסה אחד היה חציו של עורות אילים מאדמים וחציו של עורות תחשים׃ }}
{וְתַעֲבֵיד חוּפָאָה לְמַשְׁכְּנָא דְּמַשְׁכֵּי דִּכְרֵי מְסוּמְּקֵי וְחוּפָאָה דְּמַשְׁכֵּי סָסְגוֹנָא מִלְּעֵילָא׃}
{And thou shalt make a covering for the tent of rams’ skins dyed red and a covering of sealskins above.}{\arabic{verse}}
\threeverse{\aliya{רביעי}}%Ex.26:15
{וְעָשִׂ֥יתָ אֶת־הַקְּרָשִׁ֖ים לַמִּשְׁכָּ֑ן עֲצֵ֥י שִׁטִּ֖ים עֹמְדִֽים׃
\rashi{\rashiDH{ועשית את הקרשים. }היה לו לומר ועשית קרשים, כמו שנאמר בכל דבר ודבר, מהו הקרשים, מאותן העומדין ומיוחדין לכך. יעקב אבינו נטע ארזים במצרים, וכשמת, צוה לבניו להעלותם עמהם כשיצאו ממצרים, ואמר להם שעתיד הקב״ה לצוות אותן לעשות משכן במדבר מעצי שטים, ראו שיהיו מזומנים בידכם, הוא שיסד הבבלי בפיוט שלו, טַס מַטַּע מְזוֹרָזִים קורות בתינו ארזים, שנזדרזו להיות מוכנים בידם מקודם לכן׃ }\rashi{\rashiDH{עצי שטים עומדים. }אישטנבי״ש בלע״ז שיהא אורך הקרשים זקוף למעלה בקירות המשכן, ולא תעשה הכתלים בקרשים שוכבים, להיות רוחב הקרשים לגובה הכתלים קרש על קרש׃ }}
{וְתַעֲבֵיד יָת דַּפַּיָּא לְמַשְׁכְּנָא דְּאָעֵי שִׁטִּין קָיְמִין׃}
{And thou shalt make the boards for the tabernacle of acacia-wood, standing up.}{\arabic{verse}}
\threeverse{\arabic{verse}}%Ex.26:16
{עֶ֥שֶׂר אַמּ֖וֹת אֹ֣רֶךְ הַקָּ֑רֶשׁ וְאַמָּה֙ וַחֲצִ֣י הָֽאַמָּ֔ה רֹ֖חַב הַקֶּ֥רֶשׁ הָאֶחָֽד׃
\rashi{\rashiDH{עשר אמות אורך הקרש. }למדנו, גבהו של משכן עשר אמות׃ }\rashi{\rashiDH{ואמה וחצי האמה רוחב. }למדנו, ארכו של משכן לכ׳ קרשים שיהיו בצפון ובדרום מן המזרח למערב, ל׳ אמה׃ }}
{עֲשַׂר אַמִּין אוּרְכָּא דְּדַפָּא וְאַמְּתָא וּפַלְגוּת אַמְּתָא פוּתְיָא דְּדַפָּא חַד׃}
{Ten cubits shall be the length of a board, and a cubit and a half the breadth of each board.}{\arabic{verse}}
\threeverse{\arabic{verse}}%Ex.26:17
{שְׁתֵּ֣י יָד֗וֹת לַקֶּ֙רֶשׁ֙ הָאֶחָ֔ד מְשֻׁ֨לָּבֹ֔ת אִשָּׁ֖ה אֶל־אֲחֹתָ֑הּ כֵּ֣ן תַּעֲשֶׂ֔ה לְכֹ֖ל קַרְשֵׁ֥י הַמִּשְׁכָּֽן׃
\rashi{\rashiDH{שתי ידות לקרש האחד. }היה חורץ את הקרש מלמטה באמצעו בגובה אמה, ומניח רביע רחבו מכאן ורביע רחבו מכאן, והן הן הידות, והחריץ חצי רוחב הקרש באמצע, (רצונו לומר, קרש נקרא מה שנשאר לאחר שחרץ מכאן ומכאן, ואז נשאר רוחב אמה, נמצא חצי רוחב הקרש שבאמצע הוא חצי אמה. וכן פירש רש״י בהדיא בפ׳ הזורק (שבת דף צ״ח׃) שהחריץ באמצע רחב חצי אמה, והיה חורץ רביעית אמה מכל צד, וכל יד היה רוחב רביעית אמה, וכן כל שפת אדן היה רוחב רביעית אמה. ודוק היטב ואז סרה תלונת הרמב״ן ותמיהתו מסולקת) ואותן הידות מכניס באדנים שהיו חלולים, והאדנים גבהן אמה, ויושבים רצופים מ׳ זה אצל זה, וידות הקרש הנכנסות בחלל האדנים חרוצות משלשה צדיהן, רוחב החריץ כעובי שפת האדן, שיכסה הקרש את כל ראש האדן, שאם לא כן נמצא ריוח בין קרש לקרש כעובי שפת שני האדנים שיפסיקו ביניהם, וזהו שנאמר וְיִהיוּ תֹּאֲמִים מִלְּמַטָּה, שיחרוץ את צדי הידות כדי שיתחברו הקרשים זה אצל זה׃ }\rashi{\rashiDH{משולבות. }עשויות כמין שליבות סולם, מובדלות זו מזו, ומשופין ראשיהם ליכנס בתוך חלל האדן, כשליבה הנכנסת בנקב עמודי הסולם׃ }\rashi{\rashiDH{אשה אל אחתה. }מכוונות זו כנגד זו, שיהיו חריציהם שוים זו כמדת זו, כדי שלא יהיו שתי ידות זו משוכה לצד פנים וזו משוכה לצד חוץ בעובי הקרש שהוא אמה, ותרגום של ידות צִירִין, לפי שדומות לצירי הדלת הנכנסים בחורי המפתן׃ 
}}
{תְּרֵין צִירִין לְדַפָּא חַד מְשׁוּלְּבִין חַד לָקֳבֵיל חַד כֵּן תַּעֲבֵיד לְכֹל דַּפֵּי מַשְׁכְּנָא׃}
{Two tenons shall there be in each board, joined one to another; thus shalt thou make for all the boards of the tabernacle.}{\arabic{verse}}
\threeverse{\arabic{verse}}%Ex.26:18
{וְעָשִׂ֥יתָ אֶת־הַקְּרָשִׁ֖ים לַמִּשְׁכָּ֑ן עֶשְׂרִ֣ים קֶ֔רֶשׁ לִפְאַ֖ת נֶ֥גְבָּה תֵימָֽנָה׃
\rashi{\rashiDH{לפאת נגבה תימנה. }אין פאה זו לשון מקצוע, אלא כל הרוח קרויה פאה, כתרגומו לְרוּחַ עֵיבַר דְּרוֹמָא׃ }}
{וְתַעֲבֵיד יָת דַּפַּיָּא לְמַשְׁכְּנָא עֶשְׂרִין דַּפִּין לְרוּחַ עֵיבַר דָּרוֹמָא׃}
{And thou shalt make the boards for the tabernacle, twenty boards for the south side southward:}{\arabic{verse}}
\threeverse{\arabic{verse}}%Ex.26:19
{וְאַרְבָּעִים֙ אַדְנֵי־כֶ֔סֶף תַּעֲשֶׂ֕ה תַּ֖חַת עֶשְׂרִ֣ים הַקָּ֑רֶשׁ שְׁנֵ֨י אֲדָנִ֜ים תַּֽחַת־הַקֶּ֤רֶשׁ הָאֶחָד֙ לִשְׁתֵּ֣י יְדֹתָ֔יו וּשְׁנֵ֧י אֲדָנִ֛ים תַּֽחַת־הַקֶּ֥רֶשׁ הָאֶחָ֖ד לִשְׁתֵּ֥י יְדֹתָֽיו׃}
{וְאַרְבְּעִין סָמְכִין דִּכְסַף תַּעֲבֵיד תְּחוֹת עֶשְׂרִין דַּפִּין תְּרֵין סָמְכִין תְּחוֹת דַּפָּא חַד לִתְרֵין צִירוֹהִי וּתְרֵין סָמְכִין תְּחוֹת דַּפָּא חַד לִתְרֵין צִירוֹהִי׃}
{And thou shalt make forty sockets of silver under the twenty boards: two sockets under one board for its two tenons, and two sockets under another board for its two tenons;}{\arabic{verse}}
\threeverse{\arabic{verse}}%Ex.26:20
{וּלְצֶ֧לַע הַמִּשְׁכָּ֛ן הַשֵּׁנִ֖ית לִפְאַ֣ת צָפ֑וֹן עֶשְׂרִ֖ים קָֽרֶשׁ׃}
{וְלִסְטַר מַשְׁכְּנָא תִּנְיָנָא לְרוּחַ צִפּוּנָא עֶשְׂרִין דַּפִּין׃}
{and for the second side of the tabernacle, on the north side, twenty boards.}{\arabic{verse}}
\threeverse{\arabic{verse}}%Ex.26:21
{וְאַרְבָּעִ֥ים אַדְנֵיהֶ֖ם כָּ֑סֶף שְׁנֵ֣י אֲדָנִ֗ים תַּ֚חַת הַקֶּ֣רֶשׁ הָֽאֶחָ֔ד וּשְׁנֵ֣י אֲדָנִ֔ים תַּ֖חַת הַקֶּ֥רֶשׁ הָאֶחָֽד׃}
{וְאַרְבְּעִין סָמְכֵיהוֹן דִּכְסַף תְּרֵין סָמְכִין תְּחוֹת דַּפָּא חַד וּתְרֵין סָמְכִין תְּחוֹת דַּפָּא חַד׃}
{And their forty sockets of silver: two sockets under one board, and two sockets under another board.}{\arabic{verse}}
\threeverse{\arabic{verse}}%Ex.26:22
{וּֽלְיַרְכְּתֵ֥י הַמִּשְׁכָּ֖ן יָ֑מָּה תַּעֲשֶׂ֖ה שִׁשָּׁ֥ה קְרָשִֽׁים׃
\rashi{\rashiDH{ולירכתי. }לשון סוף, כתרגומו וְלִסְיָפֵי. ולפי שהפתח במזרח קרוי מזרח פנים והמערב אחורים, וזהו סוף, שהפנים הוא הראש׃ }\rashi{\rashiDH{תעשה ששה קרשים. }הרי ט׳ אמות רוחב׃}}
{וְלִסְיָפֵי מַשְׁכְּנָא מַעְרְבָא תַּעֲבֵיד שִׁתָּא דַּפִּין׃}
{And for the hinder part of the tabernacle westward thou shalt make six boards.}{\arabic{verse}}
\threeverse{\arabic{verse}}%Ex.26:23
{וּשְׁנֵ֤י קְרָשִׁים֙ תַּעֲשֶׂ֔ה לִמְקֻצְעֹ֖ת הַמִּשְׁכָּ֑ן בַּיַּרְכָתָֽיִם׃
\rashi{\rashiDH{ושני קרשים תעשה למקצעת. }אחד למקצוע צפונית מערבית ואחד למערבית דרומית, כל שמנה קרשים בסדר אחד הן, אלא שאלו השתים אינן בחלל המשכן, אלא חצי אמה מזו וחצי אמה מזו נראות בחלל להשלים רחבו לעשר, והאמה מזה והאמה מזה, באות כנגד אמות עובי קרשי המשכן הצפון והדרום, כדי שיהא המקצוע מבחוץ שוה׃ }}
{וּתְרֵין דַּפִּין תַּעֲבֵיד לְזָוְיָת מַשְׁכְּנָא בְּסוֹפְהוֹן׃}
{And two boards shalt thou make for the corners of the tabernacle in the hinder part.}{\arabic{verse}}
\threeverse{\arabic{verse}}%Ex.26:24
{וְיִֽהְי֣וּ תֹֽאֲמִם֮ מִלְּמַ֒טָּה֒ וְיַחְדָּ֗ו יִהְי֤וּ תַמִּים֙ עַל־רֹאשׁ֔וֹ אֶל־הַטַּבַּ֖עַת הָאֶחָ֑ת כֵּ֚ן יִהְיֶ֣ה לִשְׁנֵיהֶ֔ם לִשְׁנֵ֥י הַמִּקְצֹעֹ֖ת יִהְיֽוּ׃
\rashi{\rashiDH{ויהיו תואמים מלמטה. }כל הקרשים תואמים זה לזה מלמטה, שלא יפסיק עובי שפת שני האדנים ביניהם להרחיקן זו מזו, זהו שפרשתי, שיהיו צירי הידות חרוצים מצדיהן, שיהא רוחב הקרש בולט לצדיו חוץ לידי הקרש, לכסות את שפת האדן, וכן הקרש שאצלו, ונמצאו תואמים זה לזה, וקרש המקצוע שבסדר המערב, חרוץ לרחבו בעביו כנגד חריץ של צד קרש הצפוני והדרומי, כדי שלא יפרידו האדנים ביניהם׃ }\rashi{\rashiDH{ויחדו יהיו תמים. }כמו תאומים׃}\rashi{\rashiDH{על ראשו. }של קרש׃ 
}\rashi{\rashiDH{אל הטבעת האחת. }כל קרש וקרש היה חרוץ מלמעלה ברחבו שני חריצין בשני צדיו כמו עובי טבעת, ומכניסו בטבעת אחת, נמצא מתאים לקרש שאצלו. אבל אותן טבעות לא ידעתי אם קבועות הן אם מטולטלות. ובקרש שבמקצוע היה טבעת בעובי הקרש (נראה כי מלות הדרומי והצפוני אלמטה קאי, ורוצה לומר שהדרומי והצפוני וראש קרש כו׳ נכנס לתוכו, ומה שכתב היה טבעת בעובי קרש, רוצה לומר בעובי קרש המערבי. ודו״ק) הדרומי והצפוני, וראש קרש המקצוע שבסדר מערב נכנס לתוכו, נמצאו שני הכתלים מחוברים׃ }\rashi{\rashiDH{כן יהיה לשניהם. }לשני הקרשים שבמקצוע, לקרש שבסוף צפון ולקרש המערבי, וכן לשני המקצועות׃ 
}}
{וִיהוֹן מַכְוְנִין מִלְּרַע וְכַחְדָּא יְהוֹן מַכְוְנִין עַל רֵישֵׁיהוֹן בְּעִזְקְתָא חֲדָא כֵּן יְהֵי לְתַרְוֵיהוֹן לְתַרְתֵּין זָוְיָין יְהוֹן׃}
{And they shall be double beneath, and in like manner they shall be complete unto the top thereof unto the first ring; thus shall it be for them both; they shall be for the two corners.}{\arabic{verse}}
\threeverse{\arabic{verse}}%Ex.26:25
{וְהָיוּ֙ שְׁמֹנָ֣ה קְרָשִׁ֔ים וְאַדְנֵיהֶ֣ם כֶּ֔סֶף שִׁשָּׁ֥ה עָשָׂ֖ר אֲדָנִ֑ים שְׁנֵ֣י אֲדָנִ֗ים תַּ֚חַת הַקֶּ֣רֶשׁ הָאֶחָ֔ד וּשְׁנֵ֣י אֲדָנִ֔ים תַּ֖חַת הַקֶּ֥רֶשׁ הָאֶחָֽד׃
\rashi{\rashiDH{והיו שמנה קרשים. }הן האמורות למעלה תעשה ששה קרשים ושני קרשים תעשה למקוצעות, נמצאו שמנה קרשים בסדר מערבי. כך שנויה במשנה מעשה סדר הקרשים במלאכת המשכן. היה עושה את האדנים חלולים, וחורץ את הקרש מלמטה רביע מכאן ורביע מכאן, והחריץ חציו באמצע, ועשה לו שתי ידות כמין שני חמוקין (ולי נראה שהגרסא כמין שני חווקין) כמין שני שליבות סולם המובדלות זו מזו, ומשופות להכנס בחלל האדן כשליבה הנכנסת בנקב עמודי הסולם, והוא לשון משולבות, עשויות כמין שליבה, ומכניסן לתוך שני אדנים, שנאמר שְׁנֵי אֲדָנִים שְׁנֵי אֲדָנִים (שמות לו, ל), וחורץ את הקרש מלמעלה אצבע מכאן ואצבע מכאן, ונותנן לתוך טבעת אחת של זהב, כדי שלא יהיו נפרדין זה מזה, שנאמר ויהיו תואמים מלמטה וגו׳, כך היא המשנה, והפירוש שלה הצעתי למעלה בסדר המקראות׃ 
}}
{וִיהוֹן תְּמָנְיָא דַּפִּין וְסָמְכֵיהוֹן דִּכְסַף שִׁתַּת עֲשַׂר סָמְכִין תְּרֵין סָמְכִין תְּחוֹת דַּפָּא חַד וּתְרֵין סָמְכִין תְּחוֹת דַּפָּא חַד׃}
{Thus there shall be eight boards, and their sockets of silver, sixteen sockets: two sockets under one board, and two sockets under another board.}{\arabic{verse}}
\threeverse{\arabic{verse}}%Ex.26:26
{וְעָשִׂ֥יתָ בְרִיחִ֖ם עֲצֵ֣י שִׁטִּ֑ים חֲמִשָּׁ֕ה לְקַרְשֵׁ֥י צֶֽלַע־הַמִּשְׁכָּ֖ן הָאֶחָֽד׃
\rashi{\rashiDH{בריחם. }כתרגומו עַבְּרִין, ובלע״ז אשפרי״ש }\rashi{\rashiDH{חמשה לקרשי צלע המשכן. }אלו ה׳, ג׳ הן, אלא שהבריח העליון והתחתון עשוי משתי חתיכות, זה מבריח עד חצי הכותל וזה מבריח עד חצי הכותל, זה נכנס בטבעת מצד זה וזה נכנס בטבעת מצד זה, עד שמגיעין זה לזה, נמצא שעליון ותחתון שנים שהן ארבע, אבל האמצעי ארכו כנגד כל הכותל, ומבריח מקצה הכותל ועד קצהו, שנאמר והבריח התיכון וגו׳ מבריח מן הקצה אל הקצה, שהעליונים והתחתונים היו להן טבעות בקרשים להכנס לתוכן, שתי טבעות לכל קרש, משולשים בתוך עשר אמות של גובה הקרש, חלק אחד מן הטבעת העליונה ולמעלה, וחלק אחד מן התחתונה ולמטה, וכל חלק הוא רביע אורך הקרש, ושני חלקים בין טבעת לטבעת, כדי שיהיו כל הטבעות מכוונות זו כנגד זו, אבל לבריח התיכון אין טבעות, אלא הקרשים נקובין בעובים, והוא נכנס בהם דרך הנקבים שהם מכוונין זה מול זה, וזהו שנאמר בתוך הקרשים. הבריחים העליונים והתחתונים שבצפון ושבדרום, אורך כל אחת ט״ו אמה, והתיכון ארכו ל׳ אמה, וזהו מן הקצה אל הקצה, מן המזרח ועד המערב, וה׳ בריחים שבמערב אורך העליונים והתחתונים ו׳ אמות, והתיכון ארכו י״ב, כנגד רוחב ח׳ קרשים, כך היא מפורשת במלאכת המשכן (שבת צח׃)׃ }}
{וְתַעֲבֵיד עָבְרֵי דְּאָעֵי שִׁטִּין חַמְשָׁא לְדַפֵּי סְטַר מַשְׁכְּנָא חַד׃}
{And thou shalt make bars of acacia-wood: five for the boards of the one side of the tabernacle,}{\arabic{verse}}
\threeverse{\arabic{verse}}%Ex.26:27
{וַחֲמִשָּׁ֣ה בְרִיחִ֔ם לְקַרְשֵׁ֥י צֶֽלַע־הַמִּשְׁכָּ֖ן הַשֵּׁנִ֑ית וַחֲמִשָּׁ֣ה בְרִיחִ֗ם לְקַרְשֵׁי֙ צֶ֣לַע הַמִּשְׁכָּ֔ן לַיַּרְכָתַ֖יִם יָֽמָּה׃}
{וְחַמְשָׁא עָבְרִין לְדַפֵּי סְטַר מַשְׁכְּנָא תִּנְיָנָא וְחַמְשָׁא עָבְרִין לְדַפֵּי סְטַר מַשְׁכְּנָא לְסוֹפְהוֹן מַעְרְבָא׃}
{and five bars for the boards of the other side of the tabernacle, and five bars for the boards of the side of the tabernacle, for the hinder part westward;}{\arabic{verse}}
\threeverse{\arabic{verse}}%Ex.26:28
{וְהַבְּרִ֥יחַ הַתִּיכֹ֖ן בְּת֣וֹךְ הַקְּרָשִׁ֑ים מַבְרִ֕חַ מִן־הַקָּצֶ֖ה אֶל־הַקָּצֶֽה׃}
{וְעָבְרָא מְצִיעָאָה בְּגוֹ דַּפַּיָּא מַעְבַּר מִן סְיָפֵי לִסְיָפֵי׃}
{and the middle bar in the midst of the boards, which shall pass through from end to end.}{\arabic{verse}}
\threeverse{\arabic{verse}}%Ex.26:29
{וְֽאֶת־הַקְּרָשִׁ֞ים תְּצַפֶּ֣ה זָהָ֗ב וְאֶת־טַבְּעֹֽתֵיהֶם֙ תַּעֲשֶׂ֣ה זָהָ֔ב בָּתִּ֖ים לַבְּרִיחִ֑ם וְצִפִּיתָ֥ אֶת־הַבְּרִיחִ֖ם זָהָֽב׃
\rashi{\rashiDH{בתים לבריחם. }הטבעות שתעשה בהן יהיו בתים להכניס בהן הבריחים׃}\rashi{\rashiDH{וצפית את הבריחים זהב. }לא שהיה הזהב מדובק על הבריחים, שאין עליהם שום צפוי, אלא בקרש היה קובע כמין ב׳ פיפיות של זהב כמין ב׳ סדקי קנה חלול, וקובען אצל הטבעות לכאן ולכאן, ארכן ממלא את רוחב הקרש מן הטבעת לכאן וממנה לכאן, והבריח נכנס לתוכו, וממנו לטבעת, ומן הטבעת לפה השני, נמצאו הבריחים מצופים זהב כשהן תחובין בקרשים, והבריחים הללו מבחוץ היו בולטות. הטבעות והפיפיות לא היו נראות בתוך המשכן, אלא כל הכותל חלק מבפנים׃ }}
{וְיָת דַּפַּיָּא תִּחְפֵי דַּהְבָּא וְיָת עִזְקָתְהוֹן תַּעֲבֵיד דַּהְבָּא אַתְרָא לְעָבְרַיָּא וְתִחְפֵי יָת עָבְרַיָּא דַּהְבָּא׃}
{And thou shalt overlay the boards with gold, and make their rings of gold for holders for the bars; and thou shalt overlay the bars with gold.}{\arabic{verse}}
\threeverse{\arabic{verse}}%Ex.26:30
{וַהֲקֵמֹתָ֖ אֶת־הַמִּשְׁכָּ֑ן כְּמִ֨שְׁפָּט֔וֹ אֲשֶׁ֥ר הׇרְאֵ֖יתָ בָּהָֽר׃ \setuma         
\rashi{\rashiDH{והקמת את המשכן. }לאחר שיגמור, הקימהו׃ }\rashi{\rashiDH{הראית בהר. }קודם לכן, שאני עתיד ללמדך ולהראותך סדר הקמתו׃ }}
{וּתְקִים יָת מַשְׁכְּנָא כְּהִלְכְּתֵיהּ דְּאִתַּחְזִיתָא בְּטוּרָא׃}
{And thou shalt rear up the tabernacle according to the fashion thereof which hath been shown thee in the mount.}{\arabic{verse}}
\threeverse{\aliya{חמישי}}%Ex.26:31
{וְעָשִׂ֣יתָ פָרֹ֗כֶת תְּכֵ֧לֶת וְאַרְגָּמָ֛ן וְתוֹלַ֥עַת שָׁנִ֖י וְשֵׁ֣שׁ מׇשְׁזָ֑ר מַעֲשֵׂ֥ה חֹשֵׁ֛ב יַעֲשֶׂ֥ה אֹתָ֖הּ כְּרֻבִֽים׃
\rashi{\rashiDH{פרוכת. }לשון מחיצה הוא, ובלשון חכמים פרגוד, דבר המבדיל בין המלך ובין העם׃ }\rashi{\rashiDH{תכלת וארגמן. }כל מין ומין היה כפול, בכל חוט וחוט ו׳ חוטין׃ }\rashi{\rashiDH{מעשה חושב. }כבר פרשתי שזו היא אריגה של שתי קירות, והציורין שמשני עבריה אינן דומין זה לזה׃ }\rashi{\rashiDH{כרבים. }ציורין של בריות יעשה בה.}}
{וְתַעֲבֵיד פָּרוּכְתָּא דְּתַכְלָא וְאַרְגְּוָנָא וּצְבַע זְהוֹרִי וּבוּץ שְׁזִיר עוֹבָד אוּמָּן יַעֲבֵיד יָתַהּ צוּרַת כְּרוּבִין׃}
{And thou shalt make a veil of blue, and purple, and scarlet, and fine twined linen; with cherubim the work of the skilful workman shall it be made.}{\arabic{verse}}
\threeverse{\arabic{verse}}%Ex.26:32
{וְנָתַתָּ֣ה אֹתָ֗הּ עַל־אַרְבָּעָה֙ עַמּוּדֵ֣י שִׁטִּ֔ים מְצֻפִּ֣ים זָהָ֔ב וָוֵיהֶ֖ם זָהָ֑ב עַל־אַרְבָּעָ֖ה אַדְנֵי־כָֽסֶף׃
\rashi{\rashiDH{ארבעה עמודי שטים.} ד׳ עמודים תקועים בתוך ד׳ אדנים, ואונקליות קבועין בהן, עקומין למעלה להושיב עליהן כלונס שראש הפרוכת כרוך בה, והאונקליות הן הווין, שהרי כמין ווין הן עשוים, והפרוכת ארכה י׳ אמות לרחבו של משכן, ורחבה י׳ אמות כגבהן של קרשים, פרוסה בשליש של משכן, שיהא הימנה ולפנים עשר אמות, והימנה ולחוץ כ׳ אמה, נמצא בית קדשי הקדשים עשר על עשר, שנאמר וְנָתתָּה אֶת הַפָּרֹכֶת תַּחַת הַקְּרָסִים, המחברים את שתי חוברות של יריעות המשכן, ורוחב החוברת כ׳ אמה, וכשפרשם על גג המשכן מן הפתח למערב, כלתה בשני שלישי המשכן, והחוברת השנית כסתה שלישו של משכן, והמותר תלוי לאחוריו לכסות את הקרשים׃}}
{וְתִתֵּין יָתַהּ עַל אַרְבְּעָא עַמּוּדֵי שִׁטִּין מְחוּפַּן דַּהְבָּא וָוֵיהוֹן דַּהְבָּא עַל אַרְבְּעָא סָמְכִין דִּכְסַף׃}
{And thou shalt hang it upon four pillars of acacia overlaid with gold, their hooks being of gold, upon four sockets of silver.}{\arabic{verse}}
\threeverse{\arabic{verse}}%Ex.26:33
{וְנָתַתָּ֣ה אֶת־הַפָּרֹ֘כֶת֮ תַּ֣חַת הַקְּרָסִים֒ וְהֵבֵאתָ֥ שָׁ֙מָּה֙ מִבֵּ֣ית לַפָּרֹ֔כֶת אֵ֖ת אֲר֣וֹן הָעֵד֑וּת וְהִבְדִּילָ֤ה הַפָּרֹ֙כֶת֙ לָכֶ֔ם בֵּ֣ין הַקֹּ֔דֶשׁ וּבֵ֖ין קֹ֥דֶשׁ הַקֳּדָשִֽׁים׃}
{וְתִתֵּין יָת פָּרוּכְתָּא תְּחוֹת פּוּרְפַיָּא וְתַעֵיל לְתַמָּן מִגָּיו לְפָרוּכְתָּא יָת אֲרוֹנָא דְּסָהֲדוּתָא וְתַפְרֵישׁ פָּרוּכְתָּא לְכוֹן בֵּין קוּדְשָׁא וּבֵין קֹדֶשׁ קוּדְשַׁיָּא׃}
{And thou shalt hang up the veil under the clasps, and shalt bring in thither within the veil the ark of the testimony; and the veil shall divide unto you between the holy place and the most holy.}{\arabic{verse}}
\threeverse{\arabic{verse}}%Ex.26:34
{וְנָתַתָּ֙ אֶת־הַכַּפֹּ֔רֶת עַ֖ל אֲר֣וֹן הָעֵדֻ֑ת בְּקֹ֖דֶשׁ הַקֳּדָשִֽׁים׃}
{וְתִתֵּין יָת כָּפוּרְתָּא עַל אֲרוֹנָא דְּסָהֲדוּתָא בְּקֹדֶשׁ קוּדְשַׁיָּא׃}
{And thou shalt put the ark-cover upon the ark of the testimony in the most holy place.}{\arabic{verse}}
\threeverse{\arabic{verse}}%Ex.26:35
{וְשַׂמְתָּ֤ אֶת־הַשֻּׁלְחָן֙ מִח֣וּץ לַפָּרֹ֔כֶת וְאֶת־הַמְּנֹרָה֙ נֹ֣כַח הַשֻּׁלְחָ֔ן עַ֛ל צֶ֥לַע הַמִּשְׁכָּ֖ן תֵּימָ֑נָה וְהַ֨שֻּׁלְחָ֔ן תִּתֵּ֖ן עַל־צֶ֥לַע צָפֽוֹן׃
\rashi{\rashiDH{ושמת את השלחן. }שלחן בצפון, משוך מן הכותל הצפוני שתי אמות ומחצה, ומנורה בדרום, משוכה מן הכותל הדרומי שתי אמות ומחצה, ומזבח הזהב נתון כנגד אויר שבין שלחן למנורה, משוך קמעא כלפי המזרח, וכולם נתונים מן חצי המשכן ולפנים. כיצד, אורך המשכן מן הפתח לפרוכת עשרים אמה, המזבח והשלחן והמנורה משוכים מן הפתח לצד מערב עשר אמות׃ 
}}
{וּתְשַׁוֵּי יָת פָּתוּרָא מִבַּרָא לְפָרוּכְתָּא וְיָת מְנָרְתָא לָקֳבֵיל פָּתוּרָא עַל סְטַר מַשְׁכְּנָא דָּרוֹמָא וּפָתוּרָא תִּתֵּין עַל סְטַר צִפּוּנָא׃}
{And thou shalt set the table without the veil, and the candlestick over against the table on the side of the tabernacle toward the south; and thou shalt put the table on the north side.}{\arabic{verse}}
\threeverse{\arabic{verse}}%Ex.26:36
{וְעָשִׂ֤יתָ מָסָךְ֙ לְפֶ֣תַח הָאֹ֔הֶל תְּכֵ֧לֶת וְאַרְגָּמָ֛ן וְתוֹלַ֥עַת שָׁנִ֖י וְשֵׁ֣שׁ מׇשְׁזָ֑ר מַעֲשֵׂ֖ה רֹקֵֽם׃
\rashi{\rashiDH{ועשית מסך. }וילון, הוא מסך כנגד הפתח, כמו שַׂכְתָּ בַעֲדֹו (איוב א, י), לשון מגין׃ }\rashi{\rashiDH{מעשה רוקם. }הצורות עשויות בו מעשה מחט, כפרצוף של עבר זה כך פרצוף של עבר זה׃ }\rashi{\rashiDH{רוקם. }שם האומן ולא שם האומנות, ותרגומו עוֹבַד צַיָּיר, ולא עובד ציור. מדת המסך כמדת הפרוכת י׳ אמות על י׳ אמות׃ }}
{וְתַעֲבֵיד פְּרָסָא לִתְרַע מַשְׁכְּנָא תַּכְלָא וְאַרְגְּוָנָא וּצְבַע זְהוֹרִי וּבוּץ שְׁזִיר עוֹבָד צַיָּיר׃}
{And thou shalt make a screen for the door of the Tent, of blue, and purple, and scarlet, and fine twined linen, the work of the weaver in colours.}{\arabic{verse}}
\threeverse{\arabic{verse}}%Ex.26:37
{וְעָשִׂ֣יתָ לַמָּסָ֗ךְ חֲמִשָּׁה֙ עַמּוּדֵ֣י שִׁטִּ֔ים וְצִפִּיתָ֤ אֹתָם֙ זָהָ֔ב וָוֵיהֶ֖ם זָהָ֑ב וְיָצַקְתָּ֣ לָהֶ֔ם חֲמִשָּׁ֖ה אַדְנֵ֥י נְחֹֽשֶׁת׃ \setuma         }
{וְתַעֲבֵיד לִפְרָסָא חַמְשָׁא עַמּוּדֵי שִׁטִּין וְתִחְפֵי יָתְהוֹן דַּהְבָּא וָוֵיהוֹן דַּהְבָּא וְתַתֵּיךְ לְהוֹן חַמְשָׁא סָמְכִין דִּנְחָשָׁא׃}
{And thou shalt make for the screen five pillars of acacia, and overlay them with gold; their hooks shall be of gold; and thou shalt cast five sockets of brass for them.}{\arabic{verse}}
\newperek
\threeverse{\aliya{ששי}}%Ex.27:1
{וְעָשִׂ֥יתָ אֶת־הַמִּזְבֵּ֖חַ עֲצֵ֣י שִׁטִּ֑ים חָמֵשׁ֩ אַמּ֨וֹת אֹ֜רֶךְ וְחָמֵ֧שׁ אַמּ֣וֹת רֹ֗חַב רָב֤וּעַ יִהְיֶה֙ הַמִּזְבֵּ֔חַ וְשָׁלֹ֥שׁ אַמּ֖וֹת קֹמָתֽוֹ׃
\rashi{\rashiDH{ועשית את המזבח וגו׳. ושלש אמות קומתו. }דברים ככתבן, דברי רבי יהודה, רבי יוסי אומר, נאמר כאן רבוע ונאמר בפנימי רבוע, מה להלן גבהו פי שנים כארכו, אף כאן גבהו פי שנים כארכו, ומה אני מקיים ושלש אמות קומתו, משפת סובב ולמעלה (זבחים נט׃)׃ }}
{וְתַעֲבֵיד יָת מַדְבְּחָא דְּאָעֵי שִׁטִּין חֲמֵישׁ אַמִּין אוּרְכָּא וַחֲמֵישׁ אַמִּין פּוּתְיָא מְרוּבַּע יְהֵי מַדְבְּחָא וּתְלָת אַמִּין רוּמֵיהּ׃}
{And thou shalt make the altar of acacia-wood, five cubits long, and five cubits broad; the altar shall be four-square; and the height thereof shall be three cubits.}{\Roman{chap}}
\threeverse{\arabic{verse}}%Ex.27:2
{וְעָשִׂ֣יתָ קַרְנֹתָ֗יו עַ֚ל אַרְבַּ֣ע פִּנֹּתָ֔יו מִמֶּ֖נּוּ תִּהְיֶ֣יןָ קַרְנֹתָ֑יו וְצִפִּיתָ֥ אֹת֖וֹ נְחֹֽשֶׁת׃
\rashi{\rashiDH{ממנו תהיין קרנותיו. }שלא יעשם לבדם ויחברם בו׃}\rashi{\rashiDH{וצפית אותו נחושת. }לכפר על עזות מצח, שנאמר וּמִצְחֲךָ נְחוּשָׁה (ישעיה מח, ד)׃ }}
{וְתַעֲבֵיד קַרְנוֹהִי עַל אַרְבַּע זָוְיָתֵיהּ מִנֵּיהּ יִהְוְיָן קַרְנוֹהִי וְתִחְפֵי יָתֵיהּ נְחָשָׁא׃}
{And thou shalt make the horns of it upon the four corners thereof; the horns thereof shall be of one piece with it; and thou shalt overlay it with brass.}{\arabic{verse}}
\threeverse{\arabic{verse}}%Ex.27:3
{וְעָשִׂ֤יתָ סִּֽירֹתָיו֙ לְדַשְּׁנ֔וֹ וְיָעָיו֙ וּמִזְרְקֹתָ֔יו וּמִזְלְגֹתָ֖יו וּמַחְתֹּתָ֑יו לְכׇל־כֵּלָ֖יו תַּעֲשֶׂ֥ה נְחֹֽשֶׁת׃
\rashi{\rashiDH{סירותיו. }כמין יורות׃}\rashi{\rashiDH{לדשנו. }להסיר דשנו לתוכה, והוא שתרגם אונקלוס לְמִסְפֵי קִטְמֵיהּ, לספות הדשן לתוכם, כי יש מלות בלשון עברית מלה אחת מתחלפת בפתרון, לשמש בנין וסתירה כמו וַתַּשְׁרֵשׁ שָׁרָשֶׁיהָ (תהלים פ, י), אֱוִיל מַשְׁרִישׁ (איוב ה, ג), וחלופו, וּבְכָל תְּבוּאָתִי תְּשָׁרֵשׁ (שם לא, יב). וכמוהו בִּסְעִפֶיהָ פֹּרִיָּה (ישעיה יז, ו), וחלופו מְסָעֵף פֻּארָה (שם י, לג), מפשח סעיפיה. וכמוהו וְזֶה הָאַחֲרֹון עִצְּמֹו (ירמיה נ, יז), שבר עצמיו. וכמוהו וַיְסקְּלֻהוּ בָּאֲבָנִים (מלכים־א כא, יג), וחלופו סַקְּלוּ מֵאֶבֶן (ישעי׳ סב, י), הסירו אבניה, וכן וַיְעַזְקֵהוּ וַיְסַקְּלֵהוּ (שם ה, ב). אף כאן לדשנו להסיר דשנו, ובלע״ז אדשצדר״יר }\rashi{\rashiDH{ויעיו. }כתרגומו, מגרפות שנוטל בהם הדשן, והן כמין כסוי קדרה של מתכת דק, ולו בית יד, ובלע״ז וידי״ל }\rashi{\rashiDH{ומזרקותיו. }לקבל בהם דם הזבחים׃}\rashi{\rashiDH{ומזלגותיו. }כמין אונקליות כפופים, ומכה בהם בבשר ונתחבים בו, ומתהפכין בהן על גחלי המערכה שיהא ממהר שריפתן, ובלע״ז קרוצינ״ש ובלשון חכמים צנוריות׃ }\rashi{\rashiDH{ומחתותיו. }בית קבול יש להם, ליטול בהן גחלים מן המזבח לשאתם על מזבח הפנימי לקטרת, ועל שם חתייתן קרויים מחתות, כמו לַחְתֹּות אֵשׁ מִיָּקוּד (ישעיה ל, יד), לשון שאיבת אש ממקומה, וכן הַיַחֲתֶּה אִישׁ אֵשׁ בְּחֵיקֹו (משלי ו, כז)׃ }\rashi{\rashiDH{לכל כליו. }כמו כל כליו׃}}
{וְתַעֲבֵיד פְּסַכְתֵּירְוָתֵיהּ לְמִסְפֵּי קִטְמֵיהּ וּמַגְרוֹפְיָתֵיהּ וּמִזְרְקוֹהִי וְצִנּוֹרְיָתֵיהּ וּמַחְתְּיָתֵיהּ לְכָל מָנוֹהִי תַּעֲבֵיד נְחָשָׁא׃}
{And thou shalt make its pots to take away its ashes, and its shovels, and its basins, and its flesh-hooks, and its fire-pans; all the vessels thereof thou shalt make of brass.}{\arabic{verse}}
\threeverse{\arabic{verse}}%Ex.27:4
{וְעָשִׂ֤יתָ לּוֹ֙ מִכְבָּ֔ר מַעֲשֵׂ֖ה רֶ֣שֶׁת נְחֹ֑שֶׁת וְעָשִׂ֣יתָ עַל־הָרֶ֗שֶׁת אַרְבַּע֙ טַבְּעֹ֣ת נְחֹ֔שֶׁת עַ֖ל אַרְבַּ֥ע קְצוֹתָֽיו׃
\rashi{\rashiDH{מכבר. }לשון כברה שקורין קריבל״ש (זיעב) כמין לבוש עשוי לו למזבח, עשוי חורין חורין כמין רשת, ומקרא זה מסורס, וכה פתרונו, ועשית לו מכבר נחושת מעשה רשת׃ 
}}
{וְתַעֲבֵיד לֵיהּ סְרָדָא עוֹבָד מְצָדְתָא דִּנְחָשָׁא וְתַעֲבֵיד עַל מְצָדְתָא אַרְבַּע עִזְקָן דִּנְחָשָׁא עַל אַרְבְּעָא סִטְרוֹהִי׃}
{And thou shalt make for it a grating of network of brass; and upon the net shalt thou make four brazen rings in the four corners thereof.}{\arabic{verse}}
\threeverse{\arabic{verse}}%Ex.27:5
{וְנָתַתָּ֣ה אֹתָ֗הּ תַּ֛חַת כַּרְכֹּ֥ב הַמִּזְבֵּ֖חַ מִלְּמָ֑טָּה וְהָיְתָ֣ה הָרֶ֔שֶׁת עַ֖ד חֲצִ֥י הַמִּזְבֵּֽחַ׃
\rashi{\rashiDH{כרכב המזבח. }סובב, כל דבר המקיף סביב בעגול קרוי כרכב, כמו ששנינו בהכל שוחטין (חולין כה.), אלו הן גולמי כלי עץ, כל שעתיד לשוף ולכרכב, והוא כמו שעושין חריצין עגולין בקרשי דפני התיבות וספסלי העץ, אף למזבח עשה חריץ סביבו, והיה רחבו אמה בדפנו לנוי, והוא לסוף שלש (ס״א שש) אמות של גבהו, כדברי האומר גבהו פי שנים כארכו, הא מה אני מקיים ושלש אמות קומתו, משפת סובב ולמעלה, אבל סובב להלוך הכהנים, לא היה למזבח הנחשת, אלא על ראשו לפנים מקרנותיו, וכן שנינו בזבחים (סב.), איזהו כרכוב, בין קרן לקרן, והיה רוחב אמה ולפנים מהן אמה של הלוך רגלי הכהנים, שתי אמות הללו קרויים כרכוב. ודקדקנו שם, והכתיב תחת כרכוב המזבח מלמטה, למדנו שהכרכוב בדפנו הוא ולבוש המכבר תחתיו, ותירץ המתרץ, תרי הוי, חד לנוי, וחד לכהנים דלא ישתרגו, זה שבדופן לנוי היה, ומתחתיו הלבישו המכבר, והגיע רחבו עד חצי המזבח, נמצא שהמכבר רחב אמה, והוא היה סימן לחצי גבהו, להבדיל בין דמים העליונים לדמים התחתונים, וכנגדו עשו למזבח בית עולמים, דוגמת חוט הסקרא באמצעו. וכבש שהיו עולין בו, אף על פי שלא פירשו בענין זה, כבר שמענו בפרשת מזבח אדמה תעשה לי, ולא תעלה במעלות, לא תעשה לו מעלות בכבש שלו אלא כבש חלק, למדנו שהיה לו כבש. כך שנינו במכילתא (בחדש פי״א). ומזבח אדמה הוא מזבח הנחשת, שהיו ממלאין חללו אדמה במקום חנייתן, והכבש היה בדרום המזבח, מובדל מן המזבח מלא חוט השערה, ורגליו מגיעין עד אמה סמוך לקלעי החצר שבדרום, כדברי האומר י׳ אמות קומתו. ולדברי האומר דברים ככתבן, ג׳ אמות קומתו, לא היה אורך הכבש אלא י׳ אמות, כך מצאתי במשנה מ״ט מדות, וזה שהיה מובדל מן המזבח מלא החוט, במסכת זבחים (סב׃) למדנוה מן המקרא׃ }}
{וְתִתֵּין יָתַהּ תְּחוֹת סוֹבֵיבָא דְּמַדְבְּחָא מִלְּרַע וּתְהֵי מְצָדְתָא עַד פַּלְגוּת מַדְבְּחָא׃}
{And thou shalt put it under the ledge round the altar beneath, that the net may reach halfway up the altar.}{\arabic{verse}}
\threeverse{\arabic{verse}}%Ex.27:6
{וְעָשִׂ֤יתָ בַדִּים֙ לַמִּזְבֵּ֔חַ בַּדֵּ֖י עֲצֵ֣י שִׁטִּ֑ים וְצִפִּיתָ֥ אֹתָ֖ם נְחֹֽשֶׁת׃}
{וְתַעֲבֵיד אֲרִיחַיָּא לְמַדְבְּחָא אֲרִיחֵי דְּאָעֵי שִׁטִּין וְתִחְפֵי יָתְהוֹן נְחָשָׁא׃}
{And thou shalt make staves for the altar, staves of acacia-wood, and overlay them with brass.}{\arabic{verse}}
\threeverse{\arabic{verse}}%Ex.27:7
{וְהוּבָ֥א אֶת־בַּדָּ֖יו בַּטַּבָּעֹ֑ת וְהָי֣וּ הַבַּדִּ֗ים עַל־שְׁתֵּ֛י צַלְעֹ֥ת הַמִּזְבֵּ֖חַ בִּשְׂאֵ֥ת אֹתֽוֹ׃
\rashi{\rashiDH{בטבעות. }בארבע טבעות שנעשו למכבר׃ 
}}
{וְיַעֵיל יָת אֲרִיחוֹהִי בְּעִזְקָתָא וִיהוֹן אֲרִיחַיָּא עַל תְּרֵין סִטְרֵי מַדְבְּחָא בְּמִטַּל יָתֵיהּ׃}
{And the staves thereof shall be put into the rings, and the staves shall be upon the two sides of the altar, in bearing it.}{\arabic{verse}}
\threeverse{\arabic{verse}}%Ex.27:8
{נְב֥וּב לֻחֹ֖ת תַּעֲשֶׂ֣ה אֹת֑וֹ כַּאֲשֶׁ֨ר הֶרְאָ֥ה אֹתְךָ֛ בָּהָ֖ר כֵּ֥ן יַעֲשֽׂוּ׃ \setuma         
\rashi{\rashiDH{נבוב לוחות. }כתרגומו חֲלִיל לוּחִין, לוחות עצי שטים מכל צד והחלל באמצע, ולא יהא כולו עץ אחד, שיהא עביו ה׳ אמות על ה׳ אמות, כמין סדן׃ }}
{חֲלִיל לוּחִין תַּעֲבֵיד יָתֵיהּ כְּמָא דְּאַחְזִי יָתָךְ בְּטוּרָא כֵּן יַעְבְּדוּן׃}
{Hollow with planks shalt thou make it; as it hath been shown thee in the mount, so shall they make it.}{\arabic{verse}}
\threeverse{\aliya{שביעי}}%Ex.27:9
{וְעָשִׂ֕יתָ אֵ֖ת חֲצַ֣ר הַמִּשְׁכָּ֑ן לִפְאַ֣ת נֶֽגֶב־תֵּ֠ימָ֠נָה קְלָעִ֨ים לֶחָצֵ֜ר שֵׁ֣שׁ מׇשְׁזָ֗ר מֵאָ֤ה בָֽאַמָּה֙ אֹ֔רֶךְ לַפֵּאָ֖ה הָאֶחָֽת׃
\rashi{\rashiDH{קלעים.} עשויין כמין קלעי ספינה נקבים נקבים, מעשה קליעה ולא מעשה אורג, ותרגומו סְרָדִין כתרגומו של מכבר המתורגם סְרָדָא לפי שהן מנוקבין ככברה׃ }\rashi{\rashiDH{לפאה האחת. }כל הרוח קרוי פאה׃}}
{וְתַעֲבֵיד יָת דָּרַת מַשְׁכְּנָא לְרוּחַ עֵיבַר דָּרוֹמָא סְרָדֵי לְדָרְתָא דְּבוּץ שְׁזִיר מְאָה אַמִּין אוּרְכָּא לְרוּחָא חֲדָא׃}
{And thou shalt make the court of the tabernacle: for the south side southward there shall be hangings for the court of fine twined linen a hundred cubits long for one side.}{\arabic{verse}}
\threeverse{\arabic{verse}}%Ex.27:10
{וְעַמֻּדָ֣יו עֶשְׂרִ֔ים וְאַדְנֵיהֶ֥ם עֶשְׂרִ֖ים נְחֹ֑שֶׁת וָוֵ֧י הָעַמֻּדִ֛ים וַחֲשֻׁקֵיהֶ֖ם כָּֽסֶף׃
\rashi{\rashiDH{ועמודיו עשרים. }חמש אמות בין עמוד לעמוד׃}\rashi{\rashiDH{ואדניהם. }של העמודים נחשת, האדנים יושבים על הארץ, והעמודים תקועין לתוכן, היה עושה כמין קונדסין שקורין פלא״ש, ארכן ו׳ טפחים ורחבן ג׳, וטבעת נחשת קבוע בו באמצעו, וכורך שפת הקלע סביביו במיתרים כנגד כל עמוד ועמוד, ותולה הקונדס דרך טבעתו באונקליות שבעמוד העשוי כמין וי״ו, ראשו זקוף למעלה וראשו אחד תקוע בעמוד, כאותן שעושין להציב דלתות שקורין גונזי״ש, ורחב הקלע תלוי מלמטה, והיא קומת מחיצות החצר׃ }\rashi{\rashiDH{ווי העמודים. }הם האונקליות׃}\rashi{\rashiDH{וחשוקיהם. }מוקפות היו העמודים בחוטי כסף סביב, ואיני יודע אם על פני כולן, אם בראשם, ואם באמצעם, אך יודע אני שחשוק לשון חגורה, שכך מצינו בפילגש בגבעה, וְעִמֹּו צֶמֶד חֲמֹורִים חֲבוּשִׁים (שופטים יט, י), תרגומו חשוקים׃ }}
{וְעַמּוּדוֹהִי עַסְרִין וְסָמְכֵיהוֹן עַסְרִין דִּנְחָשָׁא וָוֵי עַמּוּדַיָּא וְכִבּוּשֵׁיהוֹן כְּסַף׃}
{And the pillars thereof shall be twenty, and their sockets twenty, of brass; the hooks of the pillars and their fillets shall be of silver.}{\arabic{verse}}
\threeverse{\arabic{verse}}%Ex.27:11
{וְכֵ֨ן לִפְאַ֤ת צָפוֹן֙ בָּאֹ֔רֶךְ קְלָעִ֖ים מֵ֣אָה אֹ֑רֶךְ וְעַמֻּדָ֣ו עֶשְׂרִ֗ים וְאַדְנֵיהֶ֤ם עֶשְׂרִים֙ נְחֹ֔שֶׁת וָוֵ֧י הָֽעַמֻּדִ֛ים וַחֲשֻׁקֵיהֶ֖ם כָּֽסֶף׃}
{וְכֵן לְרוּחַ צִפּוּנָא בְּאוּרְכָּא סְרָדֵי מְאָה אוּרְכָּא וְעַמּוּדוֹהִי עֶשְׂרִין וְסָמְכֵיהוֹן עְשְׂרִין דִּנְחָשָׁא וָוֵי עַמּוּדַיָּא וְכִבּוּשֵׁיהוֹן כְּסַף׃}
{And likewise for the north side in length there shall be hangings a hundred cubits long, and the pillars thereof twenty, and their sockets twenty, of brass; the hooks of the pillars and their fillets of silver.}{\arabic{verse}}
\threeverse{\arabic{verse}}%Ex.27:12
{וְרֹ֤חַב הֶֽחָצֵר֙ לִפְאַת־יָ֔ם קְלָעִ֖ים חֲמִשִּׁ֣ים אַמָּ֑ה עַמֻּדֵיהֶ֣ם עֲשָׂרָ֔ה וְאַדְנֵיהֶ֖ם עֲשָׂרָֽה׃}
{וּפוּתְיָא דְּדָרְתָא לְרוּחַ מַעְרְבָא סְרָדֵי חַמְשִׁין אַמִּין עַמּוּדֵיהוֹן עֶשְׂרָא וְסָמְכֵיהוֹן עֶשְׂרָא׃}
{And for the breadth of the court on the west side shall be hangings of fifty cubits: their pillars ten, and their sockets ten.}{\arabic{verse}}
\threeverse{\arabic{verse}}%Ex.27:13
{וְרֹ֣חַב הֶֽחָצֵ֗ר לִפְאַ֛ת קֵ֥דְמָה מִזְרָ֖חָה חֲמִשִּׁ֥ים אַמָּֽה׃
\rashi{\rashiDH{לפאת קדמה מזרחה. }פני המזרח קרוי קדם, לשון פנים, אחור, לשון אחורים, לפיכך המזרח קרוי קדם שהוא פנים, ומערב קרוי אחור, כמו דתרגם אונקלוס הַיָּם הָאַחֲרֹון (דברים יא, כד), ימא מערבא׃ }\rashi{\rashiDH{חמשים אמה. }אותן נ׳ אמה לא היו סתומים כולם בקלעים, לפי ששם הפתח, אלא ט״ו אמה קלעים לכתף הפתח מכאן, וכן לכתף השנית, נשאר רחב חלל הפתח בנתים כ׳ אמה, וזהו שנאמר וּלְשַׁעַר הֶחָצֵר מָסָךְ עֶשְׂרִים אַמָּה, וילון למסך כנגד הפתח, כ׳ אמה ארך כרוחב הפתח׃ }}
{וּפוּתְיָא דְּדָרְתָא לְרוּחַ קִדּוּמָא מַדְנְחָא חַמְשִׁין אַמִּין׃}
{And the breadth of the court on the east side eastward shall be fifty cubits.}{\arabic{verse}}
\threeverse{\arabic{verse}}%Ex.27:14
{וַחֲמֵ֨שׁ עֶשְׂרֵ֥ה אַמָּ֛ה קְלָעִ֖ים לַכָּתֵ֑ף עַמֻּדֵיהֶ֣ם שְׁלֹשָׁ֔ה וְאַדְנֵיהֶ֖ם שְׁלֹשָֽׁה׃
\rashi{\rashiDH{עמדיהם שלשה. }חמש אמות בין עמוד לעמוד, בין עמוד שבראש הדרום העומד במקצוע דרומית מזרחית, עד עמוד שהוא מן הג׳ שבמזרח ה׳ אמות, וממנו לשני חמש אמות, ומן השני לשלישי חמש אמות, וכן לכתף השנית, וארבעה עמודים למסך, הרי י׳ עמודים למזרח כנגד י׳ למערב׃ }}
{וַחֲמֵישׁ עֶשְׂרֵי אַמִּין סְרָדֵי לְעִבְרָא עַמּוּדֵיהוֹן תְּלָתָא וְסָמְכֵיהוֹן תְּלָתָא׃}
{The hangings for the one side [of the gate] shall be fifteen cubits: their pillars three, and their sockets three.}{\arabic{verse}}
\threeverse{\arabic{verse}}%Ex.27:15
{וְלַכָּתֵף֙ הַשֵּׁנִ֔ית חֲמֵ֥שׁ עֶשְׂרֵ֖ה קְלָעִ֑ים עַמֻּדֵיהֶ֣ם שְׁלֹשָׁ֔ה וְאַדְנֵיהֶ֖ם שְׁלֹשָֽׁה׃}
{וּלְעִבְרָא תִּנְיָנָא חֲמֵישׁ עֶשְׂרֵי סְרָדִין עַמּוּדֵיהוֹן תְּלָתָא וְסָמְכֵיהוֹן תְּלָתָא׃}
{And for the other side shall be hangings of fifteen cubits: their pillars three, and their sockets three.}{\arabic{verse}}
\threeverse{\arabic{verse}}%Ex.27:16
{וּלְשַׁ֨עַר הֶֽחָצֵ֜ר מָסָ֣ךְ \legarmeh  עֶשְׂרִ֣ים אַמָּ֗ה תְּכֵ֨לֶת וְאַרְגָּמָ֜ן וְתוֹלַ֧עַת שָׁנִ֛י וְשֵׁ֥שׁ מׇשְׁזָ֖ר מַעֲשֵׂ֣ה רֹקֵ֑ם עַמֻּֽדֵיהֶם֙ אַרְבָּעָ֔ה וְאַדְנֵיהֶ֖ם אַרְבָּעָֽה׃}
{וְלִתְרַע דָּרְתָא פְּרָסָא עֶשְׂרִין אַמִּין דְּתַכְלָא וְאַרְגְּוָנָא וּצְבַע זְהוֹרִי וּבוּץ שְׁזִיר עוֹבָד צַיָּיר עַמּוּדֵיהוֹן אַרְבְּעָא וְסָמְכֵיהוֹן אַרְבְּעָא׃}
{And for the gate of the court shall be a screen of twenty cubits, of blue, and purple, and scarlet, and fine twined linen, the work of the weaver in colours: their pillars four, and their sockets four.}{\arabic{verse}}
\threeverse{\aliya{מפטיר}}%Ex.27:17
{כׇּל־עַמּוּדֵ֨י הֶֽחָצֵ֤ר סָבִיב֙ מְחֻשָּׁקִ֣ים כֶּ֔סֶף וָוֵיהֶ֖ם כָּ֑סֶף וְאַדְנֵיהֶ֖ם נְחֹֽשֶׁת׃
\rashi{\rashiDH{כל עמודי החצר סביב וגו׳. }לפי שלא פירש ווין וחשוקים ואדני נחשת אלא לצפון ולדרום, אבל למזרח ולמערב לא נאמר ווין וחשוקים ואדני נחשת, לכך בא ולמד כאן׃ }}
{כָּל עַמּוּדֵי דָּרְתָא סְחוֹר סְחוֹר מְכוּבְּשִׁין כְּסַף וָוֵיהוֹן כְּסַף וְסָמְכֵיהוֹן דִּנְחָשָׁא׃}
{All the pillars of the court round about shall be filleted with silver; their hooks of silver, and their sockets of brass.}{\arabic{verse}}
\threeverse{\arabic{verse}}%Ex.27:18
{אֹ֣רֶךְ הֶֽחָצֵר֩ מֵאָ֨ה בָֽאַמָּ֜ה וְרֹ֣חַב \legarmeh  חֲמִשִּׁ֣ים בַּחֲמִשִּׁ֗ים וְקֹמָ֛ה חָמֵ֥שׁ אַמּ֖וֹת שֵׁ֣שׁ מׇשְׁזָ֑ר וְאַדְנֵיהֶ֖ם נְחֹֽשֶׁת׃
\rashi{\rashiDH{ארך החצר. }הצפון והדרום שמן המזרח למערב מאה באמה׃}\rashi{\rashiDH{ורחב חמשים בחמשים. }חצר שבמזרח היתה מרובעת חמשים על חמשים, שהמשכן ארכו שלשים ורחבו עשר, העמיד מזרח פתחו בשפת נ׳ החיצונים של אורך החצר, נמצאו כָּלוּ בחמשים הפנימים, וכלה ארכו לסוף ל׳, נמצאו כ׳ אמה ריוח לאחוריו בין הקלעים שבמערב ליריעות של אחורי המשכן, ורחב המשכן עשר אמות באמצע רוחב החצר, נמצאו לו עשרים אמה ריוח לצפון ולדרום מן קלעי החצר ליריעות המשכן, וכן למערב, וחמשים על חמשים חצר לפניו (עירובין כג׃)׃ }\rashi{\rashiDH{וקומה חמש אמות. }גובה מחיצות החצר, והוא רוחב הקלעים׃ }\rashi{\rashiDH{ואדניהם נחושת. }להביא אדני המסך, שלא תאמר לא נאמרו אדני נחושת אלא לעמודי הקלעים, אבל אדני המסך של מין אחר היו. כך נראה בעיני שלכך חזר ושנאן׃ }}
{אוּרְכָּא דְּדָרְתָא מְאָה אַמִּין וּפוּתְיָא חַמְשִׁין בְּחַמְשִׁין וְרוּמָא חֲמֵישׁ אַמִּין דְּבוּץ שְׁזִיר וְסָמְכֵיהוֹן דִּנְחָשָׁא׃}
{The length of the court shall be a hundred cubits, and the breadth fifty every where, and the height five cubits, of fine twined linen, and their sockets of brass.}{\arabic{verse}}
\threeverse{\arabic{verse}}%Ex.27:19
{לְכֹל֙ כְּלֵ֣י הַמִּשְׁכָּ֔ן בְּכֹ֖ל עֲבֹדָת֑וֹ וְכׇל־יְתֵדֹתָ֛יו וְכׇל־יִתְדֹ֥ת הֶחָצֵ֖ר נְחֹֽשֶׁת׃ \setuma         
\rashi{\rashiDH{לכל כלי המשכן. }שהיו צריכין להקמתו ולהורדתו, כגון מקבות לתקוע יתדות ועמודים׃ }\rashi{\rashiDH{יתדות. }כמין נגרי נחושת עשויין ליריעות האהל ולקלעי החצר, קשורים במיתרים סביב סביב בשפוליהן, כדי שלא תהא הרוח מגביהתן, ואיני יודע אם תחובין בארץ, או קשורין ותלויין וכובדן מכביד שפולי היריעות שלא ינועו ברוח, ואומר אני, ששמן מוכיח עליהם שהם תקועין בארץ, לכך נקראו יתדות, ומקרא זה מסייעני, אֹהֶל בַּל יִצְעָן בַּל יִסַּע יְתֵדֹתָיו לָנֶצַח (ישעיה לג, כ)׃ 
}}
{לְכֹל מָנֵי מַשְׁכְּנָא בְּכֹל פּוּלְחָנֵיהּ וְכָל סִכּוֹהִי וְכָל סִכֵּי דָּרְתָא דִּנְחָשָׁא׃}
{All the instruments of the tabernacle in all the service thereof, and all the pins thereof, and all the pins of the court, shall be of brass.}{\arabic{verse}}
\newparsha{תצוה}
\threeverse{\aliya{תצוה}}%Ex.27:20
{וְאַתָּ֞ה תְּצַוֶּ֣ה \legarmeh  אֶת־בְּנֵ֣י יִשְׂרָאֵ֗ל וְיִקְח֨וּ אֵלֶ֜יךָ שֶׁ֣מֶן זַ֥יִת זָ֛ךְ כָּתִ֖ית לַמָּא֑וֹר לְהַעֲלֹ֥ת נֵ֖ר תָּמִֽיד׃
\rashi{\rashiDH{זך. }בלי שמרים, כמו ששנינו במנחות (פו.), מגרגרו בראש הזית וכו׳׃ }\rashi{\rashiDH{כתית. }הזיתים היה כותש במכתשת ואינו טוחנן בריחים, כדי שלא יהא בו שמרים, ואחר שהוציא טפה ראשונה, מכניסן לריחים וטוחנן, והשמן השני פסול למנורה וכשר למנחות, שנאמר כתית למאור, ולא כתית למנחות׃ 
}\rashi{\rashiDH{להעלות נר תמיד. }מדליק עד שתהא שלהבת עולה מאליה (שבת כא.)׃}\rashi{\rashiDH{תמיד. }כל לילה ולילה קרוי תמיד, כמו שאתה אומר עֹלַת תָּמִיד (במדבר כח, ו), ואינה אלא מיום ליום. וכן במנחת חביתין נאמר תמיד (ויקרא ו, יג), ואינה אלא מַחֲצִיתָהּ בַּבֹּקֶר וּמַחֲצִיתָהּ בָּעָרֶב, אבל תמיד האמור בלחם הפנים, משבת לשבת הוא׃ 
}}
{וְאַתְּ תְּפַקֵּיד יָת בְּנֵי יִשְׂרָאֵל וְיִסְּבוּן לָךְ מְשַׁח זֵיתָא דָּכְיָא כָּתִישָׁא לְאַנְהָרָא לְאַדְלָקָא בּוֹצִינַיָּא תְּדִירָא׃}
{And thou shalt command the children of Israel, that they bring unto thee pure olive oil beaten for the light, to cause a lamp to burn continually.}{\arabic{verse}}
\threeverse{\arabic{verse}}%Ex.27:21
{בְּאֹ֣הֶל מוֹעֵד֩ מִח֨וּץ לַפָּרֹ֜כֶת אֲשֶׁ֣ר עַל־הָעֵדֻ֗ת יַעֲרֹךְ֩ אֹת֨וֹ אַהֲרֹ֧ן וּבָנָ֛יו מֵעֶ֥רֶב עַד־בֹּ֖קֶר לִפְנֵ֣י יְהֹוָ֑ה חֻקַּ֤ת עוֹלָם֙ לְדֹ֣רֹתָ֔ם מֵאֵ֖ת בְּנֵ֥י יִשְׂרָאֵֽל׃ \setuma         
\rashi{\rashiDH{מערב עד בוקר. }תן לה מדתה שתהא דולקת מערב ועד בוקר, ושיערו חכמים חצי לוג ללילי טבת הארוכין, וכן לכל הלילות, ואם יותר אין בכך כלום׃ }}
{בְּמַשְׁכַּן זִמְנָא מִבַּרָא לְפָרוּכְתָּא דְּעַל סָהֲדוּתָא יַסְדַּר יָתֵיהּ אַהֲרֹן וּבְנוֹהִי מֵרַמְשָׁא עַד צַפְרָא קֳדָם יְיָ קְיָם עָלַם לְדָרֵיהוֹן מִן בְּנֵי יִשְׂרָאֵל׃}
{In the tent of meeting, without the veil which is before the testimony, Aaron and his sons shall set it in order, to burn from evening to morning before the \lord; it shall be a statute for ever throughout their generations on the behalf of the children of Israel.}{\arabic{verse}}
\newperek
\threeverse{\Roman{chap}}%Ex.28:1
{וְאַתָּ֡ה הַקְרֵ֣ב אֵלֶ֩יךָ֩ אֶת־אַהֲרֹ֨ן אָחִ֜יךָ וְאֶת־בָּנָ֣יו אִתּ֗וֹ מִתּ֛וֹךְ בְּנֵ֥י יִשְׂרָאֵ֖ל לְכַהֲנוֹ־לִ֑י אַהֲרֹ֕ן נָדָ֧ב וַאֲבִיה֛וּא אֶלְעָזָ֥ר וְאִיתָמָ֖ר בְּנֵ֥י אַהֲרֹֽן׃
\rashi{\rashiDH{ואתה הקרב אליך. }לאחר שתגמור מלאכת המשכן׃ 
}}
{וְאַתְּ קָרֵיב לְוָתָךְ יָת אַהֲרֹן אֲחוּךְ וְיָת בְּנוֹהִי עִמֵּיהּ מִגּוֹ בְנֵי יִשְׂרָאֵל לְשַׁמָּשָׁא קֳדָמָי אַהֲרֹן נָדָב וַאֲבִיהוּא אֶלְעָזָר וְאִיתָמָר בְּנֵי אַהֲרֹן׃}
{And bring thou near unto thee Aaron thy brother, and his sons with him, from among the children of Israel, that they may minister unto Me in the priest’s office, even Aaron, Nadab and Abihu, Eleazar and Ithamar, Aaron’s sons.}{\Roman{chap}}
\threeverse{\arabic{verse}}%Ex.28:2
{וְעָשִׂ֥יתָ בִגְדֵי־קֹ֖דֶשׁ לְאַהֲרֹ֣ן אָחִ֑יךָ לְכָב֖וֹד וּלְתִפְאָֽרֶת׃}
{וְתַעֲבֵיד לְבוּשֵׁי קוּדְשָׁא לְאַהֲרֹן אֲחוּךְ לִיקָר וּלְתוּשְׁבְּחָא׃}
{And thou shalt make holy garments for Aaron thy brother, for splendour and for beauty.}{\arabic{verse}}
\threeverse{\arabic{verse}}%Ex.28:3
{וְאַתָּ֗ה תְּדַבֵּר֙ אֶל־כׇּל־חַכְמֵי־לֵ֔ב אֲשֶׁ֥ר מִלֵּאתִ֖יו ר֣וּחַ חׇכְמָ֑ה וְעָשׂ֞וּ אֶת־בִּגְדֵ֧י אַהֲרֹ֛ן לְקַדְּשׁ֖וֹ לְכַהֲנוֹ־לִֽי׃
\rashi{\rashiDH{לקדשו לכהנו לי. }לקדשו להכניסו בכהונה על ידי הבגדים, שיהא כהן לי, ולשון כהונה שירות הוא, שנטריא״ה בלע״ז 
}}
{וְאַתְּ תְּמַלֵּיל עִם כָּל חַכִּימֵי לִבָּא דְּאַשְׁלֵימִית עִמְּהוֹן רוּחַ חָכְמָא וְיַעְבְּדוּן יָת לְבוּשֵׁי אַהֲרֹן לְקַדָּשׁוּתֵיהּ לְשַׁמָּשָׁא קֳדָמָי׃}
{And thou shalt speak unto all that are wise-hearted, whom I have filled with the spirit of wisdom, that they make Aaron’s garments to sanctify him, that he may minister unto Me in the priest’s office.}{\arabic{verse}}
\threeverse{\arabic{verse}}%Ex.28:4
{וְאֵ֨לֶּה הַבְּגָדִ֜ים אֲשֶׁ֣ר יַעֲשׂ֗וּ חֹ֤שֶׁן וְאֵפוֹד֙ וּמְעִ֔יל וּכְתֹ֥נֶת תַּשְׁבֵּ֖ץ מִצְנֶ֣פֶת וְאַבְנֵ֑ט וְעָשׂ֨וּ בִגְדֵי־קֹ֜דֶשׁ לְאַהֲרֹ֥ן אָחִ֛יךָ וּלְבָנָ֖יו לְכַהֲנוֹ־לִֽי׃
\rashi{\rashiDH{חושן. }תכשיט כנגד הלב׃}\rashi{\rashiDH{ואפוד. }לא שמעתי (שהוא לבוש)ולא מצאתי בברייתא פירוש תבניתו, ולבי אומר לי שהוא חגור לו מאחוריו, רחבו כרוחב גב איש, כמין סינר שקורין פורצי״נט שחוגרות הַשָּׂרוֹת כשרוכבות על הסוסים, כך מעשהו מלמטה, שנאמר וְדָוִד חָגוּר אֵפֹוד בָּד (שמואל־ב ו, יד), למדנו שהאפוד חגורה היא. ואי אפשר לומר שאין בו אלא חגורה לבדה, שהרי נאמר וַיִּתֵּן עָלָיו אֶת הָאפֹד (ויקרא ח, ז) ואחר כך וַיַּחֲגֹּר אֹתֹו בְּחֵשֶׁב הָאֵפֹד (שם), ותרגם אונקלוס בְּהֶמְיַן אֵפוֹדָא, למדנו שהחשב הוא החגור, והאפוד שם תכשיט לבדו. ואי אפשר לומר שעל שם שתי הכתפות שבו הוא קרוי אפוד, שהרי נאמר שתי כתפות האפוד, למדנו שהאפוד שם לבד, והכתפות שם לבד, והחשב שם לבד. לכך אני אומר שעל שם הסינר של מטה קרוי אפוד, על שם שאופדו ומקשטו בו, כמו שנאמר ויאפד לו בו (שם), והחשב הוא חגור שלמעלה הימנו, והכתפות קבועות בו. ועוד אומר לי לבי, שיש ראיה שהוא מין לבוש, שתרגם יונתן ודוד חגור אפוד בד, כַּרְדוּט דְּבוּץ, ותרגם כמו כן מעילין, כַּרְדוּטִין, במעשה תמר אחות אבשלום, כִּי כֵן תִּלְבַּשְׁןָ בְּנֹות הַמֶּלֶךְ הַבְּתוּלֹת מְעִילִים (שמואל־ב יג, יח)׃ }\rashi{\rashiDH{ומעיל. }הוא כמין חלוק, וכן הכתונת, אלא שהכתונת סמוך לבשרו, ומעיל קרוי חלוק העליון׃ }\rashi{\rashiDH{תשבץ. }עשויין משבצות לנוי, והמשבצות הם כמין גומות העשויות בתכשיטי זהב למושב קביעת אבנים טובות ומרגליות, כמו שנאמר באבני האפוד מֻסַּבֹּת מִשְׁבְּצֹות זָהָב, ובלע״ז קוראין אותו קשטונ״ש }\rashi{\rashiDH{מצנפת. }כמין כיפת כובע שקורין קופי״א, שהרי במקום אחר קורא להם מגבעות, ומתרגמינן כּוֹבָעִין (עיין יומא כה.)׃ }\rashi{\rashiDH{ואבנט. }היא חגורה על הכתונת, והאפוד חגורה על המעיל, כמו שמצינו בסדר לבישתן, וַיִּתֵּן עָלָיו אֶת הַכֻּתֹּנֶת וַיַחֲגֹּר אֹתֹו בָּאַבְנֵט וַיַּלְבֵּשׁ אֹתֹו אֶת הַמְּעִיל וַיִּתֵּן עָלָיו אֶת הָאֵפֹד׃ }\rashi{\rashiDH{בגדי קדש. }מתרומה המקודשת לשמי יעשו אותם׃}}
{וְאִלֵּין לְבוּשַׁיָּא דְּיַעְבְּדוּן חוּשְׁנָא וְאֵיפוֹדָא וּמְעִילָא וְכִתּוּנִין מְרַמְּצָן מַצְנְפָן וְהִמְיָנִין וְיַעְבְּדוּן לְבוּשֵׁי קוּדְשָׁא לְאַהֲרֹן אֲחוּךְ וְלִבְנוֹהִי לְשַׁמָּשָׁא קֳדָמָי׃}
{And these are the garments which they shall make: a breastplate, and an ephod, and a robe, and a tunic of chequer work, a mitre, and a girdle; and they shall make holy garments for Aaron thy brother, and his sons, that he may minister unto Me in the priest’s office.}{\arabic{verse}}
\threeverse{\arabic{verse}}%Ex.28:5
{וְהֵם֙ יִקְח֣וּ אֶת־הַזָּהָ֔ב וְאֶת־הַתְּכֵ֖לֶת וְאֶת־הָֽאַרְגָּמָ֑ן וְאֶת־תּוֹלַ֥עַת הַשָּׁנִ֖י וְאֶת־הַשֵּֽׁשׁ׃ \petucha 
\rashi{\rashiDH{והם יקחו. }אותם חכמי לב שיעשו הבגדים, יקבלו מן המתנדבים את הזהב ואת התכלת, לעשות מהן את הבגדים׃ }}
{וְאִנּוּן יִסְּבוּן יָת דַּהְבָּא וְיָת תַּכְלָא וְיָת אַרְגְּוָנָא וְיָת צְבַע זְהוֹרִי וְיָת בּוּצָא׃}
{And they shall take the gold, and the blue, and the purple, and the scarlet, and the fine linen.}{\arabic{verse}}
\threeverse{\aliya{לוי}}%Ex.28:6
{וְעָשׂ֖וּ אֶת־הָאֵפֹ֑ד זָ֠הָ֠ב תְּכֵ֨לֶת וְאַרְגָּמָ֜ן תּוֹלַ֧עַת שָׁנִ֛י וְשֵׁ֥שׁ מׇשְׁזָ֖ר מַעֲשֵׂ֥ה חֹשֵֽׁב׃
\rashi{\rashiDH{ועשו את האפוד. }אם באתי לפרש מעשה האפוד והחשן על סדר המקראות, הרי פירושן פרקים וישגה הקורא בצרופן, לכך אני כותב מעשיהם כמות שהוא, למען ירוץ הקורא בו, ואחר כך אפרש על סדר המקראות. האפוד עשוי כמין סינר של נשים רוכבות סוסים, וחוגר אותו מאחוריו כנגד לבו למטה מאציליו, רחבו כמדת רוחב גבו של אדם ויותר, ומגיע עד עקביו, והחשב מחובר בראשו על פני רחבו מעשה אורג, ומאריך לכאן ולכאן כדי להקיף ולחגור בו, והכתפות מחוברות בחשב, אחת לימין ואחת לשמאל, מאחורי הכהן לשני קצות רחבו של סינר, וכשזוקפן עומדות לו על שני כתפיו, והן כמין שתי רצועות עשויות ממין האפוד, ארוכות כדי שיעור לזקפן אצל צוארו מכאן ומכאן, ונקפלות לפניו למטה מכתפיו מעט, ואבני השהם קבועות בהם, אחת על כתף ימין ואחת על כתף שמאל, והמשבצות נתונות בראשיהם לפני כתפיו, ושתי עבותות הזהב תחובות בשתי טבעות שבחשן בשני קצות רחבו העליון, אחת לימין ואחת לשמאל, ושני ראשי השרשרות תקועים במשבצות לימין, וכן שני ראשי השרשרות השמאלית תקועין במשבצות שבכתף שמאל, נמצא החשן תלוי במשבצות האפוד על לבו מלפניו, ועוד שתי טבעות בשני קצות החשן בתחתיתו, וכנגדם שתי טבעות בשתי כתפות האפוד מלמטה, בראשו התחתון המחובר בחשב, טבעות החשן אל מול טבעות האפוד שוכבים זה על זה, ומרכסן בפתיל תכלת תחוב בטבעות האפוד והחשן, שיהא תחתית החשן דבוק לחשב האפוד, ולא יהא נד ונבדל, הולך וחוזר׃ }\rashi{\rashiDH{זהב תכלת וארגמן ותולעת שני ושש משזר. }חמשה מינים הללו שזורין בכל חוט וחוט. היו מרדדין את הזהב כמין טסים דקין, וקוצצין פתילים מהם, וטווין אותן חוט של זהב עם שש חוטין של תכלת, וחוט של זהב עם שש חוטין של ארגמן, וכן בתולעת שני, וכן בשש, שכל המינין חוטן כפול ששה וחוט של זהב עם כל אחד ואחד, ואחר כך שוזר את כולם כאחד, נמצא חוטן כפול כ״ח, וכן מפורש במס׳ יומא (עב.), ולמד מן המקרא הזה וירקעו את פחי הזהב וקצץ פתילים לעשות (את פתילי הזהב) בתוך התכלת ובתוך הארגמן וגו׳, למדנו שחוט של זהב שזור עם כל מין ומין׃ }\rashi{\rashiDH{מעשה חושב. }כבר פרשתי שהוא אריגת שתי קירות שאין צורת שני עבריהם דומות זו לזו׃}}
{וְיַעְבְּדוּן יָת אֵיפוֹדָא דַּהְבָּא תַּכְלָא וְאַרְגְּוָונָא צְבַע זְהוֹרִי וּבוּץ שְׁזִיר עוֹבָד אוּמָּן׃}
{And they shall make the ephod of gold, of blue, and purple, scarlet, and fine twined linen, the work of the skilful workman.}{\arabic{verse}}
\threeverse{\arabic{verse}}%Ex.28:7
{שְׁתֵּ֧י כְתֵפֹ֣ת חֹֽבְרֹ֗ת יִֽהְיֶה־לּ֛וֹ אֶל־שְׁנֵ֥י קְצוֹתָ֖יו וְחֻבָּֽר׃
\rashi{\rashiDH{שתי כתפות וגו׳. }הסינר מלמטה, וחשב האפוד היא החגורה, וצמודה לו מלמעלה דוגמת סינר הנשים, ומגבו של כהן היו מחוברות בחשב שתי חתיכות כמין שתי רצועות רחבות, אחת כנגד כל כתף וכתף, וזוקפן על שתי כתפותיו עד שנקפלות לפניו כנגד החזה, ועל ידי חבורן לטבעות החשן נאחזין מלפניו כנגד לבו שאין נופלת, כמו שמפורש בענין, והיו זקופות והולכות כנגד כתפיו, ושתי אבני שהם קבועות בהן, אחת בכל אחת׃ }\rashi{\rashiDH{אל שני קצותיו. }אל רחבו של אפוד, שלא היה רחבו אלא כנגד גבו של כהן, וגבהו עד כנגד האצילים שקורין קודי״ש, שנאמר לֹא יַחְגְּרוּ בַּיָּזַע (יחזקאל מד, יח), אין חוגרין במקום זיעה, לא למעלה מאציליהם ולא למטה ממתניהם, אלא כנגד אציליהם׃ }\rashi{\rashiDH{וחבר. }האפוד עם אותן שתי כתפות האפוד יחבר אותם במחט למטה בחשב, ולא יארגם עמו, אלא אורגם לבד ואחר כך מחברם׃ }}
{תְּרֵין כִּתְפִין מְלָפְפָן יְהוֹן לֵיהּ עַל תְּרֵין סִטְרוֹהִי וְיִתְלָפַף׃}
{It shall have two shoulder-pieces joined to the two ends thereof, that it may be joined together.}{\arabic{verse}}
\threeverse{\arabic{verse}}%Ex.28:8
{וְחֵ֤שֶׁב אֲפֻדָּתוֹ֙ אֲשֶׁ֣ר עָלָ֔יו כְּמַעֲשֵׂ֖הוּ מִמֶּ֣נּוּ יִהְיֶ֑ה זָהָ֗ב תְּכֵ֧לֶת וְאַרְגָּמָ֛ן וְתוֹלַ֥עַת שָׁנִ֖י וְשֵׁ֥שׁ מׇשְׁזָֽר׃
\rashi{\rashiDH{וחשב אפדתו. }וחגור שעל ידו, הוא מאפדו ומתקנהו לכהן ומקשטו׃ }\rashi{\rashiDH{אשר עליו. }למעלה בשפת הסינר היא החגורה׃}\rashi{\rashiDH{כמעשהו. }כאריגת הסינר מעשה חושב ומחמשת מינין, כך אריגת החשב מעשה חושב ומחמשת המינים׃ }\rashi{\rashiDH{ממנו יהיה. }עמו יהיה ארוג, ולא יארגנו לבד ויחברנו׃ 
}}
{וְהִמְיַן תִּקּוּנֵיהּ דַּעֲלוֹהִי כְּעוֹבָדֵיהּ מִנֵּיהּ יְהֵי דַּהְבָּא תַּכְלָא וְאַרְגְּוָנָא וּצְבַע זְהוֹרִי וּבוּץ שְׁזִיר׃}
{And the skilfully woven band, which is upon it, wherewith to gird it on, shall be like the work thereof and of the same piece: of gold, of blue, and purple, and scarlet, and fine twined linen.}{\arabic{verse}}
\threeverse{\arabic{verse}}%Ex.28:9
{וְלָ֣קַחְתָּ֔ אֶת־שְׁתֵּ֖י אַבְנֵי־שֹׁ֑הַם וּפִתַּחְתָּ֣ עֲלֵיהֶ֔ם שְׁמ֖וֹת בְּנֵ֥י יִשְׂרָאֵֽל׃}
{וְתִסַּב יָת תַּרְתֵּין אַבְנֵי בוּרְלָא וְתִגְלוֹף עֲלֵיהוֹן שְׁמָהָת בְּנֵי יִשְׂרָאֵל׃}
{And thou shalt take two onyx stones, and grave on them the names of the children of Israel:}{\arabic{verse}}
\threeverse{\aliya{ישראל}}%Ex.28:10
{שִׁשָּׁה֙ מִשְּׁמֹתָ֔ם עַ֖ל הָאֶ֣בֶן הָאֶחָ֑ת וְאֶת־שְׁמ֞וֹת הַשִּׁשָּׁ֧ה הַנּוֹתָרִ֛ים עַל־הָאֶ֥בֶן הַשֵּׁנִ֖ית כְּתוֹלְדֹתָֽם׃
\rashi{\rashiDH{כתולדותם. }כסדר שנולדו, ראובן שמעון לוי יהודה דן ונפתלי, על האחת, ועל השניה, גד אשר יששכר זבולן יוסף, ובנימין מלא, שכן הוא כתוב במקום תולדותו, כ״ה אותיות בכל אחת ואחת׃ }}
{שִׁתָּא מִשְּׁמָהָתְהוֹן עַל אַבְנָא חֲדָא וְיָת שְׁמָהָת שִׁתָּא דְּאִשְׁתְּאַרוּ עַל אַבְנָא תִּנְיֵיתָא כְּתוֹלְדָתְהוֹן׃}
{six of their names on the one stone, and the names of the six that remain on the other stone, according to their birth.}{\arabic{verse}}
\threeverse{\arabic{verse}}%Ex.28:11
{מַעֲשֵׂ֣ה חָרַשׁ֮ אֶ֒בֶן֒ פִּתּוּחֵ֣י חֹתָ֗ם תְּפַתַּח֙ אֶת־שְׁתֵּ֣י הָאֲבָנִ֔ים עַל־שְׁמֹ֖ת בְּנֵ֣י יִשְׂרָאֵ֑ל מֻסַבֹּ֛ת מִשְׁבְּצ֥וֹת זָהָ֖ב תַּעֲשֶׂ֥ה אֹתָֽם׃
\rashi{\rashiDH{מעשה חרש אבן. }מעשה אומן של אבנים. חרש זה, דבוק הוא לתיבה שלאחריו, ולפיכך הוא נקוד פתח בסופו, וכן חָרַשׁ עֵצִים נָטָה קָו (ישעיה מד, יג), חרש של עצים. וכן חָרַשׁ בַּרְזֶל מַעֲצָד (שם יב), כל אלה דבוקים ופתוחים׃ }\rashi{\rashiDH{פתוחי חותם. }כתרגומו כְּתַב מְפָרָשׁ כִּגְלֹף דְּעִזְקָא, חרוצות האותיות בתוכן, כמו שחורצין חותמי טבעות שהם לחתום אגרות, כתב ניכר ומפורש׃ }\rashi{\rashiDH{על שמות. }כמו בשמות׃}\rashi{\rashiDH{מסבות משבצות. }מוקפות האבנים במשבצות זהב, שעושה מושב האבן בזהב כמין גומא למדת האבן, ומשקעה במשבצות, נמצאת המשבצת סובבת את האבן סביב, ומחבר המשבצות בכתפות האפוד׃ }}
{עוֹבָד אוּמָּן אֶבֶן טָבָא כְּתָב מְפָרַשׁ כִּגְלָף דְּעִזְקָא תִּגְלוֹף יָת תַּרְתֵּין אַבְנַיָּא עַל שְׁמָהָת בְּנֵי יִשְׂרָאֵל מְשַׁקְּעָן מְרַמְּצָן דִּדְהַב תַּעֲבֵיד יָתְהוֹן׃}
{With the work of an engraver in stone, like the engravings of a signet, shalt thou engrave the two stones, according to the names of the children of Israel; thou shalt make them to be inclosed in settings of gold.}{\arabic{verse}}
\threeverse{\arabic{verse}}%Ex.28:12
{וְשַׂמְתָּ֞ אֶת־שְׁתֵּ֣י הָאֲבָנִ֗ים עַ֚ל כִּתְפֹ֣ת הָֽאֵפֹ֔ד אַבְנֵ֥י זִכָּרֹ֖ן לִבְנֵ֣י יִשְׂרָאֵ֑ל וְנָשָׂא֩ אַהֲרֹ֨ן אֶת־שְׁמוֹתָ֜ם לִפְנֵ֧י יְהֹוָ֛ה עַל־שְׁתֵּ֥י כְתֵפָ֖יו לְזִכָּרֹֽן׃ \setuma         
\rashi{\rashiDH{לזכרון. }שיהא רואה הקב״ה את השבטים כתובים לפניו, ויזכור צדקתם׃ }}
{וּתְשַׁוֵּי יָת תַּרְתֵּין אַבְנַיָּא עַל כִּתְפֵי אֵיפוֹדָא אַבְנֵי דּוּכְרָנָא לִבְנֵי יִשְׂרָאֵל וְיִטּוֹל אַהֲרֹן יָת שְׁמָהָתְהוֹן קֳדָם יְיָ עַל תְּרֵין כִּתְפוֹהִי לְדוּכְרָנָא׃}
{And thou shalt put the two stones upon the shoulder-pieces of the ephod, to be stones of memorial for the children of Israel; and Aaron shall bear their names before the \lord\space upon his two shoulders for a memorial.}{\arabic{verse}}
\threeverse{\aliya{שני}}%Ex.28:13
{וְעָשִׂ֥יתָ מִשְׁבְּצֹ֖ת זָהָֽב׃
\rashi{\rashiDH{ועשית משבצות. }מיעוט משבצות שתים, ולא פירש לך עתה בפרשה זו אלא מקצת צרכן, ובפרשת החשן גומר לך פירושן׃ }}
{וְתַעֲבֵיד מְרַמְּצָן דִּדְהַב׃}
{And thou shalt make settings of gold;}{\arabic{verse}}
\threeverse{\arabic{verse}}%Ex.28:14
{וּשְׁתֵּ֤י שַׁרְשְׁרֹת֙ זָהָ֣ב טָה֔וֹר מִגְבָּלֹ֛ת תַּעֲשֶׂ֥ה אֹתָ֖ם מַעֲשֵׂ֣ה עֲבֹ֑ת וְנָתַתָּ֛ה אֶת־שַׁרְשְׁרֹ֥ת הָעֲבֹתֹ֖ת עַל־הַֽמִּשְׁבְּצֹֽת׃ \setuma         
\rashi{\rashiDH{שרשרות זהב. }שלשלאות׃}\rashi{\rashiDH{מגבלות. }לסוף גבול החשן תעשה אותם׃}\rashi{\rashiDH{מעשה עבות. }מעשה קליעת חוטין, ולא מעשה נקבים וכפלים כאותן שעושין לבורות, אלא כאותן שעושין לְעַרְדַּסְקָאוֹת שקורין אינשינשייר״ש (רוכפאס) (ביצה כב׃)׃ }\rashi{\rashiDH{ונתתה את שרשרות. }של עבותות העשויות מעשה עבות על משבצות הללו. ולא זה הוא מקום צוואת עשייתן של שרשרות ולא צוואת קביעותן, ואין תעשה האמור כאן לשון צווי, ואין ונתתה האמור כאן לשון צווי, אלא לשון עתיד, כי בפרשת החשן חוזר ומצוהו על עשייתן ועל קביעותן, ולא נכתב כאן אלא להודיעך מקצת צורך המשבצות שצוה לעשות עם האפוד, וכתב לך זאת, לומר לך המשבצות הללו יוזקקו לך, לכשתעשה שרשרות מגבלות על החשן, תתנם על המשבצות הללו׃ }}
{וְתַרְתֵּין תִּכִּין דִּדְהַב דְּכֵי מְתַחֲמָן תַּעֲבֵיד יָתְהוֹן עוֹבָד גְּדִילוּ וְתִתֵּין יָת תִּכַּיָּא גְּדִילָתָא עַל מְרַמְּצָתָא׃}
{and two chains of pure gold; of plaited thread shalt thou make them, of wreathen work; and thou shalt put the wreathen chains on the settings.}{\arabic{verse}}
\threeverse{\arabic{verse}}%Ex.28:15
{וְעָשִׂ֜יתָ חֹ֤שֶׁן מִשְׁפָּט֙ מַעֲשֵׂ֣ה חֹשֵׁ֔ב כְּמַעֲשֵׂ֥ה אֵפֹ֖ד תַּעֲשֶׂ֑נּוּ זָ֠הָ֠ב תְּכֵ֨לֶת וְאַרְגָּמָ֜ן וְתוֹלַ֧עַת שָׁנִ֛י וְשֵׁ֥שׁ מׇשְׁזָ֖ר תַּעֲשֶׂ֥ה אֹתֽוֹ׃
\rashi{\rashiDH{חושן משפט. }שמכפר על קלקול הדין. דבר אחר משפט, שמברר דבריו והבטחתו אמת, דרישנמ״ט בלע״ז שהמשפט משמש ג׳ לשונות, דברי טענות בעלי הדין, וגמר הדין, ועונש הדין, אם עונש מיתה אם עונש מכות אם עונש ממון, וזה משמש לשון בירור דברים, שמפרש ומברר דבריו׃ }\rashi{\rashiDH{כמעשה אפוד. }מעשה חושב ומחמשת מינין׃ 
}}
{וְתַעֲבֵיד חֹשֶׁן דִּינָא עוֹבָד אוּמָּן כְּעוֹבָד אֵיפוֹדָא תַּעְבְּדִנֵּיהּ דַּהְבָּא תַּכְלָא וְאַרְגְּוָנָא וּצְבַע זְהוֹרִי וּבוּץ שְׁזִיר תַּעֲבֵיד יָתֵיהּ׃}
{And thou shalt make a breastplate of judgment, the work of the skilful workman; like the work of the ephod thou shalt make it: of gold, of blue, and purple, and scarlet, and fine twined linen, shalt thou make it.}{\arabic{verse}}
\threeverse{\arabic{verse}}%Ex.28:16
{רָב֥וּעַ יִֽהְיֶ֖ה כָּפ֑וּל זֶ֥רֶת אׇרְכּ֖וֹ וְזֶ֥רֶת רׇחְבּֽוֹ׃
\rashi{\rashiDH{זרת ארכו וזרת רחבו. }כפול, ומוטל לו לפניו כנגד לבו, שנאמר וְהָיוּ עַל לֵב אַהֲרֹן, תלוי בכתפות האפוד הבאות מאחוריו על כתפיו, ונקפלות ויורדות לפניו מעט, והחשן תלוי בהן בשרשרות וטבעות, כמו שמפורש בענין׃ }}
{מְרוּבַּע יְהֵי עִיף זַרְתָּא אוּרְכֵּיהּ וְזַרְתָּא פוּתְיֵיהּ׃}
{Four-square it shall be and double: a span shall be the length thereof, and a span the breadth thereof.}{\arabic{verse}}
\threeverse{\arabic{verse}}%Ex.28:17
{וּמִלֵּאתָ֥ בוֹ֙ מִלֻּ֣אַת אֶ֔בֶן אַרְבָּעָ֖ה טוּרִ֣ים אָ֑בֶן ט֗וּר אֹ֤דֶם פִּטְדָה֙ וּבָרֶ֔קֶת הַטּ֖וּר הָאֶחָֽד׃
\rashi{\rashiDH{ומלאת בו. }על שם שהאבנים ממלאות גומות המשבצות המתוקנות להן, קורא אותן בלשון מלואים׃ }}
{וְתַשְׁלֵים בֵּיהּ אַשְׁלָמוּת אַבְנָא אַרְבְּעָא סִדְרִין דְּאֶבֶן טָבָא סִדְרָא קַדְמָאָה סָמְקָן יָרְקָן וּבָרְקָן סִדְרָא חַד׃}
{And thou shalt set in it settings of stones, four rows of stones: a row of carnelian, topaz, and smaragd shall be the first row;}{\arabic{verse}}
\threeverse{\arabic{verse}}%Ex.28:18
{וְהַטּ֖וּר הַשֵּׁנִ֑י נֹ֥פֶךְ סַפִּ֖יר וְיָהֲלֹֽם׃}
{וְסִדְרָא תִּנְיָנָא אִזְמַרַגְדִּין שַׁבְזֵיז וְסַבְהֲלוֹם׃}
{and the second row a carbuncle, a sapphire, and an emerald;}{\arabic{verse}}
\threeverse{\arabic{verse}}%Ex.28:19
{וְהַטּ֖וּר הַשְּׁלִישִׁ֑י לֶ֥שֶׁם שְׁב֖וֹ וְאַחְלָֽמָה׃}
{וְסִדְרָא תְּלִיתָאָה קַנְכֵּירִי טְרַקְיָא וְעֵין עִגְלָא׃}
{and the third row a jacinth, an agate, and an amethyst;}{\arabic{verse}}
\threeverse{\arabic{verse}}%Ex.28:20
{וְהַטּוּר֙ הָרְבִיעִ֔י תַּרְשִׁ֥ישׁ וְשֹׁ֖הַם וְיָשְׁפֵ֑ה מְשֻׁבָּצִ֥ים זָהָ֛ב יִהְי֖וּ בְּמִלּוּאֹתָֽם׃
\rashi{\rashiDH{משבצים זהב. }יהיו הטורים במלואותם, מוקפים משבצות זהב בעומק שיעור שיתמלא בעובי האבן, זהו לשון במלואותם, כשיעור מלוי עביין של אבנים יהיה עומק המשבצות, לא פחות ולא יותר׃ }}
{וְסִדְרָא רְבִיעָאָה כְּרוּם יַמָּא וּבוּרְלָא וּפַנְתֵּירִי מְרַמְּצָן בִּדְהַב יְהוֹן בְּאַשְׁלָמוּתְהוֹן׃}
{and the fourth row a beryl, and an onyx, and a jasper; they shall be inclosed in gold in their settings.}{\arabic{verse}}
\threeverse{\arabic{verse}}%Ex.28:21
{וְ֠הָאֲבָנִ֠ים תִּֽהְיֶ֜יןָ עַל־שְׁמֹ֧ת בְּנֵֽי־יִשְׂרָאֵ֛ל שְׁתֵּ֥ים עֶשְׂרֵ֖ה עַל־שְׁמֹתָ֑ם פִּתּוּחֵ֤י חוֹתָם֙ אִ֣ישׁ עַל־שְׁמ֔וֹ תִּֽהְיֶ֕יןָ לִשְׁנֵ֥י עָשָׂ֖ר שָֽׁבֶט׃
\rashi{\rashiDH{איש על שמו. }כסדר תולדותם סדר האבנים, אודם לראובן, פטדה לשמעון, וכן כלם׃ }}
{וְאַבְנַיָּא יִהְוְיָן עַל שְׁמָהָת בְּנֵי יִשְׂרָאֵל תַּרְתַּא עֶשְׂרֵי עַל שְׁמָהָתְהוֹן כְּתָב מְפָרַשׁ כִּגְלָף דְּעִזְקָא גְּבַר עַל שְׁמֵיהּ יִהְוְיָן לִתְרֵי עֲשַׂר שִׁבְטִין׃}
{And the stones shall be according to the names of the children of Israel, twelve, according to their names; like the engravings of a signet, every one according to his name, they shall be for the twelve tribes.}{\arabic{verse}}
\threeverse{\arabic{verse}}%Ex.28:22
{וְעָשִׂ֧יתָ עַל־הַחֹ֛שֶׁן שַֽׁרְשֹׁ֥ת גַּבְלֻ֖ת מַעֲשֵׂ֣ה עֲבֹ֑ת זָהָ֖ב טָהֽוֹר׃
\rashi{\rashiDH{על החושן. }בשביל החשן, לקבעם בטבעותיו, כמו שמפורש למטה בענין׃ }\rashi{\rashiDH{שרשת. }לשון שרשי אילן, המאחיזין לאילן להאחז ולהתקע בארץ, אף אלו יהיו מאחיזין לחשן, שבהם יהיה תלוי באפוד, והן שתי שרשרות האמורות למעלה בענין המשבצות, ואף שרשרות פתר מנחם בן סרוק לשון שרשים, ואמר שהרי״ש יתירה, כמו מ״ם שבשלשום, ומ״ם שבריקם, ואיני רואה את דבריו, אלא שרשרת בלשון עברית כשלשלת בלשון משנה׃ }\rashi{\rashiDH{גבלת. }הוא מגבלות האמור למעלה, שתתקעם בטבעות שיהיו בגבול החשן, וכל גבול לשון קצה, אשומי״ל בלע״ז }\rashi{\rashiDH{מעשה עבות. }מעשה קליעה׃}}
{וְתַעֲבֵיד עַל חוּשְׁנָא תִּכִּין מְתַחֲמָן עוֹבָד גְּדִילוּ דִּדְהַב דְּכֵי׃}
{And thou shalt make upon the breastplate plaited chains of wreathen work of pure gold.}{\arabic{verse}}
\threeverse{\arabic{verse}}%Ex.28:23
{וְעָשִׂ֙יתָ֙ עַל־הַחֹ֔שֶׁן שְׁתֵּ֖י טַבְּע֣וֹת זָהָ֑ב וְנָתַתָּ֗ אֶת־שְׁתֵּי֙ הַטַּבָּע֔וֹת עַל־שְׁנֵ֖י קְצ֥וֹת הַחֹֽשֶׁן׃
\rashi{\rashiDH{על החושן. }לצורך החושן, כדי לקבעם בו. ולא יתכן לומר שתהא תחלת עשייתן עליו, שאם כן מה הוא שחוזר ואומר ונתת את שתי הטבעות, והלא כבר נתונים בו, היה לו לכתוב בתחלת המקרא, ועשית על קצות החשן שתי טבעות זהב, ואף בשרשרות צריך אתה לפתור כן׃ }\rashi{\rashiDH{על שני קצות החושן. }לשתי פאות שכנגד הצואר לימנית ולשמאלית הבאים מול כתפות האפוד׃}}
{וְתַעֲבֵיד עַל חוּשְׁנָא תַּרְתֵּין עִזְקָן דִּדְהַב וְתִתֵּין יָת תַּרְתֵּין עִזְקָתָא עַל תְּרֵין סִטְרֵי חוּשְׁנָא׃}
{And thou shalt make upon the breastplate two rings of gold, and shalt put the two rings on the two ends of the breastplate.}{\arabic{verse}}
\threeverse{\arabic{verse}}%Ex.28:24
{וְנָתַתָּ֗ה אֶת־שְׁתֵּי֙ עֲבֹתֹ֣ת הַזָּהָ֔ב עַל־שְׁתֵּ֖י הַטַּבָּעֹ֑ת אֶל־קְצ֖וֹת הַחֹֽשֶׁן׃
\rashi{\rashiDH{ונתתה את שתי עבותות הזהב. }הן הן שרשות גבלות הכתובות למעלה, ולא פירש מקום קבוען בחשן, עכשיו מפרש לך שיהא תוחב אותן בטבעות, ותדע לך שהן הן הראשונות, שהרי בפרשת אלה פקודי לא הוכפלו׃ }}
{וְתִתֵּין יָת תַּרְתֵּין גְּדִילָן דִּדְהַב עַל תַּרְתֵּין עִזְקָתָא בְּסִטְרֵי חוּשְׁנָא׃}
{And thou shalt put the two wreathen chains of gold on the two rings at the ends of the breastplate.}{\arabic{verse}}
\threeverse{\arabic{verse}}%Ex.28:25
{וְאֵ֨ת שְׁתֵּ֤י קְצוֹת֙ שְׁתֵּ֣י הָעֲבֹתֹ֔ת תִּתֵּ֖ן עַל־שְׁתֵּ֣י הַֽמִּשְׁבְּצ֑וֹת וְנָתַתָּ֛ה עַל־כִּתְפ֥וֹת הָאֵפֹ֖ד אֶל־מ֥וּל פָּנָֽיו׃
\rashi{\rashiDH{ואת שתי קצות. }של שתי העבותות, ב׳ ראשיהם של כל אחת ואחת׃ }\rashi{\rashiDH{תתן על שתי המשבצות. }הן הן הכתובות למעלה בין פרשת החשן ופרשת האפוד, ולא פירש את צרכן ואת מקומן, עכשיו מפרש שיתקע בהן ראשי העבותות התחובות בטבעות החשן לימין ולשמאל אצל הצואר, שני ראשי שרשרות הימנית תוקע במשבצות של ימין, וכן בשל שמאל שני ראשי שרשרות השמאלית׃ }\rashi{\rashiDH{ונתתה. }המשבצות על כתפות האפוד, אחת בזו ואחת בזו, נמצאו כתפות האפוד מחזיקין את החשן שלא יפול ובהן הוא תלוי, ועדיין שפת החשן התחתונה הולכת ובאה, ונוקשת על כריסו ואינה דבוקה לו יפה, לכך הוצרך עוד ב׳ טבעות לתחתיתו, כמו שמפרש והולך׃ }\rashi{\rashiDH{אל מול פניו. }של אפוד, שלא יתן המשבצות בעבר הכתפות שכלפי המעיל, אלא בעבר העליון שכלפי החוץ, והוא קרוי מול פניו של אפוד, כי אותו עבר שאינו נראה אינו קרוי פנים׃ }}
{וְיָת תַּרְתֵּין גְּדִילָן דְּעַל תְּרֵין סִטְרוֹהִי תִּתֵּין עַל תַּרְתֵּין מְרַמְּצָתָא וְתִתֵּין עַל כִּתְפֵי אֵיפוֹדָא לָקֳבֵיל אַפּוֹהִי׃}
{And the other two ends of the two wreathen chains thou shalt put on the two settings, and put them on the shoulder-pieces of the ephod, in the forepart thereof.}{\arabic{verse}}
\threeverse{\arabic{verse}}%Ex.28:26
{וְעָשִׂ֗יתָ שְׁתֵּי֙ טַבְּע֣וֹת זָהָ֔ב וְשַׂמְתָּ֣ אֹתָ֔ם עַל־שְׁנֵ֖י קְצ֣וֹת הַחֹ֑שֶׁן עַל־שְׂפָת֕וֹ אֲשֶׁ֛ר אֶל־עֵ֥בֶר הָאֵפֹ֖ד\note{בספרי ספרד ואשכנז הָאֵפ֖וֹד} בָּֽיְתָה׃
\rashi{\rashiDH{על שני קצות החושן. }הן שתי פאותיו התחתונות לימין ולשמאל׃}\rashi{\rashiDH{על שפתו אשר אל עבר האפוד ביתה. }הרי לך שני סימנין, האחד שיתנם בשני קצות של תחתיתו, שהוא כנגד האפוד, שעליונו אינו כנגד האפוד, שהרי סמוך לצואר הוא, והאפוד נתון על מתניו, ועוד נתן סימן, שלא יקבעם בעבר החושן שכלפי החוץ, אלא בעבר שכלפי פנים, שנאמר ביתה, ואותו העבר הוא לצד האפוד, שחשב האפוד חוגרו לכהן, ונקפל הסינר לפני הכהן על מתניו, וקצת כריסו מכאן ומכאן עד כנגד קצות החשן, וקצותיו שוכבין עליו׃ }}
{וְתַעֲבֵיד תַּרְתֵּין עִזְקָן דִּדְהַב וּתְשַׁוֵּי יָתְהוֹן עַל תְּרֵין סִטְרֵי חוּשְׁנָא עַל סִפְתֵּיהּ דִּלְעִבְרָא דְּאֵיפוֹדָא לְגָיו׃}
{And thou shalt make two rings of gold, and thou shalt put them upon the two ends of the breastplate, upon the edge thereof, which is toward the side of the ephod inward.}{\arabic{verse}}
\threeverse{\arabic{verse}}%Ex.28:27
{וְעָשִׂ֘יתָ֮ שְׁתֵּ֣י טַבְּע֣וֹת זָהָב֒ וְנָתַתָּ֣ה אֹתָ֡ם עַל־שְׁתֵּי֩ כִתְפ֨וֹת הָאֵפ֤וֹד מִלְּמַ֙טָּה֙ מִמּ֣וּל פָּנָ֔יו לְעֻמַּ֖ת מַחְבַּרְתּ֑וֹ מִמַּ֕עַל לְחֵ֖שֶׁב הָאֵפֽוֹד׃
\rashi{\rashiDH{על שתי כתפות האפוד מלמטה. }שהמשבצות נתונות בראשי כתפות האפוד העליונים, הבאים על כתפיו כנגד גרונו ונקפלות ויורדות לפניו, והטבעות צוה ליתן בראשן השני שהוא מחובר לאפוד, והוא שנאמר לעומת מחברתו, סמוך למקום חבורן באפוד למעלה מן החגורה מעט, שהמחברת לעומת החגורה, ואלו נתונים מעט בגובה זקיפת הכתפות, הוא שנאמר ממעל לחשב האפוד, והן כנגד סוף החשן, ונותן פתיל תכלת באותן הטבעות ובטבעות החשן, ורוכסן באותו פתיל לימין ולשמאל, שלא יהא תחתית החשן הולך לפנים וחוזר לאחור ונוקש על כריסו, ונמצא מיושב על המעיל יפה׃ }\rashi{\rashiDH{ממול פניו. }בעבר החיצון׃}}
{וְתַעֲבֵיד תַּרְתֵּין עִזְקָן דִּדְהַב וְתִתֵּין יָתְהוֹן עַל תְּרֵין כִּתְפֵי אֵיפוֹדָא מִלְּרַע מִלָּקֳבֵיל אַפּוֹהִי לָקֳבֵיל בֵּית לוֹפִי מֵעִלָּוֵי לְהִמְיַן אֵיפוֹדָא׃}
{And thou shalt make two rings of gold, and shalt put them on the two shoulder-pieces of the ephod underneath, in the forepart thereof, close by the coupling thereof, above the skilfully woven band of the ephod.}{\arabic{verse}}
\threeverse{\arabic{verse}}%Ex.28:28
{וְיִרְכְּס֣וּ אֶת־הַ֠חֹ֠שֶׁן מִֽטַּבְּעֹתָ֞ו אֶל־טַבְּעֹ֤ת הָאֵפוֹד֙ בִּפְתִ֣יל תְּכֵ֔לֶת לִֽהְי֖וֹת עַל־חֵ֣שֶׁב הָאֵפ֑וֹד וְלֹֽא־יִזַּ֣ח הַחֹ֔שֶׁן מֵעַ֖ל הָאֵפֽוֹד׃
\rashi{\rashiDH{וירכסו. }לשון חבור, וכן מֵרֻכְסֵי אִישׁ (תהלים לא, כא), חבורי חברי רשעים. וכן וְהָרְכָסִים לְבִקְעָה (ישעיה מ, ד), הרים הסמוכים זה לזה, שאי אפשר לירד לגיא שביניהם אלא בקושי גדול, שמתוך סמיכתן הגיא זקופה ועמוקה, יהיו לבקעת מישור ונוחה לילך׃ }\rashi{\rashiDH{להיות על חשב האפוד. }להיות החשן דבוק אל חשב האפוד׃}\rashi{\rashiDH{ולא יזח. }לשון ניתוק, ולשון ערבי הוא כדברי דונש בן לברט׃ }}
{וְיַחֲדוּן יָת חוּשְׁנָא מֵעִזְקָתֵיהּ לְעִזְקָת אֵיפוֹדָא בְּחוּטָא דִּתְכִילְתָא לְמִהְוֵי עַל הִמְיַן אֵיפוֹדָא וְלָא יִתְפָּרַק חוּשְׁנָא מֵעִלָּוֵי אֵיפוֹדָא׃}
{And they shall bind the breastplate by the rings thereof unto the rings of the ephod with a thread of blue, that it may be upon the skilfully woven band of the ephod, and that the breastplate be not loosed from the ephod.}{\arabic{verse}}
\threeverse{\arabic{verse}}%Ex.28:29
{וְנָשָׂ֣א אַ֠הֲרֹ֠ן אֶת־שְׁמ֨וֹת בְּנֵֽי־יִשְׂרָאֵ֜ל בְּחֹ֧שֶׁן הַמִּשְׁפָּ֛ט עַל־לִבּ֖וֹ בְּבֹא֣וֹ אֶל־הַקֹּ֑דֶשׁ לְזִכָּרֹ֥ן לִפְנֵֽי־יְהֹוָ֖ה תָּמִֽיד׃}
{וְיִטּוֹל אַהֲרֹן יָת שְׁמָהָת בְּנֵי יִשְׂרָאֵל בְּחֹשֶׁן דִּינָא עַל לִבֵּיהּ בְּמֵיעֲלֵיהּ לְקוּדְשָׁא לְדוּכְרָנָא קֳדָם יְיָ תְּדִירָא׃}
{And Aaron shall bear the names of the children of Israel in the breastplate of judgment upon his heart, when he goeth in unto the holy place, for a memorial before the \lord\space continually. .}{\arabic{verse}}
\threeverse{\arabic{verse}}%Ex.28:30
{וְנָתַתָּ֞ אֶל־חֹ֣שֶׁן הַמִּשְׁפָּ֗ט אֶת־הָאוּרִים֙ וְאֶת־הַתֻּמִּ֔ים וְהָיוּ֙ עַל־לֵ֣ב אַהֲרֹ֔ן בְּבֹא֖וֹ לִפְנֵ֣י יְהֹוָ֑ה וְנָשָׂ֣א אַ֠הֲרֹ֠ן אֶת־מִשְׁפַּ֨ט בְּנֵי־יִשְׂרָאֵ֧ל עַל־לִבּ֛וֹ לִפְנֵ֥י יְהֹוָ֖ה תָּמִֽיד׃ \setuma         
\rashi{\rashiDH{האורים ואת התמים. }הוא כתב שם המפורש שהיה נותנו בתוך כפלי החשן, שעל ידו הוא מאיר דבריו ומתמם את דבריו (יומא עג׃). ובמקדש שני היה החשן, שאי אפשר לכהן גדול להיות מחוסר בגדים, אבל אותו השם לא היה בתוכו, ועל שם אותו הכתב הוא קרוי משפט, שנאמר וְשָׁאַל לֹו בְּמִשְׁפַּט הָאוּרִים (במדבר כז, כא)׃ }\rashi{\rashiDH{את משפט בני ישראל. }דבר שהם נשפטים ונוכחים על ידו אם לעשות דבר או לא לעשות. ולפי המדרש אגדה, שהחשן מכפר על מעוותי הדין נקרא משפט, על שם סליחת המשפט׃ 
}}
{וְתִתֵּין בְּחֹשֶׁן דִּינָא יָת אוּרַיָּא וְיָת תּוּמַּיָּא וִיהוֹן עַל לִבָּא דְּאַהֲרֹן בְּמֵיעֲלֵיהּ לִקְדָם יְיָ וְיִטּוֹל אַהֲרֹן יָת דִּין בְּנֵי יִשְׂרָאֵל עַל לִבֵּיהּ קֳדָם יְיָ תְּדִירָא׃}
{And thou shalt put in the breastplate of judgment the Urim and the Thummim; and they shall be upon Aaron’s heart, when he goeth in before the \lord; and Aaron shall bear the judgment of the children of Israel upon his heart before the \lord\space continually.}{\arabic{verse}}
\threeverse{\aliya{שלישי}}%Ex.28:31
{וְעָשִׂ֛יתָ אֶת־מְעִ֥יל הָאֵפ֖וֹד כְּלִ֥יל תְּכֵֽלֶת׃
\rashi{\rashiDH{את מעיל האפוד. }שאפוד ניתן עליו לחגורה׃}\rashi{\rashiDH{כליל תכלת. }כולו תכלת, שאין מין אחר מעורב בו׃ }}
{וְתַעֲבֵיד יָת מְעִיל אֵיפוֹדָא גְּמִיר תַּכְלָא׃}
{And thou shalt make the robe of the ephod all of blue.}{\arabic{verse}}
\threeverse{\arabic{verse}}%Ex.28:32
{וְהָיָ֥ה פִֽי־רֹאשׁ֖וֹ בְּתוֹכ֑וֹ שָׂפָ֡ה יִֽהְיֶה֩ לְפִ֨יו סָבִ֜יב מַעֲשֵׂ֣ה אֹרֵ֗ג כְּפִ֥י תַחְרָ֛א יִֽהְיֶה־לּ֖וֹ לֹ֥א יִקָּרֵֽעַ׃
\rashi{\rashiDH{והיה פי ראשו. }פי המעיל בגבהו, הוא פתיחת בית הצואר׃ }\rashi{\rashiDH{בתוכו. }כתרגומו כָּפִיל לְגַוֵּיהּ, כפול לתוכו, להיות לו לשפה כפילתו, והיה מעשה אורג ולא במחט׃ }\rashi{\rashiDH{כפי תחרא. }למדנו שהשריונים שלהם פיהם כפול לתוכן׃}\rashi{\rashiDH{לא יקרע. }כדי שלא יקרע, והקורעו עובר בלאו, שזה ממנין לאוין שבתורה, וכן לֹא יִזַּח הַחשֶׁן, וכן לֹא יָסֻרוּ מִמֶּנוּ (שמות כה, טו), הנאמר בבדי הארון׃ }}
{וִיהֵי פוּמֵּיהּ כְּפִיל לְגַוֵּיהּ תּוּרָא יְהֵי מַקַּף לְפוּמֵּיהּ סְחוֹר סְחוֹר עוֹבָד מָחֵי כְּפוֹם שִׁרְיָן יְהֵי לֵיהּ דְּלָא יִתְבְּזַע׃}
{And it shall have a hole for the head in the midst thereof; it shall have a binding of woven work round about the hole of it, as it were the hole of a coat of mail that it be not rent.}{\arabic{verse}}
\threeverse{\arabic{verse}}%Ex.28:33
{וְעָשִׂ֣יתָ עַל־שׁוּלָ֗יו רִמֹּנֵי֙ תְּכֵ֤לֶת וְאַרְגָּמָן֙ וְתוֹלַ֣עַת שָׁנִ֔י עַל־שׁוּלָ֖יו סָבִ֑יב וּפַעֲמֹנֵ֥י זָהָ֛ב בְּתוֹכָ֖ם סָבִֽיב׃
\rashi{\rashiDH{רמוני. }עגולים וחלולים היו, כמין רמונים העשויים כבצת תרנגולת׃ }\rashi{\rashiDH{ופעמוני זהב. }זגין עם עִנְבָּלִין שבתוכם׃}\rashi{\rashiDH{בתוכם סביב. }ביניהם סביב, בין שני רמונים פעמון אחד, דבוק ותלוי בשולי המעיל׃ }}
{וְתַעֲבֵיד עַל שִׁפּוֹלוֹהִי רִמּוֹנֵי תַּכְלָא וְאַרְגְּוָנָא וּצְבַע זְהוֹרִי עַל שִׁפּוֹלוֹהִי סְחוֹר סְחוֹר וְזַגִּין דִּדְהַב בֵּינֵיהוֹן סְחוֹר סְחוֹר׃}
{And upon the skirts of it thou shalt make pomegranates of blue, and of purple, and of scarlet, round about the skirts thereof; and bells of gold between them round about:}{\arabic{verse}}
\threeverse{\arabic{verse}}%Ex.28:34
{פַּעֲמֹ֤ן זָהָב֙ וְרִמּ֔וֹן פַּֽעֲמֹ֥ן זָהָ֖ב וְרִמּ֑וֹן עַל־שׁוּלֵ֥י הַמְּעִ֖יל סָבִֽיב׃
\rashi{\rashiDH{פעמון זהב ורמון וגו׳. }פעמון זהב ורמון אצלו׃}}
{זַגָּא דְּדַהְבָּא וְרִמּוֹנָא זַגָּא דְּדַהְבָּא וְרִמּוֹנָא עַל שִׁפּוֹלֵי מְעִילָא סְחוֹר סְחוֹר׃}
{a golden bell and a pomegranate, a golden bell and a pomegranate, upon the skirts of the robe round about.}{\arabic{verse}}
\threeverse{\arabic{verse}}%Ex.28:35
{וְהָיָ֥ה עַֽל־אַהֲרֹ֖ן לְשָׁרֵ֑ת וְנִשְׁמַ֣ע ק֠וֹל֠וֹ בְּבֹא֨וֹ אֶל־הַקֹּ֜דֶשׁ לִפְנֵ֧י יְהֹוָ֛ה וּבְצֵאת֖וֹ וְלֹ֥א יָמֽוּת׃ \setuma         
\rashi{\rashiDH{ולא ימות. }מכלל לאו אתה שומע הן, אם יהיו לו לא יתחייב מיתה, הא אם יכנס מחוסר אחד מן הבגדים הללו, חייב מיתה בידי שמים׃ 
}}
{וִיהֵי עַל אַהֲרֹן לְשַׁמָּשָׁא וְיִשְׁתְּמַע קָלֵיהּ בְּמֵיעֲלֵיהּ לְקוּדְשָׁא לִקְדָם יְיָ וּבְמִפְּקֵיהּ וְלָא יְמוּת׃}
{And it shall be upon Aaron to minister; and the sound thereof shall be heard when he goeth in unto the holy place before the \lord, and when he cometh out, that he die not.}{\arabic{verse}}
\threeverse{\arabic{verse}}%Ex.28:36
{וְעָשִׂ֥יתָ צִּ֖יץ זָהָ֣ב טָה֑וֹר וּפִתַּחְתָּ֤ עָלָיו֙ פִּתּוּחֵ֣י חֹתָ֔ם קֹ֖דֶשׁ לַֽיהֹוָֽה׃
\rashi{\rashiDH{ציץ. }כמין טס של זהב היה, רוחב ב׳ אצבעות, מקיף על המצח מאוזן לאוזן (סוכה ה׃)׃ }}
{וְתַעֲבֵיד צִיצָא דִּדְהַב דְּכֵי וְתִגְלוֹף עֲלוֹהִי כְּתָב מְפָרַשׁ קֹדֶשׁ לַייָ׃}
{And thou shalt make a plate of pure gold, and engrave upon it, like the engravings of a signet: HOLY TO THE \lord.}{\arabic{verse}}
\threeverse{\arabic{verse}}%Ex.28:37
{וְשַׂמְתָּ֤ אֹתוֹ֙ עַל־פְּתִ֣יל תְּכֵ֔לֶת וְהָיָ֖ה עַל־הַמִּצְנָ֑פֶת אֶל־מ֥וּל פְּנֵֽי־הַמִּצְנֶ֖פֶת יִהְיֶֽה׃
\rashi{\rashiDH{על פתיל תכלת. }ובמקום אחר הוא אומר, וַיִּתְּנוּ עָלָיו פְּתִיל תְּכֵלֶת (שמות לט, לא), ועוד, כתיב כאן והיה על המצנפת, ולמטה הוא אומר והיה על מצח אהרן, ובשחיטת קדשים שנינו (זבחים יט.׃), שערו היה נראה בין ציץ למצנפת ששם מניח תפלין, למדנו שהמצנפת למעלה בגובה הראש, ואינה עמוקה להכנס בה כל הראש עד המצח, והציץ מלמטה, והפתילים היו בנקבים, ותלויין בו בשני ראשים ובאמצעו, ששה בג׳ מקומות הללו, פתיל מלמעלה אחד מבחוץ ואחד מבפנים כנגדו, וקושר ראשי הפתילים מאחורי העורף שלשתן, ונמצאו בין אורך הטס ופתילי ראשיו מקיפין את הקדקד, ופתיל האמצעי שבראשו קשור עם ראשי השנים, והולך על פני רוחב הראש מלמעלה, נמצא עשוי כמין כובע, ועל פתיל האמצעי הוא אומר והיה על המצנפת, והיה נותן הציץ על ראשו כמין כובע על המצנפת, והפתיל האמצעי מחזיקו שאינו נופל, והטס תלוי כנגד מצחו, ונתקיימו כל המקראות, פתיל על הציץ, וציץ על הפתיל, ופתיל על המצנפת מלמעלה׃ }}
{וּתְשַׁוֵּי יָתֵיהּ עַל חוּטָא דִּתְכִילְתָא וִיהֵי עַל מַצְנַפְתָּא לָקֳבֵיל אַפֵּי מַצְנַפְתָּא יְהֵי׃}
{And thou shalt put it on a thread of blue, and it shall be upon the mitre; upon the forefront of the mitre it shall be.}{\arabic{verse}}
\threeverse{\arabic{verse}}%Ex.28:38
{וְהָיָה֮ עַל־מֵ֣צַח אַהֲרֹן֒ וְנָשָׂ֨א אַהֲרֹ֜ן אֶת־עֲוֺ֣ן הַקֳּדָשִׁ֗ים אֲשֶׁ֤ר יַקְדִּ֙ישׁוּ֙ בְּנֵ֣י יִשְׂרָאֵ֔ל לְכׇֽל־מַתְּנֹ֖ת קׇדְשֵׁיהֶ֑ם וְהָיָ֤ה עַל־מִצְחוֹ֙ תָּמִ֔יד לְרָצ֥וֹן לָהֶ֖ם לִפְנֵ֥י יְהֹוָֽה׃
\rashi{\rashiDH{ונשא אהרן. }לשון סליחה, ואף על פי כן אינו זז ממשמעו, אהרן נושא את המשא של עון, נמצא מסולק העון מן הקדשים׃ }\rashi{\rashiDH{את עון הקדשים. }לרצות על הדם ועל החלב שקרבו בטומאה, כמו ששנינו (מנחות כה.), אי זה עון הוא נושא, אם עון פגול הרי כבר נאמר לא ירצה, ואם עון נותר הרי נאמר לא יחשב, ואין לומר שיכפר על עון הכהן שהקריב טמא, שהרי עון הקדשים נאמר, ולא עון המקריבים, הא אינו מרצה אלא להכשיר הקרבן׃ }\rashi{\rashiDH{והיה על מצחו תמיד. }אי אפשר לומר שיהא על מצחו תמיד, שהרי אינו עליו אלא בשעת העבודה, אלא תמיד לרצות להם, אפילו אינו על מצחו, שלא היה כהן גדול עובד באותה שעה, ולדברי האומר עודהו על מצחו מכפר ומרצה, ואם לאו אינו מרצה, נדרש על מצחו תמיד, מלמד שממשמש בו בעודו על מצחו, שלא יסיח דעתו ממנו׃ }}
{וִיהֵי עַל בֵּית עֵינוֹהִי דְּאַהֲרֹן וְיִטּוֹל אַהֲרֹן יָת עֲוָיָת קוּדְשַׁיָּא דְּיַקְדְּשׁוּן בְּנֵי יִשְׂרָאֵל לְכל מַתְּנָת קוּדְשֵׁיהוֹן וִיהֵי עַל בֵּית עֵינוֹהִי תְּדִירָא לְרַעֲוָא לְהוֹן קֳדָם יְיָ׃}
{And it shall be upon Aaron’s forehead, and Aaron shall bear the iniquity committed in the holy things, which the children of Israel shall hallow, even in all their holy gifts; and it shall be always upon his forehead, that they may be accepted before the \lord.}{\arabic{verse}}
\threeverse{\arabic{verse}}%Ex.28:39
{וְשִׁבַּצְתָּ֙ הַכְּתֹ֣נֶת שֵׁ֔שׁ וְעָשִׂ֖יתָ מִצְנֶ֣פֶת שֵׁ֑שׁ וְאַבְנֵ֥ט תַּעֲשֶׂ֖ה מַעֲשֵׂ֥ה רֹקֵֽם׃
\rashi{\rashiDH{ושבצת. }עשה אותם משבצות משבצות, וכולם של שש׃ }}
{וּתְרַמֵּיץ כִּתּוּנָא דְּבוּצָא וְתַעֲבֵיד מַצְנַפְתָּא דְּבוּצָא וְהִמְיָן תַּעֲבֵיד עוֹבָד צַיָּיר׃}
{And thou shalt weave the tunic in chequer work of fine linen, and thou shalt make a mitre of fine linen, and thou shalt make a girdle, the work of the weaver in colours.}{\arabic{verse}}
\threeverse{\arabic{verse}}%Ex.28:40
{וְלִבְנֵ֤י אַהֲרֹן֙ תַּעֲשֶׂ֣ה כֻתֳּנֹ֔ת וְעָשִׂ֥יתָ לָהֶ֖ם אַבְנֵטִ֑ים וּמִגְבָּעוֹת֙ תַּעֲשֶׂ֣ה לָהֶ֔ם לְכָב֖וֹד וּלְתִפְאָֽרֶת׃
\rashi{\rashiDH{ולבני אהרן תעשה כתנת. }ארבעה בגדים הללו ולא יותר, כתונת, ואבנט, ומגבעות היא מצנפת, ומכנסים הכתובים למטה בפרשה׃ }}
{וְלִבְנֵי אַהֲרֹן תַּעֲבֵיד כִּתּוּנִין וְתַעֲבֵיד לְהוֹן הִמְיָנִין וְקוֹבְעִין תַּעֲבֵיד לְהוֹן לִיקָר וּלְתוּשְׁבְּחָא׃}
{And for Aaron’s sons thou shalt make tunics, and thou shalt make for them girdles, and head-tires shalt thou make for them, for splendour and for beauty.}{\arabic{verse}}
\threeverse{\arabic{verse}}%Ex.28:41
{וְהִלְבַּשְׁתָּ֤ אֹתָם֙ אֶת־אַהֲרֹ֣ן אָחִ֔יךָ וְאֶת־בָּנָ֖יו אִתּ֑וֹ וּמָשַׁחְתָּ֨ אֹתָ֜ם וּמִלֵּאתָ֧ אֶת־יָדָ֛ם וְקִדַּשְׁתָּ֥ אֹתָ֖ם וְכִהֲנ֥וּ לִֽי׃
\rashi{\rashiDH{והלבשת אותם את אהרן. }אותם האמורין באהרן, חשן, ואפוד, ומעיל, וכתונת תשבץ, מצנפת, ואבנט, וציץ, ומכנסים, הכתובים למטה בכולם׃ }\rashi{\rashiDH{ואת בניו אתו. }אותם הכתובים בהם׃}\rashi{\rashiDH{ומשחת אותם. }את אהרן ואת בניו בשמן המשחה׃}\rashi{\rashiDH{ומלאת את ידם. }כל מלוי ידים לשון חנוך, כשהוא נכנס לדבר להיות מוחזק בו מאותו יום והלאה, ובלשון לע״ז, כשממנין אדם על פקודת דבר, נותן השליט בידו בית יד של עור שקורין גוואנ״טו, ועל ידו הוא מחזיקו בדבר, וקורין לאותו מסירה וויר״סטיר, והוא מלוי ידים׃ }}
{וְתַלְבֵּישׁ יָתְהוֹן יָת אַהֲרֹן אֲחוּךְ וְיָת בְּנוֹהִי עִמֵּיהּ וּתְרַבֵּי יָתְהוֹן וּתְקָרֵיב יָת קוּרְבָּנְהוֹן וּתְקַדֵּישׁ יָתְהוֹן וִישַׁמְּשׁוּן קֳדָמָי׃}
{And thou shalt put them upon Aaron thy brother, and upon his sons with him; and shalt anoint them, and consecrate them, and sanctify them, that they may minister unto Me in the priest’s office.}{\arabic{verse}}
\threeverse{\arabic{verse}}%Ex.28:42
{וַעֲשֵׂ֤ה לָהֶם֙ מִכְנְסֵי־בָ֔ד לְכַסּ֖וֹת בְּשַׂ֣ר עֶרְוָ֑ה מִמׇּתְנַ֥יִם וְעַד־יְרֵכַ֖יִם יִהְיֽוּ׃
\rashi{\rashiDH{ועשה להם. }לאהרן ולבניו׃}\rashi{\rashiDH{מכנסי בד. }הרי ח׳ בגדים לכהן גדול וארבעה לכהן הדיוט׃}}
{וַעֲבֵיד לְהוֹן מִכְנְסִין דְּבוּץ לְכַסָּאָה בְּשַׂר עֶרְיָא מֵחַרְצִין וְעַד יִרְכָן יְהוֹן׃}
{And thou shalt make them linen breeches to cover the flesh of their nakedness; from the loins even unto the thighs they shall reach.}{\arabic{verse}}
\threeverse{\arabic{verse}}%Ex.28:43
{וְהָיוּ֩ עַל־אַהֲרֹ֨ן וְעַל־בָּנָ֜יו בְּבֹאָ֣ם \legarmeh  אֶל־אֹ֣הֶל מוֹעֵ֗ד א֣וֹ בְגִשְׁתָּ֤ם אֶל־הַמִּזְבֵּ֙חַ֙ לְשָׁרֵ֣ת בַּקֹּ֔דֶשׁ וְלֹא־יִשְׂא֥וּ עָוֺ֖ן וָמֵ֑תוּ חֻקַּ֥ת עוֹלָ֛ם ל֖וֹ וּלְזַרְע֥וֹ אַחֲרָֽיו׃ \setuma         
\rashi{\rashiDH{והיו על אהרן. }כל הבגדים האלה, על אהרן הראויין לו׃ }\rashi{\rashiDH{ועל בניו. }האמורין בהם׃}\rashi{\rashiDH{בבואם אל אהל מועד. }להיכל וכן למשכן׃ 
}\rashi{\rashiDH{ומתו. }הא למדת, שהמשמש מחוסר בגדים, במיתה׃ }\rashi{\rashiDH{חקת עולם לו. }כל מקום שנאמר חקת עולם, הוא גזירה מיד ולדורות לעכב בו׃ }}
{וִיהוֹן עַל אַהֲרֹן וְעַל בְּנוֹהִי בְּמֵיעַלְהוֹן לְמַשְׁכַּן זִמְנָא אוֹ בְמִקְרַבְהוֹן לְמַדְבְּחָא לְשַׁמָּשָׁא בְּקוּדְשָׁא וְלָא יְקַבְּלוּן חוֹבָא וְלָא יְמוּתוּן קְיָם עָלַם לֵיהּ וְלִבְנוֹהִי בָּתְרוֹהִי׃}
{And they shall be upon Aaron, and upon his sons, when they go in unto the tent of meeting, or when they come near unto the altar to minister in the holy place; that they bear not iniquity, and die; it shall be a statute for ever unto him and unto his seed after him.}{\arabic{verse}}
\newperek
\threeverse{\aliya{רביעי}}%Ex.29:1
{וְזֶ֨ה הַדָּבָ֜ר אֲשֶֽׁר־תַּעֲשֶׂ֥ה לָהֶ֛ם לְקַדֵּ֥שׁ אֹתָ֖ם לְכַהֵ֣ן לִ֑י לְ֠קַ֠ח פַּ֣ר אֶחָ֧ד בֶּן־בָּקָ֛ר וְאֵילִ֥ם שְׁנַ֖יִם תְּמִימִֽם׃
\rashi{\rashiDH{לקח. }כמו קח, ושתי גזרות הן, אחת של קיחה ואחת של לקיחה, ולהן פתרון אחד׃ }\rashi{\rashiDH{פר אחד. }לכפר על מעשה העגל שהוא פר׃ 
}}
{וְדֵין פִּתְגָמָא דְּתַעֲבֵיד לְהוֹן לְקַדָּשָׁא יָתְהוֹן לְשַׁמָּשָׁא קֳדָמָי סַב תּוֹר חַד בַּר תּוֹרֵי וְדִכְרִין תְּרֵין שַׁלְמִין׃}
{And this is the thing that thou shalt do unto them to hallow them, to minister unto Me in the priest’s office: take one young bullock and two rams without blemish,}{\Roman{chap}}
\threeverse{\arabic{verse}}%Ex.29:2
{וְלֶ֣חֶם מַצּ֗וֹת וְחַלֹּ֤ת מַצֹּת֙ בְּלוּלֹ֣ת בַּשֶּׁ֔מֶן וּרְקִיקֵ֥י מַצּ֖וֹת מְשֻׁחִ֣ים בַּשָּׁ֑מֶן סֹ֥לֶת חִטִּ֖ים תַּעֲשֶׂ֥ה אֹתָֽם׃
\rashi{\rashiDH{ולחם מצות וחלת מצת ורקיקי מצות. }הרי אלו ג׳ מינין, רבוכה, וחלות, ורקיקין. לחם מצות היא הקרויה למטה בענין חלת לחם שמן, על שם שנותן שמן ברבוכה כנגד החלות והרקיקין, וכל המינין באים עשר עשר חלות (מנחות עו.)׃ }\rashi{\rashiDH{בשמן. }כשהן קמח, יוצק בהן שמן ובוללן (שם עה.)׃ }\rashi{\rashiDH{משחים בשמן. }אחר אפייתן מושחן כמין כ״ף יונית שהיא עשויה כנו״ן שלנו (שם)׃ }}
{וּלְחֵים פַּטִּיר וּגְרִיצָן פַּטִּירָן דְּפִילָן בִּמְשַׁח וְאֶסְפּוֹגִין פַּטִּירִין דִּמְשִׁיחִין בִּמְשַׁח סֹלֶת דְּחִטִּין תַּעֲבֵיד יָתְהוֹן׃}
{and unleavened bread, and cakes unleavened mingled with oil, and wafers unleavened spread with oil; of fine wheaten flour shalt thou make them.}{\arabic{verse}}
\threeverse{\arabic{verse}}%Ex.29:3
{וְנָתַתָּ֤ אוֹתָם֙ עַל־סַ֣ל אֶחָ֔ד וְהִקְרַבְתָּ֥ אֹתָ֖ם בַּסָּ֑ל וְאֶ֨ת־הַפָּ֔ר וְאֵ֖ת שְׁנֵ֥י הָאֵילִֽם׃
\rashi{\rashiDH{והקרבת אותם. }אל חצר המשכן ביום הקמתו׃}}
{וְתִתֵּין יָתְהוֹן עַל סַלָּא חַד וּתְקָרֵיב יָתְהוֹן בְּסַלָּא וְיָת תּוֹרָא וְיָת תְּרֵין דִּכְרִין׃}
{And thou shalt put them into one basket, and bring them in the basket, with the bullock and the two rams.}{\arabic{verse}}
\threeverse{\arabic{verse}}%Ex.29:4
{וְאֶת־אַהֲרֹ֤ן וְאֶת־בָּנָיו֙ תַּקְרִ֔יב אֶל־פֶּ֖תַח אֹ֣הֶל מוֹעֵ֑ד וְרָחַצְתָּ֥ אֹתָ֖ם בַּמָּֽיִם׃
\rashi{\rashiDH{ורחצת. }זו טבילת כל הגוף׃ 
}}
{וְיָת אַהֲרֹן וְיָת בְּנוֹהִי תְּקָרֵיב לִתְרַע מַשְׁכַּן זִמְנָא וְתַסְחֵי יָתְהוֹן בְּמַיָּא׃}
{And Aaron and his sons thou shalt bring unto the door of the tent of meeting, and shalt wash them with water.}{\arabic{verse}}
\threeverse{\arabic{verse}}%Ex.29:5
{וְלָקַחְתָּ֣ אֶת־הַבְּגָדִ֗ים וְהִלְבַּשְׁתָּ֤ אֶֽת־אַהֲרֹן֙ אֶת־הַכֻּתֹּ֔נֶת וְאֵת֙ מְעִ֣יל הָאֵפֹ֔ד וְאֶת־הָאֵפֹ֖ד וְאֶת־הַחֹ֑שֶׁן וְאָפַדְתָּ֣ ל֔וֹ בְּחֵ֖שֶׁב הָאֵפֹֽד׃
\rashi{\rashiDH{ואפדת. }קשט ותקן החגורה והסינר סביבותיו׃}}
{וְתִסַּב יָת לְבוּשַׁיָּא וְתַלְבֵּישׁ יָת אַהֲרֹן יָת כִּתּוּנָא וְיָת מְעִיל אֵיפוֹדָא וְיָת אֵיפוֹדָא וְיָת חוּשְׁנָא וְתַתְקֵין לֵיהּ בְּהִמְיַן אֵיפוֹדָא׃}
{And thou shalt take the garments, and put upon Aaron the tunic, and the robe of the ephod, and the ephod, and the breastplate, and gird him with the skilfully woven band of the ephod.}{\arabic{verse}}
\threeverse{\arabic{verse}}%Ex.29:6
{וְשַׂמְתָּ֥ הַמִּצְנֶ֖פֶת עַל־רֹאשׁ֑וֹ וְנָתַתָּ֛ אֶת־נֵ֥זֶר הַקֹּ֖דֶשׁ עַל־הַמִּצְנָֽפֶת׃
\rashi{\rashiDH{נזר הקדש. }זה הציץ׃}\rashi{\rashiDH{על המצנפת. }כמו שפירשתי למעלה, על ידי הפתיל האמצעי ושני פתילין שבראשו הקשורין שלשתן מאחורי העורף, הוא נותנו על המצנפת כמין כובע׃ }}
{וּתְשַׁוֵּי מַצְנַפְתָּא עַל רֵישֵׁיהּ וְתִתֵּין יָת כְּלִילָא דְּקוּדְשָׁא עַל מַצְנַפְתָּא׃}
{And thou shalt set the mitre upon his head, and put the holy crown upon the mitre.}{\arabic{verse}}
\threeverse{\arabic{verse}}%Ex.29:7
{וְלָֽקַחְתָּ֙ אֶת־שֶׁ֣מֶן הַמִּשְׁחָ֔ה וְיָצַקְתָּ֖ עַל־רֹאשׁ֑וֹ וּמָשַׁחְתָּ֖ אֹתֽוֹ׃
\rashi{\rashiDH{ומשחת אותו. }אף משיחה זו כמין כ״ף יונית, נותן שמן על ראשו, ובין ריסי עיניו, ומחברן באצבעו׃ }}
{וְתִסַּב יָת מִשְׁחָא דִּרְבוּתָא וּתְרִיק עַל רֵישֵׁיהּ וּתְרַבִּי יָתֵיהּ׃}
{Then shalt thou take the anointing oil, and pour it upon his head, and anoint him.}{\arabic{verse}}
\threeverse{\arabic{verse}}%Ex.29:8
{וְאֶת־בָּנָ֖יו תַּקְרִ֑יב וְהִלְבַּשְׁתָּ֖ם כֻּתֳּנֹֽת׃}
{וְיָת בְּנוֹהִי תְּקָרֵיב וְתַלְבֵּישִׁנּוּן כִּתּוּנִין׃}
{And thou shalt bring his sons, and put tunics upon them.}{\arabic{verse}}
\threeverse{\arabic{verse}}%Ex.29:9
{וְחָגַרְתָּ֩ אֹתָ֨ם אַבְנֵ֜ט אַהֲרֹ֣ן וּבָנָ֗יו וְחָבַשְׁתָּ֤ לָהֶם֙ מִגְבָּעֹ֔ת וְהָיְתָ֥ה לָהֶ֛ם כְּהֻנָּ֖ה לְחֻקַּ֣ת עוֹלָ֑ם וּמִלֵּאתָ֥ יַֽד־אַהֲרֹ֖ן וְיַד־בָּנָֽיו׃
\rashi{\rashiDH{והיתה להם. }מלוי ידים זה לכהונת עולם׃}\rashi{\rashiDH{ומלאת. }על ידי הדברים האלה׃}\rashi{\rashiDH{יד אהרן ויד בניו. }במילוי ופקודת הכהונה׃ 
}}
{וּתְזָרֵיז יָתְהוֹן הִמְיָנִין אַהֲרֹן וּבְנוֹהִי וְתַתְקֵין לְהוֹן קוֹבְעִין וּתְהֵי לְהוֹן כְּהוּנְּתָא לִקְיָם עָלַם וּתְקָרֵיב קוּרְבָּנָא דְּאַהֲרֹן וְקוּרְבָּנָא דִּבְנוֹהִי׃}
{And thou shalt gird them with girdles, Aaron and his sons, and bind head-tires on them; and they shall have the priesthood by a perpetual statute; and thou shalt consecrate Aaron and his sons.}{\arabic{verse}}
\threeverse{\arabic{verse}}%Ex.29:10
{וְהִקְרַבְתָּ֙ אֶת־הַפָּ֔ר לִפְנֵ֖י אֹ֣הֶל מוֹעֵ֑ד וְסָמַ֨ךְ אַהֲרֹ֧ן וּבָנָ֛יו אֶת־יְדֵיהֶ֖ם עַל־רֹ֥אשׁ הַפָּֽר׃}
{וּתְקָרֵיב יָת תּוֹרָא לִקְדָם מַשְׁכַּן זִמְנָא וְיִסְמוֹךְ אַהֲרֹן וּבְנוֹהִי יָת יְדֵיהוֹן עַל רֵישׁ תּוֹרָא׃}
{And thou shalt bring the bullock before the tent of meeting; and Aaron and his sons shall lay their hands upon the head of the bullock.}{\arabic{verse}}
\threeverse{\arabic{verse}}%Ex.29:11
{וְשָׁחַטְתָּ֥ אֶת־הַפָּ֖ר לִפְנֵ֣י יְהֹוָ֑ה פֶּ֖תַח אֹ֥הֶל מוֹעֵֽד׃
\rashi{\rashiDH{פתח אהל מועד. }בחצר המשכן שלפני הפתח׃}}
{וְתִכּוֹס יָת תּוֹרָא קֳדָם יְיָ בִּתְרַע מַשְׁכַּן זִמְנָא׃}
{And thou shalt kill the bullock before the \lord, at the door of the tent of meeting.}{\arabic{verse}}
\threeverse{\arabic{verse}}%Ex.29:12
{וְלָֽקַחְתָּ֙ מִדַּ֣ם הַפָּ֔ר וְנָתַתָּ֛ה עַל־קַרְנֹ֥ת הַמִּזְבֵּ֖חַ בְּאֶצְבָּעֶ֑ךָ וְאֶת־כׇּל־הַדָּ֣ם תִּשְׁפֹּ֔ךְ אֶל־יְס֖וֹד הַמִּזְבֵּֽחַ׃
\rashi{\rashiDH{על קרנות. }למעלה, בקרנות ממש (זבחים נג.)׃ }\rashi{\rashiDH{ואת כל הדם. }שירי הדם׃}\rashi{\rashiDH{אל יסוד המזבח. }כמין בליטת בית קבול עשוי לו סביב סביב לאחר שעלה אמה מן הארץ (מדות פ״ג מ״א)׃ }}
{וְתִסַּב מִדְּמָא דְּתוֹרָא וְתִתֵּין עַל קַרְנָת מַדְבְּחָא בְּאֶצְבְּעָךְ וְיָת כָּל דְּמָא תִּשְׁפּוֹךְ לִיסוֹדָא דְּמַדְבְּחָא׃}
{And thou shalt take of the blood of the bullock, and put it upon the horns of the altar with thy finger; and thou shalt pour out all the remaining blood at the base of the altar.}{\arabic{verse}}
\threeverse{\arabic{verse}}%Ex.29:13
{וְלָֽקַחְתָּ֗ אֶֽת־כׇּל־הַחֵ֘לֶב֮ הַֽמְכַסֶּ֣ה אֶת־הַקֶּ֒רֶב֒ וְאֵ֗ת הַיֹּתֶ֙רֶת֙ עַל־הַכָּבֵ֔ד וְאֵת֙ שְׁתֵּ֣י הַכְּלָיֹ֔ת וְאֶת־הַחֵ֖לֶב אֲשֶׁ֣ר עֲלֵיהֶ֑ן וְהִקְטַרְתָּ֖ הַמִּזְבֵּֽחָה׃
\rashi{\rashiDH{החלב המכסה את הקרב. }הוא הקרום שעל הכרס שקורין טיל״א׃ }\rashi{\rashiDH{ואת היתרת. }הוא טַרְפְּשָׁא דְכַבְדָא שקורין איבר״ש (ראטהפלייש)׃ }\rashi{\rashiDH{על הכבד. }אף מן הכבד יטול עמה (ת״כ ג, ד)׃ }}
{וְתִסַּב יָת כָּל תַּרְבָּא דְּחָפֵי יָת גַּוָּא וְיָת חַצְרָא דְּעַל כַּבְדָּא וְיָת תַּרְתֵּין כּוֹלְיָן וְיָת תַּרְבָּא דַּעֲלֵיהוֹן וְתַסֵּיק לְמַדְבְּחָא׃}
{And thou shalt take all the fat that covereth the inwards, and the lobe above the liver, and the two kidneys, and the fat that is upon them, and make them smoke upon the altar.}{\arabic{verse}}
\threeverse{\arabic{verse}}%Ex.29:14
{וְאֶת־בְּשַׂ֤ר הַפָּר֙ וְאֶת־עֹר֣וֹ וְאֶת־פִּרְשׁ֔וֹ תִּשְׂרֹ֣ף בָּאֵ֔שׁ מִח֖וּץ לַֽמַּחֲנֶ֑ה חַטָּ֖את הֽוּא׃
\rashi{\rashiDH{תשרף באש. }לא מצינו חטאת חיצונה נשרפת אלא זו׃}}
{וְיָת בְּשַׂר תּוֹרָא וְיָת מַשְׁכֵּיהּ וְיָת אוּכְלֵיהּ תּוֹקֵיד בְּנוּרָא מִבַּרָא לְמַשְׁרִיתָא חַטָּאתָא הוּא׃}
{But the flesh of the bullock, and its skin, and its dung, shalt thou burn with fire without the camp; it is a sin-offering.}{\arabic{verse}}
\threeverse{\arabic{verse}}%Ex.29:15
{וְאֶת־הָאַ֥יִל הָאֶחָ֖ד תִּקָּ֑ח וְסָ֨מְכ֜וּ אַהֲרֹ֧ן וּבָנָ֛יו אֶת־יְדֵיהֶ֖ם עַל־רֹ֥אשׁ הָאָֽיִל׃}
{וְיָת דִּכְרָא חַד תִּסַּב וְיִסְמְכוּן אַהֲרֹן וּבְנוֹהִי יָת יְדֵיהוֹן עַל רֵישׁ דִּכְרָא׃}
{Thou shalt also take the one ram; and Aaron and his sons shall lay their hands upon the head of the ram.}{\arabic{verse}}
\threeverse{\arabic{verse}}%Ex.29:16
{וְשָׁחַטְתָּ֖ אֶת־הָאָ֑יִל וְלָֽקַחְתָּ֙ אֶת־דָּמ֔וֹ וְזָרַקְתָּ֥ עַל־הַמִּזְבֵּ֖חַ סָבִֽיב׃
\rashi{\rashiDH{וזרקת. }בכלי, אוחז במזרק וזורק כנגד הקרן, כדי שיראה לכאן ולכאן, ואין קרבן טעון מתנה באצבע אלא חטאת בלבד, אבל שאר זבחים אינן טעונין קרן ולא אצבע, שמתן דמם מחצי המזבח ולמטה, ואינו עולה בכבש, אלא עומד בארץ וזורק׃ }\rashi{\rashiDH{סביב. }כך מפורש בשחיטת קדשים (זבחים נג׃), שאין סביב אלא ב׳ מתנות שהן ארבע, האחת בקרן זוית זו, והאחת בקרן שכנגדה באלכסון, וכל מתנה נראית בשני צדי הקרן אילך ואילך, נמצא הדם נתון בד׳ רוחות סביב, לכך קרוי סביב׃ }}
{וְתִכּוֹס יָת דִּכְרָא וְתִסַּב יָת דְּמֵיהּ וְתִזְרוֹק עַל מַדְבְּחָא סְחוֹר סְחוֹר׃}
{And thou shalt slay the ram, and thou shalt take its blood, and dash it round about against the altar.}{\arabic{verse}}
\threeverse{\arabic{verse}}%Ex.29:17
{וְאֶ֨ת־הָאַ֔יִל תְּנַתֵּ֖חַ לִנְתָחָ֑יו וְרָחַצְתָּ֤ קִרְבּוֹ֙ וּכְרָעָ֔יו וְנָתַתָּ֥ עַל־נְתָחָ֖יו וְעַל־רֹאשֽׁוֹ׃
\rashi{\rashiDH{על נתחיו. }עם נתחיו, וסף על שאר הנתחים׃ }}
{וְיָת דִּכְרָא תְּפַלֵּיג לְאֶבְרוֹהִי וּתְחַלֵּיל גַּוֵּיהּ וּכְרָעוֹהִי וְתִתֵּין עַל אֶבְרוֹהִי וְעַל רֵישֵׁיהּ׃}
{And thou shalt cut the ram into its pieces, and wash its inwards, and its legs, and put them with its pieces, and with its head.}{\arabic{verse}}
\threeverse{\arabic{verse}}%Ex.29:18
{וְהִקְטַרְתָּ֤ אֶת־כׇּל־הָאַ֙יִל֙ הַמִּזְבֵּ֔חָה עֹלָ֥ה ה֖וּא לַֽיהֹוָ֑ה רֵ֣יחַ נִיח֔וֹחַ אִשֶּׁ֥ה לַיהֹוָ֖ה הֽוּא׃
\rashi{\rashiDH{ריח ניחוח. }נחת רוח לפני, שאמרתי ונעשה רצוני׃ 
}\rashi{\rashiDH{אשה. }לשון אש, והיא הקטרת איברים שעל האש׃ }}
{וְתַסֵּיק יָת כָּל דִּכְרָא לְמַדְבְּחָא עֲלָתָא הוּא קֳדָם יְיָ לְאִתְקַבָּלָא בְּרַעֲוָא קוּרְבָּנָא קֳדָם יְיָ הוּא׃}
{And thou shalt make the whole ram smoke upon the altar; it is a burnt-offering unto the \lord; it is a sweet savour, an offering made by fire unto the \lord.}{\arabic{verse}}
\threeverse{\aliya{חמישי}}%Ex.29:19
{וְלָ֣קַחְתָּ֔ אֵ֖ת הָאַ֣יִל הַשֵּׁנִ֑י וְסָמַ֨ךְ אַהֲרֹ֧ן וּבָנָ֛יו אֶת־יְדֵיהֶ֖ם עַל־רֹ֥אשׁ הָאָֽיִל׃}
{וְתִסַּב יָת דִּכְרָא תִּנְיָנָא וְיִסְמוֹךְ אַהֲרֹן וּבְנוֹהִי יָת יְדֵיהוֹן עַל רֵישׁ דִּכְרָא׃}
{And thou shalt take the other ram; and Aaron and his sons shall lay their hands upon the head of the ram.}{\arabic{verse}}
\threeverse{\arabic{verse}}%Ex.29:20
{וְשָׁחַטְתָּ֣ אֶת־הָאַ֗יִל וְלָקַחְתָּ֤ מִדָּמוֹ֙ וְנָֽתַתָּ֡ה עַל־תְּנוּךְ֩ אֹ֨זֶן אַהֲרֹ֜ן וְעַל־תְּנ֨וּךְ אֹ֤זֶן בָּנָיו֙ הַיְמָנִ֔ית וְעַל־בֹּ֤הֶן יָדָם֙ הַיְמָנִ֔ית וְעַל־בֹּ֥הֶן רַגְלָ֖ם הַיְמָנִ֑ית וְזָרַקְתָּ֧ אֶת־הַדָּ֛ם עַל־הַמִּזְבֵּ֖חַ סָבִֽיב׃
\rashi{\rashiDH{תנוך. }הוא הַסְּחוּס (ת״כ מילואים כ״א), גדר האמצעי שבתוך האוזן, שקורין טנדרו״ס׃ }\rashi{\rashiDH{בהן ידם. }הגודל, ובפרק האמצעי׃ 
}}
{וְתִכּוֹס יָת דִּכְרָא וְתִסַּב מִדְּמֵיהּ וְתִתֵּין עַל רוּם אוּדְנָא דְּאַהֲרֹן וְעַל רוּם אוּדְנָא דִּבְנוֹהִי דְּיַמִּינָא וְעַל אִלְיוֹן יַדְהוֹן דְּיַמִּינָא וְעַל אִלְיוֹן רַגְלְהוֹן דְּיַמִּינָא וְתִזְרוֹק יָת דְּמָא עַל מַדְבְּחָא סְחוֹר סְחוֹר׃}
{Then shalt thou kill the ram, and take of its blood, and put it upon the tip of the right ear of Aaron, and upon the tip of the right ear of his sons, and upon the thumb of their right hand, and upon the great toe of their right foot, and dash the blood against the altar round about.}{\arabic{verse}}
\threeverse{\arabic{verse}}%Ex.29:21
{וְלָקַחְתָּ֞ מִן־הַדָּ֨ם אֲשֶׁ֥ר עַֽל־הַמִּזְבֵּ֘חַ֮ וּמִשֶּׁ֣מֶן הַמִּשְׁחָה֒ וְהִזֵּיתָ֤ עַֽל־אַהֲרֹן֙ וְעַל־בְּגָדָ֔יו וְעַל־בָּנָ֛יו וְעַל־בִּגְדֵ֥י בָנָ֖יו אִתּ֑וֹ וְקָדַ֥שׁ הוּא֙ וּבְגָדָ֔יו וּבָנָ֛יו וּבִגְדֵ֥י בָנָ֖יו אִתּֽוֹ׃}
{וְתִסַּב מִן דְּמָא דְּעַל מַדְבְּחָא וּמִמִּשְׁחָא דִּרְבוּתָא וְתַדֵּי עַל אַהֲרֹן וְעַל לְבוּשׁוֹהִי וְעַל בְּנוֹהִי וְעַל לְבוּשֵׁי בְנוֹהִי עִמֵּיהּ וְיִתְקַדַּשׁ הוּא וּלְבוּשׁוֹהִי וּבְנוֹהִי וּלְבוּשֵׁי בְנוֹהִי עִמֵּיהּ׃}
{And thou shalt take of the blood that is upon the altar, and of the anointing oil, and sprinkle it upon Aaron, and upon his garments, and upon his sons, and upon the garments of his sons with him; and he and his garments shall be hallowed, and his sons and his sons’ garments with him.}{\arabic{verse}}
\threeverse{\arabic{verse}}%Ex.29:22
{וְלָקַחְתָּ֣ מִן־הָ֠אַ֠יִל הַחֵ֨לֶב וְהָֽאַלְיָ֜ה וְאֶת־הַחֵ֣לֶב \legarmeh  הַֽמְכַסֶּ֣ה אֶת־הַקֶּ֗רֶב וְאֵ֨ת יֹתֶ֤רֶת הַכָּבֵד֙ וְאֵ֣ת \legarmeh  שְׁתֵּ֣י הַכְּלָיֹ֗ת וְאֶת־הַחֵ֙לֶב֙ אֲשֶׁ֣ר עֲלֵיהֶ֔ן וְאֵ֖ת שׁ֣וֹק הַיָּמִ֑ין כִּ֛י אֵ֥יל מִלֻּאִ֖ים הֽוּא׃
\rashi{\rashiDH{החלב. }זה חלב הדקים או הקיבה (חולין מט׃)׃}\rashi{\rashiDH{והאליה. }מן הכליות ולמטה, כמו שמפורש בויקרא (ג, ט) שנאמר לְעֻמַּת הֶעָצֶה יְסִירֶנָּה, מקום שהכליות יועצות, ובאמורי הפר לא נאמר אליה, שאין אליה קריבה אלא בכבש וכבשה ואיל, אבל שור ועז אין טעונים אליה׃ }\rashi{\rashiDH{ואת שוק הימין. }לא מצינו הקטרה בשוק הימין עם האמורים, אלא זו בלבד׃ }\rashi{\rashiDH{כי איל מלואים הוא. }שלמים לשון שלמות, שהושלם בכל. מגיד הכתוב שהמלואים שלמים, שמשימים שלום למזבח, ולעובד העבודה, ולבעלים לכך אני מצריכו החזה, להיות לו לעובד העבודה למנה, וזה משה ששימש במלואים, והשאר אכלו אהרן ובניו, שהם בעלים, כמפורש בענין׃ }}
{וְתִסַּב מִן דִּכְרָא תַּרְבָּא וְאַלְיְתָא וְיָת תַּרְבָּא דְּחָפֵי יָת גַּוָּא וְיָת חֲצַר כַּבְדָּא וְיָת תַּרְתֵּין כּוֹלְיָן וְיָת תַּרְבָּא דַּעֲלֵיהוֹן וְיָת שָׁקָא דְּיַמִּינָא אֲרֵי דְּכַר קוּרְבָּנַיָּא הוּא׃}
{Also thou shalt take of the ram the fat, and the fat tail, and the fat that covereth the inwards, and the lobe of the liver, and the two kidneys, and the fat that is upon them, and the right thigh; for it is a ram of consecration;}{\arabic{verse}}
\threeverse{\arabic{verse}}%Ex.29:23
{וְכִכַּ֨ר לֶ֜חֶם אַחַ֗ת וְֽחַלַּ֨ת לֶ֥חֶם שֶׁ֛מֶן אַחַ֖ת וְרָקִ֣יק אֶחָ֑ד מִסַּל֙ הַמַּצּ֔וֹת אֲשֶׁ֖ר לִפְנֵ֥י יְהֹוָֽה׃
\rashi{\rashiDH{וככר לחם. }מן החלות׃}\rashi{\rashiDH{וחלת לחם שמן. }ממין הרבוכה׃}\rashi{\rashiDH{ורקיק. }מן הרקיקין אחד, מעשר שבכל מין ומין. ולא מצינו תרומת לחם הבא עם זבח נקטרת, אלא זו בלבד, שתרומת לחמי תודה ואיל נזיר נתונה לכהנים עם חזה ושוק, ומזה לא היה למשה למנה, אלא חזה בלבד׃ }}
{וּפִתָּא דִּלְחֵים חֲדָא וּגְרִיצְתָא דִּלְחֵים מְשַׁח חֲדָא וְאֶסְפּוֹג חַד מִסַּלָּא דְּפַטִּירַיָּא דִּקְדָם יְיָ׃}
{and one loaf of bread, and one cake of oiled bread, and one wafer, out of the basket of unleavened bread that is before the \lord.}{\arabic{verse}}
\threeverse{\arabic{verse}}%Ex.29:24
{וְשַׂמְתָּ֣ הַכֹּ֔ל עַ֚ל כַּפֵּ֣י אַהֲרֹ֔ן וְעַ֖ל כַּפֵּ֣י בָנָ֑יו וְהֵנַפְתָּ֥ אֹתָ֛ם תְּנוּפָ֖ה לִפְנֵ֥י יְהֹוָֽה׃
\rashi{\rashiDH{על כפי אהרן וגו׳ והנפת. }שניהם עסוקין בתנופה, הבעלים והכהן, הא כיצד, כהן מניח ידו תחת יד הבעלים ומניף, ובזה היו אהרן ובניו בעלים ומשה כהן׃ }\rashi{\rashiDH{תנופה. }מוליך ומביא למי שארבע רוחות העולם שלו, ותנופה מעכבת ומבטלת פורעניות ורוחות רעות. תרומה, מעלה ומוריד למי שהשמים והארץ שלו, ומעכבת טללים רעים (מנחות סב.)׃ }}
{וּתְשַׁוֵּי כוֹלָא עַל יְדֵי אַהֲרֹן וְעַל יְדֵי בְנוֹהִי וּתְרִים יָתְהוֹן אֲרָמָא קֳדָם יְיָ׃}
{And thou shalt put the whole upon the hands of Aaron, and upon the hands of his sons; and shalt wave them for a wave-offering before the \lord.}{\arabic{verse}}
\threeverse{\arabic{verse}}%Ex.29:25
{וְלָקַחְתָּ֤ אֹתָם֙ מִיָּדָ֔ם וְהִקְטַרְתָּ֥ הַמִּזְבֵּ֖חָה עַל־הָעֹלָ֑ה לְרֵ֤יחַ נִיח֙וֹחַ֙ לִפְנֵ֣י יְהֹוָ֔ה אִשֶּׁ֥ה ה֖וּא לַיהֹוָֽה׃
\rashi{\rashiDH{על העולה. }על האיל הראשון שהעלית עולה׃}\rashi{\rashiDH{לריח ניחוח. }לנחת רוח למי שאמר ונעשה רצונו׃}\rashi{\rashiDH{אשה. }לאש נתן׃}\rashi{\rashiDH{לה׳. }לשמו של מקום׃}}
{וְתִסַּב יָתְהוֹן מִיַּדְהוֹן וְתַסֵּיק לְמַדְבְּחָא עַל עֲלָתָא לְאִתְקַבָּלָא בְּרַעֲוָא קֳדָם יְיָ קוּרְבָּנָא הוּא קֳדָם יְיָ׃}
{And thou shalt take them from their hands, and make them smoke on the altar upon the burnt-offering, for a sweet savour before the \lord; it is an offering made by fire unto the \lord.}{\arabic{verse}}
\threeverse{\arabic{verse}}%Ex.29:26
{וְלָקַחְתָּ֣ אֶת־הֶֽחָזֶ֗ה מֵאֵ֤יל הַמִּלֻּאִים֙ אֲשֶׁ֣ר לְאַהֲרֹ֔ן וְהֵנַפְתָּ֥ אֹת֛וֹ תְּנוּפָ֖ה לִפְנֵ֣י יְהֹוָ֑ה וְהָיָ֥ה לְךָ֖ לְמָנָֽה׃}
{וְתִסַּב יָת חֶדְיָא מִדְּכַר קוּרְבָּנַיָּא דִּלְאַהֲרֹן וּתְרִים יָתֵיהּ אֲרָמָא קֳדָם יְיָ וִיהֵי לָךְ לֻחְלָק׃}
{And thou shalt take the breast of Aaron’s ram of consecration, and wave it for a wave-offering before the \lord; and it shall be thy portion.}{\arabic{verse}}
\threeverse{\arabic{verse}}%Ex.29:27
{וְקִדַּשְׁתָּ֞ אֵ֣ת \legarmeh  חֲזֵ֣ה הַתְּנוּפָ֗ה וְאֵת֙ שׁ֣וֹק הַתְּרוּמָ֔ה אֲשֶׁ֥ר הוּנַ֖ף וַאֲשֶׁ֣ר הוּרָ֑ם מֵאֵיל֙ הַמִּלֻּאִ֔ים מֵאֲשֶׁ֥ר לְאַהֲרֹ֖ן וּמֵאֲשֶׁ֥ר לְבָנָֽיו׃
\rashi{\rashiDH{וקדשת את חזה התנופה ואת שוק התרומה וגו׳. }קדשם לדורות, להיות נוהגת תרומתם והנפתם בחזה ושוק של שלמים, אבל לא להקטרה, אלא והיה לאהרן ולבניו לאכול׃ }\rashi{\rashiDH{תנופה. }לשון הולכה והבאה, וינטלי״ר בלע״ז׃ 
}\rashi{\rashiDH{הורם. }לשון מעלה ומוריד׃}}
{וּתְקַדֵּישׁ יָת חֶדְיָא דַּאֲרָמוּתָא וְיָת שָׁקָא דְּאַפְרָשׁוּתָא דְּאִתָּרַם וּדְאִתַּפְרַשׁ מִדְּכַר קוּרְבָּנַיָּא מִדִּלְאַהֲרֹן וּמִדִּלְבְּנוֹהִי׃}
{And thou shalt sanctify the breast of the wave-offering, and the thigh of the heave-offering, which is waved, and which is heaved up, of the ram of consecration, even of that which is Aaron’s, and of that which is his sons’ .}{\arabic{verse}}
\threeverse{\arabic{verse}}%Ex.29:28
{וְהָיָה֩ לְאַהֲרֹ֨ן וּלְבָנָ֜יו לְחׇק־עוֹלָ֗ם מֵאֵת֙ בְּנֵ֣י יִשְׂרָאֵ֔ל כִּ֥י תְרוּמָ֖ה ה֑וּא וּתְרוּמָ֞ה יִהְיֶ֨ה מֵאֵ֤ת בְּנֵֽי־יִשְׂרָאֵל֙ מִזִּבְחֵ֣י שַׁלְמֵיהֶ֔ם תְּרוּמָתָ֖ם לַיהֹוָֽה׃
\rashi{\rashiDH{לחק עולם מאת בני ישראל. }שהשלמים לבעלים, ואת החזה ואת השוק יתנו לכהן׃ }\rashi{\rashiDH{כי תרומה הוא. }החזה ושוק הזה׃ 
}}
{וִיהֵי לְאַהֲרֹן וְלִבְנוֹהִי לִקְיָם עָלַם מִן בְּנֵי יִשְׂרָאֵל אֲרֵי אַפְרָשׁוּתָא הוּא וְאַפְרָשׁוּתָא יְהֵי מִן בְּנֵי יִשְׂרָאֵל מִנִּכְסַת קוּדְשֵׁיהוֹן אַפְרָשׁוּתְהוֹן קֳדָם יְיָ׃}
{And it shall be for Aaron and his sons as a due for ever from the children of Israel; for it is a heave-offering; and it shall be a heave-offering from the children of Israel of their sacrifices of peace-offerings, even their heave-offering unto the \lord.}{\arabic{verse}}
\threeverse{\arabic{verse}}%Ex.29:29
{וּבִגְדֵ֤י הַקֹּ֙דֶשׁ֙ אֲשֶׁ֣ר לְאַהֲרֹ֔ן יִהְי֥וּ לְבָנָ֖יו אַחֲרָ֑יו לְמׇשְׁחָ֣ה בָהֶ֔ם וּלְמַלֵּא־בָ֖ם אֶת־יָדָֽם׃
\rashi{\rashiDH{לבניו אחריו. }למי שבא בגדולה אחריו׃}\rashi{\rashiDH{למשחה. }להתגדל בהם, שיש משיחה שהיא לשון שררה, כמו לְךָ נְתַתִּים לְמָשְׁחָה (במדבר יח, ח), אַל תִּגְעוּ בִמְשִׁיחָי (תהלים קה, טו)׃ }\rashi{\rashiDH{ולמלא בם את ידם. }על ידי הבגדים הוא מתלבש בכהונה גדולה׃}}
{וּלְבוּשֵׁי קוּדְשָׁא דִּלְאַהֲרֹן יְהוֹן לִבְנוֹהִי בָתְרוֹהִי לְרַבָּאָה בְּהוֹן וּלְקָרָבָא בְּהוֹן יָת קוּרְבָּנְהוֹן׃}
{And the holy garments of Aaron shall be for his sons after him, to be anointed in them, and to be consecrated in them.}{\arabic{verse}}
\threeverse{\arabic{verse}}%Ex.29:30
{שִׁבְעַ֣ת יָמִ֗ים יִלְבָּשָׁ֧ם הַכֹּהֵ֛ן תַּחְתָּ֖יו מִבָּנָ֑יו אֲשֶׁ֥ר יָבֹ֛א אֶל־אֹ֥הֶל מוֹעֵ֖ד לְשָׁרֵ֥ת בַּקֹּֽדֶשׁ׃
\rashi{\rashiDH{שבעת ימים. }רצופין׃}\rashi{\rashiDH{ילבשם הכהן. }אשר יקום מבניו תחתיו לכהונה גדולה, כשימנוהו להיות כהן גדול׃ 
}\rashi{\rashiDH{אשר יבא אל אהל מועד. }אותו כהן המוכן להכנס לפני ולפנים ביום הכפורים, וזהו כהן גדול, שאין עבודת יום הכפורים כשרה אלא בו (יומא עג.)׃ }\rashi{\rashiDH{תחתיו מבניו. }מלמד, שאם יש לו לכהן גדול בן ממלא את מקומו, ימנוהו כהן גדול תחתיו (ת״כ ו, טו)׃ }\rashi{\rashiDH{הכהן תחתיו מבניו. }מכאן ראיה, כל לשון כהן, לשון פועל עובד ממש, לפיכך ניגון תְּבִיר נמשך לפניו׃ }}
{שִׁבְעָא יוֹמִין יִלְבְּשִׁנּוּן כָּהֲנָא תְּחוֹתוֹהִי מִבְּנוֹהִי דְּיֵיעוֹל לְמַשְׁכַּן זִמְנָא לְשַׁמָּשָׁא בְּקוּדְשָׁא׃}
{Seven days shall the son that is priest in his stead put them on, even he who cometh into the tent of meeting to minister in the holy place.}{\arabic{verse}}
\threeverse{\arabic{verse}}%Ex.29:31
{וְאֵ֛ת אֵ֥יל הַמִּלֻּאִ֖ים תִּקָּ֑ח וּבִשַּׁלְתָּ֥ אֶת־בְּשָׂר֖וֹ בְּמָקֹ֥ם קָדֹֽשׁ׃
\rashi{\rashiDH{במקום קדש. }בחצר אהל מועד, שהשלמים הללו קדשי קדשים היו׃ }}
{וְיָת דְּכַר קוּרְבָּנַיָּא תִּסַּב וּתְבַשֵּׁיל יָת בִּשְׂרֵיהּ בַּאֲתַר קַדִּישׁ׃}
{And thou shalt take the ram of consecration, and seethe its flesh in a holy place.}{\arabic{verse}}
\threeverse{\arabic{verse}}%Ex.29:32
{וְאָכַ֨ל אַהֲרֹ֤ן וּבָנָיו֙ אֶת־בְּשַׂ֣ר הָאַ֔יִל וְאֶת־הַלֶּ֖חֶם אֲשֶׁ֣ר בַּסָּ֑ל פֶּ֖תַח אֹ֥הֶל מוֹעֵֽד׃
\rashi{\rashiDH{פתח אהל מועד. }כל החצר קרוי כן׃}}
{וְיֵיכוֹל אַהֲרֹן וּבְנוֹהִי יָת בְּשַׂר דִּכְרָא וְיָת לַחְמָא דִּבְסַלָּא בִּתְרַע מַשְׁכַּן זִמְנָא׃}
{And Aaron and his sons shall eat the flesh of the ram, and the bread that is in the basket, at the door of the tent of meeting.}{\arabic{verse}}
\threeverse{\arabic{verse}}%Ex.29:33
{וְאָכְל֤וּ אֹתָם֙ אֲשֶׁ֣ר כֻּפַּ֣ר בָּהֶ֔ם לְמַלֵּ֥א אֶת־יָדָ֖ם לְקַדֵּ֣שׁ אֹתָ֑ם וְזָ֥ר לֹא־יֹאכַ֖ל כִּי־קֹ֥דֶשׁ הֵֽם׃
\rashi{\rashiDH{ואכלו אתם. }אהרן ובניו, לפי שהם בעליהם׃ }\rashi{\rashiDH{אשר כפר בהם. }להם כל זרות ותיעוב׃ 
}\rashi{\rashiDH{למלא את ידם. }באיל ולחם הללו׃}\rashi{\rashiDH{לקדש אתם. }שעל ידי המלואים הללו נתמלאו ידיהם ונתקדשו לכהונה׃}\rashi{\rashiDH{כי קדש הם. }קדשי קדשים, ומכאן למדנו אזהרה לזר האוכל קדשי קדשים, שנתן המקרא טעם לדבר משום דקדש הם׃ }}
{וְיֵיכְלוּן יָתְהוֹן דְּאִתְכַּפַּר בְּהוֹן לְקָרָבָא יָת קוּרְבָּנְהוֹן לְקַדָּשָׁא יָתְהוֹן וְחִילוֹנַי לָא יֵיכוֹל אֲרֵי קוּדְשָׁא אִנּוּן׃}
{And they shall eat those things wherewith atonement was made, to consecrate and to sanctify them; but a stranger shall not eat thereof, because they are holy.}{\arabic{verse}}
\threeverse{\arabic{verse}}%Ex.29:34
{וְֽאִם־יִוָּתֵ֞ר מִבְּשַׂ֧ר הַמִּלֻּאִ֛ים וּמִן־הַלֶּ֖חֶם עַד־הַבֹּ֑קֶר וְשָׂרַפְתָּ֤ אֶת־הַנּוֹתָר֙ בָּאֵ֔שׁ לֹ֥א יֵאָכֵ֖ל כִּי־קֹ֥דֶשׁ הֽוּא׃}
{וְאִם יִשְׁתְּאַר מִבְּשַׂר קוּרְבָּנַיָּא וּמִן לַחְמָא עַד צַפְרָא וְתוֹקֵיד יָת דְּאִשְׁתְּאַר בְּנוּרָא לָא יִתְאֲכִיל אֲרֵי קוּדְשָׁא הוּא׃}
{And if aught of the flesh of the consecration, or of the bread, remain unto the morning, then thou shalt burn the remainder with fire; it shall not be eaten, because it is holy.}{\arabic{verse}}
\threeverse{\arabic{verse}}%Ex.29:35
{וְעָשִׂ֜יתָ לְאַהֲרֹ֤ן וּלְבָנָיו֙ כָּ֔כָה כְּכֹ֥ל אֲשֶׁר־צִוִּ֖יתִי אֹתָ֑כָה שִׁבְעַ֥ת יָמִ֖ים תְּמַלֵּ֥א יָדָֽם׃
\rashi{\rashiDH{ועשית לאהרן ולבניו ככה. }שנה הכתוב וכפל לעכב, שאם חסר דבר אחד מכל האמור בענין, לא נתמלאו ידיהם להיות כהנים, ועבודתם פסולה׃ }\rashi{\rashiDH{אתכה. }כמו אותך׃}\rashi{\rashiDH{שבעת ימים תמלא וגו׳. }בענין הזה ובקרבנות הללו בכל יום׃}}
{וְתַעֲבֵיד לְאַהֲרֹן וְלִבְנוֹהִי כְּדֵין כְּכֹל דְּפַקֵּידִית יָתָךְ שִׁבְעָא יוֹמִין תְּקָרֵיב קוּרְבָּנְהוֹן׃}
{And thus shalt thou do unto Aaron, and to his sons, according to all that I have commanded thee; seven days shalt thou consecrate them.}{\arabic{verse}}
\threeverse{\arabic{verse}}%Ex.29:36
{וּפַ֨ר חַטָּ֜את תַּעֲשֶׂ֤ה לַיּוֹם֙ עַל־הַכִּפֻּרִ֔ים וְחִטֵּאתָ֙ עַל־הַמִּזְבֵּ֔חַ בְּכַפֶּרְךָ֖ עָלָ֑יו וּמָֽשַׁחְתָּ֥ אֹת֖וֹ לְקַדְּשֽׁוֹ׃
\rashi{\rashiDH{על הכפורים. }בשביל הכפורים, לכפר על המזבח מכל זרות ותיעוב, ולפי שנאמר שבעת ימים תמלא ידם, אין לי אלא דבר הבא בשבילם, כגון האילים והלחם, אבל הבא בשביל המזבח, כגון פר שהוא לְחִטּוּי המזבח, לא שמענו, לכך הוצרך מקרא זה. ומדרש תורת כהנים (מילואים ט״ו) אומר, כפרת המזבח הוצרך, שמא התנדב איש דבר גזול במלאכת המשכן והמזבח׃ }\rashi{\rashiDH{וחטאת. }וּתְדַכֵּי, לשון מתנת דמים הנתונים באצבע קרוי חטוי׃ }\rashi{\rashiDH{ומשחת אותו. }בשמן המשחה, וכל המשיחות כמין כ״ף יונית׃ }}
{וְתוֹרָא דְּחַטָּאתָא תַּעֲבֵיד לְיוֹמָא עַל כִּפּוּרַיָּא וּתְדַכֵּי עַל מַדְבְּחָא בְּכַפָּרוּתָךְ עֲלוֹהִי וּתְרַבֵּי יָתֵיהּ לְקַדָּשׁוּתֵיהּ׃}
{And every day shalt thou offer the bullock of sin-offering, beside the other offerings of atonement; and thou shalt do the purification upon the altar when thou makest atonement for it; and thou shalt anoint it, to sanctify it.}{\arabic{verse}}
\threeverse{\arabic{verse}}%Ex.29:37
{שִׁבְעַ֣ת יָמִ֗ים תְּכַפֵּר֙ עַל־הַמִּזְבֵּ֔חַ וְקִדַּשְׁתָּ֖ אֹת֑וֹ וְהָיָ֤ה הַמִּזְבֵּ֙חַ֙ קֹ֣דֶשׁ קׇֽדָשִׁ֔ים כׇּל־הַנֹּגֵ֥עַ בַּמִּזְבֵּ֖חַ יִקְדָּֽשׁ׃ \setuma         
\rashi{\rashiDH{והיה המזבח קדש. }ומה היא קדושתו, כל הנוגע במזבח יקדש, אפילו קרבן פסול שעלה עליו, קדשו המזבח להכשירו שלא ירד, מתוך שנאמר כל הנוגע וגו׳ יקדש, שומע אני בין ראוי בין שאינו ראוי, כגון דבר שלא היה פסולו בקדש, כגון הרובע והנרבע, ומוקצה, ונעבד, והטריפה, וכיוצא בהן, תלמוד לומר וזה אשר תעשה, הסמוך אחריו, מה עולה ראויה אף כל ראויה (ת״כ ו, ב.  זבחים פג׃), שנראה לו כבר ונפסל משבא לעזרה, כגון הלן, והיוצא, והטמא, ושנשחט במחשבת חוץ לזמנו וחוץ למקומו, וכיוצא בהן׃ }}
{שִׁבְעָא יוֹמִין תְּכַפַּר עַל מַדְבְּחָא וּתְקַדֵּישׁ יָתֵיהּ וִיהֵי מַדְבְּחָא קֹדֶשׁ קוּדְשִׁין כָּל דְּיִקְרַב בְּמַדְבְּחָא יִתְקַדַּשׁ׃}
{Seven days thou shalt make atonement for the altar, and sanctify it; thus shall the altar be most holy; whatsoever toucheth the altar shall be holy.}{\arabic{verse}}
\threeverse{\aliya{ששי}}%Ex.29:38
{וְזֶ֕ה אֲשֶׁ֥ר תַּעֲשֶׂ֖ה עַל־הַמִּזְבֵּ֑חַ כְּבָשִׂ֧ים בְּנֵֽי־שָׁנָ֛ה שְׁנַ֥יִם לַיּ֖וֹם תָּמִֽיד׃}
{וְדֵין דְּתַעֲבֵיד עַל מַדְבְּחָא אִמְּרִין בְּנֵי שְׁנָא תְּרֵין לְיוֹמָא תְּדִירָא׃}
{Now this is that which thou shalt offer upon the altar: two lambs of the first year day by day continually.}{\arabic{verse}}
\threeverse{\arabic{verse}}%Ex.29:39
{אֶת־הַכֶּ֥בֶשׂ הָאֶחָ֖ד תַּעֲשֶׂ֣ה בַבֹּ֑קֶר וְאֵת֙ הַכֶּ֣בֶשׂ הַשֵּׁנִ֔י תַּעֲשֶׂ֖ה בֵּ֥ין הָעַרְבָּֽיִם׃}
{יָת אִמְּרָא חַד תַּעֲבֵיד בְּצַפְרָא וְיָת אִמְּרָא תִּנְיָנָא תַּעֲבֵיד בֵּין שִׁמְשַׁיָּא׃}
{The one lamb thou shalt offer in the morning; and the other lamb thou shalt offer at dusk.}{\arabic{verse}}
\threeverse{\arabic{verse}}%Ex.29:40
{וְעִשָּׂרֹ֨ן סֹ֜לֶת בָּל֨וּל בְּשֶׁ֤מֶן כָּתִית֙ רֶ֣בַע הַהִ֔ין וְנֵ֕סֶךְ רְבִיעִ֥ת הַהִ֖ין יָ֑יִן לַכֶּ֖בֶשׂ הָאֶחָֽד׃
\rashi{\rashiDH{ועשרון סולת. }עשירית האיפה, מ״ג ביצים וחומש ביצה׃ }\rashi{\rashiDH{בשמן כתית. }לא לחובה נאמר כתית, אלא להכשיר, לפי שנאמר כתית למאור, ומשמע למאור ולא למנחות, יכול לפסלו למנחות, תלמוד לומר כאן כתית, ולא נאמר כתית למאור אלא למעט מנחות שאין צריך כתית, שאף הטחון ברחיים כשר בהן (מנחות פו.)׃ }\rashi{\rashiDH{רבע ההין. }שלשה לוגין׃}\rashi{\rashiDH{ונסך. }לספלים, כמו ששנינו במסכת סוכה (מח.), שני ספלים של כסף היו בראש המזבח, ומנוקבים כמין שני חוטמין דקים, נותן היין לתוכו, והוא מקלח ויוצא דרך החוטם, ונופל על גג המזבח, ומשם יורד לשיתין במזבח בית עולמים, ובמזבח הנחשת יורד מן המזבח לארץ׃ }}
{וְעֶשְׂרוֹנָא סוּלְתָּא דְּפִיל בִּמְשַׁח כָּתִישָׁא רַבְעוּת הִינָא וְנִסְכָּא רַבְעוּת הִינָא חַמְרָא לְאִמְּרָא חַד׃}
{And with the one lamb a tenth part of an ephah of fine flour mingled with the fourth part of a hin of beaten oil; and the fourth part of a hin of wine for a drink-offering.}{\arabic{verse}}
\threeverse{\arabic{verse}}%Ex.29:41
{וְאֵת֙ הַכֶּ֣בֶשׂ הַשֵּׁנִ֔י תַּעֲשֶׂ֖ה בֵּ֣ין הָעַרְבָּ֑יִם כְּמִנְחַ֨ת הַבֹּ֤קֶר וּכְנִסְכָּהּ֙ תַּֽעֲשֶׂה־לָּ֔הּ לְרֵ֣יחַ נִיחֹ֔חַ אִשֶּׁ֖ה לַיהֹוָֽה׃
\rashi{\rashiDH{לריח ניחוח. }על המנחה נאמר, שמנחת נסכים כולה כליל, וסדר הקרבתם, האיברים בתחלה ואחר כך המנחה, שנאמר עֹלָה וּמִנְחָה (ויקרא כג, לז)׃ }}
{וְיָת אִמְּרָא תִּנְיָנָא תַּעֲבֵיד בֵּין שִׁמְשַׁיָּא כְּמִנְחַת צַפְרָא וּכְנִסְכַּהּ תַּעֲבֵיד לַהּ לְאִתְקַבָּלָא בְּרַעֲוָא קוּרְבָּנָא קֳדָם יְיָ׃}
{And the other lamb thou shalt offer at dusk, and shalt do thereto according to the meal-offering of the morning, and according to the drink-offering thereof, for a sweet savour, an offering made by fire unto the \lord.}{\arabic{verse}}
\threeverse{\arabic{verse}}%Ex.29:42
{עֹלַ֤ת תָּמִיד֙ לְדֹרֹ֣תֵיכֶ֔ם פֶּ֥תַח אֹֽהֶל־מוֹעֵ֖ד לִפְנֵ֣י יְהֹוָ֑ה אֲשֶׁ֨ר אִוָּעֵ֤ד לָכֶם֙ שָׁ֔מָּה לְדַבֵּ֥ר אֵלֶ֖יךָ שָֽׁם׃
\rashi{\rashiDH{תמיד. }מיום אל יום, ולא יפסיק יום בנתיים׃ }\rashi{\rashiDH{אשר אועד לכם. }כשאקבע מועד לדבר אליך, שם אקבענו לבא. ויש מרבותינו למדים מכאן, שמעל מזבח הנחשת היה הקב״ה מדבר עם משה משהוקם המשכן, ויש אומרים מעל הכפורת כמו שנאמר וְדִבַּרְתִּי אִתְּךָ מֵעַל הַכַּפֹּרֶת (שמות כה, כב). ואשר אועד לכם האמור כאן אינו אמור על המזבח, אלא על אהל מועד הנזכר במקרא׃ }}
{עֲלַת תְּדִירָא לְדָרֵיכוֹן בִּתְרַע מַשְׁכַּן זִמְנָא קֳדָם יְיָ דַּאֲזָמֵין מֵימְרִי לְכוֹן תַּמָּן לְמַלָּלָא עִמָּךְ תַּמָּן׃}
{It shall be a continual burnt-offering throughout your generations at the door of the tent of meeting before the \lord, where I will meet with you, to speak there unto thee.}{\arabic{verse}}
\threeverse{\arabic{verse}}%Ex.29:43
{וְנֹעַדְתִּ֥י שָׁ֖מָּה לִבְנֵ֣י יִשְׂרָאֵ֑ל וְנִקְדַּ֖שׁ בִּכְבֹדִֽי׃
\rashi{\rashiDH{ונועדתי שמה. }אתועד עמם בדבור, כמלך הקובע מקום מועד לדבר עם עבדיו שם׃ }\rashi{\rashiDH{ונקדש. }המשכן׃}\rashi{\rashiDH{בכבודי. }שתשרה שכינתי בו. ומדרש אגדה, אל תקרי בכבודי אלא במכובדי, במכובדים שלי, כאן רמז לו מיתת בני אהרן ביום הקמתו, וזהו שאמר משה הוּא אֲשֶׁר דִּבֶּר ה׳ לֵאמֹר בִּקְרֹבַי אֶקָּדֵשׁ (ויקרא י, ג), והיכן דִּבֶּר, ונקדש בכבודי׃ 
}}
{וַאֲזָמֵין מֵימְרִי תַּמָּן לִבְנֵי יִשְׂרָאֵל וְאֶתְקַדַּשׁ בִּיקָרִי׃}
{And there I will meet with the children of Israel; and [the Tent] shall be sanctified by My glory.}{\arabic{verse}}
\threeverse{\arabic{verse}}%Ex.29:44
{וְקִדַּשְׁתִּ֛י אֶת־אֹ֥הֶל מוֹעֵ֖ד וְאֶת־הַמִּזְבֵּ֑חַ וְאֶת־אַהֲרֹ֧ן וְאֶת־בָּנָ֛יו אֲקַדֵּ֖שׁ לְכַהֵ֥ן לִֽי׃}
{וַאֲקַדֵּישׁ יָת מַשְׁכַּן זִמְנָא וְיָת מַדְבְּחָא וְיָת אַהֲרֹן וְיָת בְּנוֹהִי אֲקַדֵּישׁ לְשַׁמָּשָׁא קֳדָמָי׃}
{And I will sanctify the tent of meeting, and the altar; Aaron also and his sons will I sanctify, to minister to Me in the priest’s office.}{\arabic{verse}}
\threeverse{\arabic{verse}}%Ex.29:45
{וְשָׁ֣כַנְתִּ֔י בְּת֖וֹךְ בְּנֵ֣י יִשְׂרָאֵ֑ל וְהָיִ֥יתִי לָהֶ֖ם לֵאלֹהִֽים׃}
{וְאַשְׁרֵי שְׁכִינְתִי בְּגוֹ בְּנֵי יִשְׂרָאֵל וְאֶהְוֵי לְהוֹן לֶאֱלָהּ׃}
{And I will dwell among the children of Israel, and will be their God.}{\arabic{verse}}
\threeverse{\arabic{verse}}%Ex.29:46
{וְיָדְע֗וּ כִּ֣י אֲנִ֤י יְהֹוָה֙ אֱלֹ֣הֵיהֶ֔ם אֲשֶׁ֨ר הוֹצֵ֧אתִי אֹתָ֛ם מֵאֶ֥רֶץ מִצְרַ֖יִם לְשׇׁכְנִ֣י בְתוֹכָ֑ם אֲנִ֖י יְהֹוָ֥ה אֱלֹהֵיהֶֽם׃ \petucha 
\rashi{\rashiDH{לשכני בתוכם. }על מנת לשכון אני בתוכם׃}}
{וְיִדְּעוּן אֲרֵי אֲנָא יְיָ אֱלָהֲהוֹן דְּאַפֵּיקִית יָתְהוֹן מֵאַרְעָא דְּמִצְרַיִם לְאַשְׁרָאָה שְׁכִינְתִי בֵּינֵיהוֹן אֲנָא יְיָ אֱלָהֲהוֹן׃}
{And they shall know that I am the \lord\space their God, that brought them forth out of the land of Egypt, that I may dwell among them. I am the \lord\space their God.}{\arabic{verse}}
\newperek
\threeverse{\aliya{שביעי}}%Ex.30:1
{וְעָשִׂ֥יתָ מִזְבֵּ֖חַ מִקְטַ֣ר קְטֹ֑רֶת עֲצֵ֥י שִׁטִּ֖ים תַּעֲשֶׂ֥ה אֹתֽוֹ׃
\rashi{\rashiDH{מקטר קטורת. }להעלות עליו קיטור עשן סמים׃}}
{וְתַעֲבֵיד מַדְבְּחָא לְאַקְטָרָא עֲלוֹהִי קְטֹרֶת בּוּסְמִין דְּאָעֵי שִׁטִּין תַּעֲבֵיד יָתֵיהּ׃}
{And thou shalt make an altar to burn incense upon; of acacia-wood shalt thou make it.}{\Roman{chap}}
\threeverse{\arabic{verse}}%Ex.30:2
{אַמָּ֨ה אׇרְכּ֜וֹ וְאַמָּ֤ה רׇחְבּוֹ֙ רָב֣וּעַ יִהְיֶ֔ה וְאַמָּתַ֖יִם קֹמָת֑וֹ מִמֶּ֖נּוּ קַרְנֹתָֽיו׃}
{אַמְּתָא אוּרְכֵּיהּ וְאַמְּתָא פוּתְיֵיהּ מְרוּבַּע יְהֵי וְתַרְתֵּין אַמִּין רוּמֵיהּ מִנֵּיהּ קַרְנוֹהִי׃}
{A cubit shall be the length thereof, and a cubit the breadth thereof; foursquare shall it be; and two cubits shall be the height thereof; the horns thereof shall be of one piece with it.}{\arabic{verse}}
\threeverse{\arabic{verse}}%Ex.30:3
{וְצִפִּיתָ֨ אֹת֜וֹ זָהָ֣ב טָה֗וֹר אֶת־גַּגּ֧וֹ וְאֶת־קִירֹתָ֛יו סָבִ֖יב וְאֶת־קַרְנֹתָ֑יו וְעָשִׂ֥יתָ לּ֛וֹ זֵ֥ר זָהָ֖ב סָבִֽיב׃
\rashi{\rashiDH{את גגו. }זה היה לו גג, אבל מזבח העולה לא היה לו גג, אלא ממלאים חללו אדמה בכל חנייתם׃ }\rashi{\rashiDH{זר זהב. }סימן הוא לכתר כהונה׃}}
{וְתִחְפֵי יָתֵיהּ דְּהַב דְּכֵי יָת אִגָּרֵיהּ וְיָת כּוּתְלוֹהִי סְחוֹר סְחוֹר וְיָת קַרְנוֹהִי וְתַעֲבֵיד לֵיהּ זִיר דִּדְהַב סְחוֹר סְחוֹר׃}
{And thou shalt overlay it with pure gold, the top thereof, and the sides thereof round about, and the horns thereof; and thou shalt make unto it a crown of gold round about.}{\arabic{verse}}
\threeverse{\arabic{verse}}%Ex.30:4
{וּשְׁתֵּי֩ טַבְּעֹ֨ת זָהָ֜ב תַּֽעֲשֶׂה־לּ֣וֹ \legarmeh  מִתַּ֣חַת לְזֵר֗וֹ עַ֚ל שְׁתֵּ֣י צַלְעֹתָ֔יו תַּעֲשֶׂ֖ה עַל־שְׁנֵ֣י צִדָּ֑יו וְהָיָה֙ לְבָתִּ֣ים לְבַדִּ֔ים לָשֵׂ֥את אֹת֖וֹ בָּהֵֽמָּה׃
\rashi{\rashiDH{צלעותיו. }כאן הוא לשון זויות כתרגומו, לפי שנאמר על שני צדיו, על שתי זויותיו שבשני צדיו׃ }\rashi{\rashiDH{והיה. }מעשה הטבעות האלה׃}\rashi{\rashiDH{לבתים לבדים. }לכל בית תהיה הטבעת לבד׃}}
{וְתַרְתֵּין עִזְקָן דִּדְהַב תַּעֲבֵיד לֵיהּ מִלְּרַע לְזֵירֵיהּ עַל תַּרְתֵּין זָוְיָתֵיהּ תַּעֲבֵיד עַל תְּרֵין סִטְרוֹהִי וִיהֵי לְאַתְרָא לַאֲרִיחַיָּא לְמִטַּל יָתֵיהּ בְּהוֹן׃}
{And two golden rings shalt thou make for it under the crown thereof, upon the two ribs thereof, upon the two sides of it shalt thou make them; and they shall be for places for staves wherewith to bear it.}{\arabic{verse}}
\threeverse{\arabic{verse}}%Ex.30:5
{וְעָשִׂ֥יתָ אֶת־הַבַּדִּ֖ים עֲצֵ֣י שִׁטִּ֑ים וְצִפִּיתָ֥ אֹתָ֖ם זָהָֽב׃}
{וְתַעֲבֵיד יָת אֲרִיחַיָּא דְּאָעֵי שִׁטִּין וְתִחְפֵי יָתְהוֹן דַּהְבָּא׃}
{And thou shalt make the staves of acacia-wood, and overlay them with gold.}{\arabic{verse}}
\threeverse{\arabic{verse}}%Ex.30:6
{וְנָתַתָּ֤ה אֹתוֹ֙ לִפְנֵ֣י הַפָּרֹ֔כֶת אֲשֶׁ֖ר עַל־אֲרֹ֣ן הָעֵדֻ֑ת לִפְנֵ֣י הַכַּפֹּ֗רֶת אֲשֶׁר֙ עַל־הָ֣עֵדֻ֔ת אֲשֶׁ֛ר אִוָּעֵ֥ד לְךָ֖ שָֽׁמָּה׃
\rashi{\rashiDH{לפני הפרכת. }שמא תאמר משוך מכנגד הארון לצפון או לדרום, תלמוד לומר לפני הכפרת, מכוון כנגד הארון מבחוץ׃ 
}}
{וְתִתֵּין יָתֵיהּ קֳדָם פָּרוּכְתָּא דְּעַל אֲרוֹנָא דְּסָהֲדוּתָא קֳדָם כָּפוּרְתָּא דְּעַל סָהֲדוּתָא דַּאֲזָמֵין מֵימְרִי לָךְ תַּמָּן׃}
{And thou shalt put it before the veil that is by the ark of the testimony, before the ark-cover that is over the testimony, where I will meet with thee.}{\arabic{verse}}
\threeverse{\arabic{verse}}%Ex.30:7
{וְהִקְטִ֥יר עָלָ֛יו אַהֲרֹ֖ן קְטֹ֣רֶת סַמִּ֑ים בַּבֹּ֣קֶר בַּבֹּ֗קֶר בְּהֵיטִיב֛וֹ אֶת־הַנֵּרֹ֖ת יַקְטִירֶֽנָּה׃
\rashi{\rashiDH{בהיטיבו. }לשון נקוי הבזיכין של המנורה מדשן הפתילות שנשרפו בלילה, והיה מטיבן בכל בקר ובקר׃ }\rashi{\rashiDH{הנרות. }לוצי״ש בלע״ז, וכן כל נרות האמורות במנורה, חוץ ממקום שנאמר שם העלאה, שהוא לשון הדלקה׃ }}
{וְיַקְטַר עֲלוֹהִי אַהֲרֹן קְטֹרֶת בּוּסְמִין בִּצְפַר בִּצְפַר בְּאַתְקָנוּתֵיהּ יָת בּוֹצִינַיָּא יַקְטְרִנַּהּ׃}
{And Aaron shall burn thereon incense of sweet spices; every morning, when he dresseth the lamps, he shall burn it.}{\arabic{verse}}
\threeverse{\aliya{מפטיר}}%Ex.30:8
{וּבְהַעֲלֹ֨ת אַהֲרֹ֧ן אֶת־הַנֵּרֹ֛ת בֵּ֥ין הָעַרְבַּ֖יִם יַקְטִירֶ֑נָּה קְטֹ֧רֶת תָּמִ֛יד לִפְנֵ֥י יְהֹוָ֖ה לְדֹרֹתֵיכֶֽם׃
\rashi{\rashiDH{ובהעלות. }כשידליקם להעלות להבתן׃}\rashi{\rashiDH{יקטירנה. }בכל יום, פרס מקטיר שחרית, ופרס מקטיר בין הערבים׃ }}
{וּבְאַדְלָקוּת אַהֲרֹן יָת בּוֹצִינַיָּא בֵּין שִׁמְשַׁיָּא יַקְטְרִנַּהּ קְטֹרֶת בּוּסְמִין תְּדִירָא קֳדָם יְיָ לְדָרֵיכוֹן׃}
{And when Aaron lighteth the lamps at dusk, he shall burn it, a perpetual incense before the \lord\space throughout your generations.}{\arabic{verse}}
\threeverse{\arabic{verse}}%Ex.30:9
{לֹא־תַעֲל֥וּ עָלָ֛יו קְטֹ֥רֶת זָרָ֖ה וְעֹלָ֣ה וּמִנְחָ֑ה וְנֵ֕סֶךְ לֹ֥א תִסְּכ֖וּ עָלָֽיו׃
\rashi{\rashiDH{לא תעלו עליו. }על מזבח זה׃}\rashi{\rashiDH{קטרת זרה. }שום קטורת של נדבה, כולן זרות לו חוץ מזו׃ }\rashi{\rashiDH{ועולה ומנחה. }ולא עולה ומנחה. עולה של בהמה ועוף, ומנחה היא של לחם׃ }}
{לָא תַסְּקוּן עֲלוֹהִי קְטֹרֶת בּוּסְמִין נוּכְרָאִין וַעֲלָתָא וּמִנְחָתָא וְנִסְכָּא לָא תְנַסְּכוּן עֲלוֹהִי׃}
{Ye shall offer no strange incense thereon, nor burnt-offering, nor meal-offering; and ye shall pour no drink-offering thereon.}{\arabic{verse}}
\threeverse{\arabic{verse}}%Ex.30:10
{וְכִפֶּ֤ר אַהֲרֹן֙ עַל־קַרְנֹתָ֔יו אַחַ֖ת בַּשָּׁנָ֑ה מִדַּ֞ם חַטַּ֣את הַכִּפֻּרִ֗ים אַחַ֤ת בַּשָּׁנָה֙ יְכַפֵּ֤ר עָלָיו֙ לְדֹרֹ֣תֵיכֶ֔ם קֹֽדֶשׁ־קׇדָשִׁ֥ים ה֖וּא לַיהֹוָֽה׃ \petucha 
\rashi{\rashiDH{וכפר אהרן. }מתן דמים׃}\rashi{\rashiDH{אחת בשנה. }ביום הכפורים, הוא שנאמר באחרי מות וְיָצָא אֶל הַמִּזְבֵּח אֲשֶׁר לִפְנֵי ה׳ וְכִפֶּר עָלָיו (ויקרא טז, יח)׃ }\rashi{\rashiDH{חטאת הכפרים. }הם פר ושעיר של יום הכפורים, המכפרים על טומאת מקדש וקדשיו׃ }\rashi{\rashiDH{קדש קדשים. }המזבח מקודש לדברים הללו בלבד, ולא לעבודה אחרת׃ }}
{וִיכַפַּר אַהֲרֹן עַל קַרְנָתֵיהּ חֲדָא בְּשַׁתָּא מִדַּם חַטַּאת כִּפּוּרַיָּא חֲדָא בְּשַׁתָּא יְכַפַּר עֲלוֹהִי לְדָרֵיכוֹן קֹדֶשׁ קוּדְשִׁין הוּא קֳדָם יְיָ׃}
{And Aaron shall make atonement upon the horns of it once in the year; with the blood of the sin-offering of atonement once in the year shall he make atonement for it throughout your generations; it is most holy unto the \lord.’}{\arabic{verse}}
\newparsha{כי תשא}
\threeverse{\aliya{כי תשא}}%Ex.30:11
{וַיְדַבֵּ֥ר יְהֹוָ֖ה אֶל־מֹשֶׁ֥ה לֵּאמֹֽר׃}
{וּמַלֵּיל יְיָ עִם מֹשֶׁה לְמֵימַר׃}
{And the \lord\space spoke unto Moses, saying:}{\arabic{verse}}
\threeverse{\arabic{verse}}%Ex.30:12
{כִּ֣י תִשָּׂ֞א אֶת־רֹ֥אשׁ בְּנֵֽי־יִשְׂרָאֵל֮ לִפְקֻדֵיהֶם֒ וְנָ֨תְנ֜וּ אִ֣ישׁ כֹּ֧פֶר נַפְשׁ֛וֹ לַיהֹוָ֖ה בִּפְקֹ֣ד אֹתָ֑ם וְלֹא־יִהְיֶ֥ה בָהֶ֛ם נֶ֖גֶף בִּפְקֹ֥ד אֹתָֽם׃
\rashi{\rashiDH{כי תשא. }לשון קבלה כתרגומו, כשתחפוץ לקבל סכום מנינם לדעת כמה הם, אל תמנם לגלגולת, אלא יתנו כל אחד מחצית השקל, ותמנה את השקלים ותדע מנינם׃ 
}\rashi{\rashiDH{ולא יהיה בהם נגף. }שהמנין שולט בו עין הרע, וְהַדֶּבֶר בא עליהם, כמו שמצינו בימי דוד׃ }}
{אֲרֵי תְקַבֵּיל יָת חוּשְׁבַּן בְּנֵי יִשְׂרָאֵל לְמִנְיָנֵיהוֹן וְיִתְּנוּן גְּבַר פּוּרְקַן נַפְשֵׁיהּ קֳדָם יְיָ כַּד תִּמְנֵי יָתְהוֹן וְלָא יְהֵי בְהוֹן מוֹתָא כַּד תִּמְנֵי יָתְהוֹן׃}
{’When thou takest the sum of the children of Israel, according to their number, then shall they give every man a ransom for his soul unto the \lord, when thou numberest them; that there be no plague among them, when thou numberest them.}{\arabic{verse}}
\threeverse{\arabic{verse}}%Ex.30:13
{זֶ֣ה \legarmeh  יִתְּנ֗וּ כׇּל־הָעֹבֵר֙ עַל־הַפְּקֻדִ֔ים מַחֲצִ֥ית הַשֶּׁ֖קֶל בְּשֶׁ֣קֶל הַקֹּ֑דֶשׁ עֶשְׂרִ֤ים גֵּרָה֙ הַשֶּׁ֔קֶל מַחֲצִ֣ית הַשֶּׁ֔קֶל תְּרוּמָ֖ה לַֽיהֹוָֽה׃
\rashi{\rashiDH{זה יתנו. }הראה לו כמין מטבע של אש ומשקלה מחצית השקל, ואמר לו כזה יתנו׃ }\rashi{\rashiDH{העובר על הפקודים. }דרך המונין מעבירין את הנמנין זה אחר זה, וכן כֹּל אֲשֶׁר יַעֲבֹר תַּחַת הַשָׁבֶט (ויקרא כז, לב), וכן תַּעֲבֹרְנָה הַצֹּאן עַל יְדֵי מֹונֶה (ירמיה לג, יג)׃ }\rashi{\rashiDH{מחצית השקל בשקל הקדש. }במשקל השקל שקצבתי לך לשקול בו שקלי הקדש, כגון שקלים האמורין בפרשת ערכין ושדה אחוזה׃ }\rashi{\rashiDH{עשרים גרה השקל. }עכשיו פירש לך כמה הוא׃}\rashi{\rashiDH{גרה. }לשון מעה, וכן בשמואל (א ב, לו) יָבֹוא לְהִשְׁתַּחֲֹות לֹו לַאֲגֹורַת כֶּסֶף וְכִכַּר לָחֶם׃ }\rashi{\rashiDH{עשרים גרה השקל. }שהשקל השלם ד׳ זוזים, והזוז מתחלתו חמש מעות, אלא באו והוסיפו עליו שתות, והעלוהו לשש מעה כסף, ומחצית השקל הזה שאמרתי לך, יתנו תרומה לה׳׃ }}
{דֵּין יִתְּנוּן כָּל דְּעָבַר עַל מִנְיָנַיָּא פַּלְגוּת סִלְעָא בְּסִלְעֵי קוּדְשָׁא עֶשְׂרִין מָעִין סִלְעָא פַּלְגוּת סִלְעָא אַפְרָשׁוּתָא קֳדָם יְיָ׃}
{This they shall give, every one that passeth among them that are numbered, half a shekel after the shekel of the sanctuary—the shekel is twenty gerahs—half a shekel for an offering to the \lord.}{\arabic{verse}}
\threeverse{\aliya{לוי}}%Ex.30:14
{כֹּ֗ל הָעֹבֵר֙ עַל־הַפְּקֻדִ֔ים מִבֶּ֛ן עֶשְׂרִ֥ים שָׁנָ֖ה וָמָ֑עְלָה יִתֵּ֖ן תְּרוּמַ֥ת יְהֹוָֽה׃
\rashi{\rashiDH{מבן עשרים שנה ומעלה. }למדך כאן, שאין פחות מבן עשרים יוצא לצבא, ונמנה בכלל אנשים׃ }}
{כֹּל דְּעָבַר עַל מִנְיָנַיָּא מִבַּר עֶשְׂרִין שְׁנִין וּלְעֵילָא יִתֵּין אַפְרָשׁוּתָא קֳדָם יְיָ׃}
{Every one that passeth among them that are numbered, from twenty years old and upward, shall give the offering of the \lord.}{\arabic{verse}}
\threeverse{\arabic{verse}}%Ex.30:15
{הֶֽעָשִׁ֣יר לֹֽא־יַרְבֶּ֗ה וְהַדַּל֙ לֹ֣א יַמְעִ֔יט מִֽמַּחֲצִ֖ית הַשָּׁ֑קֶל לָתֵת֙ אֶת־תְּרוּמַ֣ת יְהֹוָ֔ה לְכַפֵּ֖ר עַל־נַפְשֹׁתֵיכֶֽם׃
\rashi{\rashiDH{לכפר על נפשותיכם. }שלא תנגפו על ידי מנין. דבר אחר לכפר על נפשותיכם, לפי שרמז להם כאן ג׳ תרומות, שנכתב כאן תרומת ה׳ ג׳ פעמים, אחת תרומת אדנים, שמנאן כשהתחילו בנדבת המשכן, ונתנו כל אחד ואחד מחצית השקל, ועלה למאת הככר, שנאמר וְכֶסֶף פְּקוּדֵי הָעֵדָה מְאַת כִּכָּר (שמות לח, כה), ומהם נעשו האדנים שנאמר וַיְהִי מְאַת כִּכָּר הַכֶּסֶף וגו׳ (שם שם, כז). והשנית אף היא על ידי מנין, שמנאן משהוקם המשכן, הוא המנין האמור בתחלת חומש הפקודים, בְּאֶחָד לַחֹדֶשׁ הַשֵׁנִי בַּשָׁנָה הַשֵׁנִית (במדבר א, א), ונתנו כל אחד מחצית השקל, והן לקנות מהן קרבנות צבור של כל שנה ושנה, והשוו בהם עניים ועשירים, ועל אותה תרומה נאמר לכפר על נפשותיכם, שהקרבנות לכפרה הם באים. והשלישית היא תרומת המשכן, כמו שנאמר כָּל מֵרִים תְּרוּמַת כֶּסֶף וּנְחשֶׁת (שמות לה, כד), ולא היתה יד כלם שוה בה, אלא איש איש מה שנדבו לבו׃ }}
{דְּעַתִּיר לָא יַסְגֵּי וּדְמִסְכֵּין לָא יַזְעַר מִפַּלְגוּת סִלְעָא לְמִתַּן יָת אַפְרָשׁוּתָא קֳדָם יְיָ לְכַפָּרָא עַל נַפְשָׁתְכוֹן׃}
{The rich shall not give more, and the poor shall not give less, than the half shekel, when they give the offering of the \lord, to make atonement for your souls.}{\arabic{verse}}
\threeverse{\arabic{verse}}%Ex.30:16
{וְלָקַחְתָּ֞ אֶת־כֶּ֣סֶף הַכִּפֻּרִ֗ים מֵאֵת֙ בְּנֵ֣י יִשְׂרָאֵ֔ל וְנָתַתָּ֣ אֹת֔וֹ עַל־עֲבֹדַ֖ת אֹ֣הֶל מוֹעֵ֑ד וְהָיָה֩ לִבְנֵ֨י יִשְׂרָאֵ֤ל לְזִכָּרוֹן֙ לִפְנֵ֣י יְהֹוָ֔ה לְכַפֵּ֖ר עַל־נַפְשֹׁתֵיכֶֽם׃ \petucha 
\rashi{\rashiDH{ונתת אותו על עבודת אהל מועד. }למדת, שנצטוו למנותם בתחלת נדבת המשכן אחר מעשה העגל, מפני שנכנס בהם מגפה, כמו שנאמר וַיִּגֹּף ה׳ אֶת הָעָם (שמות לב, לה), משל לצאן החביבה על בעליה שנפל בה דבר, ומשפסק, אמר לו לרועה בבקשה ממך, מנה את צאני ודע כמה נותרו בהם, להודיע שהיא חביבה עליו. ואי אפשר לומר שהמנין הזה הוא האמור בחומש הפקודים, שהרי נאמר בו בְּאֶחָד לַחֹדֶשׁ הַשֵׁנִי (במדבר א, א), והמשכן הוקם באחד לחדש הראשון, שנאמר בְּיֹום הַחֹדֶש הָרִאשֹׁון בְּאֶחָד לַחֹדֶשׁ תָּקִים וגו׳ (שמות מ, ב), ומהמנין הזה נעשו האדנים משקלים שלו, שנאמר וַיְהִי מְאַת כִּכַּר הַכֶּסֶף לָצֶקֶת וגו׳ (שם לח, כז), הא למדת, ששתים היו, אחת בתחלת נדבתן אחר יום הכפורים בשנה ראשונה, ואחת בשנה שנייה באייר משהוקם המשכן. ואם תאמר, וכי אפשר שבשניהם היו ישראל שוים, ו׳ מאות אלף וג׳ אלפים וה׳ מאות ונ׳, שהרי בכסף פקודי העדה נאמר כן, ובחומש הפקודים אף בו נאמר כן, וַיִּהיוּ כָּל הַפְּקֻדִים שֵׁשׁ מֵאֹות אֶלֶף וּשְׁלשֶׁת אֲלָפִים וַחֲמֵשׁ מֵאֹות וַחֲמִשִׁים (במדבר א, מו), והלא בשתי שנים היו, ואי אפשר שלא היו בשעת מנין הראשון בני י״ט שנה שלא נמנו, ובשנייה נעשו בני כ׳. תשובה לדבר, אצל שנות האנשים בשנה אחת נמנו, אבל למנין יציאת מצרים היו שתי שנים, לפי שליציאת מצרים מונין מניסן, כמו ששנינו במסכת ראש השנה (ב׃), ונבנה המשכן בראשונה והוקם בשנייה, שנתחדשה שנה באחד בניסן, אבל שנות האנשים מנויין למנין שנות עולם המתחילין מתשרי, נמצאו שני המנינים בשנה אחת, המנין הראשון היה בתשרי לאחר יום הכפורים, שנתרצה המקום לישראל לסלוח להם ונצטוו על המשכן, והשני באחד באייר׃ }\rashi{\rashiDH{על עבודת אהל מועד. }הן אדנים שנעשו בו׃}}
{וְתִסַּב יָת כְּסַף כִּפּוּרַיָּא מִן בְּנֵי יִשְׂרָאֵל וְתִתֵּין יָתֵיהּ עַל פּוּלְחַן מַשְׁכַּן זִמְנָא וִיהֵי לִבְנֵי יִשְׂרָאֵל לְדוּכְרָנָא קֳדָם יְיָ לְכַפָּרָא עַל נַפְשָׁתְכוֹן׃}
{And thou shalt take the atonement money from the children of Israel, and shalt appoint it for the service of the tent of meeting, that it may be a memorial for the children of Israel before the \lord, to make atonement for your souls.’}{\arabic{verse}}
\threeverse{\aliya{ישראל}}%Ex.30:17
{וַיְדַבֵּ֥ר יְהֹוָ֖ה אֶל־מֹשֶׁ֥ה לֵּאמֹֽר׃}
{וּמַלֵּיל יְיָ עִם מֹשֶׁה לְמֵימַר׃}
{And the \lord\space spoke unto Moses, saying:}{\arabic{verse}}
\threeverse{\arabic{verse}}%Ex.30:18
{וְעָשִׂ֜יתָ כִּיּ֥וֹר נְחֹ֛שֶׁת וְכַנּ֥וֹ נְחֹ֖שֶׁת לְרׇחְצָ֑ה וְנָתַתָּ֣ אֹת֗וֹ בֵּֽין־אֹ֤הֶל מוֹעֵד֙ וּבֵ֣ין הַמִּזְבֵּ֔חַ וְנָתַתָּ֥ שָׁ֖מָּה מָֽיִם׃
\rashi{\rashiDH{כיור. }כמין דּוּד גדולה, ולה דדים המריקים בפיהם מים׃ }\rashi{\rashiDH{וכנו. }כתרגומו בְּסִיסֵיהּ, מושב מתוקן לכיור׃ }\rashi{\rashiDH{לרחצה. }מוסב על הכיור׃}\rashi{\rashiDH{ובין המזבח. }מזבח העולה, שכתוב בו שהוא לפני פתח משכן אהל מועד, והיה הכיור מָשׁוּךְ קמעא, ועומד כנגד אויר שבין המזבח והמשכן ואינו מפסיק כלל בנתיים, משום שנאמר וְאֵת מִזְבַּח הָעֹלָה שָׂם פֶּתַח מִשְׁכַּן אֹהֶל מֹועֵד (שמות מ, כט), כלומר מזבח לפני אהל מועד, ואין כיור לפני אהל מועד, הא כיצד, משוך קמעא כלפי הדרום, כך שנויה בזבחים (נט.)׃ }}
{וְתַעֲבֵיד כִּיּוֹרָא דִּנְחָשָׁא וּבְסִיסֵיהּ דִּנְחָשָׁא לְקִדּוּשׁ וְתִתֵּין יָתֵיהּ בֵּין מַשְׁכַּן זִמְנָא וּבֵין מַדְבְּחָא וְתִתֵּין תַּמָּן מַיָּא׃}
{’Thou shalt also make a laver of brass, and the base thereof of brass, whereat to wash; and thou shalt put it between the tent of meeting and the altar, and thou shalt put water therein.}{\arabic{verse}}
\threeverse{\arabic{verse}}%Ex.30:19
{וְרָחֲצ֛וּ אַהֲרֹ֥ן וּבָנָ֖יו מִמֶּ֑נּוּ אֶת־יְדֵיהֶ֖ם וְאֶת־רַגְלֵיהֶֽם׃
\rashi{\rashiDH{את ידיהם ואת רגליהם. }בבת אחת היה מקדש ידיו ורגליו, וכך שנינו בזבחים (יט׃), כיצד קדוש ידים ורגלים, מניח ידו הימנית על גבי רגלו הימנית, וידו השמאלית על גבי רגלו השמאלית, ומקדשן׃ }}
{וִיקַדְּשׁוּן אַהֲרֹן וּבְנוֹהִי מִנֵּיהּ יָת יְדֵיהוֹן וְיָת רַגְלֵיהוֹן׃}
{And Aaron and his sons shall wash their hands and their feet thereat;}{\arabic{verse}}
\threeverse{\arabic{verse}}%Ex.30:20
{בְּבֹאָ֞ם אֶל־אֹ֧הֶל מוֹעֵ֛ד יִרְחֲצוּ־מַ֖יִם וְלֹ֣א יָמֻ֑תוּ א֣וֹ בְגִשְׁתָּ֤ם אֶל־הַמִּזְבֵּ֙חַ֙ לְשָׁרֵ֔ת לְהַקְטִ֥יר אִשֶּׁ֖ה לַֽיהֹוָֽה׃
\rashi{\rashiDH{בבואם אל אהל מועד. }להקטיר שחרית ובין הערבים קטרת, או להזות מדם פר כהן המשיח ושעירי עבודת אלילים׃ }\rashi{\rashiDH{ולא ימותו. }הא אם לא ירחצו ימותו, שבתורה נאמרו כללות, ומכלל לאו אתה שומע הן׃ 
}\rashi{\rashiDH{אל המזבח. }החיצון, שאין כאן ביאת אהל מועד אלא בחצר׃ }}
{בְּמֵיעַלְהוֹן לְמַשְׁכַּן זִמְנָא יְקַדְּשׁוּן מַיָּא וְלָא יְמוּתוּן אוֹ בְמִקְרַבְהוֹן לְמַדְבְּחָא לְשַׁמָּשָׁא לְאַסָּקָא קוּרְבָּנָא קֳדָם יְיָ׃}
{when they go into the tent of meeting, they shall wash with water, that they die not; or when they come near to the altar to minister, to cause an offering made by fire to smoke unto the \lord;}{\arabic{verse}}
\threeverse{\aliya{ע״כ בחול}}%Ex.30:21
{וְרָחֲצ֛וּ יְדֵיהֶ֥ם וְרַגְלֵיהֶ֖ם וְלֹ֣א יָמֻ֑תוּ וְהָיְתָ֨ה לָהֶ֧ם חׇק־עוֹלָ֛ם ל֥וֹ וּלְזַרְע֖וֹ לְדֹרֹתָֽם׃ \petucha 
\rashi{\rashiDH{ולא ימותו. }לחייב מיתה על המשמש במזבח ואינו רחוץ ידים ורגלים, שממיתה הראשונה לא שמענו אלא על הנכנס להיכל׃ }}
{וִיקַדְּשׁוּן יְדֵיהוֹן וְרַגְלֵיהוֹן וְלָא יְמוּתוּן וּתְהֵי לְהוֹן קְיָם עָלַם לֵיהּ וְלִבְנוֹהִי לְדָרֵיהוֹן׃}
{so they shall wash their hands and their feet, that they die not; and it shall be a statute for ever to them, even to him and to his seed throughout their generations.’}{\arabic{verse}}
\threeverse{\arabic{verse}}%Ex.30:22
{וַיְדַבֵּ֥ר יְהֹוָ֖ה אֶל־מֹשֶׁ֥ה לֵּאמֹֽר׃}
{וּמַלֵּיל יְיָ עִם מֹשֶׁה לְמֵימַר׃}
{Moreover the \lord\space spoke unto Moses, saying:}{\arabic{verse}}
\threeverse{\arabic{verse}}%Ex.30:23
{וְאַתָּ֣ה קַח־לְךָ֮ בְּשָׂמִ֣ים רֹאשׁ֒ מׇר־דְּרוֹר֙ חֲמֵ֣שׁ מֵא֔וֹת וְקִנְּמׇן־בֶּ֥שֶׂם מַחֲצִית֖וֹ חֲמִשִּׁ֣ים וּמָאתָ֑יִם וּקְנֵה־בֹ֖שֶׂם חֲמִשִּׁ֥ים וּמָאתָֽיִם׃
\rashi{\rashiDH{בשמים ראש. }חשובים׃}\rashi{\rashiDH{וקנמן בשם. }לפי שהקנמון קליפת עץ הוא, יש שהוא טוב ויש בו ריח טוב וטעם, ויש שאינו אלא כעץ, לכך הוצרך לומר קנמן בשם, מן הטוב׃ }\rashi{\rashiDH{מחציתו חמשים ומאתים. }מחצית הבאתו תהא חמשים ומאתים, נמצא כלו חמש מאות, כמו שיעור מר דרור, אם כן למה נאמר בו חצאין, גזרת הכתוב היא להביאו לחצאין, להרבות בו ב׳ הכרעות, שאין שוקלין עין בעין, וכך שנויה בכריתות (ה.)׃ }\rashi{\rashiDH{וקנה בושם. }קנה של בשם, לפי שיש קנים שאינן של בשם, הוצרך לומר בשם׃ }\rashi{\rashiDH{חמשים ומאתים. }סך משקל כולו׃}}
{וְאַתְּ סַב לָךְ בּוּסְמִין רֵישָׁא מֵירָא דָּכְיָא מַתְקַל חֲמֵישׁ מְאָה וְקִנְּמָן בְּשַׂם פַּלְגוּתֵיהּ מַתְקַל מָאתַן וְחַמְשִׁין וּקְנֵי בוּסְמָא מַתְקַל מָאתַן וְחַמְשִׁין׃}
{’Take thou also unto thee the chief spices, of flowing myrrh five hundred shekels, and of sweet cinnamon half so much, even two hundred and fifty, and of sweet calamus two hundred and fifty,}{\arabic{verse}}
\threeverse{\arabic{verse}}%Ex.30:24
{וְקִדָּ֕ה חֲמֵ֥שׁ מֵא֖וֹת בְּשֶׁ֣קֶל הַקֹּ֑דֶשׁ וְשֶׁ֥מֶן זַ֖יִת הִֽין׃
\rashi{\rashiDH{וקדה. }שם שורש עשב, ובלשון חכמים קציעה׃ }\rashi{\rashiDH{הין. }י״ב לוגין, ונחלקו בו חכמי ישראל, ר׳ מאיר אומר, בו שלקו את העקרין, אמר לו ר׳ יהודה, והלא לסוך את העקרין אינו סיפק, אלא שראום במים שלא יבלעו את השמן, ואחר כך הציף עליהם השמן, עד שקלט הריח וקפחו לשמן שעל העקרין׃ }}
{וּקְצִיעֲתָא מַתְקַל חֲמֵישׁ מְאָה בְּסִלְעֵי קוּדְשָׁא וּמְשַׁח זֵיתָא מְלֵי הִינָא׃}
{and of cassia five hundred, after the shekel of the sanctuary, and of olive oil a hin.}{\arabic{verse}}
\threeverse{\arabic{verse}}%Ex.30:25
{וְעָשִׂ֣יתָ אֹת֗וֹ שֶׁ֚מֶן מִשְׁחַת־קֹ֔דֶשׁ רֹ֥קַח מִרְקַ֖חַת מַעֲשֵׂ֣ה רֹקֵ֑חַ שֶׁ֥מֶן מִשְׁחַת־קֹ֖דֶשׁ יִהְיֶֽה׃
\rashi{\rashiDH{רוקח מרקחת. }רוקח שם דבר הוא, והטעם מוכיח שהוא למעלה, והרי הוא כמו רקע, רגע, ואינו כמו רֹגַע הַיָּם (ישעיה נא, טו), וכמו רֹוקַע הָאָרֶץ (שם מב, ה), שהטעם למטה, וכל דבר המעורב בחבירו, עד שזה קופח מזה או ריח או טעם, קרוי מרקחת׃ }\rashi{\rashiDH{רקח מרקחת. }רקח העשוי על ידי אומנות ותערובות׃}\rashi{\rashiDH{מעשה רוקח. }שם האומן בדבר׃}}
{וְתַעֲבֵיד יָתֵיהּ מְשַׁח רְבוּת קוּדְשָׁא בּוּסֶם מְבוּסַּם עוֹבָד בּוּסְמָנוּ מְשַׁח רְבוּת קוּדְשָׁא יְהֵי׃}
{And thou shalt make it a holy anointing oil, a perfume compounded after the art of the perfumer; it shall be a holy anointing oil.}{\arabic{verse}}
\threeverse{\arabic{verse}}%Ex.30:26
{וּמָשַׁחְתָּ֥ ב֖וֹ אֶת־אֹ֣הֶל מוֹעֵ֑ד וְאֵ֖ת אֲר֥וֹן הָעֵדֻֽת׃
\rashi{\rashiDH{ומשחת בו. }כל המשיחות כמין כ״ף יונית, חוץ משל מלכים שהן כמין נזר׃ }}
{וּתְרַבֵּי בֵּיהּ יָת מַשְׁכַּן זִמְנָא וְיָת אֲרוֹנָא דְּסָהֲדוּתָא׃}
{And thou shalt anoint therewith the tent of meeting, and the ark of the testimony,}{\arabic{verse}}
\threeverse{\arabic{verse}}%Ex.30:27
{וְאֶת־הַשֻּׁלְחָן֙ וְאֶת־כׇּל־כֵּלָ֔יו וְאֶת־הַמְּנֹרָ֖ה וְאֶת־כֵּלֶ֑יהָ וְאֵ֖ת מִזְבַּ֥ח הַקְּטֹֽרֶת׃}
{וְיָת פָּתוּרָא וְיָת כָּל מָנוֹהִי וְיָת מְנָרְתָא וְיָת מָנַהָא וְיָת מַדְבְּחָא דִּקְטֹרֶת בּוּסְמַיָּא׃}
{and the table and all the vessels thereof, and the candlestick and the vessels thereof, and the altar of incense,}{\arabic{verse}}
\threeverse{\arabic{verse}}%Ex.30:28
{וְאֶת־מִזְבַּ֥ח הָעֹלָ֖ה וְאֶת־כׇּל־כֵּלָ֑יו וְאֶת־הַכִּיֹּ֖ר וְאֶת־כַּנּֽוֹ׃}
{וְיָת מַדְבְּחָא דַּעֲלָתָא וְיָת כָּל מָנוֹהִי וְיָת כִּיּוֹרָא וְיָת בְּסִיסֵיהּ׃}
{and the altar of burnt-offering with all the vessels thereof, and the laver and the base thereof.}{\arabic{verse}}
\threeverse{\arabic{verse}}%Ex.30:29
{וְקִדַּשְׁתָּ֣ אֹתָ֔ם וְהָי֖וּ קֹ֣דֶשׁ קׇֽדָשִׁ֑ים כׇּל־הַנֹּגֵ֥עַ בָּהֶ֖ם יִקְדָּֽשׁ׃
\rashi{\rashiDH{וקדשת אותם. }משיחה זו מקדשתם להיות קדש קדשים, ומה היא קדושתם, כל הנוגע וגו׳, כל הראוי לכלי שרת, משנכנס לתוכו קדוש קדושת הגוף, להפסל ביוצא, ולינה, וטבול יום, ואינו נפדה לצאת לחולין, אבל דבר שאינו ראוי להם אין מקדשין. ושנויה היא משנה שלימה אצל מזבח, מתוך שנאמר כֹּל הַנֹּגֵעַ בַּמִּזְבֵּחַ יִקְדָּשׁ (שמות כט, לז), שומע אני בין ראוי בין שאינו ראוי, תלמוד לומר כבשים, מה כבשים ראויים אף כל ראויים. כל משיחת משכן וכהנים ומלכים מתורגם לשון רבוי, לפי שאין צורך משיחתן אלא לגדולה, כי כן יסד המלך שזה חנוך גדולתן, ושאר משיחות, כמו רקיקין משוחין וְרֵאשִׁית שְׁמָנִים יִמְשָׁחוּ (עמוס ו, ו), לשון ארמית בהן כלשון עברית׃ }}
{וּתְקַדֵּישׁ יָתְהוֹן וִיהוֹן קֹדֶשׁ קוּדְשִׁין כָּל דְּיִקְרַב בְּהוֹן יִתְקַדַּשׁ׃}
{And thou shalt sanctify them, that they may be most holy; whatsoever toucheth them shall be holy.}{\arabic{verse}}
\threeverse{\arabic{verse}}%Ex.30:30
{וְאֶת־אַהֲרֹ֥ן וְאֶת־בָּנָ֖יו תִּמְשָׁ֑ח וְקִדַּשְׁתָּ֥ אֹתָ֖ם לְכַהֵ֥ן לִֽי׃}
{וְיָת אַהֲרֹן וְיָת בְּנוֹהִי תְּרַבֵּי וּתְקַדֵּישׁ יָתְהוֹן לְשַׁמָּשָׁא קֳדָמָי׃}
{And thou shalt anoint Aaron and his sons, and sanctify them, that they may minister unto Me in the priest’s office.}{\arabic{verse}}
\threeverse{\arabic{verse}}%Ex.30:31
{וְאֶל־בְּנֵ֥י יִשְׂרָאֵ֖ל תְּדַבֵּ֣ר לֵאמֹ֑ר שֶׁ֠מֶן מִשְׁחַת־קֹ֨דֶשׁ יִהְיֶ֥ה זֶ֛ה לִ֖י לְדֹרֹתֵיכֶֽם׃
\rashi{\rashiDH{לדרתיכם. }מכאן למדו רבותינו לומר שכולו קיים לעתיד לבא (הוריות יא׃)׃}\rashi{\rashiDH{זה. }בגימטריא תריסר לוגין הוו׃}}
{וְעִם בְּנֵי יִשְׂרָאֵל תְּמַלֵּיל לְמֵימַר מְשַׁח רְבוּת קוּדְשָׁא יְהֵי דֵין קֳדָמַי לְדָרֵיכוֹן׃}
{And thou shalt speak unto the children of Israel, saying: This shall be a holy anointing oil unto Me throughout your generations.}{\arabic{verse}}
\threeverse{\arabic{verse}}%Ex.30:32
{עַל־בְּשַׂ֤ר אָדָם֙ לֹ֣א יִיסָ֔ךְ וּ֨בְמַתְכֻּנְתּ֔וֹ לֹ֥א תַעֲשׂ֖וּ כָּמֹ֑הוּ קֹ֣דֶשׁ ה֔וּא קֹ֖דֶשׁ יִהְיֶ֥ה לָכֶֽם׃
\rashi{\rashiDH{לא ייסך. }בשני יודי״ן, לשון לא יפעל, כמו לְמַעַן יִיטַב לָךְ (דברים ו, יח)׃ }\rashi{\rashiDH{על בשר אדם לא ייסך. }מן השמן הזה עצמו׃}\rashi{\rashiDH{ובמתכנתו לא תעשו כמוהו. }בסכום סממניו, לא תעשו אחר כמוהו במשקל סממנין הללו, לפי מדת הין שמן, אבל אם פחת או רבה סממנין לפי מדת הין שמן, מותר, ואף העשוי במתכונתו של זה, אין הסך ממנו חייב, אלא הרוקחו (כריתות ה.)׃ }\rashi{\rashiDH{ובמתכנתו. }לשון חשבון, כמו מתכנת הלבנים, וכן במתכונתה, של קטורת׃ }}
{עַל בִּשְׂרָא דַּאֲנָשָׁא לָא יִתַּסַּךְ וּבִדְמוּתֵיהּ לָא תַעְבְּדוּן כְּוָתֵיהּ קוּדְשָׁא הוּא קוּדְשָׁא יְהֵי לְכוֹן׃}
{Upon the flesh of man shall it not be poured, neither shall ye make any like it, according to the composition thereof; it is holy, and it shall be holy unto you.}{\arabic{verse}}
\threeverse{\arabic{verse}}%Ex.30:33
{אִ֚ישׁ אֲשֶׁ֣ר יִרְקַ֣ח כָּמֹ֔הוּ וַאֲשֶׁ֥ר יִתֵּ֛ן מִמֶּ֖נּוּ עַל־זָ֑ר וְנִכְרַ֖ת מֵעַמָּֽיו׃ \setuma         
\rashi{\rashiDH{ואשר יתן ממנו. }מאותו של משה׃}\rashi{\rashiDH{על זר. }שאינו צורך כהונה ומלכות׃}}
{גְּבַר דִּיבַסֵּים דִּכְוָתֵיהּ וּדְיִתֵּין מִנֵּיהּ עַל חִילוֹנַי וְיִשְׁתֵּיצֵי מֵעַמֵּיהּ׃}
{Whosoever compoundeth any like it, or whosoever putteth any of it upon a stranger, he shall be cut off from his people.’}{\arabic{verse}}
\threeverse{\arabic{verse}}%Ex.30:34
{וַיֹּ֩אמֶר֩ יְהֹוָ֨ה אֶל־מֹשֶׁ֜ה קַח־לְךָ֣ סַמִּ֗ים נָטָ֤ף \pasek  וּשְׁחֵ֙לֶת֙ וְחֶלְבְּנָ֔ה סַמִּ֖ים וּלְבֹנָ֣ה זַכָּ֑ה בַּ֥ד בְּבַ֖ד יִהְיֶֽה׃
\rashi{\rashiDH{נטף. }הוא צרי, ועל שאינו אלא שרף הנוטף מעצי הקטף קרוי נטף, ובלע״ז גומ״א (גוממיא), והצרי קורין לו טרי״אקה (טהעריאק)׃ }\rashi{\rashiDH{ושחלת. }שורש בשם, חלק ומצהיר כצפורן, ובלשון המשנה קרוי צפורן, וזהו שתרגם אונקלוס וְטוּפְרָא׃ }\rashi{\rashiDH{וחלבנה. }בשם שריחו רע, וקורין לו גלב״נא (גאלבאן) ומנאה הכתוב בין סממני הקטורת, ללמדנו שלא יקל בעינינו לצרף עמנו באגודת תעניותינו ותפלתנו את פושעי ישראל שיהיו נמנין עמנו׃ }\rashi{\rashiDH{סמים. }אחרים׃}\rashi{\rashiDH{ולבונה זכה. }מכאן למדו רבותינו (כריתות ו׃) י״א סממנין נאמרו לו למשה בסיני, מעוט סמים שנים, נטף ושחלת וחלבנה ג׳, הרי ה׳, סמים, לרבות עוד כמו אלו, הרי עשר, ולבונה, הרי י״א. ואלו הן, הצרי, והצפורן, החלבנה, והלבונה, מור, וקציעה, שבולת נרד, וכרכום, הרי ח׳, שהשבולת ונרד אחד שהנרד דומה לשבולת, הקושט, והקילופה, והקנמון, הרי י״א. בורית כרשינה אינו נקטר, אלא בו שפין את הצפורן ללבנה שתהא נאה׃ }\rashi{\rashiDH{בד בבד יהיה. }אלו הארבעה הנזכרים כאן יהיו שוין משקל במשקל, כמשקלו של זה כך משקלו של זה, וכן שנינו, הצרי והצפורן והחלבנה והלבונה משקל שבעים שבעים מנה. ולשון בד, נראה בעיני שהוא לשון יחיד, אחד באחד יהיה, זה כמו זה׃ }}
{וַאֲמַר יְיָ לְמֹשֶׁה סַב לָךְ בּוּסְמִין נָטוֹפָא וְטוּפְרָא וְחֶלְבּוֹנְתָא בּוּסְמִין וּלְבוֹנְתָא דָּכִיתָא מַתְקַל בְּמַתְקַל יְהֵי׃}
{And the \lord\space said unto Moses: ‘Take unto thee sweet spices, stacte, and onycha, and galbanum; sweet spices with pure frankincense; of each shall there be a like weight.}{\arabic{verse}}
\threeverse{\arabic{verse}}%Ex.30:35
{וְעָשִׂ֤יתָ אֹתָהּ֙ קְטֹ֔רֶת רֹ֖קַח מַעֲשֵׂ֣ה רוֹקֵ֑חַ מְמֻלָּ֖ח טָה֥וֹר קֹֽדֶשׁ׃
\rashi{\rashiDH{ממלח. }כתרגומו מעורב, שיערב שחיקתן יפה יפה זה עם זה, ואומר אני שדומה לו וַיִּרְאוּ הַמַּלָּחִים (יונה א, ה), מַלָּחַיִך וְחֹבְלַיִךְ (יחזקאל כז, כז), על שם שמהפכין את המים במשוטות כשמנהיגים את הספינה, כאדם המהפך בכף ביצים טרופות לערבן עם המים, וכל דבר שאדם רוצה לערב יפה יפה, מהפכו באצבע או בבזך׃ }\rashi{\rashiDH{ממולח טהור קדש. }ממולח יהיה, וטהור יהיה, וקדש יהיה׃ }}
{וְתַעֲבֵיד יָתַהּ קְטֹרֶת בּוּסְמִין בּוּסֶם עוֹבָד בּוּסְמָנוּ מְעָרַב דְּכֵי לְקוּדְשָׁא׃}
{And thou shalt make of it incense, a perfume after the art of the perfumer, seasoned with salt, pure and holy.}{\arabic{verse}}
\threeverse{\arabic{verse}}%Ex.30:36
{וְשָֽׁחַקְתָּ֣ מִמֶּ֘נָּה֮ הָדֵק֒ וְנָתַתָּ֨ה מִמֶּ֜נָּה לִפְנֵ֤י הָעֵדֻת֙ בְּאֹ֣הֶל מוֹעֵ֔ד אֲשֶׁ֛ר אִוָּעֵ֥ד לְךָ֖ שָׁ֑מָּה קֹ֥דֶשׁ קׇֽדָשִׁ֖ים תִּהְיֶ֥ה לָכֶֽם׃
\rashi{\rashiDH{ונתתה ממנה וגו׳. }היא קטרת שבכל יום ויום שעל מזבח הפנימי שהוא באהל מועד׃}\rashi{\rashiDH{אשר אועד לך שמה. }כל מועדי דבור שאקבע לך, אני קובעם לאותו מקום׃ }}
{וְתִשְׁחוֹק מִנַּהּ וְתַדֵּיק וְתִתֵּין מִנַּהּ קֳדָם סָהֲדוּתָא בְּמַשְׁכַּן זִמְנָא דַּאֲזָמֵין מֵימְרִי לָךְ תַּמָּן קֹדֶשׁ קוּדְשִׁין תְּהֵי לְכוֹן׃}
{And thou shalt beat some of it very small, and put of it before the testimony in the tent of meeting, where I will meet with thee; it shall be unto you most holy. .}{\arabic{verse}}
\threeverse{\arabic{verse}}%Ex.30:37
{וְהַקְּטֹ֙רֶת֙ אֲשֶׁ֣ר תַּעֲשֶׂ֔ה בְּמַ֨תְכֻּנְתָּ֔הּ לֹ֥א תַעֲשׂ֖וּ לָכֶ֑ם קֹ֛דֶשׁ תִּהְיֶ֥ה לְךָ֖ לַיהֹוָֽה׃
\rashi{\rashiDH{במתכנתה. }במנין סממניה׃}\rashi{\rashiDH{קדש תהיה לך לה׳. }שלא תעשנה אלא לשמי׃ 
}}
{וּקְטֹרֶת בּוּסְמִין דְּתַעֲבֵיד בִּדְמוּתַהּ לָא תַעְבְּדוּן לְכוֹן קוּדְשָׁא תְּהֵי לָךְ קֳדָם יְיָ׃}
{And the incense which thou shalt make, according to the composition thereof ye shall not make for yourselves; it shall be unto thee holy for the \lord.}{\arabic{verse}}
\threeverse{\arabic{verse}}%Ex.30:38
{אִ֛ישׁ אֲשֶׁר־יַעֲשֶׂ֥ה כָמ֖וֹהָ לְהָרִ֣יחַ בָּ֑הּ וְנִכְרַ֖ת מֵעַמָּֽיו׃ \setuma         
\rashi{\rashiDH{להריח בה. }אבל עושה אתה במתכנתה משלך כדי למכרה לצבור׃}}
{גְּבַר דְּיַעֲבֵיד דִּכְוָתַהּ לְאָרָחָא בַהּ וְיִשְׁתֵּיצֵי מֵעַמֵּיהּ׃}
{Whosoever shall make like unto that, to smell thereof, he shall be cut off from his people.’}{\arabic{verse}}
\newperek
\threeverse{\Roman{chap}}%Ex.31:1
{וַיְדַבֵּ֥ר יְהֹוָ֖ה אֶל־מֹשֶׁ֥ה לֵּאמֹֽר׃}
{וּמַלֵּיל יְיָ עִם מֹשֶׁה לְמֵימַר׃}
{And the \lord\space spoke unto Moses, saying:}{\Roman{chap}}
\threeverse{\arabic{verse}}%Ex.31:2
{רְאֵ֖ה קָרָ֣אתִֽי בְשֵׁ֑ם בְּצַלְאֵ֛ל בֶּן־אוּרִ֥י בֶן־ח֖וּר לְמַטֵּ֥ה יְהוּדָֽה׃
\rashi{\rashiDH{קראתי בשם. }לעשות מלאכתי, את בצלאל׃ }}
{חֲזִי דְּרַבִּיתִי בְשׁוֹם בְּצַלְאֵל בַּר אוּרִי בַר חוּר לְשִׁבְטָא דִּיהוּדָה׃}
{’See, I have called by name Bezalel the son of Uri, the son of Hur, of the tribe of Judah;}{\arabic{verse}}
\threeverse{\arabic{verse}}%Ex.31:3
{וָאֲמַלֵּ֥א אֹת֖וֹ ר֣וּחַ אֱלֹהִ֑ים בְּחׇכְמָ֛ה וּבִתְבוּנָ֥ה וּבְדַ֖עַת וּבְכׇל־מְלָאכָֽה׃
\rashi{\rashiDH{בחכמה. }מה שאדם שומע דברים מאחרים ולמד׃}\rashi{\rashiDH{ובתבונה. }מבין דבר מלבו מתוך דברים שלמד׃}\rashi{\rashiDH{ובדעת. }רוח הקדש׃}}
{וְאַשְׁלֵימִית עִמֵּיהּ רוּחַ מִן קֳדָם יְיָ בְּחָכְמָה וּבְסוּכְלְתָנוּ וּבְמַדַּע וּבְכָל עֲבִידָא׃}
{and I have filled him with the spirit of God, in wisdom, and in understanding, and in knowledge, and in all manner of workmanship,}{\arabic{verse}}
\threeverse{\arabic{verse}}%Ex.31:4
{לַחְשֹׁ֖ב מַחֲשָׁבֹ֑ת לַעֲשׂ֛וֹת בַּזָּהָ֥ב וּבַכֶּ֖סֶף וּבַנְּחֹֽשֶׁת׃
\rashi{\rashiDH{לחשוב מחשבות. }אריגת מעשה חשב׃}}
{לְאַלָּפָא אוּמָּנְוָן לְמֶעֱבַד בְּדַהְבָּא וּבְכַסְפָּא וּבִנְחָשָׁא׃}
{to devise skilful works, to work in gold, and in silver, and in brass,}{\arabic{verse}}
\threeverse{\arabic{verse}}%Ex.31:5
{וּבַחֲרֹ֥שֶׁת אֶ֛בֶן לְמַלֹּ֖את וּבַחֲרֹ֣שֶׁת עֵ֑ץ לַעֲשׂ֖וֹת בְּכׇל־מְלָאכָֽה׃
\rashi{\rashiDH{ובחרשת. }לשון אומנות, כמו חָרָשׁ חָכָם (ישעיה מ, כ). ואונקלוס פירש, ושנה בפירושן, שאומן אבנים קרוי אומן, וחרש עץ קרוי נגר׃ }\rashi{\rashiDH{למלאת. }להושיבה במשבצות שלה במלואה, לעשות המשבצת למדת מושב האבן ועוביה׃ }}
{וּבְאוּמָּנוּת אֶבֶן טָבָא לְאַשְׁלָמָא וּבְנַגָּרוּת אָעָא לְמֶעֱבַד בְּכָל עֲבִידָא׃}
{and in cutting of stones for setting, and in carving of wood, to work in all manner of workmanship.}{\arabic{verse}}
\threeverse{\arabic{verse}}%Ex.31:6
{וַאֲנִ֞י הִנֵּ֧ה נָתַ֣תִּי אִתּ֗וֹ אֵ֣ת אׇהֳלִיאָ֞ב בֶּן־אֲחִֽיסָמָךְ֙ לְמַטֵּה־דָ֔ן וּבְלֵ֥ב כׇּל־חֲכַם־לֵ֖ב נָתַ֣תִּי חׇכְמָ֑ה וְעָשׂ֕וּ אֵ֖ת כׇּל־אֲשֶׁ֥ר צִוִּיתִֽךָ׃
\rashi{\rashiDH{ובלב כל חכם לב וגו׳. }ועוד שאר חכמי לב שבכם, וכל אשר נתתי בו חכמה, ועשו את כל אשר צויתיך׃ }}
{וַאֲנָא הָא יְהַבִית עִמֵּיהּ יָת אָהֳלִיאָב בַּר אֲחִיסָמָךְ לְשִׁבְטָא דְּדָן וּבְלֵב כָּל חַכִּימֵי לִבָּא יְהַבִית חָכְמְתָא וְיַעְבְּדוּן יָת כָּל דְּפַקֵּידְתָּךְ׃}
{And I, behold, I have appointed with him Oholiab, the son of Ahisamach, of the tribe of Dan; and in the hearts of all that are wise-hearted I have put wisdom, that they may make all that I have commanded thee:}{\arabic{verse}}
\threeverse{\arabic{verse}}%Ex.31:7
{אֵ֣ת \legarmeh  אֹ֣הֶל מוֹעֵ֗ד וְאֶת־הָֽאָרֹן֙ לָֽעֵדֻ֔ת וְאֶת־הַכַּפֹּ֖רֶת אֲשֶׁ֣ר עָלָ֑יו וְאֵ֖ת כׇּל־כְּלֵ֥י הָאֹֽהֶל׃
\rashi{\rashiDH{ואת הארון לעדות. }לצורך לוחות העדות׃}}
{יָת מַשְׁכַּן זִמְנָא וְיָת אֲרוֹנָא לְסָהֲדוּתָא וְיָת כָּפוּרְתָּא דַּעֲלוֹהִי וְיָת כָּל מָנֵי מַשְׁכְּנָא׃}
{the tent of meeting, and the ark of the testimony, and the ark-cover that is thereupon, and all the furniture of the Tent;}{\arabic{verse}}
\threeverse{\arabic{verse}}%Ex.31:8
{וְאֶת־הַשֻּׁלְחָן֙ וְאֶת־כֵּלָ֔יו וְאֶת־הַמְּנֹרָ֥ה הַטְּהֹרָ֖ה וְאֶת־כׇּל־כֵּלֶ֑יהָ וְאֵ֖ת מִזְבַּ֥ח הַקְּטֹֽרֶת׃
\rashi{\rashiDH{הטהורה. }על שם זהב טהור׃}}
{וְיָת פָּתוּרָא וְיָת מָנוֹהִי וְיָת מְנָרְתָא דָּכִיתָא וְיָת כָּל מָנַהָא וְיָת מַדְבְּחָא דִּקְטֹרֶת בּוּסְמַיָּא׃}
{and the table and its vessels, and the pure candlestick with all its vessels, and the altar of incense;}{\arabic{verse}}
\threeverse{\arabic{verse}}%Ex.31:9
{וְאֶת־מִזְבַּ֥ח הָעֹלָ֖ה וְאֶת־כׇּל־כֵּלָ֑יו וְאֶת־הַכִּיּ֖וֹר וְאֶת־כַּנּֽוֹ׃}
{וְיָת מַדְבְּחָא דַּעֲלָתָא וְיָת כָּל מָנוֹהִי וְיָת כִּיּוֹרָא וְיָת בְּסִיסֵיהּ׃}
{and the altar of burnt-offering with all its vessels, and the laver and its base;}{\arabic{verse}}
\threeverse{\arabic{verse}}%Ex.31:10
{וְאֵ֖ת בִּגְדֵ֣י הַשְּׂרָ֑ד וְאֶת־בִּגְדֵ֤י הַקֹּ֙דֶשׁ֙ לְאַהֲרֹ֣ן הַכֹּהֵ֔ן וְאֶת־בִּגְדֵ֥י בָנָ֖יו לְכַהֵֽן׃
\rashi{\rashiDH{ואת בגדי השרד. }אומר אני לפי פשוטו של מקרא, שאי אפשר לומר שבבגדי כהונה מדבר, לפי שנאמר אצלם ואת בגדי הקדש לאהרן הכהן ואת בגדי בניו לכהן, אלא אלו בגדי השרד הם, בגדי התכלת והארגמן ותולעת שני האמורים בפרשת מסעות, וְנָתְנוּ עָלָיו בֶּגֶד תְּכֵלֶת (במדבר ד, יב), וְנָתְנוּ עָלָיו בֶּגֶד אַרְגָּמָן (שם יג), וְנָתְנוּ עֲלֵיהֶם בֶּגֶד תֹּולַעַת שָׁנִי (שם ח), ונראין דברי, שנאמר וּמִן הַתְּכֵלֶת וְהָאַרְגָּמָן וְתֹולַעַת הַשָּׁנִי עָשׂוּ בִגְדֵי שְׂרָד לְשָׁרֵת בַּקֹּדֶשׁ (שמות לט, א), ולא הוזכר שש עמהם, ואם בבגדי כהונה מדבר, לא מצינו באחד מהם ארגמן או תולעת שני בלא שש׃ }\rashi{\rashiDH{בגדי השרד. }יש מפרשים לשון עבודה ושירות, כתרגומו לְבוּשֵׁי שִׁמּוּשָׁא, ואין לו דמיון במקרא, ואני אומר שהוא לשון ארמי, כתרגום של קלעים ותרגום של מכבר, שהיו ארוגים במחט, עשויים נקבים נקבים, לצי״דין בלע״ז (שלינגווערק)׃ }}
{וְיָת לְבוּשֵׁי שִׁמּוּשָׁא וְיָת לְבוּשֵׁי קוּדְשָׁא לְאַהֲרֹן כָּהֲנָא וְיָת לְבוּשֵׁי בְנוֹהִי לְשַׁמָּשָׁא׃}
{and the plaited garments, and the holy garments for Aaron the priest, and the garments of his sons, to minister in the priest’s office;}{\arabic{verse}}
\threeverse{\arabic{verse}}%Ex.31:11
{וְאֵ֨ת שֶׁ֧מֶן הַמִּשְׁחָ֛ה וְאֶת־קְטֹ֥רֶת הַסַּמִּ֖ים לַקֹּ֑דֶשׁ כְּכֹ֥ל אֲשֶׁר־צִוִּיתִ֖ךָ יַעֲשֽׂוּ׃ \petucha 
\rashi{\rashiDH{ואת קטורת הסמים לקדש. }לצורך הקטרת ההיכל שהוא קדש׃ 
}}
{וְיָת מִשְׁחָא דִּרְבוּתָא וְיָת קְטֹרֶת בּוּסְמַיָּא לְקוּדְשָׁא כְּכֹל דְּפַקֵּידְתָּךְ יַעְבְּדוּן׃}
{and the anointing oil, and the incense of sweet spices for the holy place; according to all that I have commanded thee shall they do.’}{\arabic{verse}}
\threeverse{\arabic{verse}}%Ex.31:12
{וַיֹּ֥אמֶר יְהֹוָ֖ה אֶל־מֹשֶׁ֥ה לֵּאמֹֽר׃}
{וַאֲמַר יְיָ לְמֹשֶׁה לְמֵימַר׃}
{And the \lord\space spoke unto Moses, saying:}{\arabic{verse}}
\threeverse{\arabic{verse}}%Ex.31:13
{וְאַתָּ֞ה דַּבֵּ֨ר אֶל־בְּנֵ֤י יִשְׂרָאֵל֙ לֵאמֹ֔ר אַ֥ךְ אֶת־שַׁבְּתֹתַ֖י תִּשְׁמֹ֑רוּ כִּי֩ א֨וֹת הִ֜וא בֵּינִ֤י וּבֵֽינֵיכֶם֙ לְדֹרֹ֣תֵיכֶ֔ם לָדַ֕עַת כִּ֛י אֲנִ֥י יְהֹוָ֖ה מְקַדִּשְׁכֶֽם׃
\rashi{\rashiDH{ואתה דבר אל בני ישראל. }ואתה, אף על פי שהפקדתיך לצוותם על מלאכת המשכן, אל יקל בעיניך לדחות את השבת מפני אותה מלאכה׃ }\rashi{\rashiDH{אך את שבתותי תשמורו. }אף על פי שתהיו רדופין וזריזין בזריזות המלאכה, שבת אל תדחה מפניה. כל אכין ורקין מעוטין, למעט שבת ממלאכת המשכן׃ }\rashi{\rashiDH{כי אות היא ביני וביניכם. }אות גדולה היא בינינו שבחרתי בכם, בהנחילי לכם את יום מנוחתי למנוחה׃ }\rashi{\rashiDH{לדעת. }האומות בה, כי אני ה׳ מקדשכם׃ }}
{וְאַתְּ מַלֵּיל עִם בְּנֵי יִשְׂרָאֵל לְמֵימַר בְּרַם יָת יוֹמֵי שַׁבַּיָּא דִּילִי תִּטְּרוּן אֲרֵי אָת הִיא בֵּין מֵימְרִי וּבֵינֵיכוֹן לְדָרֵיכוֹן לְמִדַּע אֲרֵי אֲנָא יְיָ מְקַדִּשְׁכוֹן׃}
{’Speak thou also unto the children of Israel, saying: Verily ye shall keep My sabbaths, for it is a sign between Me and you throughout your generations, that ye may know that I am the \lord\space who sanctify you.}{\arabic{verse}}
\threeverse{\arabic{verse}}%Ex.31:14
{וּשְׁמַרְתֶּם֙ אֶת־הַשַּׁבָּ֔ת כִּ֛י קֹ֥דֶשׁ הִ֖וא לָכֶ֑ם מְחַֽלְלֶ֙יהָ֙ מ֣וֹת יוּמָ֔ת כִּ֗י כׇּל־הָעֹשֶׂ֥ה בָהּ֙ מְלָאכָ֔ה וְנִכְרְתָ֛ה הַנֶּ֥פֶשׁ הַהִ֖וא מִקֶּ֥רֶב עַמֶּֽיהָ׃
\rashi{\rashiDH{מות יומת. }אם יש עדים והתראה׃}\rashi{\rashiDH{ונכרתה. }בלא התראה׃}\rashi{\rashiDH{מחלליה. }הנוהג בה חול בקדושתה׃}}
{וְתִטְּרוּן יָת שַׁבְּתָא אֲרֵי קוּדְשָׁא הִיא לְכוֹן דְּיַחֲלִנַּהּ אִתְקְטָלָא יִתְקְטִיל אֲרֵי כָל דְּיַעֲבֵיד בַּהּ עֲבִידְתָא וְיִשְׁתֵּיצֵי אֲנָשָׁא הַהוּא מִגּוֹ עַמֵּיהּ׃}
{Ye shall keep the sabbath therefore, for it is holy unto you; every one that profaneth it shall surely be put to death; for whosoever doeth any work therein, that soul shall be cut off from among his people.}{\arabic{verse}}
\threeverse{\arabic{verse}}%Ex.31:15
{שֵׁ֣שֶׁת יָמִים֮ יֵעָשֶׂ֣ה מְלָאכָה֒ וּבַיּ֣וֹם הַשְּׁבִיעִ֗י שַׁבַּ֧ת שַׁבָּת֛וֹן קֹ֖דֶשׁ לַיהֹוָ֑ה כׇּל־הָעֹשֶׂ֧ה מְלָאכָ֛ה בְּי֥וֹם הַשַּׁבָּ֖ת מ֥וֹת יוּמָֽת׃
\rashi{\rashiDH{שבת שבתון. }מנוחת מרגוע ולא מנוחת עראי׃ (\rashiDH{שבת שבתון. }לכך כפלו הכתוב, לומר שאסור בכל מלאכה, אפילו אוכל נפש, וכן יום הכפורים שנאמר בו שַׁבַּת שַׁבָּתֹון הוּא לָכֶם (ויקרא כג, לב), אסור בכל מלאכה, אבל יום טוב לא נאמר בו כי אם בַּיֹּום הָרִאשֹׁון שַׁבָּתֹון וּבַיֹּום הַשְׁמִינִי שַׁבָּתֹון (שם, לט), אסורים בכל מלאכת עבודה, ומותרים במלאכת אוכל נפש׃) 
}\rashi{\rashiDH{קדש לה׳. }שמירת קדושתה לשמי ובמצותי׃}}
{שִׁתָּא יוֹמִין תִּתְעֲבֵיד עֲבִידְתָא וּבְיוֹמָא שְׁבִיעָאָה שַׁבָּא שַׁבָּתָא קוּדְשָׁא קֳדָם יְיָ כָּל דְּיַעֲבֵיד עֲבִידְתָא בְּיוֹמָא דְּשַׁבְּתָא אִתְקְטָלָא יִתְקְטִיל׃}
{Six days shall work be done; but on the seventh day is a sabbath of solemn rest, holy to the \lord; whosoever doeth any work in the sabbath day, he shall surely be put to death.}{\arabic{verse}}
\threeverse{\arabic{verse}}%Ex.31:16
{וְשָׁמְר֥וּ בְנֵֽי־יִשְׂרָאֵ֖ל אֶת־הַשַּׁבָּ֑ת לַעֲשׂ֧וֹת אֶת־הַשַּׁבָּ֛ת לְדֹרֹתָ֖ם בְּרִ֥ית עוֹלָֽם׃}
{וְיִטְּרוּן בְּנֵי יִשְׂרָאֵל יָת שַׁבְּתָא לְמֶעֱבַד יָת שַׁבְּתָא לְדָרֵיהוֹן קְיָם עָלַם׃}
{Wherefore the children of Israel shall keep the sabbath, to observe the sabbath throughout their generations, for a perpetual covenant.}{\arabic{verse}}
\threeverse{\arabic{verse}}%Ex.31:17
{בֵּינִ֗י וּבֵין֙ בְּנֵ֣י יִשְׂרָאֵ֔ל א֥וֹת הִ֖וא לְעֹלָ֑ם כִּי־שֵׁ֣שֶׁת יָמִ֗ים עָשָׂ֤ה יְהֹוָה֙ אֶת־הַשָּׁמַ֣יִם וְאֶת־הָאָ֔רֶץ וּבַיּוֹם֙ הַשְּׁבִיעִ֔י שָׁבַ֖ת וַיִּנָּפַֽשׁ׃ \setuma         
\rashi{\rashiDH{וינפש. }כתרגומו וְנַח, וכל לשון נופש הוא לשון נפש, שמשיב נפשו ונשימתו בהרגיעו מטורח המלאכה, ומי שכתוב בו לֹא יִיעַף וְלֹא יִיגָע (ישעיה מ, כח), וכל פעלו במאמר הכתיב מנוחה לעצמו, לְשַׂבֵּר האוזן מה שהיא יכולה לשמוע׃ 
}}
{בֵּין מֵימְרִי וּבֵין בְּנֵי יִשְׂרָאֵל אָת הִיא לְעָלַם אֲרֵי שִׁתָּא יוֹמִין עֲבַד יְיָ יָת שְׁמַיָּא וְיָת אַרְעָא וּבְיוֹמָא שְׁבִיעָאָה שְׁבָת וְנָח׃}
{It is a sign between Me and the children of Israel for ever; for in six days the \lord\space made heaven and earth, and on the seventh day He ceased from work and rested.’}{\arabic{verse}}
\threeverse{\aliya{שני}}%Ex.31:18
{וַיִּתֵּ֣ן אֶל־מֹשֶׁ֗ה כְּכַלֹּתוֹ֙ לְדַבֵּ֤ר אִתּוֹ֙ בְּהַ֣ר סִינַ֔י שְׁנֵ֖י לֻחֹ֣ת הָעֵדֻ֑ת לֻחֹ֣ת אֶ֔בֶן כְּתֻבִ֖ים בְּאֶצְבַּ֥ע אֱלֹהִֽים׃
\rashi{\rashiDH{ויתן אל משה וגו׳. }אין מוקדם ומאוחר בתורה, מעשה העגל קודם לצווי מלאכת המשכן ימים רבים היה, שהרי בי״ז בתמוז נשתברו הלוחות, וביום הכפורים נתרצה הקב״ה לישראל, ולמחרת התחילו בנדבת המשכן והוקם באחד בניסן. (צ״ע טובא, דילמא הכל כסדר, וצווי הקב״ה למשה היה בארבעים ימים הראשונים, טרם עשותם העגל, וקודם רדתו מההר עשו העגל, ומשה לא הגיד לישראל צווי המשכן עד למחרת יום הכפורים, שהיו ישראל מרוצים להקב״ה, וכן הוא בהדיא בזוהר ויקהל, אשר על כן בצווי הקב״ה כתיב מאת כל איש, דהיינו גם ערב רב, כמו שדרשו רבותינו ז״ל, איש איש, מלמד וכו׳, ומשה בציווי אמר לישראל, קחו מאתכם דייקא, ולא מערב רב, לפי שהם גרמו בנזקין וק״ל)׃ }\rashi{\rashiDH{ככלתו. }ככלתו, כתיב חסר, שנמסרה לו תורה במתנה ככלה לחתן, שלא היה יכול ללמוד כולה בזמן מועט כזה. דבר אחר, מה כלה מתקשטת בכ״ד קשוטין, הן האמורים בספר ישעיה (ג, חכד), אף תלמיד חכם צריך להיות בקי בכ״ד ספרים׃ }\rashi{\rashiDH{לדבר אתו. }החקים והמשפטים שבואלה המשפטים׃}\rashi{\rashiDH{לדבר אתו. }מלמד שהיה משה שומע מפי הגבורה, וחוזרין ושונין את ההלכה שניהם יחד׃ }\rashi{\rashiDH{לחת. }לחת כתיב, שהיו שתיהן שוות׃ 
}}
{וִיהַב לְמֹשֶׁה כַּד שֵׁיצֵי לְמַלָּלָא עִמֵּיהּ בְּטוּרָא דְּסִינַי תְּרֵין לוּחֵי סָהֲדוּתָא לוּחֵי אַבְנָא כְּתִיבִין בְּאֶצְבְּעָא דַּייָ׃}
{And He gave unto Moses, when He had made an end of speaking with him upon mount Sinai, the two tables of the testimony, tables of stone, written with the finger of God.}{\arabic{verse}}
\newperek
\threeverse{\Roman{chap}}%Ex.32:1
{וַיַּ֣רְא הָעָ֔ם כִּֽי־בֹשֵׁ֥שׁ מֹשֶׁ֖ה לָרֶ֣דֶת מִן־הָהָ֑ר וַיִּקָּהֵ֨ל הָעָ֜ם עַֽל־אַהֲרֹ֗ן וַיֹּאמְר֤וּ אֵלָיו֙ ק֣וּם \legarmeh  עֲשֵׂה־לָ֣נוּ אֱלֹהִ֗ים אֲשֶׁ֤ר יֵֽלְכוּ֙ לְפָנֵ֔ינוּ כִּי־זֶ֣ה \legarmeh  מֹשֶׁ֣ה הָאִ֗ישׁ אֲשֶׁ֤ר הֶֽעֱלָ֙נוּ֙ מֵאֶ֣רֶץ מִצְרַ֔יִם לֹ֥א יָדַ֖עְנוּ מֶה־הָ֥יָה לֽוֹ׃
\rashi{\rashiDH{כי בשש משה. }כתרגומו לשון איחור, וכן בּ שֵׁשׁ רִכְבֹּו (שופטים ה, כח), וַיָּחִילוּ עַד בֹּושׁ (שם ג, כה), כי כשעלה משה להר אמר להם, לסוף ארבעים יום אני בא בתוך ו׳ שעות, כסבורים הם שאותו יום שעלה מן המנין הוא, והוא אמר להם שלימים, ארבעים יום ולילו עמו, ויום עלייתו אין לילו עמו, שהרי בז׳ בסיון עלה, נמצא יום ארבעים בשבעה עשר בתמוז, בי״ו בא השטן וערבב את העולם, והראה דמות חשך ואפלה וערבוביא, לומר ודאי מת משה לכך בא ערבוביא לעולם, אמר להם מת משה שכבר באו שש שעות ולא בא וכו׳, כדאיתא במסכת שבת (פט.), ואי אפשר לומר שלא טעו אלא ביום המעונן, בין קודם חצות בין לאחר חצות, שהרי לא ירד משה עד יום המחרת, שנאמר וַיַּשְׁכִּימוּ מִמָּחֳרָת וַיַּעֲלוּ עֹלֹות׃ }\rashi{\rashiDH{אשר ילכו לפנינו. }אלהות הרבה איוו להם׃ 
}\rashi{\rashiDH{כי זה משה האיש. }כמין דמות משה הראה להם השטן, שנושאים אותו באויר רקיע השמים׃ }\rashi{\rashiDH{אשר העלנו מארץ מצרים. }והיה מורה לנו דרך אשר נעלה בה, עתה צריכין אנו לאלהות אשר ילכו לפנינו׃ }}
{וַחֲזָא עַמָּא אֲרֵי אוֹחַר מֹשֶׁה לְמֵיחַת מִן טוּרָא וְאִתְכְּנֵישׁ עַמָּא עַל אַהֲרֹן וַאֲמַרוּ לֵיהּ קוּם עֲבֵיד לַנָא דַּחְלָן דִּיהָכָן קֳדָמַנָא אֲרֵי דֵין מֹשֶׁה גּוּבְרָא דְּאַסְּקַנָא מֵאַרְעָא דְּמִצְרַיִם לָא יְדַעְנָא מָא הֲוָה לֵיהּ׃}
{And when the people saw that Moses delayed to come down from the mount, the people gathered themselves together unto Aaron, and said unto him: ‘Up, make us a god who shall go before us; for as for this Moses, the man that brought us up out of the land of Egypt, we know not what is become of him.’}{\Roman{chap}}
\threeverse{\arabic{verse}}%Ex.32:2
{וַיֹּ֤אמֶר אֲלֵהֶם֙ אַהֲרֹ֔ן פָּֽרְקוּ֙ נִזְמֵ֣י הַזָּהָ֔ב אֲשֶׁר֙ בְּאׇזְנֵ֣י נְשֵׁיכֶ֔ם בְּנֵיכֶ֖ם וּבְנֹתֵיכֶ֑ם וְהָבִ֖יאוּ אֵלָֽי׃
\rashi{\rashiDH{באזני נשיכם. }אמר אהרן בלבו, הנשים והילדים חסים בתכשיטיהן, שמא יתעכב הדבר, ובתוך כך יבא משה, והם לא המתינו ופרקו מעל עצמן׃ }\rashi{\rashiDH{פרקו. }לשון צווי, מגזרת פרק ליחיד, כמו ברכו מגזרת ברך׃ }}
{וַאֲמַר לְהוֹן אַהֲרֹן פָּרִיקוּ קְדָשֵׁי דְּדַהְבָּא דִּבְאוּדְנֵי נְשֵׁיכוֹן בְּנֵיכוֹן וּבְנָתְכוֹן וְאֵיתוֹ לְוָתִי׃}
{And Aaron said unto them: ‘Break off the golden rings, which are in the ears of your wives, of your sons, and of your daughters, and bring them unto me.’}{\arabic{verse}}
\threeverse{\arabic{verse}}%Ex.32:3
{וַיִּתְפָּֽרְקוּ֙ כׇּל־הָעָ֔ם אֶת־נִזְמֵ֥י הַזָּהָ֖ב אֲשֶׁ֣ר בְּאׇזְנֵיהֶ֑ם וַיָּבִ֖יאוּ אֶֽל־אַהֲרֹֽן׃
\rashi{\rashiDH{ויתפרקו. }לשון פריקת משא, כשנטלום מאזניהם נמצאו הם מפורקים מנזמיהם, דישקריי״ר בלע״ז (ענטלאזטעט)׃ }\rashi{\rashiDH{את נזמי. }כמו מנזמי, כמו כְּצֵאתִי אֶת הָעִיר (שמות ט, כט), מן העיר׃ }}
{וּפָרִיקוּ כָל עַמָּא יָת קְדָשֵׁי דְּדַהְבָּא דִּבְאוּדְנֵיהוֹן וְאֵיתִיאוּ לְוָת אַהֲרֹן׃}
{And all the people broke off the golden rings which were in their ears, and brought them unto Aaron.}{\arabic{verse}}
\threeverse{\arabic{verse}}%Ex.32:4
{וַיִּקַּ֣ח מִיָּדָ֗ם וַיָּ֤צַר אֹתוֹ֙ בַּחֶ֔רֶט וַֽיַּעֲשֵׂ֖הוּ עֵ֣גֶל מַסֵּכָ֑ה וַיֹּ֣אמְר֔וּ אֵ֤לֶּה אֱלֹהֶ֙יךָ֙ יִשְׂרָאֵ֔ל אֲשֶׁ֥ר הֶעֱל֖וּךָ מֵאֶ֥רֶץ מִצְרָֽיִם׃
\rashi{\rashiDH{ויצר אותו בחרט. }יש לתרגמו בשני פנים, האחד ויצר לשון קשירה, בחרט לשון סודר, כמו וְהַמִּטְפָּחֹות וְהָחֲרִיטִים (ישעיה ג, כב), וַיָּצַר כִּכְּרַיִם כֶּסֶף בִּשְׁנֵי חֲרִיטִים (מלכים־ב ה, כג). והב׳ ויצר לשון צורה, בחרט כלי אומנות הצורפין שחורצין וחורטין בו צורות בזהב, כעט סופר החורט אותיות בלוחות ופנקסין, כמו וּכְתֹב עָלָיו בְּחֶרֶט אֱנֹושׁ (ישעיה ח, א), וזהו שתרגם אונקלוס וְצַר יָתֵיהּ בְּזִיפָא, לשון זיוף הוא, כלי אומנות שחורצין בו בזהב אותיות ושקדים שקורין בלע״ז ניי״ל (גראבשטיכל), ומזייפין על ידו חותמות׃ }\rashi{\rashiDH{עגל מסכה. }כיון שהשליכו לָאוּר בַּכּוּר, באו מכשפי ערב רב שעלו עמהם ממצרים ועשאוהו בכשפים, ויש אומרים מיכה היה שם, שיצא מתוך דמוסי בנין שנתמעך בו במצרים (סנהדרין קא׃), והיה בידו שם וטס, שכתב בו משה עלה שור עלה שור להעלות ארונו של יוסף מתוך נילוס, והשליכו לתוך הַכּוּר, ויצא העגל׃ }\rashi{\rashiDH{מסכה. }לשון מתכת. דבר אחר, קכ״ה קנטרין זהב היו בו, כגימטריא של מסכה׃ }\rashi{\rashiDH{אלה אלהיך. }ולא נאמר אלה אלהינו, מכאן שערב רב שעלו ממצרים הם שנקהלו על אהרן והם שעשאוהו, ואחר כך הטעו את ישראל אחריו׃ 
}}
{וּנְסֵיב מִיַּדְהוֹן וְצָר יָתֵיהּ בְּזִיפָא וְעַבְדֵיהּ עֵיגֶל מַתְּכָא וַאֲמַרוּ אִלֵּין דַּחְלָתָךְ יִשְׂרָאֵל דְּאַסְּקוּךְ מֵאַרְעָא דְּמִצְרָיִם׃}
{And he received it at their hand, and fashioned it with a graving tool, and made it a molten calf; and they said: ‘This is thy god, O Israel, which brought thee up out of the land of Egypt.’}{\arabic{verse}}
\threeverse{\arabic{verse}}%Ex.32:5
{וַיַּ֣רְא אַהֲרֹ֔ן וַיִּ֥בֶן מִזְבֵּ֖חַ לְפָנָ֑יו וַיִּקְרָ֤א אַֽהֲרֹן֙ וַיֹּאמַ֔ר חַ֥ג לַיהֹוָ֖ה מָחָֽר׃
\rashi{\rashiDH{וירא אהרן. }שהיה בו רוח חיים, שנאמר בְּתַבְנִית שֹׁור אֹכֵל עֵשֶׂב (תהלים קו, כ), וראה שהצליח מעשה שטן, ולא היה לו פה לדחותם לגמרי׃ }\rashi{\rashiDH{ויבן מזבח. }לדחותם׃}\rashi{\rashiDH{ויאמר חג לה׳ מחר. }ולא היום, שמא יבא משה קודם שיעבדוהו, זהו פשוטו. ומדרשו בויקרא רבה (י, ג), דברים הרבה ראה אהרן, ראה חור בן אחותו שהיה מוכיחם והרגוהו, וזהו ויבן (לשון בינה)מזבח לפניו, ויבן מזבוח לפניו, ועוד ראה ואמר מוטב שיתלה בי הסרחון ולא בהם, ועוד ראה ואמר, אם הם בונים את המזבח, זה מביא צרור וזה מביא אבן ונמצאת מלאכתן עשויה בבת אחת, מתוך שאני בונה אותו ומתעצל במלאכתי, בין כך ובין כך משה בא׃ }\rashi{\rashiDH{חג לה׳. }בלבו היה לשמים, בטוח היה שיבא משה ויעבדו את המקום׃ }}
{וַחֲזָא אַהֲרֹן וּבְנָא מַדְבְּחָא קֳדָמוֹהִי וּקְרָא אַהֲרֹן וַאֲמַר חַגָּא קֳדָם יְיָ מְחַר׃}
{And when Aaron saw this, he built an altar before it; and Aaron made proclamation, and said: ‘To-morrow shall be a feast to the \lord.’}{\arabic{verse}}
\threeverse{\arabic{verse}}%Ex.32:6
{וַיַּשְׁכִּ֙ימוּ֙ מִֽמׇּחֳרָ֔ת וַיַּעֲל֣וּ עֹלֹ֔ת וַיַּגִּ֖שׁוּ שְׁלָמִ֑ים וַיֵּ֤שֶׁב הָעָם֙ לֶֽאֱכֹ֣ל וְשָׁת֔וֹ וַיָּקֻ֖מוּ לְצַחֵֽק׃ \petucha 
\rashi{\rashiDH{וישכימו. }השטן זרזם כדי שיחטאו׃}\rashi{\rashiDH{לצחק. }יש במשמע הזה גלוי עריות, כמו שנאמר לְצַחֶק בִּי (בראשית לט, יז), ושפיכות דמים, כמו שנאמר יָקוּמוּ נָא הַנְּעָרִים וִישַׂחֲקוּ לְפָנֵינוּ (שמואל־ב ב, יד), אף כאן נהרג חור׃ }}
{וְאַקְדִּימוּ בְּיוֹמָא דְּבָתְרוֹהִי וְאַסִּיקוּ עֲלָוָן וְקָרִיבוּ נִכְסָן וְאַסְחַר עַמָּא לְמֵיכַל וּלְמִשְׁתֵּי וְקָמוּ לְחַיָּיכָא׃}
{And they rose up early on the morrow, and offered burnt-offerings, and brought peace-offerings; and the people sat down to eat and to drink, and rose up to make merry.}{\arabic{verse}}
\threeverse{\arabic{verse}}%Ex.32:7
{וַיְדַבֵּ֥ר יְהֹוָ֖ה אֶל־מֹשֶׁ֑ה לֶךְ־רֵ֕ד כִּ֚י שִׁחֵ֣ת עַמְּךָ֔ אֲשֶׁ֥ר הֶעֱלֵ֖יתָ מֵאֶ֥רֶץ מִצְרָֽיִם׃
\rashi{\rashiDH{וידבר. }לשון קושי הוא, כמו וַיְדַבֵּר אִתָּם קָשֹׁות (בראשית מב, ז)׃ }\rashi{\rashiDH{לך רד. }רד מגדולתך, לא נתתי לך גדולה אלא בשבילם (ברכות לב.), באותה שעה נתנדה משה מפי בית דין של מעלה׃ }\rashi{\rashiDH{שחת עמך. }שחת העם לא נאמר, אלא עמך, ערב רב שקבלת מעצמך וגיירתם, ולא נמלכת בי, ואמרת טוב שידבקו גרים בשכינה, הם שחתו והשחיתו׃ }}
{וּמַלֵּיל יְיָ עִם מֹשֶׁה אִיזֵיל חוֹת אֲרֵי חַבֵּיל עַמָּךְ דְּאַסֵּיקְתָּא מֵאַרְעָא דְּמִצְרָיִם׃}
{And the \lord\space spoke unto Moses: ‘Go, get thee down; for thy people, that thou broughtest up out of the land of Egypt, have dealt corruptly;}{\arabic{verse}}
\threeverse{\arabic{verse}}%Ex.32:8
{סָ֣רוּ מַהֵ֗ר מִן־הַדֶּ֙רֶךְ֙ אֲשֶׁ֣ר צִוִּיתִ֔ם עָשׂ֣וּ לָהֶ֔ם עֵ֖גֶל מַסֵּכָ֑ה וַיִּשְׁתַּֽחֲווּ־לוֹ֙ וַיִּזְבְּחוּ־ל֔וֹ וַיֹּ֣אמְר֔וּ אֵ֤לֶּה אֱלֹהֶ֙יךָ֙ יִשְׂרָאֵ֔ל אֲשֶׁ֥ר הֶֽעֱל֖וּךָ מֵאֶ֥רֶץ מִצְרָֽיִם׃}
{סְטוֹ בִּפְרִיעַ מִן אוֹרְחָא דְּפַקֵּידְתִּנּוּן עֲבַדוּ לְהוֹן עֵיגַל מַתְּכָא וּסְגִידוּ לֵיהּ וְדַבַּחוּ לֵיהּ וַאֲמַרוּ אִלֵּין דַּחְלָתָךְ יִשְׂרָאֵל דְּאַסְּקוּךְ מֵאַרְעָא דְּמִצְרָיִם׃}
{they have turned aside quickly out of the way which I commanded them; they have made them a molten calf, and have worshipped it, and have sacrificed unto it, and said: This is thy god, O Israel, which brought thee up out of the land of Egypt.’}{\arabic{verse}}
\threeverse{\arabic{verse}}%Ex.32:9
{וַיֹּ֥אמֶר יְהֹוָ֖ה אֶל־מֹשֶׁ֑ה רָאִ֙יתִי֙ אֶת־הָעָ֣ם הַזֶּ֔ה וְהִנֵּ֥ה עַם־קְשֵׁה־עֹ֖רֶף הֽוּא׃
\rashi{\rashiDH{קשה עורף. }מחזירין קשי ערפם לנגד מוכיחיהם וממאנים לשמוע׃}}
{וַאֲמַר יְיָ לְמֹשֶׁה גְּלֵי קֳדָמַי עַמָּא הָדֵין וְהָא עַם קְשֵׁי קְדָל הוּא׃}
{And the \lord\space said unto Moses: ‘I have seen this people, and, behold, it is a stiffnecked people.}{\arabic{verse}}
\threeverse{\arabic{verse}}%Ex.32:10
{וְעַתָּה֙ הַנִּ֣יחָה לִּ֔י וְיִֽחַר־אַפִּ֥י בָהֶ֖ם וַאֲכַלֵּ֑ם וְאֶֽעֱשֶׂ֥ה אוֹתְךָ֖ לְג֥וֹי גָּדֽוֹל׃
\rashi{\rashiDH{הניחה לי. }עדיין לא שמענו שהתפלל משה עליהם והוא אומר הניחה לי, אלא כאן פתח לו פתח והודיעו שהדבר תלוי בו, שאם יתפלל עליהם לא יכלם (שמו״ר מב, י)׃ }}
{וּכְעַן אָנַח בָּעוּתָךְ מִן קֳדָמַי וְיִתְקַף רוּגְזִי בְּהוֹן וַאֲשֵׁיצֵינוּן וְאַעֲבֵיד יָתָךְ לְעַם סַגִּי׃}
{Now therefore let Me alone, that My wrath may wax hot against them, and that I may consume them; and I will make of thee a great nation.’}{\arabic{verse}}
\threeverse{\arabic{verse}}%Ex.32:11
{וַיְחַ֣ל מֹשֶׁ֔ה אֶת־פְּנֵ֖י יְהֹוָ֣ה אֱלֹהָ֑יו וַיֹּ֗אמֶר לָמָ֤ה יְהֹוָה֙ יֶחֱרֶ֤ה אַפְּךָ֙ בְּעַמֶּ֔ךָ אֲשֶׁ֤ר הוֹצֵ֙אתָ֙ מֵאֶ֣רֶץ מִצְרַ֔יִם בְּכֹ֥חַ גָּד֖וֹל וּבְיָ֥ד חֲזָקָֽה׃
\rashi{\rashiDH{למה ה׳ יחרה אפך. }כלום מתקנא אלא חכם בחכם, גבור בגבור׃ }}
{וְצַלִּי מֹשֶׁה קֳדָם יְיָ אֱלָהֵיהּ וַאֲמַר לְמָא יְיָ יִתְקַף רוּגְזָךְ בְּעַמָּךְ דְּאַפֵּיקְתָּא מֵאַרְעָא דְּמִצְרַיִם בְּחֵיל רַב וּבְיַד תַּקִּיפָא׃}
{And Moses besought the \lord\space his God, and said: ‘\lord, why doth Thy wrath wax hot against Thy people, that Thou hast brought forth out of the land of Egypt with great power and with a mighty hand?}{\arabic{verse}}
\threeverse{\arabic{verse}}%Ex.32:12
{לָ֩מָּה֩ יֹאמְר֨וּ מִצְרַ֜יִם לֵאמֹ֗ר בְּרָעָ֤ה הֽוֹצִיאָם֙ לַהֲרֹ֤ג אֹתָם֙ בֶּֽהָרִ֔ים וּ֨לְכַלֹּתָ֔ם מֵעַ֖ל פְּנֵ֣י הָֽאֲדָמָ֑ה שׁ֚וּב מֵחֲר֣וֹן אַפֶּ֔ךָ וְהִנָּחֵ֥ם עַל־הָרָעָ֖ה לְעַמֶּֽךָ׃
\rashi{\rashiDH{והנחם. }התעשת מחשבה אחרת להיטיב להם׃}\rashi{\rashiDH{על הרעה. }אשר חשבת להם׃ 
}}
{לְמָא יֵימְרוּן מִצְרָאֵי לְמֵימַר בְּבִישָׁא אַפֵּיקִנּוּן לְקַטָּלָא יָתְהוֹן בֵּינֵי טוּרַיָּא וּלְשֵׁיצָיוּתְהוֹן מֵעַל אַפֵּי אַרְעָא תּוּב מִתְּקוֹף רוּגְזָךְ וַאֲתֵיב מִן בִּשְׁתָּא דְּמַלֵּילְתָּא לְמֶעֱבַד לְעַמָּךְ׃}
{Wherefore should the Egyptians speak, saying: For evil did He bring them forth, to slay them in the mountains, and to consume them from the face of the earth? Turn from Thy fierce wrath, and repent of this evil against Thy people.}{\arabic{verse}}
\threeverse{\arabic{verse}}%Ex.32:13
{זְכֹ֡ר לְאַבְרָהָם֩ לְיִצְחָ֨ק וּלְיִשְׂרָאֵ֜ל עֲבָדֶ֗יךָ אֲשֶׁ֨ר נִשְׁבַּ֣עְתָּ לָהֶם֮ בָּךְ֒ וַתְּדַבֵּ֣ר אֲלֵהֶ֔ם אַרְבֶּה֙ אֶֽת־זַרְעֲכֶ֔ם כְּכוֹכְבֵ֖י הַשָּׁמָ֑יִם וְכׇל־הָאָ֨רֶץ הַזֹּ֜את אֲשֶׁ֣ר אָמַ֗רְתִּי אֶתֵּן֙ לְזַרְעֲכֶ֔ם וְנָחֲל֖וּ לְעֹלָֽם׃
\rashi{\rashiDH{זכור לאברהם. }אם עברו על עשרת הדברות, אברהם אביהם נתנסה בעשר נסיונות ועדיין לא קבל שכרו, תנהו לו ויצאו עשר בעשר׃ }\rashi{\rashiDH{לאברהם ליצחק ולישראל. }אם לשרפה הם, זכור לאברהם שמסר עצמו להשרף עליך באור כשדים, אם להריגה, זכור ליצחק שפשט צוארו לעקידה, אם לגלות, זכור ליעקב שגלה לחרן (שמו״ר מד, ה), ואם אינן נצולין בזכותן, מה אתה אומר לי ואעשה אותך לגוי גדול, ואם כסא של שלש רגלים אינו עומד לפניך בשעת כעסך, קל וחמר לכסא של רגל אחד (ברכות לב.)׃ }\rashi{\rashiDH{אשר נשבעת להם בך. }לא נשבעת להם בדבר שהוא כלה, לא בשמים ולא בארץ, לא בהרים ולא בגבעות, אלא בך, שאתה קיים ושבועתך קיימת לעולם, שנאמר בִּי נִשׁבַּעְתִּי נְאֻם ה׳ (בראשית כב, טז), וליצחק נאמר, וַהֲקִימֹותִי אֶת הַשְׁבֻעָה אֲשֶׁר נִשְׁבַּעְתִּי לְאַבְרָהָם אָבִיךָ (שם כו, ג), וליעקב נאמר אֲנִי אֵל שַׁדַּי פְּרֵה וּרְבֵה (שם לה, יא), נשבע לו באל שדי׃ }}
{אִדְּכַר לְאַבְרָהָם לְיִצְחָק וּלְיִשְׂרָאֵל עַבְדָךְ דְּקַיֵּימְתָּא לְהוֹן בְּמֵימְרָךְ וּמַלֵּילְתָּא עִמְּהוֹן אַסְגֵּי יָת בְּנֵיכוֹן כְּכוֹכְבֵי שְׁמַיָּא וְכָל אַרְעָא הָדָא דַּאֲמַרִית אֶתֵּין לִבְנֵיכוֹן וְיַחְסְנוּן לְעָלַם׃}
{Remember Abraham, Isaac, and Israel, Thy servants, to whom Thou didst swear by Thine own self, and saidst unto them: I will multiply your seed as the stars of heaven, and all this land that I have spoken of will I give unto your seed, and they shall inherit it for ever.’}{\arabic{verse}}
\threeverse{\arabic{verse}}%Ex.32:14
{וַיִּנָּ֖חֶם יְהֹוָ֑ה עַל־הָ֣רָעָ֔ה אֲשֶׁ֥ר דִּבֶּ֖ר לַעֲשׂ֥וֹת לְעַמּֽוֹ׃ \petucha }
{וְתָב יְיָ מִן בִּשְׁתָּא דְּמַלֵּיל לְמֶעֱבַד לְעַמֵּיהּ׃}
{And the \lord\space repented of the evil which He said He would do unto His people.}{\arabic{verse}}
\threeverse{\arabic{verse}}%Ex.32:15
{וַיִּ֜פֶן וַיֵּ֤רֶד מֹשֶׁה֙ מִן־הָהָ֔ר וּשְׁנֵ֛י לֻחֹ֥ת הָעֵדֻ֖ת בְּיָד֑וֹ לֻחֹ֗ת כְּתֻבִים֙ מִשְּׁנֵ֣י עֶבְרֵיהֶ֔ם מִזֶּ֥ה וּמִזֶּ֖ה הֵ֥ם כְּתֻבִֽים׃
\rashi{\rashiDH{משני עבריהם. }היו האותיות נקראות, ומעשה נסים היה (שבת קד.)׃ }}
{וְאִתְפְּנִי וּנְחַת מֹשֶׁה מִן טוּרָא וּתְרֵין לוּחֵי סָהֲדוּתָא בִּידֵיהּ לוּחֵי כְּתִיבִין מִתְּרֵין עֶבְרֵיהוֹן מִכָּא וּמִכָּא אִנּוּן כְּתִיבִין׃}
{And Moses turned, and went down from the mount, with the two tables of the testimony in his hand; tables that were written on both their sides; on the one side and on the other were they written.}{\arabic{verse}}
\threeverse{\arabic{verse}}%Ex.32:16
{וְהַ֨לֻּחֹ֔ת מַעֲשֵׂ֥ה אֱלֹהִ֖ים הֵ֑מָּה וְהַמִּכְתָּ֗ב מִכְתַּ֤ב אֱלֹהִים֙ ה֔וּא חָר֖וּת עַל־הַלֻּחֹֽת׃
\rashi{\rashiDH{מעשה אלהים המה. }כמשמעו, הוא בכבודו עשאן. דבר אחר, כאדם האומר לחברו כל עסקיו של פלוני במלאכה פלונית, כך כל שעשועיו של הקב״ה בתורה׃ }\rashi{\rashiDH{חרות. }לשון חרת וחרט אחד הוא, שניהם לשון חיקוק, אנטליי״ר בלע״ז (איינשניידען)׃ }}
{וְלוּחַיָּא עוּבָדָא דַּייָ אִנּוּן וּכְתָבָא כְּתָבָא דַּייָ הוּא מְפָרַשׁ עַל לוּחַיָּא׃}
{And the tables were the work of God, and the writing was the writing of God, graven upon the tables.}{\arabic{verse}}
\threeverse{\arabic{verse}}%Ex.32:17
{וַיִּשְׁמַ֧ע יְהוֹשֻׁ֛עַ אֶת־ק֥וֹל הָעָ֖ם בְּרֵעֹ֑ה וַיֹּ֙אמֶר֙ אֶל־מֹשֶׁ֔ה ק֥וֹל מִלְחָמָ֖ה בַּֽמַּחֲנֶֽה׃
\rashi{\rashiDH{ברעה. }בהריעו, שהיו מריעים ושמחים וצוחקים׃ }}
{וּשְׁמַע יְהוֹשֻעַ יָת קָל עַמָּא כַּד מְיַבְּבִין וַאֲמַר לְמֹשֶׁה קָל קְרָבָא בְּמַשְׁרִיתָא׃}
{And when Joshua heard the noise of the people as they shouted, he said unto Moses: ‘There is a noise of war in the camp.’}{\arabic{verse}}
\threeverse{\arabic{verse}}%Ex.32:18
{וַיֹּ֗אמֶר אֵ֥ין קוֹל֙ עֲנ֣וֹת גְּבוּרָ֔ה וְאֵ֥ין ק֖וֹל עֲנ֣וֹת חֲלוּשָׁ֑ה ק֣וֹל עַנּ֔וֹת אָנֹכִ֖י שֹׁמֵֽעַ׃
\rashi{\rashiDH{אין קול ענות גבורה. }אין קול הזה נראה קול עניית גבורים הצועקים נצחון, ולא קול חלשים שצועקים וי, או ניסה׃ }\rashi{\rashiDH{קול ענות. }קול חרופין וגדופין, המענין את נפש שומען כשנאמרין לו׃ }}
{וַאֲמַר לָא קָל גִּבָּרִין דְּנָצְחִין בִּקְרָבָא וְאַף לָא קָל חַלָּשִׁין דְּמִתַּבְּרִין קָל דִּמְחָיְכִין אֲנָא שָׁמַע׃}
{And he said: ‘It is not the voice of them that shout for mastery, neither is it the voice of them that cry for being overcome, but the noise of them that sing do I hear.’}{\arabic{verse}}
\threeverse{\arabic{verse}}%Ex.32:19
{וַֽיְהִ֗י כַּאֲשֶׁ֤ר קָרַב֙ אֶל־הַֽמַּחֲנֶ֔ה וַיַּ֥רְא אֶת־הָעֵ֖גֶל וּמְחֹלֹ֑ת וַיִּֽחַר־אַ֣ף מֹשֶׁ֗ה וַיַּשְׁלֵ֤ךְ מִיָּדָו֙ אֶת־הַלֻּחֹ֔ת וַיְשַׁבֵּ֥ר אֹתָ֖ם תַּ֥חַת הָהָֽר׃
\rashi{\rashiDH{וישלך מידו וגו׳. }אמר, מה פסח שהוא אחד מן המצות, אמרה תורה כל בן נכר לא יאכל בו, התורה כולה כאן, וכל ישראל מומרים ואתננה להם׃ }\rashi{\rashiDH{תחת ההר. }לרגלי ההר׃}}
{וַהֲוָה כַּד קְרֵיב לְמַשְׁרִיתָא וַחֲזָא יָת עִגְלָא וְחִנְגִין וּתְקֵיף רוּגְזָא דְּמֹשֶׁה וּרְמָא מִיְּדוֹהִי יָת לוּחַיָּא וְתַבַּר יָתְהוֹן בְּשִׁפּוֹלֵי טוּרָא׃}
{And it came to pass, as soon as he came nigh unto the camp, that he saw the calf and the dancing; and Moses’ anger waxed hot, and he cast the tables out of his hands, and broke them beneath the mount.}{\arabic{verse}}
\threeverse{\arabic{verse}}%Ex.32:20
{וַיִּקַּ֞ח אֶת־הָעֵ֨גֶל אֲשֶׁ֤ר עָשׂוּ֙ וַיִּשְׂרֹ֣ף בָּאֵ֔שׁ וַיִּטְחַ֖ן עַ֣ד אֲשֶׁר־דָּ֑ק וַיִּ֙זֶר֙ עַל־פְּנֵ֣י הַמַּ֔יִם וַיַּ֖שְׁקְ אֶת־בְּנֵ֥י יִשְׂרָאֵֽל׃
\rashi{\rashiDH{ויזר. }לשון נפוץ, וכן יזורה על נוהו גפרית (איוב יח, טו), וכן כִּי חִנָּם מְזֹרָה הָרָשֶׁת (משלי א, יז), שזורין בה דגן וקטניות׃ }\rashi{\rashiDH{וישק את בני ישראל. }נתכוין לבדקם כסוטות, שלש מיתות נדונו שם, אם יש עדים והתראה, בסייף, כמשפט אנשי עיר הנדחת שהן מרובין, עדים בלא התראה, במגפה, שנאמר וַיִּגֹף ה׳ אֶת הָעָם, לא עדים ולא התראה, בהדרוקן (יומא סו׃), שבדקום המים וצבו בטניהם׃ }}
{וּנְסֵיב יָת עִגְלָא דַּעֲבַדוּ וְאוֹקֵיד בְּנוּרָא וְשָׁף עַד דַּהֲוָה דַּקִיק וּדְרָא עַל אַפֵּי מַיָּא וְאַשְׁקִי יָת בְּנֵי יִשְׂרָאֵל׃}
{And he took the calf which they had made, and burnt it with fire, and ground it to powder, and strewed it upon the water, and made the children of Israel drink of it.}{\arabic{verse}}
\threeverse{\arabic{verse}}%Ex.32:21
{וַיֹּ֤אמֶר מֹשֶׁה֙ אֶֽל־אַהֲרֹ֔ן מֶֽה־עָשָׂ֥ה לְךָ֖ הָעָ֣ם הַזֶּ֑ה כִּֽי־הֵבֵ֥אתָ עָלָ֖יו חֲטָאָ֥ה גְדֹלָֽה׃
\rashi{\rashiDH{מה עשה לך העם. }כמה יסורים סבלת שיסרוך עד שלא תביא עליהם חטא זה׃}}
{וַאֲמַר מֹשֶׁה לְאַהֲרֹן מָא עֲבַד לָךְ עַמָּא הָדֵין אֲרֵי אֵיתִיתָא עֲלוֹהִי חוֹבָא רַבָּא׃}
{And Moses said unto Aaron: ‘What did this people unto thee, that thou hast brought a great sin upon them?’}{\arabic{verse}}
\threeverse{\arabic{verse}}%Ex.32:22
{וַיֹּ֣אמֶר אַהֲרֹ֔ן אַל־יִ֥חַר אַ֖ף אֲדֹנִ֑י אַתָּה֙ יָדַ֣עְתָּ אֶת־הָעָ֔ם כִּ֥י בְרָ֖ע הֽוּא׃
\rashi{\rashiDH{כי ברע הוא. }בדרך רע הם הולכין תמיד, ובנסיונות לפני המקום׃ }}
{וַאֲמַר אַהֲרֹן לָא יִתְקַף רוּגְזָא דְּרִבּוֹנִי אַתְּ יְדַעְתְּ יָת עַמָּא אֲרֵי בְּבִישׁ הוּא׃}
{And Aaron said: ‘Let not the anger of my lord wax hot; thou knowest the people, that they are set on evil.}{\arabic{verse}}
\threeverse{\arabic{verse}}%Ex.32:23
{וַיֹּ֣אמְרוּ לִ֔י עֲשֵׂה־לָ֣נוּ אֱלֹהִ֔ים אֲשֶׁ֥ר יֵלְכ֖וּ לְפָנֵ֑ינוּ כִּי־זֶ֣ה \legarmeh  מֹשֶׁ֣ה הָאִ֗ישׁ אֲשֶׁ֤ר הֶֽעֱלָ֙נוּ֙ מֵאֶ֣רֶץ מִצְרַ֔יִם לֹ֥א יָדַ֖עְנוּ מֶה־הָ֥יָה לֽוֹ׃}
{וַאֲמַרוּ לִי עֲבֵיד לַנָא דַּחְלָן דִּיהָכָן קֳדָמַנָא אֲרֵי דֵין מֹשֶׁה גּוּבְרָא דְּאַסְּקַנָא מֵאַרְעָא דְּמִצְרַיִם לָא יְדַעְנָא מָא הֲוָה לֵיהּ׃}
{So they said unto me: Make us a god, which shall go before us; for as for this Moses, the man that brought us up out of the land of Egypt, we know not what is become of him.}{\arabic{verse}}
\threeverse{\arabic{verse}}%Ex.32:24
{וָאֹמַ֤ר לָהֶם֙ לְמִ֣י זָהָ֔ב הִתְפָּרָ֖קוּ וַיִּתְּנוּ־לִ֑י וָאַשְׁלִכֵ֣הוּ בָאֵ֔שׁ וַיֵּצֵ֖א הָעֵ֥גֶל הַזֶּֽה׃
\rashi{\rashiDH{ואמר להם. }דבר אחד, למי זהב לבד, והם מהרו והתפרקו ויתנו לי׃ }\rashi{\rashiDH{ואשלכהו באש. }ולא ידעתי שיצא העגל הזה, ויצא׃ }}
{וַאֲמַרִית לְהוֹן לְמַן דַּהְבָּא פָּרִיקוּ וִיהַבוּ לִי וּרְמֵיתֵיהּ בְּנוּרָא וּנְפַק עִגְלָא הָדֵין׃}
{And I said unto them: Whosoever hath any gold, let them break it off; so they gave it me; and I cast it into the fire, and there came out this calf.’}{\arabic{verse}}
\threeverse{\arabic{verse}}%Ex.32:25
{וַיַּ֤רְא מֹשֶׁה֙ אֶת־הָעָ֔ם כִּ֥י פָרֻ֖עַ ה֑וּא כִּֽי־פְרָעֹ֣ה אַהֲרֹ֔ן לְשִׁמְצָ֖ה בְּקָמֵיהֶֽם׃
\rashi{\rashiDH{פרוע. }מגולה, נתגלה שמצו וקלונו, כמו וּפָרַע אֶת רֹאשׁ הָאִשָּׁה (במדבר ה, יח)׃ }\rashi{\rashiDH{לשמצה בקמיהם. }להיות להם הדבר הזה לגנות בפי כל הקמים עליהם׃}}
{וַחֲזָא מֹשֶׁה יָת עַמָּא אֲרֵי בְּטִיל הוּא אֲרֵי בַּטֵּילִנּוּן אַהֲרֹן לְאַסָּבוּתְהוֹן שׁוֹם בִּישׁ לְדָרֵיהוֹן׃}
{And when Moses saw that the people were broken loose—for Aaron had let them loose for a derision among their enemies—}{\arabic{verse}}
\threeverse{\arabic{verse}}%Ex.32:26
{וַיַּעֲמֹ֤ד מֹשֶׁה֙ בְּשַׁ֣עַר הַֽמַּחֲנֶ֔ה וַיֹּ֕אמֶר מִ֥י לַיהֹוָ֖ה אֵלָ֑י וַיֵּאָסְפ֥וּ אֵלָ֖יו כׇּל־בְּנֵ֥י לֵוִֽי׃
\rashi{\rashiDH{מי לה׳ אלי. }יבא אלי׃}\rashi{\rashiDH{בני לוי. }מכאן שכל השבט כשר (יומא שם)׃ 
}}
{וְקָם מֹשֶׁה בִּתְרַע מַשְׁרִיתָא וַאֲמַר מַן דָּחֲלַיָּא דַּייָ יֵיתוֹן לְוָתִי וְאִתְכְּנִישׁוּ לְוָתֵיהּ כָּל בְּנֵי לֵוִי׃}
{then Moses stood in the gate of the camp, and said: ‘Whoso is on the \lord’S side, let him come unto me.’ And all the sons of Levi gathered themselves together unto him.}{\arabic{verse}}
\threeverse{\arabic{verse}}%Ex.32:27
{וַיֹּ֣אמֶר לָהֶ֗ם כֹּֽה־אָמַ֤ר יְהֹוָה֙ אֱלֹהֵ֣י יִשְׂרָאֵ֔ל שִׂ֥ימוּ אִישׁ־חַרְבּ֖וֹ עַל־יְרֵכ֑וֹ עִבְר֨וּ וָשׁ֜וּבוּ מִשַּׁ֤עַר לָשַׁ֙עַר֙ בַּֽמַּחֲנֶ֔ה וְהִרְג֧וּ אִֽישׁ־אֶת־אָחִ֛יו וְאִ֥ישׁ אֶת־רֵעֵ֖הוּ וְאִ֥ישׁ אֶת־קְרֹבֽוֹ׃
\rashi{\rashiDH{כה אמר וגו׳. }והיכן אמר, זֹבֵח לָאֱלֹהִים יָחֳרָם (שמות כב, יט), כך שנויה במכילתא (פסחא פי״ב)׃ }\rashi{\rashiDH{אחיו. }מאמו, והוא ישראל׃ 
}}
{וַאֲמַר לְהוֹן כִּדְנָן אֲמַר יְיָ אֱלָהָא דְּיִשְׂרָאֵל שַׁוּוֹ גְּבַר חַרְבֵּיהּ עַל יִרְכֵּיהּ עֵיבַרוּ וְתוּבוּ מִתְּרַע לִתְרַע בְּמַשְׁרִיתָא וּקְטוּלוּ גְּבַר יָת אֲחוּהִי וּגְבַר יָת חַבְרֵיהּ וֶאֱנָשׁ יָת קָרִיבֵיהּ׃}
{And he said unto them: ‘Thus saith the \lord, the God of Israel: Put ye every man his sword upon his thigh, and go to and fro from gate to gate throughout the camp, and slay every man his brother, and every man his companion, and every man his neighbour.’}{\arabic{verse}}
\threeverse{\arabic{verse}}%Ex.32:28
{וַיַּֽעֲשׂ֥וּ בְנֵֽי־לֵוִ֖י כִּדְבַ֣ר מֹשֶׁ֑ה וַיִּפֹּ֤ל מִן־הָעָם֙ בַּיּ֣וֹם הַה֔וּא כִּשְׁלֹ֥שֶׁת אַלְפֵ֖י אִֽישׁ׃}
{וַעֲבַדוּ בְנֵי לֵוִי כְּפִתְגָמָא דְּמֹשֶׁה וּנְפַל מִן עַמָּא בְּיוֹמָא הַהוּא כִּתְלָתָא אַלְפִין גּוּבְרָא׃}
{And the sons of Levi did according to the word of Moses; and there fell of the people that day about three thousand men.}{\arabic{verse}}
\threeverse{\arabic{verse}}%Ex.32:29
{וַיֹּ֣אמֶר מֹשֶׁ֗ה מִלְא֨וּ יֶדְכֶ֤ם הַיּוֹם֙ לַֽיהֹוָ֔ה כִּ֛י אִ֥ישׁ בִּבְנ֖וֹ וּבְאָחִ֑יו וְלָתֵ֧ת עֲלֵיכֶ֛ם הַיּ֖וֹם בְּרָכָֽה׃
\rashi{\rashiDH{מלאו ידכם. }אתם ההורגים אותם, בדבר זה תתחנכו להיות כהנים למקום׃ }\rashi{\rashiDH{כי איש. }מכם ימלא ידו בבנו ובאחיו׃ 
}}
{וַאֲמַר מֹשֶׁה קָרִיבוּ יְדֵיכוֹן יוֹמָא דֵין קוּרְבָּנָא קֳדָם יְיָ אֲרֵי גְּבַר בִּבְרֵיהּ וּבַאֲחוּהִי וּלְאֵיתָאָה עֲלֵיכוֹן יוֹמָא דֵין בִּרְכָן׃}
{And Moses said: ‘Consecrate yourselves to-day to the \lord, for every man hath been against his son and against his brother; that He may also bestow upon you a blessing this day.’}{\arabic{verse}}
\threeverse{\arabic{verse}}%Ex.32:30
{וַיְהִי֙ מִֽמׇּחֳרָ֔ת וַיֹּ֤אמֶר מֹשֶׁה֙ אֶל־הָעָ֔ם אַתֶּ֥ם חֲטָאתֶ֖ם חֲטָאָ֣ה גְדֹלָ֑ה וְעַתָּה֙ אֶֽעֱלֶ֣ה אֶל־יְהֹוָ֔ה אוּלַ֥י אֲכַפְּרָ֖ה בְּעַ֥ד חַטַּאתְכֶֽם׃
\rashi{\rashiDH{אכפרה בעד חטאתכם. }אשים כופר וקנוח וסתימה לנגד חטאתכם, להבדיל ביניכם ובין החטא׃ }}
{וַהֲוָה בְּיוֹמָא דְּבָתְרוֹהִי וַאֲמַר מֹשֶׁה לְעַמָּא אַתּוּן חַבְתּוּן חוֹבָא רַבָּא וּכְעַן אֶסַּק לִקְדָם יְיָ מָאִם אֲכַפַּר עַל חוֹבֵיכוֹן׃}
{And it came to pass on the morrow, that Moses said unto the people: ‘Ye have sinned a great sin; and now I will go up unto the \lord, peradventure I shall make atonement for your sin.’}{\arabic{verse}}
\threeverse{\arabic{verse}}%Ex.32:31
{וַיָּ֧שׇׁב מֹשֶׁ֛ה אֶל־יְהֹוָ֖ה וַיֹּאמַ֑ר אָ֣נָּ֗א חָטָ֞א הָעָ֤ם הַזֶּה֙ חֲטָאָ֣ה גְדֹלָ֔ה וַיַּֽעֲשׂ֥וּ לָהֶ֖ם אֱלֹהֵ֥י זָהָֽב׃
\rashi{\rashiDH{אלהי זהב. }אתה הוא שגרמת להם, שהשפעת להם זהב וכל חפצם, מה יעשו שלא יחטאו (יומא פו׃  ברכות לב.). משל למלך, שהיה מאכיל ומשקה את בנו ומקשטו, ותולה לו כיס בצוארו ומעמידו בפתח בית זונות, מה יעשה הבן שלא יחטא (ברכות שם)׃ }}
{וְתָב מֹשֶׁה לִקְדָם יְיָ וַאֲמַר בְּבָעוּ חָב עַמָּא הָדֵין חוֹבָא רַבָּא וַעֲבַדוּ לְהוֹן דַּחְלָן דִּדְהַב׃}
{And Moses returned unto the \lord, and said: ‘Oh, this people have sinned a great sin, and have made them a god of gold.}{\arabic{verse}}
\threeverse{\arabic{verse}}%Ex.32:32
{וְעַתָּ֖ה אִם־תִּשָּׂ֣א חַטָּאתָ֑ם וְאִם־אַ֕יִן מְחֵ֣נִי נָ֔א מִֽסִּפְרְךָ֖ אֲשֶׁ֥ר כָּתָֽבְתָּ׃
\rashi{\rashiDH{ועתה אם תשא חטאתם. }הרי טוב, איני אומר לך מחני׃ \rashiDH{ואם אין מחני. }וזה מקרא קצר, וכן הרבה׃ }\rashi{\rashiDH{מספרך. }מכל התורה כולה, שלא יאמרו עלי שלא הייתי כדאי לבקש עליהם רחמים׃ }}
{וּכְעַן אִם שְׁבַקְתְּ לְחוֹבֵיהוֹן וְאִם לָא מְחֵינִי כְעַן מִסִּפְרָךְ דִּכְתַבְתָּא׃}
{Yet now, if Thou wilt forgive their sin—; and if not, blot me, I pray Thee, out of Thy book which Thou hast written.’}{\arabic{verse}}
\threeverse{\arabic{verse}}%Ex.32:33
{וַיֹּ֥אמֶר יְהֹוָ֖ה אֶל־מֹשֶׁ֑ה מִ֚י אֲשֶׁ֣ר חָֽטָא־לִ֔י אֶמְחֶ֖נּוּ מִסִּפְרִֽי׃}
{וַאֲמַר יְיָ לְמֹשֶׁה מַן דְּחָב קֳדָמַי אֶמְחֵינֵיהּ מִסִּפְרִי׃}
{And the \lord\space said unto Moses: ‘Whosoever hath sinned against Me, him will I blot out of My book.}{\arabic{verse}}
\threeverse{\arabic{verse}}%Ex.32:34
{וְעַתָּ֞ה לֵ֣ךְ \legarmeh  נְחֵ֣ה אֶת־הָעָ֗ם אֶ֤ל אֲשֶׁר־דִּבַּ֙רְתִּי֙ לָ֔ךְ הִנֵּ֥ה מַלְאָכִ֖י יֵלֵ֣ךְ לְפָנֶ֑יךָ וּבְי֣וֹם פׇּקְדִ֔י וּפָקַדְתִּ֥י עֲלֵהֶ֖ם חַטָּאתָֽם׃
\rashi{\rashiDH{אל אשר דברתי לך. }יש כאן לך אצל דבור במקום אליך, וכן לדבר לו על אדוניהו (מלכים־א ב, יט)׃ }\rashi{\rashiDH{הנה מלאכי. }ולא אני׃ 
}\rashi{\rashiDH{וביום פקדי וגו׳. }עתה שמעתי אליך מלכלותם יחד, ותמיד תמיד כשאפקוד עליהם עונותיהם, ופקדתי עליהם מעט מן העון הזה עם שאר העונות, ואין פורענות באה על ישראל שאין בה קצת מפרעון עון העגל׃ }}
{וּכְעַן אִיזֵיל דַּבַּר יָת עַמָּא לְאַתְרָא דְּמַלֵּילִית לָךְ הָא מַלְאֲכִי יְהָךְ קֳדָמָךְ וּבְיוֹם אַסְעָרוּתִי וְאַסְעַר עֲלֵיהוֹן חוֹבֵיהוֹן׃}
{And now go, lead the people unto the place of which I have spoken unto thee; behold, Mine angel shall go before thee; nevertheless in the day when I visit, I will visit their sin upon them.’}{\arabic{verse}}
\threeverse{\arabic{verse}}%Ex.32:35
{וַיִּגֹּ֥ף יְהֹוָ֖ה אֶת־הָעָ֑ם עַ֚ל אֲשֶׁ֣ר עָשׂ֣וּ אֶת־הָעֵ֔גֶל אֲשֶׁ֥ר עָשָׂ֖ה אַהֲרֹֽן׃ \setuma         
\rashi{\rashiDH{ויגוף ה׳ את העם. }מיתה בידי שמים, לעדים בלא התראה׃ }}
{וּמְחָא יְיָ יָת עַמָּא עַל דְּאִשְׁתַּעֲבַדוּ לְעִגְלָא דַּעֲבַד אַהֲרֹן׃}
{And the \lord\space smote the people, because they made the calf, which Aaron made.}{\arabic{verse}}
\newperek
\threeverse{\Roman{chap}}%Ex.33:1
{וַיְדַבֵּ֨ר יְהֹוָ֤ה אֶל־מֹשֶׁה֙ לֵ֣ךְ עֲלֵ֣ה מִזֶּ֔ה אַתָּ֣ה וְהָעָ֔ם אֲשֶׁ֥ר הֶֽעֱלִ֖יתָ מֵאֶ֣רֶץ מִצְרָ֑יִם אֶל־הָאָ֗רֶץ אֲשֶׁ֣ר נִ֠שְׁבַּ֠עְתִּי לְאַבְרָהָ֨ם לְיִצְחָ֤ק וּֽלְיַעֲקֹב֙ לֵאמֹ֔ר לְזַרְעֲךָ֖ אֶתְּנֶֽנָּה׃
\rashi{\rashiDH{לך עלה מזה. }ארץ ישראל גבוה מכל הארצות, לכך נאמר עלה. דבר אחר, כלפי שאמר לו בשעת הכעס לך רד, אמר לו בשעת רצון לך עלה׃ }\rashi{\rashiDH{אתה והעם. }כאן לא אמר ועמך׃}}
{וּמַלֵּיל יְיָ עִם מֹשֶׁה אִיזֵיל סַק מִכָּא אַתְּ וְעַמָּא דְּאַסֵּיקְתָּא מֵאַרְעָא דְּמִצְרָיִם לְאַרְעָא דְּקַיֵּימִית לְאַבְרָהָם לְיִצְחָק וּלְיַעֲקֹב לְמֵימַר לִבְנָךְ אֶתְּנִנַּהּ׃}
{And the \lord\space spoke unto Moses: ‘Depart, go up hence, thou and the people that thou hast brought up out of the land of Egypt, unto the land of which I swore unto Abraham, to Isaac, and to Jacob, saying: Unto thy seed will I give it—}{\Roman{chap}}
\threeverse{\arabic{verse}}%Ex.33:2
{וְשָׁלַחְתִּ֥י לְפָנֶ֖יךָ מַלְאָ֑ךְ וְגֵֽרַשְׁתִּ֗י אֶת־הַֽכְּנַעֲנִי֙ הָֽאֱמֹרִ֔י וְהַֽחִתִּי֙ וְהַפְּרִזִּ֔י הַחִוִּ֖י וְהַיְבוּסִֽי׃
\rashi{\rashiDH{וגרשתי את הכנעני וגו׳. }ששה אומות הן, והגרגשי עמד ופנה מפניהם מאליו׃ }}
{וְאֶשְׁלַח קֳדָמָךְ מַלְאֲכָא וַאֲתָרֵיךְ יָת כְּנַעֲנָאֵי אֱמוֹרָאֵי וְחִתָּאֵי וּפְרִזָּאֵי חִוָּאֵי וִיבוּסָאֵי׃}
{and I will send an angel before thee; and I will drive out the Canaanite, the Amorite, and the Hittite, and the Perizzite, the Hivite, and the Jebusite—}{\arabic{verse}}
\threeverse{\arabic{verse}}%Ex.33:3
{אֶל־אֶ֛רֶץ זָבַ֥ת חָלָ֖ב וּדְבָ֑שׁ כִּי֩ לֹ֨א אֶֽעֱלֶ֜ה בְּקִרְבְּךָ֗ כִּ֤י עַם־קְשֵׁה־עֹ֙רֶף֙ אַ֔תָּה פֶּן־אֲכֶלְךָ֖ בַּדָּֽרֶךְ׃
\rashi{\rashiDH{אל ארץ זבת חלב ודבש. }אני אומר לך להעלותם׃}\rashi{\rashiDH{כי לא אעלה בקרבך. }לכן אני אומר לך ושלחתי לפניך מלאך׃}\rashi{\rashiDH{כי עם קשה עורף אתה. }וכששכינתי בקרבכם ואתם ממרים בי, מרבה אני עליכם זעם׃ 
}\rashi{\rashiDH{אכלך. }לשון כליון׃}}
{לַאֲרַע עָבְדָא חֲלָב וּדְבָשׁ אֲרֵי לָא אֲסַלֵּיק שְׁכִינְתִי מִבֵּינָךְ אֲרֵי עַם קְשֵׁי קְדָל אַתְּ דִּלְמָא אֲשֵׁיצֵינָךְ בְּאוֹרְחָא׃}
{unto a land flowing with milk and honey; for I will not go up in the midst of thee; for thou art a stiffnecked people; lest I consume thee in the way.’}{\arabic{verse}}
\threeverse{\arabic{verse}}%Ex.33:4
{וַיִּשְׁמַ֣ע הָעָ֗ם אֶת־הַדָּבָ֥ר הָרָ֛ע הַזֶּ֖ה וַיִּתְאַבָּ֑לוּ וְלֹא־שָׁ֛תוּ אִ֥ישׁ עֶדְי֖וֹ עָלָֽיו׃
\rashi{\rashiDH{הדבר הרע. }שאין השכינה שורה ומהלכת עמם׃}\rashi{\rashiDH{איש עדיו. }כתרים שניתנו להם בחורב כשאמרו נעשה ונשמע (שבת פח.)׃}}
{וּשְׁמַע עַמָּא יָת פִּתְגָמָא בִישָׁא הָדֵין וְאִתְאַבַּלוּ וְלָא שַׁוּוֹ גְּבַר תִּקּוּן זֵינֵיהּ עֲלוֹהִי׃}
{And when the people heard these evil tidings, they mourned; and no man did put on him his ornaments.}{\arabic{verse}}
\threeverse{\arabic{verse}}%Ex.33:5
{וַיֹּ֨אמֶר יְהֹוָ֜ה אֶל־מֹשֶׁ֗ה אֱמֹ֤ר אֶל־בְּנֵֽי־יִשְׂרָאֵל֙ אַתֶּ֣ם עַם־קְשֵׁה־עֹ֔רֶף רֶ֧גַע אֶחָ֛ד אֶֽעֱלֶ֥ה בְקִרְבְּךָ֖ וְכִלִּיתִ֑יךָ וְעַתָּ֗ה הוֹרֵ֤ד עֶדְיְךָ֙ מֵֽעָלֶ֔יךָ וְאֵדְעָ֖ה מָ֥ה אֶֽעֱשֶׂה־לָּֽךְ׃
\rashi{\rashiDH{רגע אחד אעלה בקרבך וכליתיך. }אם אעלה בקרבך ואתם ממרים בי בקשיות ערפכם, אזעום עליכם רגע אחד, שהוא שיעור זעמי, שנאמר חֲבִי כִמְעַט רֶגַע עַד יַעֲבָר זָעַם (ישעיה כו, כ), ואכלה אתכם, לפיכך טוב לכם שאשלח מלאך׃ }\rashi{\rashiDH{ועתה. }פורענות זו תלקו מיד, שתורידו עדיכם מעליכם׃ 
}\rashi{\rashiDH{ואדעה מה אעשה לך. }בפקודת שאר העון, אני יודע מה שבלבי לעשות לך׃ }}
{וַאֲמַר יְיָ לְמֹשֶׁה אֵימַר לִבְנֵי יִשְׂרָאֵל אַתּוּן עַם קְשֵׁי קְדָל שָׁעָה חֲדָא אֲסַלֵּיק שְׁכִינְתִי מִבֵּינָךְ וַאֲשֵׁיצֵינָךְ וּכְעַן אַעְדְּ תִּקּוּן זֵינָךְ מִנָּךְ גְּלֵי קֳדָמַי מָא אַעֲבֵיד לָךְ׃}
{And the \lord\space said unto Moses: ‘Say unto the children of Israel: Ye are a stiffnecked people; if I go up into the midst of thee for one moment, I shall consume thee; therefore now put off thy ornaments from thee, that I may know what to do unto thee.’}{\arabic{verse}}
\threeverse{\arabic{verse}}%Ex.33:6
{וַיִּֽתְנַצְּל֧וּ בְנֵֽי־יִשְׂרָאֵ֛ל אֶת־עֶדְיָ֖ם מֵהַ֥ר חוֹרֵֽב׃
\rashi{\rashiDH{את עדים מהר חורב. }את הָעֲדִי שהיה בידם מהר חורב׃}}
{וְאַעְדּוֹ בְנֵי יִשְׂרָאֵל יָת תִּקּוּן זֵינְהוֹן מִטּוּרָא דְּחוֹרֵב׃}
{And the children of Israel stripped themselves of their ornaments from mount Horeb onward.}{\arabic{verse}}
\threeverse{\arabic{verse}}%Ex.33:7
{וּמֹשֶׁה֩ יִקַּ֨ח אֶת־הָאֹ֜הֶל וְנָֽטָה־ל֣וֹ \legarmeh  מִח֣וּץ לַֽמַּחֲנֶ֗ה הַרְחֵק֙ מִן־הַֽמַּחֲנֶ֔ה וְקָ֥רָא ל֖וֹ אֹ֣הֶל מוֹעֵ֑ד וְהָיָה֙ כׇּל־מְבַקֵּ֣שׁ יְהֹוָ֔ה יֵצֵא֙ אֶל־אֹ֣הֶל מוֹעֵ֔ד אֲשֶׁ֖ר מִח֥וּץ לַֽמַּחֲנֶֽה׃
\rashi{\rashiDH{ומשה. }מאותו עון והלאה׃}\rashi{\rashiDH{יקח את האהל. }לשון הווה הוא, לוקח אהלו ונוטהו מחוץ למחנה, אמר, מנודה לרב מנודה לתלמיד׃ }\rashi{\rashiDH{הרחק. }אלפים אמה, כענין שנאמר אַךְ רָחֹוק יִהְיֶה בֵּינֵיכֶם וּבֵינָו כְּאַלְפַּיִם אַמָּה בַּמִּדָּה (יהושע ג, ד)׃ }\rashi{\rashiDH{וקרא לו. }והיה קורא לו אהל מועד, הוא בית ועד למבקשי תורה׃ }\rashi{\rashiDH{כל מבקש ה׳. }מכאן למבקש פני זקן, כמקבל פני שכינה׃ }\rashi{\rashiDH{יצא אל אהל מועד. }כמו יוצא. דבר אחר והיה כל מבקש ה׳, אפילו מלאכי השרת כשהיו שואלים מקום שכינה, חבריהם אומרים להם הרי הוא באהלו של משה׃ }}
{וּמֹשֶׁה נָסֵיב יָת מַשְׁכְּנָא וּפַרְסֵיהּ לֵיהּ מִבַּרָא לְמַשְׁרִיתָא אַרְחֵיק מִן מַשְׁרִיתָא וְקָרֵי לֵיהּ מַשְׁכַּן בֵּית אוּלְפָנָא וְהָוֵי כָּל דְּתָבַע אוּלְפָן מִן קֳדָם יְיָ נָפֵיק לְמַשְׁכַּן בֵּית אוּלְפָנָא דְּמִבַּרָא לְמַשְׁרִיתָא׃}
{Now Moses used to take the tent and to pitch it without the camp, afar off from the camp; and he called it The tent of meeting. And it came to pass, that every one that sought the \lord\space went out unto the tent of meeting, which was without the camp.}{\arabic{verse}}
\threeverse{\arabic{verse}}%Ex.33:8
{וְהָיָ֗ה כְּצֵ֤את מֹשֶׁה֙ אֶל־הָאֹ֔הֶל יָק֙וּמוּ֙ כׇּל־הָעָ֔ם וְנִ֨צְּב֔וּ אִ֖ישׁ פֶּ֣תַח אׇהֳל֑וֹ וְהִבִּ֙יטוּ֙ אַחֲרֵ֣י מֹשֶׁ֔ה עַד־בֹּא֖וֹ הָאֹֽהֱלָה׃
\rashi{\rashiDH{והיה. }לשון הווה׃}\rashi{\rashiDH{כצאת משה מן המחנה. }ללכת אל האהל׃}\rashi{\rashiDH{יקומו כל העם. }עומדים מפניו, ואין יושבין עד שנתכסה מהם׃ }\rashi{\rashiDH{והביטו אחרי משה. }לשבח, אשרי ילוד אשה שכך מובטח שהשכינה תכנס אחריו לפתח אהלו׃ }}
{וְהָוֵי כַּד נָפֵיק מֹשֶׁה לְמַשְׁכְּנָא קָיְמִין כָּל עַמָּא וּמִתְעַתְּדִין גְּבַר בִּתְרַע מַשְׁכְּנֵיהּ וּמִסְתַּכְּלִין אֲחוֹרֵי מֹשֶׁה עַד דְּעָלֵיל לְמַשְׁכְּנָא׃}
{And it came to pass, when Moses went out unto the Tent, that all the people rose up, and stood, every man at his tent door, and looked after Moses, until he was gone into the Tent.}{\arabic{verse}}
\threeverse{\arabic{verse}}%Ex.33:9
{וְהָיָ֗ה כְּבֹ֤א מֹשֶׁה֙ הָאֹ֔הֱלָה יֵרֵד֙ עַמּ֣וּד הֶֽעָנָ֔ן וְעָמַ֖ד פֶּ֣תַח הָאֹ֑הֶל וְדִבֶּ֖ר עִם־מֹשֶֽׁה׃
\rashi{\rashiDH{ודבר עם משה. }כמו ומִדַּבֵּר עם משה. תרגומו וּמִתְמַלֵּל עם משה, שהוא כבוד השכינה, כמו וַיִּשְׁמַע אֶת הַקֹּול מִדַּבֵּר אֵלָיו (במדבר ז, פט), ואינו קורא מְדבר אליו, כשהוא קורא מִדבר פתרונו הקול מדבר בינו לבין עצמו, וההדיוט שומע מאליו, וכשהוא קורא מְדבר, משמע שהמלך מדבר עם ההדיוט׃ }}
{וְהָוֵי כַּד עָלֵיל מֹשֶׁה לְמַשְׁכְּנָא נָחֵית עַמּוּדָא דַּעֲנָנָא וְקָאֵים בִּתְרַע מַשְׁכְּנָא וּמִתְמַלַּל עִם מֹשֶׁה׃}
{And it came to pass, when Moses entered into the Tent, the pillar of cloud descended, and stood at the door of the Tent; and [the \lord] spoke with Moses.}{\arabic{verse}}
\threeverse{\arabic{verse}}%Ex.33:10
{וְרָאָ֤ה כׇל־הָעָם֙ אֶת־עַמּ֣וּד הֶֽעָנָ֔ן עֹמֵ֖ד פֶּ֣תַח הָאֹ֑הֶל וְקָ֤ם כׇּל־הָעָם֙ וְהִֽשְׁתַּחֲו֔וּ אִ֖ישׁ פֶּ֥תַח אׇהֳלֽוֹ׃
\rashi{\rashiDH{והשתחוו. }לשכינה׃ 
}}
{וְחָזַן כָּל עַמָּא יָת עַמּוּדָא דַּעֲנָנָא קָאֵים בִּתְרַע מַשְׁכְּנָא וְקָיְמִין כָּל עַמָּא וְסָגְדִין גְּבַר בִּתְרַע מַשְׁכְּנֵיהּ׃}
{And when all the people saw the pillar of cloud stand at the door of the Tent, all the people rose up and worshipped, every man at his tent door.}{\arabic{verse}}
\threeverse{\arabic{verse}}%Ex.33:11
{וְדִבֶּ֨ר יְהֹוָ֤ה אֶל־מֹשֶׁה֙ פָּנִ֣ים אֶל־פָּנִ֔ים כַּאֲשֶׁ֛ר יְדַבֵּ֥ר אִ֖ישׁ אֶל־רֵעֵ֑הוּ וְשָׁב֙ אֶל־הַֽמַּחֲנֶ֔ה וּמְשָׁ֨רְת֜וֹ יְהוֹשֻׁ֤עַ בִּן־נוּן֙ נַ֔עַר לֹ֥א יָמִ֖ישׁ מִתּ֥וֹךְ הָאֹֽהֶל׃ \petucha 
\rashi{\rashiDH{ודבר ה׳ אל משה פנים אל פנים. }וּמִתְמַלֵּל עם משה׃}\rashi{\rashiDH{ושב אל המחנה. }לאחר שנדבר עמו, היה שב משה אל המחנה, ומלמד לזקנים מה שלמד, והדבר הזה נהג משה מיום הכפורים עד שהוקם המשכן ולא יותר, שהרי בשבעה עשר בתמוז נשתברו הלוחות (תענית כח׃), ובי״ח שרף את העגל ודן את החוטאים, ובי״ט עלה, שנאמר וַיְהִי מִמָּחֳרָת וַיֹאמֶר משֶׁה אֶל הָעָם וגו׳ (שמות לב, ל), ועשה שם ארבעים יום ובקש רחמים, שנאמר וָאֶתְנַפַּל לִפְנֵי ה׳ וגו׳ (דברים ט, יח), ובראש חדש אלול נאמר לו ועלית בבקר אל הר סיני, לקבל לוחות האחרונות, ועשה שם מ׳ יום, שנאמר בהם וְאָנֹכִי עָמַדְתִּי בָהָר כַּיָּמִים הָרִאשֹׁונִים וגו׳ (שם י, ו), מה הראשונים ברצון אף האחרונים ברצון, אמור מעתה, אמצעיים היו בכעס. בי׳ בתשרי נתרצה הקב״ה לישראל בשמחה ובלב שלם, ואמר לו למשה סָלַחְתִּי כִּדְבָרֶיך, ומסר לו לוחות אחרונות, וירד והתחיל לצוותן על מלאכת המשכן, ועשאוהו עד אחד בניסן, ומשהוקם לא נדבר עמו עוד, אלא מאהל מועד׃ }\rashi{\rashiDH{ושב אל המחנה. }תרגומו וְתָב לְמַשְׁרִיתָא, לפי שהוא לשון הווה, וכן כל הענין, וראה כל העם וחזן, ונצבו וקיימין. והביטו, ומסתכלין. והשתחוו, וְסָגְדִּין. ומדרשו, ודבר ה׳ אל משה שישוב אל המחנה, אמר לו אני בכעס ואתה בכעס, אם כן מי יקרבם׃ }}
{וּמְמַלֵּיל יְיָ עִם מֹשֶׁה מַמְלַל עִם מַמְלַל כְּמָא דִּימַלֵּיל גּוּבְרָא עִם חַבְרֵיהּ וְתָאֵיב לְמַשְׁרִיתָא וּמְשׁוּמְשָׁנֵיהּ יְהוֹשֻׁעַ בַּר נוּן עוּלֵימָא לָא עָדֵי מִגּוֹ מַשְׁכְּנָא׃}
{And the \lord\space spoke unto Moses face to face, as a man speaketh unto his friend. And he would return into the camp; but his minister Joshua, the son of Nun, a young man, departed not out of the Tent.}{\arabic{verse}}
\threeverse{\aliya{שלישי}}%Ex.33:12
{וַיֹּ֨אמֶר מֹשֶׁ֜ה אֶל־יְהֹוָ֗ה רְ֠אֵ֠ה אַתָּ֞ה אֹמֵ֤ר אֵלַי֙ הַ֚עַל אֶת־הָעָ֣ם הַזֶּ֔ה וְאַתָּה֙ לֹ֣א הֽוֹדַעְתַּ֔נִי אֵ֥ת אֲשֶׁר־תִּשְׁלַ֖ח עִמִּ֑י וְאַתָּ֤ה אָמַ֙רְתָּ֙ יְדַעְתִּ֣יךָֽ בְשֵׁ֔ם וְגַם־מָצָ֥אתָ חֵ֖ן בְּעֵינָֽי׃
\rashi{\rashiDH{ראה אתה אומר אלי. }ראה, תן עיניך ולבך על דבריך, אתה אומר אלי וגו׳ ואתה לא הודעתני וגו׳, ואשר אמרת לי הִנֵּה אָנֹכִי שֹׁלֵחַ מַלְאָךְ (שמות כג, כ), אין זו הודעה, שאין אני חפץ בה׃ }\rashi{\rashiDH{ואתה אמרת ידעתיך בשם. }הכרתיך משאר בני אדם בשם חשיבות, שהרי אמרת לי הִנֵּה אָנֹכִי בָּא אֵלֶיךָ בְּעַב הֶעָנָן וגו׳ וגם בְּךָ יַאֲמִינוּ לְעֹולָם (שם יט, ט)׃ }}
{וַאֲמַר מֹשֶׁה קֳדָם יְיָ חֲזִי דְּאַתְּ אָמַר לִי אַסֵּיק יָת עַמָּא הָדֵין וְאַתְּ לָא אוֹדַעְתַּנִי יָת דְּתִשְׁלַח עִמִּי וְאַתְּ אֲמַרְתְּ רַבִּיתָךְ בְּשׁוֹם וְאַף אַשְׁכַּחְתָּא רַחֲמִין קֳדָמָי׃}
{And Moses said unto the \lord: ‘See, Thou sayest unto me: Bring up this people; and Thou hast not let me know whom Thou wilt send with me. Yet Thou hast said: I know thee by name, and thou hast also found grace in My sight.}{\arabic{verse}}
\threeverse{\arabic{verse}}%Ex.33:13
{וְעַתָּ֡ה אִם־נָא֩ מָצָ֨אתִי חֵ֜ן בְּעֵינֶ֗יךָ הוֹדִעֵ֤נִי נָא֙ אֶת־דְּרָכֶ֔ךָ וְאֵדָ֣עֲךָ֔ לְמַ֥עַן אֶמְצָא־חֵ֖ן בְּעֵינֶ֑יךָ וּרְאֵ֕ה כִּ֥י עַמְּךָ֖ הַגּ֥וֹי הַזֶּֽה׃
\rashi{\rashiDH{ועתה. }אם אמת שמצאתי חן בעיניך, הודיעני נא את דרכיך, מה שכר אתה נותן למוצאי חן בעיניך׃ }\rashi{\rashiDH{ואדעך למען אמצא חן בעיניך. }ואדע בזו מדת תגמוליך, מה היא מציאת חן שמצאתי בעיניך, ופתרון למען אמצא חן, למען אכיר כמה שכר מציאת החן׃ 
}\rashi{\rashiDH{וראה כי עמך הגוי הזה. }שלא תאמר ואעשה אותך לגוי גדול, ואת אלה תעזוב, ראה כי עמך הם מקדם, ואם בהם תמאס, איני סומך על היוצאים מחלצי שיתקיימו, ואת תשלום השכר שלי בעם הזה תודיעני. ורבותינו דרשוהו במסכת ברכות (ז.), ואני ליישב המקראות על אופניהם ועל סדרם באתי׃ }}
{וּכְעַן אִם כְּעַן אַשְׁכַּחִית רַחֲמִין קֳדָמָךְ אוֹדַעְנִי כְעַן יָת אוֹרַח טוּבָךְ וְאֶדַּע רַחֲמָךְ בְּדִיל דְּאַשְׁכַּח רַחֲמִין קֳדָמָךְ וּגְלֵי קֳדָמָךְ אֲרֵי עַמָּךְ עַמָּא הָדֵין׃}
{Now therefore, I pray Thee, if I have found grace in Thy sight, show me now Thy ways, that I may know Thee, to the end that I may find grace in Thy sight; and consider that this nation is Thy people.’}{\arabic{verse}}
\threeverse{\arabic{verse}}%Ex.33:14
{וַיֹּאמַ֑ר פָּנַ֥י יֵלֵ֖כוּ וַהֲנִחֹ֥תִי לָֽךְ׃
\rashi{\rashiDH{ויאמר פני ילכו. }כתרגומו, לא אשלח עוד מלאך, אני בעצמי אלך, כמו וּפָנֶיךָ הֹלְכִים בַּקְרָב (שמואל־ב יז, יא)׃ }}
{וַאֲמַר שְׁכִינְתִי תְּהָךְ וַאֲנִיחַ לָךְ׃}
{And He said: ‘My presence shall go with thee, and I will give thee rest.’}{\arabic{verse}}
\threeverse{\arabic{verse}}%Ex.33:15
{וַיֹּ֖אמֶר אֵלָ֑יו אִם־אֵ֤ין פָּנֶ֙יךָ֙ הֹלְכִ֔ים אַֽל־תַּעֲלֵ֖נוּ מִזֶּֽה׃
\rashi{\rashiDH{ויאמר אליו. }בזו אני חפץ, כי על ידי מלאך אל תעלנו מזה׃ 
}}
{וַאֲמַר קֳדָמוֹהִי אִם לֵית שְׁכִינְתָךְ מְהַלְּכָא בֵּינַנָא לָא תַסְּקִנַּנָא מִכָּא׃}
{And he said unto Him: ‘If Thy presence go not with me, carry us not up hence.}{\arabic{verse}}
\threeverse{\arabic{verse}}%Ex.33:16
{וּבַמֶּ֣ה \legarmeh  יִוָּדַ֣ע אֵפ֗וֹא כִּֽי־מָצָ֨אתִי חֵ֤ן בְּעֵינֶ֙יךָ֙ אֲנִ֣י וְעַמֶּ֔ךָ הֲל֖וֹא בְּלֶכְתְּךָ֣ עִמָּ֑נוּ וְנִפְלִ֙ינוּ֙ אֲנִ֣י וְעַמְּךָ֔ מִכׇּ֨ל־הָעָ֔ם אֲשֶׁ֖ר עַל־פְּנֵ֥י הָאֲדָמָֽה׃ \petucha 
\rashi{\rashiDH{ובמה יודע אפוא. }יודע מציאת החן, הלא בלכתך עמנו, ועוד דבר אחר אני שואל ממך, שלא תשרה שכינתך עוד על אומות עובדי אלילים׃ }\rashi{\rashiDH{ונפלינו אני ועמך. }ונהיה מובדלים בדבר הזה מכל העם, כמו וְהִפְלָה ה׳ בֵּין מִקְנֵה יִשְׂרָאֵל וגו׳ (שמות ט, ד)׃ }}
{וּבְמָא יִתְיְדַע הָכָא אֲרֵי אַשְׁכַּחִית רַחֲמִין קֳדָמָךְ אֲנָא וְעַמָּךְ הֲלָא בִּמְהָךְ שְׁכִינְתָךְ עִמַּנָא וְיִתְעַבְדָן לַנָא פְרִישָׁן לִי וּלְעַמָּךְ מְשׁוּנַּי מִכָּל עַמָּא דְּעַל אַפֵּי אַרְעָא׃}
{For wherein now shall it be known that I have found grace in Thy sight, I and Thy people? is it not in that Thou goest with us, so that we are distinguished, I and Thy people, from all the people that are upon the face of the earth?’}{\arabic{verse}}
\threeverse{\aliya{רביעי}}%Ex.33:17
{וַיֹּ֤אמֶר יְהֹוָה֙ אֶל־מֹשֶׁ֔ה גַּ֣ם אֶת־הַדָּבָ֥ר הַזֶּ֛ה אֲשֶׁ֥ר דִּבַּ֖רְתָּ אֶֽעֱשֶׂ֑ה כִּֽי־מָצָ֤אתָ חֵן֙ בְּעֵינַ֔י וָאֵדָעֲךָ֖ בְּשֵֽׁם׃
\rashi{\rashiDH{גם את הדבר הזה. }שלא תשרה שכינתי עוד על עובדי אלילים אעשה, ואין דבריו של בלעם הרשע על ידי שריית שכינה, אלא נופל וגלוי עינים, כגון וְאֵלַי דָּבָר יְגֻנָּב (איוב ד, ב), שומעין על ידי שליח׃ }}
{וַאֲמַר יְיָ לְמֹשֶׁה אַף יָת פִּתְגָמָא הָדֵין דְּמַלֵּילְתָּא אַעֲבֵיד אֲרֵי אַשְׁכַּחְתָּא רַחֲמִין קֳדָמַי וְרַבִּיתָךְ בְּשׁוֹם׃}
{And the \lord\space said unto Moses: ‘I will do this thing also that thou hast spoken, for thou hast found grace in My sight, and I know thee by name.’}{\arabic{verse}}
\threeverse{\arabic{verse}}%Ex.33:18
{וַיֹּאמַ֑ר הַרְאֵ֥נִי נָ֖א אֶת־כְּבֹדֶֽךָ׃
\rashi{\rashiDH{ויאמר הראני נא את כבודך. }ראה משה שהיה עת רצון ודבריו מקובלים, והוסיף לשאול להראותו מראית כבודו׃ }}
{וַאֲמַר אַחְזִינִי כְעַן יָת יְקָרָךְ׃}
{And he said: ‘Show me, I pray Thee, Thy glory.’}{\arabic{verse}}
\threeverse{\arabic{verse}}%Ex.33:19
{וַיֹּ֗אמֶר אֲנִ֨י אַעֲבִ֤יר כׇּל־טוּבִי֙ עַל־פָּנֶ֔יךָ וְקָרָ֧אתִֽי בְשֵׁ֛ם יְהֹוָ֖ה לְפָנֶ֑יךָ וְחַנֹּתִי֙ אֶת־אֲשֶׁ֣ר אָחֹ֔ן וְרִחַמְתִּ֖י אֶת־אֲשֶׁ֥ר אֲרַחֵֽם׃
\rashi{\rashiDH{ויאמר אני אעביר וגו׳. }הגיעה שעה שתראה בכבודי מה שארשה אותך לראות, לפי שאני רוצה וצריך ללמדך סדר תפלה, שכשנצרכת לבקש רחמים על ישראל, הזכרת לי זכות אבות, כסבור אתה שאם תמה זכות אבות אין עוד תקוה, אני אעביר כל מדת טובי לפניך על הצור, ואתה צפון במערה׃}\rashi{\rashiDH{וקראתי בשם ה׳ לפניך.}  ללמדך סדר בקשת רחמים, אף אם תכלה זכות אבות. וכסדר זה שאתה רואה אותי מעוטף וקורא י״ג מדות (ראש השנה יז׃), הוי מלמד את ישראל לעשות כן, ועל ידי שיזכירו לפני רחום וחנון, יהיו נענין, כי רחמי לא כָלִים׃ }\rashi{\rashiDH{וחנותי את אשר אחון. }אותן פעמים שארצה לחון׃}\rashi{\rashiDH{ורחמתי. }עת שאחפוץ לרחם. עד כאן לא הבטיחו אלא עתים אענה ועתים לא אענה, אבל בשעת מעשה אמר לו הִנֵּה אָנֹכִי כֹּרֵת בְּרִית, הבטיחו שאינן חוזרות ריקם (שם)׃}}
{וַאֲמַר אֲנָא אַעְבַּר כָּל טוּבִי עַל אַפָּךְ וְאֶקְרֵי בִשְׁמָא דַּייָ קֳדָמָךְ וַאֲחוּן לְמַן דַּאֲחוּן וַאֲרַחֵים עַל מַן דַּאֲרַחֵים׃}
{And He said: ‘I will make all My goodness pass before thee, and will proclaim the name of the \lord\space before thee; and I will be gracious to whom I will be gracious, and will show mercy on whom I will show mercy.’}{\arabic{verse}}
\threeverse{\arabic{verse}}%Ex.33:20
{וַיֹּ֕אמֶר לֹ֥א תוּכַ֖ל לִרְאֹ֣ת אֶת־פָּנָ֑י כִּ֛י לֹֽא־יִרְאַ֥נִי הָאָדָ֖ם וָחָֽי׃
\rashi{\rashiDH{ויאמר לא תוכל וגו׳. }אף כשאעביר כל טובי על פניך, איני נותן לך רשות לראות את פני׃ }}
{וַאֲמַר לָא תִכּוֹל לְמִחְזֵי יָת אַפֵּי שְׁכִינְתִי אֲרֵי לָא יִחְזֵינַנִי אֲנָשָׁא וְיִתְקַיַּים׃}
{And He said: ‘Thou canst not see My face, for man shall not see Me and live.’}{\arabic{verse}}
\threeverse{\arabic{verse}}%Ex.33:21
{וַיֹּ֣אמֶר יְהֹוָ֔ה הִנֵּ֥ה מָק֖וֹם אִתִּ֑י וְנִצַּבְתָּ֖ עַל־הַצּֽוּר׃
\rashi{\rashiDH{הנה מקום אתי. }בהר אשר אני מדבר עמך תמיד, יש מקום מוכן לי לצרכך שאטמינך שם שלא תזוק, ומשם תראה מה שתראה, זו פשוטו. ומדרשו, על מקום שהשכינה שם מדבר, ואומר המקום אתי ואינו אומר אני במקום, שהקב״ה מקומו של עולם ואין עולמו מקומו׃ }}
{וַאֲמַר יְיָ הָא אֲתַר מְתוּקַּן קֳדָמָי וְתִתְעַתַּד עַל טִנָּרָא׃}
{And the \lord\space said: ‘Behold, there is a place by Me, and thou shalt stand upon the rock.}{\arabic{verse}}
\threeverse{\arabic{verse}}%Ex.33:22
{וְהָיָה֙ בַּעֲבֹ֣ר כְּבֹדִ֔י וְשַׂמְתִּ֖יךָ בְּנִקְרַ֣ת הַצּ֑וּר וְשַׂכֹּתִ֥י כַפִּ֛י עָלֶ֖יךָ עַד־עׇבְרִֽי׃
\rashi{\rashiDH{בעבור כבודי. }כשאעבור לפניך׃ 
}\rashi{\rashiDH{בנקרת הצור. }כמו הַעֵינֵי הָאֲנָשִׁים הָהֵם תְּנַקֵּר (במדבר טז, יד), יִקְּרוּהָ עֹרְבֵי נַחַל (משלי ל, יז), אֲנִי קַרְתִּי וְשָׁתִיתִי מָיִם (ישעיה לז, כה), גזרה אחת להם׃ \rashiDH{נקרת הצור. }כריית הצור׃}\rashi{\rashiDH{ושכותי כפי. }מכאן שנתנה רשות למחבלים לחבל (ת״כ פ׳ ויקרא), ותרגומו וְאָגֵין בְּמֵימְרִי, כנוי הוא לדרך כבוד של מעלה, שאינו צריך לסוכך עליו בכף ממש׃ }}
{וִיהֵי בְמִעְבַּר יְקָרִי וַאֲשַׁוֵּינָךְ בִּמְעָרַת טִנָּרָא וְאַגֵּין בְּמֵימְרִי עֲלָךְ עַד דְּאֶעְבַּר׃}
{And it shall come to pass, while My glory passeth by, that I will put thee in a cleft of the rock, and will cover thee with My hand until I have passed by.}{\arabic{verse}}
\threeverse{\arabic{verse}}%Ex.33:23
{וַהֲסִרֹתִי֙ אֶת־כַּפִּ֔י וְרָאִ֖יתָ אֶת־אֲחֹרָ֑י וּפָנַ֖י לֹ֥א יֵרָאֽוּ׃ \petucha \note{ספק פרשה סתומה בכתר ארם צובה}
\rashi{\rashiDH{והסרותי את כפי. }וְאַעְדֵי יַת דִּבְרַת יְקָרִי, כשתסתלק הנהגת כבודי מכנגד פניך, ללכת משם ולהלן׃ }\rashi{\rashiDH{וראית את אחורי. }הראהו קשר של תפילין׃}}
{וְאֶעְדֵּי יָת דִּבְרַת יְקָרִי וְתִחְזֵי יָת דְּבָתְרָי וְדִקְדָמַי לָא יִתַּחְזוֹן׃}
{And I will take away My hand, and thou shalt see My back; but My face shall not be seen.’}{\arabic{verse}}
\newperek
\threeverse{\aliya{חמישי}}%Ex.34:1
{וַיֹּ֤אמֶר יְהֹוָה֙ אֶל־מֹשֶׁ֔ה פְּסׇל־לְךָ֛ שְׁנֵֽי־לֻחֹ֥ת אֲבָנִ֖ים כָּרִאשֹׁנִ֑ים וְכָתַבְתִּי֙ עַל־הַלֻּחֹ֔ת אֶ֨ת־הַדְּבָרִ֔ים אֲשֶׁ֥ר הָי֛וּ עַל־הַלֻּחֹ֥ת הָרִאשֹׁנִ֖ים אֲשֶׁ֥ר שִׁבַּֽרְתָּ׃
\rashi{\rashiDH{פסל לך. }הראהו מחצב סנפירינון מתוך אהלו, ואמר לו הפסולת יהיה שלך, ומשם נתעשר משה הרבה׃ }\rashi{\rashiDH{פסל לך. }אתה שברת הראשונות, אתה פסל לך אחרות, משל למלך שהלך למדינת הים והניח ארוסתו עם השפחות, מתוך קלקול השפחות יצא עליה שם רע, עמד שושבינה וקרע כתובתה, אמר, אם יאמר המלך להורגה, אומר לו עדיין אינה אשתך, בדק המלך ומצא שלא היה הקלקול אלא מן השפחות, נתרצה לה, אמר לו שושבינה, כתוב לה כתובה אחרת שנקרעה הראשונה, אמר לו המלך, אתה קרעת אותה, אתה קנה לה נייר אחר ואני אכתוב לה בכתב ידי, כן המלך זה הקב״ה, השפחות אלו ערב רב, והשושבין זה משה, ארוסתו של הקב״ה אלו ישראל, לכך נאמר פסל לך׃ }}
{וַאֲמַר יְיָ לְמֹשֶׁה פְּסָל לָךְ תְּרֵין לוּחֵי אַבְנַיָּא כְּקַדְמָאֵי וְאֶכְתּוֹב עַל לוּחַיָּא יָת פִּתְגָמַיָּא דַּהֲווֹ עַל לוּחַיָּא קַדְמָאֵי דְּתַבַּרְתָּא׃}
{And the \lord\space said unto Moses: ‘Hew thee two tables of stone like unto the first; and I will write upon the tables the words that were on the first tables, which thou didst break.}{\Roman{chap}}
\threeverse{\arabic{verse}}%Ex.34:2
{וֶהְיֵ֥ה נָכ֖וֹן לַבֹּ֑קֶר וְעָלִ֤יתָ בַבֹּ֙קֶר֙ אֶל־הַ֣ר סִינַ֔י וְנִצַּבְתָּ֥ לִ֛י שָׁ֖ם עַל־רֹ֥אשׁ הָהָֽר׃
\rashi{\rashiDH{נכון. }מזומן׃ 
}}
{וִהְוִי זְמִין לְצַפְרָא וְתִסַּק בְּצַפְרָא לְטוּרָא דְּסִינַי וְתִתְעַתַּד קֳדָמַי תַּמָּן עַל רֵישׁ טוּרָא׃}
{And be ready by the morning, and come up in the morning unto mount Sinai, and present thyself there to Me on the top of the mount.}{\arabic{verse}}
\threeverse{\arabic{verse}}%Ex.34:3
{וְאִישׁ֙ לֹֽא־יַעֲלֶ֣ה עִמָּ֔ךְ וְגַם־אִ֥ישׁ אַל־יֵרָ֖א בְּכׇל־הָהָ֑ר גַּם־הַצֹּ֤אן וְהַבָּקָר֙ אַל־יִרְע֔וּ אֶל־מ֖וּל הָהָ֥ר הַהֽוּא׃
\rashi{\rashiDH{ואיש לא יעלה עמך. }הראשונות על ידי שהיו בתשואות וקולות וקהלות, שלטה בהן עין רעה, אין לך יפה מן הצניעות׃ }}
{וַאֲנָשׁ לָא יִסַּק עִמָּךְ וְאַף אֶנָשׁ לָא יִתַּחְזֵי בְּכָל טוּרָא אַף עָנָא וְתוֹרֵי לָא יִרְעוֹן לָקֳבֵיל טוּרָא הַהוּא׃}
{And no man shall come up with thee, neither let any man be seen throughout all the mount; neither let the flocks nor herds feed before that mount.’}{\arabic{verse}}
\threeverse{\arabic{verse}}%Ex.34:4
{וַיִּפְסֹ֡ל שְׁנֵֽי־לֻחֹ֨ת אֲבָנִ֜ים כָּרִאשֹׁנִ֗ים וַיַּשְׁכֵּ֨ם מֹשֶׁ֤ה בַבֹּ֙קֶר֙ וַיַּ֙עַל֙ אֶל־הַ֣ר סִינַ֔י כַּאֲשֶׁ֛ר צִוָּ֥ה יְהֹוָ֖ה אֹת֑וֹ וַיִּקַּ֣ח בְּיָד֔וֹ שְׁנֵ֖י לֻחֹ֥ת אֲבָנִֽים׃}
{וּפְסַל תְּרֵין לוּחֵי אַבְנַיָּא כְּקַדְמָאֵי וְאַקְדֵּים מֹשֶׁה בְּצַפְרָא וּסְלֵיק לְטוּרָא דְּסִינַי כְּמָא דְּפַקֵּיד יְיָ יָתֵיהּ וּנְסֵיב בִּידֵיהּ תְּרֵין לוּחֵי אַבְנַיָּא׃}
{And he hewed two tables of stone like unto the first; and Moses rose up early in the morning, and went up unto mount Sinai, as the \lord\space had commanded him, and took in his hand two tables of stone.}{\arabic{verse}}
\threeverse{\arabic{verse}}%Ex.34:5
{וַיֵּ֤רֶד יְהֹוָה֙ בֶּֽעָנָ֔ן וַיִּתְיַצֵּ֥ב עִמּ֖וֹ שָׁ֑ם וַיִּקְרָ֥א בְשֵׁ֖ם יְהֹוָֽה׃
\rashi{\rashiDH{ויקרא בשם ה׳. }מתרגמינן וּקְרָא בִשְׁמָא דה׳׃ 
}}
{וְאִתְגְּלִי יְיָ בַּעֲנָנָא וְאִתְעַתַּד עִמֵּיהּ תַּמָּן וּקְרָא בִשְׁמָא דַּייָ׃}
{And the \lord\space descended in the cloud, and stood with him there, and proclaimed the name of the \lord.}{\arabic{verse}}
\threeverse{\arabic{verse}}%Ex.34:6
{וַיַּעֲבֹ֨ר יְהֹוָ֥ה \pasek  עַל־פָּנָיו֮ וַיִּקְרָא֒ יְהֹוָ֣ה \pasek  יְהֹוָ֔ה אֵ֥ל רַח֖וּם וְחַנּ֑וּן אֶ֥רֶךְ אַפַּ֖יִם וְרַב־חֶ֥סֶד וֶאֱמֶֽת׃
\rashi{\rashiDH{יי׳ יי׳. }מדת רחמים היא, אחת קודם שיחטא, ואחת לאחר שיחטא וישוב (ראש השנה יז׃)׃ }\rashi{\rashiDH{אל. }אף זו מדת רחמים, וכן הוא אומר, אֵלִי אֵלִי לָמָה עֲזַבְתָּנִי (תהלים כב, ב), ואין לומר למדת הדין למה עזבתני, כך מצאתי במכילתא (שירה פ״ג)׃ }\rashi{\rashiDH{ארך אפים. }מאריך אפו, ואינו ממהר ליפרע, שמא יעשה תשובה׃ }\rashi{\rashiDH{ורב חסד. }לצריכים חסד, שאין להם זכיות כל כך׃ }\rashi{\rashiDH{ואמת. }לשלם שכר טוב לעושי רצונו׃ 
}}
{וְאַעְבַּר יְיָ שְׁכִינְתֵיהּ עַל אַפּוֹהִי וּקְרָא יְיָ יְיָ אֱלָהָא רַחְמָנָא וְחַנָּנָא מַרְחֵיק רְגַז וּמַסְגֵּי לְמֶעֱבַד טָבְוָן וּקְשׁוֹט׃}
{And the \lord\space passed by before him, and proclaimed: ‘The \lord, the \lord, God, merciful and gracious, long-suffering, and abundant in goodness and truth;}{\arabic{verse}}
\threeverse{\arabic{verse}}%Ex.34:7
{\large נֹ\normalsize צֵ֥ר\note{בספרי תימן נֹצֵ֥ר בנו״ן רגילה} חֶ֙סֶד֙ לָאֲלָפִ֔ים נֹשֵׂ֥א עָוֺ֛ן וָפֶ֖שַׁע וְחַטָּאָ֑ה וְנַקֵּה֙ לֹ֣א יְנַקֶּ֔ה פֹּקֵ֣ד \legarmeh  עֲוֺ֣ן אָב֗וֹת עַל־בָּנִים֙ וְעַל־בְּנֵ֣י בָנִ֔ים עַל־שִׁלֵּשִׁ֖ים וְעַל־רִבֵּעִֽים׃
\rashi{\rashiDH{נוצר חסד. }שהאדם עושה לפניו׃}\rashi{\rashiDH{לאלפים. }לשני אלפים דורות. עונות, אלו הזדונות. פשעים, אלו המרדים שאדם עושה להכעיס׃ }\rashi{\rashiDH{ונקה לא ינקה. }לפי פשוטו משמע, שאינו מוותר על העון לגמרי, אלא נפרע ממנו מעט מעט. ורבותינו דרשו (יומא פו.), מנקה הוא לשבים ולא ינקה לשאינן שבים׃ }\rashi{\rashiDH{פוקד עון אבות על בנים. }כשאוחזים מעשה אבותיהם בידיהם, שכבר פירש במקרא אחר לשונאי׃ }\rashi{\rashiDH{ועל רבעים. }דור רביעי, נמצאת מדה טובה מרובה על מדת פורענות אחת לחמש מאות, שבמדה טובה הוא אומר נוצר חסד לאלפים׃ }}
{נָטֵיר טִיבוּ לְאַלְפֵי דָרִין שָׁבֵיק לַעֲוָיָן וְלִמְרוֹד וּלְחוֹבִין סָלַח לִדְתָּיְבִין לְאוֹרָיְתֵיהּ וְלִדְלָא תָּיְבִין לָא מְזַכֵּי מַסְעַר חוֹבֵי אֲבָהָן עַל בְּנִין וְעַל בְּנֵי בְנִין מָרָדִין עַל דָּר תְּלִיתַאי וְעַל דָּר רְבִיעָאי׃}
{keeping mercy unto the thousandth generation, forgiving iniquity and transgression and sin; and that will by no means clear the guilty; visiting the iniquity of the fathers upon the children, and upon the children’s children, unto the third and unto the fourth generation.’}{\arabic{verse}}
\threeverse{\arabic{verse}}%Ex.34:8
{וַיְמַהֵ֖ר מֹשֶׁ֑ה וַיִּקֹּ֥ד אַ֖רְצָה וַיִּשְׁתָּֽחוּ׃
\rashi{\rashiDH{וימהר משה. }כשראה משה שכינה עוברת ושמע קול הקריאה, מיד וישתחו׃ }}
{וְאוֹחִי מֹשֶׁה וּכְרַע עַל אַרְעָא וּסְגֵיד׃}
{And Moses made haste, and bowed his head toward the earth, and worshipped.}{\arabic{verse}}
\threeverse{\arabic{verse}}%Ex.34:9
{וַיֹּ֡אמֶר אִם־נָא֩ מָצָ֨אתִי חֵ֤ן בְּעֵינֶ֙יךָ֙ אֲדֹנָ֔י יֵֽלֶךְ־נָ֥א אֲדֹנָ֖י בְּקִרְבֵּ֑נוּ כִּ֤י עַם־קְשֵׁה־עֹ֙רֶף֙ ה֔וּא וְסָלַחְתָּ֛ לַעֲוֺנֵ֥נוּ וּלְחַטָּאתֵ֖נוּ וּנְחַלְתָּֽנוּ׃
\rashi{\rashiDH{ילך נא ה׳ בקרבנו. }כמו שהבטחתנו, מאחר שאתה נושא עון, ואם עם קשה עורף הוא וימרו בך, ואמרת על זאת פן אכלך בדרך, אתה תסלח לעונינו וגו׳. יש כי במקום אם׃ 
}\rashi{\rashiDH{ונחלתנו. }ותתננו לך לנחלה (ס״א שתתן לנו נחלה) מיוחדת, זו היא בקשת ונפלינו אני ועמך, שלא תשרה שכינתך על האומות עובדי אלילים׃ }}
{וַאֲמַר אִם כְּעַן אַשְׁכַּחִית רַחֲמִין קֳדָמָךְ יְיָ תְּהָךְ כְּעַן שְׁכִינְתָא דַּייָ בֵּינַנָא אֲרֵי עַם קְשֵׁי קְדָל הוּא וְתִשְׁבּוֹק לְחוֹבַנָא וְלִחְטָאַנָא וְתַחְסְנִנַּנָא׃}
{And he said: ‘If now I have found grace in Thy sight, O Lord, let the Lord, I pray Thee, go in the midst of us; for it is a stiffnecked people; and pardon our iniquity and our sin, and take us for Thine inheritance.’}{\arabic{verse}}
\threeverse{\aliya{ששי}}%Ex.34:10
{וַיֹּ֗אמֶר הִנֵּ֣ה אָנֹכִי֮ כֹּרֵ֣ת בְּרִית֒ נֶ֤גֶד כׇּֽל־עַמְּךָ֙ אֶעֱשֶׂ֣ה נִפְלָאֹ֔ת אֲשֶׁ֛ר לֹֽא־נִבְרְא֥וּ בְכׇל־הָאָ֖רֶץ וּבְכׇל־הַגּוֹיִ֑ם וְרָאָ֣ה כׇל־הָ֠עָ֠ם אֲשֶׁר־אַתָּ֨ה בְקִרְבּ֜וֹ אֶת־מַעֲשֵׂ֤ה יְהֹוָה֙ כִּֽי־נוֹרָ֣א ה֔וּא אֲשֶׁ֥ר אֲנִ֖י עֹשֶׂ֥ה עִמָּֽךְ׃
\rashi{\rashiDH{כורת ברית. }על זאת׃}\rashi{\rashiDH{אעשה נפלאות. }לשון ונפלינו, שתהיו מובדלים בזו מכל האומות עובדי אלילים, שלא תשרה שכינתי עליהם׃ }}
{וַאֲמַר הָא אֲנָא גָּזַר קְיָם קֳדָם כָּל עַמָּךְ אַעֲבֵיד פְּרִישָׁן דְּלָא אִתְבְּרִיאוּ בְּכָל אַרְעָא וּבְכָל עַמְמַיָּא וְיִחְזֵי כָל עַמָּא דְּאַתְּ בֵּינֵיהוֹן יָת עוּבָדָא דַּייָ אֲרֵי דְּחִיל הוּא דַּאֲנָא עָבֵיד עִמָּךְ׃}
{And He said: ‘Behold, I make a covenant; before all thy people I will do marvels, such as have not been wrought in all the earth, nor in any nation; and all the people among which thou art shall see the work of the \lord\space that I am about to do with thee, that it is tremendous.}{\arabic{verse}}
\threeverse{\arabic{verse}}%Ex.34:11
{שְׁמׇ֨ר־לְךָ֔ אֵ֛ת אֲשֶׁ֥ר אָנֹכִ֖י מְצַוְּךָ֣ הַיּ֑וֹם הִנְנִ֧י גֹרֵ֣שׁ מִפָּנֶ֗יךָ אֶת־הָאֱמֹרִי֙ וְהַֽכְּנַעֲנִ֔י וְהַחִתִּי֙ וְהַפְּרִזִּ֔י וְהַחִוִּ֖י וְהַיְבוּסִֽי׃
\rashi{\rashiDH{את האמורי וגו׳. }ו׳ אומות יש כאן, כי הגרגשי עמד ופנה מפניהם׃ }}
{טַר לָךְ יָת דַּאֲנָא מְפַקְּדָךְ יוֹמָא דֵין הָאֲנָא מְתָרֵיךְ מִן קֳדָמָךְ יָת אֱמוֹרָאֵי וּכְנַעֲנָאֵי וְחִתָּאֵי וּפְרִזָּאֵי וְחִוָּאֵי וִיבוּסָאֵי׃}
{Observe thou that which I am commanding thee this day; behold, I am driving out before thee the Amorite, and the Canaanite, and the Hittite, and the Perizzite, and the Hivite, and the Jebusite.}{\arabic{verse}}
\threeverse{\arabic{verse}}%Ex.34:12
{הִשָּׁ֣מֶר לְךָ֗ פֶּן־תִּכְרֹ֤ת בְּרִית֙ לְיוֹשֵׁ֣ב הָאָ֔רֶץ אֲשֶׁ֥ר אַתָּ֖ה בָּ֣א עָלֶ֑יהָ פֶּן־יִהְיֶ֥ה לְמוֹקֵ֖שׁ בְּקִרְבֶּֽךָ׃}
{אִסְתְּמַר לָךְ דִּלְמָא תִגְזַר קְיָם לְיָתֵיב אַרְעָא דְּאַתְּ עָלֵיל עֲלַהּ דִּלְמָא יְהֵי לְתַקְלָא בֵּינָךְ׃}
{Take heed to thyself, lest thou make a covenant with the inhabitants of the land whither thou goest, lest they be for a snare in the midst of thee.}{\arabic{verse}}
\threeverse{\arabic{verse}}%Ex.34:13
{כִּ֤י אֶת־מִזְבְּחֹתָם֙ תִּתֹּצ֔וּן וְאֶת־מַצֵּבֹתָ֖ם תְּשַׁבֵּר֑וּן וְאֶת־אֲשֵׁרָ֖יו תִּכְרֹתֽוּן׃
\rashi{\rashiDH{אשריו. }הוא אילן שעובדים אותו׃}}
{אֲרֵי יָת אֵיגוֹרֵיהוֹן תְּתָרְעוּן וְיָת קָמָתְהוֹן תְּתַבְּרוּן וְיָת אֲשֵׁירֵיהוֹן תְּקָצְצוּן׃}
{But ye shall break down their altars, and dash in pieces their pillars, and ye shall cut down their Asherim.}{\arabic{verse}}
\threeverse{\arabic{verse}}%Ex.34:14
{כִּ֛י לֹ֥א תִֽשְׁתַּחֲוֶ֖ה לְאֵ֣ל אַחֵ֑\large ר\normalsize  כִּ֤י יְהֹוָה֙ קַנָּ֣א שְׁמ֔וֹ אֵ֥ל קַנָּ֖א הֽוּא׃
\rashi{\rashiDH{קנא שמו. }מקנא להפרע ואינו מוותר. וזהו כל לשון קנאה, אוחז בנצחונו ופורע מעוזביו׃ 
}}
{אֲרֵי לָא תִסְגּוֹד לְטָעֲוָת עַמְמַיָּא אֲרֵי יְיָ קַנָּא שְׁמֵיהּ אֵל קַנָּא הוּא׃}
{For thou shalt bow down to no other god; for the \lord, whose name is Jealous, is a jealous God;}{\arabic{verse}}
\threeverse{\arabic{verse}}%Ex.34:15
{פֶּן־תִּכְרֹ֥ת בְּרִ֖ית לְיוֹשֵׁ֣ב הָאָ֑רֶץ וְזָנ֣וּ \legarmeh  אַחֲרֵ֣י אֱלֹֽהֵיהֶ֗ם וְזָבְחוּ֙ לֵאלֹ֣הֵיהֶ֔ם וְקָרָ֣א לְךָ֔ וְאָכַלְתָּ֖ מִזִּבְחֽוֹ׃
\rashi{\rashiDH{ואכלת מזבחו. }כסבור אתה שאין עונש באכילתו, ואני מעלה עליך כמודה בעבודתו, שמתוך כך אתה בא ולוקח מבנותיו לבניך (עבודה זרה ח.)׃ }}
{דִּלְמָא תִגְזַר קְיָם לְיָתֵיב אַרְעָא וְיִטְעוֹן בָּתַר טָעֲוָתְהוֹן וִידַבְּחוּן לְטָעֲוָתְהוֹן וְיִקְרוֹן לָךְ וְתֵיכוֹל מִדִּבְחֵיהוֹן׃}
{lest thou make a covenant with the inhabitants of the land, and they go astray after their gods, and do sacrifice unto their gods, and they call thee, and thou eat of their sacrifice;}{\arabic{verse}}
\threeverse{\arabic{verse}}%Ex.34:16
{וְלָקַחְתָּ֥ מִבְּנֹתָ֖יו לְבָנֶ֑יךָ וְזָנ֣וּ בְנֹתָ֗יו אַחֲרֵי֙ אֱלֹ֣הֵיהֶ֔ן וְהִזְנוּ֙ אֶת־בָּנֶ֔יךָ אַחֲרֵ֖י אֱלֹהֵיהֶֽן׃}
{וְתִסַּב מִבְּנָתְהוֹן לִבְנָךְ וְיִטְעוֹן בְּנָתְהוֹן בָּתַר טָעֲוָתְהוֹן וְיַטְעֲיָן יָת בְּנָךְ בָּתַר טָעֲוָתְהוֹן׃}
{and thou take of their daughters unto thy sons, and their daughters go astray after their gods, and make thy sons go astray after their gods.}{\arabic{verse}}
\threeverse{\arabic{verse}}%Ex.34:17
{אֱלֹהֵ֥י מַסֵּכָ֖ה לֹ֥א תַעֲשֶׂה־לָּֽךְ׃}
{דַּחְלָן דְּמַתְּכָא לָא תַעֲבֵיד לָךְ׃}
{Thou shalt make thee no molten gods.}{\arabic{verse}}
\threeverse{\arabic{verse}}%Ex.34:18
{אֶת־חַ֣ג הַמַּצּוֹת֮ תִּשְׁמֹר֒ שִׁבְעַ֨ת יָמִ֜ים תֹּאכַ֤ל מַצּוֹת֙ אֲשֶׁ֣ר צִוִּיתִ֔ךָ לְמוֹעֵ֖ד חֹ֣דֶשׁ הָאָבִ֑יב כִּ֚י בְּחֹ֣דֶשׁ הָֽאָבִ֔יב יָצָ֖אתָ מִמִּצְרָֽיִם׃
\rashi{\rashiDH{חדש האביב. }חדש הַבִּכּוּר, שהתבואה מתבכרת בבישולה׃ }}
{יָת חַגָּא דְּפַטִּירַיָּא תִּטַּר שִׁבְעָא יוֹמִין תֵּיכוֹל פַּטִּירָא דְּפַקֵּידְתָּךְ לִזְמַן יַרְחָא דַּאֲבִיבָא אֲרֵי בְּיַרְחָא דַּאֲבִיבָא נְפַקְתָּא מִמִּצְרָיִם׃}
{The feast of unleavened bread shalt thou keep. Seven days thou shalt eat unleavened bread, as I commanded thee, at the time appointed in the month Abib, for in the month Abib thou camest out from Egypt.}{\arabic{verse}}
\threeverse{\arabic{verse}}%Ex.34:19
{כׇּל־פֶּ֥טֶר רֶ֖חֶם לִ֑י וְכׇֽל־מִקְנְךָ֙ תִּזָּכָ֔ר פֶּ֖טֶר שׁ֥וֹר וָשֶֽׂה׃
\rashi{\rashiDH{כל פטר רחם לי. }באדם׃}\rashi{\rashiDH{וכל מקנך תזכר וגו׳. }וכל מקנך אשר תזכר בפטר שור ושה, אשר יפטור זכר את רחמה. }\rashi{\rashiDH{פטר. }לשון פתיחה, וכן פֹּוטֵר מַיִם רֵאשִׁית מָדֹון (משלי יז, יד). תי״ו של תזכר לשון נקבה היא, מוסב על היולדת׃ }}
{כָּל פָּתַח וַלְדָּא דִּילִי הוּא וְכָל בְּעִירָךְ תַּקְדֵּישׁ דִּכְרִין בְּכוֹר תּוֹר וְאִמַּר׃}
{All that openeth the womb is Mine; and of all thy cattle thou shalt sanctify the males, the firstlings of ox and sheep.}{\arabic{verse}}
\threeverse{\arabic{verse}}%Ex.34:20
{וּפֶ֤טֶר חֲמוֹר֙ תִּפְדֶּ֣ה בְשֶׂ֔ה וְאִם־לֹ֥א תִפְדֶּ֖ה וַעֲרַפְתּ֑וֹ כֹּ֣ל בְּכ֤וֹר בָּנֶ֙יךָ֙ תִּפְדֶּ֔ה וְלֹֽא־יֵרָא֥וּ פָנַ֖י רֵיקָֽם׃
\rashi{\rashiDH{ופטר חמור. }ולא שאר בהמה טמאה׃ 
}\rashi{\rashiDH{תפדה בשה. }נותן שה לכהן, והוא חולין ביד כהן, ופטר חמור מותר בעבודה לבעלים׃ }\rashi{\rashiDH{וערפתו. }עורפו בקופיץ, הוא הפסיד ממון כהן, לפיכך יופסד ממונו׃ }\rashi{\rashiDH{כל בכור בניך תפדה. }חמשה סלעים פדיונו קצוב, שנאמר וּפְדוּיָו מִבֶּן חֹדֶשׁ תִּפְדֶּה (במדבר יח, טז)׃ }\rashi{\rashiDH{ולא יראו פני ריקם. }לפי פשוטו של מקרא, דבר בפני עצמו הוא, ואינו מוסב על הבכור, שאין במצות בכור ראיית פנים, אלא אזהרה אחרת היא, וכשתעלו לרגל לראות, לא יראו פני ריקם, מצוה עליכם להביא עולת ראיית פנים. ולפי מדרש ברייתא, מקרא יתר הוא, ומופנה לגזרה שוה, ללמד על הענקתו של עבד עברי שהוא חמשה סלעים מכל מין ומין, כפדיון בכור, במסכת קדושין (יז.)׃ 
}}
{וּבוּכְרָא דִּחְמָרָא תִּפְרוּק בְּאִמְּרָא וְאִם לָא תִפְרוּק וְתִקְפֵיהּ כֹּל בּוּכְרָא דִּבְנָךְ תִּפְרוּק וְלָא יִתַּחְזוֹן קֳדָמַי רֵיקָנִין׃}
{And the firstling of an ass thou shalt redeem with a lamb; and if thou wilt not redeem it, then thou shalt break its neck. All the first-born of thy sons thou shalt redeem. And none shall appear before Me empty.}{\arabic{verse}}
\threeverse{\arabic{verse}}%Ex.34:21
{שֵׁ֤שֶׁת יָמִים֙ תַּעֲבֹ֔ד וּבַיּ֥וֹם הַשְּׁבִיעִ֖י תִּשְׁבֹּ֑ת בֶּחָרִ֥ישׁ וּבַקָּצִ֖יר תִּשְׁבֹּֽת׃
\rashi{\rashiDH{בחריש ובקציר תשבות. }למה נזכר חריש וקציר, יש מרבותינו אומרים (ראש השנה ט.), על חריש של ערב שביעית הנכנס לשביעית וקציר של שביעית היוצא למוצאי שביעית, ללמדך שמוסיפין מחול על הקדש, וכך משמעו, ששת ימים תעבוד וביום השביעי תשבות, ועבודת ו׳ הימים שהתרתי לך, יש שנה שהחריש והקציר אסור, ואין צריך לומר חריש וקציר של שביעית, שהרי כבר נאמר שָׂדְךָ לֹא תִזְרָע וגו׳ (ויקרא כה, ד). ויש אומרים שאינו מדבר אלא בשבת, וחריש וקציר שהוזכר בו לומר לך, מה חריש רשות אף קציר רשות, יצא קציר העומר שהוא מצוה, ודוחה את השבת׃ 
}}
{שִׁתָּא יוֹמִין תִּפְלַח וּבְיוֹמָא שְׁבִיעָאָה תְּנוּחַ בִּזְרוּעָא וּבִחְצָדָא תְּנוּחַ׃}
{Six days thou shalt work, but on the seventh day thou shalt rest; in plowing time and in harvest thou shalt rest.}{\arabic{verse}}
\threeverse{\arabic{verse}}%Ex.34:22
{וְחַ֤ג שָׁבֻעֹת֙ תַּעֲשֶׂ֣ה לְךָ֔ בִּכּוּרֵ֖י קְצִ֣יר חִטִּ֑ים וְחַג֙ הָֽאָסִ֔יף תְּקוּפַ֖ת הַשָּׁנָֽה׃
\rashi{\rashiDH{בכורי קציר חטים. }שאתה מביא בו שתי הלחם מן החטים.}\rashi{\rashiDH{בכורם }שהיא מנחה ראשונה הבאה מן החדש של חטים למקדש, כי מנחת העומר הבאה בפסח, מן השעורים היא׃ }\rashi{\rashiDH{וחג האסיף. }בזמן שאתה אוסף תבואתך מן השדה לבית. אסיפה זו לשון הכנסה לבית, כמו וַאֲסַפְתֹּו אֶל תֹּוךְ בֵּיתֶךָ (דברים כב, ב)׃ }\rashi{\rashiDH{תקופת השנה. }שהיא בחזרת השנה, בתחלת השנה הבאה׃ }\rashi{\rashiDH{תקופת. }לשון מסיבה והקפה׃}}
{וְחַגָּא דְּשָׁבוּעַיָּא תַּעֲבֵיד לָךְ בִּכּוּרֵי חֲצָד חִטִּין וְחַגָּא דִּכְנָשָׁא בְּמִפְּקַהּ דְּשַׁתָּא׃}
{And thou shalt observe the feast of weeks, even of the first-fruits of wheat harvest, and the feast of ingathering at the turn of the year.}{\arabic{verse}}
\threeverse{\arabic{verse}}%Ex.34:23
{שָׁלֹ֥שׁ פְּעָמִ֖ים בַּשָּׁנָ֑ה יֵרָאֶה֙ כׇּל־זְכ֣וּרְךָ֔ אֶת־פְּנֵ֛י הָֽאָדֹ֥ן \pasek  יְהֹוָ֖ה אֱלֹהֵ֥י יִשְׂרָאֵֽל׃
\rashi{\rashiDH{כל זכורך. }כל הזכרים שבך. הרבה מצות בתורה נאמרו ונכפלו, ויש מהם שלש פעמים וארבע, לחייב ולענוש על מנין לאוין שבהם, ועל מנין עשה שבהם׃ }}
{תְּלָת זִמְנִין בְּשַׁתָּא יִתַּחְזוֹן כָּל דְּכוּרָךְ קֳדָם רִבּוֹן עָלְמָא יְיָ אֱלָהָא דְּיִשְׂרָאֵל׃}
{Three times in the year shall all thy males appear before the Lord \textsc{God}, the God of Israel.}{\arabic{verse}}
\threeverse{\arabic{verse}}%Ex.34:24
{כִּֽי־אוֹרִ֤ישׁ גּוֹיִם֙ מִפָּנֶ֔יךָ וְהִרְחַבְתִּ֖י אֶת־גְּבֻלֶ֑ךָ וְלֹא־יַחְמֹ֥ד אִישׁ֙ אֶֽת־אַרְצְךָ֔ בַּעֲלֹֽתְךָ֗ לֵרָאוֹת֙ אֶת־פְּנֵי֙ יְהֹוָ֣ה אֱלֹהֶ֔יךָ שָׁלֹ֥שׁ פְּעָמִ֖ים בַּשָּׁנָֽה׃
\rashi{\rashiDH{אוריש. }כתרגומו אֲתָרֵךְ, וכן הָחֵל רָשׁ (דברים ב, לא), וכן וַיֹּורֶשׁ אֶת הָאֱמֹרִי (במדבר כא, לב), לשון גירושין׃ }\rashi{\rashiDH{והרחבתי את גבלך. }ואתה רחוק מבית הבחירה, ואינך יכול לראות לפני תמיד, לכך אני קובע לך שלשה רגלים הללו׃ }}
{אֲרֵי אֲתָרֵיךְ עַמְמִין מִן קֳדָמָךְ וְאַפְתֵּי יָת תְּחוּמָךְ וְלָא יְחַמֵּיד אֶנָשׁ יָת אַרְעָךְ בְּמִסְּקָךְ לְאִתַּחְזָאָה קֳדָם יְיָ אֱלָהָךְ תְּלָת זִמְנִין בְּשַׁתָּא׃}
{For I will cast out nations before thee, and enlarge thy borders; neither shall any man covet thy land, when thou goest up to appear before the \lord\space thy God three times in the year.}{\arabic{verse}}
\threeverse{\arabic{verse}}%Ex.34:25
{לֹֽא־תִשְׁחַ֥ט עַל־חָמֵ֖ץ דַּם־זִבְחִ֑י וְלֹא־יָלִ֣ין לַבֹּ֔קֶר זֶ֖בַח חַ֥ג הַפָּֽסַח׃
\rashi{\rashiDH{לא תשחט וגו׳. }לא תשחט את הפסח ועדיין חמץ קיים, אזהרה לשוחט, או לזורק, או לאחד מבני חבורה (פסחים סג.)׃ }\rashi{\rashiDH{ולא ילין. }כתרגומו, אין לינה מועלת בראש המזבח, ואין לינה אלא בעמוד השחר׃ }\rashi{\rashiDH{זבח חג הפסח. }אמוריו, ומכאן אתה למד לכל הקטר חלבים ואברים׃ }}
{לָא תִכּוֹס עַל חֲמִיעַ דַּם פִּסְחִי וְלָא יְבִיתוּן לְצַפְרָא תַּרְבֵּי נִכְסַת חַגָּא דְּפִסְחָא׃}
{Thou shalt not offer the blood of My sacrifice with leavened bread; neither shall the sacrifice of the feast of the passover be left unto the morning.}{\arabic{verse}}
\threeverse{\arabic{verse}}%Ex.34:26
{רֵאשִׁ֗ית בִּכּוּרֵי֙ אַדְמָ֣תְךָ֔ תָּבִ֕יא בֵּ֖ית יְהֹוָ֣ה אֱלֹהֶ֑יךָ לֹא־תְבַשֵּׁ֥ל גְּדִ֖י בַּחֲלֵ֥ב אִמּֽוֹ׃ \petucha 
\rashi{\rashiDH{ראשית בכורי אדמתך. }משבעת המינין האמורים בשבח ארצך, אֶרֶץ חִטָּה וּשְׂעֹרָה וְגֶפֶן וגו׳ (דברים ח, ח), וּדְבָשׁ, הוא דבש תמרים׃ }\rashi{\rashiDH{לא תבשל גדי. }אזהרה לבשר וחלב, ושלשה פעמים כתוב בתורה, אחד לאכילה, ואחד להנאה, ואחד לאיסור בישול (חולין קטו׃)׃ }\rashi{\rashiDH{גדי. }כל ולד רך במשמע, ואף עגל וכבש, ממה שהוצרך לפרש בכמה מקומות גדי עזים, למדת שגדי סתם כל יונקים במשמע׃ }\rashi{\rashiDH{בחלב אמו. }פרט לעוף, שאין לו חלב אם, שאין איסורו מן התורה אלא מדברי סופרים׃ }}
{רֵישׁ בִּכּוּרֵי אַרְעָךְ תַּיְתֵי לְבֵית מַקְדְּשָׁא דַּייָ אֱלָהָךְ לָא תֵיכְלוּן בְּשַׂר בַּחֲלַב׃}
{The choicest first-fruits of thy land thou shalt bring unto the house of the \lord\space thy God. Thou shalt not seethe a kid in its mother’s milk.’}{\arabic{verse}}
\threeverse{\aliya{שביעי}}%Ex.34:27
{וַיֹּ֤אמֶר יְהֹוָה֙ אֶל־מֹשֶׁ֔ה כְּתׇב־לְךָ֖ אֶת־הַדְּבָרִ֣ים הָאֵ֑לֶּה כִּ֞י עַל־פִּ֣י \legarmeh  הַדְּבָרִ֣ים הָאֵ֗לֶּה כָּרַ֧תִּי אִתְּךָ֛ בְּרִ֖ית וְאֶת־יִשְׂרָאֵֽל׃
\rashi{\rashiDH{את הדברים האלה. }ולא אתה רשאי לכתוב תורה שבעל פה׃}}
{וַאֲמַר יְיָ לְמֹשֶׁה כְּתוֹב לָךְ יָת פִּתְגָמַיָּא הָאִלֵּין אֲרֵי עַל מֵימַר פִּתְגָמַיָּא הָאִלֵּין גְּזַרִית עִמָּךְ קְיָם וְעִם יִשְׂרָאֵל׃}
{And the \lord\space said unto Moses: ‘Write thou these words, for after the tenor of these words I have made a covenant with thee and with Israel.’}{\arabic{verse}}
\threeverse{\arabic{verse}}%Ex.34:28
{וַֽיְהִי־שָׁ֣ם עִם־יְהֹוָ֗ה אַרְבָּעִ֥ים יוֹם֙ וְאַרְבָּעִ֣ים לַ֔יְלָה לֶ֚חֶם לֹ֣א אָכַ֔ל וּמַ֖יִם לֹ֣א שָׁתָ֑ה וַיִּכְתֹּ֣ב עַל־הַלֻּחֹ֗ת אֵ֚ת דִּבְרֵ֣י הַבְּרִ֔ית עֲשֶׂ֖רֶת הַדְּבָרִֽים׃}
{וַהֲוָה תַּמָּן קֳדָם יְיָ אַרְבְּעִין יְמָמִין וְאַרְבְּעִין לֵילָוָן לַחְמָא לָא אֲכַל וּמַיָּא לָא שְׁתִי וּכְתַב עַל לוּחַיָּא יָת פִּתְגָמֵי קְיָמָא עֶשְׂרָא פִּתְגָמִין׃}
{And he was there with the \lord\space forty days and forty nights; he did neither eat bread, nor drink water. And he wrote upon the tables the words of the covenant, the ten words.}{\arabic{verse}}
\threeverse{\arabic{verse}}%Ex.34:29
{וַיְהִ֗י בְּרֶ֤דֶת מֹשֶׁה֙ מֵהַ֣ר סִינַ֔י וּשְׁנֵ֨י לֻחֹ֤ת הָֽעֵדֻת֙ בְּיַד־מֹשֶׁ֔ה בְּרִדְתּ֖וֹ מִן־הָהָ֑ר וּמֹשֶׁ֣ה לֹֽא־יָדַ֗ע כִּ֥י קָרַ֛ן ע֥וֹר פָּנָ֖יו בְּדַבְּר֥וֹ אִתּֽוֹ׃
\rashi{\rashiDH{ויהי ברדת משה. }כשהביא לוחות אחרונות ביום הכפורים׃ 
}\rashi{\rashiDH{כי קרן. }לשון קרנים, שהאור מבהיק ובולט כמין קרן. ומהיכן זכה משה לקרני ההוד, רבותינו אמרו מן המערה, שנתן הקב״ה ידו על פניו, שנאמר וְשַׂכֹּתִי כַפִּי (שמות לג, כב)׃ }}
{וַהֲוָה כַּד נְחַת מֹשֶׁה מִטּוּרָא דְּסִינַי וּתְרֵין לוּחֵי דְּסָהֲדוּתָא בִּידָא דְּמֹשֶׁה בְּמֵיחֲתֵיהּ מִן טוּרָא וּמֹשֶׁה לָא יְדַע אֲרֵי סְגִי זִיו יְקָרָא דְּאַפּוֹהִי בְּמַלָּלוּתֵיהּ עִמֵּיהּ׃}
{And it came to pass, when Moses came down from mount Sinai with the two tables of the testimony in Moses’ hand, when he came down from the mount, that Moses knew not that the skin of his face sent forth abeams while He talked with him.}{\arabic{verse}}
\threeverse{\arabic{verse}}%Ex.34:30
{וַיַּ֨רְא אַהֲרֹ֜ן וְכׇל־בְּנֵ֤י יִשְׂרָאֵל֙ אֶת־מֹשֶׁ֔ה וְהִנֵּ֥ה קָרַ֖ן ע֣וֹר פָּנָ֑יו וַיִּֽירְא֖וּ מִגֶּ֥שֶׁת אֵלָֽיו׃
\rashi{\rashiDH{וייראו מגשת אליו. }בא וראה כמה גדולה כחה של עבירה, שעד שלא פשטו ידיהם בעבירה מהו אומר, וּמַרְאֵה כְּבֹוד ה׳ כְּאֵשׁ אֹכֶלֶת בְּראשׁ הָהָר לְעֵינֵי בְּנֵי יִשְׂרָאֵל (שמות כד, יז), ולא יראים ולא מזדעזעים, ומשעשו את העגל, אף מקרני הודו של משה היו מרתיעים ומזדעזעים׃ }}
{וַחֲזָא אַהֲרֹן וְכָל בְּנֵי יִשְׂרָאֵל יָת מֹשֶׁה וְהָא סְגִי זִיו יְקָרָא דְּאַפּוֹהִי וּדְחִילוּ מִלְּאִתְקָרָבָא לְוָתֵיהּ׃}
{And when Aaron and all the children of Israel saw Moses, behold, the skin of his face sent forth beams; and they were afraid to come nigh him.}{\arabic{verse}}
\threeverse{\arabic{verse}}%Ex.34:31
{וַיִּקְרָ֤א אֲלֵהֶם֙ מֹשֶׁ֔ה וַיָּשֻׁ֧בוּ אֵלָ֛יו אַהֲרֹ֥ן וְכׇל־הַנְּשִׂאִ֖ים בָּעֵדָ֑ה וַיְדַבֵּ֥ר מֹשֶׁ֖ה אֲלֵהֶֽם׃
\rashi{\rashiDH{הנשאים בעדה. }כמו נשיאי העדה׃}\rashi{\rashiDH{וידבר משה אליהם. }שליחותו של מקום, ולשון הווה הוא כל הענין הזה׃ }}
{וּקְרָא לְהוֹן מֹשֶׁה וְתָבוּ לְוָתֵיהּ אַהֲרֹן וְכָל רַבְרְבַיָּא בִּכְנִשְׁתָּא וּמַלֵּיל מֹשֶׁה עִמְּהוֹן׃}
{And Moses called unto them; and Aaron and all the rulers of the congregation returned unto him; and Moses spoke to them.}{\arabic{verse}}
\threeverse{\arabic{verse}}%Ex.34:32
{וְאַחֲרֵי־כֵ֥ן נִגְּשׁ֖וּ כׇּל־בְּנֵ֣י יִשְׂרָאֵ֑ל וַיְצַוֵּ֕ם אֵת֩ כׇּל־אֲשֶׁ֨ר דִּבֶּ֧ר יְהֹוָ֛ה אִתּ֖וֹ בְּהַ֥ר סִינָֽי׃
\rashi{\rashiDH{ואחרי כן נגשו. }אחר שלמד לזקנים, חוזר ומלמד הפרשה או ההלכה לישראל. תנו רבנן, כיצד סדר המשנה, משה היה לומד מפי הגבורה, נכנס אהרן, שנה לו משה פרקו, נסתלק אהרן וישב לו לשמאל משה, נכנסו בניו, שנה להם משה פרקם, נסתלקו הם, ישב אלעזר לימין משה ואיתמר לשמאל אהרן, נכנסו זקנים, שנה להם משה פרקם, נסתלקו זקנים ישבו לצדדין, נכנסו כל העם, שנה להם משה פרקם, נמצא ביד כל העם א׳, ביד הזקנים ב׳, ביד בני אהרן שלשה, ביד אהרן ארבעה וכו׳, כדאיתא בעירובין (נד׃)׃ 
}}
{וּבָתַר כֵּן אִתְקָרַבוּ כָּל בְּנֵי יִשְׂרָאֵל וּפַקֵּידִנּוּן יָת כָּל דְּמַלֵּיל יְיָ עִמֵּיהּ בְּטוּרָא דְּסִינָי׃}
{And afterward all the children of Israel came nigh, and he gave them in commandment all that the \lord\space had spoken with him in mount Sinai.}{\arabic{verse}}
\threeverse{\aliya{מפטיר}}%Ex.34:33
{וַיְכַ֣ל מֹשֶׁ֔ה מִדַּבֵּ֖ר אִתָּ֑ם וַיִּתֵּ֥ן עַל־פָּנָ֖יו מַסְוֶֽה׃
\rashi{\rashiDH{ויתן על פניו מסוה. }כתרגומו בֵּית אַפֵּי, לשון ארמי הוא בגמרא סְוֵי לִבָּא (כתובות סב׃), ועוד בכתובות (ס.), הוה קא מסוה לאפה, לשון הבטה, היה מסתכל בה, אף כאן מסוה, בגד הניתן כנגד הפרצוף ובית העינים, ולכבוד קרני ההוד שלא יזונו הכל מהם, היה נותן המסוה כנגדן, ונוטלו בשעה שהיה מדבר עם ישראל, ובשעה שהמקום נדבר עמו עד צאתו, ובצאתו יצא בלא מסוה׃ }}
{וְשֵׁיצֵי מֹשֶׁה מִלְּמַלָּלָא עִמְּהוֹן וִיהַב עַל אַפּוֹהִי בֵּית אַפֵּי׃}
{And when Moses had done speaking with them, he put a veil on his face.}{\arabic{verse}}
\threeverse{\arabic{verse}}%Ex.34:34
{וּבְבֹ֨א מֹשֶׁ֜ה לִפְנֵ֤י יְהֹוָה֙ לְדַבֵּ֣ר אִתּ֔וֹ יָסִ֥יר אֶת־הַמַּסְוֶ֖ה עַד־צֵאת֑וֹ וְיָצָ֗א וְדִבֶּר֙ אֶל־בְּנֵ֣י יִשְׂרָאֵ֔ל אֵ֖ת אֲשֶׁ֥ר יְצֻוֶּֽה׃
\rashi{\rashiDH{ודבר אל בני ישראל. }וראו קרני ההוד בפניו, וכשהוא מסתלק מהם׃ 
}}
{וְכַד עָלֵיל מֹשֶׁה לִקְדָם יְיָ לְמַלָּלָא עִמֵּיהּ מַעְדֵּי יָת בֵּית אַפֵּי עַד מִפְּקֵיהּ וְנָפֵיק וּמְמַלֵּיל עִם בְּנֵי יִשְׂרָאֵל יָת דְּמִתְפַּקַּד׃}
{But when Moses went in before the \lord\space that He might speak with him, he took the veil off, until he came out; and he came out; and spoke unto the children of Israel that which he was commanded.}{\arabic{verse}}
\threeverse{\arabic{verse}}%Ex.34:35
{וְרָא֤וּ בְנֵֽי־יִשְׂרָאֵל֙ אֶת־פְּנֵ֣י מֹשֶׁ֔ה כִּ֣י קָרַ֔ן ע֖וֹר פְּנֵ֣י מֹשֶׁ֑ה וְהֵשִׁ֨יב מֹשֶׁ֤ה אֶת־הַמַּסְוֶה֙ עַל־פָּנָ֔יו עַד־בֹּא֖וֹ לְדַבֵּ֥ר אִתּֽוֹ׃ \setuma         
\rashi{\rashiDH{והשיב משה את המסוה על פניו עד בואו לדבר אתו. }וכשבא לדבר אתו נוטלו מעל פניו׃}}
{וְחָזַן בְּנֵי יִשְׂרָאֵל יָת אַפֵּי מֹשֶׁה אֲרֵי סְגִי זִיו יְקָרָא דְּאַפֵּי מֹשֶׁה וּמְתִיב מֹשֶׁה יָת בֵּית אַפֵּי עַל אַפּוֹהִי עַד דְּעָלֵיל לְמַלָּלָא עִמֵּיהּ׃}
{And the children of Israel saw the face of Moses, that the skin of Moses’ face sent forth beams; and Moses put the veil back upon his face, until he went in to speak with Him.}{\arabic{verse}}
\newperek
\newparsha{ויקהל}
\threeverse{\aliya{ויקהל}}%Ex.35:1
{וַיַּקְהֵ֣ל מֹשֶׁ֗ה אֶֽת־כׇּל־עֲדַ֛ת בְּנֵ֥י יִשְׂרָאֵ֖ל וַיֹּ֣אמֶר אֲלֵהֶ֑ם אֵ֚לֶּה הַדְּבָרִ֔ים אֲשֶׁר־צִוָּ֥ה יְהֹוָ֖ה לַעֲשֹׂ֥ת אֹתָֽם׃
\rashi{\rashiDH{ויקהל משה. }למחרת יום הכפורים כשירד מן ההר, והוא לשון הפעיל, שאינו אוסף אנשים בידים, אלא הן נאספים על פי דבורו, ותרגומו וְאַכְנֵשׁ׃ }}
{וּכְנַשׁ מֹשֶׁה יָת כָּל כְּנִשְׁתָּא דִּבְנֵי יִשְׂרָאֵל וַאֲמַר לְהוֹן אִלֵּין פִּתְגָמַיָּא דְּפַקֵּיד יְיָ לְמֶעֱבַד יָתְהוֹן׃}
{And Moses assembled all the congregation of the children of Israel, and said unto them: ‘These are the words which the \lord\space hath commanded, that ye should do them.}{\Roman{chap}}
\threeverse{\arabic{verse}}%Ex.35:2
{שֵׁ֣שֶׁת יָמִים֮ תֵּעָשֶׂ֣ה מְלָאכָה֒ וּבַיּ֣וֹם הַשְּׁבִיעִ֗י יִהְיֶ֨ה לָכֶ֥ם קֹ֛דֶשׁ שַׁבַּ֥ת שַׁבָּת֖וֹן לַיהֹוָ֑ה כׇּל־הָעֹשֶׂ֥ה ב֛וֹ מְלָאכָ֖ה יוּמָֽת׃
\rashi{\rashiDH{ששת ימים. }הקדים להם אזהרת שבת לצווי מלאכת המשכן, לומר, שאינו דוחה את השבת׃ }}
{שִׁתָּא יוֹמִין תִּתְעֲבֵיד עֲבִידְתָא וּבְיוֹמָא שְׁבִיעָאָה יְהֵי לְכוֹן קוּדְשָׁא שַׁבָּא שַׁבְּתָא קֳדָם יְיָ כָּל דְּיַעֲבֵיד בֵּיהּ עֲבִידְתָא יִתְקְטִיל׃}
{Six days shall work be done, but on the seventh day there shall be to you a holy day, a sabbath of solemn rest to the \lord; whosoever doeth any work therein shall be put to death.}{\arabic{verse}}
\threeverse{\arabic{verse}}%Ex.35:3
{לֹא־תְבַעֲר֣וּ אֵ֔שׁ בְּכֹ֖ל מֹשְׁבֹֽתֵיכֶ֑ם בְּי֖וֹם הַשַּׁבָּֽת׃ \petucha 
\rashi{\rashiDH{לא תבערו אש. }יש מרבותינו אומרים, הבערה ללאו יצאת, ויש אומרים לחלק יצאת (סנהדרין לה׃, יבמות ו׃)׃ }}
{לָא תְבַעֲרוּן אִישָׁתָא בְּכֹל מוֹתְבָנֵיכוֹן בְּיוֹמָא דְּשַׁבְּתָא׃}
{Ye shall kindle no fire throughout your habitations upon the sabbath day.’}{\arabic{verse}}
\threeverse{\aliya{לוי}}%Ex.35:4
{וַיֹּ֣אמֶר מֹשֶׁ֔ה אֶל־כׇּל־עֲדַ֥ת בְּנֵֽי־יִשְׂרָאֵ֖ל לֵאמֹ֑ר זֶ֣ה הַדָּבָ֔ר אֲשֶׁר־צִוָּ֥ה יְהֹוָ֖ה לֵאמֹֽר׃
\rashi{\rashiDH{זה הדבר אשר צוה ה׳. }לי לאמר לכם׃ 
}}
{וַאֲמַר מֹשֶׁה לְכָל כְּנִשְׁתָּא דִּבְנֵי יִשְׂרָאֵל לְמֵימַר דֵּין פִּתְגָמָא דְּפַקֵּיד יְיָ לְמֵימַר׃}
{And Moses spoke unto all the congregation of the children of Israel, saying: ‘This is the thing which the \lord\space commanded, saying:}{\arabic{verse}}
\threeverse{\arabic{verse}}%Ex.35:5
{קְח֨וּ מֵֽאִתְּכֶ֤ם תְּרוּמָה֙ לַֽיהֹוָ֔ה כֹּ֚ל נְדִ֣יב לִבּ֔וֹ יְבִיאֶ֕הָ אֵ֖ת תְּרוּמַ֣ת יְהֹוָ֑ה זָהָ֥ב וָכֶ֖סֶף וּנְחֹֽשֶׁת׃
\rashi{\rashiDH{נדיב לבו. }על שם שלבו נדבו קרוי נדיב לב. כבר פירשתי נדבת המשכן ומלאכתו במקום צוואתם׃ 
}}
{סַבוּ מִנְּכוֹן אַפְרָשׁוּתָא קֳדָם יְיָ כֹּל דְּיִתְרְעֵי לִבֵּיהּ יַיְתֵי יָת אַפְרָשׁוּתָא קֳדָם יְיָ דַּהְבָּא וְכַסְפָּא וּנְחָשָׁא׃}
{Take ye from among you an offering unto the \lord, whosoever is of a willing heart, let him bring it, the \lord’S offering: gold, and silver, and brass;}{\arabic{verse}}
\threeverse{\arabic{verse}}%Ex.35:6
{וּתְכֵ֧לֶת וְאַרְגָּמָ֛ן וְתוֹלַ֥עַת שָׁנִ֖י וְשֵׁ֥שׁ וְעִזִּֽים׃}
{וְתַכְלָא וְאַרְגְּוָנָא וּצְבַע זְהוֹרִי וּבוּץ וּמַעְזֵי׃}
{and blue, and purple, and scarlet, and fine linen, and goats’ hair;}{\arabic{verse}}
\threeverse{\arabic{verse}}%Ex.35:7
{וְעֹרֹ֨ת אֵילִ֧ם מְאׇדָּמִ֛ים וְעֹרֹ֥ת תְּחָשִׁ֖ים וַעֲצֵ֥י שִׁטִּֽים׃}
{וּמַשְׁכֵּי דְּדִכְרֵי מְסֻמְּקֵי וּמַשְׁכֵּי סָסְגוֹנָא וְאָעֵי שִׁטִּין׃}
{and rams’ skins dyed red, and sealskins, and acacia-wood;}{\arabic{verse}}
\threeverse{\arabic{verse}}%Ex.35:8
{וְשֶׁ֖מֶן לַמָּא֑וֹר וּבְשָׂמִים֙ לְשֶׁ֣מֶן הַמִּשְׁחָ֔ה וְלִקְטֹ֖רֶת הַסַּמִּֽים׃}
{וּמִשְׁחָא לְאַנְהָרוּתָא וּבוּסְמַיָּא לִמְשַׁח רְבוּתָא וְלִקְטֹרֶת בּוּסְמַיָּא׃}
{and oil for the light, and spices for the anointing oil, and for the sweet incense;}{\arabic{verse}}
\threeverse{\arabic{verse}}%Ex.35:9
{וְאַ֨בְנֵי־שֹׁ֔הַם וְאַבְנֵ֖י מִלֻּאִ֑ים לָאֵפ֖וֹד וְלַחֹֽשֶׁן׃}
{וְאַבְנֵי בוּרְלָא וְאַבְנֵי אַשְׁלָמוּתָא לְשַׁקָּעָא בְּאֵיפוֹדָא וּבְחוּשְׁנָא׃}
{and onyx stones, and stones to be set, for the ephod, and for the breastplate.}{\arabic{verse}}
\threeverse{\arabic{verse}}%Ex.35:10
{וְכׇל־חֲכַם־לֵ֖ב בָּכֶ֑ם יָבֹ֣אוּ וְיַעֲשׂ֔וּ אֵ֛ת כׇּל־אֲשֶׁ֥ר צִוָּ֖ה יְהֹוָֽה׃}
{וְכָל חַכִּימֵי לִבָּא דִּבְכוֹן יֵיתוֹן וְיַעְבְּדוּן יָת כָּל דְּפַקֵּיד יְיָ׃}
{And let every wise-hearted man among you come, and make all that the \lord\space hath commanded:}{\arabic{verse}}
\threeverse{\aliya{ישראל}}%Ex.35:11
{אֶ֨ת־הַמִּשְׁכָּ֔ן אֶֽת־אׇהֳל֖וֹ וְאֶת־מִכְסֵ֑הוּ אֶת־קְרָסָיו֙ וְאֶת־קְרָשָׁ֔יו אֶת־בְּרִיחָ֕ו אֶת־עַמֻּדָ֖יו וְאֶת־אֲדָנָֽיו׃
\rashi{\rashiDH{את המשכן. }יריעות התחתונות הנראות בתוכו קרוים משכן׃}\rashi{\rashiDH{את אהלו. }היא אהל יריעות עזים העשוי לגג׃}\rashi{\rashiDH{ואת מכסהו. }מכסה עורות אילים והתחשים׃}}
{יָת מַשְׁכְּנָא יָת פְּרָסֵיהּ וְיָת חוּפָאֵיהּ פּוּרְפוֹהִי דַּפּוֹהִי עָבְרוֹהִי עַמּוּדוֹהִי וְסָמְכוֹהִי׃}
{the tabernacle, its tent, and its covering, its clasps, and its boards, its bars, its pillars, and its sockets;}{\arabic{verse}}
\threeverse{\arabic{verse}}%Ex.35:12
{אֶת־הָאָרֹ֥ן וְאֶת־בַּדָּ֖יו אֶת־הַכַּפֹּ֑רֶת וְאֵ֖ת פָּרֹ֥כֶת הַמָּסָֽךְ׃
\rashi{\rashiDH{ואת פרוכת המסך. }פרוכת המחיצה. כל דבר המגין בין למעלה בין מכנגד קרוי מסך וסכך, וכן שַׂכְתָּ בַעֲדֹו (איוב א, ו), הִנְנִי שָׂךְ אֶת דַּרְכֵּךְ (הושע ב, ח)׃ }}
{יָת אֲרוֹנָא וְיָת אֲרִיחוֹהִי יָת כָּפוּרְתָּא וְיָת פָּרוּכְתָּא דִּפְרָסָא׃}
{the ark, and the staves thereof, the ark-cover, and the veil of the screen;}{\arabic{verse}}
\threeverse{\arabic{verse}}%Ex.35:13
{אֶת־הַשֻּׁלְחָ֥ן וְאֶת־בַּדָּ֖יו וְאֶת־כׇּל־כֵּלָ֑יו וְאֵ֖ת לֶ֥חֶם הַפָּנִֽים׃
\rashi{\rashiDH{לחם הפנים. }כבר פירשתי, על שם שהיו לו פנים לכאן ולכאן, שהיה עשוי כמין תיבה פרוצה׃ }}
{יָת פָּתוּרָא וְיָת אֲרִיחוֹהִי וְיָת כָּל מָנוֹהִי וְיָת לְחֵים אַפַּיָּא׃}
{the table, and its staves, and all its vessels, and the showbread;}{\arabic{verse}}
\threeverse{\arabic{verse}}%Ex.35:14
{וְאֶת־מְנֹרַ֧ת הַמָּא֛וֹר וְאֶת־כֵּלֶ֖יהָ וְאֶת־נֵרֹתֶ֑יהָ וְאֵ֖ת שֶׁ֥מֶן הַמָּאֽוֹר׃
\rashi{\rashiDH{ואת כליה. }מלקחים ומחתות׃}\rashi{\rashiDH{נרותיה. }לוציני״ש בלע״ז, בזיכים שהשמן והפתילות נתונין בהן׃ }\rashi{\rashiDH{ואת שמן המאור. }אף הוא צריך חכמי לב, שהוא משונה משאר שמנים, כמו שמפורש במנחות (פו.), מגרגרו בראש הזית, והוא כתית וזך׃ }}
{וְיָת מְנָרְתָא דְּאַנְהוֹרִי וְיָת מָנַהָא וְיָת בּוֹצִינַהָא וְיָת מִשְׁחָא דְּאַנְהָרוּתָא׃}
{the candlestick also for the light, and its vessels, and its lamps, and the oil for the light;}{\arabic{verse}}
\threeverse{\arabic{verse}}%Ex.35:15
{וְאֶת־מִזְבַּ֤ח הַקְּטֹ֙רֶת֙ וְאֶת־בַּדָּ֔יו וְאֵת֙ שֶׁ֣מֶן הַמִּשְׁחָ֔ה וְאֵ֖ת קְטֹ֣רֶת הַסַּמִּ֑ים וְאֶת־מָסַ֥ךְ הַפֶּ֖תַח לְפֶ֥תַח הַמִּשְׁכָּֽן׃
\rashi{\rashiDH{מסך הפתח. }וילון שלפני המזרח, שלא היו שם קרשים ולא יריעות׃ 
}}
{וְיָת מַדְבְּחָא דִּקְטֹרֶת בּוּסְמַיָּא וְיָת אֲרִיחוֹהִי וְיָת מִשְׁחָא דִּרְבוּתָא וְיָת קְטֹרֶת בּוּסְמַיָּא וְיָת פְּרָסָא דְּתַרְעָא לִתְרַע מַשְׁכְּנָא׃}
{and the altar of incense, and its staves, and the anointing oil, and the sweet incense, and the screen for the door, at the door of the tabernacle;}{\arabic{verse}}
\threeverse{\arabic{verse}}%Ex.35:16
{אֵ֣ת \legarmeh  מִזְבַּ֣ח הָעֹלָ֗ה וְאֶת־מִכְבַּ֤ר הַנְּחֹ֙שֶׁת֙ אֲשֶׁר־ל֔וֹ אֶת־בַּדָּ֖יו וְאֶת־כׇּל־כֵּלָ֑יו אֶת־הַכִּיֹּ֖ר וְאֶת־כַּנּֽוֹ׃}
{יָת מַדְבְּחָא דַּעֲלָתָא וְיָת סְרָדָא דִּנְחָשָׁא דִּילֵיהּ יָת אֲרִיחוֹהִי וְיָת כָּל מָנוֹהִי יָת כִּיּוֹרָא וְיָת בְּסִיסֵיהּ׃}
{the altar of burnt-offering, with its grating of brass, its staves, and all its vessels, the laver and its base;}{\arabic{verse}}
\threeverse{\arabic{verse}}%Ex.35:17
{אֵ֚ת קַלְעֵ֣י הֶחָצֵ֔ר אֶת־עַמֻּדָ֖יו וְאֶת־אֲדָנֶ֑יהָ וְאֵ֕ת מָסַ֖ךְ שַׁ֥עַר הֶחָצֵֽר׃
\rashi{\rashiDH{את עמודיו ואת אדניה. }הרי חצר קרוי כאן לשון זכר ולשון נקבה, וכן דברים הרבה׃ }\rashi{\rashiDH{ואת מסך שער החצר. }וילון פרוש לצד המזרח עשרים אמה אמצעיות, של רוחב החצר שהיה חמשים רחב, וסתומין הימנו לצד צפון ט״ו אמה, וכן לדרום, שנאמר וַחֲמֵשׁ עֶשְׂרֵה אַמָּה קְלָעִים לַכָּתֵף (שמות כז, יד)׃ }}
{יָת סְרָדֵי דָּרְתָא יָת עַמּוּדוֹהִי וְיָת סָמְכַהָא וְיָת פְּרָסָא דִּתְרַע דָּרְתָא׃}
{the hangings of the court, the pillars thereof, and their sockets, and the screen for the gate of the court;}{\arabic{verse}}
\threeverse{\arabic{verse}}%Ex.35:18
{אֶת־יִתְדֹ֧ת הַמִּשְׁכָּ֛ן וְאֶת־יִתְדֹ֥ת הֶחָצֵ֖ר וְאֶת־מֵיתְרֵיהֶֽם׃
\rashi{\rashiDH{יתדות. }לתקוע ולקשור בהם סופי היריעות בארץ, שלא ינועו ברוח׃ }\rashi{\rashiDH{מיתריהם. }חבלים לקשור׃}}
{יָת סִכֵּי מַשְׁכְּנָא וְיָת סִכֵּי דָּרְתָא וְיָת אֲטוּנֵיהוֹן׃}
{the pins of the tabernacle, and the pins of the court, and their cords;}{\arabic{verse}}
\threeverse{\arabic{verse}}%Ex.35:19
{אֶת־בִּגְדֵ֥י הַשְּׂרָ֖ד לְשָׁרֵ֣ת בַּקֹּ֑דֶשׁ אֶת־בִּגְדֵ֤י הַקֹּ֙דֶשׁ֙ לְאַהֲרֹ֣ן הַכֹּהֵ֔ן וְאֶת־בִּגְדֵ֥י בָנָ֖יו לְכַהֵֽן׃
\rashi{\rashiDH{בגדי השרד. }לכסות הארון והשלחן והמנורה והמזבחות בשעת סילוק מסעות׃ 
}}
{יָת לְבוּשֵׁי שִׁמּוּשָׁא לְשַׁמָּשָׁא בְּקוּדְשָׁא יָת לְבוּשֵׁי קוּדְשָׁא לְאַהֲרֹן כָּהֲנָא וְיָת לְבוּשֵׁי בְנוֹהִי לְשַׁמָּשָׁא׃}
{the plaited garments, for ministering in the holy place, the holy garments for Aaron the priest, and the garments of his sons, to minister in the priest’s office.’}{\arabic{verse}}
\threeverse{\arabic{verse}}%Ex.35:20
{וַיֵּ֥צְא֛וּ כׇּל־עֲדַ֥ת בְּנֵֽי־יִשְׂרָאֵ֖ל מִלִּפְנֵ֥י מֹשֶֽׁה׃}
{וּנְפַקוּ כָּל כְּנִשְׁתָּא דִּבְנֵי יִשְׂרָאֵל מִן קֳדָם מֹשֶׁה׃}
{And all the congregation of the children of Israel departed from the presence of Moses.}{\arabic{verse}}
\threeverse{\aliya{שני}}%Ex.35:21
{וַיָּבֹ֕אוּ כׇּל־אִ֖ישׁ אֲשֶׁר־נְשָׂא֣וֹ לִבּ֑וֹ וְכֹ֡ל אֲשֶׁר֩ נָדְבָ֨ה רוּח֜וֹ אֹת֗וֹ הֵ֠בִ֠יאוּ אֶת־תְּרוּמַ֨ת יְהֹוָ֜ה לִמְלֶ֨אכֶת אֹ֤הֶל מוֹעֵד֙ וּלְכׇל־עֲבֹ֣דָת֔וֹ וּלְבִגְדֵ֖י הַקֹּֽדֶשׁ׃}
{וַאֲתוֹ כָּל גְּבַר דְּאִתְרְעִי לִבֵּיהּ וְכֹל דְּאַשְׁלֵימַת רוּחֵיהּ עִמֵּיהּ אֵיתִיאוּ יָת אַפְרָשׁוּתָא קֳדָם יְיָ לַעֲבִידַת מַשְׁכַּן זִמְנָא וּלְכָל פּוּלְחָנֵיהּ וְלִלְבוּשֵׁי קוּדְשָׁא׃}
{And they came, every one whose heart stirred him up, and every one whom his spirit made willing, and brought the \lord’S offering, for the work of the tent of meeting, and for all the service thereof, and for the holy garments. .}{\arabic{verse}}
\threeverse{\arabic{verse}}%Ex.35:22
{וַיָּבֹ֥אוּ הָאֲנָשִׁ֖ים עַל־הַנָּשִׁ֑ים כֹּ֣ל \legarmeh  נְדִ֣יב לֵ֗ב הֵ֠בִ֠יאוּ חָ֣ח וָנֶ֜זֶם וְטַבַּ֤עַת וְכוּמָז֙ כׇּל־כְּלִ֣י זָהָ֔ב וְכׇל־אִ֕ישׁ אֲשֶׁ֥ר הֵנִ֛יף תְּנוּפַ֥ת זָהָ֖ב לַיהֹוָֽה׃
\rashi{\rashiDH{על הנשים. }עם הנשים, וסמוכין אליהם. (מה שהתרגום הניח על כפשוטו, משום דלא מתרגם ויבאו האנשים וַאֲתוֹ גֻּבְרַיָא, כמו שמתרגם לעיל מיניה, רק מתרגם וּמַיְיתָן, ורצה לומר שהביאו חח ונזם בעודן על הנשים, כמו שכתב רש״י על טוו את העזים)׃ }\rashi{\rashiDH{חח. }הוא תכשיט של זהב עגול, נתון על הזרוע, והוא הצמיד׃ }\rashi{\rashiDH{וכומז. }כלי זהב הוא, נתון כנגד אותו מקום לאשה, ורבותינו פירשו שם כומז, כאן מקום זמה׃ }}
{וּמֵיתַן גּוּבְרַיָּא עַל נְשַׁיָּא כֹּל דְּאִתְרְעִי לִבֵּיהּ אֵיתִיאוּ שֵׁירִין וְשַׁבִּין וְעִזְקָן וּמָחוֹךְ כָּל מָן דִּדְהַב וְכָל גְּבַר דַּאֲרֵים אֲרָמוּת דַּהְבָּא קֳדָם יְיָ׃}
{And they came, both men and women, as many as were willing-hearted, and brought nose-rings, and ear-rings, and signet-rings, and girdles, all jewels of gold; even every man that brought an offering of gold unto the \lord.}{\arabic{verse}}
\threeverse{\arabic{verse}}%Ex.35:23
{וְכׇל־אִ֞ישׁ אֲשֶׁר־נִמְצָ֣א אִתּ֗וֹ תְּכֵ֧לֶת וְאַרְגָּמָ֛ן וְתוֹלַ֥עַת שָׁנִ֖י וְשֵׁ֣שׁ וְעִזִּ֑ים וְעֹרֹ֨ת אֵילִ֧ם מְאׇדָּמִ֛ים וְעֹרֹ֥ת תְּחָשִׁ֖ים הֵבִֽיאוּ׃
\rashi{\rashiDH{וכל איש אשר נמצא אתו. }תכלת או ארגמן או תולעת שני או עורות אילים או תחשים, כולם הביאו׃ 
}}
{וְכָל גְּבַר דְּאִשְׁתְּכַח עִמֵּיהּ תַּכְלָא וְאַרְגְּוָנָא וּצְבַע זְהוֹרִי וּבוּץ וּמַעְזֵי וּמַשְׁכֵּי דְּדִכְרֵי מְסֻמְּקֵי וּמַשְׁכֵּי סָסְגוֹנָא אֵיתִיאוּ׃}
{And every man, with whom was found blue, and purple, and scarlet, and fine linen, and goats’ hair, and rams’ skins dyed red, and sealskins, brought them.}{\arabic{verse}}
\threeverse{\arabic{verse}}%Ex.35:24
{כׇּל־מֵרִ֗ים תְּר֤וּמַת כֶּ֙סֶף֙ וּנְחֹ֔שֶׁת הֵבִ֕יאוּ אֵ֖ת תְּרוּמַ֣ת יְהֹוָ֑ה וְכֹ֡ל אֲשֶׁר֩ נִמְצָ֨א אִתּ֜וֹ עֲצֵ֥י שִׁטִּ֛ים לְכׇל־מְלֶ֥אכֶת הָעֲבֹדָ֖ה הֵבִֽיאוּ׃}
{כָּל דַּאֲרֵים אֲרָמוּת כְּסַף וּנְחָשׁ אֵיתִיאוּ יָת אַפְרָשׁוּתָא קֳדָם יְיָ וְכֹל דְּאִשְׁתְּכַח עִמֵּיהּ אָעֵי שִׁטִּין לְכָל עֲבִידַת פּוּלְחָנָא אֵיתִיאוּ׃}
{Every one that did set apart an offering of silver and brass brought the \lord’S offering; and every man, with whom was found acacia-wood for any work of the service, brought it.}{\arabic{verse}}
\threeverse{\arabic{verse}}%Ex.35:25
{וְכׇל־אִשָּׁ֥ה חַכְמַת־לֵ֖ב בְּיָדֶ֣יהָ טָו֑וּ וַיָּבִ֣יאוּ מַטְוֶ֗ה אֶֽת־הַתְּכֵ֙לֶת֙ וְאֶת־הָֽאַרְגָּמָ֔ן אֶת־תּוֹלַ֥עַת הַשָּׁנִ֖י וְאֶת־הַשֵּֽׁשׁ׃}
{וְכָל אִתְּתָא חַכִּימַת לִבָּא בִּידַהָא עָזְלָא וּמֵיתַן כִּד עֲזִיל יָת תַּכְלָא וְיָת אַרְגְּוָנָא יָת צְבַע זְהוֹרִי וְיָת בּוּצָא׃}
{And all the women that were wise-hearted did spin with their hands, and brought that which they had spun, the blue, and the purple, the scarlet, and the fine linen.}{\arabic{verse}}
\threeverse{\arabic{verse}}%Ex.35:26
{וְכׇ֨ל־הַנָּשִׁ֔ים אֲשֶׁ֨ר נָשָׂ֥א לִבָּ֛ן אֹתָ֖נָה בְּחׇכְמָ֑ה טָו֖וּ אֶת־הָעִזִּֽים׃
\rashi{\rashiDH{טוו את העזים. }היא היתה אומנות יתירה, שמעל גבי העזים היו טווין אותם (שבת צט.)׃ }}
{וְכָל נְשַׁיָּא דְּאִתְרְעִי לִבְּהוֹן עִמְּהוֹן בְּחָכְמָא עָזְלָן יָת מַעַזְיָא׃}
{And all the women whose heart stirred them up in wisdom spun the goats’ hair.}{\arabic{verse}}
\threeverse{\arabic{verse}}%Ex.35:27
{וְהַנְּשִׂאִ֣ם הֵבִ֔יאוּ אֵ֚ת אַבְנֵ֣י הַשֹּׁ֔הַם וְאֵ֖ת אַבְנֵ֣י הַמִּלֻּאִ֑ים לָאֵפ֖וֹד וְלַחֹֽשֶׁן׃
\rashi{\rashiDH{והנשאם הביאו. }אמר ר׳ נתן, מה ראו נשיאים להתנדב בחנוכת המזבח בתחלה, ובמלאכת המשכן לא התנדבו בתחלה, אלא כך אמרו נשיאים, יתנדבו צבור מה שמתנדבים, ומה שמחסרין אנו משלימין אותו, כיון שהשלימו צבור את הכל, שנאמר וְהַמְּלָאכָה הָיְתָה דַיָּם (שמות לו, ז), אמרו נשיאים מה עלינו לעשות, הביאו את אבני השהם וגו׳, לכך התנדבו בחנוכת המזבח תחלה, ולפי שנתעצלו מתחלה, נחסרה אות משמם, והנשאם כתיב׃ 
}}
{וְרַבְרְבַיָּא אֵיתִיאוּ יָת אַבְנֵי בוּרְלָא וְיָת אַבְנֵי אַשְׁלָמוּתָא לְשַׁקָּעָא בְּאֵיפוֹדָא וּבְחוּשְׁנָא׃}
{And the rulers brought the onyx stones, and the stones to be set, for the ephod, and for the breastplate;}{\arabic{verse}}
\threeverse{\arabic{verse}}%Ex.35:28
{וְאֶת־הַבֹּ֖שֶׂם וְאֶת־הַשָּׁ֑מֶן לְמָא֕וֹר וּלְשֶׁ֙מֶן֙ הַמִּשְׁחָ֔ה וְלִקְטֹ֖רֶת הַסַּמִּֽים׃}
{וְיָת בּוּסְמָא וְיָת מִשְׁחָא לְאַנְהָרוּתָא וְלִמְשַׁח רְבוּתָא וְלִקְטֹרֶת בּוּסְמַיָּא׃}
{and the spice, and the oil, for the light, and for the anointing oil, and for the sweet incense.}{\arabic{verse}}
\threeverse{\arabic{verse}}%Ex.35:29
{כׇּל־אִ֣ישׁ וְאִשָּׁ֗ה אֲשֶׁ֨ר נָדַ֣ב לִבָּם֮ אֹתָם֒ לְהָבִיא֙ לְכׇל־הַמְּלָאכָ֔ה אֲשֶׁ֨ר צִוָּ֧ה יְהֹוָ֛ה לַעֲשׂ֖וֹת בְּיַד־מֹשֶׁ֑ה הֵבִ֧יאוּ בְנֵי־יִשְׂרָאֵ֛ל נְדָבָ֖ה לַיהֹוָֽה׃ \petucha }
{כּל גְּבַר וְאִתָּא דְּאִתְרְעִי לִבְּהוֹן עִמְּהוֹן לְאֵיתָאָה לְכָל עֲבִידְתָא דְּפַקֵּיד יְיָ לְמֶעֱבַד בִּידָא דְּמֹשֶׁה אֵיתִיאוּ בְנֵי יִשְׂרָאֵל נְדַבְתָּא קֳדָם יְיָ׃}
{The children of Israel brought a freewill-offering unto the \lord; every man and woman, whose heart made them willing to bring for all the work, which the \lord\space had commanded by the hand of Moses to be made.}{\arabic{verse}}
\threeverse{\aliya{שלישי\newline (שני)}}%Ex.35:30
{וַיֹּ֤אמֶר מֹשֶׁה֙ אֶל־בְּנֵ֣י יִשְׂרָאֵ֔ל רְא֛וּ קָרָ֥א יְהֹוָ֖ה בְּשֵׁ֑ם בְּצַלְאֵ֛ל בֶּן־אוּרִ֥י בֶן־ח֖וּר לְמַטֵּ֥ה יְהוּדָֽה׃
\rashi{\rashiDH{חור. }בנה של מרים היה׃ 
}}
{וַאֲמַר מֹשֶׁה לִבְנֵי יִשְׂרָאֵל חֲזוֹ דְּרַבִּי יְיָ בְּשׁוֹם בְּצַלְאֵל בַּר אוּרִי בַר חוּר לְשִׁבְטָא דִּיהוּדָה׃}
{And Moses said unto the children of Israel: ‘See, the \lord\space hath called by name Bezalel the son of Uri, the son of Hur, of the tribe of Judah.}{\arabic{verse}}
\threeverse{\arabic{verse}}%Ex.35:31
{וַיְמַלֵּ֥א אֹת֖וֹ ר֣וּחַ אֱלֹהִ֑ים בְּחׇכְמָ֛ה בִּתְבוּנָ֥ה וּבְדַ֖עַת וּבְכׇל־מְלָאכָֽה׃}
{וְאַשְׁלֵים עִמֵּיהּ רוּחַ מִן קֳדָם יְיָ בְּחָכְמָא בְּסוּכְלְתָנוּ וּבְמַדַּע וּבְכָל עֲבִידָא׃}
{And He hath filled him with the spirit of God, in wisdom, in understanding, and in knowledge, and in all manner of workmanship.}{\arabic{verse}}
\threeverse{\arabic{verse}}%Ex.35:32
{וְלַחְשֹׁ֖ב מַֽחֲשָׁבֹ֑ת לַעֲשֹׂ֛ת בַּזָּהָ֥ב וּבַכֶּ֖סֶף וּבַנְּחֹֽשֶׁת׃}
{וּלְאַלָּפָא אֻמָּנְוָן לְמֶעֱבַד בְּדַהְבָּא וּבְכַסְפָּא וּבִנְחָשָׁא׃}
{And to devise skilful works, to work in gold, and in silver, and in brass,}{\arabic{verse}}
\threeverse{\arabic{verse}}%Ex.35:33
{וּבַחֲרֹ֥שֶׁת אֶ֛בֶן לְמַלֹּ֖את וּבַחֲרֹ֣שֶׁת עֵ֑ץ לַעֲשׂ֖וֹת בְּכׇל־מְלֶ֥אכֶת מַחֲשָֽׁבֶת׃}
{וּבְאוּמָּנוּת אֶבֶן טָבָא לְאַשְׁלָמָא וּבְנַגָּרוּת אָעָא לְמֶעֱבַד בְּכָל עֲבִידַת אוּמָּנְוָן׃}
{and in cutting of stones for setting, and in carving of wood, to work in all manner of skilful workmanship.}{\arabic{verse}}
\threeverse{\arabic{verse}}%Ex.35:34
{וּלְהוֹרֹ֖ת נָתַ֣ן בְּלִבּ֑וֹ ה֕וּא וְאׇֽהֳלִיאָ֥ב בֶּן־אֲחִיסָמָ֖ךְ לְמַטֵּה־דָֽן׃
\rashi{\rashiDH{ואהליאב. }משבט דן, מן הירודין שבשבטים, מבני השפחות, והשוהו המקום לבצלאל למלאכת המשכן, והוא מגדולי השבטים, לקיים מה שנאמר וְלֹא נִכַּר שֹׁעַ לִפְנֵי דָל (איוב לד, יט)׃ 
}}
{וּלְאַלָּפָא יְהַב בְּלִבֵּיהּ הוּא וְאָהֳלִיאָב בַּר אֲחִיסָמָךְ לְשִׁבְטָא דְּדָן׃}
{And He hath put in his heart that he may teach, both he, and Oholiab, the son of Ahisamach, of the tribe of Dan.}{\arabic{verse}}
\threeverse{\arabic{verse}}%Ex.35:35
{מִלֵּ֨א אֹתָ֜ם חׇכְמַת־לֵ֗ב לַעֲשׂוֹת֮ כׇּל־מְלֶ֣אכֶת חָרָ֣שׁ \pasek  וְחֹשֵׁב֒ וְרֹקֵ֞ם בַּתְּכֵ֣לֶת וּבָֽאַרְגָּמָ֗ן בְּתוֹלַ֧עַת הַשָּׁנִ֛י וּבַשֵּׁ֖שׁ וְאֹרֵ֑ג עֹשֵׂי֙ כׇּל־מְלָאכָ֔ה וְחֹשְׁבֵ֖י מַחֲשָׁבֹֽת׃}
{אַשְׁלֵים עִמְּהוֹן חַכִּימוּת לִבָּא לְמֶעֱבַד כָּל עֲבִידַת נַגָּר וְאוּמָּן וְצַיָּיר בְּתַכְלָא וּבְאַרְגְּוָנָא בִּצְבַע זְהוֹרִי וּבְבוּצָא וּמָחֵי עָבְדֵי כָּל עֲבִידָא וּמַלְּפֵי אוּמָּנְוָן׃}
{Them hath He filled with wisdom of heart, to work all manner of workmanship, of the craftsman, and of the skilful workman, and of the weaver in colours, in blue, and in purple, in scarlet, and in fine linen, and of the weaver, even of them that do any workmanship, and of those that devise skilful works.}{\arabic{verse}}
\newperek
\threeverse{\Roman{chap}}%Ex.36:1
{וְעָשָׂה֩ בְצַלְאֵ֨ל וְאׇהֳלִיאָ֜ב וְכֹ֣ל \legarmeh  אִ֣ישׁ חֲכַם־לֵ֗ב אֲשֶׁר֩ נָתַ֨ן יְהֹוָ֜ה חׇכְמָ֤ה וּתְבוּנָה֙ בָּהֵ֔מָּה לָדַ֣עַת לַעֲשֹׂ֔ת אֶֽת־כׇּל־מְלֶ֖אכֶת עֲבֹדַ֣ת הַקֹּ֑דֶשׁ לְכֹ֥ל אֲשֶׁר־צִוָּ֖ה יְהֹוָֽה׃}
{וְיַעֲבֵיד בְּצַלְאֵל וְאָהֳלִיאָב וְכֹל גְּבַר חַכִּים לִבָּא דִּיהַב יְיָ חָכְמְתָא וְסוּכְלְתָנוּתָא בְּהוֹן לְמִדַּע לְמֶעֱבַד יָת כָּל עֲבִידַת פּוּלְחַן קוּדְשָׁא לְכָל דְּפַקֵּיד יְיָ׃}
{And Bezalel and Oholiab shall work, and every wise-hearted man, in whom the \lord\space hath put wisdom and understanding to know how to work all the work for the service of the sanctuary, according to all that the \lord\space hath commanded.’}{\Roman{chap}}
\threeverse{\arabic{verse}}%Ex.36:2
{וַיִּקְרָ֣א מֹשֶׁ֗ה אֶל־בְּצַלְאֵל֮ וְאֶל־אׇֽהֳלִיאָב֒ וְאֶל֙ כׇּל־אִ֣ישׁ חֲכַם־לֵ֔ב אֲשֶׁ֨ר נָתַ֧ן יְהֹוָ֛ה חׇכְמָ֖ה בְּלִבּ֑וֹ כֹּ֚ל אֲשֶׁ֣ר נְשָׂא֣וֹ לִבּ֔וֹ לְקׇרְבָ֥ה אֶל־הַמְּלָאכָ֖ה לַעֲשֹׂ֥ת אֹתָֽהּ׃}
{וּקְרָא מֹשֶׁה לִבְצַלְאֵל וּלְאָהֳלִיאָב וּלְכָל גְּבַר חַכִּים לִבָּא דִּיהַב יְיָ חָכְמְתָא בְּלִבֵּיהּ כֹּל דְּאִתְרְעִי לִבֵּיהּ לְמִקְרַב לַעֲבִידְתָא לְמֶעֱבַד יָתַהּ׃}
{And Moses called Bezalel and Oholiab, and every wise-hearted man, in whose heart the \lord\space had put wisdom, even every one whose heart stirred him up to come unto the work to do it.}{\arabic{verse}}
\threeverse{\arabic{verse}}%Ex.36:3
{וַיִּקְח֞וּ מִלִּפְנֵ֣י מֹשֶׁ֗ה אֵ֤ת כׇּל־הַתְּרוּמָה֙ אֲשֶׁ֨ר הֵבִ֜יאוּ בְּנֵ֣י יִשְׂרָאֵ֗ל לִמְלֶ֛אכֶת עֲבֹדַ֥ת הַקֹּ֖דֶשׁ לַעֲשֹׂ֣ת אֹתָ֑הּ וְ֠הֵ֠ם הֵבִ֨יאוּ אֵלָ֥יו ע֛וֹד נְדָבָ֖ה בַּבֹּ֥קֶר בַּבֹּֽקֶר׃}
{וּנְסִיבוּ מִן קֳדָם מֹשֶׁה יָת כָּל אַפְרָשׁוּתָא דְּאֵיתִיאוּ בְּנֵי יִשְׂרָאֵל לַעֲבִידַת פּוּלְחַן קוּדְשָׁא לְמֶעֱבַד יָתַהּ וְאִנּוּן מֵיתַן לֵיהּ עוֹד נְדַבְתָּא בִּצְפַר בִּצְפַר׃}
{And they received of Moses all the offering, which the children of Israel had brought for the work of the service of the sanctuary, wherewith to make it. And they brought yet unto him freewill-offerings every morning.}{\arabic{verse}}
\threeverse{\arabic{verse}}%Ex.36:4
{וַיָּבֹ֙אוּ֙ כׇּל־הַ֣חֲכָמִ֔ים הָעֹשִׂ֕ים אֵ֖ת כׇּל־מְלֶ֣אכֶת הַקֹּ֑דֶשׁ אִֽישׁ־אִ֥ישׁ מִמְּלַאכְתּ֖וֹ אֲשֶׁר־הֵ֥מָּה עֹשִֽׂים׃}
{וַאֲתוֹ כָּל חַכִּימַיָּא דְּעָבְדִין יָת כָּל עֲבִידַת קוּדְשָׁא גְּבַר גְּבַר מֵעֲבִידְתֵיהּ דְּאִנּוּן עָבְדִין׃}
{And all the wise men, that wrought all the work of the sanctuary, came every man from his work which they wrought.}{\arabic{verse}}
\threeverse{\arabic{verse}}%Ex.36:5
{וַיֹּאמְרוּ֙ אֶל־מֹשֶׁ֣ה לֵּאמֹ֔ר מַרְבִּ֥ים הָעָ֖ם לְהָבִ֑יא מִדֵּ֤י הָֽעֲבֹדָה֙ לַמְּלָאכָ֔ה אֲשֶׁר־צִוָּ֥ה יְהֹוָ֖ה לַעֲשֹׂ֥ת אֹתָֽהּ׃
\rashi{\rashiDH{מדי העבודה. }יותר מכדי צורך העבודה׃}}
{וַאֲמַרוּ לְמֹשֶׁה לְמֵימַר מַסְגַּן עַמָּא לְאֵיתָאָה מִסַּת פּוּלְחָנָא לַעֲבִידְתָא דְּפַקֵּיד יְיָ לְמֶעֱבַד יָתַהּ׃}
{And they spoke unto Moses, saying: ‘The people bring much more than enough for the service of the work, which the \lord\space commanded to make.’}{\arabic{verse}}
\threeverse{\arabic{verse}}%Ex.36:6
{וַיְצַ֣ו מֹשֶׁ֗ה וַיַּעֲבִ֨ירוּ ק֥וֹל בַּֽמַּחֲנֶה֮ לֵאמֹר֒ אִ֣ישׁ וְאִשָּׁ֗ה אַל־יַעֲשׂוּ־ע֛וֹד מְלָאכָ֖ה לִתְרוּמַ֣ת הַקֹּ֑דֶשׁ וַיִּכָּלֵ֥א הָעָ֖ם מֵהָבִֽיא׃
\rashi{\rashiDH{ויכלא. }לשון מניעה׃}}
{וּפַקֵּיד מֹשֶׁה וְאַעְבַּרוּ כָּרוֹז בְּמַשְׁרִיתָא לְמֵימַר גְּבַר וְאִתָּא לָא יַעְבְּדוּן עוֹד עֲבִידְתָא לְאַפְרָשׁוּת קוּדְשָׁא וּפְסַק עַמָּא מִלְּאֵיתָאָה׃}
{And Moses gave commandment, and they caused it to be proclaimed throughout the camp, saying: ‘Let neither man nor woman make any more work for the offering of the sanctuary.’ So the people were restrained from bringing.}{\arabic{verse}}
\threeverse{\arabic{verse}}%Ex.36:7
{וְהַמְּלָאכָ֗ה הָיְתָ֥ה דַיָּ֛ם לְכׇל־הַמְּלָאכָ֖ה לַעֲשׂ֣וֹת אֹתָ֑הּ וְהוֹתֵֽר׃ \setuma         
\rashi{\rashiDH{והמלאכה היתה דים לכל המלאכה }ומלאכת ההבאה היתה דים של עושי המשכן, לכל המלאכה של משכן לעשות אותה, ולהותר׃ }\rashi{\rashiDH{והותר. }כמו וְהַכְבֵּד אֶת לִבֹּו (שמות ח, יא), וְהַכֹּות אֶת מֹואָב (מלכים־ב ג, כד)׃ 
}}
{וַעֲבִידְתָא הֲוָת מִסַּת לְכָל עֲבִידְתָא לְמֶעֱבַד יָתַהּ וִיתַרַת׃}
{For the stuff they had was sufficient for all the work to make it, and too much.}{\arabic{verse}}
\threeverse{\aliya{רביעי}}%Ex.36:8
{וַיַּעֲשׂ֨וּ כׇל־חֲכַם־לֵ֜ב בְּעֹשֵׂ֧י הַמְּלָאכָ֛ה אֶת־הַמִּשְׁכָּ֖ן עֶ֣שֶׂר יְרִיעֹ֑ת שֵׁ֣שׁ מׇשְׁזָ֗ר וּתְכֵ֤לֶת וְאַרְגָּמָן֙ וְתוֹלַ֣עַת שָׁנִ֔י כְּרֻבִ֛ים מַעֲשֵׂ֥ה חֹשֵׁ֖ב עָשָׂ֥ה אֹתָֽם׃}
{וַעֲבַדוּ כָל חַכִּימֵי לִבָּא בְּעָבְדֵי עֲבִידְתָא יָת מַשְׁכְּנָא עֲשַׂר יְרִיעָן דְּבוּץ שְׁזִיר וְתַכְלָא וְאַרְגְּוָנָא וּצְבַע זְהוֹרִי צוּרַת כְּרוּבִין עוֹבָד אוּמָּן עֲבַד יָתְהוֹן׃}
{And every wise-hearted man among them that wrought the work made the tabernacle with ten curtains: of fine twined linen, and blue, and purple, and scarlet, with cherubim the work of the skilful workman made he them.}{\arabic{verse}}
\threeverse{\arabic{verse}}%Ex.36:9
{אֹ֜רֶךְ הַיְרִיעָ֣ה הָֽאַחַ֗ת שְׁמֹנֶ֤ה וְעֶשְׂרִים֙ בָּֽאַמָּ֔ה וְרֹ֙חַב֙ אַרְבַּ֣ע בָּֽאַמָּ֔ה הַיְרִיעָ֖ה הָאֶחָ֑ת מִדָּ֥ה אַחַ֖ת לְכׇל־הַיְרִיעֹֽת׃}
{אוּרְכָּא דִּירִיעֲתָא חֲדָא עֶשְׂרִין וְתַמְנֵי אַמִּין וּפוּתְיָא אַרְבַּע אַמִּין דִּירִיעֲתָא חֲדָא מִשְׁחֲתָא חֲדָא לְכָל יְרִיעָתָא׃}
{The length of each curtain was eight and twenty cubits, and the breadth of each curtain four cubits; all the curtains had one measure.}{\arabic{verse}}
\threeverse{\arabic{verse}}%Ex.36:10
{וַיְחַבֵּר֙ אֶת־חֲמֵ֣שׁ הַיְרִיעֹ֔ת אַחַ֖ת אֶל־אֶחָ֑ת וְחָמֵ֤שׁ יְרִיעֹת֙ חִבַּ֔ר אַחַ֖ת אֶל־אֶחָֽת׃}
{וְלָפֵיף יָת חֲמֵישׁ יְרִיעָן חֲדָא עִם חֲדָא וַחֲמֵישׁ יְרִיעָן לָפֵיף חֲדָא עִם חֲדָא׃}
{And he coupled five curtains one to another; and the other five curtains he coupled one to another.}{\arabic{verse}}
\threeverse{\arabic{verse}}%Ex.36:11
{וַיַּ֜עַשׂ לֻֽלְאֹ֣ת תְּכֵ֗לֶת עַ֣ל שְׂפַ֤ת הַיְרִיעָה֙ הָֽאֶחָ֔ת מִקָּצָ֖ה בַּמַּחְבָּ֑רֶת כֵּ֤ן עָשָׂה֙ בִּשְׂפַ֣ת הַיְרִיעָ֔ה הַקִּ֣יצוֹנָ֔ה בַּמַּחְבֶּ֖רֶת הַשֵּׁנִֽית׃}
{וַעֲבַד עֲנוּבִּין דְּתַכְלָא עַל סִפְתָּא דִּירִיעֲתָא חֲדָא מִסִּטְרָא בֵּית לוֹפֵי כֵּן עֲבַד בְּסִפְתָּא דִּירִיעֲתָא בְּסִטְרָא בֵּית לוֹפִי תִּנְיָנָא׃}
{And he made loops of blue upon the edge of the one curtain that was outmost in the first set; likewise he made in the edge of the curtain that was outmost in the second set.}{\arabic{verse}}
\threeverse{\arabic{verse}}%Ex.36:12
{חֲמִשִּׁ֣ים לֻלָאֹ֗ת עָשָׂה֮ בַּיְרִיעָ֣ה הָאֶחָת֒ וַחֲמִשִּׁ֣ים לֻלָאֹ֗ת עָשָׂה֙ בִּקְצֵ֣ה הַיְרִיעָ֔ה אֲשֶׁ֖ר בַּמַּחְבֶּ֣רֶת הַשֵּׁנִ֑ית מַקְבִּילֹת֙ הַלֻּ֣לָאֹ֔ת אַחַ֖ת אֶל־אֶחָֽת׃}
{חַמְשִׁין עֲנוּבִּין עֲבַד בִּירִיעֲתָא חֲדָא וְחַמְשִׁין עֲנוּבִּין עֲבַד בְּסִטְרָא דִּירִיעֲתָא דְּבֵית לוֹפֵי תִּנְיָנָא מַכְוְנָן עֲנוּבַּיָּא חֲדָא לָקֳבֵיל חֲדָא׃}
{Fifty loops made he in the one curtain, and fifty loops made he in the edge of the curtain that was in the second set; the loops were opposite one to another.}{\arabic{verse}}
\threeverse{\arabic{verse}}%Ex.36:13
{וַיַּ֕עַשׂ חֲמִשִּׁ֖ים קַרְסֵ֣י זָהָ֑ב וַיְחַבֵּ֨ר אֶת־הַיְרִיעֹ֜ת אַחַ֤ת אֶל־אַחַת֙ בַּקְּרָסִ֔ים וַֽיְהִ֥י הַמִּשְׁכָּ֖ן אֶחָֽד׃ \petucha }
{וַעֲבַד חַמְשִׁין פּוּרְפִין דִּדְהַב וְלָפֵיף יָת יְרִיעָתָא חֲדָא עִם חֲדָא בְּפוּרְפַיָּא וַהֲוָה מַשְׁכְּנָא חַד׃}
{And he made fifty clasps of gold, and coupled the curtains one to another with the clasps; so the tabernacle was one.}{\arabic{verse}}
\threeverse{\arabic{verse}}%Ex.36:14
{וַיַּ֙עַשׂ֙ יְרִיעֹ֣ת עִזִּ֔ים לְאֹ֖הֶל עַל־הַמִּשְׁכָּ֑ן עַשְׁתֵּֽי־עֶשְׂרֵ֥ה יְרִיעֹ֖ת עָשָׂ֥ה אֹתָֽם׃}
{וַעֲבַד יְרִיעָן דְּמַעְזֵי לִפְרָסָא עַל מַשְׁכְּנָא חֲדָא עֶשְׂרֵי יְרִיעָן עֲבַד יָתְהוֹן׃}
{And he made curtains of goats’ hair for a tent over the tabernacle; eleven curtains he made them.}{\arabic{verse}}
\threeverse{\arabic{verse}}%Ex.36:15
{אֹ֜רֶךְ הַיְרִיעָ֣ה הָאַחַ֗ת שְׁלֹשִׁים֙ בָּֽאַמָּ֔ה וְאַרְבַּ֣ע אַמּ֔וֹת רֹ֖חַב הַיְרִיעָ֣ה הָאֶחָ֑ת מִדָּ֣ה אַחַ֔ת לְעַשְׁתֵּ֥י עֶשְׂרֵ֖ה יְרִיעֹֽת׃}
{אוּרְכָּא דִּירִיעֲתָא חֲדָא תְּלָתִין אַמִּין וְאַרְבַּע אַמִּין פּוּתְיָא דִּירִיעֲתָא חֲדָא מִשְׁחֲתָא חֲדָא לַחֲדָא עֶשְׂרֵי יְרִיעָן׃}
{The length of each curtain was thirty cubits, and four cubits the breadth of each curtain; the eleven curtains had one measure.}{\arabic{verse}}
\threeverse{\arabic{verse}}%Ex.36:16
{וַיְחַבֵּ֛ר אֶת־חֲמֵ֥שׁ הַיְרִיעֹ֖ת לְבָ֑ד וְאֶת־שֵׁ֥שׁ הַיְרִיעֹ֖ת לְבָֽד׃}
{וְלָפֵיף יָת חֲמֵישׁ יְרִיעָן לְחוֹד וְיָת שֵׁית יְרִיעָן לְחוֹד׃}
{And he coupled five curtains by themselves, and six curtains by themselves.}{\arabic{verse}}
\threeverse{\arabic{verse}}%Ex.36:17
{וַיַּ֜עַשׂ לֻֽלָאֹ֣ת חֲמִשִּׁ֗ים עַ֚ל שְׂפַ֣ת הַיְרִיעָ֔ה הַקִּיצֹנָ֖ה בַּמַּחְבָּ֑רֶת וַחֲמִשִּׁ֣ים לֻלָאֹ֗ת עָשָׂה֙ עַל־שְׂפַ֣ת הַיְרִיעָ֔ה הַחֹבֶ֖רֶת הַשֵּׁנִֽית׃}
{וַעֲבַד עֲנוּבִּין חַמְשִׁין עַל סִפְתָּא דִּירִיעֲתָא בְּסִטְרָא בֵּית לוֹפֵי וְחַמְשִׁין עֲנוּבִּין עֲבַד עַל סִפְתָא דִּירִיעֲתָא דְּבֵית לוֹפֵי תִּנְיָנָא׃}
{And he made fifty loops on the edge of the curtain that was outmost in the first set, and fifty loops made he upon the edge of the curtain which was outmost in the second set.}{\arabic{verse}}
\threeverse{\arabic{verse}}%Ex.36:18
{וַיַּ֛עַשׂ קַרְסֵ֥י נְחֹ֖שֶׁת חֲמִשִּׁ֑ים לְחַבֵּ֥ר אֶת־הָאֹ֖הֶל לִֽהְיֹ֥ת אֶחָֽד׃}
{וַעֲבַד פּוּרְפִין דִּנְחָשׁ חַמְשִׁין לְלָפָפָא יָת מַשְׁכְּנָא לְמִהְוֵי חַד׃}
{And he made fifty clasps of brass to couple the tent together, that it might be one.}{\arabic{verse}}
\threeverse{\arabic{verse}}%Ex.36:19
{וַיַּ֤עַשׂ מִכְסֶה֙ לָאֹ֔הֶל עֹרֹ֥ת אֵילִ֖ם מְאׇדָּמִ֑ים וּמִכְסֵ֛ה עֹרֹ֥ת תְּחָשִׁ֖ים מִלְמָֽעְלָה׃ \setuma         }
{וַעֲבַד חוּפָאָה לְמַשְׁכְּנָא דְּמַשְׁכֵּי דִּכְרֵי מְסֻמְּקֵי וְחוּפָאָה דְּמַשְׁכֵּי סָסְגוֹנָא מִלְּעֵילָא׃}
{And he made a covering for the tent of rams’ skins dyed red, and a covering of sealskins above.}{\arabic{verse}}
\threeverse{\aliya{חמישי}}%Ex.36:20
{וַיַּ֥עַשׂ אֶת־הַקְּרָשִׁ֖ים לַמִּשְׁכָּ֑ן עֲצֵ֥י שִׁטִּ֖ים עֹמְדִֽים׃}
{וַעֲבַד יָת דַּפַּיָּא לְמַשְׁכְּנָא דְּאָעֵי שִׁטִּין קָיְמִין׃}
{And he made the boards for the tabernacle of acacia-wood, standing up.}{\arabic{verse}}
\threeverse{\arabic{verse}}%Ex.36:21
{עֶ֥שֶׂר אַמֹּ֖ת אֹ֣רֶךְ הַקָּ֑רֶשׁ וְאַמָּה֙ וַחֲצִ֣י הָֽאַמָּ֔ה רֹ֖חַב הַקֶּ֥רֶשׁ הָאֶחָֽד׃}
{עֲשַׂר אַמִּין אוּרְכָּא דְּדַפָּא וְאַמְּתָא וּפַלְגוּת אַמְּתָא פּוּתְיָא דְּדַפָּא חַד׃}
{Ten cubits was the length of a board, and a cubit and a half the breadth of each board.}{\arabic{verse}}
\threeverse{\arabic{verse}}%Ex.36:22
{שְׁתֵּ֣י יָדֹ֗ת לַקֶּ֙רֶשׁ֙ הָֽאֶחָ֔ד מְשֻׁ֨לָּבֹ֔ת אַחַ֖ת אֶל־אֶחָ֑ת כֵּ֣ן עָשָׂ֔ה לְכֹ֖ל קַרְשֵׁ֥י הַמִּשְׁכָּֽן׃}
{תְּרֵין צִירִין לְדַפָּא חַד מְשׁוּלְּבִין חַד לָקֳבֵיל חַד כֵּן עֲבַד לְכֹל דַּפֵּי מַשְׁכְּנָא׃}
{Each board had two tenons, joined one to another. Thus did he make for all the boards of the tabernacle.}{\arabic{verse}}
\threeverse{\arabic{verse}}%Ex.36:23
{וַיַּ֥עַשׂ אֶת־הַקְּרָשִׁ֖ים לַמִּשְׁכָּ֑ן עֶשְׂרִ֣ים קְרָשִׁ֔ים לִפְאַ֖ת נֶ֥גֶב תֵּימָֽנָה׃}
{וַעֲבַד יָת דַּפַּיָּא לְמַשְׁכְּנָא עֶשְׂרִין דַּפִּין לְרוּחַ עֵיבַר דָּרוֹמָא׃}
{And he made the boards for the tabernacle; twenty boards for the south side southward.}{\arabic{verse}}
\threeverse{\arabic{verse}}%Ex.36:24
{וְאַרְבָּעִים֙ אַדְנֵי־כֶ֔סֶף עָשָׂ֕ה תַּ֖חַת עֶשְׂרִ֣ים הַקְּרָשִׁ֑ים שְׁנֵ֨י אֲדָנִ֜ים תַּֽחַת־הַקֶּ֤רֶשׁ הָאֶחָד֙ לִשְׁתֵּ֣י יְדֹתָ֔יו וּשְׁנֵ֧י אֲדָנִ֛ים תַּֽחַת־הַקֶּ֥רֶשׁ הָאֶחָ֖ד לִשְׁתֵּ֥י יְדֹתָֽיו׃}
{וְאַרְבְּעִין סָמְכִין דִּכְסַף עֲבַד תְּחוֹת עֶשְׂרִין דַּפִּין תְּרֵין סָמְכִין תְּחוֹת דַּפָּא חַד לִתְרֵין צִירוֹהִי וּתְרֵין סָמְכִין תְּחוֹת דַּפָּא חַד לִתְרֵין צִירוֹהִי׃}
{And he made forty sockets of silver under the twenty boards: two sockets under one board for its two tenons, and two sockets under another board for its two tenons.}{\arabic{verse}}
\threeverse{\arabic{verse}}%Ex.36:25
{וּלְצֶ֧לַע הַמִּשְׁכָּ֛ן הַשֵּׁנִ֖ית לִפְאַ֣ת צָפ֑וֹן עָשָׂ֖ה עֶשְׂרִ֥ים קְרָשִֽׁים׃}
{וְלִסְטַר מַשְׁכְּנָא תִּנְיָנָא לְרוּחַ צִפּוּנָא עֲבַד עֶשְׂרִין דַּפִּין׃}
{And for the second side of the tabernacle, on the north side, he made twenty boards,}{\arabic{verse}}
\threeverse{\arabic{verse}}%Ex.36:26
{וְאַרְבָּעִ֥ים אַדְנֵיהֶ֖ם כָּ֑סֶף שְׁנֵ֣י אֲדָנִ֗ים תַּ֚חַת הַקֶּ֣רֶשׁ הָאֶחָ֔ד וּשְׁנֵ֣י אֲדָנִ֔ים תַּ֖חַת הַקֶּ֥רֶשׁ הָאֶחָֽד׃}
{וְאַרְבְּעִין סָמְכֵיהוֹן דִּכְסַף תְּרֵין סָמְכִין תְּחוֹת דַּפָּא חַד וּתְרֵין סָמְכִין תְּחוֹת דַּפָּא חַד׃}
{and their forty sockets of silver: two sockets under one board, and two sockets under another board.}{\arabic{verse}}
\threeverse{\arabic{verse}}%Ex.36:27
{וּֽלְיַרְכְּתֵ֥י הַמִּשְׁכָּ֖ן יָ֑מָּה עָשָׂ֖ה שִׁשָּׁ֥ה קְרָשִֽׁים׃}
{וְלִסְיָפֵי מַשְׁכְּנָא מַעְרְבָא עֲבַד שִׁתָּא דַּפִּין׃}
{And for the hinder part of the tabernacle westward he made six boards.}{\arabic{verse}}
\threeverse{\arabic{verse}}%Ex.36:28
{וּשְׁנֵ֤י קְרָשִׁים֙ עָשָׂ֔ה לִמְקֻצְעֹ֖ת הַמִּשְׁכָּ֑ן בַּיַּרְכָתָֽיִם׃}
{וּתְרֵין דַּפִּין עֲבַד לְזָוְיָת מַשְׁכְּנָא בְּסוֹפְהוֹן׃}
{And two boards made he for the corners of the tabernacle in the hinder part;}{\arabic{verse}}
\threeverse{\arabic{verse}}%Ex.36:29
{וְהָי֣וּ תוֹאֲמִם֮ מִלְּמַ֒טָּה֒ וְיַחְדָּ֗ו יִהְי֤וּ תַמִּים֙ אֶל־רֹאשׁ֔וֹ אֶל־הַטַּבַּ֖עַת הָאֶחָ֑ת כֵּ֚ן עָשָׂ֣ה לִשְׁנֵיהֶ֔ם לִשְׁנֵ֖י הַמִּקְצֹעֹֽת׃}
{וַהֲווֹ מַכְוְנִין מִלְּרַע וְכַחְדָּא הֲווֹ מַכְוְנִין בְּרֵישֵׁיהוֹן בְּעִזְקְתָא חַדָא כֵּן עֲבַד לְתַרְוֵיהוֹן לְתַרְתֵּין זָוְיָן׃}
{that they might be double beneath, and in like manner they should be complete unto the top thereof unto the first ring. Thus he did to both of them in the two corners.}{\arabic{verse}}
\threeverse{\arabic{verse}}%Ex.36:30
{וְהָיוּ֙ שְׁמֹנָ֣ה קְרָשִׁ֔ים וְאַדְנֵיהֶ֣ם כֶּ֔סֶף שִׁשָּׁ֥ה עָשָׂ֖ר אֲדָנִ֑ים שְׁנֵ֤י אֲדָנִים֙ שְׁנֵ֣י אֲדָנִ֔ים תַּ֖חַת הַקֶּ֥רֶשׁ הָאֶחָֽד׃}
{וַהֲווֹ תְּמָנְיָא דַּפִּין וְסָמְכֵיהוֹן דִּכְסַף שִׁתַּת עֲשַׂר סָמְכִין תְּרֵין סָמְכִין תְּרֵין סָמְכִין תְּחוֹת דַּפָּא חַד׃}
{And there were eight boards, and their sockets of silver, sixteen sockets: under every board two sockets.}{\arabic{verse}}
\threeverse{\arabic{verse}}%Ex.36:31
{וַיַּ֥עַשׂ בְּרִיחֵ֖י עֲצֵ֣י שִׁטִּ֑ים חֲמִשָּׁ֕ה לְקַרְשֵׁ֥י צֶֽלַע־הַמִּשְׁכָּ֖ן הָאֶחָֽת׃}
{וַעֲבַד עָבְרֵי דְּאָעֵי שִׁטִּין חַמְשָׁא לְדַפֵּי סְטַר מַשְׁכְּנָא חַד׃}
{And he made bars of acacia-wood: five for the boards of the one side of the tabernacle,}{\arabic{verse}}
\threeverse{\arabic{verse}}%Ex.36:32
{וַחֲמִשָּׁ֣ה בְרִיחִ֔ם לְקַרְשֵׁ֥י צֶֽלַע־הַמִּשְׁכָּ֖ן הַשֵּׁנִ֑ית וַחֲמִשָּׁ֤ה בְרִיחִם֙ לְקַרְשֵׁ֣י הַמִּשְׁכָּ֔ן לַיַּרְכָתַ֖יִם יָֽמָּה׃}
{וְחַמְשָׁא עָבְרִין לְדַפֵּי סְטַר מַשְׁכְּנָא תִּנְיָנָא וְחַמְשָׁא עָבְרִין לְדַפֵּי מַשְׁכְּנָא לְסוֹפְהוֹן מַעְרְבָא׃}
{and five bars for the boards of the other side of the tabernacle, and five bars for the boards of the tabernacle for the hinder part westward.}{\arabic{verse}}
\threeverse{\arabic{verse}}%Ex.36:33
{וַיַּ֖עַשׂ אֶת־הַבְּרִ֣יחַ הַתִּיכֹ֑ן לִבְרֹ֙חַ֙ בְּת֣וֹךְ הַקְּרָשִׁ֔ים מִן־הַקָּצֶ֖ה אֶל־הַקָּצֶֽה׃}
{וַעֲבַד יָת עָבְרָא מְצִיעָאָה לְאַעְבָּרָא בְּגוֹ דַּפַּיָּא מִן סְיָפֵי לִסְיָפֵי׃}
{And he made the middle bar to pass through in the midst of the boards from the one end to the other.}{\arabic{verse}}
\threeverse{\arabic{verse}}%Ex.36:34
{וְֽאֶת־הַקְּרָשִׁ֞ים צִפָּ֣ה זָהָ֗ב וְאֶת־טַבְּעֹתָם֙ עָשָׂ֣ה זָהָ֔ב בָּתִּ֖ים לַבְּרִיחִ֑ם וַיְצַ֥ף אֶת־הַבְּרִיחִ֖ם זָהָֽב׃}
{וְיָת דַּפַּיָּא חֲפָא דַּהְבָּא וְיָת עִזְקָתְהוֹן עֲבַד דַּהְבָּא אַתְרָא לְעָבְרַיָּא וַחֲפָא יָת עָבְרַיָּא דַּהְבָּא׃}
{And he overlaid the boards with gold, and made their rings of gold for holders for the bars, and overlaid the bars with gold.}{\arabic{verse}}
\threeverse{\arabic{verse}}%Ex.36:35
{וַיַּ֙עַשׂ֙ אֶת־הַפָּרֹ֔כֶת תְּכֵ֧לֶת וְאַרְגָּמָ֛ן וְתוֹלַ֥עַת שָׁנִ֖י וְשֵׁ֣שׁ מׇשְׁזָ֑ר מַעֲשֵׂ֥ה חֹשֵׁ֛ב עָשָׂ֥ה אֹתָ֖הּ כְּרֻבִֽים׃}
{וַעֲבַד יָת פָּרוּכְתָּא דְּתַכְלָא וְאַרְגְּוָנָא וּצְבַע זְהוֹרִי וּבוּץ שְׁזִיר עוֹבָד אוּמָּן עֲבַד יָתַהּ צוּרַת כְּרוּבִין׃}
{And he made the veil of blue, and purple, and scarlet, and fine twined linen; with the cherubim the work of the skilful workman made he it.}{\arabic{verse}}
\threeverse{\arabic{verse}}%Ex.36:36
{וַיַּ֣עַשׂ לָ֗הּ אַרְבָּעָה֙ עַמּוּדֵ֣י שִׁטִּ֔ים וַיְצַפֵּ֣ם זָהָ֔ב וָוֵיהֶ֖ם זָהָ֑ב וַיִּצֹ֣ק לָהֶ֔ם אַרְבָּעָ֖ה אַדְנֵי־כָֽסֶף׃}
{וַעֲבַד לַהּ אַרְבְּעָא עַמּוּדֵי שִׁטִּין וַחֲפָנוּן דַּהְבָּא וָוֵיהוֹן דַּהְבָּא וְאַתֵּיךְ לְהוֹן אַרְבְּעָא סָמְכִין דִּכְסַף׃}
{And he made thereunto four pillars of acacia, and overlaid them with gold, their hooks being of gold; and he cast for them four sockets of silver.}{\arabic{verse}}
\threeverse{\arabic{verse}}%Ex.36:37
{וַיַּ֤עַשׂ מָסָךְ֙ לְפֶ֣תַח הָאֹ֔הֶל תְּכֵ֧לֶת וְאַרְגָּמָ֛ן וְתוֹלַ֥עַת שָׁנִ֖י וְשֵׁ֣שׁ מׇשְׁזָ֑ר מַעֲשֵׂ֖ה רֹקֵֽם׃}
{וַעֲבַד פְּרָסָא לִתְרַע מַשְׁכְּנָא דְּתַכְלָא וְאַרְגְּוָנָא וּצְבַע זְהוֹרִי וּבוּץ שְׁזִיר עוֹבָד צַיָּיר׃}
{And he made a screen for the door of the Tent, of blue, and purple, and scarlet, and fine twined linen, the work of the weaver in colours;}{\arabic{verse}}
\threeverse{\arabic{verse}}%Ex.36:38
{וְאֶת־עַמּוּדָ֤יו חֲמִשָּׁה֙ וְאֶת־וָ֣וֵיהֶ֔ם וְצִפָּ֧ה רָאשֵׁיהֶ֛ם וַחֲשֻׁקֵיהֶ֖ם זָהָ֑ב וְאַדְנֵיהֶ֥ם חֲמִשָּׁ֖ה נְחֹֽשֶׁת׃ \petucha }
{וְיָת עַמּוּדוֹהִי חַמְשָׁא וְיָת וָוֵיהוֹן וְחַפִּי רֵישֵׁיהוֹן וְכִבּוּשֵׁיהוֹן דַּהְבָּא וְסָמְכֵיהוֹן חַמְשָׁא דִּנְחָשָׁא׃}
{and the five pillars of it with their hooks; and he overlaid their capitals and their fillets with gold; and their five sockets were of brass.}{\arabic{verse}}
\newperek
\threeverse{\Roman{chap}}%Ex.37:1
{וַיַּ֧עַשׂ בְּצַלְאֵ֛ל אֶת־הָאָרֹ֖ן עֲצֵ֣י שִׁטִּ֑ים אַמָּתַ֨יִם וָחֵ֜צִי אׇרְכּ֗וֹ וְאַמָּ֤ה וָחֵ֙צִי֙ רׇחְבּ֔וֹ וְאַמָּ֥ה וָחֵ֖צִי קֹמָתֽוֹ׃
\rashi{\rashiDH{ויעש בצלאל. }לפי שנתן נפשו על המלאכה יותר משאר חכמים, נקראת על שמו׃ 
}}
{וַעֲבַד בְּצַלְאֵל יָת אֲרוֹנָא דְּאָעֵי שִׁטִּין תַּרְתֵּין אַמִּין וּפַלְגָּא אוּרְכֵּיהּ וְאַמְּתָא וּפַלְגָּא פּוּתְיֵיהּ וְאַמְּתָא וּפַלְגָּא רוּמֵיהּ׃}
{And Bezalel made the ark of acacia-wood: two cubits and a half was the length of it, and a cubit and a half the breadth of it, and a cubit and a half the height of it.}{\Roman{chap}}
\threeverse{\arabic{verse}}%Ex.37:2
{וַיְצַפֵּ֛הוּ זָהָ֥ב טָה֖וֹר מִבַּ֣יִת וּמִח֑וּץ וַיַּ֥עַשׂ ל֛וֹ זֵ֥ר זָהָ֖ב סָבִֽיב׃}
{וַחֲפָהִי דְּהַב דְּכֵי מִגָּיו וּמִבַּרָא וַעֲבַד לֵיהּ זִיר דִּדְהַב סְחוֹר סְחוֹר׃}
{And he overlaid it with pure gold within and without, and made a crown of gold to it round about.}{\arabic{verse}}
\threeverse{\arabic{verse}}%Ex.37:3
{וַיִּצֹ֣ק ל֗וֹ אַרְבַּע֙ טַבְּעֹ֣ת זָהָ֔ב עַ֖ל אַרְבַּ֣ע פַּעֲמֹתָ֑יו וּשְׁתֵּ֣י טַבָּעֹ֗ת עַל־צַלְעוֹ֙ הָֽאֶחָ֔ת וּשְׁתֵּי֙ טַבָּעֹ֔ת עַל־צַלְע֖וֹ הַשֵּׁנִֽית׃}
{וְאַתֵּיךְ לֵיהּ אַרְבַּע עִזְקָן דִּדְהַב עַל אַרְבַּע זָוְיָתֵיהּ וְתַרְתֵּין עִזְקָן עַל סִטְרֵיהּ חַד וְתַרְתֵּין עִזְקָן עַל סִטְרֵיהּ תִּנְיָנָא׃}
{And he cast for it four rings of gold, in the four feet thereof: even two rings on the one side of it, and two rings on the other side of it.}{\arabic{verse}}
\threeverse{\arabic{verse}}%Ex.37:4
{וַיַּ֥עַשׂ בַּדֵּ֖י עֲצֵ֣י שִׁטִּ֑ים וַיְצַ֥ף אֹתָ֖ם זָהָֽב׃}
{וַעֲבַד אֲרִיחֵי דְּאָעֵי שִׁטִּין וַחֲפָא יָתְהוֹן דַּהְבָּא׃}
{And he made staves of acacia-wood, and overlaid them with gold.}{\arabic{verse}}
\threeverse{\arabic{verse}}%Ex.37:5
{וַיָּבֵ֤א אֶת־הַבַּדִּים֙ בַּטַּבָּעֹ֔ת עַ֖ל צַלְעֹ֣ת הָאָרֹ֑ן לָשֵׂ֖את אֶת־הָאָרֹֽן׃}
{וְאַעֵיל יָת אֲרִיחַיָּא בְּעִזְקָתָא עַל סִטְרֵי אֲרוֹנָא לְמִטַּל יָת אֲרוֹנָא׃}
{And he put the staves into the rings on the sides of the ark, to bear the ark.}{\arabic{verse}}
\threeverse{\arabic{verse}}%Ex.37:6
{וַיַּ֥עַשׂ כַּפֹּ֖רֶת זָהָ֣ב טָה֑וֹר אַמָּתַ֤יִם וָחֵ֙צִי֙ אׇרְכָּ֔הּ וְאַמָּ֥ה וָחֵ֖צִי רׇחְבָּֽהּ׃}
{וַעֲבַד כָּפוּרְתָּא דִּדְהַב דְּכֵי תַּרְתֵּין אַמִּין וּפַלְגָּא אוּרְכַּהּ וְאַמְּתָא וּפַלְגָּא פוּתְיַהּ׃}
{And he made an ark-cover of pure gold: two cubits and a half was the length thereof, and a cubit and a half the breadth thereof.}{\arabic{verse}}
\threeverse{\arabic{verse}}%Ex.37:7
{וַיַּ֛עַשׂ שְׁנֵ֥י כְרֻבִ֖ים זָהָ֑ב מִקְשָׁה֙ עָשָׂ֣ה אֹתָ֔ם מִשְּׁנֵ֖י קְצ֥וֹת הַכַּפֹּֽרֶת׃}
{וַעֲבַד תְּרֵין כְּרוּבִין דִּדְהַב נְגִיד עֲבַד יָתְהוֹן מִתְּרֵין סִטְרֵי כָּפוּרְתָּא׃}
{And he made two cherubim of gold: of beaten work made he them, at the two ends of the ark-cover:}{\arabic{verse}}
\threeverse{\arabic{verse}}%Ex.37:8
{כְּרוּב־אֶחָ֤ד מִקָּצָה֙ מִזֶּ֔ה וּכְרוּב־אֶחָ֥ד מִקָּצָ֖ה מִזֶּ֑ה מִן־הַכַּפֹּ֛רֶת עָשָׂ֥ה אֶת־הַכְּרֻבִ֖ים מִשְּׁנֵ֥י \qk{קְצוֹתָֽיו}{קצוותו}׃}
{כְּרוּבָא חַד מִסִּטְרָא מִכָּא וּכְרוּבָא חַד מִסִּטְרָא מִכָּא מִן כָּפוּרְתָּא עֲבַד יָת כְּרוּבַיָּא מִתְּרֵין סִטְרוֹהִי׃}
{one cherub at the one end, and one cherub at the other end; of one piece with the ark-cover made he the cherubim at the two ends thereof.}{\arabic{verse}}
\threeverse{\arabic{verse}}%Ex.37:9
{וַיִּהְי֣וּ הַכְּרֻבִים֩ פֹּרְשֵׂ֨י כְנָפַ֜יִם לְמַ֗עְלָה סֹֽכְכִ֤ים בְּכַנְפֵיהֶם֙ עַל־הַכַּפֹּ֔רֶת וּפְנֵיהֶ֖ם אִ֣ישׁ אֶל־אָחִ֑יו אֶ֨ל־הַכַּפֹּ֔רֶת הָי֖וּ פְּנֵ֥י הַכְּרֻבִֽים׃ \petucha }
{וַהֲווֹ כְּרוּבַיָּא פְּרִיסִין גַּדְפֵּיהוֹן לְעֵילָא מְטַלַּן בְּגַדְפֵּיהוֹן עַל כָּפוּרְתָּא וְאַפֵּיהוֹן חַד לָקֳבֵיל חַד לָקֳבֵיל כָּפוּרְתָּא הֲווֹ אַפֵּי כְּרוּבַיָּא׃}
{And the cherubim spread out their wings on high, screening the ark-cover with their wings, with their faces one to another; toward the ark-cover were the faces of the cherubim.}{\arabic{verse}}
\threeverse{\arabic{verse}}%Ex.37:10
{וַיַּ֥עַשׂ אֶת־הַשֻּׁלְחָ֖ן עֲצֵ֣י שִׁטִּ֑ים אַמָּתַ֤יִם אׇרְכּוֹ֙ וְאַמָּ֣ה רׇחְבּ֔וֹ וְאַמָּ֥ה וָחֵ֖צִי קֹמָתֽוֹ׃}
{וַעֲבַד יָת פָּתוּרָא דְּאָעֵי שִׁטִּין תַּרְתֵּין אַמִּין אוּרְכֵּיהּ וְאַמְּתָא פוּתְיֵיהּ וְאַמְּתָא וּפַלְגָּא רוּמֵיהּ׃}
{And he made the table of acacia-wood: two cubits was the length thereof, and a cubit the breadth thereof, and a cubit and a half the height thereof.}{\arabic{verse}}
\threeverse{\arabic{verse}}%Ex.37:11
{וַיְצַ֥ף אֹת֖וֹ זָהָ֣ב טָה֑וֹר וַיַּ֥עַשׂ ל֛וֹ זֵ֥ר זָהָ֖ב סָבִֽיב׃}
{וַחֲפָא יָתֵיהּ דְּהַב דְּכֵי וַעֲבַד לֵיהּ זִיר דִּדְהַב סְחוֹר סְחוֹר׃}
{And he overlaid it with pure gold, and made thereto a crown of gold round about.}{\arabic{verse}}
\threeverse{\arabic{verse}}%Ex.37:12
{וַיַּ֨עַשׂ ל֥וֹ מִסְגֶּ֛רֶת טֹ֖פַח סָבִ֑יב וַיַּ֧עַשׂ זֵר־זָהָ֛ב לְמִסְגַּרְתּ֖וֹ סָבִֽיב׃}
{וַעֲבַד לֵיהּ גְּדָנְפָא רוּמֵיהּ פּוּשְׁכָּא סְחוֹר סְחוֹר וַעֲבַד זִיר דִּדְהַב לִגְדָנְפֵיהּ סְחוֹר סְחוֹר׃}
{And he made unto it a border of a hand-breadth round about, and made a golden crown to the border thereof round about.}{\arabic{verse}}
\threeverse{\arabic{verse}}%Ex.37:13
{וַיִּצֹ֣ק ל֔וֹ אַרְבַּ֖ע טַבְּעֹ֣ת זָהָ֑ב וַיִּתֵּן֙ אֶת־הַטַּבָּעֹ֔ת עַ֚ל אַרְבַּ֣ע הַפֵּאֹ֔ת אֲשֶׁ֖ר לְאַרְבַּ֥ע רַגְלָֽיו׃}
{וְאַתֵּיךְ לֵיהּ אַרְבַּע עִזְקָן דִּדְהַב וִיהַב יָת עִזְקָתָא עַל אַרְבַּע זָוְיָתָא דִּלְאַרְבַּע רַגְלוֹהִי׃}
{And he cast for it four rings of gold, and put the rings in the four corners that were on the four feet thereof.}{\arabic{verse}}
\threeverse{\arabic{verse}}%Ex.37:14
{לְעֻמַּת֙ הַמִּסְגֶּ֔רֶת הָי֖וּ הַטַּבָּעֹ֑ת בָּתִּים֙ לַבַּדִּ֔ים לָשֵׂ֖את אֶת־הַשֻּׁלְחָֽן׃}
{לָקֳבֵיל גְּדָנְפָא הֲוַאָה עִזְקָתָא אַתְרָא לַאֲרִיחַיָּא לְמִטַּל יָת פָּתוּרָא׃}
{Close by the border were the rings, the holders for the staves to bear the table.}{\arabic{verse}}
\threeverse{\arabic{verse}}%Ex.37:15
{וַיַּ֤עַשׂ אֶת־הַבַּדִּים֙ עֲצֵ֣י שִׁטִּ֔ים וַיְצַ֥ף אֹתָ֖ם זָהָ֑ב לָשֵׂ֖את אֶת־הַשֻּׁלְחָֽן׃}
{וַעֲבַד יָת אֲרִיחַיָּא דְּאָעֵי שִׁטִּין וַחֲפָא יָתְהוֹן דַּהְבָּא לְמִטַּל יָת פָּתוּרָא׃}
{And he made the staves of acacia-wood, and overlaid them with gold, to bear the table.}{\arabic{verse}}
\threeverse{\arabic{verse}}%Ex.37:16
{וַיַּ֜עַשׂ אֶֽת־הַכֵּלִ֣ים \legarmeh  אֲשֶׁ֣ר עַל־הַשֻּׁלְחָ֗ן אֶת־קְעָרֹתָ֤יו וְאֶת־כַּפֹּתָיו֙ וְאֵת֙ מְנַקִּיֹּתָ֔יו וְאֶ֨ת־הַקְּשָׂוֺ֔ת אֲשֶׁ֥ר יֻסַּ֖ךְ בָּהֵ֑ן זָהָ֖ב טָהֽוֹר׃ \petucha }
{וַעֲבַד יָת מָנַיָּא דְּעַל פָּתוּרָא יָת מְגִסּוֹהִי וְיָת בָּזִכּוֹהִי וְיָת מְכִילָתֵיהּ וְיָת קָסְוָתָא דְּיִתְנַסַּךְ בְּהוֹן דִּדְהַב דְּכֵי׃}
{And he made the vessels which were upon the table, the dishes thereof, and the pans thereof, and the bowls thereof, and the jars thereof, wherewith to pour out, of pure gold.}{\arabic{verse}}
\threeverse{\aliya{ששי\newline (שלישי)}}%Ex.37:17
{וַיַּ֥עַשׂ אֶת־הַמְּנֹרָ֖ה זָהָ֣ב טָה֑וֹר מִקְשָׁ֞ה עָשָׂ֤ה אֶת־הַמְּנֹרָה֙ יְרֵכָ֣הּ וְקָנָ֔הּ גְּבִיעֶ֛יהָ כַּפְתֹּרֶ֥יהָ וּפְרָחֶ֖יהָ מִמֶּ֥נָּה הָיֽוּ׃}
{וַעֲבַד יָת מְנָרְתָא דִּדְהַב דְּכֵי נְגִיד עֲבַד יָת מְנָרְתָא שִׁדַּהּ וּקְנַהּ כַּלִּידַהָא חַזּוּרַהָא וְשׁוֹשַׁנַּהָא מִנַּהּ הֲווֹ׃}
{And he made the candlestick of pure gold: of beaten work made he the candlestick, even its base, and its shaft; its cups, its knops, and its flowers, were of one piece with it.}{\arabic{verse}}
\threeverse{\arabic{verse}}%Ex.37:18
{וְשִׁשָּׁ֣ה קָנִ֔ים יֹצְאִ֖ים מִצִּדֶּ֑יהָ שְׁלֹשָׁ֣ה \legarmeh  קְנֵ֣י מְנֹרָ֗ה מִצִּדָּהּ֙ הָֽאֶחָ֔ד וּשְׁלֹשָׁה֙ קְנֵ֣י מְנֹרָ֔ה מִצִּדָּ֖הּ הַשֵּׁנִֽי׃}
{וְשִׁתָּא קְנִין נָפְקִין מִסִּטְרַהָא תְּלָתָא קְנֵי מְנָרְתָא מִסִּטְרַהּ חַד וּתְלָתָא קְנֵי מְנָרְתָא מִסִּטְרַהּ תִּנְיָנָא׃}
{And there were six branches going out of the sides thereof: three branches of the candlestick out of the one side thereof, and three branches of the candlestick out of the other side thereof;}{\arabic{verse}}
\threeverse{\arabic{verse}}%Ex.37:19
{שְׁלֹשָׁ֣ה גְ֠בִעִ֠ים מְֽשֻׁקָּדִ֞ים בַּקָּנֶ֣ה הָאֶחָד֮ כַּפְתֹּ֣ר וָפֶ֒רַח֒ וּשְׁלֹשָׁ֣ה גְבִעִ֗ים מְשֻׁקָּדִ֛ים בְּקָנֶ֥ה אֶחָ֖ד כַּפְתֹּ֣ר וָפָ֑רַח כֵּ֚ן לְשֵׁ֣שֶׁת הַקָּנִ֔ים הַיֹּצְאִ֖ים מִן־הַמְּנֹרָֽה׃}
{תְּלָתָא כַלִּידִין מְצָיְרִין בְּקַנְיָא חַד חַזּוּר וְשׁוֹשָׁן וּתְלָתָא כַלִּידִין מְצָיְרִין בְּקַנְיָא חַד חַזּוּר וְשׁוֹשָׁן כֵּן לְשִׁתָּא קְנִין דְּנָפְקִין מִן מְנָרְתָא׃}
{three cups made like almond-blossoms in one branch, a knop and a flower; and three cups made like almond-blossoms in the other branch, a knop and a flower. So for the six branches going out of the candlestick.}{\arabic{verse}}
\threeverse{\arabic{verse}}%Ex.37:20
{וּבַמְּנֹרָ֖ה אַרְבָּעָ֣ה גְבִעִ֑ים מְשֻׁ֨קָּדִ֔ים כַּפְתֹּרֶ֖יהָ וּפְרָחֶֽיהָ׃}
{וּבִמְנָרְתָא אַרְבְּעָא כַלִּידִין מְצָיְרִין חַזּוּרַהָא וְשׁוֹשַׁנַּהָא׃}
{And in the candlestick were four cups made like almond-blossoms, the knops thereof, and the flowers thereof;}{\arabic{verse}}
\threeverse{\arabic{verse}}%Ex.37:21
{וְכַפְתֹּ֡ר תַּ֩חַת֩ שְׁנֵ֨י הַקָּנִ֜ים מִמֶּ֗נָּה וְכַפְתֹּר֙ תַּ֣חַת שְׁנֵ֤י הַקָּנִים֙ מִמֶּ֔נָּה וְכַפְתֹּ֕ר תַּֽחַת־שְׁנֵ֥י הַקָּנִ֖ים מִמֶּ֑נָּה לְשֵׁ֙שֶׁת֙ הַקָּנִ֔ים הַיֹּצְאִ֖ים מִמֶּֽנָּה׃}
{וְחַזּוּר תְּחוֹת תְּרֵין קְנִין דְּמִנַּהּ וְחַזּוּר תְּחוֹת תְּרֵין קְנִין דְּמִנַּהּ וְחַזּוּר תְּחוֹת תְּרֵין קְנִין דְּמִנַּהּ לְשִׁתָּא קְנִין דְּנָפְקִין מִנַּהּ׃}
{and a knop under two branches of one piece with it, and a knop under two branches of one piece with it, and a knop under two branches of one piece with it, for the six branches going out of it.}{\arabic{verse}}
\threeverse{\arabic{verse}}%Ex.37:22
{כַּפְתֹּרֵיהֶ֥ם וּקְנֹתָ֖ם מִמֶּ֣נָּה הָי֑וּ כֻּלָּ֛הּ מִקְשָׁ֥ה אַחַ֖ת זָהָ֥ב טָהֽוֹר׃}
{חַזּוּרֵיהוֹן וּקְנֵיהוֹן מִנַּהּ הֲווֹ כּוּלַּהּ נְגִידָא חֲדָא דִּדְהַב דְּכֵי׃}
{Their knops and their branches were of one piece with it; the whole of it was one beaten work of pure gold.}{\arabic{verse}}
\threeverse{\arabic{verse}}%Ex.37:23
{וַיַּ֥עַשׂ אֶת־נֵרֹתֶ֖יהָ שִׁבְעָ֑ה וּמַלְקָחֶ֥יהָ וּמַחְתֹּתֶ֖יהָ זָהָ֥ב טָהֽוֹר׃}
{וַעֲבַד יָת בּוֹצִינַהָא שִׁבְעָא וְצֵיבְתַהָא וּמַחְתְּיָתַהָא דִּדְהַב דְּכֵי׃}
{And he made the lamps thereof, seven, and the tongs thereof, and the snuffdishes thereof, of pure gold.}{\arabic{verse}}
\threeverse{\arabic{verse}}%Ex.37:24
{כִּכָּ֛ר זָהָ֥ב טָה֖וֹר עָשָׂ֣ה אֹתָ֑הּ וְאֵ֖ת כׇּל־כֵּלֶֽיהָ׃ \petucha }
{כַּכְּרָא דְּדַהְבָּא דָּכְיָא עֲבַד יָתַהּ וְיָת כָּל מָנַהָא׃}
{Of a talent of pure gold made he it, and all the vessels thereof.}{\arabic{verse}}
\threeverse{\arabic{verse}}%Ex.37:25
{וַיַּ֛עַשׂ אֶת־מִזְבַּ֥ח הַקְּטֹ֖רֶת עֲצֵ֣י שִׁטִּ֑ים אַמָּ֣ה אׇרְכּוֹ֩ וְאַמָּ֨ה רׇחְבּ֜וֹ רָב֗וּעַ וְאַמָּתַ֙יִם֙ קֹֽמָת֔וֹ מִמֶּ֖נּוּ הָי֥וּ קַרְנֹתָֽיו׃}
{וַעֲבַד יָת מַדְבְּחָא דִּקְטֹרֶת בּוּסְמַיָּא דְּאָעֵי שִׁטִּין אַמְּתָא אוּרְכֵּיהּ וְאַמְּתָא פוּתְיֵיהּ מְרֻבַּע וְתַרְתֵּין אַמִּין רוּמֵיהּ מִנֵּיהּ הֲוַאָה קַרְנוֹהִי׃}
{And he made the altar of incense of acacia-wood: a cubit was the length thereof, and a cubit the breadth thereof, four-square; and two cubits was the height thereof; the horns thereof were of one piece with it.}{\arabic{verse}}
\threeverse{\arabic{verse}}%Ex.37:26
{וַיְצַ֨ף אֹת֜וֹ זָהָ֣ב טָה֗וֹר אֶת־גַּגּ֧וֹ וְאֶת־קִירֹתָ֛יו סָבִ֖יב וְאֶת־קַרְנֹתָ֑יו וַיַּ֥עַשׂ ל֛וֹ זֵ֥ר זָהָ֖ב סָבִֽיב׃}
{וַחֲפָא יָתֵיהּ דְּהַב דְּכֵי יָת אִגָּרֵיהּ וְיָת כּוּתְלוֹהִי סְחוֹר סְחוֹר וְיָת קַרְנוֹהִי וַעֲבַד לֵיהּ זִיר דִּדְהַב סְחוֹר סְחוֹר׃}
{And he overlaid it with pure gold, the top thereof, and the sides thereof round about, and the horns of it; and he made unto it a crown of gold round about.}{\arabic{verse}}
\threeverse{\arabic{verse}}%Ex.37:27
{וּשְׁתֵּי֩ טַבְּעֹ֨ת זָהָ֜ב עָֽשָׂה־ל֣וֹ \legarmeh  מִתַּ֣חַת לְזֵר֗וֹ עַ֚ל שְׁתֵּ֣י צַלְעֹתָ֔יו עַ֖ל שְׁנֵ֣י צִדָּ֑יו לְבָתִּ֣ים לְבַדִּ֔ים לָשֵׂ֥את אֹת֖וֹ בָּהֶֽם׃}
{וְתַרְתֵּין עִזְקָן דִּדְהַב עֲבַד לֵיהּ מִלְּרַע לְזֵירֵיהּ עַל תַּרְתֵּין זָוְיָתֵיהּ עַל תְּרֵין סִטְרוֹהִי לְאַתְרָא לַאֲרִיחַיָּא לְמִטַּל יָתֵיהּ בְּהוֹן׃}
{And he made for it two golden rings under the crown thereof, upon the two ribs thereof, upon the two sides of it, for holders for staves wherewith to bear it.}{\arabic{verse}}
\threeverse{\arabic{verse}}%Ex.37:28
{וַיַּ֥עַשׂ אֶת־הַבַּדִּ֖ים עֲצֵ֣י שִׁטִּ֑ים וַיְצַ֥ף אֹתָ֖ם זָהָֽב׃}
{וַעֲבַד יָת אֲרִיחַיָּא דְּאָעֵי שִׁטִּין וַחֲפָא יָתְהוֹן דַּהְבָּא׃}
{And he made the staves of acacia-wood, and overlaid them with gold.}{\arabic{verse}}
\threeverse{\arabic{verse}}%Ex.37:29
{וַיַּ֜עַשׂ אֶת־שֶׁ֤מֶן הַמִּשְׁחָה֙ קֹ֔דֶשׁ וְאֶת־קְטֹ֥רֶת הַסַּמִּ֖ים טָה֑וֹר מַעֲשֵׂ֖ה רֹקֵֽחַ׃ \setuma         }
{וַעֲבַד יָת מִשְׁחָא דִּרְבוּתָא קוּדְשָׁא וְיָת קְטֹרֶת בּוּסְמַיָּא דְּכֵי עוֹבָד בּוּסְמָנוּ׃}
{And he made the holy anointing oil, and the pure incense of sweet spices, after the art of the perfumer.}{\arabic{verse}}
\newperek
\threeverse{\aliya{שביעי\newline (רביעי)}}%Ex.38:1
{וַיַּ֛עַשׂ אֶת־מִזְבַּ֥ח הָעֹלָ֖ה עֲצֵ֣י שִׁטִּ֑ים חָמֵשׁ֩ אַמּ֨וֹת אׇרְכּ֜וֹ וְחָֽמֵשׁ־אַמּ֤וֹת רׇחְבּוֹ֙ רָב֔וּעַ וְשָׁלֹ֥שׁ אַמּ֖וֹת קֹמָתֽוֹ׃}
{וַעֲבַד יָת מַדְבְּחָא דַּעֲלָתָא דְּאָעֵי שִׁטִּין חֲמֵישׁ אַמִּין אוּרְכֵּיהּ וַחֲמֵישׁ אַמִּין פּוּתְיֵיהּ מְרֻבַּע וּתְלָת אַמִּין רוּמֵיהּ׃}
{And he made the altar of burnt-offering of acacia-wood: five cubits was the length thereof, and five cubits the breadth thereof, four-square, and three cubits the height thereof.}{\Roman{chap}}
\threeverse{\arabic{verse}}%Ex.38:2
{וַיַּ֣עַשׂ קַרְנֹתָ֗יו עַ֚ל אַרְבַּ֣ע פִּנֹּתָ֔יו מִמֶּ֖נּוּ הָי֣וּ קַרְנֹתָ֑יו וַיְצַ֥ף אֹת֖וֹ נְחֹֽשֶׁת׃}
{וַעֲבַד קַרְנוֹהִי עַל אַרְבַּע זָוְיָתֵיהּ מִנֵּיהּ הֲוָאָה קַרְנוֹהִי וַחֲפָא יָתֵיהּ נְחָשָׁא׃}
{And he made the horns thereof upon the four corners of it; the horns thereof were of one piece with it; and he overlaid it with brass.}{\arabic{verse}}
\threeverse{\arabic{verse}}%Ex.38:3
{וַיַּ֜עַשׂ אֶֽת־כׇּל־כְּלֵ֣י הַמִּזְבֵּ֗חַ אֶת־הַסִּירֹ֤ת וְאֶת־הַיָּעִים֙ וְאֶת־הַמִּזְרָקֹ֔ת אֶת־הַמִּזְלָגֹ֖ת וְאֶת־הַמַּחְתֹּ֑ת כׇּל־כֵּלָ֖יו עָשָׂ֥ה נְחֹֽשֶׁת׃}
{וַעֲבַד יָת כָּל מָנֵי מַדְבְּחָא יָת פְּסַכְתֵּירְוָתָא וְיָת מַגְרוֹפְיָתָא וְיָת מִזְרְקַיָּא יָת צִנּוֹרְיָתָא וְיָת מַחְתְּיָתָא כָּל מָנוֹהִי עֲבַד נְחָשָׁא׃}
{And he made all the vessels of the altar, the pots, and the shovels, and the basins, the flesh-hooks, and the fire-pans; all the vessels thereof made he of brass.}{\arabic{verse}}
\threeverse{\arabic{verse}}%Ex.38:4
{וַיַּ֤עַשׂ לַמִּזְבֵּ֙חַ֙ מִכְבָּ֔ר מַעֲשֵׂ֖ה רֶ֣שֶׁת נְחֹ֑שֶׁת תַּ֧חַת כַּרְכֻּבּ֛וֹ מִלְּמַ֖טָּה עַד־חֶצְיֽוֹ׃}
{וַעֲבַד לְמַדְבְּחָא סְרָדָא עוֹבָד מְצָדְתָא דִּנְחָשָׁא תְּחוֹת סוֹבֵיבֵיהּ מִלְּרַע עַד פַּלְגֵּיהּ׃}
{And he made for the altar a grating of network of brass, under the ledge round it beneath, reaching halfway up.}{\arabic{verse}}
\threeverse{\arabic{verse}}%Ex.38:5
{וַיִּצֹ֞ק אַרְבַּ֧ע טַבָּעֹ֛ת בְּאַרְבַּ֥ע הַקְּצָוֺ֖ת לְמִכְבַּ֣ר הַנְּחֹ֑שֶׁת בָּתִּ֖ים לַבַּדִּֽים׃}
{וְאַתֵּיךְ אַרְבַּע עִזְקָן בְּאַרְבַּע זָוְיָתָא לִסְרָדָא דִּנְחָשָׁא אַתְרָא לַאֲרִיחַיָּא׃}
{And he cast four rings for the four ends of the grating of brass, to be holders for the staves.}{\arabic{verse}}
\threeverse{\arabic{verse}}%Ex.38:6
{וַיַּ֥עַשׂ אֶת־הַבַּדִּ֖ים עֲצֵ֣י שִׁטִּ֑ים וַיְצַ֥ף אֹתָ֖ם נְחֹֽשֶׁת׃}
{וַעֲבַד יָת אֲרִיחַיָּא דְּאָעֵי שִׁטִּין וַחֲפָא יָתְהוֹן נְחָשָׁא׃}
{And he made the staves of acacia-wood, and overlaid them with brass.}{\arabic{verse}}
\threeverse{\arabic{verse}}%Ex.38:7
{וַיָּבֵ֨א אֶת־הַבַּדִּ֜ים בַּטַּבָּעֹ֗ת עַ֚ל צַלְעֹ֣ת הַמִּזְבֵּ֔חַ לָשֵׂ֥את אֹת֖וֹ בָּהֶ֑ם נְב֥וּב לֻחֹ֖ת עָשָׂ֥ה אֹתֽוֹ׃ \setuma         
\rashi{\rashiDH{נבוב לוחות. }נבוב הוא חלול, וכן וְעָבְיֹו אַרְבַּע אֶצְבָּעֹות נָבוּב (ירמיה נב, כא)׃ }\rashi{\rashiDH{נבוב לוחות. }הלוחות של עצי שטים לכל רוח, והחלל באמצע׃ }}
{וְאַעֵיל יָת אֲרִיחַיָּא בְּעִזְקָתָא עַל סִטְרֵי מַדְבְּחָא לְמִטַּל יָתֵיהּ בְּהוֹן חֲלִיל לוּחִין עֲבַד יָתֵיהּ׃}
{And he put the staves into the rings on the sides of the altar, wherewith to bear it; he made it hollow with planks.}{\arabic{verse}}
\threeverse{\arabic{verse}}%Ex.38:8
{וַיַּ֗עַשׂ אֵ֚ת הַכִּיּ֣וֹר נְחֹ֔שֶׁת וְאֵ֖ת כַּנּ֣וֹ נְחֹ֑שֶׁת בְּמַרְאֹת֙ הַצֹּ֣בְאֹ֔ת אֲשֶׁ֣ר צָֽבְא֔וּ פֶּ֖תַח אֹ֥הֶל מוֹעֵֽד׃ \setuma         
\rashi{\rashiDH{במראות הצובאות. }בנות ישראל היו בידן מראות, שרואות בהן כשהן מתקשטות, ואף אותן לא עכבו מלהביא לנדבת המשכן, והיה מואס משה בהן, מפני שעשוים ליצר הרע, אמר לו הקב״ה קבל, כי אלו חביבין עלי מן הכל, שעל ידיהם העמידו הנשים צבאות רבות במצרים, כשהיו בעליהם יגעים בעבודת פרך, היו הולכות ומוליכות להם מאכל ומשתה ומאכילות אותם, ונוטלות המראות, וכל אחת רואה עצמה עם בעלה במראה, ומשדלתו בדברים, לומר אני נאה ממך, ומתוך כך מביאות לבעליהם לידי תאוה, ונזקקות להם, ומתעברות ויולדות שם, שנאמר תַּחַת הַתַּפּוּחַ עֹורַרְתִּיךָ (שיר השירים ח, ה), וזה שנאמר במראות הצובאות, ונעשה הכיור מהם, שהוא לשום שלום בין איש לאשתו, להשקות ממים שבתוכו למי שקנא לה בעלה ונסתרה, ותדע לך שהן מראות ממש, שהרי נאמר ונחשת התנופה שבעים ככר וגו׳ ויעש בה וגו׳, וכיור וכנו לא הוזכרו שם, למדת, שלא היה נחשת של כיור מנחשת התנופה, כך דרש רבי תנחומא (פקודי ט), וכן תרגם אונקלוס בְּמֶחְזְיַת נְשַׁיָא, והוא תרגום של מראות, מירוא״ש בלע״ז (שפיעגעל), וכן מצינו בישעיה (ג, כג), וְהַגִּלְיֹנִים, מתרגמינן וּמֶחְזְיָתָא׃ }\rashi{\rashiDH{אשר צבאו. }להביא נדבתן׃ 
}}
{וַעֲבַד יָת כִּיּוֹרָא דִּנְחָשָׁא וְיָת בְּסִיסֵיהּ דִּנְחָשָׁא בְּמִחְזְיָת נְשַׁיָּא דְּאָתְיָן לְצַלָּאָה בִּתְרַע מַשְׁכַּן זִמְנָא׃}
{And he made the laver of brass, and the base thereof of brass, of the mirrors of the serving women that did service at the door of the tent of meeting.}{\arabic{verse}}
\threeverse{\arabic{verse}}%Ex.38:9
{וַיַּ֖עַשׂ אֶת־הֶחָצֵ֑ר לִפְאַ֣ת \legarmeh  נֶ֣גֶב תֵּימָ֗נָה קַלְעֵ֤י הֶֽחָצֵר֙ שֵׁ֣שׁ מׇשְׁזָ֔ר מֵאָ֖ה בָּאַמָּֽה׃}
{וַעֲבַד יָת דָּרְתָא לְרוּחַ עֵיבַר דָּרוֹמָא סְרָדֵי דָּרְתָא דְּבוּץ שְׁזִיר מְאָה אַמִּין׃}
{And he made the court; for the south side southward the hangings of the court were of fine twined linen, a hundred cubits.}{\arabic{verse}}
\threeverse{\arabic{verse}}%Ex.38:10
{עַמּוּדֵיהֶ֣ם עֶשְׂרִ֔ים וְאַדְנֵיהֶ֥ם עֶשְׂרִ֖ים נְחֹ֑שֶׁת וָוֵ֧י הָעַמּוּדִ֛ים וַחֲשֻׁקֵיהֶ֖ם כָּֽסֶף׃}
{עַמּוּדֵיהוֹן עֶשְׂרִין וְסָמְכֵיהוֹן עֶשְׂרִין דִּנְחָשָׁא וָוֵי עַמּוּדַיָּא וְכִבּוּשֵׁיהוֹן כְּסַף׃}
{Their pillars were twenty, and their sockets twenty, of brass; the hooks of the pillars and their fillets were of silver.}{\arabic{verse}}
\threeverse{\arabic{verse}}%Ex.38:11
{וְלִפְאַ֤ת צָפוֹן֙ מֵאָ֣ה בָֽאַמָּ֔ה עַמּוּדֵיהֶ֣ם עֶשְׂרִ֔ים וְאַדְנֵיהֶ֥ם עֶשְׂרִ֖ים נְחֹ֑שֶׁת וָוֵ֧י הָֽעַמּוּדִ֛ים וַחֲשֻׁקֵיהֶ֖ם כָּֽסֶף׃}
{וּלְרוּחַ צִפּוּנָא מְאָה אַמִּין עַמּוּדֵיהוֹן עֶשְׂרִין וְסָמְכֵיהוֹן עֶשְׂרִין דִּנְחָשָׁא וָוֵי עַמּוּדַיָּא וְכִבּוּשֵׁיהוֹן כְּסַף׃}
{And for the north side a hundred cubits, their pillars twenty, and their sockets twenty, of brass; the hooks of the pillars and their fillets of silver.}{\arabic{verse}}
\threeverse{\arabic{verse}}%Ex.38:12
{וְלִפְאַת־יָ֗ם קְלָעִים֙ חֲמִשִּׁ֣ים בָּֽאַמָּ֔ה עַמּוּדֵיהֶ֣ם עֲשָׂרָ֔ה וְאַדְנֵיהֶ֖ם עֲשָׂרָ֑ה וָוֵ֧י הָעַמֻּדִ֛ים וַחֲשׁוּקֵיהֶ֖ם כָּֽסֶף׃}
{וּלְרוּחַ מַעְרָבָא סְרָדֵי חַמְשִׁין אַמִּין עַמּוּדֵיהוֹן עֶשְׂרָא וְסָמְכֵיהוֹן עֶשְׂרָא וָוֵי עַמּוּדַיָּא וְכִבּוּשֵׁיהוֹן כְּסַף׃}
{And for the west side were hangings of fifty cubits, their pillars ten, and their sockets ten; the hooks of the pillars and their fillets of silver.}{\arabic{verse}}
\threeverse{\arabic{verse}}%Ex.38:13
{וְלִפְאַ֛ת קֵ֥דְמָה מִזְרָ֖חָה חֲמִשִּׁ֥ים אַמָּֽה׃}
{וּלְרוּחַ קִדּוּמָא מַדְנְחָא חַמְשִׁין אַמִּין׃}
{And for the east side eastward fifty cubits.}{\arabic{verse}}
\threeverse{\arabic{verse}}%Ex.38:14
{קְלָעִ֛ים חֲמֵשׁ־עֶשְׂרֵ֥ה אַמָּ֖ה אֶל־הַכָּתֵ֑ף עַמּוּדֵיהֶ֣ם שְׁלֹשָׁ֔ה וְאַדְנֵיהֶ֖ם שְׁלֹשָֽׁה׃}
{סְרָדֵי חֲמֵישׁ עֶשְׂרָא אַמִּין לְעִבְרָא עַמּוּדֵיהוֹן תְּלָתָא וְסָמְכֵיהוֹן תְּלָתָא׃}
{The hangings for the one side [of the gate] were fifteen cubits; their pillars three, and their sockets three.}{\arabic{verse}}
\threeverse{\arabic{verse}}%Ex.38:15
{וְלַכָּתֵ֣ף הַשֵּׁנִ֗ית מִזֶּ֤ה וּמִזֶּה֙ לְשַׁ֣עַר הֶֽחָצֵ֔ר קְלָעִ֕ים חֲמֵ֥שׁ עֶשְׂרֵ֖ה אַמָּ֑ה עַמֻּדֵיהֶ֣ם שְׁלֹשָׁ֔ה וְאַדְנֵיהֶ֖ם שְׁלֹשָֽׁה׃}
{וּלְעִבְרָא תִּנְיָנָא מִכָּא וּמִכָּא לִתְרַע דָּרְתָא סְרָדֵי חֲמֵישׁ עֶשְׂרָא אַמִּין עַמּוּדֵיהוֹן תְּלָתָא וְסָמְכֵיהוֹן תְּלָתָא׃}
{And so for the other side; on this hand and that hand by the gate of the court were hangings of fifteen cubits; their pillars three, and their sockets three.}{\arabic{verse}}
\threeverse{\arabic{verse}}%Ex.38:16
{כׇּל־קַלְעֵ֧י הֶחָצֵ֛ר סָבִ֖יב שֵׁ֥שׁ מׇשְׁזָֽר׃}
{כָּל סְרָדֵי דָּרְתָא סְחוֹר סְחוֹר דְּבוּץ שְׁזִיר׃}
{All the hangings of the court round about were of fine twined linen.}{\arabic{verse}}
\threeverse{\arabic{verse}}%Ex.38:17
{וְהָאֲדָנִ֣ים לָֽעַמֻּדִים֮ נְחֹ֒שֶׁת֒ וָוֵ֨י הָֽעַמּוּדִ֜ים וַחֲשׁוּקֵיהֶם֙ כֶּ֔סֶף וְצִפּ֥וּי רָאשֵׁיהֶ֖ם כָּ֑סֶף וְהֵם֙ מְחֻשָּׁקִ֣ים כֶּ֔סֶף כֹּ֖ל עַמֻּדֵ֥י הֶחָצֵֽר׃}
{וְסָמְכַיָּא לְעַמּוּדַיָּא דִּנְחָשָׁא וָוֵי עַמּוּדַיָּא וְכִבּוּשֵׁיהוֹן כְּסַף וְחִפּוּי רֵישֵׁיהוֹן כְּסַף וְאִנּוּן מְכוּבְּשִׁין כְּסַף כֹּל עַמּוּדֵי דָּרְתָא׃}
{And the sockets for the pillars were of brass; the hooks of the pillars and their fillets of silver; and the overlaying of their capitals of silver; and all the pillars of the court were filleted with silver.}{\arabic{verse}}
\threeverse{\aliya{מפטיר}}%Ex.38:18
{וּמָסַ֞ךְ שַׁ֤עַר הֶחָצֵר֙ מַעֲשֵׂ֣ה רֹקֵ֔ם תְּכֵ֧לֶת וְאַרְגָּמָ֛ן וְתוֹלַ֥עַת שָׁנִ֖י וְשֵׁ֣שׁ מָשְׁזָ֑ר וְעֶשְׂרִ֤ים אַמָּה֙ אֹ֔רֶךְ וְקוֹמָ֤ה בְרֹ֙חַב֙ חָמֵ֣שׁ אַמּ֔וֹת לְעֻמַּ֖ת קַלְעֵ֥י הֶחָצֵֽר׃
\rashi{\rashiDH{לעמת קלעי החצר. }כמדת קלעי החצר׃ 
}}
{וּפְרָסָא דִּתְרַע דָּרְתָא עוֹבָד צַיָּיר דְּתַכְלָא וְאַרְגְּוָנָא וּצְבַע זְהוֹרִי וּבוּץ שְׁזִיר וְעֶשְׂרִין אַמִּין אוּרְכָּא וְרוּמָא בְּפוּתְיָא חֲמֵישׁ אַמִּין לָקֳבֵיל סְרָדֵי דָּרְתָא׃}
{And the screen for the gate of the court was the work of the weaver in colours, of blue, and purple, and scarlet, and fine twined linen; and twenty cubits was the length, and the height in the breadth was five cubits, answerable to the hangings of the court.}{\arabic{verse}}
\threeverse{\arabic{verse}}%Ex.38:19
{וְעַמֻּֽדֵיהֶם֙ אַרְבָּעָ֔ה וְאַדְנֵיהֶ֥ם אַרְבָּעָ֖ה נְחֹ֑שֶׁת וָוֵיהֶ֣ם כֶּ֔סֶף וְצִפּ֧וּי רָאשֵׁיהֶ֛ם וַחֲשֻׁקֵיהֶ֖ם כָּֽסֶף׃}
{וְעַמּוּדֵיהוֹן אַרְבְּעָא וְסָמְכֵיהוֹן אַרְבְּעָא דִּנְחָשָׁא וָוֵיהוֹן כְּסַף וְחִפּוּי רֵישֵׁיהוֹן וְכִבּוּשֵׁיהוֹן כְּסַף׃}
{And their pillars were four, and their sockets four of brass; their hooks of silver, and the overlaying of their capitals and their fillets of silver.}{\arabic{verse}}
\threeverse{\arabic{verse}}%Ex.38:20
{וְֽכׇל־הַיְתֵדֹ֞ת לַמִּשְׁכָּ֧ן וְלֶחָצֵ֛ר סָבִ֖יב נְחֹֽשֶׁת׃ \setuma         }
{וְכָל סִכַּיָּא לְמַשְׁכְּנָא וּלְדָרְתָא סְחוֹר סְחוֹר דִּנְחָשָׁא׃}
{And all the pins of the tabernacle, and of the court round about, were of brass.}{\arabic{verse}}
\newparsha{פקודי}
\threeverse{\aliya{פקודי}}%Ex.38:21
{אֵ֣לֶּה פְקוּדֵ֤י הַמִּשְׁכָּן֙ מִשְׁכַּ֣ן הָעֵדֻ֔ת אֲשֶׁ֥ר פֻּקַּ֖ד עַל־פִּ֣י מֹשֶׁ֑ה עֲבֹדַת֙ הַלְוִיִּ֔ם בְּיַד֙ אִֽיתָמָ֔ר בֶּֽן־אַהֲרֹ֖ן הַכֹּהֵֽן׃
\rashi{\rashiDH{אלה פקודי. }בפרשה זו נמנו כל משקלי נדבת המשכן לכסף ולזהב ולנחשת, ונמנו כל כליו לכל עבודתו׃ }\rashi{\rashiDH{המשכן משכן. }שני פעמים, רמז למקדש שנתמשכן בשני חורבנין על עונותיהן של ישראל׃ }\rashi{\rashiDH{משכן העדות. }עדות לישראל שויתר להם הקב״ה על מעשה העגל, שהרי השרה שכינתו ביניהם׃ }\rashi{\rashiDH{עבודת הלוים. }פקודי המשכן וכליו, הוא עבודה המסורה ללוים במדבר, לשאת ולהוריד ולהקים, איש איש למשאו המופקד עליו, כמו שאמור בפרשת נשא (במדבר ד)׃ }\rashi{\rashiDH{ביד איתמר. }הוא היה פקיד עליהם למסור לכל בית אב עבודה שעליו׃}}
{אִלֵּין מִנְיָנֵי מַשְׁכְּנָא מַשְׁכְּנָא דְּסָהֲדוּתָא דְּאִתְמְנִיאוּ עַל מֵימְרָא דְּמֹשֶׁה פּוּלְחַן לֵיוָאֵי בִּידָא דְּאִיתָמָר בַּר אַהֲרֹן כָּהֲנָא׃}
{These are the accounts of the tabernacle, even the tabernacle of the testimony, as they were rendered according to the commandment of Moses, through the service of the Levites, by the hand of Ithamar, the son of Aaron the priest.}{\arabic{verse}}
\threeverse{\arabic{verse}}%Ex.38:22
{וּבְצַלְאֵ֛ל בֶּן־אוּרִ֥י בֶן־ח֖וּר לְמַטֵּ֣ה יְהוּדָ֑ה עָשָׂ֕ה אֵ֛ת כׇּל־אֲשֶׁר־צִוָּ֥ה יְהֹוָ֖ה אֶת־מֹשֶֽׁה׃
\rashi{\rashiDH{ובצלאל בן אורי וגו׳ עשה אל כל אשר צוה ה׳ את משה. }אשר צוה אותו משה אין כתיב כאן, אלא כל אשר צוה ה׳ את משה, אפילו דברים שלא אמר לו רבו, הסכימה דעתו למה שנאמר למשה בסיני, כי משה צוה לבצלאל לעשות תחלה כלים ואחר כך משכן, (לא לענין צווי להתנדב קאמר, דהא אדרבה להיפך צוה הקב״ה בפרשת תרומה, מתחלה הכלים שלחן מנורה יריעות, ואחר כך ציווי הקרשים, וציווי משה רבינו ע״ה ריש ויקהל, תחלה המשכן ואהלו ואחר כך הכלים, הא מיירי לענין ציווי לפועל איך יפעול כסדר, ותמצא בפרשת כי תשא ראה קראתי בשם בצלאל וגו׳, הוזכר מתחלה את אהל מועד, ואחר כך הכלים, אבל לענין להתנדב להכין מה שיהיו צריכין, מה לי מה שמתנדב תחלה, ועיין בתוספות פרק הרואה (ברכות נה.). ואם תאמר מנלן שמשה רבינו ע״ה צוה לבצלאל הפך הענין, ויש לומר, כיון דכתיב בפרשת ויקהל ויקרא משה אל בצלאל ואל אהליאב וגו׳, וקצר מה שדיבר עמהם, והאי קרא מדכתיב כל אשר צוה ה׳ את משה, חזינן דהיה מצוה להם בהיפוך. ודוק היטב). אמר לו בצלאל מנהג העולם לעשות תחלה בית ואחר כך משים כלים בתוכו, אמר לו כך שמעתי מפי הקב״ה, אמר משה, בצל אל היית, כי וודאי כך צוה לי הקב״ה, וכן עשה, המשכן תחלה ואחר כך עשה כלים׃ 
}}
{וּבְצַלְאֵל בַּר אוּרִי בַר חוּר לְשִׁבְטָא דִּיהוּדָה עֲבַד יָת כָּל דְּפַקֵּיד יְיָ יָת מֹשֶׁה׃}
{And Bezalel the son of Uri, the son of Hur, of the tribe of Judah, made all that the \lord\space commanded Moses.}{\arabic{verse}}
\threeverse{\arabic{verse}}%Ex.38:23
{וְאִתּ֗וֹ אׇהֳלִיאָ֞ב בֶּן־אֲחִיסָמָ֛ךְ לְמַטֵּה־דָ֖ן חָרָ֣שׁ וְחֹשֵׁ֑ב וְרֹקֵ֗ם בַּתְּכֵ֙לֶת֙ וּבָֽאַרְגָּמָ֔ן וּבְתוֹלַ֥עַת הַשָּׁנִ֖י וּבַשֵּֽׁשׁ׃ \setuma         }
{וְעִמֵּיהּ אָהֳלִיאָב בַּר אֲחִיסָמָךְ לְשִׁבְטָא דְּדָן נַגָּר וְאוּמָּן וְצַיָּיר בְּתַכְלָא וּבְאַרְגְּוָנָא וּבִצְבַע זְהוֹרִי וּבְבוּצָא׃}
{And with him was Oholiab, the son of Ahisamach, of the tribe of Dan, a craftsman, and a skilful workman, and a weaver in colours, in blue, and in purple, and in scarlet, and fine linen.—}{\arabic{verse}}
\threeverse{\aliya{לוי}}%Ex.38:24
{כׇּל־הַזָּהָ֗ב הֶֽעָשׂוּי֙ לַמְּלָאכָ֔ה בְּכֹ֖ל מְלֶ֣אכֶת הַקֹּ֑דֶשׁ וַיְהִ֣י \legarmeh  זְהַ֣ב הַתְּנוּפָ֗ה תֵּ֤שַׁע וְעֶשְׂרִים֙ כִּכָּ֔ר וּשְׁבַ֨ע מֵא֧וֹת וּשְׁלֹשִׁ֛ים שֶׁ֖קֶל בְּשֶׁ֥קֶל הַקֹּֽדֶשׁ׃
\rashi{\rashiDH{ככר. }ששים מנה, ומנה של קדש כפול היה, הרי הככר ק״כ מנה, והמנה כ״ה סלעים, הרי ככר של קדש שלשת אלפים שקלים, לפיכך מנה בִּפְרוֹטְרוֹט כל השקלים שפחותין במנינם מג׳ אלפים, שאין מגיעין לככר׃ 
}}
{כָּל דַּהְבָּא דְּאִתְעֲבֵיד לַעֲבִידְתָא בְּכֹל עֲבִידַת קוּדְשָׁא וַהֲוָה דְּהַב אֲרָמוּתָא עֶשְׂרִין וּתְשַׁע כַּכְּרִין וּשְׁבַע מְאָה וּתְלָתִין סִלְעִין בְּסִלְעֵי קוּדְשָׁא׃}
{All the gold that was used for the work in all the work of the sanctuary, even the gold of the offering, was twenty and nine talents, and seven hundred and thirty shekels, after the shekel of the sanctuary.}{\arabic{verse}}
\threeverse{\arabic{verse}}%Ex.38:25
{וְכֶ֛סֶף פְּקוּדֵ֥י הָעֵדָ֖ה מְאַ֣ת כִּכָּ֑ר וְאֶ֩לֶף֩ וּשְׁבַ֨ע מֵא֜וֹת וַחֲמִשָּׁ֧ה וְשִׁבְעִ֛ים שֶׁ֖קֶל בְּשֶׁ֥קֶל הַקֹּֽדֶשׁ׃}
{וּכְסַף מִנְיָנֵי כְּנִשְׁתָּא מְאָה כַּכְּרִין וְאֶלֶף וּשְׁבַע מְאָה וְשִׁבְעִין וַחֲמֵישׁ סִלְעִין בְּסִלְעֵי קוּדְשָׁא׃}
{And the silver of them that were numbered of the congregation was a hundred talents, and a thousand seven hundred and three-score and fifteen shekels, after the shekel of the sanctuary:}{\arabic{verse}}
\threeverse{\arabic{verse}}%Ex.38:26
{בֶּ֚קַע לַגֻּלְגֹּ֔לֶת מַחֲצִ֥ית הַשֶּׁ֖קֶל בְּשֶׁ֣קֶל הַקֹּ֑דֶשׁ לְכֹ֨ל הָעֹבֵ֜ר עַל־הַפְּקֻדִ֗ים מִבֶּ֨ן עֶשְׂרִ֤ים שָׁנָה֙ וָמַ֔עְלָה לְשֵׁשׁ־מֵא֥וֹת אֶ֙לֶף֙ וּשְׁלֹ֣שֶׁת אֲלָפִ֔ים וַחֲמֵ֥שׁ מֵא֖וֹת וַחֲמִשִּֽׁים׃
\rashi{\rashiDH{בקע. }הוא שם משקל של מחצית השקל׃}\rashi{\rashiDH{לשש מאות אלף וגו׳. }כך היו ישראל, וכך עלה מנינם אחר שהוקם המשכן בספר במדבר, ואף עתה בנדבת המשכן כך היו, ומנין חצאי השקלים של שש מאות אלף, עולה מאת ככר כל אחד של שלשת אלפים שקלים, כיצד, שש מאות אלף חצאין הרי הן ג׳ מאות, אלף שלימים, הרי מאת ככר, והשלשת אלפים וחמש מאות וחמשים חצאין, עולין אלף ושבע מאות וחמשה ושבעים שקלים׃ }}
{תִּקְלָא לְגוּלְגּוּלְתָּא פַּלְגוּת סִלְעָא בְּסִלְעֵי קוּדְשָׁא לְכֹל דְּעָבַר עַל מִנְיָנַיָּא מִבַּר עַסְרִין שְׁנִין וּלְעֵילָא לְשֵׁית מְאָה וּתְלָתָא אַלְפִין וַחֲמֵישׁ מְאָה וְחַמְשִׁין׃}
{a beka a head, that is, half a shekel, after the shekel of the sanctuary, for every one that passed over to them that are numbered, from twenty years old and upward, for six hundred thousand and three thousand and five hundred and fifty men.}{\arabic{verse}}
\threeverse{\arabic{verse}}%Ex.38:27
{וַיְהִ֗י מְאַת֙ כִּכַּ֣ר הַכֶּ֔סֶף לָצֶ֗קֶת אֵ֚ת אַדְנֵ֣י הַקֹּ֔דֶשׁ וְאֵ֖ת אַדְנֵ֣י הַפָּרֹ֑כֶת מְאַ֧ת אֲדָנִ֛ים לִמְאַ֥ת הַכִּכָּ֖ר כִּכָּ֥ר לָאָֽדֶן׃
\rashi{\rashiDH{לצקת. }כתרגומו לְאַתָּכָא׃}\rashi{\rashiDH{את אדני הקדש. }של קרשי המשכן, שהם מ״ח קרשים, ולהן צ״ו אדנים, ואדני פרכת ארבעה, הרי מאה, וכל שאר האדנים נחשת כתיב בהם׃ }}
{וַהֲוָאָה מְאָה כַּכְּרֵי כַסְפָּא לְאַתָּכָא יָת סָמְכֵי קוּדְשָׁא וְיָת סָמְכֵי פָרוּכְתָּא מְאָה סָמְכִין לִמְאָה כַּכְּרִין כַּכְּרָא לְסָמְכָא׃}
{And the hundred talents of silver were for casting the sockets of the sanctuary, and the sockets of the veil: a hundred sockets for the hundred talents, a talent for a socket.}{\arabic{verse}}
\threeverse{\aliya{ישראל}}%Ex.38:28
{וְאֶת־הָאֶ֜לֶף וּשְׁבַ֤ע הַמֵּאוֹת֙ וַחֲמִשָּׁ֣ה וְשִׁבְעִ֔ים עָשָׂ֥ה וָוִ֖ים לָעַמּוּדִ֑ים וְצִפָּ֥ה רָאשֵׁיהֶ֖ם וְחִשַּׁ֥ק אֹתָֽם׃
\rashi{\rashiDH{וצפה ראשיהם. }של עמודים מהן, שבכולן כתיב וצפה ראשיהם וחשוקיהם כסף׃ 
}}
{וְיָת אֶלֶף וּשְׁבַע מְאָה וְשִׁבְעִין וַחֲמֵישׁ עֲבַד וָוִין לְעַמּוּדַיָּא וְחַפִּי רֵישֵׁיהוֹן וְכַבֵּישׁ יָתְהוֹן׃}
{And of the thousand seven hundred seventy and five shekels he made hooks for the pillars, and overlaid their capitals, and made fillets for them.}{\arabic{verse}}
\threeverse{\arabic{verse}}%Ex.38:29
{וּנְחֹ֥שֶׁת הַתְּנוּפָ֖ה שִׁבְעִ֣ים כִּכָּ֑ר וְאַלְפַּ֥יִם וְאַרְבַּע־מֵא֖וֹת שָֽׁקֶל׃}
{וּנְחָשׁ אֲרָמוּתָא שִׁבְעִין כַּכְּרִין וּתְרֵין אַלְפִין וְאַרְבַּע מְאָה סִלְעִין׃}
{And the brass of the offering was seventy talents and two thousand and four hundred shekels.}{\arabic{verse}}
\threeverse{\arabic{verse}}%Ex.38:30
{וַיַּ֣עַשׂ בָּ֗הּ אֶת־אַדְנֵי֙ פֶּ֚תַח אֹ֣הֶל מוֹעֵ֔ד וְאֵת֙ מִזְבַּ֣ח הַנְּחֹ֔שֶׁת וְאֶת־מִכְבַּ֥ר הַנְּחֹ֖שֶׁת אֲשֶׁר־ל֑וֹ וְאֵ֖ת כׇּל־כְּלֵ֥י הַמִּזְבֵּֽחַ׃}
{וַעֲבַד בַּהּ יָת סָמְכֵי תְּרַע מַשְׁכַּן זִמְנָא וְיָת מַדְבְּחָא דִּנְחָשָׁא וְיָת סְרָדָא דִּנְחָשָׁא דִּילֵיהּ וְיָת כָּל מָנֵי מַדְבְּחָא׃}
{And therewith he made the sockets to the door of the tent of meeting, and the brazen altar, and the brazen grating for it, and all the vessels of the altar,}{\arabic{verse}}
\threeverse{\arabic{verse}}%Ex.38:31
{וְאֶת־אַדְנֵ֤י הֶֽחָצֵר֙ סָבִ֔יב וְאֶת־אַדְנֵ֖י שַׁ֣עַר הֶחָצֵ֑ר וְאֵ֨ת כׇּל־יִתְדֹ֧ת הַמִּשְׁכָּ֛ן וְאֶת־כׇּל־יִתְדֹ֥ת הֶחָצֵ֖ר סָבִֽיב׃}
{וְיָת סָמְכֵי דָּרְתָא סְחוֹר סְחוֹר וְיָת סָמְכֵי תְּרַע דָּרְתָא וְיָת כָּל סִכֵּי מַשְׁכְּנָא וְיָת כָּל סִכֵּי דָּרְתָא סְחוֹר סְחוֹר׃}
{and the sockets of the court round about, and the sockets of the gate of the court, and all the pins of the tabernacle, and all the pins of the court round about.}{\arabic{verse}}
\newperek
\threeverse{\Roman{chap}}%Ex.39:1
{וּמִן־הַתְּכֵ֤לֶת וְהָֽאַרְגָּמָן֙ וְתוֹלַ֣עַת הַשָּׁנִ֔י עָשׂ֥וּ בִגְדֵי־שְׂרָ֖ד לְשָׁרֵ֣ת בַּקֹּ֑דֶשׁ וַֽיַּעֲשׂ֞וּ אֶת־בִּגְדֵ֤י הַקֹּ֙דֶשׁ֙ אֲשֶׁ֣ר לְאַהֲרֹ֔ן כַּאֲשֶׁ֛ר צִוָּ֥ה יְהֹוָ֖ה אֶת־מֹשֶֽׁה׃ \petucha 
\rashi{\rashiDH{ומן התכלת והארגמן וגו׳. }שש לא נאמר כאן, ומכאן אני אומר שאין בגדי שרד הללו בגדי כהונה, שבבגדי כהונה היה שש, אלא הם בגדים שמכסים בהם כלי הקדש בשעת סלוק מסעות, שלא היה בהם שש׃ 
}}
{וּמִן תַּכְלָא וְאַרְגְּוָנָא וּצְבַע זְהוֹרִי עֲבַדוּ לְבוּשֵׁי שִׁמּוּשָׁא לְשַׁמָּשָׁא בְּקוּדְשָׁא וַעֲבַדוּ יָת לְבוּשֵׁי קוּדְשָׁא דִּלְאַהֲרֹן כְּמָא דְּפַקֵּיד יְיָ יָת מֹשֶׁה׃}
{And of the blue, and purple, and scarlet, they made plaited garments, for ministering in the holy place, and made the holy garments for Aaron, as the \lord\space commanded Moses.}{\Roman{chap}}
\threeverse{\aliya{שני\newline (חמישי)}}%Ex.39:2
{וַיַּ֖עַשׂ אֶת־הָאֵפֹ֑ד זָהָ֗ב תְּכֵ֧לֶת וְאַרְגָּמָ֛ן וְתוֹלַ֥עַת שָׁנִ֖י וְשֵׁ֥שׁ מׇשְׁזָֽר׃}
{וַעֲבַד יָת אֵיפוֹדָא דַּהְבָּא תַּכְלָא וְאַרְגְּוָנָא וּצְבַע זְהוֹרִי וּבוּץ שְׁזִיר׃}
{And he made the ephod of gold, blue, and purple, and scarlet, and fine twined linen.}{\arabic{verse}}
\threeverse{\arabic{verse}}%Ex.39:3
{וַֽיְרַקְּע֞וּ אֶת־פַּחֵ֣י הַזָּהָב֮ וְקִצֵּ֣ץ פְּתִילִם֒ לַעֲשׂ֗וֹת בְּת֤וֹךְ הַתְּכֵ֙לֶת֙ וּבְת֣וֹךְ הָֽאַרְגָּמָ֔ן וּבְת֛וֹךְ תּוֹלַ֥עַת הַשָּׁנִ֖י וּבְת֣וֹךְ הַשֵּׁ֑שׁ מַעֲשֵׂ֖ה חֹשֵֽׁב׃
\rashi{\rashiDH{וירקעו. }כמו לְרֹקַע הָאָרֶץ (תהלים קלו, ו), כתרגומו וְרַדִּידוּ טַסִּין, היו מרדדין מן הזהב, אשטנ״דרא בלע״ז (אויזדעהנען), טסין דקות. כאן הוא מלמדך היאך היו טווין את הזהב עם החוטין, מרדדים טסין דקין, וקוצצין מהן פתילים לאורך הטס, לעשות אותן פתילים מעורבים עם כל מין ומין בחשן ואפוד, שנאמר בהן זהב, חוט אחד של זהב עם ששה חוטין של תכלת, וכן עם כל מין ומין, שכל המינין חוטן כפול ששה, והזהב חוט שביעי עם כל אחד ואחד׃ }}
{וְרַדִידוּ יָת טַסֵי דְּדַהְבָּא וְקַצִּיצוּ חוּטִין לְמֶעֱבַד בְּגוֹ תַּכְלָא וּבְגוֹ אַרְגְּוָנָא וּבְגוֹ צְבַע זְהוֹרִי וּבְגוֹ בוּצָא עוֹבָד אוּמָּן׃}
{And they did beat the gold into thin plates, and cut it into threads, to work it in the blue, and in the purple, and in the scarlet, and in the fine linen, the work of the skilful workman.}{\arabic{verse}}
\threeverse{\arabic{verse}}%Ex.39:4
{כְּתֵפֹ֥ת עָֽשׂוּ־ל֖וֹ חֹבְרֹ֑ת עַל־שְׁנֵ֥י \qk{קְצוֹתָ֖יו}{קצוותו} חֻבָּֽר׃}
{כַּתְפִּין עֲבַדוּ לֵיהּ מְלָפְפָן עַל תְּרֵין סִטְרוֹהִי מְלָפַף׃}
{They made shoulder-pieces for it, joined together; at the two ends was it joined together.}{\arabic{verse}}
\threeverse{\arabic{verse}}%Ex.39:5
{וְחֵ֨שֶׁב אֲפֻדָּת֜וֹ אֲשֶׁ֣ר עָלָ֗יו מִמֶּ֣נּוּ הוּא֮ כְּמַעֲשֵׂ֒הוּ֒ זָהָ֗ב תְּכֵ֧לֶת וְאַרְגָּמָ֛ן וְתוֹלַ֥עַת שָׁנִ֖י וְשֵׁ֣שׁ מׇשְׁזָ֑ר כַּאֲשֶׁ֛ר צִוָּ֥ה יְהֹוָ֖ה אֶת־מֹשֶֽׁה׃ \setuma         }
{וְהִמְיַן תִּקּוּנֵיהּ דַּעֲלוֹהִי מִנֵּיהּ הוּא כְּעוֹבָדוֹהִי דַּהְבָּא תַּכְלָא וְאַרְגְּוָנָא וּצְבַע זְהוֹרִי וּבוּץ שְׁזִיר כְּמָא דְּפַקֵּיד יְיָ יָת מֹשֶׁה׃}
{And the skilfully woven band, that was upon it, wherewith to gird it on, was of the same piece and like the work thereof: of gold, of blue, and purple, and scarlet, and fine twined linen, as the \lord\space commanded Moses.}{\arabic{verse}}
\threeverse{\arabic{verse}}%Ex.39:6
{וַֽיַּעֲשׂוּ֙ אֶת־אַבְנֵ֣י הַשֹּׁ֔הַם מֻֽסַבֹּ֖ת מִשְׁבְּצֹ֣ת זָהָ֑ב מְפֻתָּחֹת֙ פִּתּוּחֵ֣י חוֹתָ֔ם עַל־שְׁמ֖וֹת בְּנֵ֥י יִשְׂרָאֵֽל׃}
{וַעֲבַדוּ יָת אַבְנֵי בוּרְלָא מְשַׁקְּעָן מְרַמְּצָן בִּדְהַב גְּלִיפָן כְּתָב מְפָרַשׁ עַל שְׁמָהָת בְּנֵי יִשְׂרָאֵל׃}
{And they wrought the onyx stones, inclosed in settings of gold, graven with the engravings of a signet, according to the names of the children of Israel.}{\arabic{verse}}
\threeverse{\arabic{verse}}%Ex.39:7
{וַיָּ֣שֶׂם אֹתָ֗ם עַ֚ל כִּתְפֹ֣ת הָאֵפֹ֔ד אַבְנֵ֥י זִכָּר֖וֹן לִבְנֵ֣י יִשְׂרָאֵ֑ל כַּאֲשֶׁ֛ר צִוָּ֥ה יְהֹוָ֖ה אֶת־מֹשֶֽׁה׃ \petucha }
{וְשַׁוִּי יָתְהוֹן עַל כִּתְפֵי אֵיפוֹדָא אַבְנֵי דּוּכְרָנָא לִבְנֵי יִשְׂרָאֵל כְּמָא דְּפַקֵּיד יְיָ יָת מֹשֶׁה׃}
{And he put them on the shoulder-pieces of the ephod, to be stones of memorial for the children of Israel, as the \lord\space commanded Moses.}{\arabic{verse}}
\threeverse{\arabic{verse}}%Ex.39:8
{וַיַּ֧עַשׂ אֶת־הַחֹ֛שֶׁן מַעֲשֵׂ֥ה חֹשֵׁ֖ב כְּמַעֲשֵׂ֣ה אֵפֹ֑ד זָהָ֗ב תְּכֵ֧לֶת וְאַרְגָּמָ֛ן וְתוֹלַ֥עַת שָׁנִ֖י וְשֵׁ֥שׁ מׇשְׁזָֽר׃}
{וַעֲבַד יָת חוּשְׁנָא עוֹבָד אוּמָּן כְּעוֹבָד אֵיפוֹדָא דַּהְבָּא תַּכְלָא וְאַרְגְּוָנָא וּצְבַע זְהוֹרִי וּבוּץ שְׁזִיר׃}
{And he made the breastplate, the work of the skilful workman, like the work of the ephod: of gold, of blue, and purple, and scarlet, and fine twined linen.}{\arabic{verse}}
\threeverse{\arabic{verse}}%Ex.39:9
{רָב֧וּעַ הָיָ֛ה כָּפ֖וּל עָשׂ֣וּ אֶת־הַחֹ֑שֶׁן זֶ֧רֶת אׇרְכּ֛וֹ וְזֶ֥רֶת רׇחְבּ֖וֹ כָּפֽוּל׃}
{מְרֻבַּע הֲוָה עִיף עֲבַדוּ יָת חוּשְׁנָא זַרְתָּא אוּרְכֵּיהּ וְזַרְתָּא פּוּתְיֵיהּ עִיף׃}
{It was four-square; they made the breastplate double; a span was the length thereof, and a span the breadth thereof, being double.}{\arabic{verse}}
\threeverse{\arabic{verse}}%Ex.39:10
{וַיְמַ֨לְאוּ־ב֔וֹ אַרְבָּעָ֖ה ט֣וּרֵי אָ֑בֶן ט֗וּר אֹ֤דֶם פִּטְדָה֙ וּבָרֶ֔קֶת הַטּ֖וּר הָאֶחָֽד׃}
{וְאַשְׁלִימוּ בֵּיהּ אַרְבְּעָא סִדְרִין דְּאֶבֶן טָבָא סִדְרָא קַדְמָאָה סָמְקָן יָרְקָן וּבָרְקָן סִדְרָא חַד׃}
{And they set in it four rows of stones: a row of carnelian, topaz, and smaragd was the first row.}{\arabic{verse}}
\threeverse{\arabic{verse}}%Ex.39:11
{וְהַטּ֖וּר הַשֵּׁנִ֑י נֹ֥פֶךְ סַפִּ֖יר וְיָהֲלֹֽם׃}
{וְסִדְרָא תִּנְיָנָא אִזְמַרַגְדִּין שַׁבְזֵיז וְסַבְהֲלוֹם׃}
{And the second row, a carbuncle, a sapphire, and an emerald.}{\arabic{verse}}
\threeverse{\arabic{verse}}%Ex.39:12
{וְהַטּ֖וּר הַשְּׁלִישִׁ֑י לֶ֥שֶׁם שְׁב֖וֹ וְאַחְלָֽמָה׃}
{וְסִדְרָא תְּלִיתָאָה קַנְכֵּירִי טְרַקְיָא וְעֵין עִגְלָא׃}
{And the third row, a jacinth, an agate, and an amethyst.}{\arabic{verse}}
\threeverse{\arabic{verse}}%Ex.39:13
{וְהַטּוּר֙ הָֽרְבִיעִ֔י תַּרְשִׁ֥ישׁ שֹׁ֖הַם וְיָשְׁפֵ֑ה מֽוּסַבֹּ֛ת מִשְׁבְּצֹ֥ת זָהָ֖ב בְּמִלֻּאֹתָֽם׃}
{וְסִדְרָא רְבִיעָאָה כְּרוּם יַמָּא וּבוּרְלָא וּפַנְתֵּירִי מְשַׁקְּעָן מְרַמְּצָן בִּדְהַב בְּאַשְׁלָמוּתְהוֹן׃}
{And the fourth row, a beryl, an onyx, and a jasper; they were inclosed in fittings of gold in their settings.}{\arabic{verse}}
\threeverse{\arabic{verse}}%Ex.39:14
{וְ֠הָאֲבָנִ֠ים עַל־שְׁמֹ֨ת בְּנֵי־יִשְׂרָאֵ֥ל הֵ֛נָּה שְׁתֵּ֥ים עֶשְׂרֵ֖ה עַל־שְׁמֹתָ֑ם פִּתּוּחֵ֤י חֹתָם֙ אִ֣ישׁ עַל־שְׁמ֔וֹ לִשְׁנֵ֥ים עָשָׂ֖ר שָֽׁבֶט׃}
{וְאַבְנַיָּא עַל שְׁמָהָת בְּנֵי יִשְׂרָאֵל אִנִּין תַּרְתַּא עֶשְׂרֵי עַל שְׁמָהָתְהוֹן כְּתָב מְפָרַשׁ כִּגְלָף דְּעִזְקָא גְּבַר עַל שְׁמֵיהּ לִתְרֵי עֲשַׂר שִׁבְטִין׃}
{And the stones were according to the names of the children of Israel, twelve, according to their names, like the engravings of a signet, every one according to his name, for the twelve tribes.}{\arabic{verse}}
\threeverse{\arabic{verse}}%Ex.39:15
{וַיַּעֲשׂ֧וּ עַל־הַחֹ֛שֶׁן שַׁרְשְׁרֹ֥ת גַּבְלֻ֖ת מַעֲשֵׂ֣ה עֲבֹ֑ת זָהָ֖ב טָהֽוֹר׃}
{וַעֲבַדוּ עַל חוּשְׁנָא תִּכִּין מְתַחֲמָן עוֹבָד גְּדִילוּ דִּדְהַב דְּכֵי׃}
{And they made upon the breastplate plaited chains, of wreathen work of pure gold.}{\arabic{verse}}
\threeverse{\arabic{verse}}%Ex.39:16
{וַֽיַּעֲשׂ֗וּ שְׁתֵּי֙ מִשְׁבְּצֹ֣ת זָהָ֔ב וּשְׁתֵּ֖י טַבְּעֹ֣ת זָהָ֑ב וַֽיִּתְּנ֗וּ אֶת־שְׁתֵּי֙ הַטַּבָּעֹ֔ת עַל־שְׁנֵ֖י קְצ֥וֹת הַחֹֽשֶׁן׃}
{וַעֲבַדוּ תַּרְתֵּין מְרַמְּצָן דִּדְהַב וְתַרְתֵּין עִזְקָן דִּדְהַב וִיהַבוּ יָת תַּרְתֵּין עִזְקָתָא עַל תְּרֵין סִטְרֵי חוּשְׁנָא׃}
{And they made two settings of gold, and two gold rings; and put the two rings on the two ends of the breastplate.}{\arabic{verse}}
\threeverse{\arabic{verse}}%Ex.39:17
{וַֽיִּתְּנ֗וּ שְׁתֵּי֙ הָעֲבֹתֹ֣ת הַזָּהָ֔ב עַל־שְׁתֵּ֖י הַטַּבָּעֹ֑ת עַל־קְצ֖וֹת הַחֹֽשֶׁן׃}
{וִיהַבוּ תַּרְתֵּין גְּדִילָן דִּדְהַב עַל תַּרְתֵּין עִזְקָתָא עַל סִטְרֵי חוּשְׁנָא׃}
{And they put the two wreathen chains of gold on the two rings at the ends of the breastplate.}{\arabic{verse}}
\threeverse{\arabic{verse}}%Ex.39:18
{וְאֵ֨ת שְׁתֵּ֤י קְצוֹת֙ שְׁתֵּ֣י הָֽעֲבֹתֹ֔ת נָתְנ֖וּ עַל־שְׁתֵּ֣י הַֽמִּשְׁבְּצֹ֑ת וַֽיִּתְּנֻ֛ם עַל־כִּתְפֹ֥ת הָאֵפֹ֖ד אֶל־מ֥וּל פָּנָֽיו׃}
{וְיָת תַּרְתֵּין גְּדִילָן דְּעַל תְּרֵין סִטְרוֹהִי יְהַבוּ עַל תַּרְתֵּין מְרַמְּצָתָא וִיהַבוּנִין עַל כִּתְפֵי אֵיפוֹדָא לָקֳבֵיל אַפּוֹהִי׃}
{And the other two ends of the two wreathen chains they put on the two settings, and put them on the shoulder-pieces of the ephod, in the forepart thereof.}{\arabic{verse}}
\threeverse{\arabic{verse}}%Ex.39:19
{וַֽיַּעֲשׂ֗וּ שְׁתֵּי֙ טַבְּעֹ֣ת זָהָ֔ב וַיָּשִׂ֕ימוּ עַל־שְׁנֵ֖י קְצ֣וֹת הַחֹ֑שֶׁן עַל־שְׂפָת֕וֹ אֲשֶׁ֛ר אֶל־עֵ֥בֶר הָאֵפֹ֖ד בָּֽיְתָה׃}
{וַעֲבַדוּ תַּרְתֵּין עִזְקָן דִּדְהַב וְשַׁוִּיאוּ עַל תְּרֵין סִטְרֵי חוּשְׁנָא עַל סִפְתֵּיהּ דִּלְעִבְרָא דְּאֵיפוֹדָא לְגָיו׃}
{And they made two rings of gold, and put them upon the two ends of the breastplate, upon the edge thereof, which was toward the side of the ephod inward.}{\arabic{verse}}
\threeverse{\arabic{verse}}%Ex.39:20
{וַֽיַּעֲשׂוּ֮ שְׁתֵּ֣י טַבְּעֹ֣ת זָהָב֒ וַֽיִּתְּנֻ֡ם עַל־שְׁתֵּי֩ כִתְפֹ֨ת הָאֵפֹ֤ד מִלְּמַ֙טָּה֙ מִמּ֣וּל פָּנָ֔יו לְעֻמַּ֖ת מַחְבַּרְתּ֑וֹ מִמַּ֕עַל לְחֵ֖שֶׁב הָאֵפֹֽד׃}
{וַעֲבַדוּ תַּרְתֵּין עִזְקָן דִּדְהַב וִיהַבוּנִין עַל תְּרֵין כִּתְפֵי אֵיפוֹדָא מִלְּרַע מִלָּקֳבֵיל אַפּוֹהִי לָקֳבֵיל בֵּית לוֹפֵי מֵעִלָּוֵי לְהִמְיַן אֵיפוֹדָא׃}
{And they made two rings of gold, and put them on the two shoulder-pieces of the ephod underneath, in the forepart thereof, close by the coupling thereof, above the skilfully woven band of the ephod.}{\arabic{verse}}
\threeverse{\arabic{verse}}%Ex.39:21
{וַיִּרְכְּס֣וּ אֶת־הַחֹ֡שֶׁן מִטַּבְּעֹתָיו֩ אֶל־טַבְּעֹ֨ת הָאֵפֹ֜ד בִּפְתִ֣יל תְּכֵ֗לֶת לִֽהְיֹת֙ עַל־חֵ֣שֶׁב הָאֵפֹ֔ד וְלֹֽא־יִזַּ֣ח הַחֹ֔שֶׁן מֵעַ֖ל הָאֵפֹ֑ד כַּאֲשֶׁ֛ר צִוָּ֥ה יְהֹוָ֖ה אֶת־מֹשֶֽׁה׃ \petucha }
{וַאֲחַדוּ יָת חוּשְׁנָא מֵעִזְקָתֵיהּ לְעִזְקָת אֵיפוֹדָא בְּחוּטָא דִּתְכִילְתָא לְמִהְוֵי עַל הִמְיַן אֵיפוֹדָא וְלָא יִתְפָּרַק חוּשְׁנָא מֵעִלָּוֵי אֵיפוֹדָא כְּמָא דְּפַקֵּיד יְיָ יָת מֹשֶׁה׃}
{And they did bind the breastplate by the rings thereof unto the rings of the ephod with a thread of blue, that it might be upon the skilfully woven band of the ephod, and that the breastplate might not be loosed from the ephod; as the \lord\space commanded Moses.}{\arabic{verse}}
\threeverse{\aliya{שלישי\newline (ששי)}}%Ex.39:22
{וַיַּ֛עַשׂ אֶת־מְעִ֥יל הָאֵפֹ֖ד מַעֲשֵׂ֣ה אֹרֵ֑ג כְּלִ֖יל תְּכֵֽלֶת׃}
{וַעֲבַד יָת מְעִיל אֵיפוֹדָא עוֹבָד מָחֵי גְּמִיר תַּכְלָא׃}
{And he made the robe of the ephod of woven work, all of blue;}{\arabic{verse}}
\threeverse{\arabic{verse}}%Ex.39:23
{וּפִֽי־הַמְּעִ֥יל בְּתוֹכ֖וֹ כְּפִ֣י תַחְרָ֑א שָׂפָ֥ה לְפִ֛יו סָבִ֖יב לֹ֥א יִקָּרֵֽעַ׃}
{וּפוּמֵּיהּ דִּמְעִילָא כְּפִיל לְגַוֵּיהּ כְּפוֹם שִׁרְיָן תּוּרָא מַקַּף לְפוּמֵּיהּ סְחוֹר סְחוֹר לָא יִתְבְּזַע׃}
{and the hole of the robe in the midst thereof, as the hole of a coat of mail, with a binding round about the hole of it, that it should not be rent.}{\arabic{verse}}
\threeverse{\arabic{verse}}%Ex.39:24
{וַֽיַּעֲשׂוּ֙ עַל־שׁוּלֵ֣י הַמְּעִ֔יל רִמּוֹנֵ֕י תְּכֵ֥לֶת וְאַרְגָּמָ֖ן וְתוֹלַ֣עַת שָׁנִ֑י מׇשְׁזָֽר׃}
{וַעֲבַדוּ עַל שִׁפּוֹלֵי מְעִילָא רִמּוֹנֵי תַּכְלָא וְאַרְגְּוָנָא וּצְבַע זְהוֹרִי שְׁזִיר׃}
{And they made upon the skirts of the robe pomegranates of blue, and purple, and scarlet, and twined linen.}{\arabic{verse}}
\threeverse{\arabic{verse}}%Ex.39:25
{וַיַּעֲשׂ֥וּ פַעֲמֹנֵ֖י זָהָ֣ב טָה֑וֹר וַיִּתְּנ֨וּ אֶת־הַפַּֽעֲמֹנִ֜ים בְּת֣וֹךְ הָרִמֹּנִ֗ים עַל־שׁוּלֵ֤י הַמְּעִיל֙ סָבִ֔יב בְּת֖וֹךְ הָרִמֹּנִֽים׃}
{וַעֲבַדוּ זַגִּין דִּדְהַב דְּכֵי וִיהַבוּ יָת זַגַּיָּא בְּגוֹ רִמּוֹנַיָּא עַל שִׁפּוֹלֵי מְעִילָא סְחוֹר סְחוֹר בְּגוֹ רִמּוֹנַיָּא׃}
{And they made bells of pure gold, and put the bells between the pomegranates upon the skirts of the robe round about, between the pomegranates:}{\arabic{verse}}
\threeverse{\arabic{verse}}%Ex.39:26
{פַּעֲמֹ֤ן וְרִמֹּן֙ פַּעֲמֹ֣ן וְרִמֹּ֔ן עַל־שׁוּלֵ֥י הַמְּעִ֖יל סָבִ֑יב לְשָׁרֵ֕ת כַּאֲשֶׁ֛ר צִוָּ֥ה יְהֹוָ֖ה אֶת־מֹשֶֽׁה׃ \setuma         }
{זַגָּא וְרִמּוֹנָא זַגָּא וְרִמּוֹנָא עַל שִׁפּוֹלֵי מְעִילָא סְחוֹר סְחוֹר לְשַׁמָּשָׁא כְּמָא דְּפַקֵּיד יְיָ יָת מֹשֶׁה׃}
{a bell and a pomegranate, a bell and a pomegranate, upon the skirts of the robe round about, to minister in; as the \lord\space commanded Moses.}{\arabic{verse}}
\threeverse{\arabic{verse}}%Ex.39:27
{וַֽיַּעֲשׂ֛וּ אֶת־הַכׇּתְנֹ֥ת שֵׁ֖שׁ מַעֲשֵׂ֣ה אֹרֵ֑ג לְאַהֲרֹ֖ן וּלְבָנָֽיו׃}
{וַעֲבַדוּ יָת כִּתּוּנִין דְּבוּצָא עוֹבָד מָחֵי לְאַהֲרֹן וְלִבְנוֹהִי׃}
{And they made the tunics of fine linen of woven work for Aaron, and for his sons,}{\arabic{verse}}
\threeverse{\arabic{verse}}%Ex.39:28
{וְאֵת֙ הַמִּצְנֶ֣פֶת שֵׁ֔שׁ וְאֶת־פַּאֲרֵ֥י הַמִּגְבָּעֹ֖ת שֵׁ֑שׁ וְאֶת־מִכְנְסֵ֥י הַבָּ֖ד שֵׁ֥שׁ מׇשְׁזָֽר׃
\rashi{\rashiDH{ואת פארי המגבעות. }תפארת המגבעות, המגבעות המפוארות׃ }}
{וְיָת מַצְנַפְתָּא דְּבוּצָא וְיָת שְׁבָח קוֹבְעַיָּא דְּבוּצָא וְיָת מִכְנְסֵי בוּצָא דְּבוּץ שְׁזִיר׃}
{and the mitre of fine linen, and the goodly head-tires of fine linen, and the linen breeches of fine twined linen,}{\arabic{verse}}
\threeverse{\arabic{verse}}%Ex.39:29
{וְֽאֶת־הָאַבְנֵ֞ט שֵׁ֣שׁ מׇשְׁזָ֗ר וּתְכֵ֧לֶת וְאַרְגָּמָ֛ן וְתוֹלַ֥עַת שָׁנִ֖י מַעֲשֵׂ֣ה רֹקֵ֑ם כַּאֲשֶׁ֛ר צִוָּ֥ה יְהֹוָ֖ה אֶת־מֹשֶֽׁה׃ \setuma         }
{וְיָת הִמְיָנָא דְּבוּץ שְׁזִיר וְתַכְלָא וְאַרְגְּוָנָא וּצְבַע זְהוֹרִי עוֹבָד צַיָּיר כְּמָא דְּפַקֵּיד יְיָ יָת מֹשֶׁה׃}
{and the girdle of fine twined linen, and blue, and purple, and scarlet, the work of the weaver in colours; as the \lord\space commanded Moses.}{\arabic{verse}}
\threeverse{\arabic{verse}}%Ex.39:30
{וַֽיַּעֲשׂ֛וּ אֶת־צִ֥יץ נֵֽזֶר־הַקֹּ֖דֶשׁ זָהָ֣ב טָה֑וֹר וַיִּכְתְּב֣וּ עָלָ֗יו מִכְתַּב֙ פִּתּוּחֵ֣י חוֹתָ֔ם קֹ֖דֶשׁ לַיהֹוָֽה׃}
{וַעֲבַדוּ יָת צִיצָא כְּלִילָא דְּקוּדְשָׁא דִּדְהַב דְּכֵי וּכְתַבוּ עֲלוֹהִי כְּתָב מְפָרַשׁ קֹדֶשׁ לַייָ׃}
{And they made the plate of the holy crown of pure gold, and wrote upon it a writing, like the engravings of a signet: HOLY TO THE \lord.}{\arabic{verse}}
\threeverse{\arabic{verse}}%Ex.39:31
{וַיִּתְּנ֤וּ עָלָיו֙ פְּתִ֣יל תְּכֵ֔לֶת לָתֵ֥ת עַל־הַמִּצְנֶ֖פֶת מִלְמָ֑עְלָה כַּאֲשֶׁ֛ר צִוָּ֥ה יְהֹוָ֖ה אֶת־מֹשֶֽׁה׃ \setuma         
\rashi{\rashiDH{לתת על המצנפת מלמעלה. }ועל ידי הפתילים היה מושיבן על המצנפת כמין כתר, ואי אפשר לומר הציץ על המצנפת, שהרי בשחיטת קדשים שנינו (זבחים יט.׃), שערו היה נראה בין ציץ למצנפת ששם מניח תפילין, והציץ היה נתון על המצח, הרי המצנפת למעלה והציץ למטה, ומהו על המצנפת מלמעלה. ועוד הקשיתי בה, כאן הוא אומר ויתנו עליו פתיל תכלת, ובענין הצוואה הוא אומר וְשַׂמְתָּ אֹתֹו עַל פְּתִיל תְּכֵלָת (שמות כח, לז). ואומר אני, פתיל תכלת זה חוטין הן, לקשרו בהן במצנפת, לפי שהציץ אינו אלא מאוזן לאוזן ובמה יקשרנו במצחו, והיו קבועין בו חוטי תכלת לשני ראשיו ובאמצעיתו, שבהן קושרו ותולהו במצנפת כשהוא בראשו, ושני חוטין היו בכל קצה וקצה, אחת ממעל ואחת מתחת לצד מצחו, וכן באמצעו, שכך הוא נוח לקשור, ואין דרך קשירה בפחות משני חוטין, לכך נאמר על פתיל תכלת, ועליו פתיל תכלת, וקושר ראשיהם השנים כולם יחד מאחוריו למול ערפו, ומושיבו על המצנפת. ואל תתמה שלא נאמר פתילי תכלת, הואיל ומרובין הן, שהרי מצינו בחשן ואפוד וַיִּרְכְּסוּ אֶת הַחשֶׁן וגו׳, ועל כרחך פחות משנים לא היו, שהרי בשתי קצות החשן היו ב׳ טבעות החשן, ובב׳ כתפות האפוד היו ב׳ טבעות האפוד שכנגדן, ולפי דרך קשירה ד׳ חוטין היו, ומכל מקום פחות משנים אי אפשר׃ }}
{וִיהַבוּ עֲלוֹהִי חוּטָא דִּתְכִילְתָא לְמִתַּן עַל מַצְנַפְתָּא מִלְּעֵילָא כְּמָא דְּפַקֵּיד יְיָ יָת מֹשֶׁה׃}
{And they tied unto it a thread of blue, to fasten it upon the mitre above; as the \lord\space commanded Moses.}{\arabic{verse}}
\threeverse{\arabic{verse}}%Ex.39:32
{וַתֵּ֕כֶל כׇּל־עֲבֹדַ֕ת מִשְׁכַּ֖ן אֹ֣הֶל מוֹעֵ֑ד וַֽיַּעֲשׂוּ֙ בְּנֵ֣י יִשְׂרָאֵ֔ל כְּ֠כֹ֠ל אֲשֶׁ֨ר צִוָּ֧ה יְהֹוָ֛ה אֶת־מֹשֶׁ֖ה כֵּ֥ן עָשֽׂוּ׃ \petucha 
\rashi{\rashiDH{ויעשו בני ישראל. }את המלאכה ככל אשר צוה ה׳ וגו׳׃ 
}}
{וּשְׁלֵימַת כָּל עֲבִידַת מַשְׁכְּנָא מַשְׁכַּן זִמְנָא וַעֲבַדוּ בְנֵי יִשְׂרָאֵל כְּכֹל דְּפַקֵּיד יְיָ יָת מֹשֶׁה כֵּן עֲבַדוּ׃}
{Thus was finished all the work of the tabernacle of the tent of meeting; and the children of Israel did according to all that the \lord\space commanded Moses, so did they.}{\arabic{verse}}
\threeverse{\aliya{רביעי}}%Ex.39:33
{וַיָּבִ֤יאוּ אֶת־הַמִּשְׁכָּן֙ אֶל־מֹשֶׁ֔ה אֶת־הָאֹ֖הֶל וְאֶת־כׇּל־כֵּלָ֑יו קְרָסָ֣יו קְרָשָׁ֔יו בְּרִיחָ֖ו וְעַמֻּדָ֥יו וַאֲדָנָֽיו׃
\rashi{\rashiDH{ויביאו את המשכן וגו׳. }שלא היו יכולין להקימו, ולפי שלא עשה משה שום מלאכה במשכן, הניח לו הקב״ה הקמתו, שלא היה יכול להקימו שום אדם, מחמת כובד הקרשים שאין כח באדם לזקפן, ומשה העמידו, אמר משה לפני הקב״ה, איך אפשר הקמתו על ידי אדם, אמר לו עסוק אתה בידך, נראה כמקימו והוא נזקף וקם מאליו, וזהו שנאמר (שמות מ, יז) הוּקַם הַמִּשְׁכָּן, הוקם מאליו. מדרש רבי תנחומא (פקודי י״א)׃ 
}}
{וְאֵיתִיאוּ יָת מַשְׁכְּנָא לְוָת מֹשֶׁה יָת מַשְׁכְּנָא וְיָת כָּל מָנוֹהִי פּוּרְפוֹהִי דַּפּוֹהִי עָבְרוֹהִי וְעַמּוּדוֹהִי וְסָמְכוֹהִי׃}
{And they brought the tabernacle unto Moses, the Tent, and all its furniture, its clasps, its boards, its bars, and its pillars, and its sockets;}{\arabic{verse}}
\threeverse{\arabic{verse}}%Ex.39:34
{וְאֶת־מִכְסֵ֞ה עוֹרֹ֤ת הָֽאֵילִם֙ הַמְאׇדָּמִ֔ים וְאֶת־מִכְסֵ֖ה עֹרֹ֣ת הַתְּחָשִׁ֑ים וְאֵ֖ת פָּרֹ֥כֶת הַמָּסָֽךְ׃}
{וְיָת חוּפָאָה דְּמַשְׁכֵּי דִּכְרֵי מְסֻמְּקֵי וְיָת חוּפָאָה דְּמַשְׁכֵּי סָסְגוֹנָא וְיָת פָּרוּכְתָּא דִּפְרָסָא׃}
{and the covering of rams’ skins dyed red, and the covering of sealskins, and the veil of the screen;}{\arabic{verse}}
\threeverse{\arabic{verse}}%Ex.39:35
{אֶת־אֲר֥וֹן הָעֵדֻ֖ת וְאֶת־בַּדָּ֑יו וְאֵ֖ת הַכַּפֹּֽרֶת׃}
{יָת אֲרוֹנָא דְּסָהֲדוּתָא וְיָת אֲרִיחוֹהִי וְיָת כָּפוּרְתָּא׃}
{the ark of the testimony, and the staves thereof, and the ark-cover;}{\arabic{verse}}
\threeverse{\arabic{verse}}%Ex.39:36
{אֶת־הַשֻּׁלְחָן֙ אֶת־כׇּל־כֵּלָ֔יו וְאֵ֖ת לֶ֥חֶם הַפָּנִֽים׃}
{יָת פָּתוּרָא יָת כָּל מָנוֹהִי וְיָת לְחֵים אַפַּיָּא׃}
{the table, all the vessels thereof, and the showbread;}{\arabic{verse}}
\threeverse{\arabic{verse}}%Ex.39:37
{אֶת־הַמְּנֹרָ֨ה הַטְּהֹרָ֜ה אֶת־נֵרֹתֶ֗יהָ נֵרֹ֛ת הַמַּֽעֲרָכָ֖ה וְאֶת־כׇּל־כֵּלֶ֑יהָ וְאֵ֖ת שֶׁ֥מֶן הַמָּאֽוֹר׃}
{יָת מְנָרְתָא דָּכִיתָא יָת בּוֹצִינַהָא בּוֹצִינֵי סִדְרָא וְיָת כָּל מָנַהָא וְיָת מִשְׁחָא דְּאַנְהָרוּתָא׃}
{the pure candlestick, the lamps thereof, even the lamps to be set in order, and all the vessels thereof, and the oil for the light;}{\arabic{verse}}
\threeverse{\arabic{verse}}%Ex.39:38
{וְאֵת֙ מִזְבַּ֣ח הַזָּהָ֔ב וְאֵת֙ שֶׁ֣מֶן הַמִּשְׁחָ֔ה וְאֵ֖ת קְטֹ֣רֶת הַסַּמִּ֑ים וְאֵ֕ת מָסַ֖ךְ פֶּ֥תַח הָאֹֽהֶל׃}
{וְיָת מַדְבְּחָא דְּדַהְבָּא וְיָת מִשְׁחָא דִּרְבוּתָא וְיָת קְטֹרֶת בּוּסְמַיָּא וְיָת פְּרָסָא דִּתְרַע מַשְׁכְּנָא׃}
{and the golden altar, and the anointing oil, and the sweet incense, and the screen for the door of the Tent;}{\arabic{verse}}
\threeverse{\arabic{verse}}%Ex.39:39
{אֵ֣ת \legarmeh  מִזְבַּ֣ח הַנְּחֹ֗שֶׁת וְאֶת־מִכְבַּ֤ר הַנְּחֹ֙שֶׁת֙ אֲשֶׁר־ל֔וֹ אֶת־בַּדָּ֖יו וְאֶת־כׇּל־כֵּלָ֑יו אֶת־הַכִּיֹּ֖ר וְאֶת־כַּנּֽוֹ׃}
{יָת מַדְבְּחָא דִּנְחָשָׁא וְיָת סְרָדָא דִּנְחָשָׁא דִּילֵיהּ יָת אֲרִיחוֹהִי וְיָת כָּל מָנוֹהִי יָת כִּיּוֹרָא וְיָת בְּסִיסֵיהּ׃}
{the brazen altar, and its grating of brass, its staves, and all its vessels, the laver and its base;}{\arabic{verse}}
\threeverse{\arabic{verse}}%Ex.39:40
{אֵת֩ קַלְעֵ֨י הֶחָצֵ֜ר אֶת־עַמֻּדֶ֣יהָ וְאֶת־אֲדָנֶ֗יהָ וְאֶת־הַמָּסָךְ֙ לְשַׁ֣עַר הֶֽחָצֵ֔ר אֶת־מֵיתָרָ֖יו וִיתֵדֹתֶ֑יהָ וְאֵ֗ת כׇּל־כְּלֵ֛י עֲבֹדַ֥ת הַמִּשְׁכָּ֖ן לְאֹ֥הֶל מוֹעֵֽד׃}
{יָת סְרָדֵי דָּרְתָא יָת עַמּוּדַהָא וְיָת סָמְכַהָא וְיָת פְּרָסָא לִתְרַע דָּרְתָא יָת אֲטוּנוֹהִי וְסִכַּהָא וְיָת כָּל מָנֵי פּוּלְחַן מַשְׁכְּנָא לְמַשְׁכַּן זִמְנָא׃}
{the hangings of the court, its pillars, and its sockets, and the screen for the gate of the court, the cords thereof, and the pins thereof, and all the instruments of the service of the tabernacle of the tent of meeting;}{\arabic{verse}}
\threeverse{\arabic{verse}}%Ex.39:41
{אֶת־בִּגְדֵ֥י הַשְּׂרָ֖ד לְשָׁרֵ֣ת בַּקֹּ֑דֶשׁ אֶת־בִּגְדֵ֤י הַקֹּ֙דֶשׁ֙ לְאַהֲרֹ֣ן הַכֹּהֵ֔ן וְאֶת־בִּגְדֵ֥י בָנָ֖יו לְכַהֵֽן׃}
{יָת לְבוּשֵׁי שִׁמּוּשָׁא לְשַׁמָּשָׁא בְּקוּדְשָׁא יָת לְבוּשֵׁי קוּדְשָׁא לְאַהֲרֹן כָּהֲנָא וְיָת לְבוּשֵׁי בְנוֹהִי לְשַׁמָּשָׁא׃}
{the plaited garments for ministering in the holy place; the holy garments for Aaron the priest, and the garments of his sons, to minister in the priest’s office.}{\arabic{verse}}
\threeverse{\arabic{verse}}%Ex.39:42
{כְּכֹ֛ל אֲשֶׁר־צִוָּ֥ה יְהֹוָ֖ה אֶת־מֹשֶׁ֑ה כֵּ֤ן עָשׂוּ֙ בְּנֵ֣י יִשְׂרָאֵ֔ל אֵ֖ת כׇּל־הָעֲבֹדָֽה׃}
{כְּכֹל דְּפַקֵּיד יְיָ יָת מֹשֶׁה כֵּן עֲבַדוּ בְנֵי יִשְׂרָאֵל יָת כָּל פּוּלְחָנָא׃}
{According to all that the \lord\space commanded Moses, so the children of Israel did all the work.}{\arabic{verse}}
\threeverse{\arabic{verse}}%Ex.39:43
{וַיַּ֨רְא מֹשֶׁ֜ה אֶת־כׇּל־הַמְּלָאכָ֗ה וְהִנֵּה֙ עָשׂ֣וּ אֹתָ֔הּ כַּאֲשֶׁ֛ר צִוָּ֥ה יְהֹוָ֖ה כֵּ֣ן עָשׂ֑וּ וַיְבָ֥רֶךְ אֹתָ֖ם מֹשֶֽׁה׃ \petucha 
\rashi{\rashiDH{ויברך אותם משה. }אמר להם יהי רצון שתשרה שכינה במעשה ידיכם, וִיהִי נֹעַם ה׳ אֱלֹהֵינוּ עָלֵינוּ וגו׳ (תהלים צ, יז), והוא אחד מי״א מזמורים שבתפלה למשה׃ 
}}
{וַחֲזָא מֹשֶׁה יָת כָּל עֲבִידְתָא וְהָא עֲבַדוּ יָתַהּ כְּמָא דְּפַקֵּיד יְיָ כֵּן עֲבַדוּ וּבָרֵיךְ יָתְהוֹן מֹשֶׁה׃}
{And Moses saw all the work, and, behold, they had done it; as the \lord\space had commanded, even so had they done it. And Moses blessed them.}{\arabic{verse}}
\newperek
\threeverse{\aliya{חמישי\newline (שביעי)}}%Ex.40:1
{וַיְדַבֵּ֥ר יְהֹוָ֖ה אֶל־מֹשֶׁ֥ה לֵּאמֹֽר׃}
{וּמַלֵּיל יְיָ עִם מֹשֶׁה לְמֵימַר׃}
{And the \lord\space spoke unto Moses, saying:}{\Roman{chap}}
\threeverse{\arabic{verse}}%Ex.40:2
{בְּיוֹם־הַחֹ֥דֶשׁ הָרִאשׁ֖וֹן בְּאֶחָ֣ד לַחֹ֑דֶשׁ תָּקִ֕ים אֶת־מִשְׁכַּ֖ן אֹ֥הֶל מוֹעֵֽד׃}
{בְּיוֹם יַרְחָא קַדְמָאָה בְּחַד לְיַרְחָא תְּקִים יָת מַשְׁכְּנָא מַשְׁכַּן זִמְנָא׃}
{’On the first day of the first month shalt thou rear up the tabernacle of the tent of meeting.}{\arabic{verse}}
\threeverse{\arabic{verse}}%Ex.40:3
{וְשַׂמְתָּ֣ שָׁ֔ם אֵ֖ת אֲר֣וֹן הָעֵד֑וּת וְסַכֹּתָ֥ עַל־הָאָרֹ֖ן אֶת־הַפָּרֹֽכֶת׃
\rashi{\rashiDH{וסכות על הארון. }לשון הגנה, שהרי מחיצה היתה׃ }}
{וּתְשַׁוֵּי תַּמָּן יָת אֲרוֹנָא דְּסָהֲדוּתָא וְתַטֵּיל עַל אֲרוֹנָא יָת פָּרוּכְתָּא׃}
{And thou shalt put therein the ark of the testimony, and thou shalt screen the ark with the veil.}{\arabic{verse}}
\threeverse{\arabic{verse}}%Ex.40:4
{וְהֵבֵאתָ֙ אֶת־הַשֻּׁלְחָ֔ן וְעָרַכְתָּ֖ אֶת־עֶרְכּ֑וֹ וְהֵבֵאתָ֙ אֶת־הַמְּנֹרָ֔ה וְהַעֲלֵיתָ֖ אֶת־נֵרֹתֶֽיהָ׃
\rashi{\rashiDH{וערכת את ערכו. }שתי מערכות של לחם הפנים׃}}
{וְתַעֵיל יָת פָּתוּרָא וְתַסְדֵּר יָת סִדְרֵיהּ וְתַעֵיל יָת מְנָרְתָא וְתַדְלֵיק יָת בּוֹצִינַהָא׃}
{And thou shalt bring in the table, and set in order the bread that is upon it; and thou shalt bring in the candlestick, and light the lamps thereof.}{\arabic{verse}}
\threeverse{\arabic{verse}}%Ex.40:5
{וְנָתַתָּ֞ה אֶת־מִזְבַּ֤ח הַזָּהָב֙ לִקְטֹ֔רֶת לִפְנֵ֖י אֲר֣וֹן הָעֵדֻ֑ת וְשַׂמְתָּ֛ אֶת־מָסַ֥ךְ הַפֶּ֖תַח לַמִּשְׁכָּֽן׃}
{וְתִתֵּין יָת מַדְבְּחָא דְּדַהְבָּא לִקְטֹרֶת בּוּסְמַיָּא קֳדָם אֲרוֹנָא דְּסָהֲדוּתָא וּתְשַׁוֵּי יָת פְּרָסָא דְּתַרְעָא לְמַשְׁכְּנָא׃}
{And thou shalt set the golden altar for incense before the ark of the testimony, and put the screen of the door to the tabernacle.}{\arabic{verse}}
\threeverse{\arabic{verse}}%Ex.40:6
{וְנָ֣תַתָּ֔ה אֵ֖ת מִזְבַּ֣ח הָעֹלָ֑ה לִפְנֵ֕י פֶּ֖תַח מִשְׁכַּ֥ן אֹֽהֶל־מוֹעֵֽד׃}
{וְתִתֵּין יָת מַדְבְּחָא דַּעֲלָתָא קֳדָם תְּרַע מַשְׁכְּנָא מַשְׁכַּן זִמְנָא׃}
{And thou shalt set the altar of burnt-offering before the door of the tabernacle of the tent of meeting.}{\arabic{verse}}
\threeverse{\arabic{verse}}%Ex.40:7
{וְנָֽתַתָּ֙ אֶת־הַכִּיֹּ֔ר בֵּֽין־אֹ֥הֶל מוֹעֵ֖ד וּבֵ֣ין הַמִּזְבֵּ֑חַ וְנָתַתָּ֥ שָׁ֖ם מָֽיִם׃}
{וְתִתֵּין יָת כִּיּוֹרָא בֵּין מַשְׁכַּן זִמְנָא וּבֵין מַדְבְּחָא וְתִתֵּין תַּמָּן מַיָּא׃}
{And thou shalt set the laver between the tent of meeting and the altar, and shalt put water therein.}{\arabic{verse}}
\threeverse{\arabic{verse}}%Ex.40:8
{וְשַׂמְתָּ֥ אֶת־הֶחָצֵ֖ר סָבִ֑יב וְנָ֣תַתָּ֔ אֶת־מָסַ֖ךְ שַׁ֥עַר הֶחָצֵֽר׃}
{וּתְשַׁוֵּי יָת דָּרְתָא סְחוֹר סְחוֹר וְתִתֵּין יָת פְּרָסָא דִּתְרַע דָּרְתָא׃}
{And thou shalt set up the court round about, and hang up the screen of the gate of the court.}{\arabic{verse}}
\threeverse{\arabic{verse}}%Ex.40:9
{וְלָקַחְתָּ֙ אֶת־שֶׁ֣מֶן הַמִּשְׁחָ֔ה וּמָשַׁחְתָּ֥ אֶת־הַמִּשְׁכָּ֖ן וְאֶת־כׇּל־אֲשֶׁר־בּ֑וֹ וְקִדַּשְׁתָּ֥ אֹת֛וֹ וְאֶת־כׇּל־כֵּלָ֖יו וְהָ֥יָה קֹֽדֶשׁ׃}
{וְתִסַּב יָת מִשְׁחָא דִּרְבוּתָא וּתְרַבֵּי יָת מַשְׁכְּנָא וְיָת כָּל דְּבֵיהּ וּתְקַדֵּישׁ יָתֵיהּ וְיָת כָּל מָנוֹהִי וִיהֵי קוּדְשָׁא׃}
{And thou shalt take the anointing oil, and anoint the tabernacle, and all that is therein, and shalt hallow it, and all the furniture thereof; and it shall be holy.}{\arabic{verse}}
\threeverse{\arabic{verse}}%Ex.40:10
{וּמָשַׁחְתָּ֛ אֶת־מִזְבַּ֥ח הָעֹלָ֖ה וְאֶת־כׇּל־כֵּלָ֑יו וְקִדַּשְׁתָּ֙ אֶת־הַמִּזְבֵּ֔חַ וְהָיָ֥ה הַמִּזְבֵּ֖חַ קֹ֥דֶשׁ קׇֽדָשִֽׁים׃}
{וּתְרַבֵּי יָת מַדְבְּחָא דַּעֲלָתָא וְיָת כָּל מָנוֹהִי וּתְקַדֵּשׁ יָת מַדְבְּחָא וִיהֵי מַדְבְּחָא קֹדֶשׁ קוּדְשִׁין׃}
{And thou shalt anoint the altar of burnt-offering, and all its vessels, and sanctify the altar; and the altar shall be most holy.}{\arabic{verse}}
\threeverse{\arabic{verse}}%Ex.40:11
{וּמָשַׁחְתָּ֥ אֶת־הַכִּיֹּ֖ר וְאֶת־כַּנּ֑וֹ וְקִדַּשְׁתָּ֖ אֹתֽוֹ׃}
{וּתְרַבֵּי יָת כִּיּוֹרָא וְיָת בְּסִיסֵיהּ וּתְקַדֵּישׁ יָתֵיהּ׃}
{And thou shalt anoint the laver and its base, and sanctify it.}{\arabic{verse}}
\threeverse{\arabic{verse}}%Ex.40:12
{וְהִקְרַבְתָּ֤ אֶֽת־אַהֲרֹן֙ וְאֶת־בָּנָ֔יו אֶל־פֶּ֖תַח אֹ֣הֶל מוֹעֵ֑ד וְרָחַצְתָּ֥ אֹתָ֖ם בַּמָּֽיִם׃}
{וּתְקָרֵיב יָת אַהֲרֹן וְיָת בְּנוֹהִי לִתְרַע מַשְׁכַּן זִמְנָא וְתַסְחֵי יָתְהוֹן בְּמַיָּא׃}
{And thou shalt bring Aaron and his sons unto the door of the tent of meeting, and shalt wash them with water.}{\arabic{verse}}
\threeverse{\arabic{verse}}%Ex.40:13
{וְהִלְבַּשְׁתָּ֙ אֶֽת־אַהֲרֹ֔ן אֵ֖ת בִּגְדֵ֣י הַקֹּ֑דֶשׁ וּמָשַׁחְתָּ֥ אֹת֛וֹ וְקִדַּשְׁתָּ֥ אֹת֖וֹ וְכִהֵ֥ן לִֽי׃}
{וְתַלְבֵּישׁ יָת אַהֲרֹן יָת לְבוּשֵׁי קוּדְשָׁא וּתְרַבֵּי יָתֵיהּ וּתְקַדֵּישׁ יָתֵיהּ וִישַׁמֵּישׁ קֳדָמָי׃}
{And thou shalt put upon Aaron the holy garments; and thou shalt anoint him, and sanctify him, that he may minister unto Me in the priest’s office.}{\arabic{verse}}
\threeverse{\arabic{verse}}%Ex.40:14
{וְאֶת־בָּנָ֖יו תַּקְרִ֑יב וְהִלְבַּשְׁתָּ֥ אֹתָ֖ם כֻּתֳּנֹֽת׃}
{וְיָת בְּנוֹהִי תְקָרֵיב וְתַלְבֵּישׁ יָתְהוֹן כִּתּוּנִין׃}
{And thou shalt bring his sons, and put tunics upon them.}{\arabic{verse}}
\threeverse{\arabic{verse}}%Ex.40:15
{וּמָשַׁחְתָּ֣ אֹתָ֗ם כַּאֲשֶׁ֤ר מָשַׁ֙חְתָּ֙ אֶת־אֲבִיהֶ֔ם וְכִהֲנ֖וּ לִ֑י וְ֠הָיְתָ֠ה לִהְיֹ֨ת לָהֶ֧ם מׇשְׁחָתָ֛ם לִכְהֻנַּ֥ת עוֹלָ֖ם לְדֹרֹתָֽם׃}
{וּתְרַבֵּי יָתְהוֹן כְּמָא דְּרַבִּיתָא יָת אֲבוּהוֹן וִישַׁמְּשׁוּן קֳדָמָי וּתְהֵי לְמִהְוֵי לְהוֹן רְבוּתְהוֹן לִכְהוּנַּת עָלַם לְדָרֵיהוֹן׃}
{And thou shalt anoint them, as thou didst anoint their father, that they may minister unto Me in the priest’s office; and their anointing shall be to them for an everlasting priesthood throughout their generations.’}{\arabic{verse}}
\threeverse{\arabic{verse}}%Ex.40:16
{וַיַּ֖עַשׂ מֹשֶׁ֑ה כְּ֠כֹ֠ל אֲשֶׁ֨ר צִוָּ֧ה יְהֹוָ֛ה אֹת֖וֹ כֵּ֥ן עָשָֽׂה׃ \setuma         }
{וַעֲבַד מֹשֶׁה כְּכֹל דְּפַקֵּיד יְיָ יָתֵיהּ כֵּן עֲבַד׃}
{Thus did Moses; according to all that the \lord\space commanded him, so did he.}{\arabic{verse}}
\threeverse{\aliya{ששי}}%Ex.40:17
{וַיְהִ֞י בַּחֹ֧דֶשׁ הָרִאשׁ֛וֹן בַּשָּׁנָ֥ה הַשֵּׁנִ֖ית בְּאֶחָ֣ד לַחֹ֑דֶשׁ הוּקַ֖ם הַמִּשְׁכָּֽן׃}
{וַהֲוָה בְּיַרְחָא קַדְמָאָה בְּשַׁתָּא תִּנְיֵיתָא בְּחַד לְיַרְחָא אִתָּקַם מַשְׁכְּנָא׃}
{And it came to pass in the first month in the second year, on the first day of the month, that the tabernacle was reared up.}{\arabic{verse}}
\threeverse{\arabic{verse}}%Ex.40:18
{וַיָּ֨קֶם מֹשֶׁ֜ה אֶת־הַמִּשְׁכָּ֗ן וַיִּתֵּן֙ אֶת־אֲדָנָ֔יו וַיָּ֙שֶׂם֙ אֶת־קְרָשָׁ֔יו וַיִּתֵּ֖ן אֶת־בְּרִיחָ֑יו וַיָּ֖קֶם אֶת־עַמּוּדָֽיו׃}
{וַאֲקֵים מֹשֶׁה יָת מַשְׁכְּנָא וִיהַב יָת סָמְכוֹהִי וְשַׁוִּי יָת דַּפּוֹהִי וִיהַב יָת עָבְרוֹהִי וַאֲקֵים יָת עַמּוּדוֹהִי׃}
{And Moses reared up the tabernacle, and laid its sockets, and set up the boards thereof, and put in the bars thereof, and reared up its pillars.}{\arabic{verse}}
\threeverse{\arabic{verse}}%Ex.40:19
{וַיִּפְרֹ֤שׂ אֶת־הָאֹ֙הֶל֙ עַל־הַמִּשְׁכָּ֔ן וַיָּ֜שֶׂם אֶת־מִכְסֵ֥ה הָאֹ֛הֶל עָלָ֖יו מִלְמָ֑עְלָה כַּאֲשֶׁ֛ר צִוָּ֥ה יְהֹוָ֖ה אֶת־מֹשֶֽׁה׃ \setuma         
\rashi{\rashiDH{ויפרש את האהל. }הן יריעות העזים׃}}
{וּפְרַס יָת פְּרָסָא עַל מַשְׁכְּנָא וְשַׁוִּי יָת חוּפָאָה דְּמַשְׁכְּנָא עֲלוֹהִי מִלְּעֵילָא כְּמָא דְּפַקֵּיד יְיָ יָת מֹשֶׁה׃}
{And he spread the tent over the tabernacle, and put the covering of the tent above upon it; as the \lord\space commanded Moses.}{\arabic{verse}}
\threeverse{\arabic{verse}}%Ex.40:20
{וַיִּקַּ֞ח וַיִּתֵּ֤ן אֶת־הָעֵדֻת֙ אֶל־הָ֣אָרֹ֔ן וַיָּ֥שֶׂם אֶת־הַבַּדִּ֖ים עַל־הָאָרֹ֑ן וַיִּתֵּ֧ן אֶת־הַכַּפֹּ֛רֶת עַל־הָאָרֹ֖ן מִלְמָֽעְלָה׃
\rashi{\rashiDH{את העדות. }הלוחות׃ 
}}
{וּנְסֵיב וִיהַב יָת סָהֲדוּתָא בַּאֲרוֹנָא וְשַׁוִּי יָת אֲרִיחַיָּא עַל אֲרוֹנָא וִיהַב יָת כָּפוּרְתָּא עַל אֲרוֹנָא מִלְּעֵילָא׃}
{And he took and put the testimony into the ark, and set the staves on the ark, and put the ark-cover above upon the ark.}{\arabic{verse}}
\threeverse{\arabic{verse}}%Ex.40:21
{וַיָּבֵ֣א אֶת־הָאָרֹן֮ אֶל־הַמִּשְׁכָּן֒ וַיָּ֗שֶׂם אֵ֚ת פָּרֹ֣כֶת הַמָּסָ֔ךְ וַיָּ֕סֶךְ עַ֖ל אֲר֣וֹן הָעֵד֑וּת כַּאֲשֶׁ֛ר צִוָּ֥ה יְהֹוָ֖ה אֶת־מֹשֶֽׁה׃ \setuma         }
{וְאַעֵיל יָת אֲרוֹנָא לְמַשְׁכְּנָא וְשַׁוִּי יָת פָּרוּכְתָּא דִּפְרָסָא וְאַטֵּיל עַל אֲרוֹנָא דְּסָהֲדוּתָא כְּמָא דְּפַקֵּיד יְיָ יָת מֹשֶׁה׃}
{And he brought the ark into the tabernacle, and set up the veil of the screen, and screened the ark of the testimony; as the \lord\space commanded Moses.}{\arabic{verse}}
\threeverse{\arabic{verse}}%Ex.40:22
{וַיִּתֵּ֤ן אֶת־הַשֻּׁלְחָן֙ בְּאֹ֣הֶל מוֹעֵ֔ד עַ֛ל יֶ֥רֶךְ הַמִּשְׁכָּ֖ן צָפֹ֑נָה מִח֖וּץ לַפָּרֹֽכֶת׃
\rashi{\rashiDH{על ירך המשכן צפונה. }בחצי הצפוני של רוחב הבית (יומא לג׃)׃}\rashi{\rashiDH{ירך. }כתרגומו צִדָּא, כירך הזה שהוא בצדו של אדם׃ }}
{וִיהַב יָת פָּתוּרָא בְּמַשְׁכַּן זִמְנָא עַל שִׁדָּא דְּמַשְׁכְּנָא צִפּוּנָא מִבַּרָא לְפָרוּכְתָּא׃}
{And he put the table in the tent of meeting, upon the side of the tabernacle northward, without the veil.}{\arabic{verse}}
\threeverse{\arabic{verse}}%Ex.40:23
{וַיַּעֲרֹ֥ךְ עָלָ֛יו עֵ֥רֶךְ לֶ֖חֶם לִפְנֵ֣י יְהֹוָ֑ה כַּאֲשֶׁ֛ר צִוָּ֥ה יְהֹוָ֖ה אֶת־מֹשֶֽׁה׃ \setuma         }
{וְסַדַּר עֲלוֹהִי סִדְרִין דִּלְחֵים קֳדָם יְיָ כְּמָא דְּפַקֵּיד יְיָ יָת מֹשֶׁה׃}
{And he set a row of bread in order upon it before the \lord; as the \lord\space commanded Moses.}{\arabic{verse}}
\threeverse{\arabic{verse}}%Ex.40:24
{וַיָּ֤שֶׂם אֶת־הַמְּנֹרָה֙ בְּאֹ֣הֶל מוֹעֵ֔ד נֹ֖כַח הַשֻּׁלְחָ֑ן עַ֛ל יֶ֥רֶךְ הַמִּשְׁכָּ֖ן נֶֽגְבָּה׃}
{וְשַׁוִּי יָת מְנָרְתָא בְּמַשְׁכַּן זִמְנָא לָקֳבֵיל פָּתוּרָא עַל שִׁדָּא דְּמַשְׁכְּנָא דָּרוֹמָא׃}
{And he put the candlestick in the tent of meeting, over against the table, on the side of the tabernacle southward.}{\arabic{verse}}
\threeverse{\arabic{verse}}%Ex.40:25
{וַיַּ֥עַל הַנֵּרֹ֖ת לִפְנֵ֣י יְהֹוָ֑ה כַּאֲשֶׁ֛ר צִוָּ֥ה יְהֹוָ֖ה אֶת־מֹשֶֽׁה׃ \setuma         }
{וְאַדְלֵיק בּוֹצִינַיָּא קֳדָם יְיָ כְּמָא דְּפַקֵּיד יְיָ יָת מֹשֶׁה׃}
{And he lighted the lamps before the \lord; as the \lord\space commanded Moses.}{\arabic{verse}}
\threeverse{\arabic{verse}}%Ex.40:26
{וַיָּ֛שֶׂם אֶת־מִזְבַּ֥ח הַזָּהָ֖ב בְּאֹ֣הֶל מוֹעֵ֑ד לִפְנֵ֖י הַפָּרֹֽכֶת׃}
{וְשַׁוִּי יָת מַדְבְּחָא דְּדַהְבָּא בְּמַשְׁכַּן זִמְנָא קֳדָם פָּרוּכְתָּא׃}
{And he put the golden altar in the tent of meeting before the veil;}{\arabic{verse}}
\threeverse{\arabic{verse}}%Ex.40:27
{וַיַּקְטֵ֥ר עָלָ֖יו קְטֹ֣רֶת סַמִּ֑ים כַּאֲשֶׁ֛ר צִוָּ֥ה יְהֹוָ֖ה אֶת־מֹשֶֽׁה׃ \setuma         
\rashi{\rashiDH{ויקטר עליו קטורת. }שחרית וערבית, כמו שנאמר בַּבֹּקֶר בַּבֹּקֶר בְּהֵיטִיבֹו אֶת הַנֵּרֹת וגו׳ (שמות ל, ז)׃ 
}}
{וְאַקְטַר עֲלוֹהִי קְטֹרֶת בּוּסְמִין כְּמָא דְּפַקֵּיד יְיָ יָת מֹשֶׁה׃}
{and he burnt thereon incense of sweet spices; as the \lord\space commanded Moses.}{\arabic{verse}}
\threeverse{\aliya{שביעי}}%Ex.40:28
{וַיָּ֛שֶׂם אֶת־מָסַ֥ךְ הַפֶּ֖תַח לַמִּשְׁכָּֽן׃}
{וְשַׁוִּי יָת פְּרָסָא דְּתַרְעָא לְמַשְׁכְּנָא׃}
{And he put the screen of the door to the tabernacle.}{\arabic{verse}}
\threeverse{\arabic{verse}}%Ex.40:29
{וְאֵת֙ מִזְבַּ֣ח הָעֹלָ֔ה שָׂ֕ם פֶּ֖תַח מִשְׁכַּ֣ן אֹֽהֶל־מוֹעֵ֑ד וַיַּ֣עַל עָלָ֗יו אֶת־הָעֹלָה֙ וְאֶת־הַמִּנְחָ֔ה כַּאֲשֶׁ֛ר צִוָּ֥ה יְהֹוָ֖ה אֶת־מֹשֶֽׁה׃ \setuma         
\rashi{\rashiDH{ויעל עליו וגו׳. }אף ביום השמיני למלואים שהוא יום הקמת המשכן, שמש משה והקריב קרבנות צבור, חוץ מאותן שנצטוה אהרן בו ביום, שנאמר קְרַב אֶל הַמִּזְבַּח וגו׳ (ויקרא ט, ז)׃ }\rashi{\rashiDH{את העולה. }עולת התמיד׃}\rashi{\rashiDH{ואת המנחה.} מנחת נסכים של תמיד, כמו שנאמר וְעִשָׂרֹן סֹלֶת בָּלוּל בַּשֶׁמֶן וגו׳ (שמות כט, מ)׃ }}
{וְיָת מַדְבְּחָא דַּעֲלָתָא שַׁוִּי בִּתְרַע מַשְׁכְּנָא מַשְׁכַּן זִמְנָא וְאַסֵּיק עֲלוֹהִי יָת עֲלָתָא וְיָת מִנְחָתָא כְּמָא דְּפַקֵּיד יְיָ יָת מֹשֶׁה׃}
{And the altar of burnt-offering he set at the door of the tabernacle of the tent of meeting, and offered upon it the burnt-offering and the meal-offering; as the \lord\space commanded Moses.}{\arabic{verse}}
\threeverse{\arabic{verse}}%Ex.40:30
{וַיָּ֙שֶׂם֙ אֶת־הַכִּיֹּ֔ר בֵּֽין־אֹ֥הֶל מוֹעֵ֖ד וּבֵ֣ין הַמִּזְבֵּ֑חַ וַיִּתֵּ֥ן שָׁ֛מָּה מַ֖יִם לְרׇחְצָֽה׃}
{וְשַׁוִּי יָת כִּיּוֹרָא בֵּין מַשְׁכַּן זִמְנָא וּבֵין מַדְבְּחָא וִיהַב תַּמָּן מַיָּא לְקִדּוּשׁ׃}
{And he set the laver between the tent of meeting and the altar, and put water therein, wherewith to wash;}{\arabic{verse}}
\threeverse{\arabic{verse}}%Ex.40:31
{וְרָחֲצ֣וּ מִמֶּ֔נּוּ מֹשֶׁ֖ה וְאַהֲרֹ֣ן וּבָנָ֑יו אֶת־יְדֵיהֶ֖ם וְאֶת־רַגְלֵיהֶֽם׃
\rashi{\rashiDH{ורחצו ממנו משה ואהרן ובניו. }יום שמיני למלואים הושוו כולם לכהונה, ותרגומו וִיקַדְּשׁוּן מִנֵּיהּ, בו ביום קָדַשׁ משה עמהם׃ }}
{וּמְקַדְּשִׁין מִנֵּיהּ מֹשֶׁה וְאַהֲרֹן וּבְנוֹהִי יָת יְדֵיהוֹן וְיָת רַגְלֵיהוֹן׃}
{that Moses and Aaron and his sons might wash their hands and their feet thereat;}{\arabic{verse}}
\threeverse{\arabic{verse}}%Ex.40:32
{בְּבֹאָ֞ם אֶל־אֹ֣הֶל מוֹעֵ֗ד וּבְקׇרְבָתָ֛ם אֶל־הַמִּזְבֵּ֖חַ יִרְחָ֑צוּ כַּאֲשֶׁ֛ר צִוָּ֥ה יְהֹוָ֖ה אֶת־מֹשֶֽׁה׃ \setuma         
\rashi{\rashiDH{ובקרבתם. }כמו ובקרבם, כשיקרבו׃ }}
{בְּמֵיעַלְהוֹן לְמַשְׁכַּן זִמְנָא וּבְמִקְרַבְהוֹן לְמַדְבְּחָא מְקַדְּשִׁין כְּמָא דְּפַקֵּיד יְיָ יָת מֹשֶׁה׃}
{when they went into the tent of meeting, and when they came near unto the altar, they should wash; as the \lord\space commanded Moses.}{\arabic{verse}}
\threeverse{\arabic{verse}}%Ex.40:33
{וַיָּ֣קֶם אֶת־הֶחָצֵ֗ר סָבִיב֙ לַמִּשְׁכָּ֣ן וְלַמִּזְבֵּ֔חַ וַיִּתֵּ֕ן אֶת־מָסַ֖ךְ שַׁ֣עַר הֶחָצֵ֑ר וַיְכַ֥ל מֹשֶׁ֖ה אֶת־הַמְּלָאכָֽה׃ \petucha }
{וַאֲקֵים יָת דָּרְתָא סְחוֹר סְחוֹר לְמַשְׁכְּנָא וּלְמַדְבְּחָא וִיהַב יָת פְּרָסָא דִּתְרַע דָּרְתָא וְשֵׁיצִי מֹשֶׁה יָת עֲבִידְתָא׃}
{And he reared up the court round about the tabernacle and the altar, and set up the screen of the gate of the court. So Moses finished the work.}{\arabic{verse}}
\threeverse{\aliya{מפטיר}}%Ex.40:34
{וַיְכַ֥ס הֶעָנָ֖ן אֶת־אֹ֣הֶל מוֹעֵ֑ד וּכְב֣וֹד יְהֹוָ֔ה מָלֵ֖א אֶת־הַמִּשְׁכָּֽן׃}
{וַחֲפָא עֲנָנָא יָת מַשְׁכַּן זִמְנָא וִיקָרָא דַּייָ אִתְמְלִי יָת מַשְׁכְּנָא׃}
{Then the cloud covered the tent of meeting, and the glory of the \lord\space filled the tabernacle.}{\arabic{verse}}
\threeverse{\arabic{verse}}%Ex.40:35
{וְלֹא־יָכֹ֣ל מֹשֶׁ֗ה לָבוֹא֙ אֶל־אֹ֣הֶל מוֹעֵ֔ד כִּֽי־שָׁכַ֥ן עָלָ֖יו הֶעָנָ֑ן וּכְב֣וֹד יְהֹוָ֔ה מָלֵ֖א אֶת־הַמִּשְׁכָּֽן׃
\rashi{\rashiDH{ולא יכול משה לבוא אל אהל מועד. }וכתוב אחד אומר, ובבא משה אל אהל מועד (במדבר ז, פט), בא הכתוב השלישי והכריע ביניהם, כי שכן עליו הענן, אמר מעתה, כל זמן שהיה עליו הענן, לא היה יכול לבוא, נסתלק הענן, נכנס ומדבר עמו (פתיחה לת״כ ח)׃ }}
{וְלָא יָכֵיל מֹשֶׁה לְמֵיעַל לְמַשְׁכַּן זִמְנָא אֲרֵי שְׁרָא עֲלוֹהִי עֲנָנָא וִיקָרָא דַּייָ אִתְמְלִי יָת מַשְׁכְּנָא׃}
{And Moses was not able to enter into the tent of meeting, because the cloud abode thereon, and the glory of the \lord\space filled the tabernacle.—}{\arabic{verse}}
\threeverse{\arabic{verse}}%Ex.40:36
{וּבְהֵעָל֤וֹת הֶֽעָנָן֙ מֵעַ֣ל הַמִּשְׁכָּ֔ן יִסְע֖וּ בְּנֵ֣י יִשְׂרָאֵ֑ל בְּכֹ֖ל מַסְעֵיהֶֽם׃}
{וּבְאִסְתַּלָּקוּת עֲנָנָא מֵעִלָּוֵי מַשְׁכְּנָא נָטְלִין בְּנֵי יִשְׂרָאֵל בְּכֹל מַטְּלָנֵיהוֹן׃}
{And whenever the cloud was taken up from over the tabernacle, the children of Israel went onward, throughout all their journeys.}{\arabic{verse}}
\threeverse{\arabic{verse}}%Ex.40:37
{וְאִם־לֹ֥א יֵעָלֶ֖ה הֶעָנָ֑ן וְלֹ֣א יִסְע֔וּ עַד־י֖וֹם הֵעָלֹתֽוֹ׃}
{וְאִם לָא מִסְתַּלַּק עֲנָנָא וְלָא נָטְלִין עַד יוֹם אִסְתַּלָּקוּתֵיהּ׃}
{But if the cloud was not taken up, then they journeyed not till the day that it was taken up.}{\arabic{verse}}
\threeverse{\arabic{verse}}%Ex.40:38
{כִּי֩ עֲנַ֨ן יְהֹוָ֤ה עַֽל־הַמִּשְׁכָּן֙ יוֹמָ֔ם וְאֵ֕שׁ תִּהְיֶ֥ה לַ֖יְלָה בּ֑וֹ לְעֵינֵ֥י כׇל־בֵּֽית־יִשְׂרָאֵ֖ל בְּכׇל־מַסְעֵיהֶֽם׃
\rashi{\rashiDH{לעיני כל בית ישראל בכל מסעיהם. }בכל מסע שהיו נוסעים, היה הענן שוכן במקום אשר יחנו שם. מקום חנייתם אף הוא קרוי מסע, וכן וַיֵּלֶךְ לְמַסָּעיו (בראשית יג, ג), וכן אֵלֶּה מַסְעֵי (במדבר לג, א), לפי שממקום החנייה חזרו ונסעו, לכך נקראו כולן מסעות׃ }}
{אֲרֵי עֲנַן יְקָרָא דַּייָ עַל מַשְׁכְּנָא בִּימָמָא וְחֵיזוּ אִישָׁתָא הָוֵי בְּלֵילְיָא בֵיהּ לְעֵינֵי כָל בֵּית יִשְׂרָאֵל בְּכָל מַטְּלָנֵיהוֹן׃}
{For the cloud of the \lord\space was upon the tabernacle by day, and there was fire therein by night, in the sight of all the house of Israel, throughout all their journeys.—}{\arabic{verse}}
\newperek
