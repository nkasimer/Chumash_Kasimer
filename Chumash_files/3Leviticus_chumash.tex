\aliyacounter{ראשון}
\newparsha{ויקרא}
\newseder{1}
\threeverse{\aliya{ויקרא}\newline\vspace{-4pt}\newline\seder{א}}%Leviticus1:1
{וַיִּקְרָ֖\footnotesize א\normalsize\space אֶל\maqqaf מֹשֶׁ֑ה וַיְדַבֵּ֤ר יְהֹוָה֙ אֵלָ֔יו מֵאֹ֥הֶל מוֹעֵ֖ד לֵאמֹֽר׃}
{וּקְרָא לְמֹשֶׁה וּמַלֵּיל יְיָ עִמֵּיהּ מִמַּשְׁכַּן זִמְנָא לְמֵימַר׃}
{And the \lord\space called unto Moses, and spoke unto him out of the tent of meeting, saying:}{\Roman{chap}}
\rashi{\rashiDH{ויקרא אל משה.} לכל דברות ולכל אמירות ולכל צוויים קדמה קריאה, לשון חבה, לשון שמלאכי השרת משתמשין בו, שנאמר, וְקָרָא זֶה אֶל זֶה (ישעיה ו, ג), אבל לנביאי האומות עכו״ם נגלה אליהן בלשון עראי וטומאה, שנאמר וַיִּקָּר אֱלֹהִים אֶל בִּלְעָם (במדבר כג, ד)׃\quad \rashiDH{ויקרא אל משה.} הקול הולך ומגיע לאזניו, וכל ישראל לא שומעין. יכול אף להפסקות היתה קריאה, תלמוד לומר וידבר, לדבור היתה קריאה ולא להפסקות, ומה היו הפסקות משמשות, ליתן ריוח למשה להתבונן בין פרשה לפרשה ובין ענין לענין, קל וחומר להדיוט הלומד מן ההדיוט׃\quad \rashiDH{אליו.} למעט את אהרן, רבי יהודה אומר י״ג דברות נאמרו בתורה למשה ולאהרן, וכנגדן נאמרו י״ג מיעוטין, ללמדך שלא לאהרן נאמרו אלא למשה שיאמר לאהרן, ואלו הן י״ג מיעוטין לְדַבֵּר אִתּוֹ (במדבר ז, פט), מִדַּבֵּר אֵלָיו (שם), וַיְדַבֵּר אֵלָיו, וְנוֹעַדְתִּי לְךָ (שמות כה, כב), כולן בתורת כהנים, יכול ישמעו את קול הקריאה, תלמוד לומר קול לו, קול אליו, משה שומע וכל ישראל לא שמעו.׃\quad \rashiDH{מאהל מועד.} מלמד שהיה הקול נפסק ולא היה יוצא חוץ לאהל, יכול מפני שהקול נמוך, תלמוד לומר אֶת הַקּוֹל (במדבר ז, פט), מהו הקול, הוא הקול המפורש בתהלים קוֹל ה׳ בַּכֹּחַ, קוֹל ה׳ בֶּהָדָר, קוֹל ה׳ שֹׁבֵר אֲרָזִים (תהלים כט ד, ה), אם כן למה נאמר מאהל מועד, מלמד שהיה הקול נפסק, כיוצא בו וְקוֹל כַּנְפֵי הַכְּרוּבִים נִשְׁמַע עַד הֶחָצֵר הַחִיצֹנָה (יחזקאל י, ה), יכול מפני שהקול נמוך, תלמוד לומר כְּקוֹל אֵל שַׁדַּי בְּדַבְּרוֹ (שם), אם כן למה נאמר עד החצר החיצונה, שכיון שמגיע שם היה נפסק׃\quad \rashiDH{מאהל מועד לאמר.} יכול מכל הבית, תלמוד לומר מֵעַל הַכַּפֹּרֶת (במדבר ז, פט), יכול מעל הכפורת כולה, תלמוד לומר מִבֵּין שְׁנֵי הַכְּרֻבִים (שם)׃\quad \rashiDH{לאמר.} צא ואמור להם דברי כבושים, בשבילכם הוא נדבר עמי, שכן מצינו שכל ל״ח שנה שהיו ישראל במדבר כמנודים מן המרגלים ואילך, לא נתייחד הדבור עם משה, שנאמר וַיְהִי כַּאֲשֶׁר תַּמּוּ כָּל אַנְשֵׁי הַמִּלְחָמָה לָמוּת. וַיְדַבֵּר ה׳ אֵלַי לֵאמֹר(דברים ב טז, יז), אלי היה הדבור. דבר אחר, צא ואמור להן דברי, והשיבני אם יקבלום, כמו שנאמר וַיָּשֶׁב משֶׁה אֶת דִּבְרֵי הָעָם וגו׳ (שמות יט, ח)׃}
\threeverse{\arabic{verse}}%Leviticus1:2
{דַּבֵּ֞ר אֶל\maqqaf בְּנֵ֤י יִשְׂרָאֵל֙ וְאָמַרְתָּ֣ אֲלֵהֶ֔ם אָדָ֗ם כִּֽי\maqqaf יַקְרִ֥יב מִכֶּ֛ם קׇרְבָּ֖ן לַֽיהֹוָ֑ה מִן\maqqaf הַבְּהֵמָ֗ה מִן\maqqaf הַבָּקָר֙ וּמִן\maqqaf הַצֹּ֔אן תַּקְרִ֖יבוּ אֶת\maqqaf קׇרְבַּנְכֶֽם׃}
{מַלֵּיל עִם בְּנֵי יִשְׂרָאֵל וְתֵימַר לְהוֹן אֱנָשׁ אֲרֵי יְקָרֵיב מִנְּכוֹן קוּרְבָּנָא קֳדָם יְיָ מִן בְּעִירָא מִן תּוֹרֵי וּמִן עָנָא תְּקָרְבוּן יָת קוּרְבָּנְכוֹן׃}
{Speak unto the children of Israel, and say unto them: When any man of you bringeth an offering unto the \lord, ye shall bring your offering of the cattle, even of the herd or of the flock.}{\arabic{verse}}
\rashi{\rashiDH{אדם כי יקריב מכם.} כשיקריב, בקרבנות נדבה דִּבֵּר הענין׃\quad \rashiDH{אדם.} למה נאמר, מה אדם הראשון לא הקריב מן הגזל שהכל היה שלו, אף אתם לא תקריבו מן הגזל׃\quad \rashiDH{הבהמה.} יכול אף חיה בכלל, תלמוד לומר בקר וצאן׃\quad \rashiDH{מן הבהמה.} ולא כולה, להוציא את הרובע ואת הנרבע׃\quad \rashiDH{מן הבקר.} להוציא את הנעבד׃ 
\quad \rashiDH{מן הצאן.} להוציא את המוקצה׃\quad \rashiDH{ומן הצאן.} להוציא את הנוגח שהמית, כשהוא אומר למטה מן הענין מן הבקר, שאין תלמוד לומר, להוציא את הטריפה׃\quad \rashiDH{תקריבו.} מלמד שֶׁשְּׁנַיִם מתנדבים עולה בשותפות׃\quad \rashiDH{קרבנכם.} מלמד שהיא באה נדבת צבור, היא עולת קיץ המזבח הבאה מן המותרות (שבועות יב.)׃}
\threeverse{\arabic{verse}}%Leviticus1:3
{אִם\maqqaf עֹלָ֤ה קׇרְבָּנוֹ֙ מִן\maqqaf הַבָּקָ֔ר זָכָ֥ר תָּמִ֖ים יַקְרִיבֶ֑נּוּ אֶל\maqqaf פֶּ֜תַח אֹ֤הֶל מוֹעֵד֙ יַקְרִ֣יב אֹת֔וֹ לִרְצֹנ֖וֹ לִפְנֵ֥י יְהֹוָֽה׃}
{אִם עֲלָתָא קוּרְבָּנֵיהּ מִן תּוֹרֵי דְּכַר שְׁלִים יְקָרְבִנֵּיהּ לִתְרַע מַשְׁכַּן זִמְנָא יְקָרֵיב יָתֵיהּ לְרַעֲוָא לֵיהּ קֳדָם יְיָ׃}
{If his offering be a burnt-offering of the herd, he shall offer it a male without blemish; he shall bring it to the door of the tent of meeting, that he may be accepted before the \lord.}{\arabic{verse}}
\rashi{\rashiDH{זכר.} ולא נקבה, כשהוא אומר זכר למטה, שאין תלמוד לומר, זכר ולא טומטום ואנדרוגינוס (בכורות מא׃)׃\quad \rashiDH{תמים.} בלא מום׃ 
\quad \rashiDH{אל פתח אהל מועד.} מטפל בהבאתו עד העזרה. מהו אומר יקריב יקריב, אפילו נתערבה עולת ראובן בעולת שמעון יקריב כל אחד, לשם מי שהוא, וכן עולה בחולין ימכרו החולין לצרכי עולות, והרי הן כולן עולות, ותקרב כל אחת לשם מי שהוא, יכול אפילו נתערבה בפסולין או בשאינו מינו, תלמוד לומר יקריבנו׃\quad \rashiDH{יקריב אותו.} מלמד שכופין אותו, יכול בעל כרחו תלמוד לומר לרצונו, הא כיצד, כופין אותו עד שיאמר רוצה אני (ראש השנה ו.)׃\quad \rashiDH{לפני ה׳ וסמך.} אין סמיכה בְּבָמָה׃}
\threeverse{\arabic{verse}}%Leviticus1:4
{וְסָמַ֣ךְ יָד֔וֹ עַ֖ל רֹ֣אשׁ הָעֹלָ֑ה וְנִרְצָ֥ה ל֖וֹ לְכַפֵּ֥ר עָלָֽיו׃}
{וְיִסְמוֹךְ יְדֵיהּ עַל רֵישׁ עֲלָתָא וְיִתְרְעֵי לֵיהּ לְכַפָּרָא עֲלוֹהִי׃}
{And he shall lay his hand upon the head of the burnt-offering; and it shall be accepted for him to make atonement for him.}{\arabic{verse}}
\rashi{\rashiDH{על ראש העלה.} להביא עולת חובה לסמיכה, ולהביא עולת הצאן׃\quad \rashiDH{העלה.} פרט לעולת העוף׃\quad \rashiDH{ונרצה לו.} על מה הוא מרצה לו, אם תאמר על כריתות ומיתות ״ד, או מיתה בידי שמים, או מלקות, הרי עונשן אמור, הא אינו מרצה אלא על עשה, ועל לאו שנתק לעשה׃}
\threeverse{\aliya{לוי}}%Leviticus1:5
{וְשָׁחַ֛ט אֶת\maqqaf בֶּ֥ן הַבָּקָ֖ר לִפְנֵ֣י יְהֹוָ֑ה וְ֠הִקְרִ֠יבוּ בְּנֵ֨י אַהֲרֹ֤ן הַכֹּֽהֲנִים֙ אֶת\maqqaf הַדָּ֔ם וְזָרְק֨וּ אֶת\maqqaf הַדָּ֤ם עַל\maqqaf הַמִּזְבֵּ֙חַ֙ סָבִ֔יב אֲשֶׁר\maqqaf פֶּ֖תַח אֹ֥הֶל מוֹעֵֽד׃}
{וְיִכּוֹס יָת בַּר תּוֹרֵי קֳדָם יְיָ וִיקָרְבוּן בְּנֵי אַהֲרֹן כָּהֲנַיָּא יָת דְּמָא וְיִזְרְקוּן יָת דְּמָא עַל מַדְבְּחָא סְחוֹר סְחוֹר דְּבִתְרַע מַשְׁכַּן זִמְנָא׃}
{And he shall kill the bullock before the \lord; and Aaron’s sons, the priests, shall present the blood, and dash the blood round about against the altar that is at the door of the tent of meeting.}{\arabic{verse}}
\rashi{\rashiDH{ושחט והקריבו הכהנים.} מִקַּבָּלָה ואילך מצות כהונה, למד על השחיטה שכשרה בזר׃ 
\quad \rashiDH{לפני ה׳.} בעזרה׃\quad \rashiDH{והקריבו.} זו קבלה שהיא הראשונה, ומשמעה לשון הולכה, למדנו שתיהן (ס״א ששתיהן) בבני אהרן (חגיגה יא.)׃\quad \rashiDH{בני אהרן.} יכול חללים, תלמוד לומר הכהנים׃\quad \rashiDH{את הדם וזרקו את הדם.} מה תלמוד לומר דם דם ב׳ פעמים, להביא את שנתערב במינו או בשאינו מינו, יכול אף בפסולים, או בחטאות הפנימיות, או בחטאות החצוניות, שאלו למעלה והיא למטה, תלמוד לומר במקום אחר אֶת דָּמוֹ (פסוק יא)׃\quad \rashiDH{וזרקו.} עומד למטה, וזורק מן הכלי לכותל המזבח למטה מחוט הסיקרא כנגד הזויות, לכך נאמר סביב, שיהא הדם ניתן בד׳ רוחות המזבח, או יכול יקיפנו כחוט, תלמוד לומר וזרקו, ואי אפשר להקיף בזריקה, אי וזרקו יכול בזריקה אחת, תלמוד לומר סביב, הא כיצד, נותן שתי מתנות שהן ד׳׃\quad \rashiDH{אשר פתח אהל מועד.} ולא בזמן שהוא מפורק׃ 
}
\threeverse{\arabic{verse}}%Leviticus1:6
{וְהִפְשִׁ֖יט אֶת\maqqaf הָעֹלָ֑ה וְנִתַּ֥ח אֹתָ֖הּ לִנְתָחֶֽיהָ׃}
{וְיַשְׁלַח יָת עֲלָתָא וִיפַלֵּיג יָתַהּ לְאֶבְרַֽהָא׃}
{And he shall flay the burnt-offering, and cut it into its pieces.}{\arabic{verse}}
\rashi{\rashiDH{והפשיט את העולה.} מה תלמוד לומר העולה, לרבות את כל העולות להפשט ונתוח׃\quad \rashiDH{אותה לנתחיה.} ולא נתחיה לנתחים (חולין יא.)׃}
\threeverse{\arabic{verse}}%Leviticus1:7
{וְ֠נָתְנ֠וּ בְּנֵ֨י אַהֲרֹ֧ן הַכֹּהֵ֛ן אֵ֖שׁ עַל\maqqaf הַמִּזְבֵּ֑חַ וְעָרְכ֥וּ עֵצִ֖ים עַל\maqqaf הָאֵֽשׁ׃}
{וְיִתְּנוּן בְּנֵי אַהֲרֹן כָּהֲנָא אִישָׁתָא עַל מַדְבְּחָא וִיסַדְּרוּן אָעַיָּא עַל אִישָׁתָא׃}
{And the sons of Aaron the priest shall put fire upon the altar, and lay wood in order upon the fire.}{\arabic{verse}}
\rashi{\rashiDH{ונתנו אש.} אף על פי שהאש יורדת מן השמים, מצוה להביא מן ההדיוט׃\quad \rashiDH{בני אהרן הכהן.} כשהוא בכיהונו, הא אם עבד בבגדי כהן הדיוט, עבודתו פסולה׃}
\threeverse{\arabic{verse}}%Leviticus1:8
{וְעָרְכ֗וּ בְּנֵ֤י אַהֲרֹן֙ הַכֹּ֣הֲנִ֔ים אֵ֚ת הַנְּתָחִ֔ים אֶת\maqqaf הָרֹ֖אשׁ וְאֶת\maqqaf הַפָּ֑דֶר עַל\maqqaf הָעֵצִים֙ אֲשֶׁ֣ר עַל\maqqaf הָאֵ֔שׁ אֲשֶׁ֖ר עַל\maqqaf הַמִּזְבֵּֽחַ׃}
{וִיסַדְּרוּן בְּנֵי אַהֲרֹן כָּהֲנַיָּא יָת אֶבְרַיָּא יָת רֵישָׁא וְיָת תַּרְבָּא עַל אָעַיָּא דְּעַל אִישָׁתָא דְּעַל מַדְבְּחָא׃}
{And Aaron’s sons, the priests, shall lay the pieces, and the head, and the suet, in order upon the wood that is on the fire which is upon the altar;}{\arabic{verse}}
\rashi{\rashiDH{בני אהרן הכהנים.} כשהם בכיהונם, הא כהן הדיוט שעבד בשמונה בגדים, עבודתו פסולה׃\quad \rashiDH{את הנתחים את הראש.} לפי שאין הראש בכלל הפשט, שכבר הותז בשחיטה, לפיכך הוצרך למנותו לעצמו (חולין כז.)׃\quad \rashiDH{ואת הפדר.} למה נאמר, ללמדך שמעלהו עם הראש ומכסה בו את בית השחיטה, וזהו דרך כבוד של מעלה׃\quad \rashiDH{אשר על המזבח.} שלא יהיו הַגְּזִירִין יוצאין חוץ למערכה׃}
\threeverse{\arabic{verse}}%Leviticus1:9
{וְקִרְבּ֥וֹ וּכְרָעָ֖יו יִרְחַ֣ץ בַּמָּ֑יִם וְהִקְטִ֨יר הַכֹּהֵ֤ן אֶת\maqqaf הַכֹּל֙ הַמִּזְבֵּ֔חָה עֹלָ֛ה אִשֵּׁ֥ה רֵֽיחַ\maqqaf נִיח֖וֹחַ לַֽיהֹוָֽה׃ \setuma }
{וְגַוֵּיהּ וּכְרָעוֹהִי יְחַלֵּיל בְּמַיָּא וְיַסֵּיק כָּהֲנָא יָת כּוֹלָא לְמַדְבְּחָא עֲלָתָא קוּרְבַּן דְּמִתְקַבַּל בְּרַעֲוָא קֳדָם יְיָ׃}
{but its inwards and its legs shall he wash with water; and the priest shall make the whole smoke on the altar, for a burnt-offering, an offering made by fire, of a sweet savour unto the \lord.}{\arabic{verse}}
\rashi{\rashiDH{עולה.} לשם עולה יקטירנו׃\quad \rashiDH{אשה.} כשישחטנו יהא שוחטו לשם האש, וכל אשה לשון אש, פושיי״ר בלע״ז׃\quad \rashiDH{ניחוח.} נחת רוח לְפָנַי, שאמרתי ונעשה רצוני׃}
\threeverse{\aliya{ישראל}}%Leviticus1:10
{וְאִם\maqqaf מִן\maqqaf הַצֹּ֨אן קׇרְבָּנ֧וֹ מִן\maqqaf הַכְּשָׂבִ֛ים א֥וֹ מִן\maqqaf הָעִזִּ֖ים לְעֹלָ֑ה זָכָ֥ר תָּמִ֖ים יַקְרִיבֶֽנּוּ׃}
{וְאִם מִן עָנָא קוּרְבָּנֵיהּ מִן אִמְּרַיָּא אוֹ מִן בְּנֵי עִזַּיָּא לַעֲלָתָא דְּכַר שְׁלִים יְקָרְבִנֵּיהּ׃}
{And if his offering be of the flock, whether of the sheep, or of the goats, for a burnt-offering, he shall offer it a male without blemish.}{\arabic{verse}}
\rashi{\rashiDH{ואם מן הצאן.} וי״ו מוסיף על ענין ראשון, ולמה הפסיק, ליתן ריוח למשה להתבונן בין פרשה לפרשה׃\quad \rashiDH{מן הצאן מן הכשבים או מן העזים.} הרי אלו ג׳ מיעוטין, פרט לזקן לחולה ולמזוהם׃ 
}
\threeverse{\arabic{verse}}%Leviticus1:11
{וְשָׁחַ֨ט אֹת֜וֹ עַ֣ל יֶ֧רֶךְ הַמִּזְבֵּ֛חַ צָפֹ֖נָה לִפְנֵ֣י יְהֹוָ֑ה וְזָרְק֡וּ בְּנֵי֩ אַהֲרֹ֨ן הַכֹּהֲנִ֧ים אֶת\maqqaf דָּמ֛וֹ עַל\maqqaf הַמִּזְבֵּ֖חַ סָבִֽיב׃}
{וְיִכּוֹס יָתֵיהּ עַל שִׁדָּא דְּמַדְבְּחָא צִפּוּנָא קֳדָם יְיָ וְיִזְרְקוּן בְּנֵי אַהֲרֹן כָּהֲנַיָּא יָת דְּמֵיהּ עַל מַדְבְּחָא סְחוֹר סְחוֹר׃}
{And he shall kill it on the side of the altar northward before the \lord; and Aaron’s sons, the priests, shall dash its blood against the altar round about.}{\arabic{verse}}
\rashi{\rashiDH{על ירך המזבח.} על צד המזבח׃\quad \rashiDH{צפונה לפני ה׳.} ואין צפון בבמה׃ 
}
\threeverse{\arabic{verse}}%Leviticus1:12
{וְנִתַּ֤ח אֹתוֹ֙ לִנְתָחָ֔יו וְאֶת\maqqaf רֹאשׁ֖וֹ וְאֶת\maqqaf פִּדְר֑וֹ וְעָרַ֤ךְ הַכֹּהֵן֙ אֹתָ֔ם עַל\maqqaf הָֽעֵצִים֙ אֲשֶׁ֣ר עַל\maqqaf הָאֵ֔שׁ אֲשֶׁ֖ר עַל\maqqaf הַמִּזְבֵּֽחַ׃}
{וִיפַלֵּיג יָתֵיהּ לְאֶבְרוֹהִי וְיָת רֵישֵׁיהּ וְיָת תַּרְבֵּיהּ וְיַסְדַּר כָּהֲנָא יָתְהוֹן עַל אָעַיָּא דְּעַל אִישָׁתָא דְּעַל מַדְבְּחָא׃}
{And he shall cut it into its pieces; and the priest shall lay them, with its head and its suet, in order on the wood that is on the fire which is upon the altar.}{\arabic{verse}}
\threeverse{\arabic{verse}}%Leviticus1:13
{וְהַקֶּ֥רֶב וְהַכְּרָעַ֖יִם יִרְחַ֣ץ בַּמָּ֑יִם וְהִקְרִ֨יב הַכֹּהֵ֤ן אֶת\maqqaf הַכֹּל֙ וְהִקְטִ֣יר הַמִּזְבֵּ֔חָה עֹלָ֣ה ה֗וּא אִשֵּׁ֛ה רֵ֥יחַ נִיחֹ֖חַ לַיהֹוָֽה׃ \petucha }
{וְגַוָּא וּכְרָעַיָּא יְחַלֵּיל בְּמַיָּא וִיקָרֵיב כָּהֲנָא יָת כּוֹלָא וְיַסֵּיק לְמַדְבְּחָא עֲלָתָא הוּא קוּרְבַּן דְּמִתְקַבַּל בְּרַעֲוָא קֳדָם יְיָ׃}
{But the inwards and the legs shall he wash with water; and the priest shall offer the whole, and make it smoke upon the altar; it is a burnt-offering, an offering made by fire, of a sweet savour unto the \lord.}{\arabic{verse}}
\aliyacounter{שני}
\threeverse{\aliya{שני}}%Leviticus1:14
{וְאִ֧ם מִן\maqqaf הָע֛וֹף עֹלָ֥ה קׇרְבָּנ֖וֹ לַֽיהֹוָ֑ה וְהִקְרִ֣יב מִן\maqqaf הַתֹּרִ֗ים א֛וֹ מִן\maqqaf בְּנֵ֥י הַיּוֹנָ֖ה אֶת\maqqaf קׇרְבָּנֽוֹ׃}
{וְאִם מִן עוֹפָא עֲלָתָא קוּרְבָּנֵיהּ קֳדָם יְיָ וִיקָרֵיב מִן שַׁפְנִינַיָּא אוֹ מִן בְּנֵי יוֹנָה יָת קוּרְבָּנֵיהּ׃}
{And if his offering to the \lord\space be a burnt-offering of fowls, then he shall bring his offering of turtle-doves, or of young pigeons.}{\arabic{verse}}
\rashi{\rashiDH{מן העוף.} ולא כל העוף. לפי שנאמר תָּמִים זָכָר בַּבָּקָר בַּכְּשָׂבִים וּבָעִזִּים (להלן כב, יט), תמות וזכרות בבהמה, ואין תמות וזכרות בעופות, יכול אף מחוסר אבר, תלמוד לומר מן העוף׃\quad \rashiDH{התורים.} גדולים ולא קטנים׃\quad \rashiDH{בני היונה.} קטנים ולא גדולים׃\quad \rashiDH{מן התורים או מן בני היונה.} פרט לתחלת הציהוב, שבזה ושבזה שהוא פסול, שהוא גדול אצל בני יונה וקטן אצל תורים׃}
\threeverse{\arabic{verse}}%Leviticus1:15
{וְהִקְרִיב֤וֹ הַכֹּהֵן֙ אֶל\maqqaf הַמִּזְבֵּ֔חַ וּמָלַק֙ אֶת\maqqaf רֹאשׁ֔וֹ וְהִקְטִ֖יר הַמִּזְבֵּ֑חָה וְנִמְצָ֣ה דָמ֔וֹ עַ֖ל קִ֥יר הַמִּזְבֵּֽחַ׃}
{וִיקָרְבִנֵּיהּ כָּהֲנָא לְמַדְבְּחָא וְיִמְלוֹק יָת רֵישֵׁיהּ וְיַסֵּיק לְמַדְבְּחָא וְיִתְמְצֵי דְּמֵיהּ עַל כּוֹתֶל מַדְבְּחָא׃}
{And the priest shall bring it unto the altar, and pinch off its head, and make it smoke on the altar; and the blood thereof shall be drained out on the side of the altar.}{\arabic{verse}}
\rashi{\rashiDH{והקריבו.} אפילו פרידה אחת יביא׃\quad \rashiDH{הכהן ומלק.} אין מליקה בכלי, אלא בעצמו של כהן, קוצץ בצפרנו ממול העורף וחותך מפרקת עד שמגיע לסימנין וקוצצן׃ 
\quad \rashiDH{ונמצה דמו.} לשון מִיץ אַפַּיִם (משלי ל, ג), כִּי אָפֵס הַמֵּץ (ישעיה טז, ד), כובש בית השחיטה על קיר המזבח, והדם מתמצה ויורד׃\quad \rashiDH{ומלק והקטיר ונמצה.} אפשר לומר כן, מאחר שהוא מקטיר הוא מוצה, אלא מה הקטרה הראש בעצמו והגוף בעצמו וכו׳, אף מליקה כן. ופשוטו של מקרא מסורס הוא ומלק והקטיר, וקודם הקטרה ונמצה דמו כבר׃}
\threeverse{\arabic{verse}}%Leviticus1:16
{וְהֵסִ֥יר אֶת\maqqaf מֻרְאָת֖וֹ בְּנֹצָתָ֑הּ וְהִשְׁלִ֨יךְ אֹתָ֜הּ אֵ֤צֶל הַמִּזְבֵּ֙חַ֙ קֵ֔דְמָה אֶל\maqqaf מְק֖וֹם הַדָּֽשֶׁן׃}
{וְיַעְדֵּי יָת זְפָקֵיהּ בְּאוּכְלֵיהּ וְיִרְמֵי יָתַהּ בִּסְטַר מַדְבְּחָא קִדּוּמָא בַּאֲתַר דְּמַקְרִין קִטְמָא׃}
{And he shall take away its crop with the feathers thereof, and cast it beside the altar on the east part, in the place of the ashes.}{\arabic{verse}}
\rashi{\rashiDH{מראתו.} מקום הרעי, וזה הזפק׃\quad \rashiDH{בנוצתה.} עם בני מעיה (זבחים סד׃), ונוצה לשון דבר המאוס, כמו כִּי נָצוּ גַּם נָעוּ (איכה ד, טו), וזהו שתרגם אונקלוס בְּאוּכְלֵיהּ, וזה מדרשו של אבא יוסי בן חנן, שאמר, נוטל את הקורקבן עמה. ורבותינו ז״ל אמרו, קודר סביב הזפק בסכין כעין ארובה ונוטלו עם הנוצה שעל העור (זבחים סה.). בעולת בהמה שאינה אוכלת אלא באבוס בעליה, נאמר והקרב והכרעים ירחץ במים, והקטיר, בעוף שנזון מן הגזל, נאמר, והשליך את המעים שאכלו מן הגזל׃\quad \rashiDH{אצל המזבח קדמה.} במזרחו של כבש׃\quad \rashiDH{אל מקום הדשן.} מקום שנותנין שם תרומת הדשן בכל בוקר, ודישון מזבח הפנימי, והמנורה, וכולם נבלעים שם במקומן (יומא כא.)׃}
\threeverse{\arabic{verse}}%Leviticus1:17
{וְשִׁסַּ֨ע אֹת֣וֹ בִכְנָפָיו֮ לֹ֣א יַבְדִּיל֒ וְהִקְטִ֨יר אֹת֤וֹ הַכֹּהֵן֙ הַמִּזְבֵּ֔חָה עַל\maqqaf הָעֵצִ֖ים אֲשֶׁ֣ר עַל\maqqaf הָאֵ֑שׁ עֹלָ֣ה ה֗וּא אִשֵּׁ֛ה רֵ֥יחַ נִיחֹ֖חַ לַיהֹוָֽה׃ \setuma }
{וִיפָרֵיק יָתֵיהּ בְּכַנְפוֹהִי לָא יַפְרֵישׁ וְיַסֵּיק יָתֵיהּ כָּהֲנָא לְמַדְבְּחָא עַל אָעַיָּא דְּעַל אִישָׁתָא עֲלָתָא הוּא קוּרְבַּן דְּמִתְקַבַּל בְּרַעֲוָא קֳדָם יְיָ׃}
{And he shall rend it by the wings thereof, but shall not divide it asunder; and the priest shall make it smoke upon the altar, upon the wood that is upon the fire; it is a burnt-offering, an offering made by fire, of a sweet savour unto the \lord.}{\arabic{verse}}
\rashi{\rashiDH{ושסע.} אין שיסוע אלא ביד, וכן הוא אומר בשמשון וַיְשַׁסְּעֵהוּ כְּשַׁסַּע הַגְּדִי (שופטים יד, ו  זבחים סה׃)׃\quad \rashiDH{בכנפיו.} עם כנפיו, אינו צריך למרוט כנפי נוצתו׃\quad \rashiDH{בכנפיו.} נוצה ממש. והלא אין לך הדיוט שמריח ריח רע של כנפים נשרפים ואין נפשו קצה עליו, ולמה אמר הכתוב והקטיר, כדי שיהא המזבח שָׂבֵעַ ומהודר בקרבנו של עני׃\quad \rashiDH{לא יבדיל.} אינו מפרקו לגמרי לב׳ חתיכות, אלא קורעו מגבו. נאמר בעוף ריח ניחוח (פסוק ט), ונאמר בבהמה ריח ניחוח, לומר לך אחד המרבה ואחד הממעיט ובלבד שיכוין את לבו לשמים׃}
\newperek
\threeverse{\Roman{chap}}%Leviticus2:1
{וְנֶ֗פֶשׁ כִּֽי\maqqaf תַקְרִ֞יב קׇרְבַּ֤ן מִנְחָה֙ לַֽיהֹוָ֔ה סֹ֖לֶת יִהְיֶ֣ה קׇרְבָּנ֑וֹ וְיָצַ֤ק עָלֶ֙יהָ֙ שֶׁ֔מֶן וְנָתַ֥ן עָלֶ֖יהָ לְבֹנָֽה׃}
{וֶאֱנָשׁ אֲרֵי יְקָרֵיב קוּרְבַּן מִנְחָתָא קֳדָם יְיָ סוּלְתָּא יְהֵי קוּרְבָּנֵיהּ וִירִיק עֲלַהּ מִשְׁחָא וְיִתֵּין עֲלַהּ לְבוֹנְתָא׃}
{And when any one bringeth a meal-offering unto the \lord, his offering shall be of fine flour; and he shall pour oil upon it, and put frankincense thereon.}{\Roman{chap}}
\rashi{\rashiDH{ונפש כי תקריב.} לא נאמר נפש בכל קרבנות נדבה אלא במנחה, מי דרכו להתנדב מנחה, עני, אמר הקב״ה מעלה אני עליו כאילו הקריב נפשו (מנחות קד׃)׃\quad \rashiDH{סלת יהיה קרבנו.} האומר הרי עלי מנחה סתם, מביא מנחת סלת, שהיא הראשונה שבמנחות (שם) ונקמצת כשהיא סלת, כמו שמפורש בענין. לפי שנאמרו כאן ה׳ מיני מנחות, וכולן באות אפויות קודם קמיצה חוץ מזו, לכך קרויה מנחת סלת׃\quad \rashiDH{סלת.} אין סלת אלא מן החטין, שנאמר סֹלֶת חִטִּים (שמות כט, ב), ואין מנחה פחותה מעשרון, שנאמר וְעִשָׂרֹון סֹלֶת לְמִנְחָה (להלן יד, כא), עשרון לכל מנחה׃ 
\quad \rashiDH{ויצק עליה שמן.} על כולה׃\quad \rashiDH{ונתן עליה לבונה.} על מקצתה, מניח קומץ לבונה עליה לצד אחד, ומה ראית לומר כן, שאין ריבוי אחר ריבוי בתורה אלא למעט ד״א שמן על כולה מפני שהוא נבלל עמה, ונקמץ עמה, כמו שנאמר מסלתה ומשמנה, ולבונה על מקצתה שאינה נבללת עמה ולא נקמצת עמה, שנאמר על כל לבונתה, שלאחר שקמץ מלקט את הלבונה כולה מעליה, ומקטירה׃\quad \rashiDH{ויצק ונתן והביאה.} מלמד שיציקה וּבְלִילָה כשרים בזר׃ 
}
\threeverse{\arabic{verse}}%Leviticus2:2
{וֶֽהֱבִיאָ֗הּ אֶל\maqqaf בְּנֵ֣י אַהֲרֹן֮ הַכֹּהֲנִים֒ וְקָמַ֨ץ מִשָּׁ֜ם מְלֹ֣א קֻמְצ֗וֹ מִסׇּלְתָּהּ֙ וּמִשַּׁמְנָ֔הּ עַ֖ל כׇּל\maqqaf לְבֹנָתָ֑הּ וְהִקְטִ֨יר הַכֹּהֵ֜ן אֶת\maqqaf אַזְכָּרָתָהּ֙ הַמִּזְבֵּ֔חָה אִשֵּׁ֛ה רֵ֥יחַ נִיחֹ֖חַ לַיהֹוָֽה׃}
{וְיַיְתֵינַהּ לְוָת בְּנֵי אַהֲרֹן כָּהֲנַיָּא וְיִקְמוֹץ מִתַּמָּן מְלֵי קוּמְצֵיהּ מִסֻּלְתַּהּ וּמִמִּשְׁחַהּ עַל כָּל לְבוֹנְתַהּ וְיַסֵּיק כָּהֲנָא יָת אַדְכָרְתַהּ לְמַדְבְּחָא קוּרְבַּן דְּמִתְקַבַּל בְּרַעֲוָא קֳדָם יְיָ׃}
{And he shall bring it to Aaron’s sons the priests; and he shall take thereout his handful of the fine flour thereof, and of the oil thereof, together with all the frankincense thereof; and the priest shall make the memorial-part thereof smoke upon the altar, an offering made by fire, of a sweet savour unto the \lord.}{\arabic{verse}}
\rashi{\rashiDH{הכהנים וקמץ.} מקמיצה ואילך מצות כהונה׃\quad \rashiDH{וקמץ משם.} ממקום שרגלי הזר עומדות, ללמדך שהקמיצה כשרה בכל מקום בעזרה, אף בי״א אמה של מקום דריסת רגלי ישראל (יומא טז׃)׃\quad \rashiDH{מלא קמצו.} יכול מְבֹרָץ, מְבַצְבֵּץ ויוצא לכל צד, תלמוד לומר במקום אחר וְהֵרִים מִמֶּנּוּ בְּקֻמְצוֹ (להלן ו, ח), לא יהא כשר אלא מה שבתוך הקומץ, אי בקמצו יכול חסר, תלמוד לומר מלא, הא כיצד, חופה ג׳ אצבעותיו על פס ידו (מנחות יא.), וזהו קומץ במשמע, לשון העברית׃\quad \rashiDH{על כל לבונתה.} לבד כל הלבונה יהא הקומץ מלא׃\quad \rashiDH{לבונתה והקטיר.} אף הלבונה בהקטרה׃\quad \rashiDH{מלא קמצו מסלתה ומשמנה.} הא אם קמץ ועלה בידו גרגיר מלח או קורט לבונה פסולה׃\quad \rashiDH{אזכרתה.} הקומץ העולה לגבוה הוא זכרון המנחה, שבו נזכר בעליה לטובה ולנחת רוח׃ 
}
\threeverse{\arabic{verse}}%Leviticus2:3
{וְהַנּוֹתֶ֙רֶת֙ מִן\maqqaf הַמִּנְחָ֔ה לְאַהֲרֹ֖ן וּלְבָנָ֑יו קֹ֥דֶשׁ קׇֽדָשִׁ֖ים מֵאִשֵּׁ֥י יְהֹוָֽה׃ \setuma }
{וּדְיִשְׁתְּאַר מִן מִנְחָתָא לְאַהֲרֹן וְלִבְנוֹהִי קֹדֶשׁ קוּדְשִׁין מִקּוּרְבָּנַיָּא דַּייָ׃}
{But that which is left of the meal-offering shall be Aaron’s and his sons’ ; it is a thing most holy of the offerings of the \lord\space made by fire.}{\arabic{verse}}
\rashi{\rashiDH{לאהרן ולבניו.} כהן גדול נוטל חלק בראש שלא במחלוקת, וההדיוט במחלוקת׃\quad \rashiDH{קדש קדשים.} היא להם׃\quad \rashiDH{מאשי ה׳.} אין להם חלק בה אלא לאחר מתנות האישים׃ 
}
\threeverse{\arabic{verse}}%Leviticus2:4
{וְכִ֥י תַקְרִ֛ב קׇרְבַּ֥ן מִנְחָ֖ה מַאֲפֵ֣ה תַנּ֑וּר סֹ֣לֶת חַלּ֤וֹת מַצֹּת֙ בְּלוּלֹ֣ת בַּשֶּׁ֔מֶן וּרְקִיקֵ֥י מַצּ֖וֹת מְשֻׁחִ֥ים בַּשָּֽׁמֶן׃ \setuma }
{וַאֲרֵי תְקָרֵיב קוּרְבַּן מִנְחָתָא מַאֲפֵה תַנּוּר סֹלֶת גְּרִיצָן פַּטִּירָן דְּפִילָן בִּמְשַׁח וְאֶסְפּוֹגִין פַּטִּירִין דִּמְשִׁיחִין בִּמְשַׁח׃}
{And when thou bringest a meal-offering baked in the oven, it shall be unleavened cakes of fine flour mingled with oil, or unleavened wafers spread with oil.}{\arabic{verse}}
\rashi{\rashiDH{וכי תקריב וגו׳.} שאמר הרי עלי מנחת מאפה תנור, ולימד הכתוב שיביא או חלות או רקיקין, החלות בלולות, והרקיקין משוחין (מנחות עד׃), ונחלקו רבותינו במשיחתן (שם עה.), יש אומרים מושחן וחוזרן ומושחן עד שיכלה כל השמן שבלוג, שכל המנחות טעונות לוג שמן, ויש אומרים מושחן כמין כף יונית, ושאר השמן נאכל בפני עצמו לכהנים, מה תלמוד לומר בשמן בשמן שני פעמים, להכשיר שמן שני ושלישי היוצא מן הזיתים, ואין צריך שמן ראשון אלא למנורה שנאמר בו זך. ושנינו במנחות (עו.) כל המנחות האפויות לפני קמיצתן ונקמצות על ידי פתיתה כולן באות עשר עשר חלות, והאמור בה רקיקין בא עשר רקיקין׃}
\threeverse{\arabic{verse}}%Leviticus2:5
{וְאִם\maqqaf מִנְחָ֥ה עַל\maqqaf הַֽמַּחֲבַ֖ת קׇרְבָּנֶ֑ךָ סֹ֛לֶת בְּלוּלָ֥ה בַשֶּׁ֖מֶן מַצָּ֥ה תִהְיֶֽה׃}
{וְאִם מִנְחָתָא עַל מַסְרֵיתָא קוּרְבָּנָךְ סוּלְתָּא דְּפִילָא בִמְשַׁח פַּטִּיר תְּהֵי׃}
{And if thy offering be a meal-offering baked on a griddle, it shall be of fine flour unleavened, mingled with oil.}{\arabic{verse}}
\rashi{\rashiDH{ואם מנחה על המחבת.} שאמר הרי עלי מנחת מחבת, וכלי הוא שהיה במקדש שאופין בו מנחה על האור בשמן, והכלי אינו עמוק אלא צף, ומעשה המנחה שבתוכו קשין, שמתוך שהיא צפה האור שורף את השמן (מנחות סג.), וכולן טעונות ׳ מתנות, שמן יציקה, ובלילה, ומתן שמן בכלי, קודם לעשייתן׃\quad \rashiDH{סלת בלולה בשמן.} מלמד שבוללן בעודן סלת׃}
\threeverse{\arabic{verse}}%Leviticus2:6
{פָּת֤וֹת אֹתָהּ֙ פִּתִּ֔ים וְיָצַקְתָּ֥ עָלֶ֖יהָ שָׁ֑מֶן מִנְחָ֖ה הִֽוא׃ \setuma }
{בַּצַּע יָתַהּ בִּצּוּעִין וּתְרִיק עֲלַהּ מִשְׁחָא מִנְחָתָא הִיא׃}
{Thou shalt break it in pieces, and pour oil thereon; it is a meal-offering.}{\arabic{verse}}
\rashi{\rashiDH{פתות אותה פתים.} לרבות כל המנחות הנאפות קודם קמיצה לפתיתה (שם עה.)׃\quad \rashiDH{ויצקת עליה שמן מנחה הוא.} לרבות כל המנחות ליציקה, יכול אף מנחת מאפה תנור כן, תלמוד לומר עליה, אוציא את החלות ולא אוציא את הרקיקין, תלמוד לומר הוא׃}
\aliyacounter{שלישי}
\threeverse{\aliya{שלישי}}%Leviticus2:7
{וְאִם\maqqaf מִנְחַ֥ת מַרְחֶ֖שֶׁת קׇרְבָּנֶ֑ךָ סֹ֥לֶת בַּשֶּׁ֖מֶן תֵּעָשֶֽׂה׃}
{וְאִם מִנְחָתָא רָדְתָא קוּרְבָּנָךְ סֹלֶת בִּמְשַׁח תִּתְעֲבֵיד׃}
{And if thy offering be a meal-offering of the stewing-pan, it shall be made of fine flour with oil.}{\arabic{verse}}
\rashi{\rashiDH{מרחשת.} כלי הוא שהיה במקדש עמוק, ומתוך שהיא עמוקה שַׁמְנָהּ צָבוּר ואין הָאוּר שורפו, לפיכך מעשה מנחה העשויין לתוכה רוחשין, כל דבר רך ע״י משקה נראה כרוחש ומנענע׃}
\threeverse{\arabic{verse}}%Leviticus2:8
{וְהֵבֵאתָ֣ אֶת\maqqaf הַמִּנְחָ֗ה אֲשֶׁ֧ר יֵעָשֶׂ֛ה מֵאֵ֖לֶּה לַיהֹוָ֑ה וְהִקְרִיבָהּ֙ אֶל\maqqaf הַכֹּהֵ֔ן וְהִגִּישָׁ֖הּ אֶל\maqqaf הַמִּזְבֵּֽחַ׃}
{וְתַיְתֵי יָת מִנְחָתָא דְּתִתְעֲבֵיד מֵאִלֵּין קֳדָם יְיָ וִיקָרְבִנַּהּ לְכָהֲנָא וִיקָרְבִנַּהּ לְמַדְבְּחָא׃}
{And thou shalt bring the meal-offering that is made of these things unto the \lord; and it shall be presented unto the priest, and he shall bring it unto the altar.}{\arabic{verse}}
\rashi{\rashiDH{אשר יעשה מאלה.} מאחד מן המינים הללו׃\quad \rashiDH{והקריבה.} בעליה אל הכהן׃ 
\quad \rashiDH{והגישה.} הכהן׃\quad \rashiDH{אל המזבח.} מגישה לקרן דרומית מערבית של מזבח (זבחים סג׃)׃}
\threeverse{\arabic{verse}}%Leviticus2:9
{וְהֵרִ֨ים הַכֹּהֵ֤ן מִן\maqqaf הַמִּנְחָה֙ אֶת\maqqaf אַזְכָּ֣רָתָ֔הּ וְהִקְטִ֖יר הַמִּזְבֵּ֑חָה אִשֵּׁ֛ה רֵ֥יחַ נִיחֹ֖חַ לַיהֹוָֽה׃}
{וְיַפְרֵישׁ כָּהֲנָא מִן מִנְחָתָא יָת אַדְכָרְתַהּ וְיַסֵּיק לְמַדְבְּחָא קוּרְבַּן דְּמִתְקַבַּל בְּרַעֲוָא קֳדָם יְיָ׃}
{And the priest shall take off from the meal-offering the memorial-part thereof, and shall make it smoke upon the altar—an offering made by fire, of a sweet savour unto the \lord.}{\arabic{verse}}
\rashi{\rashiDH{את אזכרתה.} הוא הקומץ׃}
\threeverse{\arabic{verse}}%Leviticus2:10
{וְהַנּוֹתֶ֙רֶת֙ מִן\maqqaf הַמִּנְחָ֔ה לְאַהֲרֹ֖ן וּלְבָנָ֑יו קֹ֥דֶשׁ קׇֽדָשִׁ֖ים מֵאִשֵּׁ֥י יְהֹוָֽה׃}
{וּדְיִשְׁתְּאַר מִן מִנְחָתָא לְאַהֲרֹן וְלִבְנוֹהִי קֹדֶשׁ קוּדְשִׁין מִקּוּרְבָּנַיָּא דַּייָ׃}
{But that which is left of the meal-offering shall be Aaron’s and his sons’ ; it is a thing most holy of the offerings of the \lord\space made by fire.}{\arabic{verse}}
\threeverse{\arabic{verse}}%Leviticus2:11
{כׇּל\maqqaf הַמִּנְחָ֗ה אֲשֶׁ֤ר תַּקְרִ֙יבוּ֙ לַיהֹוָ֔ה לֹ֥א תֵעָשֶׂ֖ה חָמֵ֑ץ כִּ֤י כׇל\maqqaf שְׂאֹר֙ וְכׇל\maqqaf דְּבַ֔שׁ לֹֽא\maqqaf תַקְטִ֧ירוּ מִמֶּ֛נּוּ אִשֶּׁ֖ה לַֽיהֹוָֽה׃}
{כָּל מִנְחָתָא דִּתְקָרְבוּן קֳדָם יְיָ לָא תִתְעֲבֵיד חֲמִיעַ אֲרֵי כָל חֲמִיר וְכָל דְּבַשׁ לָא תַסְּקוּן מִנֵּיהּ קוּרְבָּנָא קֳדָם יְיָ׃}
{No meal-offering, which ye shall bring unto the \lord, shall be made with leaven; for ye shall make no leaven, nor any honey, smoke as an offering made by fire unto the \lord.}{\arabic{verse}}
\rashi{\rashiDH{וכל דבש.} כל מתיקת פרי קרוי דבש׃}
\threeverse{\arabic{verse}}%Leviticus2:12
{קׇרְבַּ֥ן רֵאשִׁ֛ית תַּקְרִ֥יבוּ אֹתָ֖ם לַיהֹוָ֑ה וְאֶל\maqqaf הַמִּזְבֵּ֥חַ לֹא\maqqaf יַעֲל֖וּ לְרֵ֥יחַ נִיחֹֽחַ׃}
{קוּרְבַּן קַדְמַאי תְּקָרְבוּן יָתְהוֹן קֳדָם יְיָ וּלְמַדְבְּחָא לָא יִתַּסְקוּן לְאִתְקַבָּלָא בְּרַעֲוָא׃}
{As an offering of first-fruits ye may bring them unto the \lord; but they shall not come up for a sweet savour on the altar.}{\arabic{verse}}
\rashi{\rashiDH{קרבן ראשית תקריבו.} מה יש לך להביא מן השאור ומן הדבש, קרבן ראשית, שתי הלחם של עצרת הבאים מן השאור, שנאמר חָמֵץ תֵּאָפֶינָה (להלן כג, יז), ובכורים מן הדבש, כמו בכורי תאנים ותמרים (מנחות נח.)׃}
\threeverse{\arabic{verse}}%Leviticus2:13
{וְכׇל\maqqaf קׇרְבַּ֣ן מִנְחָתְךָ֮ בַּמֶּ֣לַח תִּמְלָח֒ וְלֹ֣א תַשְׁבִּ֗ית מֶ֚לַח בְּרִ֣ית אֱלֹהֶ֔יךָ מֵעַ֖ל מִנְחָתֶ֑ךָ עַ֥ל כׇּל\maqqaf קׇרְבָּנְךָ֖ תַּקְרִ֥יב מֶֽלַח׃ \setuma }
{וְכָל קוּרְבַּן מִנְחָתָךְ בְּמִלְחָא תִמְלַח וְלָא תְבַטֵּיל מְלַח קְיָם אֱלָהָךְ מֵעַל מִנְחָתָךְ עַל כָּל קוּרְבָּנָךְ תְּקָרֵיב מִלְחָא׃}
{And every meal-offering of thine shalt thou season with salt; neither shalt thou suffer the salt of the covenant of thy God to be lacking from thy meal-offering; with all thy offerings thou shalt offer salt.}{\arabic{verse}}
\rashi{\rashiDH{מלח ברית.} שהברית כרותה למלח מששת ימי בראשית שהובטחו המים התחתונים ליקרב במזבח במלח, וניסוך המים בחג׃ 
\quad \rashiDH{על כל קרבנך.} על עולת בהמה ועוף, ואמורי כל הקדשים כולן (מנחות כ.)׃ 
}
\threeverse{\arabic{verse}}%Leviticus2:14
{וְאִם\maqqaf תַּקְרִ֛יב מִנְחַ֥ת בִּכּוּרִ֖ים לַיהֹוָ֑ה אָבִ֞יב קָל֤וּי בָּאֵשׁ֙ גֶּ֣רֶשׂ כַּרְמֶ֔ל תַּקְרִ֕יב אֵ֖ת מִנְחַ֥ת בִּכּוּרֶֽיךָ׃}
{וְאִם תְּקָרֵיב מִנְחַת בִּכּוּרִין קֳדָם יְיָ אֲבִיב קְלֵי בְנוּר פֵּירוּכָן רַכִּיכָן תְּקָרֵיב יָת מִנְחַת בִּכּוּרָךְ׃}
{And if thou bring a meal-offering of first-fruits unto the \lord, thou shalt bring for the meal-offering of thy first-fruits corn in the ear parched with fire, even groats of the fresh ear.}{\arabic{verse}}
\rashi{\rashiDH{ואם תקריב.} הרי אם משמש בלשון כי, שהרי אין זה רשות, שהרי במנחת העומר הכתוב מדבר שהיא חובה, וכן וְאִם יִהְיֶה הַיּוֹבֵל וגו׳ (במדבר לו, ב)׃\quad \rashiDH{מנחת בכורים.} במנחת העומר הכתוב מדבר, שהיא באה אביב בשעת בישול התבואה, ומן השעורים היא באה, נאמר כאן אביב, ונאמר להלן כי השעורה אביב (מנחות סח׃)׃\quad \rashiDH{קלוי באש.} שמיבשין אותו על האור באבוב של קַלָּאִים, (פירש״י במנחות שם הכלי של מוכרי קליות) שאלולי כן אינה נטחנת בריחים לפי שהיא לחה׃\quad \rashiDH{גרש כרמל.} גרוסה בעודה לחה׃\quad \rashiDH{גרש.} לשון שבירה וטחינה גורסה בריחים של גרוסות, כמו וַיַּגְרֵס בֶּחָצָץ (איכה ג, טז), וכן גָּרְסָה נַפְשִׁי (תהלים קיט, כ)׃\quad \rashiDH{כרמל.} בעוד הכר מלא (מנחות סו׃), שהתבואה לחה ומלאה בקשין שלה, ועל כן נקראים המלילות כרמל, וכן כַּרְמֶל בְּצִקְלוֹנוֹ (מלכים־ב ד, מב)׃}
\threeverse{\arabic{verse}}%Leviticus2:15
{וְנָתַתָּ֤ עָלֶ֙יהָ֙ שֶׁ֔מֶן וְשַׂמְתָּ֥ עָלֶ֖יהָ לְבֹנָ֑ה מִנְחָ֖ה הִֽוא׃}
{וְתִתֵּין עֲלַהּ מִשְׁחָא וּתְשַׁוֵּי עֲלַהּ לְבוֹנְתָא מִנְחָתָא הִיא׃}
{And thou shalt put oil upon it, and lay frankincense thereon; it is a meal-offering.}{\arabic{verse}}

\threeverse{\arabic{verse}}%Leviticus2:16
{וְהִקְטִ֨יר הַכֹּהֵ֜ן אֶת\maqqaf אַזְכָּרָתָ֗הּ מִגִּרְשָׂהּ֙ וּמִשַּׁמְנָ֔הּ עַ֖ל כׇּל\maqqaf לְבֹנָתָ֑הּ אִשֶּׁ֖ה לַיהֹוָֽה׃ \petucha }
{וְיַסֵּיק כָּהֲנָא יָת אַדְכָרְתַהּ מִגִּרְסַהּ וּמִמִּשְׁחַהּ עַל כָּל לְבוֹנְתַהּ קוּרְבָּנָא קֳדָם יְיָ׃}
{And the priest shall make the memorial-part of it smoke, even of the groats thereof, and of the oil thereof, with all the frankincense thereof; it is an offering made by fire unto the \lord.}{\arabic{verse}}

\newperek
\aliyacounter{רביעי}
\threeverse{\aliya{רביעי}}%Leviticus3:1
{וְאִם\maqqaf זֶ֥בַח שְׁלָמִ֖ים קׇרְבָּנ֑וֹ אִ֤ם מִן\maqqaf הַבָּקָר֙ ה֣וּא מַקְרִ֔יב אִם\maqqaf זָכָר֙ אִם\maqqaf נְקֵבָ֔ה תָּמִ֥ים יַקְרִיבֶ֖נּוּ לִפְנֵ֥י יְהֹוָֽה׃}
{וְאִם נִכְסַת קוּדְשַׁיָּא קוּרְבָּנֵיהּ אִם מִן תּוֹרֵי הוּא מְקָרֵיב אִם דְּכַר אִם נוּקְבָּא שְׁלִים יְקָרְבִנֵּיהּ קֳדָם יְיָ׃}
{And if his offering be a sacrifice of peace-offerings: if he offer of the herd, whether male or female, he shall offer it without blemish before the \lord.}{\Roman{chap}}
\rashi{\rashiDH{שלמים.} שמטילים שלום בעולם. דבר אחר שלמים שיש בהם שלום למזבח ולכהנים ולבעלים׃}
\threeverse{\arabic{verse}}%Leviticus3:2
{וְסָמַ֤ךְ יָדוֹ֙ עַל\maqqaf רֹ֣אשׁ קׇרְבָּנ֔וֹ וּשְׁחָט֕וֹ פֶּ֖תַח אֹ֣הֶל מוֹעֵ֑ד וְזָרְק֡וּ בְּנֵי֩ אַהֲרֹ֨ן הַכֹּהֲנִ֧ים אֶת\maqqaf הַדָּ֛ם עַל\maqqaf הַמִּזְבֵּ֖חַ סָבִֽיב׃}
{וְיִסְמוֹךְ יְדֵיהּ עַל רֵישׁ קוּרְבָּנֵיהּ וְיִכְּסִנֵּיהּ בִּתְרַע מַשְׁכַּן זִמְנָא וְיִזְרְקוּן בְּנֵי אַהֲרֹן כָּהֲנַיָּא יָת דְּמָא עַל מַדְבְּחָא סְחוֹר סְחוֹר׃}
{And he shall lay his hand upon the head of his offering, and kill it at the door of the tent of meeting; and Aaron’s sons the priests shall dash the blood against the altar round about.}{\arabic{verse}}
\threeverse{\arabic{verse}}%Leviticus3:3
{וְהִקְרִיב֙ מִזֶּ֣בַח הַשְּׁלָמִ֔ים אִשֶּׁ֖ה לַיהֹוָ֑ה אֶת\maqqaf הַחֵ֙לֶב֙ הַֽמְכַסֶּ֣ה אֶת\maqqaf הַקֶּ֔רֶב וְאֵת֙ כׇּל\maqqaf הַחֵ֔לֶב אֲשֶׁ֖ר עַל\maqqaf הַקֶּֽרֶב׃}
{וִיקָרֵיב מִנִּכְסַת קוּדְשַׁיָּא קוּרְבָּנָא קֳדָם יְיָ יָת תַּרְבָּא דְּחָפֵי יָת גַּוָּא וְיָת כָּל תַּרְבָּא דְּעַל גַּוָּא׃}
{And he shall present of the sacrifice of peace-offerings an offering made by fire unto the \lord: the fat that covereth the inwards, and all the fat that is upon the inwards,}{\arabic{verse}}
\rashi{\rashiDH{ואת כל החלב וגו׳.} להביא חֵלֶב שעל ַקֵּבָה דברי רבי ישמעאל, רבי עקיבא אומר, להביא חֵלֶב שעל הדקין׃ 
}
\threeverse{\arabic{verse}}%Leviticus3:4
{וְאֵת֙ שְׁתֵּ֣י הַכְּלָיֹ֔ת וְאֶת\maqqaf הַחֵ֙לֶב֙ אֲשֶׁ֣ר עֲלֵהֶ֔ן אֲשֶׁ֖ר עַל\maqqaf הַכְּסָלִ֑ים וְאֶת\maqqaf הַיֹּתֶ֙רֶת֙ עַל\maqqaf הַכָּבֵ֔ד עַל\maqqaf הַכְּלָי֖וֹת יְסִירֶֽנָּה׃}
{וְיָת תַּרְתֵּין כּוֹלְיָן וְיָת תַּרְבָּא דַּעֲלֵיהוֹן דְּעַל גִּסְסַיָּא וְיָת חַצְרָא דְּעַל כַּבְדָּא עַל כּוֹלְיָתָא יַעְדֵּינַהּ׃}
{and the two kidneys, and the fat that is on them, which is by the loins, and the lobe above the liver, which he shall take away hard by the kidneys.}{\arabic{verse}}
\rashi{\rashiDH{הכסלים.} (פלנקי״ן בלע״ז) שהחלב שעל הכליות כשהבהמה חיה הוא בגובה הכסלים, והם מלמטה, וזהו החלב שתחת המתנים שקורין בלע״ז לונבילו״ש, לובן הנראה למעלה בגובה הכסלים, ובתחתיתו הבשר חופהו׃\quad \rashiDH{היותרת.} היא דופן המסך שקורין איברי״ש, ובלשון ארמי חַצְרָא דְכַבְדָא׃\quad \rashiDH{על הכבד.} שיטול מן הכבד עמה מעט, ובמקום אחר הוא אומר וְאֶת הַיֹּתֶרֶת מִן הַכָּבֵד׃\quad \rashiDH{על הכבד על הכליות.} לבד מן הכבד, ולבד מן הכליות יסירנה לזו׃}
\threeverse{\arabic{verse}}%Leviticus3:5
{וְהִקְטִ֨ירוּ אֹת֤וֹ בְנֵֽי\maqqaf אַהֲרֹן֙ הַמִּזְבֵּ֔חָה עַל\maqqaf הָ֣עֹלָ֔ה אֲשֶׁ֥ר עַל\maqqaf הָעֵצִ֖ים אֲשֶׁ֣ר עַל\maqqaf הָאֵ֑שׁ אִשֵּׁ֛ה רֵ֥יחַ נִיחֹ֖חַ לַֽיהֹוָֽה׃ \petucha }
{וְיַסְּקוּן יָתֵיהּ בְּנֵי אַהֲרֹן לְמַדְבְּחָא עַל עֲלָתָא דְּעַל אָעַיָּא דְּעַל אִישָׁתָא קוּרְבַּן דְּמִתְקַבַּל בְּרַעֲוָא קֳדָם יְיָ׃}
{And Aaron’s sons shall make it smoke on the altar upon the burnt-offering, which is upon the wood that is on the fire; it is an offering made by fire, of a sweet savour unto the \lord.}{\arabic{verse}}
\rashi{\rashiDH{על העולה.} מלבד העולה, למדנו שתקדים עולת תמידלכל קרבן, על המערכה׃}
\threeverse{\arabic{verse}}%Leviticus3:6
{וְאִם\maqqaf מִן\maqqaf הַצֹּ֧אן קׇרְבָּנ֛וֹ לְזֶ֥בַח שְׁלָמִ֖ים לַיהֹוָ֑ה זָכָר֙ א֣וֹ נְקֵבָ֔ה תָּמִ֖ים יַקְרִיבֶֽנּוּ׃}
{וְאִם מִן עָנָא קוּרְבָּנֵיהּ לְנִכְסַת קוּדְשַׁיָּא קֳדָם יְיָ דְּכַר אוֹ נוּקְבָּא שְׁלִים יְקָרְבִנֵּיהּ׃}
{And if his offering for a sacrifice of peace-offerings unto the \lord\space be of the flock, male or female, he shall offer it without blemish.}{\arabic{verse}}
\threeverse{\arabic{verse}}%Leviticus3:7
{אִם\maqqaf כֶּ֥שֶׂב הֽוּא\maqqaf מַקְרִ֖יב אֶת\maqqaf קׇרְבָּנ֑וֹ וְהִקְרִ֥יב אֹת֖וֹ לִפְנֵ֥י יְהֹוָֽה׃}
{אִם אִמַּר הוּא מְקָרֵיב יָת קוּרְבָּנֵיהּ וִיקָרֵיב יָתֵיהּ קֳדָם יְיָ׃}
{If he bring a lamb for his offering, then shall he present it before the \lord.}{\arabic{verse}}
\rashi{\rashiDH{אם כשב.} לפי שיש באימורי הכשב מה שאין באימורי העז, שהכשב אליתו קריבה, לכך נחלקו שתי פרשיות׃}
\threeverse{\arabic{verse}}%Leviticus3:8
{וְסָמַ֤ךְ אֶת\maqqaf יָדוֹ֙ עַל\maqqaf רֹ֣אשׁ קׇרְבָּנ֔וֹ וְשָׁחַ֣ט אֹת֔וֹ לִפְנֵ֖י אֹ֣הֶל מוֹעֵ֑ד וְ֠זָרְק֠וּ בְּנֵ֨י אַהֲרֹ֧ן אֶת\maqqaf דָּמ֛וֹ עַל\maqqaf הַמִּזְבֵּ֖חַ סָבִֽיב׃}
{וְיִסְמוֹךְ יָת יְדֵיהּ עַל רֵישׁ קוּרְבָּנֵיהּ וְיִכּוֹס יָתֵיהּ קֳדָם מַשְׁכַּן זִמְנָא וְיִזְרְקוּן בְּנֵי אַהֲרֹן יָת דְּמֵיהּ עַל מַדְבְּחָא סְחוֹר סְחוֹר׃}
{And he shall lay his hand upon the head of his offering, and kill it before the tent of meeting; and Aaron’s sons shall dash the blood thereof against the altar round about.}{\arabic{verse}}
\rashi{\rashiDH{וזרקו.} שתי מתנות שהן ד׳, ועל ידי הכלי הוא זורק, ואינו נותן באצבע אלא בחטאת׃}
\threeverse{\arabic{verse}}%Leviticus3:9
{וְהִקְרִ֨יב מִזֶּ֣בַח הַשְּׁלָמִים֮ אִשֶּׁ֣ה לַיהֹוָה֒ חֶלְבּוֹ֙ הָאַלְיָ֣ה תְמִימָ֔ה לְעֻמַּ֥ת הֶעָצֶ֖ה יְסִירֶ֑נָּה וְאֶת\maqqaf הַחֵ֙לֶב֙ הַֽמְכַסֶּ֣ה אֶת\maqqaf הַקֶּ֔רֶב וְאֵת֙ כׇּל\maqqaf הַחֵ֔לֶב אֲשֶׁ֖ר עַל\maqqaf הַקֶּֽרֶב׃}
{וִיקָרֵיב מִנִּכְסַת קוּדְשַׁיָּא קוּרְבָּנָא קֳדָם יְיָ תַּרְבֵּיהּ אַלְיְתָא שַׁלְמְתָא לָקֳבֵיל שַׁזַרְתָּא יַעְדֵּינַהּ וְיָת תַּרְבָּא דְּחָפֵי יָת גַּוָּא וְיָת כָּל תַּרְבָּא דְּעַל גַּוָּא׃}
{And he shall present of the sacrifice of peace-offerings an offering made by fire unto the \lord: the fat thereof, the fat tail entire, which he shall take away hard by the rump-bone; and the fat that covereth the inwards, and all the fat that is upon the inwards,}{\arabic{verse}}
\rashi{\rashiDH{חלבו.} המובחר שבו, ומהו, זה האליה תמימה׃\quad \rashiDH{לעמת העצה.} למעלה מן הכליות היועצות (חולין יא.)׃}
\threeverse{\arabic{verse}}%Leviticus3:10
{וְאֵת֙ שְׁתֵּ֣י הַכְּלָיֹ֔ת וְאֶת\maqqaf הַחֵ֙לֶב֙ אֲשֶׁ֣ר עֲלֵהֶ֔ן אֲשֶׁ֖ר עַל\maqqaf הַכְּסָלִ֑ים וְאֶת\maqqaf הַיֹּתֶ֙רֶת֙ עַל\maqqaf הַכָּבֵ֔ד עַל\maqqaf הַכְּלָיֹ֖ת יְסִירֶֽנָּה׃}
{וְיָת תַּרְתֵּין כּוֹלְיָן וְיָת תַּרְבָּא דַּעֲלֵיהוֹן דְּעַל גִּסְסַיָּא וְיָת חַצְרָא דְּעַל כַּבְדָּא עַל כּוֹלְיָתָא יַעְדֵּינַהּ׃}
{and the two kidneys, and the fat that is upon them, which is by the loins, and the lobe above the liver, which he shall take away by the kidneys.}{\arabic{verse}}
\threeverse{\arabic{verse}}%Leviticus3:11
{וְהִקְטִיר֥וֹ הַכֹּהֵ֖ן הַמִּזְבֵּ֑חָה לֶ֥חֶם אִשֶּׁ֖ה לַיהֹוָֽה׃ \petucha }
{וְיַסְּקִנֵּיהּ כָּהֲנָא לְמַדְבְּחָא לְחֵים קוּרְבָּנָא קֳדָם יְיָ׃}
{And the priest shall make it smoke upon the altar; it is the food of the offering made by fire unto the \lord.}{\arabic{verse}}
\rashi{\rashiDH{לחם אשה לה׳.} לחמו של אש לשם גבוה׃\quad \rashiDH{לחם.} לשון מאכל, וכן נַשְׁחִיתָה עֵץ בְּלַחְמוֹ (ירמיה יא, יט), עֲבַד לְחֶם רַב (דניאל ה, א), לִשְׂחוֹק עֹשִׂים לֶחֶם (קהלת י, יט)׃}
\threeverse{\arabic{verse}}%Leviticus3:12
{וְאִ֥ם עֵ֖ז קׇרְבָּנ֑וֹ וְהִקְרִיב֖וֹ לִפְנֵ֥י יְהֹוָֽה׃}
{וְאִם מִן בְּנֵי עִזַּיָּא קוּרְבָּנֵיהּ וִיקָרְבִנֵּיהּ קֳדָם יְיָ׃}
{And if his offering be a goat, then he shall present it before the \lord.}{\arabic{verse}}
\threeverse{\arabic{verse}}%Leviticus3:13
{וְסָמַ֤ךְ אֶת\maqqaf יָדוֹ֙ עַל\maqqaf רֹאשׁ֔וֹ וְשָׁחַ֣ט אֹת֔וֹ לִפְנֵ֖י אֹ֣הֶל מוֹעֵ֑ד וְ֠זָרְק֠וּ בְּנֵ֨י אַהֲרֹ֧ן אֶת\maqqaf דָּמ֛וֹ עַל\maqqaf הַמִּזְבֵּ֖חַ סָבִֽיב׃}
{וְיִסְמוֹךְ יָת יְדֵיהּ עַל רֵישֵׁיהּ וְיִכּוֹס יָתֵיהּ קֳדָם מַשְׁכַּן זִמְנָא וְיִזְרְקוּן בְּנֵי אַהֲרֹן יָת דְּמֵיהּ עַל מַדְבְּחָא סְחוֹר סְחוֹר׃}
{And he shall lay his hand upon the head of it, and kill it before the tent of meeting; and the sons of Aaron shall dash the blood thereof against the altar round about.}{\arabic{verse}}
\threeverse{\arabic{verse}}%Leviticus3:14
{וְהִקְרִ֤יב מִמֶּ֙נּוּ֙ קׇרְבָּנ֔וֹ אִשֶּׁ֖ה לַֽיהֹוָ֑ה אֶת\maqqaf הַחֵ֙לֶב֙ הַֽמְכַסֶּ֣ה אֶת\maqqaf הַקֶּ֔רֶב וְאֵת֙ כׇּל\maqqaf הַחֵ֔לֶב אֲשֶׁ֖ר עַל\maqqaf הַקֶּֽרֶב׃}
{וִיקָרֵיב מִנֵּיהּ קוּרְבָּנֵיהּ קוּרְבָּנָא קֳדָם יְיָ יָת תַּרְבָּא דְּחָפֵי יָת גַּוָּא וְיָת כָּל תַּרְבָּא דְּעַל גַּוָּא׃}
{And he shall present thereof his offering, even an offering made by fire unto the \lord: the fat that covereth the inwards, and all the fat that is upon the inwards,}{\arabic{verse}}
\threeverse{\arabic{verse}}%Leviticus3:15
{וְאֵת֙ שְׁתֵּ֣י הַכְּלָיֹ֔ת וְאֶת\maqqaf הַחֵ֙לֶב֙ אֲשֶׁ֣ר עֲלֵהֶ֔ן אֲשֶׁ֖ר עַל\maqqaf הַכְּסָלִ֑ים וְאֶת\maqqaf הַיֹּתֶ֙רֶת֙ עַל\maqqaf הַכָּבֵ֔ד עַל\maqqaf הַכְּלָיֹ֖ת יְסִירֶֽנָּה׃}
{וְיָת תַּרְתֵּין כּוֹלְיָן וְיָת תַּרְבָּא דַּעֲלֵיהוֹן דְּעַל גִּסְסַיָּא וְיָת חַצְרָא דְּעַל כַּבְדָּא עַל כּוֹלְיָתָא יַעְדֵּינַהּ׃}
{and the two kidneys, and the fat that is upon them, which is by the loins, and the lobe above the liver, which he shall take away by the kidneys.}{\arabic{verse}}
\threeverse{\arabic{verse}}%Leviticus3:16
{וְהִקְטִירָ֥ם הַכֹּהֵ֖ן הַמִּזְבֵּ֑חָה לֶ֤חֶם אִשֶּׁה֙ לְרֵ֣יחַ נִיחֹ֔חַ כׇּל\maqqaf חֵ֖לֶב לַיהֹוָֽה׃}
{וְיַסֵּיקִנּוּן כָּהֲנָא לְמַדְבְּחָא לְחֵים קוּרְבָּנָא לְאִתְקַבָּלָא בְּרַעֲוָא כָּל תַּרְבָּא קֳדָם יְיָ׃}
{And the priest shall make them smoke upon the altar; it is the food of the offering made by fire, for a sweet savour; all the fat is the \lord’S.}{\arabic{verse}}
\threeverse{\arabic{verse}}%Leviticus3:17
{חֻקַּ֤ת עוֹלָם֙ לְדֹרֹ֣תֵיכֶ֔ם בְּכֹ֖ל מוֹשְׁבֹֽתֵיכֶ֑ם כׇּל\maqqaf חֵ֥לֶב וְכׇל\maqqaf דָּ֖ם לֹ֥א תֹאכֵֽלוּ׃ \petucha }
{קְיָם עָלַם לְדָרֵיכוֹן בְּכֹל מוֹתְבָנֵיכוֹן כָּל תַּרְבָּא וְכָל דְּמָא לָא תֵיכְלוּן׃}
{It shall be a perpetual statute throughout your generations in all your dwellings, that ye shall eat neither fat nor blood.}{\arabic{verse}}
\rashi{\rashiDH{חקת עולם.} יפה מפורש בתורת כהנים (פרק כ, ו) כל הפסוק הזה׃ 
}
\newperek
\aliyacounter{חמישי}
\newseder{2}
\threeverse{\aliya{חמישי}\newline\vspace{-4pt}\newline\seder{ב}}%Leviticus4:1
{וַיְדַבֵּ֥ר יְהֹוָ֖ה אֶל\maqqaf מֹשֶׁ֥ה לֵּאמֹֽר׃}
{וּמַלֵּיל יְיָ עִם מֹשֶׁה לְמֵימַר׃}
{And the \lord\space spoke unto Moses, saying:}{\Roman{chap}}
\threeverse{\arabic{verse}}%Leviticus4:2
{דַּבֵּ֞ר אֶל\maqqaf בְּנֵ֣י יִשְׂרָאֵל֮ לֵאמֹר֒ נֶ֗פֶשׁ כִּֽי\maqqaf תֶחֱטָ֤א בִשְׁגָגָה֙ מִכֹּל֙ מִצְוֺ֣ת יְהֹוָ֔ה אֲשֶׁ֖ר לֹ֣א תֵעָשֶׂ֑ינָה וְעָשָׂ֕ה מֵאַחַ֖ת מֵהֵֽנָּה׃}
{מַלֵּיל עִם בְּנֵי יִשְׂרָאֵל לְמֵימַר אֲנָשׁ אֲרֵי יְחוּב בְּשָׁלוּ מִכֹּל פִּקּוֹדַיָּא דַּייָ דְּלָא כָשְׁרִין לְאִתְעֲבָדָא וְיַעֲבֵיד מִן חַד מִנְּהוֹן׃}
{Speak unto the children of Israel, saying: If any one shall sin through error, in any of the things which the \lord\space hath commanded not to be done, and shall do any one of them:}{\arabic{verse}}
\rashi{\rashiDH{מכל מצות ה׳.} פירשו רבותינו אין חטאת באה אלא על דבר שזדונו לאו וכרת ושגגתו חטאת (שבת סט.)׃\quad \rashiDH{מאחת מהנה.} ממקצת אחת מהן, כגון הכותב בשבת, שם משמעון, נח מנחור, דן מדניאל׃ 
}
\threeverse{\arabic{verse}}%Leviticus4:3
{אִ֣ם הַכֹּהֵ֧ן הַמָּשִׁ֛יחַ יֶחֱטָ֖א לְאַשְׁמַ֣ת הָעָ֑ם וְהִקְרִ֡יב עַ֣ל חַטָּאתוֹ֩ אֲשֶׁ֨ר חָטָ֜א פַּ֣ר בֶּן\maqqaf בָּקָ֥ר תָּמִ֛ים לַיהֹוָ֖ה לְחַטָּֽאת׃}
{אִם כָּהֲנָא רַבָּא יְחוּב לְחוֹבַת עַמָּא וִיקָרֵיב עַל חוֹבְתֵיהּ דְּחָב תּוֹר בַּר תּוֹרֵי שְׁלִים קֳדָם יְיָ לְחַטָּתָא׃}
{if the anointed priest shall sin so as to bring guilt on the people, then let him offer for his sin, which he hath sinned, a young bullock without blemish unto the \lord\space for a sin-offering.}{\arabic{verse}}
\rashi{\rashiDH{אם הכהן המשיח יחטא לאשמת העם.} מדרשו אינו חייב אלא בהעלם דבר עם שגגת מעשה, כמו שנאמר לאשמת העם. ונעלם דבר מעיני הקהל ועשו. ופשוטו לפי אגדה כשהכהן גדול חוטא אשמת העם הוא זה, שהן תלויין בו לכפר עליהם ולהתפלל בעדם, ונעשה מקולקל׃\quad \rashiDH{פר.} יכול זקן תלמוד לומר בן, אי בן יכול טן, תלמוד לומר פר, הא כיצד, זה פר בן ג׳׃}
\threeverse{\arabic{verse}}%Leviticus4:4
{וְהֵבִ֣יא אֶת\maqqaf הַפָּ֗ר אֶל\maqqaf פֶּ֛תַח אֹ֥הֶל מוֹעֵ֖ד לִפְנֵ֣י יְהֹוָ֑ה וְסָמַ֤ךְ אֶת\maqqaf יָדוֹ֙ עַל\maqqaf רֹ֣אשׁ הַפָּ֔ר וְשָׁחַ֥ט אֶת\maqqaf הַפָּ֖ר לִפְנֵ֥י יְהֹוָֽה׃}
{וְיַיְתֵי יָת תּוֹרָא לִתְרַע מַשְׁכַּן זִמְנָא לִקְדָם יְיָ וְיִסְמוֹךְ יָת יְדֵיהּ עַל רֵישׁ תּוֹרָא וְיִכּוֹס יָת תּוֹרָא קֳדָם יְיָ׃}
{And he shall bring the bullock unto the door of the tent of meeting before the \lord; and he shall lay his hand upon the head of the bullock, and kill the bullock before the \lord.}{\arabic{verse}}
\threeverse{\arabic{verse}}%Leviticus4:5
{וְלָקַ֛ח הַכֹּהֵ֥ן הַמָּשִׁ֖יחַ מִדַּ֣ם הַפָּ֑ר וְהֵבִ֥יא אֹת֖וֹ אֶל\maqqaf אֹ֥הֶל מוֹעֵֽד׃}
{וְיִסַּב כָּהֲנָא רַבָּא מִדְּמָא דְּתוֹרָא וְיַעֵיל יָתֵיהּ לְמַשְׁכַּן זִמְנָא׃}
{And the anointed priest shall take of the blood of the bullock, and bring it to the tent of meeting.}{\arabic{verse}}
\rashi{\rashiDH{אל אהל מועד.} למשכן. ובבית עולמים להיכל׃}
\threeverse{\arabic{verse}}%Leviticus4:6
{וְטָבַ֧ל הַכֹּהֵ֛ן אֶת\maqqaf אֶצְבָּע֖וֹ בַּדָּ֑ם וְהִזָּ֨ה מִן\maqqaf הַדָּ֜ם שֶׁ֤בַע פְּעָמִים֙ לִפְנֵ֣י יְהֹוָ֔ה אֶת\maqqaf פְּנֵ֖י פָּרֹ֥כֶת הַקֹּֽדֶשׁ׃}
{וְיִטְבּוֹל כָּהֲנָא יָת אֶצְבְּעֵיהּ בִּדְמָא וְיַדֵּי מִן דְּמָא שְׁבַע זִמְנִין קֳדָם יְיָ קֳדָם פָרוּכְתָּא דְּקוּדְשָׁא׃}
{And the priest shall dip his finger in the blood, and sprinkle of the blood seven times before the \lord, in front of the veil of the sanctuary.}{\arabic{verse}}
\rashi{\rashiDH{את פני פרכת הקדש.} כנגד מקום קדושתה, כוון כנגד בין הבדים, ולא היו נוגעים דמים בפרכת, ואם נגעו נגעו (יומא נז.)׃ 
}
\threeverse{\arabic{verse}}%Leviticus4:7
{וְנָתַן֩ הַכֹּהֵ֨ן מִן\maqqaf הַדָּ֜ם עַל\maqqaf קַ֠רְנ֠וֹת מִזְבַּ֨ח קְטֹ֤רֶת הַסַּמִּים֙ לִפְנֵ֣י יְהֹוָ֔ה אֲשֶׁ֖ר בְּאֹ֣הֶל מוֹעֵ֑ד וְאֵ֣ת \legarmeh  כׇּל\maqqaf דַּ֣ם הַפָּ֗ר יִשְׁפֹּךְ֙ אֶל\maqqaf יְסוֹד֙ מִזְבַּ֣ח הָעֹלָ֔ה אֲשֶׁר\maqqaf פֶּ֖תַח אֹ֥הֶל מוֹעֵֽד׃}
{וְיִתֵּין כָּהֲנָא מִן דְּמָא עַל קַרְנָת מַדְבַּח קְטוֹרַת בֻּסְמַיָּא קֳדָם יְיָ דִּבְמַשְׁכַּן זִמְנָא וְיָת כָּל דְּמָא דְּתוֹרָא יִשְׁפּוֹךְ לִיסוֹדָא דְּמַדְבְּחָא דַּעֲלָתָא דְּבִתְרַע מַשְׁכַּן זִמְנָא׃}
{And the priest shall put of the blood upon the horns of the altar of sweet incense before the \lord, which is in the tent of meeting; and all the remaining blood of the bullock shall he pour out at the base of the altar of burnt-offering, which is at the door of the tent of meeting.}{\arabic{verse}}
\rashi{\rashiDH{ואת כל דם.} שירי הדם׃}
\threeverse{\arabic{verse}}%Leviticus4:8
{וְאֶת\maqqaf כׇּל\maqqaf חֵ֛לֶב פַּ֥ר הַֽחַטָּ֖את יָרִ֣ים מִמֶּ֑נּוּ אֶת\maqqaf הַחֵ֙לֶב֙ הַֽמְכַסֶּ֣ה עַל\maqqaf הַקֶּ֔רֶב וְאֵת֙ כׇּל\maqqaf הַחֵ֔לֶב אֲשֶׁ֖ר עַל\maqqaf הַקֶּֽרֶב׃}
{וְיָת כָּל תְּרַב תּוֹרָא דְּחַטָּתָא יַפְרֵישׁ מִנֵּיהּ יָת תַּרְבָּא דְּחָפֵי עַל גַּוָּא וְיָת כָּל תַּרְבָּא דְּעַל גַּוָּא׃}
{And all the fat of the bullock of the sin-offering he shall take off from it; the fat that covereth the inwards, and all the fat that is upon the inwards,}{\arabic{verse}}
\rashi{\rashiDH{ואת כל חלב פר.} חלבו היה לו לומר, מה תלמוד לומר פר, לרבות פר של יום הכפורים לכליות ולחלבים ויותרת׃\quad \rashiDH{החטאת.} להביא שעירי עבודת אלילים לכליות ולחלבים ויותרת׃\quad \rashiDH{ירים ממנו.} מן המחובר, שלא ינתחנו קודם הסרת חלבו (ת״כ פרק ד, א)׃}
\threeverse{\arabic{verse}}%Leviticus4:9
{וְאֵת֙ שְׁתֵּ֣י הַכְּלָיֹ֔ת וְאֶת\maqqaf הַחֵ֙לֶב֙ אֲשֶׁ֣ר עֲלֵיהֶ֔ן אֲשֶׁ֖ר עַל\maqqaf הַכְּסָלִ֑ים וְאֶת\maqqaf הַיֹּתֶ֙רֶת֙ עַל\maqqaf הַכָּבֵ֔ד עַל\maqqaf הַכְּלָי֖וֹת יְסִירֶֽנָּה׃}
{וְיָת תַּרְתֵּין כּוֹלְיָן וְיָת תַּרְבָּא דַּעֲלֵיהוֹן דְּעַל גִּסְסַיָּא וְיָת חַצְרָא דְּעַל כַּבְדָּא עַל כּוֹלְיָתָא יַעְדֵּינַהּ׃}
{and the two kidneys, and the fat that is upon them, which is by the loins, and the lobe above the liver, which he shall take away by kidneys,}{\arabic{verse}}
\threeverse{\arabic{verse}}%Leviticus4:10
{כַּאֲשֶׁ֣ר יוּרַ֔ם מִשּׁ֖וֹר זֶ֣בַח הַשְּׁלָמִ֑ים וְהִקְטִירָם֙ הַכֹּהֵ֔ן עַ֖ל מִזְבַּ֥ח הָעֹלָֽה׃}
{כְּמָא דְּמִתַּפְרַשׁ מִתּוֹר נִכְסַת קוּדְשַׁיָּא וְיַסֵּיקִנּוּן כָּהֲנָא עַל מַדְבְּחָא דַּעֲלָתָא׃}
{as it is taken off from the ox of the sacrifice of peace-offerings; and the priest shall make them smoke upon the altar of burnt-offering.}{\arabic{verse}}
\rashi{\rashiDH{כאשר יורם.} מאותן אמורין המפורשין בשור זבח השלמים, וכי מה פירש בזבח השלמים שלא פירש כאן, אלא להקישו לשלמים, מה שלמים לשמן, אף זה לשמו, ומה שלמים שלום לעולם, אף זה שלום לעולם. ובשחיטת קדשים (זבחים מט׃) מצריכו ללמוד הימנו, שאין למדין למד מן הלמד בקדשים, בפרק איזהו מקומן׃\quad \rashiDH{על הכבד על הכליות על ראשו ועל כרעיו.} כולן לשון תוספת הן, כמו מלבד׃}
\threeverse{\arabic{verse}}%Leviticus4:11
{וְאֶת\maqqaf ע֤וֹר הַפָּר֙ וְאֶת\maqqaf כׇּל\maqqaf בְּשָׂר֔וֹ עַל\maqqaf רֹאשׁ֖וֹ וְעַל\maqqaf כְּרָעָ֑יו וְקִרְבּ֖וֹ וּפִרְשֽׁוֹ׃}
{וְיָת מְשַׁךְ תּוֹרָא וְיָת כָּל בִּסְרֵיהּ עַל רֵישֵׁיהּ וְעַל כְּרָעוֹהִי וְגַוֵּיהּ וְאוּכְלֵיהּ׃}
{But the skin of the bullock, and all its flesh, with its head, and with its legs, and its inwards, and its dung,}{\arabic{verse}}
\threeverse{\arabic{verse}}%Leviticus4:12
{וְהוֹצִ֣יא אֶת\maqqaf כׇּל\maqqaf הַ֠פָּ֠ר אֶל\maqqaf מִח֨וּץ לַֽמַּחֲנֶ֜ה אֶל\maqqaf מָק֤וֹם טָהוֹר֙ אֶל\maqqaf שֶׁ֣פֶךְ הַדֶּ֔שֶׁן וְשָׂרַ֥ף אֹת֛וֹ עַל\maqqaf עֵצִ֖ים בָּאֵ֑שׁ עַל\maqqaf שֶׁ֥פֶךְ הַדֶּ֖שֶׁן יִשָּׂרֵֽף׃ \petucha }
{וְיַפֵּיק יָת כָּל תּוֹרָא לְמִבַּרָא לְמַשְׁרִיתָא לַאֲתַר דְּכֵי לַאֲתַר בֵּית מֵישַׁד קִטְמָא וְיוֹקֵיד יָתֵיהּ עַל אָעַיָּא בְּאִישָׁתָא עַל אֲתַר בֵּית מֵישַׁד קִטְמָא יִתּוֹקַד׃}
{even the whole bullock shall he carry forth without the camp unto a clean place, where the ashes are poured out, and burn it on wood with fire; where the ashes are poured out shall it be burnt.}{\arabic{verse}}
\rashi{\rashiDH{אל מקום טהור.} לפי שיש מחוץ לעיר מקום מוכן לטומאה להשליך אבנים מנוגעות ולבית הקברות, הוצרך לומר מחוץ למחנה, זה שהוא חוץ לעיר, שיהא המקום טהור׃\quad \rashiDH{מחוץ למחנה.} חוץ לשלש מחנות, ובבית עולמים חוץ לעיר, מו שפירשוהו רבותינו במס׳ יומא (סח.) ובסנהדרין (מב׃)׃\quad \rashiDH{אל שפך הדשן.} מקום ששופכין בו הדשן המסולק מן המזבח, כמו שנאמר וְהוֹצִיא אֶת הַדֶּשֶׁן אֶל מִחוּץ לַמַּחֲנֶה (להלן ו, ד)׃\quad \rashiDH{על שפך הדשן ישרף.} שאין תלמוד לומר, אלא מלמד שאפילו אין שם דשן׃}
\threeverse{\arabic{verse}}%Leviticus4:13
{וְאִ֨ם כׇּל\maqqaf עֲדַ֤ת יִשְׂרָאֵל֙ יִשְׁגּ֔וּ וְנֶעְלַ֣ם דָּבָ֔ר מֵעֵינֵ֖י הַקָּהָ֑ל וְ֠עָשׂ֠וּ אַחַ֨ת מִכׇּל\maqqaf מִצְוֺ֧ת יְהֹוָ֛ה אֲשֶׁ֥ר לֹא\maqqaf תֵעָשֶׂ֖ינָה וְאָשֵֽׁמוּ׃}
{וְאִם כָּל כְּנִשְׁתָּא דְּיִשְׂרָאֵל יִשְׁתְּלוֹן וִיהֵי מְכוּסַּא פִּתְגָמָא מֵעֵינֵי קְהָלָא וְיַעְבְּדוּן חַד מִכָּל פִּקּוֹדַיָּא דַּייָ דְּלָא כָשְׁרִין לְאִתְעֲבָדָא וִיחוּבוּן׃}
{And if the whole congregation of Israel shall err, the thing being hid from the eyes of the assembly, and do any of the things which the \lord\space hath commanded not to be done, and are guilty:}{\arabic{verse}}
\rashi{\rashiDH{עדת ישראל.} אלו סנהדרין׃\quad \rashiDH{ונעלם דבר.} טעו להורות באחת מכל כריתות שבתורה שהוא מותר (הוריות ז׃)׃\quad \rashiDH{הקהל ועשו.} שעשו צבור על פיהם׃}
\threeverse{\arabic{verse}}%Leviticus4:14
{וְנֽוֹדְעָה֙ הַֽחַטָּ֔את אֲשֶׁ֥ר חָטְא֖וּ עָלֶ֑יהָ וְהִקְרִ֨יבוּ הַקָּהָ֜ל פַּ֤ר בֶּן\maqqaf בָּקָר֙ לְחַטָּ֔את וְהֵבִ֣יאוּ אֹת֔וֹ לִפְנֵ֖י אֹ֥הֶל מוֹעֵֽד׃}
{וְתִתְיְדַע חוֹבְתָא דְּחָבוּ עֲלַהּ וִיקָרְבוּן קְהָלָא תּוֹר בַּר תּוֹרֵי לְחַטָּתָא וְיַיְתוֹן יָתֵיהּ לִקְדָם מַשְׁכַּן זִמְנָא׃}
{when the sin wherein they have sinned is known, then the assembly shall offer a young bullock for a sin-offering, and bring it before the tent of meeting.}{\arabic{verse}}
\threeverse{\arabic{verse}}%Leviticus4:15
{וְ֠סָמְכ֠וּ זִקְנֵ֨י הָעֵדָ֧ה אֶת\maqqaf יְדֵיהֶ֛ם עַל\maqqaf רֹ֥אשׁ הַפָּ֖ר לִפְנֵ֣י יְהֹוָ֑ה וְשָׁחַ֥ט אֶת\maqqaf הַפָּ֖ר לִפְנֵ֥י יְהֹוָֽה׃}
{וְיִסְמְכוּן סָבֵי כְּנִשְׁתָּא יָת יְדֵיהוֹן עַל רֵישׁ תּוֹרָא קֳדָם יְיָ וְיִכּוֹס יָת תּוֹרָא קֳדָם יְיָ׃}
{And the elders of the congregation shall lay their hands upon the head of the bullock before the \lord; and the bullock shall be killed before the \lord.}{\arabic{verse}}
\threeverse{\arabic{verse}}%Leviticus4:16
{וְהֵבִ֛יא הַכֹּהֵ֥ן הַמָּשִׁ֖יחַ מִדַּ֣ם הַפָּ֑ר אֶל\maqqaf אֹ֖הֶל מוֹעֵֽד׃}
{וְיַעֵיל כָּהֲנָא רַבָּא מִדְּמָא דְּתוֹרָא לְמַשְׁכַּן זִמְנָא׃}
{And the anointed priest shall bring of the blood of the bullock to the tent of meeting.}{\arabic{verse}}
\threeverse{\arabic{verse}}%Leviticus4:17
{וְטָבַ֧ל הַכֹּהֵ֛ן אֶצְבָּע֖וֹ מִן\maqqaf הַדָּ֑ם וְהִזָּ֞ה שֶׁ֤בַע פְּעָמִים֙ לִפְנֵ֣י יְהֹוָ֔ה אֵ֖ת פְּנֵ֥י הַפָּרֹֽכֶת׃}
{וְיִטְבּוֹל כָּהֲנָא אֶצְבְּעֵיהּ מִן דְּמָא וְיַדֵּי שְׁבַע זִמְנִין קֳדָם יְיָ קֳדָם פָּרוּכְתָּא׃}
{And the priest shall dip his finger in the blood, and sprinkle it seven times before the \lord, in front of the veil.}{\arabic{verse}}
\rashi{\rashiDH{את פני הפרכת.} ולמעלה הוא אומר את פני פרכת הקדש, משל למלך שסרחה עליו מדינה, אם מעוטה סרחה פמליא שלו מתקיימת, ואם כולם סרחו, אין פמליא שלו מתקיימת, אף כאן כשחטא כהן משיח עדיין שם קדושת המקום על המקדש, משחטאו כולם ח״ו נסתלקה הקדושה (זבחים מא׃)׃}
\threeverse{\arabic{verse}}%Leviticus4:18
{וּמִן\maqqaf הַדָּ֞ם יִתֵּ֣ן \legarmeh  עַל\maqqaf קַרְנֹ֣ת הַמִּזְבֵּ֗חַ אֲשֶׁר֙ לִפְנֵ֣י יְהֹוָ֔ה אֲשֶׁ֖ר בְּאֹ֣הֶל מוֹעֵ֑ד וְאֵ֣ת כׇּל\maqqaf הַדָּ֗ם יִשְׁפֹּךְ֙ אֶל\maqqaf יְסוֹד֙ מִזְבַּ֣ח הָעֹלָ֔ה אֲשֶׁר\maqqaf פֶּ֖תַח אֹ֥הֶל מוֹעֵֽד׃}
{וּמִן דְּמָא יִתֵּין עַל קַרְנָת מַדְבְּחָא דִּקְדָם יְיָ דִּבְמַשְׁכַּן זִמְנָא וְיָת כָּל דְּמָא יִשְׁפּוֹךְ לִיסוֹדָא דְּמַדְבְּחָא דַּעֲלָתָא דְּבִתְרַע מַשְׁכַּן זִמְנָא׃}
{And he shall put of the blood upon the horns of the altar which is before the \lord, that is in the tent of meeting, and all the remaining blood shall he pour out at the base of the altar of burnt-offering, which is at the door of the tent of meeting.}{\arabic{verse}}
\rashi{\rashiDH{יסוד מזבח העולה אשר פתח אהל מועד.} זה יסוד מערבי שהוא כנגד הפתח׃ 
}
\threeverse{\arabic{verse}}%Leviticus4:19
{וְאֵ֥ת כׇּל\maqqaf חֶלְבּ֖וֹ יָרִ֣ים מִמֶּ֑נּוּ וְהִקְטִ֖יר הַמִּזְבֵּֽחָה׃}
{וְיָת כָּל תַּרְבֵּיהּ יַפְרֵישׁ מִנֵּיהּ וְיַסֵּיק לְמַדְבְּחָא׃}
{And all the fat thereof shall he take off from it, and make it smoke upon the altar.}{\arabic{verse}}
\rashi{\rashiDH{ואת כל חלבו ירים.} ואע״פ שלא פירש כאן יותרת ושתי כליות, למדין הם מִוְּעָשָׂה לפר כאשר עשה וגו׳. ומפני מה לא נתפרשו בו, תנא דבי ר׳ ישמעאל, משל למלך שזעם על אוהבו, ומיעט בסרחונו מפני חיבתו (שם)׃}
\threeverse{\arabic{verse}}%Leviticus4:20
{וְעָשָׂ֣ה לַפָּ֔ר כַּאֲשֶׁ֤ר עָשָׂה֙ לְפַ֣ר הַֽחַטָּ֔את כֵּ֖ן יַעֲשֶׂה\maqqaf לּ֑וֹ וְכִפֶּ֧ר עֲלֵהֶ֛ם הַכֹּהֵ֖ן וְנִסְלַ֥ח לָהֶֽם׃}
{וְיַעֲבֵיד לְתוֹרָא כְּמָא דַּעֲבַד לְתוֹרָא דְּחַטָּתָא כֵּן יַעֲבֵיד לֵיהּ וִיכַפַּר עֲלֵיהוֹן כָּהֲנָא וְיִשְׁתְּבֵיק לְהוֹן׃}
{Thus shall he do with the bullock; as he did with the bullock of the sin-offering, so shall he do with this; and the priest shall make atonement for them, and they shall be forgiven.}{\arabic{verse}}
\rashi{\rashiDH{ועשה לפר.} זה, \rashiDH{כאשר עשה לפר החטאת}כמו שמפורש בפר כהן משיח, להביא יותרת ושתי כליות, שפירש שם מה שלא פירש כאן, ולכפול במצות העבודות, ללמד שאם חסר אחת מכל המתנות פסול, לפי שמצינו בניתנין על המזבח החיצון שנתנן במתנה אחת כפר, הוצרך לומר כאן שמתנה אחת מהן מעכבת׃}
\threeverse{\arabic{verse}}%Leviticus4:21
{וְהוֹצִ֣יא אֶת\maqqaf הַפָּ֗ר אֶל\maqqaf מִחוּץ֙ לַֽמַּחֲנֶ֔ה וְשָׂרַ֣ף אֹת֔וֹ כַּאֲשֶׁ֣ר שָׂרַ֔ף אֵ֖ת הַפָּ֣ר הָרִאשׁ֑וֹן חַטַּ֥את הַקָּהָ֖ל הֽוּא׃ \petucha }
{וְיַפֵּיק יָת תּוֹרָא לְמִבַּרָא לְמַשְׁרִיתָא וְיוֹקֵיד יָתֵיהּ כְּמָא דְּאוֹקֵיד יָת תּוֹרָא קַדְמָאָה חַטַּת קְהָלָא הוּא׃}
{And he shall carry forth the bullock without the camp, and burn it as he burned the first bullock; it is the sin-offering for the assembly.}{\arabic{verse}}
\threeverse{\arabic{verse}}%Leviticus4:22
{אֲשֶׁ֥ר נָשִׂ֖יא יֶֽחֱטָ֑א וְעָשָׂ֡ה אַחַ֣ת מִכׇּל\maqqaf מִצְוֺת֩ יְהֹוָ֨ה אֱלֹהָ֜יו אֲשֶׁ֧ר לֹא\maqqaf תֵעָשֶׂ֛ינָה בִּשְׁגָגָ֖ה וְאָשֵֽׁם׃}
{אִם רַבָּא יְחוּב וְיַעֲבֵיד חַד מִכָּל פִּקּוֹדַיָּא דַּייָ אֱלָהֵיהּ דְּלָא כָשְׁרִין לְאִתְעֲבָדָא בְּשָׁלוּ וִיחוּב׃}
{When a ruler sinneth, and doeth through error any one of all the things which the \lord\space his God hath commanded not to be done, and is guilty:}{\arabic{verse}}
\rashi{\rashiDH{אשר נשיא יחטא.} לשון אשרי, אשרי הדור שהנשיא שלו נותן לב להביא כפרה על שגגתו, ק״ו שמתחרט על זדונותיו׃}
\threeverse{\arabic{verse}}%Leviticus4:23
{אֽוֹ\maqqaf הוֹדַ֤ע אֵלָיו֙ חַטָּאת֔וֹ אֲשֶׁ֥ר חָטָ֖א בָּ֑הּ וְהֵבִ֧יא אֶת\maqqaf קׇרְבָּנ֛וֹ שְׂעִ֥יר עִזִּ֖ים זָכָ֥ר תָּמִֽים׃}
{אוֹ אִתְיְדַע לֵיהּ חוֹבְתֵיהּ דְּחָב בַּהּ וְיַיְתֵי יָת קוּרְבָּנֵיהּ צְפִיר בַּר עִזִּין דְּכַר שְׁלִים׃}
{if his sin, wherein he hath sinned, be known to him, he shall bring for his offering a goat, a male without blemish.}{\arabic{verse}}
\rashi{\rashiDH{או הודע.} כמו אם הודע הדבר, הרבה או יש שמשמשין בלשון אם, ואם במקום או, וכן אוֹ נוֹדַע כִּי שׁוֹר נַגָּח הוּא (שמות כא, לו)׃\quad \rashiDH{הודע אליו.} כשחטא היה סבור שהוא היתר, ולאחר מכאן נודע לו שאיסור היה׃}
\threeverse{\arabic{verse}}%Leviticus4:24
{וְסָמַ֤ךְ יָדוֹ֙ עַל\maqqaf רֹ֣אשׁ הַשָּׂעִ֔יר וְשָׁחַ֣ט אֹת֗וֹ בִּמְק֛וֹם אֲשֶׁר\maqqaf יִשְׁחַ֥ט אֶת\maqqaf הָעֹלָ֖ה לִפְנֵ֣י יְהֹוָ֑ה חַטָּ֖את הֽוּא׃}
{וְיִסְמוֹךְ יְדֵיהּ עַל רֵישׁ צְפִירָא וְיִכּוֹס יָתֵיהּ בְּאַתְרָא דְּיִכּוֹס יָת עֲלָתָא קֳדָם יְיָ חַטָּתָא הוּא׃}
{And he shall lay his hand upon the head of the goat, and kill it in the place where they kill the burnt-offering before the \lord; it is a sin-offering.}{\arabic{verse}}
\rashi{\rashiDH{במקום אשר ישחט את העולה.} בצפון שהוא מפורש בעולה׃\quad \rashiDH{חטאת הוא.} לשמו כשר, שלא לשמו פסול׃}
\threeverse{\arabic{verse}}%Leviticus4:25
{וְלָקַ֨ח הַכֹּהֵ֜ן מִדַּ֤ם הַֽחַטָּאת֙ בְּאֶצְבָּע֔וֹ וְנָתַ֕ן עַל\maqqaf קַרְנֹ֖ת מִזְבַּ֣ח הָעֹלָ֑ה וְאֶת\maqqaf דָּמ֣וֹ יִשְׁפֹּ֔ךְ אֶל\maqqaf יְס֖וֹד מִזְבַּ֥ח הָעֹלָֽה׃}
{וְיִסַּב כָּהֲנָא מִדְּמָא דְּחַטָּתָא בְּאֶצְבְּעֵיהּ וְיִתֵּין עַל קַרְנָת מַדְבְּחָא דַּעֲלָתָא וְיָת דְּמֵיהּ יִשְׁפּוֹךְ לִיסוֹדָא דְּמַדְבְּחָא דַּעֲלָתָא׃}
{And the priest shall take of the blood of the sin-offering with his finger, and put it upon the horns of the altar of burnt-offering, and the remaining blood thereof shall he pour out at the base of the altar of burnt-offering.}{\arabic{verse}}
\rashi{\rashiDH{ואת דמו.} שירי הדם׃ 
}
\threeverse{\arabic{verse}}%Leviticus4:26
{וְאֶת\maqqaf כׇּל\maqqaf חֶלְבּוֹ֙ יַקְטִ֣יר הַמִּזְבֵּ֔חָה כְּחֵ֖לֶב זֶ֣בַח הַשְּׁלָמִ֑ים וְכִפֶּ֨ר עָלָ֧יו הַכֹּהֵ֛ן מֵחַטָּאת֖וֹ וְנִסְלַ֥ח לֽוֹ׃ \petucha }
{וְיָת כָּל תַּרְבֵּיהּ יַסֵּיק לְמַדְבְּחָא כִּתְרַב נִכְסַת קוּדְשַׁיָּא וִיכַפַּר עֲלוֹהִי כָּהֲנָא מֵחוֹבְתֵיהּ וְיִשְׁתְּבֵיק לֵיהּ׃}
{And all the fat thereof shall he make smoke upon the altar, as the fat of the sacrifice of peace-offerings; and the priest shall make atonement for him as concerning his sin, and he shall be forgiven.}{\arabic{verse}}
\rashi{\rashiDH{כחלב זבח השלמים.} כאותן אמורין המפורשים בעז, האמור אצל שלמים׃ 
}
\aliyacounter{ששי}
\threeverse{\aliya{ששי}}%Leviticus4:27
{וְאִם\maqqaf נֶ֧פֶשׁ אַחַ֛ת תֶּחֱטָ֥א בִשְׁגָגָ֖ה מֵעַ֣ם הָאָ֑רֶץ בַּ֠עֲשֹׂתָ֠הּ אַחַ֨ת מִמִּצְוֺ֧ת יְהֹוָ֛ה אֲשֶׁ֥ר לֹא\maqqaf תֵעָשֶׂ֖ינָה וְאָשֵֽׁם׃}
{וְאִם אֲנָשׁ חַד יְחוּב בְּשָׁלוּ מֵעַמָּא דְּאַרְעָא בְּמִעְבְּדֵיהּ חַד מִפִּקּוֹדַיָּא דַּייָ דְּלָא כָשְׁרִין לְאִתְעֲבָדָא וִיחוּב׃}
{And if any one of the common people sin through error, in doing any of the things which the \lord\space hath commanded not to be done, and be guilty:}{\arabic{verse}}
\threeverse{\arabic{verse}}%Leviticus4:28
{א֚וֹ הוֹדַ֣ע אֵלָ֔יו חַטָּאת֖וֹ אֲשֶׁ֣ר חָטָ֑א וְהֵבִ֨יא קׇרְבָּנ֜וֹ שְׂעִירַ֤ת עִזִּים֙ תְּמִימָ֣ה נְקֵבָ֔ה עַל\maqqaf חַטָּאת֖וֹ אֲשֶׁ֥ר חָטָֽא׃}
{אוֹ אִתְיְדַע לֵיהּ חוֹבְתֵיהּ דְּחָב וְיַיְתֵי קוּרְבָּנֵיהּ צְפִירַת עִזֵּי שַׁלְמָא נוּקְבָּא עַל חוֹבְתֵיהּ דְּחָב׃}
{if his sin, which he hath sinned, be known to him, then he shall bring for his offering a goat, a female without blemish, for his sin which he hath sinned.}{\arabic{verse}}
\threeverse{\arabic{verse}}%Leviticus4:29
{וְסָמַךְ֙ אֶת\maqqaf יָד֔וֹ עַ֖ל רֹ֣אשׁ הַֽחַטָּ֑את וְשָׁחַט֙ אֶת\maqqaf הַ֣חַטָּ֔את בִּמְק֖וֹם הָעֹלָֽה׃}
{וְיִסְמוֹךְ יָת יְדֵיהּ עַל רֵישׁ חַטָּאתָא וְיִכּוֹס יָת חַטָּאתָא בְּאַתְרָא דַּעֲלָתָא׃}
{And he shall lay his hand upon the head of the sin-offering, and kill the sin-offering in the place of burnt-offering.}{\arabic{verse}}
\threeverse{\arabic{verse}}%Leviticus4:30
{וְלָקַ֨ח הַכֹּהֵ֤ן מִדָּמָהּ֙ בְּאֶצְבָּע֔וֹ וְנָתַ֕ן עַל\maqqaf קַרְנֹ֖ת מִזְבַּ֣ח הָעֹלָ֑ה וְאֶת\maqqaf כׇּל\maqqaf דָּמָ֣הּ יִשְׁפֹּ֔ךְ אֶל\maqqaf יְס֖וֹד הַמִּזְבֵּֽחַ׃}
{וְיִסַּב כָּהֲנָא מִדְּמַהּ בְּאֶצְבְּעֵיהּ וְיִתֵּין עַל קַרְנָת מַדְבְּחָא דַּעֲלָתָא וְיָת כָּל דְּמַהּ יִשְׁפּוֹךְ לִיסוֹדָא דְּמַדְבְּחָא׃}
{And the priest shall take of the blood thereof with his finger, and put it upon the horns of the altar of burnt-offering, and all the remaining blood thereof shall he pour out at the base of the altar.}{\arabic{verse}}
\threeverse{\arabic{verse}}%Leviticus4:31
{וְאֶת\maqqaf כׇּל\maqqaf חֶלְבָּ֣הּ יָסִ֗יר כַּאֲשֶׁ֨ר הוּסַ֣ר חֵ֘לֶב֮ מֵעַ֣ל זֶ֣בַח הַשְּׁלָמִים֒ וְהִקְטִ֤יר הַכֹּהֵן֙ הַמִּזְבֵּ֔חָה לְרֵ֥יחַ נִיחֹ֖חַ לַיהֹוָ֑ה וְכִפֶּ֥ר עָלָ֛יו הַכֹּהֵ֖ן וְנִסְלַ֥ח לֽוֹ׃ \petucha }
{וְיָת כָּל תַּרְבַּהּ יַעְדֵּי כְּמָא דְּאִתַּעְדַּא תְּרַב מֵעַל נִכְסַת קוּדְשַׁיָּא וְיַסֵּיק כָּהֲנָא לְמַדְבְּחָא לְאִתְקַבָּלָא בְּרַעֲוָא קֳדָם יְיָ וִיכַפַּר עֲלוֹהִי כָּהֲנָא וְיִשְׁתְּבֵיק לֵיהּ׃}
{And all the fat thereof shall he take away, as the fat is taken away from off the sacrifice of peace-offerings; and the priest shall make it smoke upon the altar for a sweet savour unto the \lord; and the priest shall make atonement for him, and he shall be forgiven.}{\arabic{verse}}
\rashi{\rashiDH{כאשר הוסר חלב מעל זבח השלמים.} כאימורי עז האמורים בשלמים׃}
\threeverse{\arabic{verse}}%Leviticus4:32
{וְאִם\maqqaf כֶּ֛בֶשׂ יָבִ֥יא קׇרְבָּנ֖וֹ לְחַטָּ֑את נְקֵבָ֥ה תְמִימָ֖ה יְבִיאֶֽנָּה׃}
{וְאִם אִמַּר יַיְתֵי קוּרְבָּנֵיהּ לְחַטָּתָא נוּקְבָּא שַׁלְמָא יַיְתֵינַהּ׃}
{And if he bring a lamb as his offering for a sin-offering, he shall bring it a female without blemish.}{\arabic{verse}}
\threeverse{\arabic{verse}}%Leviticus4:33
{וְסָמַךְ֙ אֶת\maqqaf יָד֔וֹ עַ֖ל רֹ֣אשׁ הַֽחַטָּ֑את וְשָׁחַ֤ט אֹתָהּ֙ לְחַטָּ֔את בִּמְק֕וֹם אֲשֶׁ֥ר יִשְׁחַ֖ט אֶת\maqqaf הָעֹלָֽה׃}
{וְיִסְמוֹךְ יָת יְדֵיהּ עַל רֵישׁ חַטָּתָא וְיִכּוֹס יָתַהּ לְחַטָּתָא בְּאַתְרָא דְּיִכּוֹס יָת עֲלָתָא׃}
{And he shall lay his hand upon the head of the sin-offering, and kill it for a sin-offering in the place where they kill the burnt-offering.}{\arabic{verse}}
\rashi{\rashiDH{ושחט אותה לחטאת.} שתהא שחיטתה לשם חטאת׃ 
}
\threeverse{\arabic{verse}}%Leviticus4:34
{וְלָקַ֨ח הַכֹּהֵ֜ן מִדַּ֤ם הַֽחַטָּאת֙ בְּאֶצְבָּע֔וֹ וְנָתַ֕ן עַל\maqqaf קַרְנֹ֖ת מִזְבַּ֣ח הָעֹלָ֑ה וְאֶת\maqqaf כׇּל\maqqaf דָּמָ֣הּ יִשְׁפֹּ֔ךְ אֶל\maqqaf יְס֖וֹד הַמִּזְבֵּֽחַ׃}
{וְיִסַּב כָּהֲנָא מִדְּמָא דְּחַטָּתָא בְּאֶצְבְּעֵיהּ וְיִתֵּין עַל קַרְנָת מַדְבְּחָא דַּעֲלָתָא וְיָת כָּל דְּמַהּ יִשְׁפּוֹךְ לִיסוֹדָא דְּמַדְבְּחָא׃}
{And the priest shall take of the blood of the sin-offering with his finger, and put it upon the horns of the altar of burnt-offering, and all the remaining blood thereof shall he pour out at the base of the altar.}{\arabic{verse}}
\threeverse{\arabic{verse}}%Leviticus4:35
{וְאֶת\maqqaf כׇּל\maqqaf חֶלְבָּ֣הּ יָסִ֗יר כַּאֲשֶׁ֨ר יוּסַ֥ר חֵֽלֶב\maqqaf הַכֶּ֘שֶׂב֮ מִזֶּ֣בַח הַשְּׁלָמִים֒ וְהִקְטִ֨יר הַכֹּהֵ֤ן אֹתָם֙ הַמִּזְבֵּ֔חָה עַ֖ל אִשֵּׁ֣י יְהֹוָ֑ה וְכִפֶּ֨ר עָלָ֧יו הַכֹּהֵ֛ן עַל\maqqaf חַטָּאת֥וֹ אֲשֶׁר\maqqaf חָטָ֖א וְנִסְלַ֥ח לֽוֹ׃ \petucha }
{וְיָת כָּל תַּרְבַּהּ יַעְדֵּי כְּמָא דְּמִתַּעְדַּא תְּרַב אִמַּר מִנִּכְסַת קוּדְשַׁיָּא וְיַסֵּיק כָּהֲנָא יָתְהוֹן לְמַדְבְּחָא עַל קוּרְבָּנַיָּא דַּייָ וִיכַפַּר עֲלוֹהִי כָּהֲנָא עַל חוֹבְתֵיהּ דְּחָב וְיִשְׁתְּבֵיק לֵיהּ׃}
{And all the fat thereof shall he take away, as the fat of the lamb is taken away from the sacrifice of peace-offerings; and the priest shall make them smoke on the altar, upon the offerings of the \lord\space made by fire; and the priest shall make atonement for him as touching his sin that he hath sinned, and he shall be forgiven.}{\arabic{verse}}
\rashi{\rashiDH{כאשר יוסר חלב הכשב.} שנתרבו אמורין וְאַלְיָה, אף חטאת כשהיא באה כבשה טעונה אַלְיָה עם האמורין׃\quad \rashiDH{על אשי ה׳.} על מדורות האש העשויות לשם, פואיילי״ש בלע״ז׃}
\newperek
\threeverse{\seder{ב*}}%Leviticus5:1
{וְנֶ֣פֶשׁ כִּֽי\maqqaf תֶחֱטָ֗א וְשָֽׁמְעָה֙ ק֣וֹל אָלָ֔ה וְה֣וּא עֵ֔ד א֥וֹ רָאָ֖ה א֣וֹ יָדָ֑ע אִם\maqqaf ל֥וֹא יַגִּ֖יד וְנָשָׂ֥א עֲוֺנֽוֹ׃}
{וַאֲנָשׁ אֲרֵי יְחוּב וְיִשְׁמַע קָל מוֹמֵי וְהוּא סָהִיד אוֹ חֲזָא אוֹ יְדַע אִם לָא יְחַוֵּי וִיקַבֵּיל חוֹבֵיהּ׃}
{And if any one sin, in that he heareth the voice of adjuration, he being a witness, whether he hath seen or known, if he do not utter it, then he shall bear his iniquity;}{\Roman{chap}}
\rashi{\rashiDH{ושמעה קול אלה.} בדבר שהוא עד בו, שהשביעוהו שבועה שאם יודע לו בעדות שיעיד לו׃}
\threeverse{\arabic{verse}}%Leviticus5:2
{א֣וֹ נֶ֗פֶשׁ אֲשֶׁ֣ר תִּגַּע֮ בְּכׇל\maqqaf דָּבָ֣ר טָמֵא֒ אוֹ֩ בְנִבְלַ֨ת חַיָּ֜ה טְמֵאָ֗ה א֚וֹ בְּנִבְלַת֙ בְּהֵמָ֣ה טְמֵאָ֔ה א֕וֹ בְּנִבְלַ֖ת שֶׁ֣רֶץ טָמֵ֑א וְנֶעְלַ֣ם מִמֶּ֔נּוּ וְה֥וּא טָמֵ֖א וְאָשֵֽׁם׃}
{אוֹ אֲנָשׁ דְּיִקְרַב בְּכָל מִדָּעַם מְסָאַב אוֹ בְנִבְלַת חַיְתָא מְסָאַבְתָּא אוֹ בְנִבְלַת בְּעִירָא מְסָאֲבָא אוֹ בְּנִבְלַת רְחֵישׁ מְסָאַב וִיהֵי מְכוּסַּא מִנֵּיהּ וְהוּא מְסָאַב וְחָב׃}
{or if any one touch any unclean thing, whether it be the carcass of an unclean beast, or the carcass of unclean cattle, or the carcass of unclean swarming things, and be guilty, it being hidden from him that he is unclean;}{\arabic{verse}}
\rashi{\rashiDH{או נפש אשר תגע וגו׳.} ולאחר הטומאה הזו יאכל קדשים, או יכנס למקדש, שהוא דבר שזדונו כרת, במסכת שבועות (ז.) נדרש כן׃ 
\quad \rashiDH{ונעלם ממנו.} הטומאה׃\quad \rashiDH{ואשם.} באכילת קודש או בביאת מקדש׃}
\threeverse{\arabic{verse}}%Leviticus5:3
{א֣וֹ כִ֤י יִגַּע֙ בְּטֻמְאַ֣ת אָדָ֔ם לְכֹל֙ טֻמְאָת֔וֹ אֲשֶׁ֥ר יִטְמָ֖א בָּ֑הּ וְנֶעְלַ֣ם מִמֶּ֔נּוּ וְה֥וּא יָדַ֖ע וְאָשֵֽׁם׃}
{אוֹ אֲרֵי יִקְרַב בְּסוֹאֲבָת אֲנָשָׁא לְכֹל סְאוֹבְתֵיהּ דְּיִסְתָּאַב בַּהּ וִיהֵי מְכוּסַּא מִנֵּיהּ וְהוּא יְדַע וְחָב׃}
{or if he touch the uncleanness of man, whatsoever his uncleanness be wherewith he is unclean, and it be hid from him; and, when he knoweth of it, be guilty;}{\arabic{verse}}
\rashi{\rashiDH{בטומאת אדם.} זו טומאת מת׃\quad \rashiDH{לכל טומאתו.} לרבות טומאת מגע זבין וזבות׃\quad \rashiDH{אשר יטמא.} לרבות הנוגע בבועל נדה׃\quad \rashiDH{בה.} לרבות בולע נבלת עוף טהור׃\quad \rashiDH{ונעלם.} ולא ידע. ששכח הטומאה׃\quad \rashiDH{ואשם.} באכילת קודש או בביאת מקדש׃}
\threeverse{\arabic{verse}}%Leviticus5:4
{א֣וֹ נֶ֡פֶשׁ כִּ֣י תִשָּׁבַע֩ לְבַטֵּ֨א בִשְׂפָתַ֜יִם לְהָרַ֣ע \legarmeh  א֣וֹ לְהֵיטִ֗יב לְ֠כֹ֠ל אֲשֶׁ֨ר יְבַטֵּ֧א הָאָדָ֛ם בִּשְׁבֻעָ֖ה וְנֶעְלַ֣ם מִמֶּ֑נּוּ וְהוּא\maqqaf יָדַ֥ע וְאָשֵׁ֖ם לְאַחַ֥ת מֵאֵֽלֶּה׃}
{אוֹ אֲנָשׁ אֲרֵי יְקַיֵּים לְפָרָשָׁא בְשִׂפְוָן לְאַבְאָשָׁא אוֹ לְאֵיטָבָא לְכֹל דִּיפָרֵישׁ אֲנָשָׁא בְּקִיּוּם וִיהֵי מְכוּסַּא מִנֵּיהּ וְהוּא יְדַע וְחָב לַחֲדָא מֵאִלֵּין׃}
{or if any one swear clearly with his lips to do evil, or to do good, whatsoever it be that a man shall utter clearly with an oath, and it be hid from him; and, when he knoweth of it, be guilty in one of these things;}{\arabic{verse}}
\rashi{\rashiDH{בשפתים.} ולא בלב׃\quad \rashiDH{להרע.} לעצמו׃\quad \rashiDH{או להיטיב.} לעצמו כגון אוכל ולא אוכל אישן ולא אישן׃\quad \rashiDH{לכל אשר יבטא.} לרבות לשעבר (שבועות כו.)׃\quad \rashiDH{ונעלם ממנו.} ועבר על שבועתו. כל אלה בקרבן עולה ויורד כמפורש כאן, אבל שבועה שיש בה כפירת ממון אינה בקרבן זו אלא באשם׃ 
}
\threeverse{\arabic{verse}}%Leviticus5:5
{וְהָיָ֥ה כִֽי\maqqaf יֶאְשַׁ֖ם לְאַחַ֣ת מֵאֵ֑לֶּה וְהִ֨תְוַדָּ֔ה אֲשֶׁ֥ר חָטָ֖א עָלֶֽיהָ׃}
{וִיהֵי אֲרֵי יְחוּב לַחֲדָא מֵאִלֵּין וִיוַדֵּי דְּחָב עֲלַהּ׃}
{and it shall be, when he shall be guilty in one of these things, that he shall confess that wherein he hath sinned;}{\arabic{verse}}
\threeverse{\arabic{verse}}%Leviticus5:6
{וְהֵבִ֣יא אֶת\maqqaf אֲשָׁמ֣וֹ לַיהֹוָ֡ה עַ֣ל חַטָּאתוֹ֩ אֲשֶׁ֨ר חָטָ֜א נְקֵבָ֨ה מִן\maqqaf הַצֹּ֥אן כִּשְׂבָּ֛ה אֽוֹ\maqqaf שְׂעִירַ֥ת עִזִּ֖ים לְחַטָּ֑את וְכִפֶּ֥ר עָלָ֛יו הַכֹּהֵ֖ן מֵחַטָּאתֽוֹ׃}
{וְיַיְתֵי יָת אֲשָׁמֵיהּ לִקְדָם יְיָ עַל חוֹבְתֵיהּ דְּחָב נוּקְבָּא מִן עָנָא אִמַּרְתָּא אוֹ צְפִירַת עִזֵּי לְחַטָּתָא וִיכַפַּר עֲלוֹהִי כָּהֲנָא מֵחוֹבְתֵיהּ׃}
{and he shall bring his forfeit unto the \lord\space for his sin which he hath sinned, a female from the flock, a lamb or a goat, for a sin-offering; and the priest shall make atonement for him as concerning his sin.}{\arabic{verse}}
\threeverse{\arabic{verse}}%Leviticus5:7
{וְאִם\maqqaf לֹ֨א תַגִּ֣יעַ יָדוֹ֮ דֵּ֣י שֶׂה֒ וְהֵבִ֨יא אֶת\maqqaf אֲשָׁמ֜וֹ אֲשֶׁ֣ר חָטָ֗א שְׁתֵּ֥י תֹרִ֛ים אֽוֹ\maqqaf שְׁנֵ֥י בְנֵֽי\maqqaf יוֹנָ֖ה לַֽיהֹוָ֑ה אֶחָ֥ד לְחַטָּ֖את וְאֶחָ֥ד לְעֹלָֽה׃}
{וְאִם לָא תִמְטֵי יְדֵיהּ כְּמִסַּת סִיתָא וְיַיְתֵי יָת חוֹבְתֵיהּ דְּחָב תְּרֵין שַׁפְנִינִין אוֹ תְּרֵין בְּנֵי יוֹנָה לִקְדָם יְיָ חַד לְחַטָּתָא וְחַד לַעֲלָתָא׃}
{And if his means suffice not for a lamb, then he shall bring his forfeit for that wherein he hath sinned, two turtle-doves, or two young pigeons, unto the \lord: one for a sin-offering, and the other for a burnt-offering.}{\arabic{verse}}
\threeverse{\arabic{verse}}%Leviticus5:8
{וְהֵבִ֤יא אֹתָם֙ אֶל\maqqaf הַכֹּהֵ֔ן וְהִקְרִ֛יב אֶת\maqqaf אֲשֶׁ֥ר לַחַטָּ֖את רִאשׁוֹנָ֑ה וּמָלַ֧ק אֶת\maqqaf רֹאשׁ֛וֹ מִמּ֥וּל עׇרְפּ֖וֹ וְלֹ֥א יַבְדִּֽיל׃}
{וְיַיְתֵי יָתְהוֹן לְוָת כָּהֲנָא וִיקָרֵיב יָת דִּלְחַטָּתָא קַדְמוּתָא וְיִמְלוֹק יָת רֵישֵׁיהּ מִקֳּבֵיל קְדָלֵיהּ וְלָא יַפְרֵישׁ׃}
{And he shall bring them unto the priest, who shall offer that which is for the sin-offering first, and pinch off its head close by its neck, but shall not divide it asunder.}{\arabic{verse}}
\rashi{\rashiDH{והקריב את אשר לחטאת ראשונה.} חטאת קודמת לעולה למה הדבר דומה לפרקליט שנכנס לרצות, ריצה פרקליט נכנס דורון אחריו (זבחים ז׃)׃\quad \rashiDH{ולא יבדיל.} אינו מולק אלא סימן אחד (חולין כא.)׃\quad \rashiDH{עורף.} הוא גובה הראש המשופע לצד הצואר׃\quad \rashiDH{מול עורף.} מול הרואה את העורף, והוא אורך כל אחורי הצואר׃}
\threeverse{\arabic{verse}}%Leviticus5:9
{וְהִזָּ֞ה מִדַּ֤ם הַחַטָּאת֙ עַל\maqqaf קִ֣יר הַמִּזְבֵּ֔חַ וְהַנִּשְׁאָ֣ר בַּדָּ֔ם יִמָּצֵ֖ה אֶל\maqqaf יְס֣וֹד הַמִּזְבֵּ֑חַ חַטָּ֖את הֽוּא׃}
{וְיַדֵּי מִדְּמָא דְּחַטָּאתָא עַל כֹּתֶל מַדְבְּחָא וּדְיִשְׁתְּאַר בִּדְמָא יִתְמְצֵי לִיסוֹדָא דְּמַדְבְּחָא חַטָּתָא הוּא׃}
{And he shall sprinkle of the blood of the sin-offering upon the side of the altar; and the rest of the blood shall be drained out at the base of the altar; it is a sin-offering.}{\arabic{verse}}
\rashi{\rashiDH{והזה מדם החטאת.} בעולה לא הטעין אלא מצוי, ובחטאת הזאה ומצוי, אוחז בעורף ומתיז, והדם ניתז והולך למזבח (זבחים סד׃)׃\quad \rashiDH{חטאת הוא.} לשמה כשרה, שלא לשמה פסולה׃}
\threeverse{\arabic{verse}}%Leviticus5:10
{וְאֶת\maqqaf הַשֵּׁנִ֛י יַעֲשֶׂ֥ה עֹלָ֖ה כַּמִּשְׁפָּ֑ט וְכִפֶּ֨ר עָלָ֧יו הַכֹּהֵ֛ן מֵחַטָּאת֥וֹ אֲשֶׁר\maqqaf חָטָ֖א וְנִסְלַ֥ח לֽוֹ׃ \setuma }
{וְיָת תִּנְיָנָא יַעֲבֵיד עֲלָתָא כְּדַחְזֵי וִיכַפַּר עֲלוֹהִי כָהֲנָא מֵחוֹבְתֵיהּ דְּחָב וְיִשְׁתְּבֵיק לֵיהּ׃}
{And he shall prepare the second for a burnt-offering, according to the ordinance; and the priest shall make atonement for him as concerning his sin which he hath sinned, and he shall be forgiven.}{\arabic{verse}}
\rashi{\rashiDH{כמשפט.} כדת האמור בעולת העוף של נדבה בראש הפרשה׃}
\aliyacounter{שביעי}
\threeverse{\aliya{שביעי}}%Leviticus5:11
{וְאִם\maqqaf לֹא֩ תַשִּׂ֨יג יָד֜וֹ לִשְׁתֵּ֣י תֹרִ֗ים אוֹ֮ לִשְׁנֵ֣י בְנֵי\maqqaf יוֹנָה֒ וְהֵבִ֨יא אֶת\maqqaf קׇרְבָּנ֜וֹ אֲשֶׁ֣ר חָטָ֗א עֲשִׂירִ֧ת הָאֵפָ֛ה סֹ֖לֶת לְחַטָּ֑את לֹא\maqqaf יָשִׂ֨ים עָלֶ֜יהָ שֶׁ֗מֶן וְלֹא\maqqaf יִתֵּ֤ן עָלֶ֙יהָ֙ לְבֹנָ֔ה כִּ֥י חַטָּ֖את הִֽוא׃}
{וְאִם לָא תַדְבֵּיק יְדֵיהּ לִתְרֵין שַׁפְנִינִין אוֹ לִתְרֵין בְּנֵי יוֹנָה וְיַיְתֵי יָת קוּרְבָּנֵיהּ דְּחָב חַד מִן עַסְרָא בִּתְלָת סְאִין סוּלְתָּא לְחַטָּתָא לָא יְשַׁוֵּי עֲלַהּ מִשְׁחָא וְלָא יִתֵּין עֲלַהּ לְבוֹנְתָא אֲרֵי חַטָּתָא הִיא׃}
{But if his means suffice not for two turtledoves, or two young pigeons, then he shall bring his offering for that wherein he hath sinned, the tenth part of an ephah of fine flour for a sin-offering; he shall put no oil upon it, neither shall he put any frankincense thereon; for it is a sin-offering.}{\arabic{verse}}
\rashi{\rashiDH{כי חטאת הוא.} ואין בדין שיהא קָרְבָּנוֹ מְהֻדָּר (מנחות ו.)׃}
\threeverse{\arabic{verse}}%Leviticus5:12
{וֶהֱבִיאָהּ֮ אֶל\maqqaf הַכֹּהֵן֒ וְקָמַ֣ץ הַכֹּהֵ֣ן \pasek  מִ֠מֶּ֠נָּה מְל֨וֹא קֻמְצ֜וֹ אֶת\maqqaf אַזְכָּרָתָהּ֙ וְהִקְטִ֣יר הַמִּזְבֵּ֔חָה עַ֖ל אִשֵּׁ֣י יְהֹוָ֑ה חַטָּ֖את הִֽוא׃}
{וְיַיְתֵינַהּ לְוָת כָּהֲנָא וְיִקְמוֹץ כָּהֲנָא מִנַּהּ מְלֵי קֻמְצֵיהּ יָת אַדְכָרְתַהּ וְיַסֵּיק לְמַדְבְּחָא עַל קוּרְבָּנַיָּא דַּייָ חַטָּתָא הִיא׃}
{And he shall bring it to the priest, and the priest shall take his handful of it as the memorial-part thereof, and make it smoke on the altar, upon the offerings of the \lord\space made by fire; it is a sin-offering.}{\arabic{verse}}
\rashi{\rashiDH{חטאת הוא.} נקמצה ונקטרה לשמה כשרה שלא לשמה פסולה׃ 
}
\threeverse{\arabic{verse}}%Leviticus5:13
{וְכִפֶּר֩ עָלָ֨יו הַכֹּהֵ֜ן עַל\maqqaf חַטָּאת֧וֹ אֲשֶׁר\maqqaf חָטָ֛א מֵֽאַחַ֥ת מֵאֵ֖לֶּה וְנִסְלַ֣ח ל֑וֹ וְהָיְתָ֥ה לַכֹּהֵ֖ן כַּמִּנְחָֽה׃ \setuma }
{וִיכַפַּר עֲלוֹהִי כָהֲנָא עַל חוֹבְתֵיהּ דְּחָב מֵחֲדָא מֵאִלֵּין וְיִשְׁתְּבֵיק לֵיהּ וּתְהֵי לְכָהֲנָא כְּמִנְחָתָא׃}
{And the priest shall make atonement for him as touching his sin that he hath sinned in any of these things, and he shall be forgiven; and the remnant shall be the priest’s, as the meal-offering.}{\arabic{verse}}
\rashi{\rashiDH{על חטאתו אשר חטא.} כאן שנה הכתוב, שהרי בעשירות ובדלות נאמר מחטאתו, וכאן בדלי דלות נאמר על חטאתו, דקדקו רבותינו (כריתות כז׃) מכאן שאם חטא כשהוא עשיר והפריש מעות לכשבה או שעירה והעני, יביא ממקצתן שתי תורים, הפריש מעות לשתי תורים והעני, יביא ממקצתן עשירית האיפה, (לכך נאמר מחטאתו), הפריש מעות לעשירית האיפה והעשיר, יוסיף עליהן ויביא קרבן עשיר, לכך נאמר כאן על חטאתו׃\quad \rashiDH{מאחת מאלה.} מאחת משלש כפרות האמורות בענין, או בעשירות, או בדלות, או בדלי דלות. ומה תלמוד לומר, שיכול הַחֲמוּרִים שבהם יהיו בכשבה או שעירה, והקלין יהיו בעוף, והקלין שבקלין יהיו בעשירית האיפה, תלמוד לומר מאחת מאלה, להשוות קלין לחמורין לכשבה ושעירה אם השיגה ידו, ואת החמורין לקלין לעשירית האיפה בדלי דלות (ת״כ פרק יט, י)׃\quad \rashiDH{והיתה לכהן כמנחה.} ללמד על מנחת חוטא שיהיו שיריה נאכלין, זהו לפי פשוטו. ורבותינו דרשו (שם יא. מנחות עג׃), והיתה לכהן, ואם חוטא זה כהן הוא, תהא כשאר מנחת נדבת כהן, שהוא כָּלִיל תִּהְיֶה לֹא תֵאָכֵל (ויקרא ו, טז)׃}
\threeverse{\arabic{verse}}%Leviticus5:14
{וַיְדַבֵּ֥ר יְהֹוָ֖ה אֶל\maqqaf מֹשֶׁ֥ה לֵּאמֹֽר׃}
{וּמַלֵּיל יְיָ עִם מֹשֶׁה לְמֵימַר׃}
{And the \lord\space spoke unto Moses, saying:}{\arabic{verse}}
\threeverse{\arabic{verse}}%Leviticus5:15
{נֶ֚פֶשׁ כִּֽי\maqqaf תִמְעֹ֣ל מַ֔עַל וְחָֽטְאָה֙ בִּשְׁגָגָ֔ה מִקׇּדְשֵׁ֖י יְהֹוָ֑ה וְהֵבִיא֩ אֶת\maqqaf אֲשָׁמ֨וֹ לַֽיהֹוָ֜ה אַ֧יִל תָּמִ֣ים מִן\maqqaf הַצֹּ֗אן בְּעֶרְכְּךָ֛ כֶּֽסֶף\maqqaf שְׁקָלִ֥ים בְּשֶֽׁקֶל\maqqaf הַקֹּ֖דֶשׁ לְאָשָֽׁם׃}
{אֱנָשׁ אֲרֵי יְשַׁקַּר שְׁקַר וִיחוּב בְּשָׁלוּ מִקּוּדְשַׁיָּא דַּייָ וְיַיְתֵי יָת אֲשָׁמֵיהּ לִקְדָם יְיָ דְּכַר שְׁלִים מִן עָנָא בְּפוּרְסָנֵיהּ כְּסַף סִלְעִין בְּסִלְעֵי קוּדְשָׁא לַאֲשָׁמָא׃}
{If any one commit a trespass, and sin through error, in the holy things of the \lord, then he shall bring his forfeit unto the \lord, a ram without blemish out of the flock, according to thy valuation in silver by shekels, after the shekel of the sanctuary, for a guilt-offering.}{\arabic{verse}}
\rashi{\rashiDH{כי תמעל מעל.} אין מעילה בכל מקום אלא שינוי (ת״כ פרשתא יא, יב), וכן הוא אומר וַיִּמְעֲלוּ בֵּאלֹהֵי אֲבֹותֵיהֶם וַיִּזְנוּ אַחֲרֵי אֱלֹהֵי עַמֵּי הָאָרֶץ (דברי הימים־א ה, כה), וכן הוא אומר בסוטה וּמָעֲלָה בוֹ מָעַל (במדבר ה, יב)׃\quad \rashiDH{וחטאה בשגגה מקדשי ה׳.} שנהנה מן ההקדש, והיכן הוזהר, נאמר כאן חטא, ונאמר להלן חטא בתרומה וְלֹא יִשְׂאוּ עָלָיו חֵטְא (ויקרא כב, ט), מה להלן הזהיר אף כאן הזהיר, אי מה להלן לא הזהיר אלא על האוכל אף כאן לא הזהיר אלא על האוכל, תלמוד לומר תמעול מעל ריבה (מעילה יח׃)׃\quad \rashiDH{מקדשי ה׳.} המיוחדים לשם, יצאו קדשים קלים׃ 
\quad \rashiDH{איל.} ל׳ קשה, כמו וְאֶת אֵילֵי הָאָרֶץ לָקָח (יחזקאל יז, יג), אף כאן קשה בן שתי שנים׃\quad \rashiDH{בערכך כסף שקלים.} שיהא שוה שתי סלעים׃}
\threeverse{\arabic{verse}}%Leviticus5:16
{וְאֵ֣ת אֲשֶׁר֩ חָטָ֨א מִן\maqqaf הַקֹּ֜דֶשׁ יְשַׁלֵּ֗ם וְאֶת\maqqaf חֲמִֽישִׁתוֹ֙ יוֹסֵ֣ף עָלָ֔יו וְנָתַ֥ן אֹת֖וֹ לַכֹּהֵ֑ן וְהַכֹּהֵ֗ן יְכַפֵּ֥ר עָלָ֛יו בְּאֵ֥יל הָאָשָׁ֖ם וְנִסְלַ֥ח לֽוֹ׃ \petucha }
{וְיָת דְּחָב מִן קוּדְשָׁא יְשַׁלֵּים וְיָת חוּמְשֵׁיהּ יוֹסֵיף עֲלוֹהִי וְיִתֵּין יָתֵיהּ לְכָהֲנָא וְכָהֲנָא יְכַפַּר עֲלוֹהִי בְּדִכְרָא דַּאֲשָׁמָא וְיִשְׁתְּבֵיק לֵיהּ׃}
{And he shall make restitution for that which he hath done amiss in the holy thing, and shall add the fifth part thereto, and give it unto the priest; and the priest shall make atonement for him with the ram of the guilt-offering, and he shall be forgiven.}{\arabic{verse}}
\rashi{\rashiDH{ואת אשר חטא מן הקדש ישלם.} קרן וחומש להקדש (כריתות כו׃)׃}
\threeverse{\arabic{verse}}%Leviticus5:17
{וְאִם\maqqaf נֶ֙פֶשׁ֙ כִּ֣י תֶֽחֱטָ֔א וְעָֽשְׂתָ֗ה אַחַת֙ מִכׇּל\maqqaf מִצְוֺ֣ת יְהֹוָ֔ה אֲשֶׁ֖ר לֹ֣א תֵעָשֶׂ֑ינָה וְלֹֽא\maqqaf יָדַ֥ע וְאָשֵׁ֖ם וְנָשָׂ֥א עֲוֺנֽוֹ׃}
{וְאִם אֱנָשׁ אֲרֵי יְחוּב וְיַעֲבֵיד חַד מִכָּל פִּקּוֹדַיָּא דַּייָ דְּלָא כָשְׁרִין לְאִתְעֲבָדָא וְלָא יְדַע וְחָב וִיקַבֵּיל חוֹבֵיהּ׃}
{And if any one sin, and do any of the things which the \lord\space hath commanded not to be done, though he know it not, yet is he guilty, and shall bear his iniquity.}{\arabic{verse}}
\rashi{\rashiDH{ולא ידע ואשם והביא.} הענין הזה מדבר במי שבא ספק כרת לידו ולא ידע אם עבר עליו אם לאו, כגון חֵלֶב ושומן לפניו, וכסבור ששתיהן היתר, ואכל את האחת, אמרו לו אחת של חֵלֶב היתה, ולא ידע אם זו של חֵלֶב אכל, הרי זה מביא אשם תלוי ומגין עליו כל זמן שלא נודע לו שודאי חטא, ואם יודע לו לאחר זמן יביא חטאת׃\quad \rashiDH{ולא ידע ואשם ונשא עונו.} (ת״כ) ר׳ יוסי הגלילי אומר (שם פרשתא יב, ז), הרי הכתוב ענש את מי שלא ידע, על אחת כמה וכמה שיעניש את שידע. רבי יוסי אומר (שם׃) אם נפשך לידע מתן שכרן של צדיקים, צא ולמד מאדם הראשון, שלא נצטוה אלא על מצוַת לא תעשה, ועבר עליה, ראה כמה מיתות נקנסו עליו ולדורותיו, וכי איזו מדה מרובה, של טובה או של פורענות, הוי אומר מדה טובה, אם מדת פורענות המעוטה ראה כמה מיתות נקנסו לו ולדורותיו, מדה טובה המרובה, היושב לו מן הפגולין והנותרות והמתענה ביום הכיפורים, על אחת כמה וכמה שיזכה לו ולדורותיו ולדורות דורותיו עד סוף כל הדורות. ר׳ עקיבא אומר (שם יא) הרי הוא אומר עַל פִּי ב׳ עֵדִים אֹו ג׳ עֵדִים וגו׳ (דברים יז, ו), אם מתקיימת העדות בשנים למה פרט לך הכתוב בג׳, אלא להביא שלישי להחמיר עליו ולעשות דינו כיוצא באלו לענין עונש והזמה, אם כך ענש הכתוב לנטפל לעוברי עבירה כעוברי עבירה, על אחת כמה וכמה שישלם שכר טוב לנטפל לעושי מצוה כעושי מצוה. ר׳ אלעזר בן עזריא אומר כִּי תִקְצֹור קִצִירְךָ בְּשָׂדֶךָ (דברים כד, יט) וְשָׁכַחְתָּ עֹומֶר בַּשָׁדֶה (שם יג), הרי הוא אומר למען יברכך וגו׳, קבע הכתוב ברכה למי שבאת על ידו מצוה בלא ידע, אמור מעתה היתה סלע צרורה בכנפיו, ונפלה הימנו, ומצאה העני ונתפרנס בה, הרי הקב״ה קובע לו ברכה׃}
\threeverse{\arabic{verse}}%Leviticus5:18
{וְ֠הֵבִ֠יא אַ֣יִל תָּמִ֧ים מִן\maqqaf הַצֹּ֛אן בְּעֶרְכְּךָ֥ לְאָשָׁ֖ם אֶל\maqqaf הַכֹּהֵ֑ן וְכִפֶּר֩ עָלָ֨יו הַכֹּהֵ֜ן עַ֣ל שִׁגְגָת֧וֹ אֲשֶׁר\maqqaf שָׁגָ֛ג וְה֥וּא לֹֽא\maqqaf יָדַ֖ע וְנִסְלַ֥ח לֽוֹ׃}
{וְיַיְתֵי דְּכַר שְׁלִים מִן עָנָא בְּפֻרְסָנֵיהּ לַאֲשָׁמָא לְוָת כָּהֲנָא וִיכַפַּר עֲלוֹהִי כָּהֲנָא עַל שָׁלוּתֵיהּ דְּאִשְׁתְּלִי וְהוּא לָא יְדַע וְיִשְׁתְּבֵיק לֵיהּ׃}
{And he shall bring a ram without blemish out of the flock, according to thy valuation, for a guilt-offering, unto the priest; and the priest shall make atonement for him concerning the error which he committed, though he knew it not, and he shall be forgiven.}{\arabic{verse}}
\rashi{\rashiDH{בערכך לאשם.} בערך האמור למעלה׃\quad \rashiDH{אשר שגג והוא לא ידע.} הא אם ידע לאחר זמן לא נתכפר לו באשם זה, עד שיביא חטאת, הא למה זה דומה, לעגלה ערופה שנתערפה ואח״כ נמצא ההורג, הרי זה יהרג (ת״כ פרק כא, ב)׃}
\threeverse{\arabic{verse}}%Leviticus5:19
{אָשָׁ֖ם ה֑וּא אָשֹׁ֥ם אָשַׁ֖ם לַיהֹוָֽה׃ \petucha }
{אֲשָׁמָא הוּא עַל חוֹבְתֵיהּ דְּחָב אֲשָׁמָא יְקָרֵיב קֳדָם יְיָ׃}
{It is a guilt-offering—he is certainly guilty before the \lord.}{\arabic{verse}}
\rashi{\rashiDH{אשם הוא אשם אשם.} הראשון כולו קמץ שהוא שם דבר, והאחרון חציו קמץ וחציו פתח שהוא לשון פעל, ואם תאמר מקרא שלא לצורך הוא, כבר נדרש הוא בת״כ (פרק כא, ג)׃ \rashiDH{אשם אשם.} להביא אשם שפחה חרופה שיהא איל (בן שתי שנים) [שוה שתי סלעים]. יכול שאני מרבה אשם נזיר ואשם מצורע תלמוד לומר, הוא׃}
\threeverse{\arabic{verse}}%Leviticus5:20
{וַיְדַבֵּ֥ר יְהֹוָ֖ה אֶל\maqqaf מֹשֶׁ֥ה לֵּאמֹֽר׃}
{וּמַלֵּיל יְיָ עִם מֹשֶׁה לְמֵימַר׃}
{And the \lord\space spoke unto Moses, saying:}{\arabic{verse}}
\threeverse{\arabic{verse}}%Leviticus5:21
{נֶ֚פֶשׁ כִּ֣י תֶחֱטָ֔א וּמָעֲלָ֥ה מַ֖עַל בַּיהֹוָ֑ה וְכִחֵ֨שׁ בַּעֲמִית֜וֹ בְּפִקָּד֗וֹן אֽוֹ\maqqaf בִתְשׂ֤וּמֶת יָד֙ א֣וֹ בְגָזֵ֔ל א֖וֹ עָשַׁ֥ק אֶת\maqqaf עֲמִיתֽוֹ׃}
{אֱנָשׁ אֲרֵי יְחוּב וִישַׁקַּר שְׁקַר קֳדָם יְיָ וִיכַדֵּיב בְּחַבְרֵיהּ בְּפֻקְדָנָא אוֹ בְּשוּתָּפוּת יְדָא אוֹ בְּגָזֵילָא אוֹ עֲשַׁק יָת חַבְרֵיהּ׃}
{If any one sin, and commit a trespass against the \lord, and deal falsely with his neighbour in a matter of deposit, or of pledge, or of robbery, or have oppressed his neighbour;}{\arabic{verse}}
\rashi{\rashiDH{נפש כי תחטא.} אמר ר׳ עקיבא (שם פרק כב, ד) מה תלמוד לומר ומעלה מעל בה׳, לפי שכל המלוה והלוה והנושא והנותן אינו עושה אלא בעדים ובשטר, לפיכך בזמן שהיא מכחש, מכחש בעדים ובשטר, אבל המפקיד אצל חבירו ואינו רוצה שתדע בו נשמה אלא שלישי שביניהם, לפיכך כשהוא מכחש, מכחש בשלישי שביניהם׃\quad \rashiDH{בתשומת יד.} שֶׁשָׂם בידו ממון להתעסק, או במלוה׃\quad \rashiDH{או בגזל.} שגזל מידו כלום׃\quad \rashiDH{או עשק.} הוא שכר שכיר׃}
\threeverse{\arabic{verse}}%Leviticus5:22
{אֽוֹ\maqqaf מָצָ֧א אֲבֵדָ֛ה וְכִ֥חֶשׁ בָּ֖הּ וְנִשְׁבַּ֣ע עַל\maqqaf שָׁ֑קֶר עַל\maqqaf אַחַ֗ת מִכֹּ֛ל אֲשֶׁר\maqqaf יַעֲשֶׂ֥ה הָאָדָ֖ם לַחֲטֹ֥א בָהֵֽנָּה׃}
{אוֹ אַשְׁכַּח אֲבֵידְתָא וְכַדֵּיב בַּהּ וְאִשְׁתְּבַע עַל שִׁקְרָא עַל חֲדָא מִכֹּל דְּיַעֲבֵיד אֲנָשָׁא לִמְחָב בְּהוֹן׃}
{or have found that which was lost, and deal falsely therein, and swear to a lie; in any of all these that a man doeth, sinning therein;}{\arabic{verse}}
\rashi{\rashiDH{וכחש בה.} שכפר על אחת מכל אלה אשר יעשה האדם לחטוא ולהשבע על שקר לכפירת ממון׃}
\threeverse{\arabic{verse}}%Leviticus5:23
{וְהָיָה֮ כִּֽי\maqqaf יֶחֱטָ֣א וְאָשֵׁם֒ וְהֵשִׁ֨יב אֶת\maqqaf הַגְּזֵלָ֜ה אֲשֶׁ֣ר גָּזָ֗ל א֤וֹ אֶת\maqqaf הָעֹ֙שֶׁק֙ אֲשֶׁ֣ר עָשָׁ֔ק א֚וֹ אֶת\maqqaf הַפִּקָּד֔וֹן אֲשֶׁ֥ר הׇפְקַ֖ד אִתּ֑וֹ א֥וֹ אֶת\maqqaf הָאֲבֵדָ֖ה אֲשֶׁ֥ר מָצָֽא׃}
{וִיהֵי אֲרֵי יִחְטֵי וִיחוּב וְיָתִיב יָת גָּזֵילָא דִּגְזַל אוֹ יָת עוּשְׁקָא דַּעֲשַׁק אוֹ יָת פִּקְדוֹנָא דְּאִתַּפְקַד לְוָתֵיהּ אוֹ יָת אֲבֵידְתָא דְּאַשְׁכַּח׃}
{then it shall be, if he hath sinned, and is guilty, that he shall restore that which he took by robbery, or the thing which he hath gotten by oppression, or the deposit which was deposited with him, or the lost thing which he found,}{\arabic{verse}}
\rashi{\rashiDH{כי יחטא ואשם.} כשיכיר בעצמו לשוב בתשובה, ולדעת ולהתודות כי יחטא (גי׳ ס״א ובדעתו להתודות כי חטא) ואשם׃ 
}
\threeverse{\aliya{מפטיר}}%Leviticus5:24
{א֠וֹ מִכֹּ֞ל אֲשֶׁר\maqqaf יִשָּׁבַ֣ע עָלָיו֮ לַשֶּׁ֒קֶר֒ וְשִׁלַּ֤ם אֹתוֹ֙ בְּרֹאשׁ֔וֹ וַחֲמִשִׁתָ֖יו יֹסֵ֣ף עָלָ֑יו לַאֲשֶׁ֨ר ה֥וּא ל֛וֹ יִתְּנֶ֖נּוּ בְּי֥וֹם אַשְׁמָתֽוֹ׃}
{אוֹ מִכּוֹלָא דְּיִשְׁתְּבַע עֲלוֹהִי לְשִׁקְרָא וִישַׁלֵּים יָתֵיהּ בְּרֵישֵׁיהּ וְחֻמְשׁוֹהִי יוֹסֵיף עֲלוֹהִי לִדְהוּא דִּילֵיהּ יִתְּנִנֵּיהּ בְּיוֹמָא דְּחוֹבְתֵיהּ׃}
{or any thing about which he hath sworn falsely, he shall even restore it in full, and shall add the fifth part more thereto; unto him to whom it appertaineth shall he give it, in the day of his being guilty.}{\arabic{verse}}
\rashi{\rashiDH{בראשו.} הוא הקרן ראש הממון׃\quad \rashiDH{וחמשתיו.} (ב״ק קח.) רבתה תורה חמשיות הרבה לקרן אחת, שאם כפר בחומש ונשבע והודה חוזר ומביא חומש על אותו חומש, וכן מוסיף והולך עד שיתמעט הקרן שנשבע עליו פחות משוה פרוטה׃\quad \rashiDH{לאשר הוא לו.} (לאפוקי בנו ושלוחו ת״כ) למי שהממון שלו׃}
\threeverse{\arabic{verse}}%Leviticus5:25
{וְאֶת\maqqaf אֲשָׁמ֥וֹ יָבִ֖יא לַיהֹוָ֑ה אַ֣יִל תָּמִ֧ים מִן\maqqaf הַצֹּ֛אן בְּעֶרְכְּךָ֥ לְאָשָׁ֖ם אֶל\maqqaf הַכֹּהֵֽן׃}
{וְיָת אֲשָׁמֵיהּ יַיְתֵי לִקְדָם יְיָ דְּכַר שְׁלִים מִן עָנָא בְּפֻרְסָנֵיהּ לַאֲשָׁמָא לְוָת כָּהֲנָא׃}
{And he shall bring his forfeit unto the \lord, a ram without blemish out of the flock, according to thy valuation, for a guilt-offering, unto the priest.}{\arabic{verse}}

\threeverse{\aliya{\Hebrewnumeral{111}}}%Leviticus5:26
{וְכִפֶּ֨ר עָלָ֧יו הַכֹּהֵ֛ן לִפְנֵ֥י יְהֹוָ֖ה וְנִסְלַ֣ח ל֑וֹ עַל\maqqaf אַחַ֛ת מִכֹּ֥ל אֲשֶֽׁר\maqqaf יַעֲשֶׂ֖ה לְאַשְׁמָ֥ה בָֽהּ׃ \petucha }
{וִיכַפַּר עֲלוֹהִי כָּהֲנָא קֳדָם יְיָ וְיִשְׁתְּבֵיק לֵיהּ עַל חֲדָא מִכֹּל דְּיַעֲבֵיד לִמְחָב בַּהּ׃}
{And the priest shall make atonement for him before the \lord, and he shall be forgiven, concerning whatsoever he doeth so as to be guilty thereby.}{\arabic{verse}}

\engnote{The Haftarah is Isaiah 43:21\verserangechar 44:23 on page \pageref{haft_24}. For Shabbat Zachor the maftir and Haftara are on page \pageref{maft_zachor}.  On Shabbat Ha\d{H}odesh read the Maftir and Haftara on page \pageref{maft_hachodesh}.}
\newperek
\aliyacounter{ראשון}
\newparsha{צו}
\threeverse{\aliya{צו}}%Leviticus6:1
{וַיְדַבֵּ֥ר יְהֹוָ֖ה אֶל\maqqaf מֹשֶׁ֥ה לֵּאמֹֽר׃}
{וּמַלֵּיל יְיָ עִם מֹשֶׁה לְמֵימַר׃}
{And the \lord\space spoke unto Moses, saying:}{\Roman{chap}}
\threeverse{\arabic{verse}}%Leviticus6:2
{צַ֤ו אֶֽת\maqqaf אַהֲרֹן֙ וְאֶת\maqqaf בָּנָ֣יו לֵאמֹ֔ר זֹ֥את תּוֹרַ֖ת הָעֹלָ֑ה הִ֣וא הָעֹלָ֡ה עַל֩ \footnotesize מ\normalsize וֹקְדָ֨ה\note{בספרי תימן מוֹקְדָ֨ה במ״ם רגילה} עַל\maqqaf הַמִּזְבֵּ֤חַ כׇּל\maqqaf הַלַּ֙יְלָה֙ עַד\maqqaf הַבֹּ֔קֶר וְאֵ֥שׁ הַמִּזְבֵּ֖חַ תּ֥וּקַד בּֽוֹ׃}
{פַּקֵּיד יָת אַהֲרֹן וְיָת בְּנוֹהִי לְמֵימַר דָּא אוֹרָיְתָא דַּעֲלָתָא הִיא עֲלָתָא דְּמִתּוֹקְדָא עַל מַדְבְּחָא כָּל לֵילְיָא עַד צַפְרָא וְאִישָׁתָא דְּמַדְבְּחָא תְּהֵי יָקְדָא בֵיהּ׃}
{Command Aaron and his sons, saying: This is the law of the burnt-offering: it is that which goeth up on its firewood upon the altar all night unto the morning; and the fire of the altar shall be kept burning thereby.}{\arabic{verse}}
\rashi{\rashiDH{צו את אהרן.} אין צו אלה לשון זרוז מיד ולדורות אר״ש ביותר צריך הכתוב לזרז במקום שיש בו חסרון כיס׃\quad \rashiDH{זאת תורת העולה וגו׳.} הרי הענין הזה בא ללמד על הקטר חלבים ואיברים שיהא כשר כל הלילה (מגילה כא.), וללמד על הפסולין איזה אם עלה ירד, ואיזה אם עלה לא ירד, שכל תורה לרבות הוא בא, לומר תורה אחת לכל העולים, ואפילו פסולין, שאם עלו לא ירדו׃\quad \rashiDH{הוא העולה.} למעט את הרובע ואת הנרבע, וכיוצא בהן, שלא היה פסולן בקדש שנפסלו קודם שבאו לעזרה (כל הענין בת״כ (פרשתא א, ח) וזבחים פ״ג ופ״ד)׃ 
}
\threeverse{\arabic{verse}}%Leviticus6:3
{וְלָבַ֨שׁ הַכֹּהֵ֜ן מִדּ֣וֹ בַ֗ד וּמִֽכְנְסֵי\maqqaf בַד֮ יִלְבַּ֣שׁ עַל\maqqaf בְּשָׂרוֹ֒ וְהֵרִ֣ים אֶת\maqqaf הַדֶּ֗שֶׁן אֲשֶׁ֨ר תֹּאכַ֥ל הָאֵ֛שׁ אֶת\maqqaf הָעֹלָ֖ה עַל\maqqaf הַמִּזְבֵּ֑חַ וְשָׂמ֕וֹ אֵ֖צֶל הַמִּזְבֵּֽחַ׃}
{וְיִלְבַּשׁ כָּהֲנָא לְבוּשִׁין דְּבוּץ וּמִכְנְסִין דְּבוּץ יִלְבַּשׁ עַל בִּסְרֵיהּ וְיַפְרֵישׁ יָת דִּשְׁנָא דְּתֵיכוֹל אִישָׁתָא יָת עֲלָתָא עַל מַדְבְּחָא וִישַׁוֵּינֵיהּ בִּסְטַר מַדְבְּחָא׃}
{And the priest shall put on his linen garment, and his linen breeches shall he put upon his flesh; and he shall take up the ashes whereto the fire hath consumed the burnt-offering on the altar, and he shall put them beside the altar.}{\arabic{verse}}
\rashi{\rashiDH{מדו בד.} היא הכתונת, ומה ת״ל מדו, שתהא כמדתו (ת״כ פרק ב, א)׃\quad \rashiDH{על בשרו.} שלא יהא דבר חוצץ בנתים (ערכין ג׃)׃\quad \rashiDH{והרים את הדשן.} היה חותה מלא המחתה מן המאוכלות הפנימיות ונותנן במזרחו של כבש (תמיד כח׃)׃\quad \rashiDH{הדשן אשר תאכל האש את העולה.} ועשאתה דשן, מאותו דשן ירים תרומה׃\quad \rashiDH{ושמו אצל המזבח.} (על המזבח. מצא אברים שעדיין לא נתאכלו, מחזירן על המזבח לאחר שחתה גחלים אילך ואילך ונטל מן הפנימיות שנאמר את העולה על המזבח (יומא מה.))׃ 
}
\threeverse{\aliya{לוי}}%Leviticus6:4
{וּפָשַׁט֙ אֶת\maqqaf בְּגָדָ֔יו וְלָבַ֖שׁ בְּגָדִ֣ים אֲחֵרִ֑ים וְהוֹצִ֤יא אֶת\maqqaf הַדֶּ֙שֶׁן֙ אֶל\maqqaf מִח֣וּץ לַֽמַּחֲנֶ֔ה אֶל\maqqaf מָק֖וֹם טָהֽוֹר׃}
{וְיַשְׁלַח יָת לְבוּשׁוֹהִי וְיִלְבַּשׁ לְבוּשִׁין אָחֳרָנִין וְיַפֵּיק יָת דִּשְׁנָא לְמִבַּרָא לְמַשְׁרִיתָא לַאֲתַר דְּכֵי׃}
{And he shall put off his garments, and put on other garments, and carry forth the ashes without the camp unto a clean place.}{\arabic{verse}}
\rashi{\rashiDH{ופשט את בגדיו.} אין זו חובה אלא דרך ארץ, שלא ילכלך בהוצאת הדשן בגדים שהוא משמש בהן תמיד, בגדים שבשל בהן קדרה לרבו אל ימזוג בהן כוס לרבו, לכך ולבש בגדים אחרים, פחותין מהן (יומא כג׃)׃\quad \rashiDH{והוציא את הדשן.} הַצָּבוּר בַּתַּפּוּחַ כשהוא רָבָה ואין מקום למערכה, מוציאו משם (תמיד כח׃), ואין זה חובה בכל יום, אבל התרומה חובה בכל יום׃}
\threeverse{\arabic{verse}}%Leviticus6:5
{וְהָאֵ֨שׁ עַל\maqqaf הַמִּזְבֵּ֤חַ תּֽוּקַד\maqqaf בּוֹ֙ לֹ֣א תִכְבֶּ֔ה וּבִעֵ֨ר עָלֶ֧יהָ הַכֹּהֵ֛ן עֵצִ֖ים בַּבֹּ֣קֶר בַּבֹּ֑קֶר וְעָרַ֤ךְ עָלֶ֙יהָ֙ הָֽעֹלָ֔ה וְהִקְטִ֥יר עָלֶ֖יהָ חֶלְבֵ֥י הַשְּׁלָמִֽים׃}
{וְאִישָׁתָא עַל מַדְבְּחָא תְהֵי יָקְדָא בֵיהּ לָא תִטְפֵי וְיַבְעַר עֲלַהּ כָּהֲנָא אָעִין בִּצְפַר בִּצְפַר וְיַסְדַּר עֲלַהּ עֲלָתָא וְיַסֵּיק עֲלַהּ תַּרְבֵּי נִכְסַת קוּדְשַׁיָּא׃}
{And the fire upon the altar shall be kept burning thereby, it shall not go out; and the priest shall kindle wood on it every morning; and he shall lay the burnt-offering in order upon it, and shall make smoke thereon the fat of the peace-offerings.}{\arabic{verse}}
\rashi{\rashiDH{והאש על המזבח תוקד בו}.ריבה כאן יקידות הרבה, על מוקדה, ואש המזבח תוקד בו, והאש על המזבח תוקד בו, אש תמיד תוקד על המזבח, כולן נדרשו במס׳ יומא (מה.) שנחלקו רבותינו במנין המערכות שהיו שם׃ 
\quad \rashiDH{וערך עליה העולה.} עולת תמיד היא תקדים (פסחים נח׃). (בר״י ומנין שלא יהא דבר קודם על המערכה לתמיד של שחר, תלמוד לומר העולה עולה ראשונה)\quad \rashiDH{חלבי השלמים.} אם יביאו שם שלמים. ורבותינו (שם נט.) למדו מכאן עליה, על עולת הבוקר השלם כל הקרבנות כולם, מכאן שלא יהא דבר מאוחר לתמיד של בין הערבים׃}
\threeverse{\arabic{verse}}%Leviticus6:6
{אֵ֗שׁ תָּמִ֛יד תּוּקַ֥ד עַל\maqqaf הַמִּזְבֵּ֖חַ לֹ֥א תִכְבֶּֽה׃ \setuma }
{אִישָׁתָא תְּדִירָא תְּהֵי יָקְדָא עַל מַדְבְּחָא לָא תִטְפֵי׃}
{Fire shall be kept burning upon the altar continually; it shall not go out.}{\arabic{verse}}
\rashi{\rashiDH{אש תמיד.} אש שנאמר בה תמיד היא שמדליקין בה את הנרות שנאמר בה לְהַעֲלֹת נֵר תָּמִיד (שמות כז, כ), אף היא מעל המזבח החיצון תוקד (יומא מה׃)׃\quad \rashiDH{לא תכבה.} המכבה אש על המזבח עובר בשני לאוין׃}
\threeverse{\aliya{ישראל}}%Leviticus6:7
{וְזֹ֥את תּוֹרַ֖ת הַמִּנְחָ֑ה הַקְרֵ֨ב אֹתָ֤הּ בְּנֵֽי\maqqaf אַהֲרֹן֙ לִפְנֵ֣י יְהֹוָ֔ה אֶל\maqqaf פְּנֵ֖י הַמִּזְבֵּֽחַ׃}
{וְדָא אוֹרָיְתָא דְּמִנְחָתָא דִּיקָרְבוּן יָתַהּ בְּנֵי אַהֲרֹן קֳדָם יְיָ לִקְדָם מַדְבְּחָא׃}
{And this is the law of the meal-offering: the sons of Aaron shall offer it before the \lord, in front of the altar.}{\arabic{verse}}
\rashi{\rashiDH{וזאת תורת המנחה.} תורה אחת לכולן להטעינן שמן ולבונה האמורין בענין. שיכול אין לי טעונות שמן ולבונה אלא מנחת ישראל שהיא נקמצת, מנחת כהנים שהיא כליל מנין, תלמוד לומר תורת׃\quad \rashiDH{הקרב אותה.} היא הגשה בקרן דרומית מערבית׃\quad \rashiDH{לפני ה׳.} הוא מערב, שהוא לצד אהל מועד׃\quad \rashiDH{אל פני המזבח.} הוא הדרום שהוא פניו של מזבח, שהכבש נתון לאותו הרוח׃}
\threeverse{\arabic{verse}}%Leviticus6:8
{וְהֵרִ֨ים מִמֶּ֜נּוּ בְּקֻמְצ֗וֹ מִסֹּ֤לֶת הַמִּנְחָה֙ וּמִשַּׁמְנָ֔הּ וְאֵת֙ כׇּל\maqqaf הַלְּבֹנָ֔ה אֲשֶׁ֖ר עַל\maqqaf הַמִּנְחָ֑ה וְהִקְטִ֣יר הַמִּזְבֵּ֗חַ רֵ֧יחַ נִיחֹ֛חַ אַזְכָּרָתָ֖הּ לַיהֹוָֽה׃}
{וְיַפְרֵישׁ מִנַּהּ בְּקֻמְצֵיהּ מִסֻּלְתָּא דְּמִנְחָתָא וּמִמִּשְׁחַהּ וְיָת כָּל לְבוֹנְתָא דְּעַל מִנְחָתָא וְיַסֵּיק לְמַדְבְּחָא לְאִתְקַבָּלָא בְרַעֲוָא אַדְכָרְתַהּ קֳדָם יְיָ׃}
{And he shall take up therefrom his handful, of the fine flour of the meal-offering, and of the oil thereof, and all the frankincense which is upon the meal-offering, and shall make the memorial-part thereof smoke upon the altar for a sweet savour unto the \lord.}{\arabic{verse}}
\rashi{\rashiDH{(והרים ממנו.} מהמחובר, שיהא עשרון שלם בבת אחת בשעת קמיצה. בר״י) 
\quad \rashiDH{בקמצו.} שלא יעשה מדה לקומץ (יומא מז.)׃\quad \rashiDH{מסלת המנחה ומשמנה.} מכאן שקומץ ממקום שנתרבה שמנה (סוטה יד׃)׃\quad \rashiDH{המנחה.} שלא תהא מעורבת באחרת (ת״כ פרשתא ב, ה)׃\quad \rashiDH{ואת כל הלבונה אשר על המנחה והקטיר.} שמלקט את לבונתה לאחר קמיצה ומקטירו, (ת״כ שם) ולפי שלא פירש כן אלא באחת מן המנחות בויקרא (לעיל ב, ב), הוצרך לשנות פרשה זו לכלול כל המנחות כמשפטן. 
}
\threeverse{\arabic{verse}}%Leviticus6:9
{וְהַנּוֹתֶ֣רֶת מִמֶּ֔נָּה יֹאכְל֖וּ אַהֲרֹ֣ן וּבָנָ֑יו מַצּ֤וֹת תֵּֽאָכֵל֙ בְּמָק֣וֹם קָדֹ֔שׁ בַּחֲצַ֥ר אֹֽהֶל\maqqaf מוֹעֵ֖ד יֹאכְלֽוּהָ׃}
{וּדְיִשְׁתְּאַר מִנַּהּ יֵיכְלוּן אַהֲרֹן וּבְנוֹהִי פַּטִּיר תִּתְאֲכִיל בַּאֲתַר קַדִּישׁ בְּדָרַת מַשְׁכַּן זִמְנָא יֵיכְלוּנַּהּ׃}
{And that which is left thereof shall Aaron and his sons eat; it shall be eaten without leaven in a holy place; in the court of the tent of meeting they shall eat it.}{\arabic{verse}}
\rashi{\rashiDH{במקום קדש.} ואיזהו, בחצר אהל מועד׃}
\threeverse{\arabic{verse}}%Leviticus6:10
{לֹ֤א תֵאָפֶה֙ חָמֵ֔ץ חֶלְקָ֛ם נָתַ֥תִּי אֹתָ֖הּ מֵאִשָּׁ֑י קֹ֤דֶשׁ קׇֽדָשִׁים֙ הִ֔וא כַּחַטָּ֖את וְכָאָשָֽׁם׃}
{לָא תִתְאֲפֵי חֲמִיעַ חוּלָקְהוֹן יְהַבִית יָתַהּ מִקּוּרְבָּנָי קֹדֶשׁ קוּדְשִׁין הִיא כְּחַטָּתָא וְכַאֲשָׁמָא׃}
{It shall not be baked with leaven. I have given it as their portion of My offerings made by fire; it is most holy, as the sin-offering, and as the guilt-offering.}{\arabic{verse}}
\rashi{\rashiDH{לא תאפה חמץ חלקם.} אף הַשִּׁירַיִם אסורים בחמץ (מנחות נה.)׃\quad \rashiDH{כחטאת וכאשם.} מנחת חוטא הרי היא כחטאת, לפיכך קמצה שלא לשמה פסולה, מנחת נדבה הרי היא כאשם, לפיכך קמצה שלא לשמה כשרה (ת״כ פרק ג, ד)׃}
\threeverse{\arabic{verse}}%Leviticus6:11
{כׇּל\maqqaf זָכָ֞ר בִּבְנֵ֤י אַהֲרֹן֙ יֹֽאכְלֶ֔נָּה חׇק\maqqaf עוֹלָם֙ לְדֹרֹ֣תֵיכֶ֔ם מֵאִשֵּׁ֖י יְהֹוָ֑ה כֹּ֛ל אֲשֶׁר\maqqaf יִגַּ֥ע בָּהֶ֖ם יִקְדָּֽשׁ׃ \petucha }
{כָּל דְּכוּרָא בִּבְנֵי אַהֲרֹן יֵיכְלִנַּהּ קְיָם עָלַם לְדָרֵיכוֹן מִקּוּרְבָּנַיָּא דַּייָ כֹּל דְּיִקְרַב בְּהוֹן יִתְקַדַּשׁ׃}
{Every male among the children of Aaron may eat of it, as a due for ever throughout your generations, from the offerings of the \lord\space made by fire; whatsoever toucheth them shall be holy.}{\arabic{verse}}
\rashi{\rashiDH{כל זכר.} אפילו בעל מום. למה נאמר, אם לאכילה, הרי כבר אמור לֶחֶם אֱלֹהָיו מִקָּדְשֵׁי הַקֳּדָשִׁים וגו׳ (ויקרא כא, כב), אלא לרבות בעלי מומין למחלוקת (זבחים קב.)׃\quad \rashiDH{כל אשר יגע וגו׳.} קדשים קלים או חולין שיגעו בה ויבלעו ממנה (ת״כ שם ו)׃\quad \rashiDH{יקדש.} להיות כמוה, שאם פסולה יפסלו, ואם כשרה יאכלו כחומר המנחה (שם  זבחים צז׃)׃ 
}
\aliyacounter{שני}
\newseder{3}
\threeverse{\aliya{שני}\newline\vspace{-4pt}\newline\seder{ג}}%Leviticus6:12
{וַיְדַבֵּ֥ר יְהֹוָ֖ה אֶל\maqqaf מֹשֶׁ֥ה לֵּאמֹֽר׃}
{וּמַלֵּיל יְיָ עִם מֹשֶׁה לְמֵימַר׃}
{And the \lord\space spoke unto Moses, saying:}{\arabic{verse}}
\threeverse{\arabic{verse}}%Leviticus6:13
{זֶ֡ה קׇרְבַּן֩ אַהֲרֹ֨ן וּבָנָ֜יו אֲשֶׁר\maqqaf יַקְרִ֣יבוּ לַֽיהֹוָ֗ה בְּיוֹם֙ הִמָּשַׁ֣ח אֹת֔וֹ עֲשִׂירִ֨ת הָאֵפָ֥ה סֹ֛לֶת מִנְחָ֖ה תָּמִ֑יד מַחֲצִיתָ֣הּ בַּבֹּ֔קֶר וּמַחֲצִיתָ֖הּ בָּעָֽרֶב׃}
{דֵּין קוּרְבָּנָא דְּאַהֲרֹן וְדִבְנוֹהִי דִּיקָרְבוּן קֳדָם יְיָ בְּיוֹמָא דִּירַבּוֹן יָתֵיהּ חַד מִן עַסְרָא בִּתְלָת סְאִין סוּלְתָּא מִנְחָתָא תְּדִירָא פַּלְגוּתַהּ בְּצַפְרָא וּפַלְגוּתַהּ בְּרַמְשָׁא׃}
{This is the offering of Aaron and of his sons, which they shall offer unto the \lord\space in the day when he is anointed: the tenth part of an ephah of fine flour for a meal-offering perpetually, half of it in the morning, and half thereof in the evening.}{\arabic{verse}}
\rashi{\rashiDH{זה קרבן אהרן ובניו.} אף ההדיוטות מקריבין עשירית האיפה ביום שהן מתחנכין לעבודה, אבל כהן גדול בכל יום, שנאמר מנחה תמיד וגו׳, והכהן המשיח תחתיו מבניו וגו׳ חק חקת עולם וגו׳ (מנחות נא׃)׃}
\threeverse{\arabic{verse}}%Leviticus6:14
{עַֽל\maqqaf מַחֲבַ֗ת בַּשֶּׁ֛מֶן תֵּעָשֶׂ֖ה מֻרְבֶּ֣כֶת תְּבִיאֶ֑נָּה תֻּפִינֵי֙ מִנְחַ֣ת פִּתִּ֔ים תַּקְרִ֥יב רֵֽיחַ\maqqaf נִיחֹ֖חַ לַיהֹוָֽה׃}
{עַל מַסְרֵיתָא בִּמְשַׁח תִּתְעֲבֵיד רְבִיכָא תַּיְתֵינַהּ תּוּפִינֵי מִנְחַת בִּצּוּעִין תְּקָרֵיב לְאִתְקַבָּלָא בְרַעֲוָא קֳדָם יְיָ׃}
{On a griddle it shall be made with oil; when it is soaked, thou shalt bring it in; in broken pieces shalt thou offer the meal-offering for a sweet savour unto the \lord.}{\arabic{verse}}
\rashi{\rashiDH{מרבכת.} חלוטה ברותחין כל צרכה (ת״כ פרק ד, ה)׃\quad \rashiDH{תפיני.} אפויה אפיות הרבה, שאחר חליטתה אופה בתנור, וחוזר ומטגנה במחבת (מנחות נ׃)׃\quad \rashiDH{מנחת פתים.} מלמד שטעונה פתיתה׃}
\threeverse{\arabic{verse}}%Leviticus6:15
{וְהַכֹּהֵ֨ן הַמָּשִׁ֧יחַ תַּחְתָּ֛יו מִבָּנָ֖יו יַעֲשֶׂ֣ה אֹתָ֑הּ חׇק\maqqaf עוֹלָ֕ם לַיהֹוָ֖ה כָּלִ֥יל תׇּקְטָֽר׃}
{וְכָהֲנָא דְּיִתְרַבָּא תְּחוֹתוֹהִי מִבְּנוֹהִי יַעֲבֵיד יָתַהּ קְיָם עָלַם קֳדָם יְיָ גְּמִיר תִּתַּסַּק׃}
{And the anointed priest that shall be in his stead from among his sons shall offer it, it is a due for ever; it shall be wholly made to smoke unto the \lord.}{\arabic{verse}}
\rashi{\rashiDH{המשיח תחתיו מבניו.} המשיח מבניו תחתיו׃\quad \rashiDH{כליל תקטר.} אין נקמצת להיות שיריה נאכלין, אלא כולה כליל, וכן כל מנחת כהן של נדבה כליל תהיה׃}
\threeverse{\arabic{verse}}%Leviticus6:16
{וְכׇל\maqqaf מִנְחַ֥ת כֹּהֵ֛ן כָּלִ֥יל תִּהְיֶ֖ה לֹ֥א תֵאָכֵֽל׃ \petucha }
{וְכָל מִנְחָתָא דְּכָהֲנָא גְּמִיר תְּהֵי לָא תִתְאֲכִיל׃}
{And every meal-offering of the priest shall be wholly made to smoke; it shall not be eaten.}{\arabic{verse}}
\rashi{\rashiDH{כליל.} כולה שוה לגבוה׃ 
}
\threeverse{\arabic{verse}}%Leviticus6:17
{וַיְדַבֵּ֥ר יְהֹוָ֖ה אֶל\maqqaf מֹשֶׁ֥ה לֵּאמֹֽר׃}
{וּמַלֵּיל יְיָ עִם מֹשֶׁה לְמֵימַר׃}
{And the \lord\space spoke unto Moses, saying:}{\arabic{verse}}
\threeverse{\arabic{verse}}%Leviticus6:18
{דַּבֵּ֤ר אֶֽל\maqqaf אַהֲרֹן֙ וְאֶל\maqqaf בָּנָ֣יו לֵאמֹ֔ר זֹ֥את תּוֹרַ֖ת הַֽחַטָּ֑את בִּמְק֡וֹם אֲשֶׁר֩ תִּשָּׁחֵ֨ט הָעֹלָ֜ה תִּשָּׁחֵ֤ט הַֽחַטָּאת֙ לִפְנֵ֣י יְהֹוָ֔ה קֹ֥דֶשׁ קׇֽדָשִׁ֖ים הִֽוא׃}
{מַלֵּיל עִם אַהֲרֹן וְעִם בְּנוֹהִי לְמֵימַר דָּא אוֹרָיְתָא דְּחַטָּתָא בְּאַתְרָא דְּתִתְנְכֵיס עֲלָתָא תִּתְנְכֵיס חַטָּתָא קֳדָם יְיָ קֹדֶשׁ קוּדְשִׁין הִיא׃}
{Speak unto Aaron and to his sons, saying: This is the law of the sin-offering: in the place where the burnt-offering is killed shall the sin-offering be killed before the \lord; it is most holy.}{\arabic{verse}}
\threeverse{\arabic{verse}}%Leviticus6:19
{הַכֹּהֵ֛ן הַֽמְחַטֵּ֥א אֹתָ֖הּ יֹאכְלֶ֑נָּה בְּמָק֤וֹם קָדֹשׁ֙ תֵּֽאָכֵ֔ל בַּחֲצַ֖ר אֹ֥הֶל מוֹעֵֽד׃}
{כָּהֲנָא דִּמְכַפַּר בִּדְמַהּ יֵיכְלִנַּהּ בַּאֲתַר קַדִּישׁ תִּתְאֲכִיל בְּדָרַת מַשְׁכַּן זִמְנָא׃}
{The priest that offereth it for sin shall eat it; in a holy place shall it be eaten, in the court of the tent of meeting.}{\arabic{verse}}
\rashi{\rashiDH{המחטא אותה.} העובד עבודותיה, שהיא נעשית חטאת על ידו׃\quad \rashiDH{המחטא אותה יאכלנה.} הראוי לעבודה, יצא טמא בשעת זריקת דמים שאינו חולק בבשר (זבחים צט.), ואי אפשר לומר שאוסר שאר כהנים באכילתה חוץ מן הזורק דמה, שהרי נאמר למטה כל זכר בכהנים יאכל אותה׃}
\threeverse{\arabic{verse}}%Leviticus6:20
{כֹּ֛ל אֲשֶׁר\maqqaf יִגַּ֥ע בִּבְשָׂרָ֖הּ יִקְדָּ֑שׁ וַאֲשֶׁ֨ר יִזֶּ֤ה מִדָּמָהּ֙ עַל\maqqaf הַבֶּ֔גֶד אֲשֶׁר֙ יִזֶּ֣ה עָלֶ֔יהָ תְּכַבֵּ֖ס בְּמָק֥וֹם קָדֹֽשׁ׃}
{כֹּל דְּיִקְרַב בְּבִשְׂרַהּ יִתְקַדַּשׁ וּדְיַדֵּי מִדְּמַהּ עַל לְבוּשָׁא דְּיַדֵּי עֲלַהּ תְּחַוַּר בַּאֲתַר קַדִּישׁ׃}
{Whatsoever shall touch the flesh thereof shall be holy; and when there is sprinkled of the blood thereof upon any garment, thou shalt wash that whereon it was sprinkled in a holy place.}{\arabic{verse}}
\rashi{\rashiDH{כל אשר יגע בבשרה.} כל דבר אוכל אשר יגע ויבלע ממנה׃\quad \rashiDH{יקדש.} להיות כמוה, אם פסולה תפסל, ואם היא כשרה תאכל כחומר שבה (שם צז׃)׃\quad \rashiDH{ואשר יזה מדמה על הבגד.} ואם הוזה מדמה על הבגד אותו מקום דם (הבגד אשר יזה עליה) תכבס בתוך העזרה (ת״כ פרק ו, ז)׃\quad \rashiDH{אשר יזה.} יהא נזה כמו וְלֹא יִטֶּה לָאָרֶץ מִנְלָם (איוב טו, כט), יהא נטוי׃}
\threeverse{\arabic{verse}}%Leviticus6:21
{וּכְלִי\maqqaf חֶ֛רֶשׂ אֲשֶׁ֥ר תְּבֻשַּׁל\maqqaf בּ֖וֹ יִשָּׁבֵ֑ר וְאִם\maqqaf בִּכְלִ֤י נְחֹ֙שֶׁת֙ בֻּשָּׁ֔לָה וּמֹרַ֥ק וְשֻׁטַּ֖ף בַּמָּֽיִם׃}
{וּמָן דַּחֲסַף דְּתִתְבַּשַּׁל בֵּיהּ יִתְּבַר וְאִם בְּמָנָא דִּנְחָשָׁא תִּתְבַּשַּׁל וְיִתְמְרֵיק וְיִשְׁתְּטֵיף בְּמַיָּא׃}
{But the earthen vessel wherein it is sodden shall be broken; and if it be sodden in a brazen vessel, it shall be scoured, and rinsed in water.}{\arabic{verse}}
\rashi{\rashiDH{ישבר.} לפי שהבליעה שנבלעת בו נעשה נותר והוא הדין לכל הקדשים׃\quad \rashiDH{ומרק.} לשון תַּמְרוּקֵי הַנָּשִׁים (אסתר ב, יב) אשקורי״ר בלע״ז׃\quad \rashiDH{ומרק ושטף.} לפלוט את בליעתו, אבל כלי חרס למדך הכתוב כאן שאינו יוצא מידי דפיו לעולם (פסחים ל׃)׃ 
}
\threeverse{\arabic{verse}}%Leviticus6:22
{כׇּל\maqqaf זָכָ֥ר בַּכֹּהֲנִ֖ים יֹאכַ֣ל אֹתָ֑הּ קֹ֥דֶשׁ קׇֽדָשִׁ֖ים הִֽוא׃}
{כָּל דְּכוּרָא בְּכָהֲנַיָּא יֵיכוֹל יָתַהּ קֹדֶשׁ קוּדְשִׁין הִיא׃}
{Every male among the priests may eat thereof; it is most holy.}{\arabic{verse}}
\rashi{\rashiDH{כל זכר בכהנים יאכל אותה.} הא למדת שֶׁהַמְחַטֵּא אותה האמור למעלה לא להוציא שאר הכהנים, אלא להוציא את שאינו ראוי לְחִטּוּי׃}
\threeverse{\arabic{verse}}%Leviticus6:23
{וְכׇל\maqqaf חַטָּ֡את אֲשֶׁר֩ יוּבָ֨א מִדָּמָ֜הּ אֶל\maqqaf אֹ֧הֶל מוֹעֵ֛ד לְכַפֵּ֥ר בַּקֹּ֖דֶשׁ לֹ֣א תֵאָכֵ֑ל בָּאֵ֖שׁ תִּשָּׂרֵֽף׃ \petucha }
{וְכָל חַטָּתָא דְּיִתָּעַל מִדְּמַהּ לְמַשְׁכַּן זִמְנָא לְכַפָּרָא בְּקוּדְשָׁא לָא תִתְאֲכִיל בְּנוּרָא תִּתּוֹקַד׃}
{And no sin-offering, whereof any of the blood is brought into the tent of meeting to make atonement in the holy place, shall be eaten; it shall be burnt with fire.}{\arabic{verse}}
\rashi{\rashiDH{וכל חטאת וגו׳.} שאם הכניס מדם חטאת החיצונה לפנים פסולה (זבחים פב.)׃\quad \rashiDH{וכל.} לרבות שאר קדשים׃}
\newperek
\threeverse{\Roman{chap}}%Leviticus7:1
{וְזֹ֥את תּוֹרַ֖ת הָאָשָׁ֑ם קֹ֥דֶשׁ קׇֽדָשִׁ֖ים הֽוּא׃}
{וְדָא אוֹרָיְתָא דַּאֲשָׁמָא קֹדֶשׁ קוּדְשִׁין הוּא׃}
{And this is the law of the guilt-offering: it is most holy.}{\Roman{chap}}
\rashi{\rashiDH{קדש קדשים הוא.} הוא קרב ואין תמורתו קרבה (תמורה יז׃  ת״כ פרשתא ד, ב)׃}
\threeverse{\arabic{verse}}%Leviticus7:2
{בִּמְק֗וֹם אֲשֶׁ֤ר יִשְׁחֲטוּ֙ אֶת\maqqaf הָ֣עֹלָ֔ה יִשְׁחֲט֖וּ אֶת\maqqaf הָאָשָׁ֑ם וְאֶת\maqqaf דָּמ֛וֹ יִזְרֹ֥ק עַל\maqqaf הַמִּזְבֵּ֖חַ סָבִֽיב׃}
{בְּאַתְרָא דְּיִכְּסוּן יָת עֲלָתָא יִכְּסוּן יָת אֲשָׁמָא וְיָת דְּמֵיהּ יִזְרוֹק עַל מַדְבְּחָא סְחוֹר סְחוֹר׃}
{In the place where they kill the burnt-offering shall they kill the guilt-offering: and the blood thereof shall be dashed against the altar round about.}{\arabic{verse}}
\rashi{\rashiDH{ישחטו.} ריבה לנו שחיטות הרבה, לפי שמצינו אשם בצבור נאמר ישחטו, רבים, ותלאו בעולה להביא עולת צבור לצפון׃}
\threeverse{\arabic{verse}}%Leviticus7:3
{וְאֵ֥ת כׇּל\maqqaf חֶלְבּ֖וֹ יַקְרִ֣יב מִמֶּ֑נּוּ אֵ֚ת הָֽאַלְיָ֔ה וְאֶת\maqqaf הַחֵ֖לֶב הַֽמְכַסֶּ֥ה אֶת\maqqaf הַקֶּֽרֶב׃}
{וְיָת כָּל תַּרְבֵּיהּ יְקָרֵיב מִנֵּיהּ יָת אַלְיְתָא וְיָת תַּרְבָּא דְּחָפֵי יָת גַּוָּא׃}
{And he shall offer of it all the fat thereof: the fat tail, and the fat that covereth the inwards,}{\arabic{verse}}
\rashi{\rashiDH{ואת כל חלבו וגו׳.} עד כאן לא נתפרשו אמורין באשם לכך הוצרך לפרשם כאן, אבל חטאת כבר נתפרשו בה בפרשת ויקרא׃\quad \rashiDH{את האליה.} לפי שאשם אינו בא אלא איל או כבש, ואיל וכבש נתרבו באליה׃}
\threeverse{\arabic{verse}}%Leviticus7:4
{וְאֵת֙ שְׁתֵּ֣י הַכְּלָיֹ֔ת וְאֶת\maqqaf הַחֵ֙לֶב֙ אֲשֶׁ֣ר עֲלֵיהֶ֔ן אֲשֶׁ֖ר עַל\maqqaf הַכְּסָלִ֑ים וְאֶת\maqqaf הַיֹּתֶ֙רֶת֙ עַל\maqqaf הַכָּבֵ֔ד עַל\maqqaf הַכְּלָיֹ֖ת יְסִירֶֽנָּה׃}
{וְיָת תַּרְתֵּין כּוֹלְיָן וְיָת תַּרְבָּא דַּעֲלֵיהוֹן דְּעַל גִּסְסַיָּא וְיָת חַצְרָא דְּעַל כַּבְדָּא עַל כּוֹלְיָתָא יַעְדֵּינַהּ׃}
{and the two kidneys, and the fat that is on them, which is by the loins, and the lobe above the liver, which he shall take away by the kidneys.}{\arabic{verse}}
\threeverse{\arabic{verse}}%Leviticus7:5
{וְהִקְטִ֨יר אֹתָ֤ם הַכֹּהֵן֙ הַמִּזְבֵּ֔חָה אִשֶּׁ֖ה לַיהֹוָ֑ה אָשָׁ֖ם הֽוּא׃}
{וְיַסֵּיק יָתְהוֹן כָּהֲנָא לְמַדְבְּחָא קוּרְבָּנָא קֳדָם יְיָ אֲשָׁמָא הוּא׃}
{And the priest shall make them smoke upon the altar for an offering made by fire unto the \lord; it is a guilt-offering.}{\arabic{verse}}
\rashi{\rashiDH{אשם הוא.} עד שינתק שמו ממנו, לימד על אשם שמתו בעליו או שנתכפרו בעליו אף על פי שעומד להיות דמיו עולה לקיץ המזבח, אם שחטו סתם אינו כשר לעולה קודם שנתק לרעיה. ואינו בא ללמד על האשם שיהא פסול שלא לשמו, כמו שדרשו, הוא הכתוב בחטאת, לפי שאשם לא נאמר בו אשם הוא, אלא לאחר הקטרת אמורין, והוא עצמו שלא הוקטרו אמוריו כשר (זבחים ה׃ וע״ש ברש״י)׃}
\threeverse{\arabic{verse}}%Leviticus7:6
{כׇּל\maqqaf זָכָ֥ר בַּכֹּהֲנִ֖ים יֹאכְלֶ֑נּוּ בְּמָק֤וֹם קָדוֹשׁ֙ יֵאָכֵ֔ל קֹ֥דֶשׁ קׇֽדָשִׁ֖ים הֽוּא׃}
{כָּל דְּכוּרָא בְּכָהֲנַיָּא יֵיכְלִנֵּיהּ בַּאֲתַר קַדִּישׁ יִתְאֲכִיל קֹדֶשׁ קוּדְשִׁין הוּא׃}
{Every male among the priests may eat thereof; it shall be eaten in a holy place; it is most holy.}{\arabic{verse}}
\rashi{\rashiDH{קדש קדשים הוא.} בתורת כהנים הוא נדרש (פרשתא ה, י)׃ 
}
\threeverse{\arabic{verse}}%Leviticus7:7
{כַּֽחַטָּאת֙ כָּֽאָשָׁ֔ם תּוֹרָ֥ה אַחַ֖ת לָהֶ֑ם הַכֹּהֵ֛ן אֲשֶׁ֥ר יְכַפֶּר\maqqaf בּ֖וֹ ל֥וֹ יִהְיֶֽה׃}
{כְּחַטָּתָא כֵּן אֲשָׁמָא אוֹרָיְתָא חֲדָא לְהוֹן כָּהֲנָא דִּיכַפַּר בֵּיהּ דִּילֵיהּ יְהֵי׃}
{As is the sin-offering, so is the guilt-offering; there is one law for them; the priest that maketh atonement therewith, he shall have it.}{\arabic{verse}}
\rashi{\rashiDH{תורה אחת להם.} בדבר זה׃\quad \rashiDH{הכהן אשר יכפר בו.} הראוי לכפרה חולק בו, פרט לטבול יום ומחוסר כפורים ואונן׃}
\threeverse{\arabic{verse}}%Leviticus7:8
{וְהַ֨כֹּהֵ֔ן הַמַּקְרִ֖יב אֶת\maqqaf עֹ֣לַת אִ֑ישׁ ע֤וֹר הָֽעֹלָה֙ אֲשֶׁ֣ר הִקְרִ֔יב לַכֹּהֵ֖ן ל֥וֹ יִהְיֶֽה׃}
{וְכָהֲנָא דִּמְקָרֵיב יָת עֲלַת גְּבַר מְשַׁךְ עֲלָתָא דִּיקָרֵיב לְכָהֲנָא דִּילֵיהּ יְהֵי׃}
{And the priest that offereth any man’s burnt-offering, even the priest shall have to himself the skin of the burnt-offering which he hath offered.}{\arabic{verse}}
\rashi{\rashiDH{עור העולה אשר הקריב לכהן לו יהיה.} פרט לטבול יום ומחוסר כפורים ואונן שאינן חולקים בעורות (זבחים קג׃)׃}
\threeverse{\arabic{verse}}%Leviticus7:9
{וְכׇל\maqqaf מִנְחָ֗ה אֲשֶׁ֤ר תֵּֽאָפֶה֙ בַּתַּנּ֔וּר וְכׇל\maqqaf נַעֲשָׂ֥ה בַמַּרְחֶ֖שֶׁת וְעַֽל\maqqaf מַחֲבַ֑ת לַכֹּהֵ֛ן הַמַּקְרִ֥יב אֹתָ֖הּ ל֥וֹ תִֽהְיֶֽה׃}
{וְכָל מִנְחָתָא דְּתִתְאֲפֵי בְּתַנּוּרָא וְכָל דְּתִתְעֲבֵיד בְּרָדְתָא וְעַל מַסְרֵיתָא לְכָהֲנָא דִּמְקָרֵיב יָתַהּ דִּילֵיהּ תְּהֵי׃}
{And every meal-offering that is baked in the oven, and all that is dressed in the stewing-pan, and on the griddle, shall be the priest’s that offereth it.}{\arabic{verse}}
\rashi{\rashiDH{לכהן המקריב אתה וגו׳.} יכול לו לבדו, תלמוד לומר לכל בני אהרן תהיה, יכול לכולן, תלמוד לומר לכהן המקריב, הא כיצד לבית אב של אותו יום שמקריבין אותה (ת״כ פרק י, ג)׃ 
}
\threeverse{\arabic{verse}}%Leviticus7:10
{וְכׇל\maqqaf מִנְחָ֥ה בְלוּלָֽה\maqqaf בַשֶּׁ֖מֶן וַחֲרֵבָ֑ה לְכׇל\maqqaf בְּנֵ֧י אַהֲרֹ֛ן תִּהְיֶ֖ה אִ֥ישׁ כְּאָחִֽיו׃ \petucha }
{וְכָל מִנְחָתָא דְּפִילָא בִמְשַׁח וּדְלָא פִילָא לְכָל בְּנֵי אַהֲרֹן תְּהֵי גְּבַר כַּאֲחוּהִי׃}
{And every meal-offering, mingled with oil, or dry, shall all the sons of Aaron have, one as well as another.}{\arabic{verse}}
\rashi{\rashiDH{בלולה בשמן.} זו מנחת נדבה׃\quad \rashiDH{וחרבה.} זו מנחת חוטא, ומנחת קנאות, שאין בהן שמן׃}
\aliyacounter{שלישי}
\threeverse{\aliya{שלישי}}%Leviticus7:11
{וְזֹ֥את תּוֹרַ֖ת זֶ֣בַח הַשְּׁלָמִ֑ים אֲשֶׁ֥ר יַקְרִ֖יב לַיהֹוָֽה׃}
{וְדָא אוֹרָיְתָא דְּנִכְסַת קוּדְשַׁיָּא דִּיקָרֵיב קֳדָם יְיָ׃}
{And this is the law of the sacrifice of peace-offerings, which one may offer unto the \lord.}{\arabic{verse}}
\threeverse{\arabic{verse}}%Leviticus7:12
{אִ֣ם עַל\maqqaf תּוֹדָה֮ יַקְרִיבֶ֒נּוּ֒ וְהִקְרִ֣יב \legarmeh  עַל\maqqaf זֶ֣בַח הַתּוֹדָ֗ה חַלּ֤וֹת מַצּוֹת֙ בְּלוּלֹ֣ת בַּשֶּׁ֔מֶן וּרְקִיקֵ֥י מַצּ֖וֹת מְשֻׁחִ֣ים בַּשָּׁ֑מֶן וְסֹ֣לֶת מֻרְבֶּ֔כֶת חַלֹּ֖ת בְּלוּלֹ֥ת בַּשָּֽׁמֶן׃}
{אִם עַל תּוֹדְתָא יְקָרְבִנֵּיהּ וִיקָרֵיב עַל נִכְסַת תּוֹדְתָא גְּרִיצָן פַּטִּירָן דְּפִילָן בִּמְשַׁח וְאֶסְפּוֹגִין פַּטִּירִין דִּמְשִׁיחִין בִּמְשַׁח וְסֹלֶת רְבִיכָא גְּרִיצָן דְּפִילָן בִּמְשַׁח׃}
{If he offer it for a thanksgiving, then he shall offer with the sacrifice of thanksgiving unleavened cakes mingled with oil, and unleavened wafers spread with oil, and cakes mingled with oil, of fine flour soaked.}{\arabic{verse}}
\rashi{\rashiDH{אם על תודה יקריבנו.} אם על דבר הודאה על נס שנעשה לו, כגון יורדי הים, והולכי מדברות, וחבושי בית האסורים, וחולה שנתרפא, שהם צריכין להודות שכתוב בהן יוֹדוּ לַה׳ חַסְדּוֹ וְנִפְלְאוֹתָיו לִבְנֵי אָדָם. וְיִזְבְּחוּ זִבְחֵי תוֹדָה (תהלים קז, כא־כב), אם על אחת מאלה נדר שלמים הללו, שלמי תודה הן, וטעונות לחם האמור בענין, ואינן נאכלין אלא ליום ולילה, כמו שמפורש כאן׃ 
\quad \rashiDH{והקריב על זבח התודה.} ד׳ מיני לחם, חלות, ורקיקין, ורבוכה, ג׳ מיני מצה, וכתיב על חלת לחם חמץ וגו׳, וכל מין ומין י׳ חלות, כך מפורש במנחות (דף עז.), ושעורן ה׳ סאין ירושלמיות שהן ו׳ מדבריות כ׳ עשרון׃\quad \rashiDH{מרבכת.} לחם חלוט ברותחין כל צרכו׃}
\threeverse{\arabic{verse}}%Leviticus7:13
{עַל\maqqaf חַלֹּת֙ לֶ֣חֶם חָמֵ֔ץ יַקְרִ֖יב קׇרְבָּנ֑וֹ עַל\maqqaf זֶ֖בַח תּוֹדַ֥ת שְׁלָמָֽיו׃}
{עַל גְּרִיצָן דִּלְחֵים חֲמִיעַ יְקָרֵיב קוּרְבָּנֵיהּ עַל נִכְסַת תּוֹדַת קוּדְשׁוֹהִי׃}
{With cakes of leavened bread he shall present his offering with the sacrifice of his peace-offerings for thanksgiving.}{\arabic{verse}}
\rashi{\rashiDH{יקריב קרבנו על זבח.} מגיד שאין הלחם קדוש קדושת הגוף (שם עח׃) ליפסל ביוצא וטבול יום, ומלצאת לחולין בפדיון עד שישחט הזבח׃}
\threeverse{\arabic{verse}}%Leviticus7:14
{וְהִקְרִ֨יב מִמֶּ֤נּוּ אֶחָד֙ מִכׇּל\maqqaf קׇרְבָּ֔ן תְּרוּמָ֖ה לַיהֹוָ֑ה לַכֹּהֵ֗ן הַזֹּרֵ֛ק אֶת\maqqaf דַּ֥ם הַשְּׁלָמִ֖ים ל֥וֹ יִהְיֶֽה׃}
{וִיקָרֵיב מִנֵּיהּ חַד מִכָּל קוּרְבָּנָא אַפְרָשׁוּתָא קֳדָם יְיָ לְכָהֲנָא דְּיִזְרוֹק יָת דַּם נִכְסַת קוּדְשַׁיָּא דִּילֵיהּ יְהֵי׃}
{And of it he shall present one out of each offering for a gift unto the \lord; it shall be the priest’s that dasheth the blood of the peace-offerings against the altar.}{\arabic{verse}}
\rashi{\rashiDH{אחד מכל קרבן.} לחם אחד מכל מין ומין יטול לתרומה לכהן העובד עבודתו והשאר נאכל לבעלים (מנחות עז׃), ובשרה לבעלים חוץ מחזה ושוק שבה, כמו שמפורש למטה תנופת חזה ושוק בשלמים (פסוק לד), והתודה קרויה שלמים (זבחים ד.)׃}
\threeverse{\arabic{verse}}%Leviticus7:15
{וּבְשַׂ֗ר זֶ֚בַח תּוֹדַ֣ת שְׁלָמָ֔יו בְּי֥וֹם קׇרְבָּנ֖וֹ יֵאָכֵ֑ל לֹֽא\maqqaf יַנִּ֥יחַ מִמֶּ֖נּוּ עַד\maqqaf בֹּֽקֶר׃}
{וּבְשַׂר נִכְסַת תּוֹדַת קוּדְשׁוֹהִי בְּיוֹם קוּרְבָּנֵיהּ יִתְאֲכִיל לָא יַצְנַע מִנֵּיהּ עַד צַפְרָא׃}
{And the flesh of the sacrifice of his peace-offerings for thanksgiving shall be eaten on the day of his offering; he shall not leave any of it until the morning.}{\arabic{verse}}
\rashi{\rashiDH{ובשר זבח תודת שלמיו.} יש כאן רבויין הרבה, לרבות חטאת ואשם, ואיל נזיר, וחגיגת י״ד, שיהיו נאכלין ליום ולילה (זבחים לו.  ת״כ פרק יב, א)׃\quad \rashiDH{ביום קרבנו יאכל.} וכזמן בשרה זמן לחמה׃\quad \rashiDH{לא יניח ממנו עד בוקר.} אבל אוכל הוא כל הלילה, אם כן למה אמרו עד חצות, כדי להרחיק אדם מן העבירה (ברכות ב.)׃}
\threeverse{\arabic{verse}}%Leviticus7:16
{וְאִם\maqqaf נֶ֣דֶר \legarmeh  א֣וֹ נְדָבָ֗ה זֶ֚בַח קׇרְבָּנ֔וֹ בְּי֛וֹם הַקְרִיב֥וֹ אֶת\maqqaf זִבְח֖וֹ יֵאָכֵ֑ל וּמִֽמׇּחֳרָ֔ת וְהַנּוֹתָ֥ר מִמֶּ֖נּוּ יֵאָכֵֽל׃}
{וְאִם נִדְרָא אוֹ נְדַבְתָּא נִכְסַת קוּרְבָּנֵיהּ בְּיוֹמָא דִּיקָרֵיב יָת נִכְסְתֵיהּ יִתְאֲכִיל וּבְיוֹמָא דְּבָתְרוֹהִי וּדְיִשְׁתְּאַר מִנֵּיהּ יִתְאֲכִיל׃}
{But if the sacrifice of his offering be a vow, or a freewill-offering, it shall be eaten on the day that he offereth his sacrifice; and on the morrow that which remaineth of it may be eaten.}{\arabic{verse}}
\rashi{\rashiDH{ואם נדר או נדבה.} שלא הביאה על הודאה של נס, אינה טעונה לחם ונאכלת לב׳ ימים, כמו שמפורש בענין׃\quad \rashiDH{וממחרת והנותר ממנו.} בראשון, יאכל. (ס״א והנותר ממנו יאכל) וי״ו זו יתירה היא, ויש כמוה הרבה במקרא, כגון וְאֵלֶּה בְנֵי צִבְעוֹן וְאַיָּה וַעֲנָה (בראשית לו, כד) תֵּת וְקֹדֶשׁ וְצָבָא מִרְמָס (דניאל ח, יג)׃}
\threeverse{\arabic{verse}}%Leviticus7:17
{וְהַנּוֹתָ֖ר מִבְּשַׂ֣ר הַזָּ֑בַח בַּיּוֹם֙ הַשְּׁלִישִׁ֔י בָּאֵ֖שׁ יִשָּׂרֵֽף׃}
{וּדְיִשְׁתְּאַר מִבְּשַׂר נִכְסְתָא בְּיוֹמָא תְּלִיתָאָה בְּנוּרָא יִתּוֹקַד׃}
{But that which remaineth of the flesh of the sacrifice on the third day shall be burnt with fire.}{\arabic{verse}}
\threeverse{\arabic{verse}}%Leviticus7:18
{וְאִ֣ם הֵאָכֹ֣ל יֵ֠אָכֵ֠ל מִבְּשַׂר\maqqaf זֶ֨בַח שְׁלָמָ֜יו בַּיּ֣וֹם הַשְּׁלִישִׁי֮ לֹ֣א יֵרָצֶה֒ הַמַּקְרִ֣יב אֹת֗וֹ לֹ֧א יֵחָשֵׁ֛ב ל֖וֹ פִּגּ֣וּל יִהְיֶ֑ה וְהַנֶּ֛פֶשׁ הָאֹכֶ֥לֶת מִמֶּ֖נּוּ עֲוֺנָ֥הּ תִּשָּֽׂא׃}
{וְאִם אִתְאֲכָלָא יִתְאֲכִיל מִבְּשַׂר נִכְסַת קוּדְשׁוֹהִי בְּיוֹמָא תְּלִיתָאָה לָא יְהֵי לְרַעֲוָא דִּמְקָרֵיב יָתֵיהּ לָא יִתְחֲשֵׁיב לֵיהּ מְרַחַק יְהֵי וֶאֱנָשׁ דְּיֵיכוֹל מִנֵּיהּ חוֹבֵיהּ יְקַבֵּיל׃}
{And if any of the flesh of the sacrifice of his peace-offerings be at all eaten on the third day, it shall not be accepted, neither shall it be imputed unto him that offereth it; it shall be an abhorred thing, and the soul that eateth of it shall bear his iniquity.}{\arabic{verse}}
\rashi{\rashiDH{ואם האכל יאכל וגו׳.} במחשב בשחיטה לאכלו בשלישי הכתוב מדבר יכול אם אכל ממנו בשלישי יפסל למפרע, תלמוד לומר המקריב אותו לא יחשב, בשעת הקרבה הוא נפסל, ואינו נפסל בשלישי (ת״כ פרשתא ח, א). וכן פירושו בשעת הקרבתו לא תעלה זאת במחשבה, ואם חשב פיגול יהיה׃\quad \rashiDH{והנפש האכלת ממנו.} אפילו בתוך הזמן, עונה תשא׃ 
}
\threeverse{\arabic{verse}}%Leviticus7:19
{וְהַבָּשָׂ֞ר אֲשֶׁר\maqqaf יִגַּ֤ע בְּכׇל\maqqaf טָמֵא֙ לֹ֣א יֵֽאָכֵ֔ל בָּאֵ֖שׁ יִשָּׂרֵ֑ף וְהַ֨בָּשָׂ֔ר כׇּל\maqqaf טָה֖וֹר יֹאכַ֥ל בָּשָֽׂר׃}
{וּבְשַׂר קוּדְשָׁא דְּיִקְרַב בְּכָל מְסָאַב לָא יִתְאֲכִיל בְּנוּרָא יִתּוֹקַד וּבְשַׂר קוּדְשָׁא כָּל דְּיִדְכֵּי לְקוּדְשָׁא יֵיכוֹל בְּשַׂר קוּדְשָׁא׃}
{And the flesh that toucheth any unclean thing shall not be eaten; it shall be burnt with fire. And as for the flesh, every one that is clean may eat thereof.}{\arabic{verse}}
\rashi{\rashiDH{והבשר.} של קדש שלמים אשר יגע בכל טמא לא יאכל׃\quad \rashiDH{והבשר.} לרבות אבר שיצא מקצתו, שהפנימי מותר׃\quad \rashiDH{כל טהור יאכל בשר.} מה תלמוד לומר לפי שנאמר וְדַם זְבָחֶיךָ יִשָּׁפֵךְ וְהַבָּשָׂר תֹּאכֵל (דברים יב, כז), יכול לא יאכלו שלמים אלא הבעלים, לכך נאמר כל טהור יאכל בשר׃\quad \rashiDH{(והבשר כל טהור יאכל בשר.} כלומר כל מה שאסרתי לך בחטאת ואשם שאם יצאו חוץ לקלעים אסורה, כמו שכתוב בחצר אהל מועד יאכלוה, בבשר זה אני אומר לך כל טהור יאכל בשר, אפילו בכל העיר)׃ 
}
\threeverse{\arabic{verse}}%Leviticus7:20
{וְהַנֶּ֜פֶשׁ אֲשֶׁר\maqqaf תֹּאכַ֣ל בָּשָׂ֗ר מִזֶּ֤בַח הַשְּׁלָמִים֙ אֲשֶׁ֣ר לַיהֹוָ֔ה וְטֻמְאָת֖וֹ עָלָ֑יו וְנִכְרְתָ֛ה הַנֶּ֥פֶשׁ הַהִ֖וא מֵעַמֶּֽיהָ׃}
{וֶאֱנָשׁ דְּיֵיכוֹל בִּשְׂרָא מִנִּכְסַת קוּדְשַׁיָּא דִּקְדָם יְיָ וּסְאוֹבְתֵיהּ עֲלוֹהִי וְיִשְׁתֵּיצֵי אֲנָשָׁא הַהוּא מֵעַמֵּיהּ׃}
{But the soul that eateth of the flesh of the sacrifice of peace-offerings, that pertain unto the \lord, having his uncleanness upon him, that soul shall be cut off from his people.}{\arabic{verse}}
\rashi{\rashiDH{וטמאתו עליו.} בטומאת הגוף הכתוב מדבר (זבחים מג׃), אבל טהור שאכל את הטמא, אינו ענוש כרת, אלא אזהרה, והבשר אשר יגע בכל טמא וגו׳. ואזהרת טמא שאכל את הטהור אינה מפורשת בתורה, אלא חכמים למדוה בגזירה שוה (ת״כ יד, ג.  מכות יד׃). ג׳ כריתות אמורות באוכלי קדשים בטומאת הגוף, ודרשוה רבותינו בשבועות (ז.), אחת לכלל, ואחת לפרט, ואחת ללמד על קרבן עולה ויורד שלא נאמר אלא על טומאת מקדש וקדשיו׃}
\threeverse{\arabic{verse}}%Leviticus7:21
{וְנֶ֜פֶשׁ כִּֽי\maqqaf תִגַּ֣ע בְּכׇל\maqqaf טָמֵ֗א בְּטֻמְאַ֤ת אָדָם֙ א֣וֹ \legarmeh  בִּבְהֵמָ֣ה טְמֵאָ֗ה א֚וֹ בְּכׇל\maqqaf שֶׁ֣קֶץ טָמֵ֔א וְאָכַ֛ל מִבְּשַׂר\maqqaf זֶ֥בַח הַשְּׁלָמִ֖ים אֲשֶׁ֣ר לַיהֹוָ֑ה וְנִכְרְתָ֛ה הַנֶּ֥פֶשׁ הַהִ֖וא מֵעַמֶּֽיהָ׃ \petucha \note{אין פרשה בספרי ספרד ואשכנז}}
{וֶאֱנָשׁ אֲרֵי יִקְרַב בְּכָל מְסָאַב בְּסוֹאֲבָת אֲנָשָׁא אוֹ בִּבְעִירָא מְסָאֲבָא אוֹ בְּכָל שְׁקֵיץ מְסָאַב וְיֵיכוֹל מִבְּשַׂר נִכְסַת קוּדְשַׁיָּא דִּקְדָם יְיָ וְיִשְׁתֵּיצֵי אֲנָשָׁא הַהוּא מֵעַמֵּיהּ׃}
{And when any one shall touch any unclean thing, whether it be the uncleanness of man, or an unclean beast, or any unclean detestable thing, and eat of the flesh of the sacrifice of peace-offerings, which pertain unto the \lord, that soul shall be cut off from his people.}{\arabic{verse}}
\threeverse{\arabic{verse}}%Leviticus7:22
{וַיְדַבֵּ֥ר יְהֹוָ֖ה אֶל\maqqaf מֹשֶׁ֥ה לֵּאמֹֽר׃}
{וּמַלֵּיל יְיָ עִם מֹשֶׁה לְמֵימַר׃}
{And the \lord\space spoke unto Moses, saying:}{\arabic{verse}}
\threeverse{\arabic{verse}}%Leviticus7:23
{דַּבֵּ֛ר אֶל\maqqaf בְּנֵ֥י יִשְׂרָאֵ֖ל לֵאמֹ֑ר כׇּל\maqqaf חֵ֜לֶב שׁ֥וֹר וְכֶ֛שֶׂב וָעֵ֖ז לֹ֥א תֹאכֵֽלוּ׃}
{מַלֵּיל עִם בְּנֵי יִשְׂרָאֵל לְמֵימַר כָּל תְּרַב תּוֹר וְאִמַּר וְעֵז לָא תֵיכְלוּן׃}
{Speak unto the children of Israel, saying: Ye shall eat no fat, of ox, or sheep, or goat.}{\arabic{verse}}
\threeverse{\arabic{verse}}%Leviticus7:24
{וְחֵ֤לֶב נְבֵלָה֙ וְחֵ֣לֶב טְרֵפָ֔ה יֵעָשֶׂ֖ה לְכׇל\maqqaf מְלָאכָ֑ה וְאָכֹ֖ל לֹ֥א תֹאכְלֻֽהוּ׃}
{וּתְרַב נְבִילָא וּתְרַב תְּבִירָא יִתְעֲבֵיד לְכָל עֲבִידָא וּמֵיכָל לָא תֵיכְלוּנֵּיהּ׃}
{And the fat of that which dieth of itself, and the fat of that which is torn of beasts, may be used for any other service; but ye shall in no wise eat of it.}{\arabic{verse}}
\rashi{\rashiDH{יעשה לכל מלאכה.} בא ולימד על החלב שאינו מטמא טומאת נבלות (פסחים כג.  ת״כ פרשתא י, ח)׃\quad \rashiDH{ואכל לא תאכלהו.} אמרה תורה יבוא איסור נבילה וטרפה ויחול על איסור חלב, שאם אכלו יתחייב אף על לאו של נבילה, ולא תאמר אין איסור חל על איסור (חולין לז.)׃}
\threeverse{\arabic{verse}}%Leviticus7:25
{כִּ֚י כׇּל\maqqaf אֹכֵ֣ל חֵ֔לֶב מִ֨ן\maqqaf הַבְּהֵמָ֔ה אֲשֶׁ֨ר יַקְרִ֥יב מִמֶּ֛נָּה אִשֶּׁ֖ה לַיהֹוָ֑ה וְנִכְרְתָ֛ה הַנֶּ֥פֶשׁ הָאֹכֶ֖לֶת מֵֽעַמֶּֽיהָ׃}
{אֲרֵי כָל דְּיֵיכוֹל תַּרְבָּא מִן בְּעִירָא דִּיקָרֵיב מִנַּהּ קוּרְבָּנָא קֳדָם יְיָ וְיִשְׁתֵּיצֵי אֲנָשָׁא דְּיֵיכוֹל מֵעַמֵּיהּ׃}
{For whosoever eateth the fat of the beast, of which men present an offering made by fire unto the \lord, even the soul that eateth it shall be cut off from his people.}{\arabic{verse}}
\threeverse{\arabic{verse}}%Leviticus7:26
{וְכׇל\maqqaf דָּם֙ לֹ֣א תֹאכְל֔וּ בְּכֹ֖ל מוֹשְׁבֹתֵיכֶ֑ם לָע֖וֹף וְלַבְּהֵמָֽה׃}
{וְכָל דְּמָא לָא תֵיכְלוּן בְּכֹל מוֹתְבָנֵיכוֹן דְּעוֹפָא וְדִבְעִירָא׃}
{And ye shall eat no manner of blood, whether it be of fowl or of beast, in any of your dwellings.}{\arabic{verse}}
\rashi{\rashiDH{לעוף ולבהמה.} פרט לדם דגים וחגבים׃\quad \rashiDH{בכל מושבותיכם.} לפי שהיא חובת הגוף ואינה חובת קרקע נוהגת בכל מושבות, ובמסכת קדושין בפ״א (לז׃) מפרש למה הוצרך לומר׃}
\threeverse{\arabic{verse}}%Leviticus7:27
{כׇּל\maqqaf נֶ֖פֶשׁ אֲשֶׁר\maqqaf תֹּאכַ֣ל כׇּל\maqqaf דָּ֑ם וְנִכְרְתָ֛ה הַנֶּ֥פֶשׁ הַהִ֖וא מֵֽעַמֶּֽיהָ׃ \petucha \note{אין פרשה בספרי תימן}}
{כָּל אֱנָשׁ דְּיֵיכוֹל כָּל דַּם וְיִשְׁתֵּיצֵי אֲנָשָׁא הַהוּא מֵעַמֵּיהּ׃}
{Whosoever it be that eateth any blood, that soul shall be cut off from his people.}{\arabic{verse}}
\threeverse{\arabic{verse}}%Leviticus7:28
{וַיְדַבֵּ֥ר יְהֹוָ֖ה אֶל\maqqaf מֹשֶׁ֥ה לֵּאמֹֽר׃}
{וּמַלֵּיל יְיָ עִם מֹשֶׁה לְמֵימַר׃}
{And the \lord\space spoke unto Moses, saying:}{\arabic{verse}}
\threeverse{\arabic{verse}}%Leviticus7:29
{דַּבֵּ֛ר אֶל\maqqaf בְּנֵ֥י יִשְׂרָאֵ֖ל לֵאמֹ֑ר הַמַּקְרִ֞יב אֶת\maqqaf זֶ֤בַח שְׁלָמָיו֙ לַיהֹוָ֔ה יָבִ֧יא אֶת\maqqaf קׇרְבָּנ֛וֹ לַיהֹוָ֖ה מִזֶּ֥בַח שְׁלָמָֽיו׃}
{מַלֵּיל עִם בְּנֵי יִשְׂרָאֵל לְמֵימַר דִּמְקָרֵיב יָת נִכְסַת קוּדְשׁוֹהִי קֳדָם יְיָ יַיְתֵי יָת קוּרְבָּנֵיהּ לִקְדָם יְיָ מִנִּכְסַת קוּדְשׁוֹהִי׃}
{Speak unto the children of Israel, saying: He that offereth his sacrifice of peace-offerings unto the \lord\space shall bring his offering unto the \lord\space out of his sacrifice of peace-offerings.}{\arabic{verse}}
\threeverse{\arabic{verse}}%Leviticus7:30
{יָדָ֣יו תְּבִיאֶ֔ינָה אֵ֖ת אִשֵּׁ֣י יְהֹוָ֑ה אֶת\maqqaf הַחֵ֤לֶב עַל\maqqaf הֶֽחָזֶה֙ יְבִיאֶ֔נּוּ אֵ֣ת הֶחָזֶ֗ה לְהָנִ֥יף אֹת֛וֹ תְּנוּפָ֖ה לִפְנֵ֥י יְהֹוָֽה׃}
{יְדוֹהִי יֵיתְיָן יָת קוּרְבְּנַיָּא דַּייָ יָת תַּרְבָּא עַל חַדְיָא יַיְתֵינֵיהּ יָת חַדְיָא לְאָרָמָא יָתֵיהּ אֲרָמָא קֳדָם יְיָ׃}
{His own hands shall bring the offerings of the \lord\space made by fire: the fat with the breast shall he bring, that the breast may be waved for a wave-offering before the \lord.}{\arabic{verse}}
\rashi{\rashiDH{ידיו תביאינה וגו׳.} שתהא יד הבעלים מלמעלה והחלב והחזות נתונין בה, ויד כהן מלמטה ומניפן (מנחות סא׃)׃\quad \rashiDH{את אשי ה׳.} ומה הן האשים, את החלב על החזה׃\quad \rashiDH{יביאנו.} כשמביאו מבית המטבחים נותן החלב על החזה, וכשנותנו ליד הכהן המניף נמצא החזה למעלה והחלב למטה, וזהו האמור במקום אחר שֹׁוק הַתְּרוּמָה וַחֲזֵה הַתְּנוּפָה עַל אִשֵּׁי הַחֲלָבִים יָבִיאוּ לְהָנִיף וגו׳ (ויקרא י, טו), ולאחר התנופה נותנו לכהן המקטיר, ונמצא החזה למטה, וזהו שנאמר וַיָּשִׂימוּ אֶת הַחֲלָבִים עַל הֶחָזֹות וַיַּקְטֵר הַחֲלָבִים הַמִּזְבֵּחָה (שם ט, כ), למדנו שג׳ כהנים זקוקין לה, כך מפורש במנחות (דף סב׃)׃\quad \rashiDH{את החלב על החזה יביאנו.} ואת החזה למה מביא, להניף אותו הוא מביאו, ולא שיהא הוא מן האשים, לפי שנאמר את אשי ה׳ את החלב על החזה, יכול שיהא אף החזה לאשים, לכך נאמר את החזה להניף וגו׳׃}
\threeverse{\arabic{verse}}%Leviticus7:31
{וְהִקְטִ֧יר הַכֹּהֵ֛ן אֶת\maqqaf הַחֵ֖לֶב הַמִּזְבֵּ֑חָה וְהָיָה֙ הֶֽחָזֶ֔ה לְאַהֲרֹ֖ן וּלְבָנָֽיו׃}
{וְיַסֵּיק כָּהֲנָא יָת תַּרְבָּא לְמַדְבְּחָא וִיהֵי חַדְיָא לְאַהֲרֹן וְלִבְנוֹהִי׃}
{And the priest shall make the fat smoke upon the altar; but the breast shall be Aaron’s and his sons’ .}{\arabic{verse}}
\rashi{\rashiDH{והקטיר הכהן את החלב.} ואחר כך והיה החזה לאהרן למדנו שאין הבשר נאכל בעוד שהאימורים למטה מן המזבח (ת״כ פרק טז, ד)׃}
\threeverse{\arabic{verse}}%Leviticus7:32
{וְאֵת֙ שׁ֣וֹק הַיָּמִ֔ין תִּתְּנ֥וּ תְרוּמָ֖ה לַכֹּהֵ֑ן מִזִּבְחֵ֖י שַׁלְמֵיכֶֽם׃}
{וְיָת שָׁקָא דְּיַמִּינָא תִּתְּנוּן אַפְרָשׁוּתָא לְכָהֲנָא מִנִּכְסַת קוּדְשֵׁיכוֹן׃}
{And the right thigh shall ye give unto the priest for a heave-offering out of your sacrifices of peace-offerings.}{\arabic{verse}}
\rashi{\rashiDH{שוק.} מן הפרק של ארכובה הנמכרת עם הראש עד הפרק האמצעי שהוא סובך של ירך (חולין קלד׃)׃}
\threeverse{\arabic{verse}}%Leviticus7:33
{הַמַּקְרִ֞יב אֶת\maqqaf דַּ֧ם הַשְּׁלָמִ֛ים וְאֶת\maqqaf הַחֵ֖לֶב מִבְּנֵ֣י אַהֲרֹ֑ן ל֧וֹ תִהְיֶ֛ה שׁ֥וֹק הַיָּמִ֖ין לְמָנָֽה׃}
{דִּמְקָרֵיב יָת דַּם נִכְסַת קוּדְשַׁיָּא וְיָת תַּרְבָּא מִבְּנֵי אַהֲרֹן דִּילֵיהּ תְּהֵי שָׁקָא דְּיַמִּינָא לֻחְלָק׃}
{He among the sons of Aaron, that offereth the blood of the peace-offerings, and the fat, shall have the right thigh for a portion.}{\arabic{verse}}
\rashi{\rashiDH{המקריב את דם וגו׳.} מי שהוא ראוי לזריקתו ולהקטיר חלביו, יצא טמא בשעת זריקת דמים או בשעת הקטר חלבים שאינו חולק בבשר׃}
\threeverse{\arabic{verse}}%Leviticus7:34
{כִּי֩ אֶת\maqqaf חֲזֵ֨ה הַתְּנוּפָ֜ה וְאֵ֣ת \legarmeh  שׁ֣וֹק הַתְּרוּמָ֗ה לָקַ֙חְתִּי֙ מֵאֵ֣ת בְּנֵֽי\maqqaf יִשְׂרָאֵ֔ל מִזִּבְחֵ֖י שַׁלְמֵיהֶ֑ם וָאֶתֵּ֣ן אֹ֠תָ֠ם לְאַהֲרֹ֨ן הַכֹּהֵ֤ן וּלְבָנָיו֙ לְחׇק\maqqaf עוֹלָ֔ם מֵאֵ֖ת בְּנֵ֥י יִשְׂרָאֵֽל׃}
{אֲרֵי יָת חַדְיָא דַּאֲרָמוּתָא וְיָת שָׁקָא דְּאַפְרָשׁוּתָא נְסֵיבִית מִן בְּנֵי יִשְׂרָאֵל מִנִּכְסַת קוּדְשֵׁיהוֹן וִיהַבִית יָתְהוֹן לְאַהֲרֹן כָּהֲנָא וְלִבְנוֹהִי לִקְיָם עָלַם מִן בְּנֵי יִשְׂרָאֵל׃}
{For the breast of waving and the thigh of heaving have I taken of the children of Israel out of their sacrifices of peace-offerings, and have given them unto Aaron the priest and unto his sons as a due for ever from the children of Israel.}{\arabic{verse}}
\rashi{\rashiDH{התנופה התרומה.} מוליך ומביא מעלה ומוריד (סוכה לז׃)׃ 
}
\threeverse{\arabic{verse}}%Leviticus7:35
{זֹ֣את מִשְׁחַ֤ת אַהֲרֹן֙ וּמִשְׁחַ֣ת בָּנָ֔יו מֵאִשֵּׁ֖י יְהֹוָ֑ה בְּיוֹם֙ הִקְרִ֣יב אֹתָ֔ם לְכַהֵ֖ן לַיהֹוָֽה׃}
{דָּא רְבוּת אַהֲרֹן וּרְבוּת בְּנוֹהִי מְקּוּרְבָּנַיָּא דַּייָ בְּיוֹמָא דִּיקָרְבוּן יָתְהוֹן לְשַׁמָּשָׁא קֳדָם יְיָ׃}
{This is the consecrated portion of Aaron, and the consecrated portion of his sons, out of the offerings of the \lord\space made by fire, in the day when they were presented to minister unto the \lord\space in the priest’s office;}{\arabic{verse}}
\threeverse{\arabic{verse}}%Leviticus7:36
{אֲשֶׁר֩ צִוָּ֨ה יְהֹוָ֜ה לָתֵ֣ת לָהֶ֗ם בְּיוֹם֙ מׇשְׁח֣וֹ אֹתָ֔ם מֵאֵ֖ת בְּנֵ֣י יִשְׂרָאֵ֑ל חֻקַּ֥ת עוֹלָ֖ם לְדֹרֹתָֽם׃}
{דְּפַקֵּיד יְיָ לְמִתַּן לְהוֹן בְּיוֹמָא דִּירַבּוֹן יָתְהוֹן מִן בְּנֵי יִשְׂרָאֵל קְיָם עָלַם לְדָרֵיהוֹן׃}
{which the \lord\space commanded to be given them of the children of Israel, in the day that they were anointed. It is a due for ever throughout their generations.}{\arabic{verse}}
\threeverse{\arabic{verse}}%Leviticus7:37
{זֹ֣את הַתּוֹרָ֗ה לָֽעֹלָה֙ לַמִּנְחָ֔ה וְלַֽחַטָּ֖את וְלָאָשָׁ֑ם וְלַ֨מִּלּוּאִ֔ים וּלְזֶ֖בַח הַשְּׁלָמִֽים׃}
{דָּא אוֹרָיְתָא לַעֲלָתָא לְמִנְחָתָא וּלְחַטָּתָא וּלְאֲשָׁמָא וּלְקוּרְבָּנַיָּא וּלְנִכְסַת קוּדְשַׁיָּא׃}
{This is the law of the burnt-offering, of the meal-offering, and of the sin-offering, and of the guilt-offering, and of the consecration-offering, and of the sacrifice of peace-offerings;}{\arabic{verse}}
\rashi{\rashiDH{ולמלואים.} ליום חינוך הכהונה׃ 
}
\threeverse{\arabic{verse}}%Leviticus7:38
{אֲשֶׁ֨ר צִוָּ֧ה יְהֹוָ֛ה אֶת\maqqaf מֹשֶׁ֖ה בְּהַ֣ר סִינָ֑י בְּי֨וֹם צַוֺּת֜וֹ אֶת\maqqaf בְּנֵ֣י יִשְׂרָאֵ֗ל לְהַקְרִ֧יב אֶת\maqqaf קׇרְבְּנֵיהֶ֛ם לַיהֹוָ֖ה בְּמִדְבַּ֥ר סִינָֽי׃ \petucha }
{דְּפַקֵּיד יְיָ יָת מֹשֶׁה בְּטוּרָא דְּסִינָי בְּיוֹמָא דְּפַקֵּיד יָת בְּנֵי יִשְׂרָאֵל לְקָרָבָא יָת קוּרְבָּנְהוֹן קֳדָם יְיָ בְּמַדְבְּרָא דְּסִינָי׃}
{which the \lord\space commanded Moses in mount Sinai, in the day that he commanded the children of Israel to present their offerings unto the \lord, in the wilderness of Sinai.}{\arabic{verse}}

\newperek
\aliyacounter{רביעי}
\newseder{4}
\threeverse{\aliya{רביעי}\newline\vspace{-4pt}\newline\seder{ד}}%Leviticus8:1
{וַיְדַבֵּ֥ר יְהֹוָ֖ה אֶל\maqqaf מֹשֶׁ֥ה לֵּאמֹֽר׃}
{וּמַלֵּיל יְיָ עִם מֹשֶׁה לְמֵימַר׃}
{And the \lord\space spoke unto Moses, saying:}{\Roman{chap}}
\threeverse{\arabic{verse}}%Leviticus8:2
{קַ֤ח אֶֽת\maqqaf אַהֲרֹן֙ וְאֶת\maqqaf בָּנָ֣יו אִתּ֔וֹ וְאֵת֙ הַבְּגָדִ֔ים וְאֵ֖ת שֶׁ֣מֶן הַמִּשְׁחָ֑ה וְאֵ֣ת \legarmeh  פַּ֣ר הַֽחַטָּ֗את וְאֵת֙ שְׁנֵ֣י הָֽאֵילִ֔ים וְאֵ֖ת סַ֥ל הַמַּצּֽוֹת׃}
{קָרֵיב יָת אַהֲרֹן וְיָת בְּנוֹהִי עִמֵּיהּ וְיָת לְבוּשַׁיָּא וְיָת מִשְׁחָא דִּרְבוּתָא וְיָת תּוֹרָא דְּחַטָּתָא וְיָת תְּרֵין דִּכְרִין וְיָת סַלָּא דְּפַטִּירַיָּא׃}
{‘Take Aaron and his sons with him, and the garments, and the anointing oil, and the bullock of the sin-offering, and the two rams, and the basket of unleavened bread;}{\arabic{verse}}
\rashi{\rashiDH{קח את אהרן.} פרשה זו נאמרה שבעת ימים קודם הקמת המשכן, שאין מוקדם ומאוחר בתורה׃\quad \rashiDH{קח את אהרן.} קחנו בדברים ומשכהו׃\quad \rashiDH{ואת פר החטאת וגו׳.} אלו האמורים בענין צוואת המלואים בואתה תצוה (שמות כט) ועכשיו ביום ראשון למלואים חזר וזרזו בשעת מעשה׃ 
}
\threeverse{\arabic{verse}}%Leviticus8:3
{וְאֵ֥ת כׇּל\maqqaf הָעֵדָ֖ה הַקְהֵ֑ל אֶל\maqqaf פֶּ֖תַח אֹ֥הֶל מוֹעֵֽד׃}
{וְיָת כָּל כְּנִשְׁתָּא כְּנוֹשׁ לִתְרַע מַשְׁכַּן זִמְנָא׃}
{and assemble thou all the congregation at the door of the tent of meeting.’}{\arabic{verse}}
\rashi{\rashiDH{הקהל אל פתח אהל מועד.} זה אחד מן המקומות שהחזיק מועט את המרובה׃}
\threeverse{\arabic{verse}}%Leviticus8:4
{וַיַּ֣עַשׂ מֹשֶׁ֔ה כַּֽאֲשֶׁ֛ר צִוָּ֥ה יְהֹוָ֖ה אֹת֑וֹ וַתִּקָּהֵל֙ הָֽעֵדָ֔ה אֶל\maqqaf פֶּ֖תַח אֹ֥הֶל מוֹעֵֽד׃}
{וַעֲבַד מֹשֶׁה כְּמָא דְּפַקֵּיד יְיָ יָתֵיהּ וְאִתְכְּנֵישַׁת כְּנִשְׁתָּא לִתְרַע מַשְׁכַּן זִמְנָא׃}
{And Moses did as the \lord\space commanded him; and the congregation was assembled at the door of the tent of meeting.}{\arabic{verse}}
\threeverse{\arabic{verse}}%Leviticus8:5
{וַיֹּ֥אמֶר מֹשֶׁ֖ה אֶל\maqqaf הָעֵדָ֑ה זֶ֣ה הַדָּבָ֔ר אֲשֶׁר\maqqaf צִוָּ֥ה יְהֹוָ֖ה לַעֲשֽׂוֹת׃}
{וַאֲמַר מֹשֶׁה לִכְנִשְׁתָּא דֵּין פִּתְגָמָא דְּפַקֵּיד יְיָ לְמֶעֱבַד׃}
{And Moses said unto the congregation: ‘This is the thing which the \lord\space hath commanded to be done.’}{\arabic{verse}}
\rashi{\rashiDH{זה הדבר.} דברים שתראו שאני עושה לפניכם צוני הקב״ה לעשות, ואל תאמרו לכבודי ולכבוד אחי אני עושה. כל הענין הזה בפרשת המלואים פירשתי בואתה תצוה (שם)׃}
\threeverse{\arabic{verse}}%Leviticus8:6
{וַיַּקְרֵ֣ב מֹשֶׁ֔ה אֶֽת\maqqaf אַהֲרֹ֖ן וְאֶת\maqqaf בָּנָ֑יו וַיִּרְחַ֥ץ אֹתָ֖ם בַּמָּֽיִם׃}
{וְקָרֵיב מֹשֶׁה יָת אַהֲרֹן וְיָת בְּנוֹהִי וְאַסְחִי יָתְהוֹן בְּמַיָּא׃}
{And Moses brought Aaron and his sons, and washed them with water.}{\arabic{verse}}
\threeverse{\arabic{verse}}%Leviticus8:7
{וַיִּתֵּ֨ן עָלָ֜יו אֶת\maqqaf הַכֻּתֹּ֗נֶת וַיַּחְגֹּ֤ר אֹתוֹ֙ בָּֽאַבְנֵ֔ט וַיַּלְבֵּ֤שׁ אֹתוֹ֙ אֶֽת\maqqaf הַמְּעִ֔יל וַיִּתֵּ֥ן עָלָ֖יו אֶת\maqqaf הָאֵפֹ֑ד וַיַּחְגֹּ֣ר אֹת֗וֹ בְּחֵ֙שֶׁב֙ הָֽאֵפֹ֔ד וַיֶּאְפֹּ֥ד ל֖וֹ בּֽוֹ׃}
{וִיהַב עֲלוֹהִי יָת כִּתּוּנָא וְזָרֵיז יָתֵיהּ בְּהִמְיָנָא וְאַלְבֵּישׁ יָתֵיהּ יָת מְעִילָא וִיהַב עֲלוֹהִי יָת אֵיפוֹדָא וְזָרֵיז יָתֵיהּ בְּהִמְיַן אֵיפוֹדָא וְאַתְקֵין לֵיהּ בֵּיהּ׃}
{And he put upon him the tunic, and girded him with the girdle, and clothed him with the robe, and put the ephod upon him, and he girded him with the skilfully woven band of the ephod, and bound it unto him therewith.}{\arabic{verse}}
\threeverse{\arabic{verse}}%Leviticus8:8
{וַיָּ֥שֶׂם עָלָ֖יו אֶת\maqqaf הַחֹ֑שֶׁן וַיִּתֵּן֙ אֶל\maqqaf הַחֹ֔שֶׁן אֶת\maqqaf הָאוּרִ֖ים וְאֶת\maqqaf הַתֻּמִּֽים׃}
{וְשַׁוִּי עֲלוֹהִי יָת חוּשְׁנָא וִיהַב בְּחוּשְׁנָא יָת אוּרַיָּא וְיָת תּוּמַּיָּא׃}
{And he placed the breastplate upon him; and in the breastplate he put the Urim and the Thummim.}{\arabic{verse}}
\rashi{\rashiDH{את האורים.} כתב של שם המפורש׃}
\threeverse{\arabic{verse}}%Leviticus8:9
{וַיָּ֥שֶׂם אֶת\maqqaf הַמִּצְנֶ֖פֶת עַל\maqqaf רֹאשׁ֑וֹ וַיָּ֨שֶׂם עַֽל\maqqaf הַמִּצְנֶ֜פֶת אֶל\maqqaf מ֣וּל פָּנָ֗יו אֵ֣ת צִ֤יץ הַזָּהָב֙ נֵ֣זֶר הַקֹּ֔דֶשׁ כַּאֲשֶׁ֛ר צִוָּ֥ה יְהֹוָ֖ה אֶת\maqqaf מֹשֶֽׁה׃}
{וְשַׁוִּי יָת מַצְנַפְתָּא עַל רֵישֵׁיהּ וְשַׁוִּי עַל מַצְנַפְתָּא לָקֳבֵיל אַפּוֹהִי יָת צִיצָא דְּדַהְבָּא כְּלִילָא דְּקוּדְשָׁא כְּמָא דְּפַקֵּיד יְיָ יָת מֹשֶׁה׃}
{And he set the mitre upon his head; and upon the mitre, in front, did he set the golden plate, the holy crown; as the \lord\space commanded Moses.}{\arabic{verse}}
\rashi{\rashiDH{וישם על המצנפת.} פתילי תכלת הקבועים בציץ נתן על המצנפת נמצא הציץ תלוי במצנפת׃}
\threeverse{\arabic{verse}}%Leviticus8:10
{וַיִּקַּ֤ח מֹשֶׁה֙ אֶת\maqqaf שֶׁ֣מֶן הַמִּשְׁחָ֔ה וַיִּמְשַׁ֥ח אֶת\maqqaf הַמִּשְׁכָּ֖ן וְאֶת\maqqaf כׇּל\maqqaf אֲשֶׁר\maqqaf בּ֑וֹ וַיְקַדֵּ֖שׁ אֹתָֽם׃}
{וּנְסֵיב מֹשֶׁה יָת מִשְׁחָא דִּרְבוּתָא וְרַבִּי יָת מַשְׁכְּנָא וְיָת כָּל דְּבֵיהּ וְקַדֵּישׁ יָתְהוֹן׃}
{And Moses took the anointing oil, and anointed the tabernacle and all that was therein, and sanctified them.}{\arabic{verse}}
\threeverse{\arabic{verse}}%Leviticus8:11
{וַיַּ֥ז מִמֶּ֛נּוּ עַל\maqqaf הַמִּזְבֵּ֖חַ שֶׁ֣בַע פְּעָמִ֑ים וַיִּמְשַׁ֨ח אֶת\maqqaf הַמִּזְבֵּ֜חַ וְאֶת\maqqaf כׇּל\maqqaf כֵּלָ֗יו וְאֶת\maqqaf הַכִּיֹּ֛ר וְאֶת\maqqaf כַּנּ֖וֹ לְקַדְּשָֽׁם׃}
{וְאַדִּי מִנֵּיהּ עַל מַדְבְּחָא שְׁבַע זִמְנִין וְרַבִּי יָת מַדְבְּחָא וְיָת כָּל מָנוֹהִי וְיָת כִּיּוֹרָא וְיָת בְּסִיסֵיהּ לְקַדָּשׁוּתְהוֹן׃}
{And he sprinkled thereof upon the altar seven times, and anointed the altar and all its vessels, and the laver and its base, to sanctify them.}{\arabic{verse}}
\rashi{\rashiDH{ויז ממנו על המזבח.} לא ידעתי היכן נצטוה בהזאות הללו׃}
\threeverse{\arabic{verse}}%Leviticus8:12
{וַיִּצֹק֙ מִשֶּׁ֣מֶן הַמִּשְׁחָ֔ה עַ֖ל רֹ֣אשׁ אַהֲרֹ֑ן וַיִּמְשַׁ֥ח אֹת֖וֹ לְקַדְּשֽׁוֹ׃}
{וַאֲרֵיק מִמִּשְׁחָא דִּרְבוּתָא עַל רֵישָׁא דְּאַהֲרֹן וְרַבִּי יָתֵיהּ לְקַדָּשׁוּתֵיהּ׃}
{And he poured of the anointing oil upon Aaron’s head, and anointed him, to sanctify him.}{\arabic{verse}}
\rashi{\rashiDH{ויצק. וימשח.} בתחלה יוצק על ראשו, ואח״כ נותן בין ריסי עיניו, ומושך באצבעו מזה לזה (כריתות ה׃)׃}
\threeverse{\arabic{verse}}%Leviticus8:13
{וַיַּקְרֵ֨ב מֹשֶׁ֜ה אֶת\maqqaf בְּנֵ֣י אַהֲרֹ֗ן וַיַּלְבִּשֵׁ֤ם כֻּתֳּנֹת֙ וַיַּחְגֹּ֤ר אֹתָם֙ אַבְנֵ֔ט וַיַּחֲבֹ֥שׁ לָהֶ֖ם מִגְבָּע֑וֹת כַּאֲשֶׁ֛ר צִוָּ֥ה יְהֹוָ֖ה אֶת\maqqaf מֹשֶֽׁה׃}
{וְקָרֵיב מֹשֶׁה יָת בְּנֵי אַהֲרֹן וְאַלְבֵּישִׁנּוּן כִּתּוּנִין וְזָרֵיז יָתְהוֹן הִמְיָנִין וְאַתְקֵין לְהוֹן קוֹבְעִין כְּמָא דְּפַקֵּיד יְיָ יָת מֹשֶׁה׃}
{And Moses brought Aaron’s sons, and clothed them with tunics, and girded them with girdles, and bound head-tires upon them; as the \lord\space commanded Moses.}{\arabic{verse}}
\rashi{\rashiDH{ויחבש.} לשון קשירה׃}
\aliyacounter{חמישי}
\threeverse{\aliya{חמישי}}%Leviticus8:14
{וַיַּגֵּ֕שׁ אֵ֖ת פַּ֣ר הַֽחַטָּ֑את וַיִּסְמֹ֨ךְ אַהֲרֹ֤ן וּבָנָיו֙ אֶת\maqqaf יְדֵיהֶ֔ם עַל\maqqaf רֹ֖אשׁ פַּ֥ר הַֽחַטָּֽאת׃}
{וְקָרֵיב יָת תּוֹרָא דְּחַטָּתָא וּסְמַךְ אַהֲרֹן וּבְנוֹהִי יָת יְדֵיהוֹן עַל רֵישׁ תּוֹרָא דְּחַטָּתָא׃}
{And the bullock of the sin-offering was brought; and Aaron and his sons laid their hands upon the head of the bullock of the sin-offering.}{\arabic{verse}}
\threeverse{\arabic{verse}}%Leviticus8:15
{וַיִּשְׁחָ֗ט וַיִּקַּ֨ח מֹשֶׁ֤ה אֶת\maqqaf הַדָּם֙ וַ֠יִּתֵּ֠ן עַל\maqqaf קַרְנ֨וֹת הַמִּזְבֵּ֤חַ סָבִיב֙ בְּאֶצְבָּע֔וֹ וַיְחַטֵּ֖א אֶת\maqqaf הַמִּזְבֵּ֑חַ וְאֶת\maqqaf הַדָּ֗ם יָצַק֙ אֶל\maqqaf יְס֣וֹד הַמִּזְבֵּ֔חַ וַֽיְקַדְּשֵׁ֖הוּ לְכַפֵּ֥ר עָלָֽיו׃}
{וּנְכַס וּנְסֵיב מֹשֶׁה יָת דְּמָא וִיהַב עַל קַרְנָת מַדְבְּחָא סְחוֹר סְחוֹר בְּאֶצְבְּעֵיהּ וְדַכִּי יָת מַדְבְּחָא וְיָת דְּמָא אֲרֵיק לִיסוֹדָא דְּמַדְבְּחָא וְקַדְּשֵׁיהּ לְכַפָּרָא עֲלוֹהִי׃}
{And when it was slain, Moses took the blood, and put it upon the horns of the altar round about with his finger, and purified the altar, and poured out the remaining blood at the base of the altar, and sanctified it, to make atonement for it.}{\arabic{verse}}
\rashi{\rashiDH{ויחטא את המזבח.} חִטְּאוֹ וְטִהֲרוֹ מִזָּרוּת ליכנס לקדושה׃\quad \rashiDH{ויקדשהו.} בעבודה זו׃\quad \rashiDH{לכפר עליו.} מעתה כל הכפרות׃ 
}
\threeverse{\arabic{verse}}%Leviticus8:16
{וַיִּקַּ֗ח אֶֽת\maqqaf כׇּל\maqqaf הַחֵ֘לֶב֮ אֲשֶׁ֣ר עַל\maqqaf הַקֶּ֒רֶב֒ וְאֵת֙ יֹתֶ֣רֶת הַכָּבֵ֔ד וְאֶת\maqqaf שְׁתֵּ֥י הַכְּלָיֹ֖ת וְאֶֽת\maqqaf חֶלְבְּהֶ֑ן וַיַּקְטֵ֥ר מֹשֶׁ֖ה הַמִּזְבֵּֽחָה׃}
{וּנְסֵיב יָת כָּל תַּרְבָּא דְּעַל גַּוָּא וְיָת חֲצַר כַּבְדָּא וְיָת תַּרְתֵּין כּוֹלְיָן וְיָת תַּרְבְּהוֹן וְאַסֵּיק מֹשֶׁה לְמַדְבְּחָא׃}
{And he took all the fat that was upon the inwards, and the lobe of the liver, and the two kidneys, and their fat, and Moses made it smoke upon the altar.}{\arabic{verse}}
\rashi{\rashiDH{ואת יתרת הכבד.} לבד הכבד, שהיה נוטל מעט מן הכבד עמה׃ 
}
\threeverse{\arabic{verse}}%Leviticus8:17
{וְאֶת\maqqaf הַפָּ֤ר וְאֶת\maqqaf עֹרוֹ֙ וְאֶת\maqqaf בְּשָׂר֣וֹ וְאֶת\maqqaf פִּרְשׁ֔וֹ שָׂרַ֣ף בָּאֵ֔שׁ מִח֖וּץ לַֽמַּחֲנֶ֑ה כַּאֲשֶׁ֛ר צִוָּ֥ה יְהֹוָ֖ה אֶת\maqqaf מֹשֶֽׁה׃}
{וְיָת תּוֹרָא וְיָת מַשְׁכֵּיהּ וְיָת בִּשְׂרֵיהּ וְיָת אוּכְלֵיהּ אוֹקֵיד בְּנוּרָא מִבַּרָא לְמַשְׁרִיתָא כְּמָא דְּפַקֵּיד יְיָ יָת מֹשֶׁה׃}
{But the bullock, and its skin, and its flesh, and its dung, were burnt with fire without the camp; as the \lord\space commanded Moses.}{\arabic{verse}}
\threeverse{\arabic{verse}}%Leviticus8:18
{וַיַּקְרֵ֕ב אֵ֖ת אֵ֣יל הָעֹלָ֑ה וַֽיִּסְמְכ֞וּ אַהֲרֹ֧ן וּבָנָ֛יו אֶת\maqqaf יְדֵיהֶ֖ם עַל\maqqaf רֹ֥אשׁ הָאָֽיִל׃}
{וְקָרֵיב יָת דִּכְרָא דַּעֲלָתָא וּסְמַכוּ אַהֲרֹן וּבְנוֹהִי יָת יְדֵיהוֹן עַל רֵישׁ דִּכְרָא׃}
{And the ram of the burnt-offering was presented; and Aaron and his sons laid their hands upon the head of the ram.}{\arabic{verse}}
\threeverse{\arabic{verse}}%Leviticus8:19
{וַיִּשְׁחָ֑ט וַיִּזְרֹ֨ק מֹשֶׁ֧ה אֶת\maqqaf הַדָּ֛ם עַל\maqqaf הַמִּזְבֵּ֖חַ סָבִֽיב׃}
{וּנְכַס וּזְרַק מֹשֶׁה יָת דְּמָא עַל מַדְבְּחָא סְחוֹר סְחוֹר׃}
{And when it was killed, Moses dashed the blood against the altar round about.}{\arabic{verse}}
\threeverse{\arabic{verse}}%Leviticus8:20
{וְאֶ֨ת\maqqaf הָאַ֔יִל נִתַּ֖ח לִנְתָחָ֑יו וַיַּקְטֵ֤ר מֹשֶׁה֙ אֶת\maqqaf הָרֹ֔אשׁ וְאֶת\maqqaf הַנְּתָחִ֖ים וְאֶת\maqqaf הַפָּֽדֶר׃}
{וְיָת דִּכְרָא פַּלֵּיג לְאֶבְרוֹהִי וְאַסֵּיק מֹשֶׁה יָת רֵישָׁא וְיָת אֶבְרַיָּא וְיָת תַּרְבָּא׃}
{And when the ram was cut into its pieces, Moses made the head, and the pieces, and the suet smoke.}{\arabic{verse}}
\threeverse{\arabic{verse}}%Leviticus8:21
{וְאֶת\maqqaf הַקֶּ֥רֶב וְאֶת\maqqaf הַכְּרָעַ֖יִם רָחַ֣ץ בַּמָּ֑יִם וַיַּקְטֵר֩ מֹשֶׁ֨ה אֶת\maqqaf כׇּל\maqqaf הָאַ֜יִל הַמִּזְבֵּ֗חָה עֹלָ֨ה ה֤וּא לְרֵֽיחַ\maqqaf נִיחֹ֙חַ֙ אִשֶּׁ֥ה הוּא֙ לַיהֹוָ֔ה כַּאֲשֶׁ֛ר צִוָּ֥ה יְהֹוָ֖ה אֶת\maqqaf מֹשֶֽׁה׃}
{וְיָת גַּוָּא וְיָת כְּרָעַיָּא חַלֵּיל בְּמַיָּא וְאַסֵּיק מֹשֶׁה יָת כָּל דִּכְרָא לְמַדְבְּחָא עֲלָתָא הוּא לְאִתְקַבָּלָא בְּרַעֲוָא קוּרְבָּנָא הוּא קֳדָם יְיָ כְּמָא דְּפַקֵּיד יְיָ יָת מֹשֶׁה׃}
{And when the inwards and the legs were washed with water, Moses made the whole ram smoke upon the altar; it was a burnt-offering for a sweet savour; it was an offering made by fire unto the \lord; as the \lord\space commanded Moses.}{\arabic{verse}}
\aliyacounter{ששי}
\threeverse{\aliya{ששי}}%Leviticus8:22
{וַיַּקְרֵב֙ אֶת\maqqaf הָאַ֣יִל הַשֵּׁנִ֔י אֵ֖יל הַמִּלֻּאִ֑ים וַֽיִּסְמְכ֞וּ אַהֲרֹ֧ן וּבָנָ֛יו אֶת\maqqaf יְדֵיהֶ֖ם עַל\maqqaf רֹ֥אשׁ הָאָֽיִל׃}
{וְקָרֵיב יָת דִּכְרָא תִּנְיָנָא דְּכַר קוּרְבָּנַיָּא וּסְמַכוּ אַהֲרֹן וּבְנוֹהִי יָת יְדֵיהוֹן עַל רֵישׁ דִּכְרָא׃}
{And the other ram was presented, the ram of consecration, and Aaron and his sons laid their hands upon the head of the ram.}{\arabic{verse}}
\rashi{\rashiDH{איל המלאים.} איל השלמים שמלואים לשון שלמים שממלאים ומשלימים את הכהנים בכהונתם׃}
\threeverse{\arabic{verse}}%Leviticus8:23
{וַיִּשְׁחָ֓ט \legarmeh  וַיִּקַּ֤ח מֹשֶׁה֙ מִדָּמ֔וֹ וַיִּתֵּ֛ן עַל\maqqaf תְּנ֥וּךְ אֹֽזֶן\maqqaf אַהֲרֹ֖ן הַיְמָנִ֑ית וְעַל\maqqaf בֹּ֤הֶן יָדוֹ֙ הַיְמָנִ֔ית וְעַל\maqqaf בֹּ֥הֶן רַגְל֖וֹ הַיְמָנִֽית׃}
{וּנְכַס וּנְסֵיב מֹשֶׁה מִדְּמֵיהּ וִיהַב עַל רוּם אוּדְנָא דְּאַהֲרֹן דְּיַמִּינָא וְעַל אִלְיוֹן יְדֵיהּ דְּיַמִּינָא וְעַל אִלְיוֹן רַגְלֵיהּ דְּיַמִּינָא׃}
{And when it was slain, Moses took of the blood thereof, and put it upon the tip of Aaron’s right ear, and upon the thumb of his right hand, and upon the great toe of his right foot.}{\arabic{verse}}
\threeverse{\arabic{verse}}%Leviticus8:24
{וַיַּקְרֵ֞ב אֶת\maqqaf בְּנֵ֣י אַהֲרֹ֗ן וַיִּתֵּ֨ן מֹשֶׁ֤ה מִן\maqqaf הַדָּם֙ עַל\maqqaf תְּנ֤וּךְ אׇזְנָם֙ הַיְמָנִ֔ית וְעַל\maqqaf בֹּ֤הֶן יָדָם֙ הַיְמָנִ֔ית וְעַל\maqqaf בֹּ֥הֶן רַגְלָ֖ם הַיְמָנִ֑ית וַיִּזְרֹ֨ק מֹשֶׁ֧ה אֶת\maqqaf הַדָּ֛ם עַל\maqqaf הַֽמִּזְבֵּ֖חַ סָבִֽיב׃}
{וְקָרֵיב יָת בְּנֵי אַהֲרֹן וִיהַב מֹשֶׁה מִן דְּמָא עַל רוּם אוּדְנְהוֹן דְּיַמִּינָא וְעַל אִלְיוֹן יַדְהוֹן דְּיַמִּינָא וְעַל אִלְיוֹן רַגְלְהוֹן דְּיַמִּינָא וּזְרַק מֹשֶׁה יָת דְּמָא עַל מַדְבְּחָא סְחוֹר סְחוֹר׃}
{And Aaron’s sons were brought, and Moses put of the blood upon the tip of their right ear, and upon the thumb of their right hand, and upon the great toe of their right foot; and Moses dashed the blood against the altar round about.}{\arabic{verse}}
\threeverse{\arabic{verse}}%Leviticus8:25
{וַיִּקַּ֞ח אֶת\maqqaf הַחֵ֣לֶב וְאֶת\maqqaf הָֽאַלְיָ֗ה וְאֶֽת\maqqaf כׇּל\maqqaf הַחֵ֘לֶב֮ אֲשֶׁ֣ר עַל\maqqaf הַקֶּ֒רֶב֒ וְאֵת֙ יֹתֶ֣רֶת הַכָּבֵ֔ד וְאֶת\maqqaf שְׁתֵּ֥י הַכְּלָיֹ֖ת וְאֶֽת\maqqaf חֶלְבְּהֶ֑ן וְאֵ֖ת שׁ֥וֹק הַיָּמִֽין׃}
{וּנְסֵיב יָת תַּרְבָּא וְיָת אַלְיְתָא וְיָת כָּל תַּרְבָּא דְּעַל גַּוָּא וְיָת חֲצַר כַּבְדָּא וְיָת תַּרְתֵּין כּוֹלְיָן וְיָת תַּרְבְּהוֹן וְיָת שָׁקָא דְּיַמִּינָא׃}
{And he took the fat, and the fat tail, and all the fat that was upon the inwards, and the lobe of the liver, and the two kidneys, and their fat, and the right thigh.}{\arabic{verse}}
\threeverse{\arabic{verse}}%Leviticus8:26
{וּמִסַּ֨ל הַמַּצּ֜וֹת אֲשֶׁ֣ר \legarmeh  לִפְנֵ֣י יְהֹוָ֗ה לָ֠קַ֠ח חַלַּ֨ת מַצָּ֤ה אַחַת֙ וְֽחַלַּ֨ת לֶ֥חֶם שֶׁ֛מֶן אַחַ֖ת וְרָקִ֣יק אֶחָ֑ד וַיָּ֙שֶׂם֙ עַל\maqqaf הַ֣חֲלָבִ֔ים וְעַ֖ל שׁ֥וֹק הַיָּמִֽין׃}
{וּמִסַּלָּא דְּפַטִּירַיָּא דִּקְדָם יְיָ נְסֵיב גְּרִיצְתָא פַּטִּירְתָא חֲדָא וּגְרִיצְתָא דִּלְחֵים מְשַׁח חֲדָא וְאֶסְפּוֹג חַד וְשַׁוִּי עַל תַּרְבַּיָּא וְעַל שָׁקָא דְּיַמִּינָא׃}
{And out of the basket of unleavened bread, that was before the \lord, he took one unleavened cake, and one cake of oiled bread, and one wafer, and placed them on the fat, and upon the right thigh.}{\arabic{verse}}
\rashi{\rashiDH{וחלת לחם שמן.} היא רבוכה, שהיה מרבה בה שמן כנגד החלות והרקיקין, כך מפורש במנחות (עח.)׃}
\threeverse{\arabic{verse}}%Leviticus8:27
{וַיִּתֵּ֣ן אֶת\maqqaf הַכֹּ֔ל עַ֚ל כַּפֵּ֣י אַהֲרֹ֔ן וְעַ֖ל כַּפֵּ֣י בָנָ֑יו וַיָּ֧נֶף אֹתָ֛ם תְּנוּפָ֖ה לִפְנֵ֥י יְהֹוָֽה׃}
{וִיהַב יָת כּוֹלָא עַל יְדֵי אַהֲרֹן וְעַל יְדֵי בְּנוֹהִי וַאֲרֵים יָתְהוֹן אֲרָמָא קֳדָם יְיָ׃}
{And he put the whole upon the hands of Aaron, and upon the hands of his sons, and waved them for a wave-offering before the \lord.}{\arabic{verse}}
\threeverse{\arabic{verse}}%Leviticus8:28
{וַיִּקַּ֨ח מֹשֶׁ֤ה אֹתָם֙ מֵעַ֣ל כַּפֵּיהֶ֔ם וַיַּקְטֵ֥ר הַמִּזְבֵּ֖חָה עַל\maqqaf הָעֹלָ֑ה מִלֻּאִ֥ים הֵם֙ לְרֵ֣יחַ נִיחֹ֔חַ אִשֶּׁ֥ה ה֖וּא לַיהֹוָֽה׃}
{וּנְסֵיב מֹשֶׁה יָתְהוֹן מֵעַל יַדְהוֹן וְאַסֵּיק לְמַדְבְּחָא עַל עֲלָתָא קוּרְבָּנַיָּא אִנּוּן לְאִתְקַבָּלָא בְּרַעֲוָא קוּרְבָּנָא הוּא קֳדָם יְיָ׃}
{And Moses took them from off their hands, and made them smoke on the altar upon the burnt-offering; they were a consecration-offering for a sweet savour; it was an offering made by fire unto the \lord.}{\arabic{verse}}
\rashi{\rashiDH{ויקטר המזבחה.} משה שמש כל שבעת ימי המלואים בחלוק לבן (עבודה זרה לד.)׃ 
\quad \rashiDH{על העולה.} אחר העולה ולא מצינו שוק של שלמים קרב בכל מקום חוץ מזה׃ 
}
\threeverse{\arabic{verse}}%Leviticus8:29
{וַיִּקַּ֤ח מֹשֶׁה֙ אֶת\maqqaf הֶ֣חָזֶ֔ה וַיְנִיפֵ֥הוּ תְנוּפָ֖ה לִפְנֵ֣י יְהֹוָ֑ה מֵאֵ֣יל הַמִּלֻּאִ֗ים לְמֹשֶׁ֤ה הָיָה֙ לְמָנָ֔ה כַּאֲשֶׁ֛ר צִוָּ֥ה יְהֹוָ֖ה אֶת\maqqaf מֹשֶֽׁה׃}
{וּנְסֵיב מֹשֶׁה יָת חַדְיָא וַאֲרִימֵיהּ אֲרָמָא קֳדָם יְיָ מִדְּכַר קוּרְבָּנַיָּא לְמֹשֶׁה הֲוָה לֻחְלָק כְּמָא דְּפַקֵּיד יְיָ יָת מֹשֶׁה׃}
{And Moses took the breast, and waved it for a wave-offering before the \lord; it was Moses’ portion of the ram of consecration; as the \lord\space commanded Moses.}{\arabic{verse}}
\aliyacounter{שביעי}
\threeverse{\aliya{שביעי}}%Leviticus8:30
{וַיִּקַּ֨ח מֹשֶׁ֜ה מִשֶּׁ֣מֶן הַמִּשְׁחָ֗ה וּמִן\maqqaf הַדָּם֮ אֲשֶׁ֣ר עַל\maqqaf הַמִּזְבֵּ֒חַ֒ וַיַּ֤ז עַֽל\maqqaf אַהֲרֹן֙ עַל\maqqaf בְּגָדָ֔יו וְעַל\maqqaf בָּנָ֛יו וְעַל\maqqaf בִּגְדֵ֥י בָנָ֖יו אִתּ֑וֹ וַיְקַדֵּ֤שׁ אֶֽת\maqqaf אַהֲרֹן֙ אֶת\maqqaf בְּגָדָ֔יו וְאֶת\maqqaf בָּנָ֛יו וְאֶת\maqqaf בִּגְדֵ֥י בָנָ֖יו אִתּֽוֹ׃}
{וּנְסֵיב מֹשֶׁה מִמִּשְׁחָא דִּרְבוּתָא וּמִן דְּמָא דְּעַל מַדְבְּחָא וְאַדִּי עַל אַהֲרֹן עַל לְבוּשׁוֹהִי וְעַל בְּנוֹהִי וְעַל לְבוּשֵׁי בְּנוֹהִי עִמֵּיהּ וְקַדֵּישׁ יָת אַהֲרֹן יָת לְבוּשׁוֹהִי וְיָת בְּנוֹהי וְיָת לְבוּשֵׁי בְּנוֹהִי עִמֵּיהּ׃}
{And Moses took of the anointing oil, and of the blood which was upon the altar, and sprinkled it upon Aaron, and upon his garments, and upon his sons, and upon his sons’ garments with him, and sanctified Aaron, and his garments, and his sons, and his sons’ garments with him.}{\arabic{verse}}
\threeverse{\arabic{verse}}%Leviticus8:31
{וַיֹּ֨אמֶר מֹשֶׁ֜ה אֶל\maqqaf אַהֲרֹ֣ן וְאֶל\maqqaf בָּנָ֗יו בַּשְּׁל֣וּ אֶת\maqqaf הַבָּשָׂר֮ פֶּ֣תַח אֹ֣הֶל מוֹעֵד֒ וְשָׁם֙ תֹּאכְל֣וּ אֹת֔וֹ וְאֶ֨ת\maqqaf הַלֶּ֔חֶם אֲשֶׁ֖ר בְּסַ֣ל הַמִּלֻּאִ֑ים כַּאֲשֶׁ֤ר צִוֵּ֙יתִי֙ לֵאמֹ֔ר אַהֲרֹ֥ן וּבָנָ֖יו יֹאכְלֻֽהוּ׃}
{וַאֲמַר מֹשֶׁה לְאַהֲרֹן וְלִבְנוֹהִי בַּשִּׁילוּ יָת בִּשְׂרָא בִּתְרַע מַשְׁכַּן זִמְנָא וְתַמָּן תֵּיכְלוּן יָתֵיהּ וְיָת לַחְמָא דִּבְסַל קוּרְבָּנַיָּא כְּמָא דְּפַקֵּידִית לְמֵימַר אַהֲרֹן וּבְנוֹהִי יֵיכְלוּנֵּיהּ׃}
{And Moses said unto Aaron and to his sons: ‘Boil the flesh at the door of the tent of meeting; and there eat it and the bread that is in the basket of consecration, as I commanded, saying: Aaron and his sons shall eat it.}{\arabic{verse}}
\threeverse{\arabic{verse}}%Leviticus8:32
{וְהַנּוֹתָ֥ר בַּבָּשָׂ֖ר וּבַלָּ֑חֶם בָּאֵ֖שׁ תִּשְׂרֹֽפוּ׃}
{וּדְיִשְׁתְּאַר בְּבִשְׂרָא וּבְלַחְמָא בְּנוּרָא תֵּיקְדוּן׃}
{And that which remaineth of the flesh and of the bread shall ye burn with fire.}{\arabic{verse}}
\threeverse{\aliya{מפטיר}}%Leviticus8:33
{וּמִפֶּ֩תַח֩ אֹ֨הֶל מוֹעֵ֜ד לֹ֤א תֵֽצְאוּ֙ שִׁבְעַ֣ת יָמִ֔ים עַ֚ד י֣וֹם מְלֹ֔את יְמֵ֖י מִלֻּאֵיכֶ֑ם כִּ֚י שִׁבְעַ֣ת יָמִ֔ים יְמַלֵּ֖א אֶת\maqqaf יֶדְכֶֽם׃}
{וּמִתְּרַע מַשְׁכַּן זִמְנָא לָא תִפְּקוּן שִׁבְעָא יוֹמִין עַד יוֹם מִשְׁלַם יוֹמֵי קוּרְבָּנְכוֹן אֲרֵי שִׁבְעָא יוֹמִין יִתְקָרַב קוּרְבָּנְכוֹן׃}
{And ye shall not go out from the door of the tent of meeting seven days, until the days of your consecration be fulfilled; for He shall consecrate you seven days.}{\arabic{verse}}
\threeverse{\arabic{verse}}%Leviticus8:34
{כַּאֲשֶׁ֥ר עָשָׂ֖ה בַּיּ֣וֹם הַזֶּ֑ה צִוָּ֧ה יְהֹוָ֛ה לַעֲשֹׂ֖ת לְכַפֵּ֥ר עֲלֵיכֶֽם׃}
{כְּמָא דַּעֲבַד בְּיוֹמָא הָדֵין פַּקֵּיד יְיָ לְמֶעֱבַד לְכַפָּרָא עֲלֵיכוֹן׃}
{As hath been done this day, so the \lord\space hath commanded to do, to make atonement for you.}{\arabic{verse}}
\rashi{\rashiDH{צוה ה׳ לעשות.} כל שבעת הימים ורז״ל דרשו (יומא ב׃) לעשות, זה מעשה פרה, לכפר זה מעשה יום הכפורים, וללמד שכהן גדול טעון פרישה קודם יום הכפורים שבעת ימים, וכן הכהן השורף את הפרה׃ 
}
\threeverse{\arabic{verse}}%Leviticus8:35
{וּפֶ֩תַח֩ אֹ֨הֶל מוֹעֵ֜ד תֵּשְׁב֨וּ יוֹמָ֤ם וָלַ֙יְלָה֙ שִׁבְעַ֣ת יָמִ֔ים וּשְׁמַרְתֶּ֛ם אֶת\maqqaf מִשְׁמֶ֥רֶת יְהֹוָ֖ה וְלֹ֣א תָמ֑וּתוּ כִּי\maqqaf כֵ֖ן צֻוֵּֽיתִי׃}
{וּבִתְרַע מַשְׁכַּן זִמְנָא תִּתְּבוּן יֵימָם וְלֵילֵי שִׁבְעָא יוֹמִין וְתִטְּרוּן יָת מַטְּרַת מֵימְרָא דַּייָ וְלָא תְמוּתוּן אֲרֵי כֵן אִתְפַּקַּדִית׃}
{And at the door of the tent of meeting shall ye abide day and night seven days, and keep the charge of the \lord, that ye die not; for so I am commanded.}{\arabic{verse}}
\rashi{\rashiDH{ולא תמותו.} הא אם לא תעשו כן, רי אתם חייבים מיתה׃}
\threeverse{\aliya{\Hebrewnumeral{97}}}%Leviticus8:36
{וַיַּ֥עַשׂ אַהֲרֹ֖ן וּבָנָ֑יו אֵ֚ת כׇּל\maqqaf הַדְּבָרִ֔ים אֲשֶׁר\maqqaf צִוָּ֥ה יְהֹוָ֖ה בְּיַד\maqqaf מֹשֶֽׁה׃ \setuma }
{וַעֲבַד אַהֲרֹן וּבְנוֹהִי יָת כָּל פִּתְגָמַיָּא דְּפַקֵּיד יְיָ בִּידָא דְּמֹשֶׁה׃}
{And Aaron and his sons did all the things which the \lord\space commanded by the hand of Moses.}{\arabic{verse}}
\rashi{\rashiDH{ויעש אהרן ובניו.} להגיד שבחן שלא הטו ימין ושמאל׃ 
}
\engnote{The Haftarah is Jeremiah 7:21\verserangechar 8:3 \tworange 9:22\verserangechar 9:23 on page \pageref{haft_25}. For Shabbat Zachor the maftir and Haftara are on page \pageref{maft_zachor}. On Shabbat Parah, read Maftir and Haftara on page \pageref{maft_parah}.  On the Shabbat before Pesa\d{h}, read the Haftara on page \pageref{haft_hagadol}.}
\newperek
\aliyacounter{ראשון}
\newparsha{שמיני}
\threeverse{\aliya{שמיני}}%Leviticus9:1
{וַיְהִי֙ בַּיּ֣וֹם הַשְּׁמִינִ֔י קָרָ֣א מֹשֶׁ֔ה לְאַהֲרֹ֖ן וּלְבָנָ֑יו וּלְזִקְנֵ֖י יִשְׂרָאֵֽל׃}
{וַהֲוָה בְּיוֹמָא תְּמִינָאָה קְרָא מֹשֶׁה לְאַהֲרֹן וְלִבְנוֹהִי וּלְסָבֵי יִשְׂרָאֵל׃}
{And it came to pass on the eighth day, that Moses called Aaron and his sons, and the elders of Israel;}{\Roman{chap}}
\rashi{\rashiDH{ויהי ביום השמיני.} שמיני למלואים, הוא ר״ח ניסן, שהוקם המשכן בו ביום, ונטל י׳ עטרות השנויות בסדר עולם (פרק ז)׃\quad \rashiDH{ולזקני ישראל.} להשמיעם שעל פי הדבור אהרן נכנס ומשמש בכהונה גדולה, ולא יאמרו מאליו נכנס׃ 
}
\threeverse{\arabic{verse}}%Leviticus9:2
{וַיֹּ֣אמֶר אֶֽל\maqqaf אַהֲרֹ֗ן קַח\maqqaf לְ֠ךָ֠ עֵ֣גֶל בֶּן\maqqaf בָּקָ֧ר לְחַטָּ֛את וְאַ֥יִל לְעֹלָ֖ה תְּמִימִ֑ם וְהַקְרֵ֖ב לִפְנֵ֥י יְהֹוָֽה׃}
{וַאֲמַר לְאַהֲרֹן סַב לָךְ עֵיגַל בַּר תּוֹרֵי לְחַטָּתָא וּדְכַר לַעֲלָתָא שַׁלְמִין וְקָרֵיב קֳדָם יְיָ׃}
{and he said unto Aaron: ‘Take thee a bull-calf for a sin-offering, and a ram for a burnt-offering, without blemish, and offer them before the \lord.}{\arabic{verse}}
\rashi{\rashiDH{קח לך עגל.} להודיע שמכפר לו הקב״ה ע״י עגל זה על מעשה העגל שעשה׃ 
}
\threeverse{\arabic{verse}}%Leviticus9:3
{וְאֶל\maqqaf בְּנֵ֥י יִשְׂרָאֵ֖ל תְּדַבֵּ֣ר לֵאמֹ֑ר קְח֤וּ שְׂעִיר\maqqaf עִזִּים֙ לְחַטָּ֔את וְעֵ֨גֶל וָכֶ֧בֶשׂ בְּנֵי\maqqaf שָׁנָ֛ה תְּמִימִ֖ם לְעֹלָֽה׃}
{וְעִם בְּנֵי יִשְׂרָאֵל תְּמַלֵּיל לְמֵימַר סַבוּ צְפִיר בַּר עִזִּין לְחַטָּתָא וְעֵיגַל וְאִמַּר בְּנֵי שְׁנָא שַׁלְמִין לַעֲלָתָא׃}
{And unto the children of Israel thou shalt speak, saying: Take ye a he-goat for a sin-offering; and a calf and a lamb, both of the first year, without blemish, for a burnt-offering;}{\arabic{verse}}
\threeverse{\arabic{verse}}%Leviticus9:4
{וְשׁ֨וֹר וָאַ֜יִל לִשְׁלָמִ֗ים לִזְבֹּ֙חַ֙ לִפְנֵ֣י יְהֹוָ֔ה וּמִנְחָ֖ה בְּלוּלָ֣ה בַשָּׁ֑מֶן כִּ֣י הַיּ֔וֹם יְהֹוָ֖ה נִרְאָ֥ה אֲלֵיכֶֽם׃}
{וְתוֹר וּדְכַר לְנִכְסַת קוּדְשַׁיָּא לְדַבָּחָא קֳדָם יְיָ וּמִנְחָתָא דְּפִילָא בִּמְשַׁח אֲרֵי יוֹמָא דֵין יְקָרָא דַּייָ מִתְגְּלֵי לְכוֹן׃}
{and an ox and a ram for peace-offerings, to sacrifice before the \lord; and a meal-offering mingled with oil; for to-day the \lord\space appeareth unto you.’}{\arabic{verse}}
\rashi{\rashiDH{כי היום ה׳ נראה אליכם.} להשרות שכינתו במעשה ידיכם, לכך קרבנות הללו באין חובה ליום זה׃}
\threeverse{\arabic{verse}}%Leviticus9:5
{וַיִּקְח֗וּ אֵ֚ת אֲשֶׁ֣ר צִוָּ֣ה מֹשֶׁ֔ה אֶל\maqqaf פְּנֵ֖י אֹ֣הֶל מוֹעֵ֑ד וַֽיִּקְרְבוּ֙ כׇּל\maqqaf הָ֣עֵדָ֔ה וַיַּֽעַמְד֖וּ לִפְנֵ֥י יְהֹוָֽה׃}
{וְקָרִיבוּ יָת דְּפַקֵּיד מֹשֶׁה לִקְדָם מַשְׁכַּן זִמְנָא וּקְרִיבוּ כָּל כְּנִשְׁתָּא וְקָמוּ קֳדָם יְיָ׃}
{And they brought that which Moses commanded before the tent of meeting; and all the congregation drew near and stood before the \lord.}{\arabic{verse}}
\threeverse{\arabic{verse}}%Leviticus9:6
{וַיֹּ֣אמֶר מֹשֶׁ֔ה זֶ֧ה הַדָּבָ֛ר אֲשֶׁר\maqqaf צִוָּ֥ה יְהֹוָ֖ה תַּעֲשׂ֑וּ וְיֵרָ֥א אֲלֵיכֶ֖ם כְּב֥וֹד יְהֹוָֽה׃}
{וַאֲמַר מֹשֶׁה דֵּין פִּתְגָמָא דְּפַקֵּיד יְיָ תַּעְבְּדוּן וְיִתְגְּלֵי לְכוֹן יְקָרָא דַּייָ׃}
{And Moses said: ‘This is the thing which the \lord\space commanded that ye should do; that the glory of the \lord\space may appear unto you.’}{\arabic{verse}}
\threeverse{\aliya{לוי}}%Leviticus9:7
{וַיֹּ֨אמֶר מֹשֶׁ֜ה אֶֽל\maqqaf אַהֲרֹ֗ן קְרַ֤ב אֶל\maqqaf הַמִּזְבֵּ֙חַ֙ וַעֲשֵׂ֞ה אֶת\maqqaf חַטָּֽאתְךָ֙ וְאֶת\maqqaf עֹ֣לָתֶ֔ךָ וְכַפֵּ֥ר בַּֽעַדְךָ֖ וּבְעַ֣ד הָעָ֑ם וַעֲשֵׂ֞ה אֶת\maqqaf קׇרְבַּ֤ן הָעָם֙ וְכַפֵּ֣ר בַּֽעֲדָ֔ם כַּאֲשֶׁ֖ר צִוָּ֥ה יְהֹוָֽה׃}
{וַאֲמַר מֹשֶׁה לְאַהֲרֹן קְרַב לְמַדְבְּחָא וַעֲבֵיד יָת חַטָּתָךְ וְיָת עֲלָתָךְ וְכַפַּר עֲלָךְ וְעַל עַמָּא וַעֲבֵיד יָת קוּרְבַּן עַמָּא וְכַפַּר עֲלֵיהוֹן כְּמָא דְּפַקֵּיד יְיָ׃}
{And Moses said unto Aaron: ‘Draw near unto the altar, and offer thy sin-offering, and thy burnt-offering, and make atonement for thyself, and for the people; and present the offering of the people, and make atonement for them; as the \lord\space commanded.’}{\arabic{verse}}
\rashi{\rashiDH{קרב אל המזבח.} שהיה אהרן בוש וירא לגשת, אמר לו משה למה אתה בוש, לכך נבחרת (ת״כ פרשתא א, ח)׃\quad \rashiDH{את חטאתך.} עגל בן בקר׃ 
\quad \rashiDH{ואת עולתך.} איל׃\quad \rashiDH{קרבן העם.} שעיר עזים ועגל וכבש. כל מקום שנאמר עגל בן שנה הוא, ומכאן אתה למד׃}
\threeverse{\arabic{verse}}%Leviticus9:8
{וַיִּקְרַ֥ב אַהֲרֹ֖ן אֶל\maqqaf הַמִּזְבֵּ֑חַ וַיִּשְׁחַ֛ט אֶת\maqqaf עֵ֥גֶל הַחַטָּ֖את אֲשֶׁר\maqqaf לֽוֹ׃}
{וּקְרֵיב אַהֲרֹן לְמַדְבְּחָא וּנְכַס יָת עִגְלָא דְּחַטָּתָא דִּילֵיהּ׃}
{So Aaron drew near unto the altar, and slew the calf of the sin-offering, which was for himself.}{\arabic{verse}}
\threeverse{\arabic{verse}}%Leviticus9:9
{וַ֠יַּקְרִ֠בוּ בְּנֵ֨י אַהֲרֹ֣ן אֶת\maqqaf הַדָּם֮ אֵלָיו֒ וַיִּטְבֹּ֤ל אֶצְבָּעוֹ֙ בַּדָּ֔ם וַיִּתֵּ֖ן עַל\maqqaf קַרְנ֣וֹת הַמִּזְבֵּ֑חַ וְאֶת\maqqaf הַדָּ֣ם יָצַ֔ק אֶל\maqqaf יְס֖וֹד הַמִּזְבֵּֽחַ׃}
{וְקָרִיבוּ בְּנֵי אַהֲרֹן יָת דְּמָא לֵיהּ וּטְבַל אֶצְבְּעֵיהּ בִּדְמָא וִיהַב עַל קַרְנָת מַדְבְּחָא וְיָת דְּמָא אֲרֵיק לִיסוֹדָא דְּמַדְבְּחָא׃}
{And the sons of Aaron presented the blood unto him; and he dipped his finger in the blood, and put it upon the horns of the altar, and poured out the blood at the base of the altar.}{\arabic{verse}}
\threeverse{\arabic{verse}}%Leviticus9:10
{וְאֶת\maqqaf הַחֵ֨לֶב וְאֶת\maqqaf הַכְּלָיֹ֜ת וְאֶת\maqqaf הַיֹּתֶ֤רֶת מִן\maqqaf הַכָּבֵד֙ מִן\maqqaf הַ֣חַטָּ֔את הִקְטִ֖יר הַמִּזְבֵּ֑חָה כַּאֲשֶׁ֛ר צִוָּ֥ה יְהֹוָ֖ה אֶת\maqqaf מֹשֶֽׁה׃}
{וְיָת תַּרְבָּא וְיָת כּוֹלְיָתָא וְיָת חַצְרָא מִן כַּבְדָּא מִן חַטָּתָא אַסֵּיק לְמַדְבְּחָא כְּמָא דְּפַקֵּיד יְיָ יָת מֹשֶׁה׃}
{But the fat, and the kidneys, and the lobe of the liver of the sin-offering, he made smoke upon the altar; as the \lord\space commanded Moses.}{\arabic{verse}}
\threeverse{\aliya{ישראל}}%Leviticus9:11
{וְאֶת\maqqaf הַבָּשָׂ֖ר וְאֶת\maqqaf הָע֑וֹר שָׂרַ֣ף בָּאֵ֔שׁ מִח֖וּץ לַֽמַּחֲנֶֽה׃}
{וְיָת בִּשְׂרָא וְיָת מַשְׁכָּא אוֹקֵיד בְּנוּרָא מִבַּרָא לְמַשְׁרִיתָא׃}
{And the flesh and the skin were burnt with fire without the camp.}{\arabic{verse}}
\rashi{\rashiDH{ואת הבשר ואת העור וגו׳.} לא מצינו חטאת חיצונה נשרפת אלא זו, ושל מלואים, וכולן על פי הדבור׃}
\threeverse{\arabic{verse}}%Leviticus9:12
{וַיִּשְׁחַ֖ט אֶת\maqqaf הָעֹלָ֑ה וַ֠יַּמְצִ֠אוּ בְּנֵ֨י אַהֲרֹ֤ן אֵלָיו֙ אֶת\maqqaf הַדָּ֔ם וַיִּזְרְקֵ֥הוּ עַל\maqqaf הַמִּזְבֵּ֖חַ סָבִֽיב׃}
{וּנְכַס יָת עֲלָתָא וְאַמְטִיאוּ בְּנֵי אַהֲרֹן לֵיהּ יָת דְּמָא וְזַרְקֵיהּ עַל מַדְבְּחָא סְחוֹר סְחוֹר׃}
{And he slew the burnt-offering; and Aaron’s sons delivered unto him the blood, and he dashed it against the altar round about.}{\arabic{verse}}
\rashi{\rashiDH{וימציאו.} לשון הושטה והזמנה׃}
\threeverse{\arabic{verse}}%Leviticus9:13
{וְאֶת\maqqaf הָעֹלָ֗ה הִמְצִ֧יאוּ אֵלָ֛יו לִנְתָחֶ֖יהָ וְאֶת\maqqaf הָרֹ֑אשׁ וַיַּקְטֵ֖ר עַל\maqqaf הַמִּזְבֵּֽחַ׃}
{וְיָת עֲלָתָא אַמְטִיאוּ לֵיהּ לְאֶבְרַהָא וְיָת רֵישָׁא וְאַסֵּיק עַל מַדְבְּחָא׃}
{And they delivered the burnt-offering unto him, piece by piece, and the head; and he made them smoke upon the altar.}{\arabic{verse}}
\threeverse{\arabic{verse}}%Leviticus9:14
{וַיִּרְחַ֥ץ אֶת\maqqaf הַקֶּ֖רֶב וְאֶת\maqqaf הַכְּרָעָ֑יִם וַיַּקְטֵ֥ר עַל\maqqaf הָעֹלָ֖ה הַמִּזְבֵּֽחָה׃}
{וְחַלֵּיל יָת גַּוָּא וְיָת כְּרָעַיָּא וְאַסֵּיק עַל עֲלָתָא לְמַדְבְּחָא׃}
{And he washed the inwards and the legs, and made them smoke upon the burnt-offering on the altar.}{\arabic{verse}}
\threeverse{\arabic{verse}}%Leviticus9:15
{וַיַּקְרֵ֕ב אֵ֖ת קׇרְבַּ֣ן הָעָ֑ם וַיִּקַּ֞ח אֶת\maqqaf שְׂעִ֤יר הַֽחַטָּאת֙ אֲשֶׁ֣ר לָעָ֔ם וַיִּשְׁחָטֵ֥הוּ וַֽיְחַטְּאֵ֖הוּ כָּרִאשֽׁוֹן׃}
{וְקָרֵיב יָת קוּרְבַּן עַמָּא וּנְסֵיב יָת צְפִירָא דְּחַטָּתָא דִּלְעַמָּא וְנַכְסֵיהּ וְכַפַּר בִּדְמֵיהּ כְּקַדְמָאָה׃}
{And the people’s offering was presented; and he took the goat of the sin-offering which was for the people, and slew it, and offered it for sin, as the first.}{\arabic{verse}}
\rashi{\rashiDH{ויחטאהו.} עשהו כמשפט חטאת׃\quad \rashiDH{כראשון.} כעגל שלו׃}
\threeverse{\arabic{verse}}%Leviticus9:16
{וַיַּקְרֵ֖ב אֶת\maqqaf הָעֹלָ֑ה וַֽיַּעֲשֶׂ֖הָ כַּמִּשְׁפָּֽט׃}
{וְקָרֵיב יָת עֲלָתָא וְעַבְדַהּ כְּדַחְזֵי׃}
{And the burnt-offering was presented; and he offered it according to the ordinance.}{\arabic{verse}}
\rashi{\rashiDH{ויעשה כמשפט.} המפורש בעולת נדבה בויקרא (ביצה כ.)׃ 
}
\aliyacounter{שני}
\threeverse{\aliya{שני}}%Leviticus9:17
{וַיַּקְרֵב֮ אֶת\maqqaf הַמִּנְחָה֒ וַיְמַלֵּ֤א כַפּוֹ֙ מִמֶּ֔נָּה וַיַּקְטֵ֖ר עַל\maqqaf הַמִּזְבֵּ֑חַ מִלְּבַ֖ד עֹלַ֥ת הַבֹּֽקֶר׃}
{וְקָרֵיב יָת מִנְחָתָא וּמְלָא יְדֵיהּ מִנַּהּ וְאַסֵּיק עַל מַדְבְּחָא בָּר מֵעֲלַת צַפְרָא׃}
{And the meal-offering was presented; and he filled his hand therefrom, and made it smoke upon the altar, besides the burnt-offering of the morning.}{\arabic{verse}}
\rashi{\rashiDH{וימלא כפו.} היא קמיצה׃\quad \rashiDH{מלבד עלת הבקר.} כל אלה עשה אחר עולת התמיד׃ 
}
\threeverse{\arabic{verse}}%Leviticus9:18
{וַיִּשְׁחַ֤ט אֶת\maqqaf הַשּׁוֹר֙ וְאֶת\maqqaf הָאַ֔יִל זֶ֥בַח הַשְּׁלָמִ֖ים אֲשֶׁ֣ר לָעָ֑ם וַ֠יַּמְצִ֠אוּ בְּנֵ֨י אַהֲרֹ֤ן אֶת\maqqaf הַדָּם֙ אֵלָ֔יו וַיִּזְרְקֵ֥הוּ עַל\maqqaf הַמִּזְבֵּ֖חַ סָבִֽיב׃}
{וּנְכַס יָת תּוֹרָא וְיָת דִּכְרָא נִכְסַת קוּדְשַׁיָּא דִּלְעַמָּא וְאַמְטִיאוּ בְּנֵי אַהֲרֹן יָת דְּמָא לֵיהּ וְזַרְקֵיהּ עַל מַדְבְּחָא סְחוֹר סְחוֹר׃}
{He slew also the ox and the ram, the sacrifice of peace-offerings, which was for the people; and Aaron’s sons delivered unto him the blood, and he dashed it against the altar round about,}{\arabic{verse}}
\threeverse{\arabic{verse}}%Leviticus9:19
{וְאֶת\maqqaf הַחֲלָבִ֖ים מִן\maqqaf הַשּׁ֑וֹר וּמִ֨ן\maqqaf הָאַ֔יִל הָֽאַלְיָ֤ה וְהַֽמְכַסֶּה֙ וְהַכְּלָיֹ֔ת וְיֹתֶ֖רֶת הַכָּבֵֽד׃}
{וְיָת תַּרְבַּיָּא מִן תּוֹרָא וּמִן דִּכְרָא אַלְיְתָא וְחָפֵי גַּוָּא וְכוֹלְיָתָא וַחֲצַר כַּבְדָּא׃}
{and the fat of the ox, and of the ram, the fat tail, and that which covereth the inwards, and the kidneys, and the lobe of the liver.}{\arabic{verse}}
\rashi{\rashiDH{והמכסה.} חלב המכסה את הקרב׃}
\threeverse{\arabic{verse}}%Leviticus9:20
{וַיָּשִׂ֥ימוּ אֶת\maqqaf הַחֲלָבִ֖ים עַל\maqqaf הֶחָז֑וֹת וַיַּקְטֵ֥ר הַחֲלָבִ֖ים הַמִּזְבֵּֽחָה׃}
{וְשַׁוִּיאוּ יָת תַּרְבַּיָּא עַל חֲדָוָתָא וְאַסֵּיק תַּרְבַּיָּא לְמַדְבְּחָא׃}
{And they put the fat upon the breasts, and he made the fat smoke upon the altar.}{\arabic{verse}}
\rashi{\rashiDH{וישימו את החלבים על החזות.} לאחר התנופה נתנן כהן המניף לכהן אחר להקטירם, נמצאו העליונים למטה (מנחות סב.)׃ 
}
\threeverse{\arabic{verse}}%Leviticus9:21
{וְאֵ֣ת הֶחָז֗וֹת וְאֵת֙ שׁ֣וֹק הַיָּמִ֔ין הֵנִ֧יף אַהֲרֹ֛ן תְּנוּפָ֖ה לִפְנֵ֣י יְהֹוָ֑ה כַּאֲשֶׁ֖ר צִוָּ֥ה מֹשֶֽׁה׃}
{וְיָת חֲדָוָתָא וְיָת שָׁקָא דְּיַמִּינָא אֲרֵים אַהֲרֹן אֲרָמָא קֳדָם יְיָ כְּמָא דְּפַקֵּיד מֹשֶׁה׃}
{And the breasts and the right thigh Aaron waved for a wave-offering before the \lord; as Moses commanded.}{\arabic{verse}}
\threeverse{\arabic{verse}}%Leviticus9:22
{וַיִּשָּׂ֨א אַהֲרֹ֧ן אֶת\maqqaf יָדָ֛ו אֶל\maqqaf הָעָ֖ם וַֽיְבָרְכֵ֑ם וַיֵּ֗רֶד מֵעֲשֹׂ֧ת הַֽחַטָּ֛את וְהָעֹלָ֖ה וְהַשְּׁלָמִֽים׃}
{וַאֲרֵים אַהֲרֹן יָת יְדוֹהִי לְעַמָּא וּבָרֵיכִנּוּן וּנְחַת מִלְּמֶעֱבַד חַטָּתָא וַעֲלָתָא וְנִכְסַת קוּדְשַׁיָּא׃}
{And Aaron lifted up his hands toward the people, and blessed them; and he came down from offering the sin-offering, and the burnt-offering, and the peace-offerings.}{\arabic{verse}}
\rashi{\rashiDH{ויברכם.} ברכת כהנים, יברכך, יאר, ישא׃\quad \rashiDH{וירד.} מעל המזבח׃ 
}
\threeverse{\arabic{verse}}%Leviticus9:23
{וַיָּבֹ֨א מֹשֶׁ֤ה וְאַהֲרֹן֙ אֶל\maqqaf אֹ֣הֶל מוֹעֵ֔ד וַיֵּ֣צְא֔וּ וַֽיְבָרְכ֖וּ אֶת\maqqaf הָעָ֑ם וַיֵּרָ֥א כְבוֹד\maqqaf יְהֹוָ֖ה אֶל\maqqaf כׇּל\maqqaf הָעָֽם׃}
{וְעָאל מֹשֶׁה וְאַהֲרֹן לְמַשְׁכַּן זִמְנָא וּנְפַקוּ וּבָרִיכוּ יָת עַמָּא וְאִתְגְּלִי יְקָרָא דַּייָ לְכָל עַמָּא׃}
{And Moses and Aaron went into the tent of meeting, and came out, and blessed the people; and the glory of the \lord\space appeared unto all the people.}{\arabic{verse}}
\rashi{\rashiDH{ויבא משה ואהרן וגו׳.} למה נכנסו, מצאתי בפרשת מלואים בברייתא הנוספת על תורת כהנים שלנו, למה נכנס משה עם אהרן, ללמדו על מעשה הקטרת, או לא נכנס אלא לדבר אחר, הריני דן ירידה וביאה טעונות ברכה, מה ירידה מעין עבודה, אף ביאה מעין עבודה, הא למדת למה נכנס משה עם אהרן למדו על מעשה הקטרת. דבר אחר כיון שראה אהרן שקרבו כל הקרבנות ונעשו כל המעשים, ולא ירדה שכינה לישראל, היה מצטער ואומר, יודע אני שכעס הקב״ה עלי, ובשבילי לא ירדה שכינה לישראל, אמר לו למשה משה אחי כך עשית לי שנכנסתי ונתביישתי, מיד נכנס משה עמו ובקשו רחמים וירדה שכינה לישראל׃ 
\quad \rashiDH{ויצאו ויברכו את העם.} אמרו וִיהִי נֹעַם ה׳ אֱלֹהֵינוּ עָלֵינוּ (תהלים צ, יז), יהי רצון שתשרה שכינה במעשה ידיכם. לפי שכל ז׳ ימי המלואים שהעמידו משה למשכן ושמש בו, ופרקו בכל יום, לא שרתה בו שכינה, והיו ישראל נכלמים ואומרים למשה, משה רבינו כל הטורח שטרחנו שתשרה שכינה בינינו ונדע שנתכפר לנו עון העגל, לכך אמר להם זה הדבר אשר צוה ה׳ תעשו וירא אליכם כבוד ה׳, אהרן אחי כדאי וחשוב ממני, שעל ידי קרבנותיו ועבודתו תשרה שכינה בכם, ותדעו שהמקום בחר בו׃}
\aliyacounter{שלישי}
\threeverse{\aliya{שלישי}}%Leviticus9:24
{וַתֵּ֤צֵא אֵשׁ֙ מִלִּפְנֵ֣י יְהֹוָ֔ה וַתֹּ֙אכַל֙ עַל\maqqaf הַמִּזְבֵּ֔חַ אֶת\maqqaf הָעֹלָ֖ה וְאֶת\maqqaf הַחֲלָבִ֑ים וַיַּ֤רְא כׇּל\maqqaf הָעָם֙ וַיָּרֹ֔נּוּ וַֽיִּפְּל֖וּ עַל\maqqaf פְּנֵיהֶֽם׃}
{וּנְפַקַת אִישָׁתָא מִן קֳדָם יְיָ וַאֲכַלַת עַל מַדְבְּחָא יָת עֲלָתָא וְיָת תַּרְבַּיָּא וַחֲזָא כָּל עַמָּא וְשַׁבַּחוּ וּנְפַלוּ עַל אַפֵּיהוֹן׃}
{And there came forth fire from before the \lord, and consumed upon the altar the burnt-offering and the fat; and when all the people saw it, they shouted, and fell on their faces.}{\arabic{verse}}
\rashi{\rashiDH{וירנו.} כתרגומו׃}
\newperek
\threeverse{\Roman{chap}}%Leviticus10:1
{וַיִּקְח֣וּ בְנֵֽי\maqqaf אַ֠הֲרֹ֠ן נָדָ֨ב וַאֲבִיה֜וּא אִ֣ישׁ מַחְתָּת֗וֹ וַיִּתְּנ֤וּ בָהֵן֙ אֵ֔שׁ וַיָּשִׂ֥ימוּ עָלֶ֖יהָ קְטֹ֑רֶת וַיַּקְרִ֜יבוּ לִפְנֵ֤י יְהֹוָה֙ אֵ֣שׁ זָרָ֔ה אֲשֶׁ֧ר לֹ֦א צִוָּ֖ה אֹתָֽם׃}
{וּנְסִיבוּ בְנֵי אַהֲרֹן נָדָב וַאֲבִיהוּא גְּבַר מַחְתִּיתֵיהּ וִיהַבוּ בְהוֹן אִישָׁתָא וְשַׁוִּיאוּ עֲלַהּ קְטֹרֶת בֻּסְמִין וְקָרִיבוּ קֳדָם יְיָ אִישָׁתָא נוּכְרֵיתָא דְּלָא פַקֵּיד יָתְהוֹן׃}
{And Nadab and Abihu, the sons of Aaron, took each of them his censer, and put fire therein, and laid incense thereon, and offered strange fire before the \lord, which He had not commanded them.}{\Roman{chap}}
\threeverse{\arabic{verse}}%Leviticus10:2
{וַתֵּ֥צֵא אֵ֛שׁ מִלִּפְנֵ֥י יְהֹוָ֖ה וַתֹּ֣אכַל אוֹתָ֑ם וַיָּמֻ֖תוּ לִפְנֵ֥י יְהֹוָֽה׃}
{וּנְפַקַת אִישָׁתָא מִן קֳדָם יְיָ וַאֲכַלַת יָתְהוֹן וּמִיתוּ קֳדָם יְיָ׃}
{And there came forth fire from before the \lord, and devoured them, and they died before the \lord.}{\arabic{verse}}
\rashi{\rashiDH{ותצא אש.} רבי אליעזר אומר לא מתו בני אהרן אלא על ידי שהורו הלכה בפני משה רבן, רבי ישמעאל אומר שתויי יין נכנסו למקדש. תדע שאחר מיתתן הזהיר הנותרים שלא יכנסו שתויי יין למקדש, משל למלך שהיה לו בן בית וכו׳, כדאיתא בויקרא רבה (יב, א)׃ 
}
\threeverse{\arabic{verse}}%Leviticus10:3
{וַיֹּ֨אמֶר מֹשֶׁ֜ה אֶֽל\maqqaf אַהֲרֹ֗ן הוּא֩ אֲשֶׁר\maqqaf דִּבֶּ֨ר יְהֹוָ֤ה \pasek  לֵאמֹר֙ בִּקְרֹבַ֣י אֶקָּדֵ֔שׁ וְעַל\maqqaf פְּנֵ֥י כׇל\maqqaf הָעָ֖ם אֶכָּבֵ֑ד וַיִּדֹּ֖ם אַהֲרֹֽן׃}
{וַאֲמַר מֹשֶׁה לְאַהֲרֹן הוּא דְּמַלֵּיל יְיָ לְמֵימַר בְּקָרִיבַי אֶתְקַדַּשׁ וְעַל אַפֵּי כָּל עַמָּא אֶתְיַקַּר וּשְׁתֵיק אַהֲרֹן׃}
{Then Moses said unto Aaron: ‘This is it that the \lord\space spoke, saying: Through them that are nigh unto Me I will be sanctified, and before all the people I will be glorified.’ And Aaron held his peace.}{\arabic{verse}}
\rashi{\rashiDH{הוא אשר דבר וגו׳.} היכן דבר, וְנֹעַדְתִּי שָׁמָּה לִבְנֵי יִשְׂרָאֵל וְנִקְדָּשׁ בִּכְבֹדִי (שמות כט, מג), אל תקרי בכבודי אלא במכובדי (זבחים קטו׃). אמר משה לאהרן, אהרן אחי, יודע הייתי שיתקדש הבית במיודעיו של מקום, והייתי סבור או בי או בך, עכשיו רואה אני שהם גדולים ממני וממך (ת״כ פרשתא א, כג.  ויקרא רבה יב, ב)׃ 
\quad \rashiDH{וידום אהרן.} קבל שכר על שתיקתו, ומה שכר קבל, שנתייחד עמו הדבור, שנאמרה לו לבדו פרשת שתויי יין (ת״כ שם לו.  ויק״ר שם).\quad \rashiDH{בקרובי.} בבחירי׃\quad \rashiDH{ועל פני כל העם אכבד.} כשהקב״ה עושה דין בצדיקים מתיירא, ומתעלה, ומתקלס, אם כן באלו, כל שכן ברשעים, וכן הוא אומר נוֹרָא אֱלֹהִים מִמִּקְדָּשֶׁיךָ (תהלים סח, לו), אל תקרי ממקדשיך אלא ממקודשיך׃}
\threeverse{\arabic{verse}}%Leviticus10:4
{וַיִּקְרָ֣א מֹשֶׁ֗ה אֶל\maqqaf מִֽישָׁאֵל֙ וְאֶ֣ל אֶלְצָפָ֔ן בְּנֵ֥י עֻזִּיאֵ֖ל דֹּ֣ד אַהֲרֹ֑ן וַיֹּ֣אמֶר אֲלֵהֶ֗ם קִ֞֠רְב֞֠וּ שְׂא֤וּ אֶת\maqqaf אֲחֵיכֶם֙ מֵאֵ֣ת פְּנֵי\maqqaf הַקֹּ֔דֶשׁ אֶל\maqqaf מִח֖וּץ לַֽמַּחֲנֶֽה׃}
{וּקְרָא מֹשֶׁה לְמִישָׁאֵל וּלְאֶלְצָפָן בְּנֵי עֻזִּיאֵל אַחְבּוּהִי דְּאַהֲרֹן וַאֲמַר לְהוֹן קְרוּבוּ טוּלוּ יָת אֲחֵיכוֹן מִן קֳדָם קוּדְשָׁא לְמִבַּרָא לְמַשְׁרִיתָא׃}
{And Moses called Mishael and Elzaphan, the sons of Uzziel the uncle of Aaron, and said unto them: ‘Draw near, carry your brethren from before the sanctuary out of the camp.’}{\arabic{verse}}
\rashi{\rashiDH{דד אהרן.} עוזיאל אחי עמרם היה, שנאמר וּבְנֵי קְהָת וגו׳ (שמות ו, יח)׃\quad \rashiDH{שאו את אחיכם וגו׳.} כאדם האומר לחבירו, העבר את המת מלפני הכלה, שלא לערבב את השמחה׃ 
}
\threeverse{\arabic{verse}}%Leviticus10:5
{וַֽיִּקְרְב֗וּ וַיִּשָּׂאֻם֙ בְּכֻתֳּנֹתָ֔ם אֶל\maqqaf מִח֖וּץ לַֽמַּחֲנֶ֑ה כַּאֲשֶׁ֖ר דִּבֶּ֥ר מֹשֶֽׁה׃}
{וּקְרִיבוּ וּנְטַלוּנִין בְּכִתּוּנֵיהוֹן לְמִבַּרָא לְמַשְׁרִיתָא כְּמָא דְּמַלֵּיל מֹשֶׁה׃}
{So they drew near, and carried them in their tunics out of the camp, as Moses had said.}{\arabic{verse}}
\rashi{\rashiDH{בכתנתם.} של מתים, מלמד, שלא נשרפו בגדיהם, אלא נשמתם, כמין שני חוטין של אש נכנסו לתוך חוטמיהם (סנהדרין נב. שם כג)׃}
\threeverse{\arabic{verse}}%Leviticus10:6
{וַיֹּ֣אמֶר מֹשֶׁ֣ה אֶֽל\maqqaf אַהֲרֹ֡ן וּלְאֶלְעָזָר֩ וּלְאִֽיתָמָ֨ר \pasek  בָּנָ֜יו רָֽאשֵׁיכֶ֥ם אַל\maqqaf תִּפְרָ֣עוּ \legarmeh  וּבִגְדֵיכֶ֤ם לֹֽא\maqqaf תִפְרֹ֙מוּ֙ וְלֹ֣א תָמֻ֔תוּ וְעַ֥ל כׇּל\maqqaf הָעֵדָ֖ה יִקְצֹ֑ף וַאֲחֵיכֶם֙ כׇּל\maqqaf בֵּ֣ית יִשְׂרָאֵ֔ל יִבְכּוּ֙ אֶת\maqqaf הַשְּׂרֵפָ֔ה אֲשֶׁ֖ר שָׂרַ֥ף יְהֹוָֽה׃}
{וַאֲמַר מֹשֶׁה לְאַהֲרֹן וּלְאֶלְעָזָר וּלְאִיתָמָר בְּנוֹהִי רֵישֵׁיכוֹן לָא תְרַבּוֹן פֵּירוּעַ וּלְבוּשֵׁיכוֹן לָא תְבַזְּעוּן וְלָא תְמוּתוּן וְעַל כָּל כְּנִשְׁתָּא יְהֵי רוּגְזָא וַאֲחֵיכוֹן כָּל בֵּית יִשְׂרָאֵל יִבְכּוֹן יָת יְקֵידְתָא דְּאוֹקֵיד יְיָ׃}
{And Moses said unto Aaron, and unto Eleazar and unto Ithamar, his sons: ‘Let not the hair of your heads go loose, neither rend your clothes, that ye die not, and that He be not wroth with all the congregation; but let your brethren, the whole house of Israel, bewail the burning which the \lord\space hath kindled.}{\arabic{verse}}
\rashi{\rashiDH{אל תפרעו.} אל תגדלו שער, מכאן שֶׁאָבֵל אסור בתספורת (מועד קטן יד׃), אבל אתם אל תערבבו שמחתו של מקום׃ 
\quad \rashiDH{ולא תמותו.} הא אם תעשו כן, מותו׃\quad \rashiDH{ואחיכם כל בית ישראל.} מכאן, שצרתן של תלמידי חכמים מוטלת על הכל להתאבל בה׃ 
}
\threeverse{\arabic{verse}}%Leviticus10:7
{וּמִפֶּ֩תַח֩ אֹ֨הֶל מוֹעֵ֜ד לֹ֤א תֵֽצְאוּ֙ פֶּן\maqqaf תָּמֻ֔תוּ כִּי\maqqaf שֶׁ֛מֶן מִשְׁחַ֥ת יְהֹוָ֖ה עֲלֵיכֶ֑ם וַֽיַּעֲשׂ֖וּ כִּדְבַ֥ר מֹשֶֽׁה׃ \petucha }
{וּמִתְּרַע מַשְׁכַּן זִמְנָא לָא תִפְּקוּן דִּלְמָא תְמוּתוּן אֲרֵי מְשַׁח רְבוּתָא דַּייָ עֲלֵיכוֹן וַעֲבַדוּ כְּפִתְגָמָא דְּמֹשֶׁה׃}
{And ye shall not go out from the door of the tent of meeting, lest ye die; for the anointing oil of the \lord\space is upon you.’ And they did according to the word of Moses.}{\arabic{verse}}
\newseder{5}
\threeverse{\seder{ה}}%Leviticus10:8
{וַיְדַבֵּ֣ר יְהֹוָ֔ה אֶֽל\maqqaf אַהֲרֹ֖ן לֵאמֹֽר׃}
{וּמַלֵּיל יְיָ עִם אַהֲרֹן לְמֵימַר׃}
{And the \lord\space spoke unto Aaron, saying:}{\arabic{verse}}
\threeverse{\arabic{verse}}%Leviticus10:9
{יַ֣יִן וְשֵׁכָ֞ר אַל\maqqaf תֵּ֣שְׁתְּ \legarmeh  אַתָּ֣ה \legarmeh  וּבָנֶ֣יךָ אִתָּ֗ךְ בְּבֹאֲכֶ֛ם אֶל\maqqaf אֹ֥הֶל מוֹעֵ֖ד וְלֹ֣א תָמֻ֑תוּ חֻקַּ֥ת עוֹלָ֖ם לְדֹרֹתֵיכֶֽם׃}
{חֲמַר וּמְרַוֵּי לָא תִשְׁתֵּי אַתְּ וּבְנָךְ עִמָּךְ בְּמֵיעַלְכוֹן לְמַשְׁכַּן זִמְנָא וְלָא תְמוּתוּן קְיָם עָלַם לְדָרֵיכוֹן׃}
{‘Drink no wine nor strong drink, thou, nor thy sons with thee, when ye go into the tent of meeting, that ye die not; it shall be a statute forever throughout your generations.}{\arabic{verse}}
\rashi{\rashiDH{יין ושכר.} יין דרך שכרותו׃\quad \rashiDH{בבאכם אל אהל מועד.} אין לי אלא בבואם להיכל, בגשתם למזבח מנין, נאמר כאן ביאת אהל מועד, ונאמר בקידוש ידים ורגלים ביאת אהל מועד (שמות ל, כ), מה להלן עשה גישת מזבח כביאת אהל מועד, אף כאן עשה גישת מזבח כביאת אהל מועד (ת״כ פרשתא א, ד)׃}
\threeverse{\arabic{verse}}%Leviticus10:10
{וּֽלְהַבְדִּ֔יל בֵּ֥ין הַקֹּ֖דֶשׁ וּבֵ֣ין הַחֹ֑ל וּבֵ֥ין הַטָּמֵ֖א וּבֵ֥ין הַטָּהֽוֹר׃}
{וּלְאַפְרָשָׁא בֵּין קוּדְשָׁא וּבֵין חוּלָּא וּבֵין מְסָאֲבָא וּבֵין דָּכְיָא׃}
{And that ye may put difference between the holy and the common, and between the unclean and the clean;}{\arabic{verse}}
\rashi{\rashiDH{ולהבדיל.} כדי שתבדילו בין עבודה קדושה למחוללת, הא למדת שאם עבד עבודתו פסולה (שם ח. זבחים יז׃)׃}
\threeverse{\arabic{verse}}%Leviticus10:11
{וּלְהוֹרֹ֖ת אֶת\maqqaf בְּנֵ֣י יִשְׂרָאֵ֑ל אֵ֚ת כׇּל\maqqaf הַ֣חֻקִּ֔ים אֲשֶׁ֨ר דִּבֶּ֧ר יְהֹוָ֛ה אֲלֵיהֶ֖ם בְּיַד\maqqaf מֹשֶֽׁה׃ \petucha }
{וּלְאַלָּפָא יָת בְּנֵי יִשְׂרָאֵל יָת כָּל קְיָמַיָּא דְּמַלֵּיל יְיָ לְהוֹן בִּידָא דְּמֹשֶׁה׃}
{and that ye may teach the children of Israel all the statutes which the \lord\space hath spoken unto them by the hand of Moses.’}{\arabic{verse}}
\rashi{\rashiDH{ולהורת.} למד, שאסור שיכור בהוראה (ת״כ), יכול יהא חייב מיתה, תלמוד לומר אתה ובניך אִתָּךְ ולא תמותו, כהנים בעבודתם מיתה, ואין חכמים בהוראתם במיתה׃ 
}
\aliyacounter{רביעי}
\threeverse{\aliya{רביעי}}%Leviticus10:12
{וַיְדַבֵּ֨ר מֹשֶׁ֜ה אֶֽל\maqqaf אַהֲרֹ֗ן וְאֶ֣ל אֶ֠לְעָזָ֠ר וְאֶל\maqqaf אִ֨יתָמָ֥ר \pasek  בָּנָיו֮ הַנּֽוֹתָרִים֒ קְח֣וּ אֶת\maqqaf הַמִּנְחָ֗ה הַנּוֹתֶ֙רֶת֙ מֵאִשֵּׁ֣י יְהֹוָ֔ה וְאִכְל֥וּהָ מַצּ֖וֹת אֵ֣צֶל הַמִּזְבֵּ֑חַ כִּ֛י קֹ֥דֶשׁ קׇֽדָשִׁ֖ים הִֽוא׃}
{וּמַלֵּיל מֹשֶׁה עִם אַהֲרֹן וְעִם אֶלְעָזָר וְעִם אִיתָמָר בְּנוֹהִי דְּאִשְׁתְּאַרוּ סַבוּ יָת מִנְחָתָא דְּאִשְׁתְּאַרַת מִקּוּרְבָּנַיָּא דַּייָ וְאִכְלוּהָא פַטִּיר בִּסְטַר מַדְבְּחָא אֲרֵי קֹדֶשׁ קוּדְשִׁין הִיא׃}
{And Moses spoke unto Aaron, and unto Eleazar and unto Ithamar, his sons that were left: ‘Take the meal-offering that remaineth of the offerings of the \lord\space made by fire, and eat it without leaven beside the altar; for it is most holy.}{\arabic{verse}}
\rashi{\rashiDH{הנותרים.} מן המיתה, מלמד שאף עליהם קנסה מיתה על עון העגל, הוא שנאמר וּבְאַהֲרֹן הִתְאַנַּף ה׳ מְאֹד לְהַשְׁמִידוֹ (דברים ט, כ), ואין השמדה אלא כִּלּוּי בנים, שנאמר וָאַשְׁמִיד פִּרְיוֹ מִמַּעַל (עמוס ב, ט), ותפלתו של משה בטלה מחצה, שנאמר וָאֶתְפַּלֵּל גַּם בְּעַד אַהֲרֹן בָּעֵת הַהִוא (דברים שם)׃\quad \rashiDH{קחו את המנחה.} אף על פי שאתם אוננין, וקדשים אסורים לאונן (זבחים קא׃)׃\quad \rashiDH{את המנחה.} זו מנחת שמיני ומנחת נחשון׃ 
\quad \rashiDH{ואכלוה מצות.} מה תלמוד לומר, לפי שהיא מנחת צבור ומנחת שעה, ואין כיוצא בה לדורות, הוצרך לפרש בה דין שאר מנחות׃}
\threeverse{\arabic{verse}}%Leviticus10:13
{וַאֲכַלְתֶּ֤ם אֹתָהּ֙ בְּמָק֣וֹם קָד֔וֹשׁ כִּ֣י חׇקְךָ֤ וְחׇק\maqqaf בָּנֶ֙יךָ֙ הִ֔וא מֵאִשֵּׁ֖י יְהֹוָ֑ה כִּי\maqqaf כֵ֖ן צֻוֵּֽיתִי׃}
{וְתֵיכְלוּן יָתַהּ בַּאֲתַר קַדִּישׁ אֲרֵי חוּלָקָךְ וְחוּלָק בְּנָךְ הִיא מִקּוּרְבָּנַיָּא דַּייָ אֲרֵי כֵן אִתְפַּקַּדִית׃}
{And ye shall eat it in a holy place, because it is thy due, and thy sons’ due, of the offerings of the \lord\space made by fire; for so I am commanded.}{\arabic{verse}}
\rashi{\rashiDH{וחק בניך.} אין לְבָנוֹת חק בקדשים׃ 
\quad \rashiDH{כי כן צויתי.}באנינות יאכלוה (ת״כ פרק א, ח  זבחים קא.)׃}
\threeverse{\arabic{verse}}%Leviticus10:14
{וְאֵת֩ חֲזֵ֨ה הַתְּנוּפָ֜ה וְאֵ֣ת \legarmeh  שׁ֣וֹק הַתְּרוּמָ֗ה תֹּֽאכְלוּ֙ בְּמָק֣וֹם טָה֔וֹר אַתָּ֕ה וּבָנֶ֥יךָ וּבְנֹתֶ֖יךָ אִתָּ֑ךְ כִּֽי\maqqaf חׇקְךָ֤ וְחׇק\maqqaf בָּנֶ֙יךָ֙ נִתְּנ֔וּ מִזִּבְחֵ֥י שַׁלְמֵ֖י בְּנֵ֥י יִשְׂרָאֵֽל׃}
{וְיָת חַדְיָא דַּאֲרָמוּתָא וְיָת שָׁקָא דְּאַפְרָשׁוּתָא תֵּיכְלוּן בַּאֲתַר דְּכֵי אַתְּ וּבְנָךְ וּבְנָתָךְ עִמָּךְ אֲרֵי חוּלָקָךְ וְחוּלָק בְּנָךְ אִתְיְהִיבוּ מִנִּכְסַת קוּדְשַׁיָּא דִּבְנֵי יִשְׂרָאֵל׃}
{And the breast of waving and the thigh of heaving shall ye eat in a clean place; thou, and thy sons, and thy daughters with thee; for they are given as thy due, and thy sons’ due, out of the sacrifices of the peace-offerings of the children of Israel.}{\arabic{verse}}
\rashi{\rashiDH{ואת חזה התנופה.} של שלמי צבור׃\quad \rashiDH{תאכלו במקום טהור.} וכי את הראשונים אכלו במקום טמא, אלא הראשונים שהם קדשי קדשים, הוזקק אכילתם במקום קדוש, אבל אלו אין צריכים תוך הקלעים, אבל צריכים הם להאכל תוך מחנה ישראל, שהוא טהור מליכנס שם מצורעים, מכאן שקדשים קלים נאכלין בכל העיר (שם נה.)׃\quad \rashiDH{אתה ובניך ובנותיך.} אתה ובניך בחלק, אבל בנותיך לא בחלק, אלא אם תתנו להם מתנות רשאות הן לאכול בחזה ושוק, או אינו אלא אף הבנות בחלק, תלמוד לומר כי חקך וחק בניך נתנו, חק לבנים, ואין חק לבנות (ת״כ שם י)׃}
\threeverse{\arabic{verse}}%Leviticus10:15
{שׁ֣וֹק הַתְּרוּמָ֞ה וַחֲזֵ֣ה הַתְּנוּפָ֗ה עַ֣ל אִשֵּׁ֤י הַחֲלָבִים֙ יָבִ֔יאוּ לְהָנִ֥יף תְּנוּפָ֖ה לִפְנֵ֣י יְהֹוָ֑ה וְהָיָ֨ה לְךָ֜ וּלְבָנֶ֤יךָ אִתְּךָ֙ לְחׇק\maqqaf עוֹלָ֔ם כַּאֲשֶׁ֖ר צִוָּ֥ה יְהֹוָֽה׃}
{שָׁקָא דְּאַפְרָשׁוּתָא וְחַדְיָא דַּאֲרָמוּתָא עַל קוּרְבָּנֵי תַּרְבַּיָּא יִתֵּיתוֹן לְאָרָמָא אֲרָמָא קֳדָם יְיָ וִיהֵי לָךְ וְלִבְנָךְ עִמָּךְ לִקְיָם עָלַם כְּמָא דְּפַקֵּיד יְיָ׃}
{The thigh of heaving and the breast of waving shall they bring with the offerings of the fat made by fire, to wave it for a wave-offering before the \lord; and it shall be thine, and thy sons’ with thee, as a due for ever; as the \lord\space hath commanded.’}{\arabic{verse}}
\rashi{\rashiDH{שוק התרומה וחזה התנופה.} לשון אשר הונף ואשר הורם. תנופה מוליך ומביא תרומה מעלה ומוריד. ולמה חלקן הכתוב תרומה בשוק ותנופה בחזה. לא ידענו. ששניהם בהרמה והנפה׃\quad \rashiDH{על אשי החלבים.} מכאן שהחלבים למטה בשעת תנופה, וישוב המקראות שלא יכחישו זה את זה כבר פרשתי שלשתן בצו את אהרן (לעיל ז, ל)׃}
\aliyacounter{חמישי}
\threeverse{\aliya{חמישי}}%Leviticus10:16
{וְאֵ֣ת \legarmeh  שְׂעִ֣יר הַֽחַטָּ֗את דָּרֹ֥שׁ דָּרַ֛שׁ מֹשֶׁ֖ה וְהִנֵּ֣ה שֹׂרָ֑ף וַ֠יִּקְצֹ֠ף עַל\maqqaf אֶלְעָזָ֤ר וְעַל\maqqaf אִֽיתָמָר֙ בְּנֵ֣י אַהֲרֹ֔ן הַנּוֹתָרִ֖ם לֵאמֹֽר׃}
{וְיָת צְפִירָא דְּחַטָּתָא מִתְבָּע תַּבְעֵיהּ מֹשֶׁה וְהָא אִתּוֹקַד וּרְגֵיז עַל אֶלְעָזָר וְעַל אִיתָמָר בְּנֵי אַהֲרֹן דְּאִשְׁתְּאַרוּ לְמֵימַר׃}
{And Moses diligently inquired for the goat of the sin-offering, and, behold, it was burnt; and he was angry with Eleazar and with Ithamar, the sons of Aaron that were left, saying:}{\arabic{verse}}
\rashi{\rashiDH{שעיר החטאת.} שעיר מוספי ראש חודש. ושלשה שעירי חטאות קרבו בו ביום שעיר עזים, ושעיר נחשון, ושעיר ראש חודש, ומכולן לא נשרף אלא זה, ונחלקו בדבר חכמי ישראל, (בת״כ פרק ב, ח־י) יש אומרים מפני טומאה שנגעה בו נשרף, ויש אומרים מפני אנינות נשרף, לפי שהוא קדשי דורות, אבל בקדשי שעה סמכו על משה שאמר להם במנחה ואכלוה מצות׃\quad \rashiDH{דרוש דרש.} שתי דרישות הללו, מפני מה נשרף זה, ומפני מה נאכלו אלו, כך הוא בתורת כהנים (פרק ב, ב)\quad \rashiDH{על אלעזר ועל איתמר.} בשביל כבודו של אהרן הפך פניו כנגד הבנים וכעס׃\quad \rashiDH{לאמר.} אמר להם השיבוני על דברי׃}
\threeverse{\arabic{verse}}%Leviticus10:17
{מַדּ֗וּעַ לֹֽא\maqqaf אֲכַלְתֶּ֤ם אֶת\maqqaf הַחַטָּאת֙ בִּמְק֣וֹם הַקֹּ֔דֶשׁ כִּ֛י קֹ֥דֶשׁ קׇֽדָשִׁ֖ים הִ֑וא וְאֹתָ֣הּ \legarmeh  נָתַ֣ן לָכֶ֗ם לָשֵׂאת֙ אֶת\maqqaf עֲוֺ֣ן הָעֵדָ֔ה לְכַפֵּ֥ר עֲלֵיהֶ֖ם לִפְנֵ֥י יְהֹוָֽה׃}
{מָדֵין לָא אֲכַלְתּוּן יָת חַטָּתָא בַּאֲתַר קַדִּישׁ אֲרֵי קֹדֶשׁ קוּדְשִׁין הִיא וְיָתַהּ יְהַב לְכוֹן לְסַלָּחָא עַל חוֹבֵי כְּנִשְׁתָּא לְכַפָּרָא עֲלֵיהוֹן קֳדָם יְיָ׃}
{‘Wherefore have ye not eaten the sin-offering in the place of the sanctuary, seeing it is most holy, and He hath given it you to bear the iniquity of the congregation, to make atonement for them before the \lord?}{\arabic{verse}}
\rashi{\rashiDH{מדוע לא אכלתם את החטאת במקום הקדש.} וכי חוץ לקדש אכלוה, והלא שרפוה, ומהו אומר במקום הקדש, אלא אמר להם שמא חוץ לקלעים יצאה ונפסלה׃ 
\quad \rashiDH{כי קדש קדשים הוא.} ונפסלת ביוצא, והם אמרו לו לאו, אמר להם הואיל ובמקום הקדש היתה, מדוע לא אכלתם אותה׃\quad \rashiDH{ואותה נתן לכם לשאת וגו׳.} שהכהנים אוכלים ובעליהם מתכפרים׃\quad \rashiDH{לשאת את עון העדה.} מכאן למדנו ששעיר ראש חודש היה, שהוא מכפר על עון טומאת מקדש וקדשיו, שחטאת שמיני וחטאת נחשון לא לכפרה באו׃}
\threeverse{\arabic{verse}}%Leviticus10:18
{הֵ֚ן לֹא\maqqaf הוּבָ֣א אֶת\maqqaf דָּמָ֔הּ אֶל\maqqaf הַקֹּ֖דֶשׁ פְּנִ֑ימָה אָכ֨וֹל תֹּאכְל֥וּ אֹתָ֛הּ בַּקֹּ֖דֶשׁ כַּאֲשֶׁ֥ר צִוֵּֽיתִי׃}
{הָא לָא אִתָּעַל מִדְּמַהּ לְבֵית קוּדְשָׁא גַּוָּואָה מֵיכָל תֵּיכְלוּן יָתַהּ בְּקוּדְשָׁא כְּמָא דְּפַקֵּידִית׃}
{Behold, the blood of it was not brought into the sanctuary within; ye should certainly have eaten it in the sanctuary, as I commanded.’}{\arabic{verse}}
\rashi{\rashiDH{הן לא הובא וגו׳.} שאילו הובא היה לכם לשרפה, כמו שנאמר וְכָל חַטָּאת אֲשֶׁר יוּבָא מִדָּמָהּ וגו׳ (ויקרא ו, כג)׃\quad \rashiDH{אכול תאכלו אתה.} היה לכם לאכלה אף על פי שאתם אוננים׃\quad \rashiDH{כאשר צויתי.} לכם במנחה׃}
\threeverse{\arabic{verse}}%Leviticus10:19
{וַיְדַבֵּ֨ר אַהֲרֹ֜ן אֶל\maqqaf מֹשֶׁ֗ה הֵ֣ן הַ֠יּ֠וֹם הִקְרִ֨יבוּ אֶת\maqqaf חַטָּאתָ֤ם וְאֶת\maqqaf עֹֽלָתָם֙ לִפְנֵ֣י יְהֹוָ֔ה וַתִּקְרֶ֥אנָה אֹתִ֖י כָּאֵ֑לֶּה וְאָכַ֤לְתִּי חַטָּאת֙ הַיּ֔וֹם הַיִּיטַ֖ב בְּעֵינֵ֥י יְהֹוָֽה׃}
{וּמַלֵּיל אַהֲרֹן עִם מֹשֶׁה הָא יוֹמָא דֵין קָרִיבוּ יָת חַטָּוָותְהוֹן וְיָת עֲלָוָותְהוֹן קֳדָם יְיָ וְעָרַעָא יָתִי עָקָן כְּאִלֵּין אִלּוּ פוֹן אֲכַלִית חַטָּתָא יוֹמָא דֵין הֲתָקֵין קֳדָם יְיָ׃}
{And Aaron spoke unto Moses: ‘Behold, this day have they offered their sin-offering and their burnt-offering before the \lord, and there have befallen me such things as these; and if I had eaten the sin-offering to-day, would it have been well-pleasing in the sight of the \lord?}{\arabic{verse}}
\rashi{\rashiDH{וידבר אהרן.} אין לשון דבור אלא לשון עז, שנאמר וַיְדַבֵּר הָעָם וגו׳ (במדבר כא, ה). אפשר משה קצף על אלעזר ועל איתמר, ואהרן מדבר, הא ידעת שלא היתה אלא מדרך כבוד, אמרו, אינו בדין שיהא אבינו יושב ואנו מדברים לפניו, ואינו בדין שיהא תלמיד משיב את רבו, יכול מפני שלא היה באלעזר להשיב, תלמוד לומר וַיֹּאמֶר אֶלְעָזָר הַכֹּהֵן אֶל אַנְשֵׁי הַצָּבָא וגו׳ (שם לא, כא), הרי כשרצה דבר לפני משה ולפני הנשיאים, זו מצאתי בספרי של פנים שני׃\quad \rashiDH{הן היום הקריבו.} מהו אומר, אלא אמר להם משה שמא זרקתם דמה אוננים, שהאונן שעבד חלל, אמר לו אהרן, וכי הם הקריבו שהם הדיוטות, אני הקרבתי, שאני כהן גדול, ומקריב אונן (זבחים קא.)׃\quad \rashiDH{ותקראנה אותי כאלה.} אפילו לא היו המתים בָּנַי, אלא שאר קרובים שאני חייב להיות אונן עליהם כאלו, כגון כל האמורים בפרשת כהנים שהכהן מטמא להם׃\quad \rashiDH{ואכלתי חטאת.} ואם אכלתי, הייטב וגו׳׃\quad \rashiDH{היום.} אבל אנינות לילה מותר, שאין אונן אלא יום קבורה (שם ק׃)׃\quad \rashiDH{הייטב בעיני ה׳.} אם שמעת בקדשי שעה, אין לך להקל בקדשי דורות׃}
\threeverse{\arabic{verse}}%Leviticus10:20
{וַיִּשְׁמַ֣ע מֹשֶׁ֔ה וַיִּיטַ֖ב בְּעֵינָֽיו׃ \petucha }
{וּשְׁמַע מֹשֶׁה וּשְׁפַר בְּעֵינוֹהִי׃}
{And when Moses heard that, it was well-pleasing in his sight.}{\arabic{verse}}
\rashi{\rashiDH{וייטב בעיניו.} הודה ולא בוש לומר לא שמעתי (ת״כ פרק ב, יב)׃}
\newperek
\aliyacounter{ששי}
\newseder{6}
\threeverse{\aliya{ששי}\newline\vspace{-4pt}\newline\seder{ו}}%Leviticus11:1
{וַיְדַבֵּ֧ר יְהֹוָ֛ה אֶל\maqqaf מֹשֶׁ֥ה וְאֶֽל\maqqaf אַהֲרֹ֖ן לֵאמֹ֥ר אֲלֵהֶֽם׃}
{וּמַלֵּיל יְיָ עִם מֹשֶׁה וּלְאַהֲרֹן לְמֵימַר לְהוֹן׃}
{And the \lord\space spoke unto Moses and to Aaron, saying unto them:}{\Roman{chap}}
\rashi{\rashiDH{אל משה ואל אהרן.} למשה אמר, שיאמר לאהרן׃\quad \rashiDH{לאמר אליהם.} אמר שיאמר לאלעזר ולאיתמר, או אינו אלא לאמר לישראל, כשהוא אומר דברו אל בני ישראל, הרי דבור אמור לישראל, הא מה אני מקיים לאמר אליהם, לבניו, לאלעזר ולאיתמר׃}
\threeverse{\arabic{verse}}%Leviticus11:2
{דַּבְּר֛וּ אֶל\maqqaf בְּנֵ֥י יִשְׂרָאֵ֖ל לֵאמֹ֑ר זֹ֤את הַֽחַיָּה֙ אֲשֶׁ֣ר תֹּאכְל֔וּ מִכׇּל\maqqaf הַבְּהֵמָ֖ה אֲשֶׁ֥ר עַל\maqqaf הָאָֽרֶץ׃}
{מַלִּילוּ עִם בְּנֵי יִשְׂרָאֵל לְמֵימַר דָּא חַיְתָא דְּתֵיכְלוּן מִכָּל בְּעִירָא דְּעַל אַרְעָא׃}
{Speak unto the children of Israel, saying: These are the living things which ye may eat among all the beasts that are on the earth.}{\arabic{verse}}
\rashi{\rashiDH{דברו אל בני ישראל.} את כולם השוה להיות שלוחים בדבור זה, לפי שהושוו בִּדְמִימָה וקבלו עליהם גזירת המקום מאהבה׃\quad \rashiDH{זאת החיה.} לשון חיים, לפי שישראל דבוקים במקום וראויין להיות חיים, לפיכך הבדילם מן הטומאה וגזר עליהם מצות, ולאומות העולם לא אסר כלום. משל לרופא שנכנס לבקר את החולה וכו׳, כדאיתא במדרש רבי תנחומא (פ״ו, ויקרא רבה יג, ב)׃\quad \rashiDH{זאת החיה.} מלמד שהיה משה אוחז בחיה ומראה אותה לישראל זאת תאכלו וזאת לא תאכלו. את זה תאכלו וגו׳. אף בשרצי המים אחז מכל מין ומין והראה להם, וכן בעוף ואת אלה תשקצו מן העוף, וכן בשרצים וזה לכם הטמא׃\quad \rashiDH{זאת החיה מכל הבהמה.} מלמד שהבהמה בכלל חיה (חולין עא.)׃}
\threeverse{\arabic{verse}}%Leviticus11:3
{כֹּ֣ל \legarmeh  מַפְרֶ֣סֶת פַּרְסָ֗ה וְשֹׁסַ֤עַת שֶׁ֙סַע֙ פְּרָסֹ֔ת מַעֲלַ֥ת גֵּרָ֖ה בַּבְּהֵמָ֑ה אֹתָ֖הּ תֹּאכֵֽלוּ׃}
{כֹּל דִּסְדִיקָא פַרְסְתַהּ וּמַטְלְפָן טִלְפִין פַּרְסָתַהּ מַסְּקָא פִשְׁרָא בִּבְעִירָא יָתַהּ תֵּיכְלוּן׃}
{Whatsoever parteth the hoof, and is wholly cloven-footed, and cheweth the cud, among the beasts, that may ye eat.}{\arabic{verse}}
\rashi{\rashiDH{מפרסת.} כתרגומו סְדִיקָא׃\quad \rashiDH{פרסה.} פלאנט״ה בלע״ז׃\quad \rashiDH{ושסעת שסע.} שמובדלת מלמעלה ומלמטה בשתי צפרנין, כתרגומו וּמַטִּלְפָא טִלְפִין, שיש שפרסותיו סדוקות מלמעלה ואינן שסועות ומובדלות לגמרי שמלמטה מחוברות׃\quad \rashiDH{מעלת גרה.} מעלה ומקיאה האוכל ממעיה ומחזרת אותו לתוך פיה לכתשו ולטחנו הדק׃\quad \rashiDH{גרה.} כך שמו, ויתכן להיותו מגזרת מַּיִם הַנִּגָּרִים (שמואל־ב יד, יד), שהוא נגרר אחר הפה, ותרגומו פִּשְׁרָא, שע״י הגרה האוכל נפשר ונמוח׃\quad \rashiDH{בבהמה.} תיבה זו יתירה היא לדרשה, להתיר את השליל הנמצא במעי אמו׃\quad \rashiDH{אתה תאכלו.} ולא בהמה טמאה, והלא באזהרה היא, אלא לעבור עליה בעשה ולא תעשה (זבחים לד.)׃ 
}
\threeverse{\arabic{verse}}%Leviticus11:4
{אַ֤ךְ אֶת\maqqaf זֶה֙ לֹ֣א תֹֽאכְל֔וּ מִֽמַּעֲלֵי֙ הַגֵּרָ֔ה וּמִמַּפְרִסֵ֖י הַפַּרְסָ֑ה אֶֽת\maqqaf הַ֠גָּמָ֠ל כִּֽי\maqqaf מַעֲלֵ֨ה גֵרָ֜ה ה֗וּא וּפַרְסָה֙ אֵינֶ֣נּוּ מַפְרִ֔יס טָמֵ֥א ה֖וּא לָכֶֽם׃}
{בְּרַם יָת דֵּין לָא תֵיכְלוּן מִמַּסְּקֵי פִשְׁרָא וּמִסְּדִיקֵי פַרְסְתָא יָת גַּמְלָא אֲרֵי מַסֵּיק פִּשְׁרָא הוּא וּפַרְסְתֵיהּ לָא סְדִיקָא מְסָאַב הוּא לְכוֹן׃}
{Nevertheless these shall ye not eat of them that only chew the cud, or of them that only part the hoof: the camel, because he cheweth the cud but parteth not the hoof, he is unclean unto you.}{\arabic{verse}}
\threeverse{\arabic{verse}}%Leviticus11:5
{וְאֶת\maqqaf הַשָּׁפָ֗ן כִּֽי\maqqaf מַעֲלֵ֤ה גֵרָה֙ ה֔וּא וּפַרְסָ֖ה לֹ֣א יַפְרִ֑יס טָמֵ֥א ה֖וּא לָכֶֽם׃}
{וְיָת טַבְזָא אֲרֵי מַסֵּיק פִּשְׁרָא הוּא וּפַרְסְתֵיהּ לָא סְדִיקָא מְסָאַב הוּא לְכוֹן׃}
{And the rock-badger, because he cheweth the cud but parteth not the hoof, he is unclean unto you.}{\arabic{verse}}
\threeverse{\arabic{verse}}%Leviticus11:6
{וְאֶת\maqqaf הָאַרְנֶ֗בֶת כִּֽי\maqqaf מַעֲלַ֤ת גֵּרָה֙ הִ֔וא וּפַרְסָ֖ה לֹ֣א הִפְרִ֑יסָה טְמֵאָ֥ה הִ֖וא לָכֶֽם׃}
{וְיָת אַרְנְבָא אֲרֵי מַסְּקָא פִשְׁרָא הִיא וּפַרְסְתַהּ לָא סְדִיקָא מְסָאֲבָא הִיא לְכוֹן׃}
{And the hare, because she cheweth the cud but parteth not the hoof, she is unclean unto you}{\arabic{verse}}
\threeverse{\arabic{verse}}%Leviticus11:7
{וְאֶת\maqqaf הַ֠חֲזִ֠יר כִּֽי\maqqaf מַפְרִ֨יס פַּרְסָ֜ה ה֗וּא וְשֹׁסַ֥ע שֶׁ֙סַע֙ פַּרְסָ֔ה וְה֖וּא גֵּרָ֣ה לֹֽא\maqqaf יִגָּ֑ר טָמֵ֥א ה֖וּא לָכֶֽם׃}
{וְיָת חֲזִירָא אֲרֵי סְדִיק פַּרְסְתָא הוּא וּמַטְלְפָן טִלְפִין פַּרְסָתֵיהּ וְהוּא פִשְׁרָא לָא פָשַׁר מְסָאַב הוּא לְכוֹן׃}
{And the swine, because he parteth the hoof, and is cloven-footed, but cheweth not the cud, he is unclean unto you.}{\arabic{verse}}
\threeverse{\arabic{verse}}%Leviticus11:8
{מִבְּשָׂרָם֙ לֹ֣א תֹאכֵ֔לוּ וּבְנִבְלָתָ֖ם לֹ֣א תִגָּ֑עוּ טְמֵאִ֥ים הֵ֖ם לָכֶֽם׃}
{מִבִּשְׂרְהוֹן לָא תֵיכְלוּן וּבִנְבִילַתְהוֹן לָא תִקְרְבוּן מְסָאֲבִין אִנּוּן לְכוֹן׃}
{Of their flesh ye shall not eat, and their carcasses ye shall not touch; they are unclean unto you.}{\arabic{verse}}
\rashi{\rashiDH{מבשרם לא תאכלו.} אין לי אלא אלו, שאר בהמה טמאה שאין לה שום סימן טהרה מנין, אמרת קל וחומר, ומה אלו שיש בהן קצת סימני טהרה אסורות וכו׳ (כל הענין בת״כ פרק ג, ב)׃\quad \rashiDH{מבשרם.} על בשרם באזהרה, ולא על עצמות וגידין וקרנים וטלפים׃\quad \rashiDH{ובנבלתם לא תגעו.} יכול יהו ישראל מוזהרים על מגע נבלה, תלמוד לומר אֱמֹור אֶל הַכֹּהֲנִים וגו׳ (ויקרא כא, א), כהנים מוזהרין, ואין ישראל מוזהרין, קל וחומר מעתה, ומה טומאת המת חמורה לא הזהיר בה אלא כהנים, טומאת נבלה קלה לא כל שכן, ומה תלמוד לומר לא תגעו, ברגל (ראש השנה טז׃). (זהו שאמרו חייב אדם לטהר עצמו ברגל)׃}
\threeverse{\arabic{verse}}%Leviticus11:9
{אֶת\maqqaf זֶה֙ תֹּֽאכְל֔וּ מִכֹּ֖ל אֲשֶׁ֣ר בַּמָּ֑יִם כֹּ֣ל אֲשֶׁר\maqqaf לוֹ֩ סְנַפִּ֨יר וְקַשְׂקֶ֜שֶׂת בַּמַּ֗יִם בַּיַּמִּ֛ים וּבַנְּחָלִ֖ים אֹתָ֥ם תֹּאכֵֽלוּ׃}
{יָת דֵּין תֵּיכְלוּן מִכֹּל דִּבְמַיָּא כֹּל דְּלֵיהּ צִיצִין וְקַלְפִין בְּמַיָּא בְּיַמְמַיָּא וּבְנַחְלַיָּא יָתְהוֹן תֵּיכְלוּן׃}
{These may ye eat of all that are in the waters: whatsoever hath fins and scales in the waters, in the seas, and in the rivers, them may ye eat.}{\arabic{verse}}
\rashi{\rashiDH{סנפיר.} אלו ששט בהם׃\quad \rashiDH{קשקשת.} אלו קליפין הקבועים בו (חולין נט.), כמו שנאמר וְשִׁרְיוֹן קַשְׂקַשִּׂים הוּא לָבוּשׁ (שמואל־א יז, ה)׃}
\threeverse{\arabic{verse}}%Leviticus11:10
{וְכֹל֩ אֲשֶׁ֨ר אֵֽין\maqqaf ל֜וֹ סְנַפִּ֣יר וְקַשְׂקֶ֗שֶׂת בַּיַּמִּים֙ וּבַנְּחָלִ֔ים מִכֹּל֙ שֶׁ֣רֶץ הַמַּ֔יִם וּמִכֹּ֛ל נֶ֥פֶשׁ הַחַיָּ֖ה אֲשֶׁ֣ר בַּמָּ֑יִם שֶׁ֥קֶץ הֵ֖ם לָכֶֽם׃}
{וְכֹל דְּלֵית לֵיהּ צִיצִין וְקַלְפִין בְּיַמְמַיָּא וּבְנַחְלַיָּא מִכֹּל רִחְשָׁא דְּמַיָּא וּמִכֹּל נַפְשָׁא חַיְתָא דִּבְמַיָּא שִׁקְצָא אִנּוּן לְכוֹן׃}
{And all that have not fins and scales in the seas, and in the rivers, of all that swarm in the waters, and of all the living creatures that are in the waters, they are a detestable thing unto you,}{\arabic{verse}}
\rashi{\rashiDH{שרץ.} בכל מקום משמעו דבר נמוך שרוחש ונד על הארץ׃ 
}
\threeverse{\arabic{verse}}%Leviticus11:11
{וְשֶׁ֖קֶץ יִהְי֣וּ לָכֶ֑ם מִבְּשָׂרָם֙ לֹ֣א תֹאכֵ֔לוּ וְאֶת\maqqaf נִבְלָתָ֖ם תְּשַׁקֵּֽצוּ׃}
{וְשִׁקְצָא יְהוֹן לְכוֹן מִבִּסְרְהוֹן לָא תֵיכְלוּן וְיָת נְבִילַתְהוֹן תְּשַׁקְּצוּן׃}
{and they shall be a detestable thing unto you; ye shall not eat of their flesh, and their carcasses ye shall have in detestation.}{\arabic{verse}}
\rashi{\rashiDH{ושקץ יהיו.} לאסור את עירוביהן אם יש בו בנותן טעם (חולין צט.)׃\quad \rashiDH{מבשרם.} אינו מוזהר על הסנפירים ועל העצמות׃\quad \rashiDH{ואת נבלתם תשקצו.} לרבות יבחושין שסיננן (חולין סז.), יבחושין מושיילונ״ש בלע״ז׃}
\threeverse{\arabic{verse}}%Leviticus11:12
{כֹּ֣ל אֲשֶׁ֥ר אֵֽין\maqqaf ל֛וֹ סְנַפִּ֥יר וְקַשְׂקֶ֖שֶׂת בַּמָּ֑יִם שֶׁ֥קֶץ ה֖וּא לָכֶֽם׃}
{כֹּל דְּלֵית לֵיהּ צִיצִין וְקַלְפִין בְּמַיָּא שִׁקְצָא הוּא לְכוֹן׃}
{Whatsoever hath no fins nor scales in the waters, that is a detestable thing unto you.}{\arabic{verse}}
\rashi{\rashiDH{אשר אין לו וגו׳.} (ת״כ) מה תלמוד לומר, שיכול אין לי שיהא מותר אלא המעלה סימנין שלו ליבשה, השירן במים מנין, תלמוד לומר כל אשר אין לו סנפיר וקשקשת במים, הא אם היו לו במים אף על פי שהשירן בעלייתו מותר (ת״כ פרשתא א, יא)׃}
\threeverse{\arabic{verse}}%Leviticus11:13
{וְאֶת\maqqaf אֵ֙לֶּה֙ תְּשַׁקְּצ֣וּ מִן\maqqaf הָע֔וֹף לֹ֥א יֵאָכְל֖וּ שֶׁ֣קֶץ הֵ֑ם אֶת\maqqaf הַנֶּ֙שֶׁר֙ וְאֶת\maqqaf הַפֶּ֔רֶס וְאֵ֖ת הָעׇזְנִיָּֽה׃}
{וְיָת אִלֵּין תְּשַׁקְּצוּן מִן עוֹפָא לָא יִתְאַכְלוּן שִׁקְצָא אִנּוּן נִשְׁרָא וְעָר וְעָזְיָא׃}
{And these ye shall have in detestation among the fowls; they shall not be eaten, they are a detestable thing: the great vulture, and the bearded vulture, and the ospray;}{\arabic{verse}}
\rashi{\rashiDH{לא יאכלו.} לחייב את המאכילן לקטנים (יבמות קיד.), שכך משמעו לא יהיו נאכלים על ידך, או אינו אלא לאסרן בהנאה, תלמוד לומר לא תאכלו (דברים יד, יב), באכילה אסורין, בהנאה מותרין. כל עוף שנאמר בו למינה, למינו, למינהו, יש באותו המין שאין דומין זה לזה לא במראיהם ולא בשמותם, וכולן מין אחד׃}
\threeverse{\arabic{verse}}%Leviticus11:14
{וְאֶ֨ת\maqqaf הַדָּאָ֔ה וְאֶת\maqqaf הָאַיָּ֖ה לְמִינָֽהּ׃}
{וְדַיְתָא וְטָרָפִיתָא לִזְנַהּ׃}
{and the kite, and the falcon after its kinds;}{\arabic{verse}}
\threeverse{\arabic{verse}}%Leviticus11:15
{אֵ֥ת כׇּל\maqqaf עֹרֵ֖ב לְמִינֽוֹ׃}
{יָת כָּל עוֹרְבָא לִזְנֵיהּ׃}
{every raven after its kinds;}{\arabic{verse}}
\threeverse{\arabic{verse}}%Leviticus11:16
{וְאֵת֙ בַּ֣ת הַֽיַּעֲנָ֔ה וְאֶת\maqqaf הַתַּחְמָ֖ס וְאֶת\maqqaf הַשָּׁ֑חַף וְאֶת\maqqaf הַנֵּ֖ץ לְמִינֵֽהוּ׃}
{וְיָת בַּת נַעָמִיתָא וְצִיצָא וְצִפַּר שַׁחְפָּא וְנַצָּא לִזְנוֹהִי׃}
{and the ostrich, and the night-hawk, and the sea-mew, and the hawk after its kinds;}{\arabic{verse}}
\rashi{\rashiDH{הנץ.} אישפרוי״ר׃}
\threeverse{\arabic{verse}}%Leviticus11:17
{וְאֶת\maqqaf הַכּ֥וֹס וְאֶת\maqqaf הַשָּׁלָ֖ךְ וְאֶת\maqqaf הַיַּנְשֽׁוּף׃}
{וְקָדְיָא וְשָׁלֵינוּנָא וְקִפּוּפָא׃}
{and the little owl, and the cormorant, and the great owl;}{\arabic{verse}}
\rashi{\rashiDH{השלך.} פירשו רבותינו (חולין סג.) זה השולה (פי׳ מגביה) דגים מן הים, וזהו שתרגם אונקלוס וְשָׁלֵינוּנָא׃\quad \rashiDH{כוס וינשוף.} הם צואיטי״ש הצועקים בלילה, ויש להם לסתות כאדם, ועוד אחר דומה לו שקורין ייב״ץ׃ 
}
\threeverse{\arabic{verse}}%Leviticus11:18
{וְאֶת\maqqaf הַתִּנְשֶׁ֥מֶת וְאֶת\maqqaf הַקָּאָ֖ת וְאֶת\maqqaf הָרָחָֽם׃}
{וּבָוְתָא וְקָתָא וִירַקְרֵיקָא׃}
{and the horned owl, and the pelican, and the carrion-vulture;}{\arabic{verse}}
\rashi{\rashiDH{התנשמת.} היא קלב״א שורי״ץ, ודומה לעכבר ופורחת בלילה, ותנשמת האמורה בשרצים היא דומה לה, ואין לה עינים, וקורין לה טלפ״א׃}
\threeverse{\arabic{verse}}%Leviticus11:19
{וְאֵת֙ הַחֲסִידָ֔ה הָאֲנָפָ֖ה לְמִינָ֑הּ וְאֶת\maqqaf הַדּוּכִיפַ֖ת וְאֶת\maqqaf הָעֲטַלֵּֽף׃}
{וְחָוָרִיתָא וְאִבּוֹ לִזְנַהּ וְנַגָּר טוּרָא וַעֲטַלֵּיפָא׃}
{and the stork, and the heron after its kinds, and the hoopoe, and the bat.}{\arabic{verse}}
\rashi{\rashiDH{החסידה.} זו דיה לבנה, ציגוני״ה, ולמה נקרא שמה חסידה שעושה חסידות עם חברותיה במזונות (שם)׃\quad \rashiDH{האנפה.} היא דיה רגזנית, ונראה לי שהיא שקורין לה היירו״ן׃\quad \rashiDH{הדוכיפת.} תרנגול הבר, וְכַרְבַּלְתּוֹ כפולה, ובלעז הרופ״א, ולמה נקרא שמו דוכיפת, שהודו כפות, וזו היא כרבלתו, וְנַגַּר טוּרָא, נקרא על שם מעשיו, כמו שפירשו רבותינו במס׳ גיטין בפרק מי שאחזו (דף סח׃)׃}
\threeverse{\arabic{verse}}%Leviticus11:20
{כֹּ֚ל שֶׁ֣רֶץ הָע֔וֹף הַהֹלֵ֖ךְ עַל\maqqaf אַרְבַּ֑ע שֶׁ֥קֶץ ה֖וּא לָכֶֽם׃}
{כֹּל רִחְשָׁא דְּעוֹפָא דִּמְהַלֵּיךְ עַל אַרְבַּע שִׁקְצָא הוּא לְכוֹן׃}
{All winged swarming things that go upon all fours are a detestable thing unto you.}{\arabic{verse}}
\rashi{\rashiDH{שרץ העוף.} הם הדקים הנמוכים הרוחשין על הארץ, כגון זבובים וצרעין ויתושין וחגבים׃}
\threeverse{\arabic{verse}}%Leviticus11:21
{אַ֤ךְ אֶת\maqqaf זֶה֙ תֹּֽאכְל֔וּ מִכֹּל֙ שֶׁ֣רֶץ הָע֔וֹף הַהֹלֵ֖ךְ עַל\maqqaf אַרְבַּ֑ע אֲשֶׁר\maqqaf \qk{ל֤וֹ}{לא} כְרָעַ֙יִם֙ מִמַּ֣עַל לְרַגְלָ֔יו לְנַתֵּ֥ר בָּהֵ֖ן עַל\maqqaf הָאָֽרֶץ׃}
{בְּרַם יָת דֵּין תֵּיכְלוּן מִכֹּל רִחְשָׁא דְּעוֹפָא דִּמְהַלֵּיךְ עַל אַרְבַּע דְּלֵיהּ קַרְסוּלִּין מֵעִלָּוֵי רַגְלוֹהִי לְקַפָּצָא בְּהוֹן עַל אַרְעָא׃}
{Yet these may ye eat of all winged swarming things that go upon all fours, which have jointed legs above their feet, wherewith to leap upon the earth;}{\arabic{verse}}
\rashi{\rashiDH{על ארבע.} על ד׳ רגלים׃\quad \rashiDH{ממעל לרגליו.} סמוך לצוארו יש לו כמין שתי רגלים לבד ד׳ רגליו, וכשרוצה לעוף ולקפוץ מן הארץ מתחזק באותן שתי כרעים ופורח, ויש מהן הרבה, כאותן שקורין לנגושט״א, אבל אין אנו בקיאין בהן, ארבעה סימני טהרה נאמרו בהם, ארבע רגלים, וד׳ כנפים, וקרסולין אלו כרעים הכתובים כאן, וכנפיו חופין את רובו (חולין נט.). וכל סימנים הללו מצויין באותן שבינותינו, אבל יש שראשן ארוך ויש שאין להם זנב (שם סה׃), וצריך שיהא שמו חגב, ובזה אין אנו יודעים להבדיל ביניהם׃}
\threeverse{\arabic{verse}}%Leviticus11:22
{אֶת\maqqaf אֵ֤לֶּה מֵהֶם֙ תֹּאכֵ֔לוּ אֶת\maqqaf הָֽאַרְבֶּ֣ה לְמִינ֔וֹ וְאֶת\maqqaf הַסׇּלְעָ֖ם לְמִינֵ֑הוּ וְאֶת\maqqaf הַחַרְגֹּ֣ל לְמִינֵ֔הוּ וְאֶת\maqqaf הֶחָגָ֖ב לְמִינֵֽהוּ׃}
{יָת אִלֵּין מִנְּהוֹן תֵּיכְלוּן יָת גּוֹבָא לִזְנֵיהּ וְיָת רָשׁוֹנָא לִזְנוֹהִי וְיָת חַרְגְּלָא לִזְנוֹהִי וְיָת חַגְבָּא לִזְנוֹהִי׃}
{even these of them ye may eat: the locust after its kinds, and the bald locust after its kinds, and the cricket after its kinds, and the grasshopper after its kinds.}{\arabic{verse}}
\threeverse{\arabic{verse}}%Leviticus11:23
{וְכֹל֙ שֶׁ֣רֶץ הָע֔וֹף אֲשֶׁר\maqqaf ל֖וֹ אַרְבַּ֣ע רַגְלָ֑יִם שֶׁ֥קֶץ ה֖וּא לָכֶֽם׃}
{וְכֹל רִחְשָׁא דְּעוֹפָא דְּלֵיהּ אַרְבַּע רַגְלִין שִׁקְצָא הוּא לְכוֹן׃}
{But all winged swarming things, which have four feet, are a detestable thing unto you.}{\arabic{verse}}
\rashi{\rashiDH{וכל שרץ העוף וגו׳.} בא ללמד שאם יש לו חמש טהור (ת״כ פרק ה, י)׃ 
}
\threeverse{\arabic{verse}}%Leviticus11:24
{וּלְאֵ֖לֶּה תִּטַּמָּ֑אוּ כׇּל\maqqaf הַנֹּגֵ֥עַ בְּנִבְלָתָ֖ם יִטְמָ֥א עַד\maqqaf הָעָֽרֶב׃}
{וּלְאִלֵּין תִּסְתָּאֲבוּן כָּל דְּיִקְרַב בִּנְבִילַתְהוֹן יְהֵי מְסָאַב עַד רַמְשָׁא׃}
{And by these ye shall become unclean; whosoever toucheth the carcass of them shall be unclean until even.}{\arabic{verse}}
\rashi{\rashiDH{ולאלה.} (ת״כ) העתידין להאמר למטה בענין׃\quad \rashiDH{תטמאו.} כלומר בנגיעתם יש טומאה׃}
\threeverse{\arabic{verse}}%Leviticus11:25
{וְכׇל\maqqaf הַנֹּשֵׂ֖א מִנִּבְלָתָ֑ם יְכַבֵּ֥ס בְּגָדָ֖יו וְטָמֵ֥א עַד\maqqaf הָעָֽרֶב׃}
{וְכָל דְּיִטּוֹל מִנְּבִילַתְהוֹן יְצַבַּע לְבוּשׁוֹהִי וִיהֵי מְסָאַב עַד רַמְשָׁא׃}
{And whosoever beareth aught of the carcass of them shall wash his clothes, and be unclean until the even.}{\arabic{verse}}
\rashi{\rashiDH{וכל הנשא מנבלתם.} כל מקום שנאמרה טומאת משא, חמורה מטומאת מגע, שהיא טעונה כבוס בגדים (שם פרשתא ד, ז)׃}
\threeverse{\arabic{verse}}%Leviticus11:26
{לְֽכׇל\maqqaf הַבְּהֵמָ֡ה אֲשֶׁ֣ר הִוא֩ מַפְרֶ֨סֶת פַּרְסָ֜ה וְשֶׁ֣סַע \legarmeh  אֵינֶ֣נָּה שֹׁסַ֗עַת וְגֵרָה֙ אֵינֶ֣נָּה מַעֲלָ֔ה טְמֵאִ֥ים הֵ֖ם לָכֶ֑ם כׇּל\maqqaf הַנֹּגֵ֥עַ בָּהֶ֖ם יִטְמָֽא׃}
{לְכָל בְּעִירָא דְּהִיא סְדִיקָא פַרְסְתַהּ וְטִלְפִין לָיְתַהָא מַטְלְפָא וּפִשְׁרָא לָיְתַהָא מַסְּקָא מְסָאֲבִין אִנּוּן לְכוֹן כָּל דְּיִקְרַב בְּהוֹן יְהֵי מְסָאַב׃}
{Every beast which parteth the hoof, but is not cloven footed, nor cheweth the cud, is unclean unto you; every one that to toucheth them shall be unclean.}{\arabic{verse}}
\rashi{\rashiDH{מפרסת פרסה ושסע איננה שוסעת.} כגון גמל שפרסתו סדוקה למעלה, אבל למטה היא מחוברת. כאן למדך שנבלת בהמה טמאה מטמאה, ובענין שבסוף הפרשה פירש על בהמה טהורה׃}
\threeverse{\arabic{verse}}%Leviticus11:27
{וְכֹ֣ל \legarmeh  הוֹלֵ֣ךְ עַל\maqqaf כַּפָּ֗יו בְּכׇל\maqqaf הַֽחַיָּה֙ הַהֹלֶ֣כֶת עַל\maqqaf אַרְבַּ֔ע טְמֵאִ֥ים הֵ֖ם לָכֶ֑ם כׇּל\maqqaf הַנֹּגֵ֥עַ בְּנִבְלָתָ֖ם יִטְמָ֥א עַד\maqqaf הָעָֽרֶב׃}
{וְכֹל דִּמְהַלֵּיךְ עַל יְדוֹהִי בְּכָל חַיְתָא דִּמְהַלְּכָא עַל אַרְבַּע מְסָאֲבִין אִנּוּן לְכוֹן כָּל דְּיִקְרַב בִּנְבִילַתְהוֹן יְהֵי מְסָאַב עַד רַמְשָׁא׃}
{And whatsoever goeth upon its paws, among all beasts that go on all fours, they are unclean unto you; whoso toucheth their carcass shall be unclean until the even.}{\arabic{verse}}
\rashi{\rashiDH{על כפיו.} כגון כלב ודוב וחתול׃\quad \rashiDH{טמאים הם לכם.} למגע׃}
\threeverse{\arabic{verse}}%Leviticus11:28
{וְהַנֹּשֵׂא֙ אֶת\maqqaf נִבְלָתָ֔ם יְכַבֵּ֥ס בְּגָדָ֖יו וְטָמֵ֣א עַד\maqqaf הָעָ֑רֶב טְמֵאִ֥ים הֵ֖מָּה לָכֶֽם׃ \setuma }
{וּדְיִטּוֹל יָת נְבִילַתְהוֹן יְצַבַּע לְבוּשׁוֹהִי וִיהֵי מְסָאַב עַד רַמְשָׁא מְסָאֲבִין אִנּוּן לְכוֹן׃}
{And he that beareth the carcass of them shall wash his clothes, and be unclean until the even; they are unclean unto you.}{\arabic{verse}}
\threeverse{\arabic{verse}}%Leviticus11:29
{וְזֶ֤ה לָכֶם֙ הַטָּמֵ֔א בַּשֶּׁ֖רֶץ הַשֹּׁרֵ֣ץ עַל\maqqaf הָאָ֑רֶץ הַחֹ֥לֶד וְהָעַכְבָּ֖ר וְהַצָּ֥ב לְמִינֵֽהוּ׃}
{וְדֵין לְכוֹן דִּמְסָאַב בְּרִחְשָׁא דְּרָחֵישׁ עַל אַרְעָא חוּלְדָּא וְעַכְבְּרָא וְצַבָּא לִזְנוֹהִי׃}
{And these are they which are unclean unto you among the swarming things that swarm upon the earth: the weasel, and the mouse, and the great lizard after its kinds,}{\arabic{verse}}
\rashi{\rashiDH{וזה לכם הטמא.} כל טומאות הללו אינן לאיסור אכילה, אלא לטומאה ממש, להיות טמא במגען, ונאסר לאכול תרומה וקדשים, וליכנס למקדש׃\quad \rashiDH{החלד.} מוש״טילה׃\quad \rashiDH{והצב.} פויי״ט, שדומה לצפרדע׃}
\threeverse{\arabic{verse}}%Leviticus11:30
{וְהָאֲנָקָ֥ה וְהַכֹּ֖חַ וְהַלְּטָאָ֑ה וְהַחֹ֖מֶט וְהַתִּנְשָֽׁמֶת׃}
{וְיַלָּא וְכוֹחָא וְהַלְטְתָא וְחוּמְטָא וְאָשׁוּתָא׃}
{and the gecko, and the land-crocodile, and the lizard, and the sand-lizard, and the chameleon.}{\arabic{verse}}
\rashi{\rashiDH{אנקה.} הריצו״ן׃\quad \rashiDH{הלטאה.} לישרד״ה׃\quad \rashiDH{החמט.} לימצ״א׃\quad \rashiDH{תנשמת.} טלפ״א׃}
\threeverse{\arabic{verse}}%Leviticus11:31
{אֵ֛לֶּה הַטְּמֵאִ֥ים לָכֶ֖ם בְּכׇל\maqqaf הַשָּׁ֑רֶץ כׇּל\maqqaf הַנֹּגֵ֧עַ בָּהֶ֛ם בְּמֹתָ֖ם יִטְמָ֥א עַד\maqqaf הָעָֽרֶב׃}
{אִלֵּין דִּמְסָאֲבִין לְכוֹן בְּכָל רִחְשָׁא כָּל דְּיִקְרַב בְּהוֹן בְּמוֹתְהוֹן יְהֵי מְסָאַב עַד רַמְשָׁא׃}
{These are they which are unclean to you among all that swarm; whosoever doth touch them, when they are dead, shall be unclean until the even.}{\arabic{verse}}
\threeverse{\arabic{verse}}%Leviticus11:32
{וְכֹ֣ל אֲשֶׁר\maqqaf יִפֹּל\maqqaf עָלָיו֩ מֵהֶ֨ם \pasek  בְּמֹתָ֜ם יִטְמָ֗א מִכׇּל\maqqaf כְּלִי\maqqaf עֵץ֙ א֣וֹ בֶ֤גֶד אוֹ\maqqaf עוֹר֙ א֣וֹ שָׂ֔ק כׇּל\maqqaf כְּלִ֕י אֲשֶׁר\maqqaf יֵעָשֶׂ֥ה מְלָאכָ֖ה בָּהֶ֑ם בַּמַּ֧יִם יוּבָ֛א וְטָמֵ֥א עַד\maqqaf הָעֶ֖רֶב וְטָהֵֽר׃}
{וְכֹל דְּיִפּוֹל עֲלוֹהִי מִנְּהוֹן בְּמוֹתְהוֹן יְהֵי מְסָאַב מִכָּל מָאן דְּאָע אוֹ לְבוּשׁ אוֹ מְשַׁךְ אוֹ שָׂק כָּל מָאן דְּיִתְעֲבֵיד עֲבִידָא בְּהוֹן בְּמַיָּא יִתָּעַל וִיהֵי מְסָאַב עַד רַמְשָׁא וְיִדְכֵּי׃}
{And upon whatsoever any of them, when they are dead, doth fall, it shall be unclean; whether it be any vessel of wood, or raiment, or skin, or sack, whatsoever vessel it be, wherewith any work is done, it must be put into water, and it shall be unclean until the even; then shall it be clean.}{\arabic{verse}}
\rashi{\rashiDH{במים יובא.} ואף לאחר טבילתו טמא הוא לתרומה׃\quad \rashiDH{עד הערב.} ואחר כך \rashiDH{וטהר.} בהערב השמש (יבמות עה.)׃}
\aliyacounter{שביעי}
\threeverse{\aliya{שביעי}}%Leviticus11:33
{וְכׇ֨ל\maqqaf כְּלִי\maqqaf חֶ֔רֶשׂ אֲשֶׁר\maqqaf יִפֹּ֥ל מֵהֶ֖ם אֶל\maqqaf תּוֹכ֑וֹ כֹּ֣ל אֲשֶׁ֧ר בְּתוֹכ֛וֹ יִטְמָ֖א וְאֹת֥וֹ תִשְׁבֹּֽרוּ׃}
{וְכָל מָאן דַּחֲסַף דְּיִפּוֹל מִנְּהוֹן לְגַוֵּיהּ כֹּל דִּבְגַוֵּיהּ יִסְתָּאַב וְיָתֵיהּ תִּתְבְּרוּן׃}
{And every earthen vessel whereinto any of them falleth, whatsoever is in it shall be unclean, and it ye shall break.}{\arabic{verse}}
\rashi{\rashiDH{אל תוכו.} אין כלי חרס מיטמא אלא מאוירו (חולין כד׃)׃\quad \rashiDH{כל אשר בתוכו יטמא.} הכלי חוזר ומטמא מה שבאוירו (ס״א צ״ל בתוכו)׃\quad \rashiDH{ואתו תשבורו.} למד שאין לו טהרה במקוה (ת״כ פרשתא ז. יג)׃}
\threeverse{\arabic{verse}}%Leviticus11:34
{מִכׇּל\maqqaf הָאֹ֜כֶל אֲשֶׁ֣ר יֵאָכֵ֗ל אֲשֶׁ֨ר יָב֥וֹא עָלָ֛יו מַ֖יִם יִטְמָ֑א וְכׇל\maqqaf מַשְׁקֶה֙ אֲשֶׁ֣ר יִשָּׁתֶ֔ה בְּכׇל\maqqaf כְּלִ֖י יִטְמָֽא׃}
{מִכָּל מֵיכָל דְּמִתְאֲכִיל דְּיֵיעֲלוּן עֲלוֹהִי מַיָּא יְהֵי מְסָאַב וְכָל מַשְׁקְיָא דְּיִשְׁתְּתֵי בְּכָל מָאן יְהֵי מְסָאַב׃}
{All food therein which may be eaten, that on which water cometh, shall be unclean; and all drink in every such vessel that may be drunk shall be unclean.}{\arabic{verse}}
\rashi{\rashiDH{מכל האכל אשר יאכל.} מוסב על מקרא העליון כל אשר בתוכו יטמא, מכל האכל אשר יאכל אשר יבוא עליו מים, והוא בתוך כלי חרס הטמא, יטמא, וכן כל משקה אשר ישתה בכל כלי והוא בתוך כלי חרס הטמא, יטמא. למדנו מכאן דברים הרבה, למדנו שאין אוכל מוכשר ומתוקן לקבל טומאה עד שיבאו עליו מים פעם אחת, ומשבאו עליו מים פעם אחת, מקבל טומאה לעולם, ואפילו נגוב, והיין והשמן וכל הנקרא משקה מכשיר זרעים לטומאה כמים, שכך יש לדרוש המקרא אשר יבוא עליו מים או כל משקה אשר ישתה בכל כלי יטמא האוכל. ועוד למדו רבותינו מכאן שאין ולד הטומאה מטמא כלים, שכך שנינו (פסחים כ.׃) יכול יהיו כל הכלים מטמאין מאויר כלי חרס, תלמוד לומר כל אשר בתוכו יטמא, מכל האוכל, אוכל ומשקה מיטמא מאויר כלי חרס ואין כל הכלים מיטמאין מאויר כלי חרס, לפי שהשרץ אב הטומאה והכלי שנטמא ממנו ולד הטומאה, לפיכך אינו חוזר ומטמא כלים שבתוכו. ולמדנו עוד שהשרץ שנפל לאויר תנור והפת בתוכו ולא נגע השרץ בפת, התנור ראשון והפת שנייה, ולא נאמר רואין את התנור כאלו מלא טומאה ותהא הפת תחילה, שאם אתה אומר כן, לא נתמעטו כל הכלים מליטמא מאויר כלי חרס, שהרי טומאה עצמה נגעה בהן מגבן. ולמדנו עוד, על ביאת מים שאינה מכשרת זרעים אלא אם כן נפלו עליהן משנתלשו, שאם אתה אומר מקבלין הכשר במחובר, אין לך שלא באו עליו מים, ומהו אומר אשר יבוא עליו מים, משנתלשו. ולמדנו עוד שאין אוכל מטמא אחרים אלא אם כן יש בו כביצה (ת״כ פרק ט, א.  יומא פ.) שנאמר אשר יאכל, אוכל הנאכל בבת אחת, ושיערו חכמים אין בית הבליעה מחזיק יותר מביצת תרנגולת׃}
\threeverse{\arabic{verse}}%Leviticus11:35
{וְ֠כֹ֠ל אֲשֶׁר\maqqaf יִפֹּ֨ל מִנִּבְלָתָ֥ם \pasek  עָלָיו֮ יִטְמָא֒ תַּנּ֧וּר וְכִירַ֛יִם יֻתָּ֖ץ טְמֵאִ֣ים הֵ֑ם וּטְמֵאִ֖ים יִהְי֥וּ לָכֶֽם׃}
{וְכֹל דְּיִפּוֹל מִנְּבִילַתְהוֹן עֲלוֹהִי יְהֵי מְסָאַב תַּנּוּר וְכִירַיִם יִתָּרְעוּן מְסָאֲבִין אִנּוּן וּמְסָאֲבִין יְהוֹן לְכוֹן׃}
{And every thing whereupon any part of their carcass falleth shall be unclean; whether oven, or range for pots, it shall be broken in pieces; they are unclean, and shall be unclean unto you.}{\arabic{verse}}
\rashi{\rashiDH{תנור וכירים.} כלים המטלטלין הם, והם של חרס, ויש להן תוך, ושופת את הקדרה על נקב החלל, ושניהם פיהם למעלה׃\quad \rashiDH{יתץ.} שאין לכלי חרס טהרה בטבילה׃\quad \rashiDH{וטמאים יהיו לכם.} שלא תאמר מצווה אני לנותצם, תלמוד לומר וטמאים יהיו לכם, אם רצה לקיימן בטומאתן רשאי׃}
\threeverse{\arabic{verse}}%Leviticus11:36
{אַ֣ךְ מַעְיָ֥ן וּב֛וֹר מִקְוֵה\maqqaf מַ֖יִם יִהְיֶ֣ה טָה֑וֹר וְנֹגֵ֥עַ בְּנִבְלָתָ֖ם יִטְמָֽא׃}
{בְּרַם מַעְיָן וְגוּב בֵּית כְּנֵישַׁת מַיָּא יְהֵי דְּכֵי וּדְיִקְרַב בִּנְבִילַתְהוֹן יְהֵי מְסָאַב׃}
{Nevertheless a fountain or a cistern wherein is a gathering of water shall be clean; but he who toucheth their carcass shall be unclean.}{\arabic{verse}}
\rashi{\rashiDH{אך מעין ובור מקוה מים.} המחוברים לקרקע, אין מקבלין טומאה. ועוד יש לך ללמוד׃ \rashiDH{יהיה טהור.}הטובל בהם מטומאתו׃\quad \rashiDH{ונוגע בנבלתם יטמא.} אפי׳ הוא בתוך מעין ובור ונוגע בנבלתם יטמא, (ת״כ פרשתא ט, ה) שלא תאמר קל וחומר, אם מטהר את הטמאים מטומאתם, קל וחומר שיציל את הטהור מליטמא, לכך נאמר ונוגע בנבלתם יטמא׃}
\threeverse{\arabic{verse}}%Leviticus11:37
{וְכִ֤י יִפֹּל֙ מִנִּבְלָתָ֔ם עַל\maqqaf כׇּל\maqqaf זֶ֥רַע זֵר֖וּעַ אֲשֶׁ֣ר יִזָּרֵ֑עַ טָה֖וֹר הֽוּא׃}
{וַאֲרֵי יִפּוֹל מִנְּבִילַתְהוֹן עַל כָּל בַּר זְרַע זֵירוּעַ דְּיִזְדְּרַע דְּכֵי הוּא׃}
{And if aught of their carcass fall upon any sowing seed which is to be sown, it is clean.}{\arabic{verse}}
\rashi{\rashiDH{זרע זרוע.} זריעה של מיני זרעונין. זרוע שם דבר הוא, כמו וְיִתְּנוּ לָנוּ מִן הַזֵּרֹעִים (דניאל א, יב)׃\quad \rashiDH{טהור הוא.} למדך הכתוב שלא הוכשר ונתקן לקרות אוכל לקבל טומאה עד שיבואו עליו מים׃}
\threeverse{\arabic{verse}}%Leviticus11:38
{וְכִ֤י יֻתַּן\maqqaf מַ֙יִם֙ עַל\maqqaf זֶ֔רַע וְנָפַ֥ל מִנִּבְלָתָ֖ם עָלָ֑יו טָמֵ֥א ה֖וּא לָכֶֽם׃ \setuma }
{וַאֲרֵי יִתְיַהְבּוּן מַיָּא עַל בַּר זַרְעָא וְיִפּוֹל מִנְּבִילַתְהוֹן עֲלוֹהִי מְסָאַב הוּא לְכוֹן׃}
{But if water be put upon the seed, and aught of their carcass fall thereon, it is unclean unto you.}{\arabic{verse}}
\rashi{\rashiDH{וכי יתן מים על זרע.} לאחר שנתלש, שאם תאמר יש הכשר במחובר אין לך זרע שלא הוכשר (חולין קיח׃)׃\quad \rashiDH{מים על זרע.} בין מים, בין שאר משקין, בין הם על הזרע, בין הזרע נפל לתוכן, הכל נדרש בתורת כהנים (פרק יא, ו)׃\quad \rashiDH{ונפל מנבלתם עליו.} אף משנגב מן המים, שלא הקפידה תורה אלא להיות עליו שם אוכל, ומשירד עליו הכשר קבלת טומאה פעם אחת שוב אינו נעקר הימנו׃}
\threeverse{\arabic{verse}}%Leviticus11:39
{וְכִ֤י יָמוּת֙ מִן\maqqaf הַבְּהֵמָ֔ה אֲשֶׁר\maqqaf הִ֥יא לָכֶ֖ם לְאׇכְלָ֑ה הַנֹּגֵ֥עַ בְּנִבְלָתָ֖הּ יִטְמָ֥א עַד\maqqaf הָעָֽרֶב׃}
{וַאֲרֵי יְמוּת מִן בְּעִירָא דְּהִיא לְכוֹן לְמֵיכַל דְּיִקְרַב בִּנְבִילְתַהּ יְהֵי מְסָאַב עַד רַמְשָׁא׃}
{And if any beast, of which ye may eat, die, he that toucheth the carcass thereof shall be unclean until the even.}{\arabic{verse}}
\rashi{\rashiDH{בנבלתה.} ולא בעצמות וגידים, ולא בקרנים וטלפים, ולא בעור (ת״כ פרשתא י, ה  חולין קיח.)׃}
\threeverse{\arabic{verse}}%Leviticus11:40
{וְהָֽאֹכֵל֙ מִנִּבְלָתָ֔הּ יְכַבֵּ֥ס בְּגָדָ֖יו וְטָמֵ֣א עַד\maqqaf הָעָ֑רֶב וְהַנֹּשֵׂא֙ אֶת\maqqaf נִבְלָתָ֔הּ יְכַבֵּ֥ס בְּגָדָ֖יו וְטָמֵ֥א עַד\maqqaf הָעָֽרֶב׃}
{וּדְיֵיכוֹל מִנְּבִילְתַהּ יְצַבַּע לְבוּשׁוֹהִי וִיהֵי מְסָאַב עַד רַמְשָׁא וּדְיִטּוֹל יָת נְבִילְתַהּ יְצַבַּע לְבוּשׁוֹהִי וִיהֵי מְסָאַב עַד רַמְשָׁא׃}
{And he that eateth of the carcass of it shall wash his clothes, and be unclean until the even; he also that beareth the carcass of it shall wash his clothes, and be unclean until the even.}{\arabic{verse}}
\rashi{\rashiDH{והנשא את נבלתה.} חמורה טומאת משא, מטומאת מגע, שהנושא מטמא בגדים, והנוגע אין בגדיו טמאין, שלא נאמר בו יכבס בגדיו׃\quad \rashiDH{והאכל מנבלתה.} יכול תטמאנו אכילתו, כשהוא אומר בנבלת עוף טהור נְבֵלָה וּטְרֵפָה לֹא יֹאכַל לְטָמְאָה בָהּ (ויקרא כב, ח), אותה מטמאה בגדים באכילתה, ואין נבלת בהמה מטמאה בגדים באכילתה, בלא משא, כגון, אם תחבה לו חבירו בבית הבליעה, אם כן מה תלמוד לומר האכל, ליתן שיעור לנושא ולנוגע כדי אכילה, והוא כזית (נדה מב׃)׃\quad \rashiDH{וטמא עד הערב.} אף על פי שטבל, צריך הערב שמש׃}
\threeverse{\arabic{verse}}%Leviticus11:41
{וְכׇל\maqqaf הַשֶּׁ֖רֶץ הַשֹּׁרֵ֣ץ עַל\maqqaf הָאָ֑רֶץ שֶׁ֥קֶץ ה֖וּא לֹ֥א יֵאָכֵֽל׃}
{וְכָל רִחְשָׁא דְּרָחֵישׁ עַל אַרְעָא שִׁקְצָא הוּא לָא יִתְאֲכִיל׃}
{And every swarming thing that swarmeth upon the earth is a detestable thing; it shall not be eaten.}{\arabic{verse}}
\rashi{\rashiDH{השורץ על הארץ.} להוציא את היתושין שֶׁבַּכְּלִיסִין וְשֶׁבַּפּוֹלִין, ואת הזיזין שבעדשים (חולין סז׃) שהרי לא שרצו על הארץ אלא בתוך האוכל, אבל משיצאו לאויר ושרצו הרי נאסרו׃\quad \rashiDH{לא יאכל.} לחייב על המאכיל כאוכל, ואין קרוי שרץ, אלא דבר נמוך קצר רגלים שאינו נראה אלא כרוחש ונד׃ 
}
\threeverse{\arabic{verse}}%Leviticus11:42
{כֹּל֩ הוֹלֵ֨ךְ עַל\maqqaf גָּח֜\large וֹ\normalsize ן וְכֹ֣ל \legarmeh  הוֹלֵ֣ךְ עַל\maqqaf אַרְבַּ֗ע עַ֚ד כׇּל\maqqaf מַרְבֵּ֣ה רַגְלַ֔יִם לְכׇל\maqqaf הַשֶּׁ֖רֶץ הַשֹּׁרֵ֣ץ עַל\maqqaf הָאָ֑רֶץ לֹ֥א תֹאכְל֖וּם כִּי\maqqaf שֶׁ֥קֶץ הֵֽם׃}
{כֹּל דִּמְהַלֵּיךְ עַל מְעוֹהִי וְכֹל דִּמְהַלֵּיךְ עַל אַרְבַּע עַד כָּל סַגְיוּת רַגְלִין בְּכָל רִחְשָׁא דְּרָחֵישׁ עַל אַרְעָא לָא תֵיכְלוּנוּנוּן אֲרֵי שִׁקְצָא אִנּוּן׃}
{Whatsoever goeth upon the belly, and whatsoever goeth upon all fours, or whatsoever hath many feet, even all swarming things that swarm upon the earth, them ye shall not eat; for they are a detestable thing.}{\arabic{verse}}
\rashi{\rashiDH{הולך על גחון.} זה נחש, ולשון גחון, שְׁחִיָּה, שהולך שָׁח ונופל על מעיו׃\quad \rashiDH{כל הולך.} להביא השלשולין ואת הדומה לדומה׃\quad \rashiDH{הולך על ארבע.} זה עקרב׃\quad \rashiDH{כל.} להביא את החפושית, אשקרבי״ט בלע״ז, ואת הדומה לדומה׃\quad \rashiDH{מרבה רגלים.} זה נַדָּל, שֶׁרֶץ, שיש לו רגלים מראשו ועד זנבו לכאן ולכאן, וקורין צינטפיד״ש׃ 
}
\threeverse{\arabic{verse}}%Leviticus11:43
{אַל\maqqaf תְּשַׁקְּצוּ֙ אֶת\maqqaf נַפְשֹׁ֣תֵיכֶ֔ם בְּכׇל\maqqaf הַשֶּׁ֖רֶץ הַשֹּׁרֵ֑ץ וְלֹ֤א תִֽטַּמְּאוּ֙ בָּהֶ֔ם וְנִטְמֵתֶ֖ם בָּֽם׃}
{לָא תְשַׁקְּצוּן יָת נַפְשָׁתְכוֹן בְּכָל רִחְשָׁא דְּרָחֵישׁ וְלָא תִסְתָּאֲבוּן בְּהוֹן וְתִסְתָּאֲבוּן פּוֹן בְּהוֹן׃}
{Ye shall not make yourselves detestable with any swarming thing that swarmeth, neither shall ye make yourselves unclean with them, that ye should be defiled thereby.}{\arabic{verse}}
\rashi{\rashiDH{אל תשקצו.} באכילתן, שהרי כתיב נפשותיכם, ואין שקוץ נפש במגע, וכן ולא תטמאו באכילתם׃\quad \rashiDH{ונטמתם בם.} אם אתם מטמאין בהן בארץ, אף אני מטמא אתכם בעולם הבא, ובישיבת מעלה׃}
\threeverse{\arabic{verse}}%Leviticus11:44
{כִּ֣י אֲנִ֣י יְהֹוָה֮ אֱלֹֽהֵיכֶם֒ וְהִתְקַדִּשְׁתֶּם֙ וִהְיִיתֶ֣ם קְדֹשִׁ֔ים כִּ֥י קָד֖וֹשׁ אָ֑נִי וְלֹ֤א תְטַמְּאוּ֙ אֶת\maqqaf נַפְשֹׁ֣תֵיכֶ֔ם בְּכׇל\maqqaf הַשֶּׁ֖רֶץ הָרֹמֵ֥שׂ עַל\maqqaf הָאָֽרֶץ׃}
{אֲרֵי אֲנָא יְיָ אֱלָהֲכוֹן וְתִתְקַדְּשׁוּן וּתְהוֹן קַדִּישִׁין אֲרֵי קַדִּישׁ אֲנָא וְלָא תְסָאֲבוּן יָת נַפְשָׁתְכוֹן בְּכָל רִחְשָׁא דְּרָחֵישׁ עַל אַרְעָא׃}
{For I am the \lord\space your God; sanctify yourselves therefore, and be ye holy; for I am holy; neither shall ye defile yourselves with any manner of swarming thing that moveth upon the earth.}{\arabic{verse}}
\rashi{\rashiDH{כי אני ה׳ אלהיכם.} כשם שאני קדוש, שאני ה׳ אלהיכם, כך והתקדשתם, קדשו את עצמכם למטה׃\quad \rashiDH{והייתם קדושים.} לפי שאני אקדש אתכם למעלה, ולעולם הבא׃\quad \rashiDH{ולא תטמאו וגו׳.} לעבור עליהם בלאוין הרבה וכל לאו מלקות, וזהו שאמרו בגמ׳ (מכות טז׃) אכל פּוּטִיתָא לוקה ארבע, נמלה לוקה חמש, צרעה לוקה שש׃}
\threeverse{\aliya{מפטיר}}%Leviticus11:45
{כִּ֣י \legarmeh  אֲנִ֣י יְהֹוָ֗ה הַֽמַּעֲלֶ֤ה אֶתְכֶם֙ מֵאֶ֣רֶץ מִצְרַ֔יִם לִהְיֹ֥ת לָכֶ֖ם לֵאלֹהִ֑ים וִהְיִיתֶ֣ם קְדֹשִׁ֔ים כִּ֥י קָד֖וֹשׁ אָֽנִי׃}
{אֲרֵי אֲנָא יְיָ דְּאַסֵּיק יָתְכוֹן מֵאַרְעָא דְּמִצְרַיִם לְמִהְוֵי לְכוֹן לֶאֱלָהּ וּתְהוֹן קַדִּישִׁין אֲרֵי קַדִּישׁ אֲנָא׃}
{For I am the \lord\space that brought you up out of the land of Egypt, to be your God; ye shall therefore be holy, for I am holy. .}{\arabic{verse}}
\rashi{\rashiDH{כי אני ה׳ המעלה אתכם.} על מנת שתקבלו מצותי העליתי אתכם. (דבר אחר כי אני ה׳ המעלה אתכם, בכולן כתיב הוצאתי, וכאן כתיב המעלה, תנא דבי רבי ישמעאל אלמלי לא העליתי את ישראל ממצרים אלא בשביל שאין מטמאין בשרצים כשאר אומות דיים, ומעליותא היא גבייהו, והוא לשון מעלה (ב״מ סא׃))׃}
\threeverse{\arabic{verse}}%Leviticus11:46
{זֹ֣את תּוֹרַ֤ת הַבְּהֵמָה֙ וְהָע֔וֹף וְכֹל֙ נֶ֣פֶשׁ הַֽחַיָּ֔ה הָרֹמֶ֖שֶׂת בַּמָּ֑יִם וּלְכׇל\maqqaf נֶ֖פֶשׁ הַשֹּׁרֶ֥צֶת עַל\maqqaf הָאָֽרֶץ׃}
{דָּא אוֹרָיְתָא דִּבְעִירָא וּדְעוֹפָא וּלְכֹל נַפְשָׁא חַיְתָא דְּרָחֲשָׁא בְּמַיָּא וּלְכָל נַפְשָׁא דְּרָחֲשָׁא עַל אַרְעָא׃}
{This is the law of the beast, and of the fowl, and of every living creature that moveth in the waters, and of every creature that swarmeth upon the earth;}{\arabic{verse}}
\threeverse{\aliya{\Hebrewnumeral{91}}}%Leviticus11:47
{לְהַבְדִּ֕יל בֵּ֥ין הַטָּמֵ֖א וּבֵ֣ין הַטָּהֹ֑ר וּבֵ֤ין הַֽחַיָּה֙ הַֽנֶּאֱכֶ֔לֶת וּבֵין֙ הַֽחַיָּ֔ה אֲשֶׁ֖ר לֹ֥א תֵאָכֵֽל׃ \petucha }
{לְאַפְרָשָׁא בֵּין מְסָאֲבָא וּבֵין דָּכְיָא וּבֵין חַיְתָא דְּמִתְאַכְלָא וּבֵין חַיְתָא דְּלָא מִתְאַכְלָא׃}
{to make a difference between the unclean and the clean, and between the living thing that may be eaten and the living thing that may not be eaten.}{\arabic{verse}}
\rashi{\rashiDH{להבדיל.} לא בלבד השונה אלא שתהא יודע ומכיר ובקי בהן׃\quad \rashiDH{בין הטמא ובין הטהור.} צריך לומר בין חמור לפרה, והלא כבר מפורשים הם, אלא בין טמאה לך, לטהורה לך, בין נשחט חציו של קנה, לנשחט רובו׃\quad \rashiDH{ובין החיה הנאכלת.} צריך לומר בין צבי לערוד, והלא כבר מפורשים הם, אלא בין שנולדו בה סימני טרפה כשרה, לנולדו בה סימני טרפה פסולה׃ 
}
\engnote{The Haftarah is II Samuel 6:1\verserangechar 7:17 on page \pageref{haft_26}. Sepharadim read II Samuel 6:1\verserangechar 6:19. On Shabbat Parah, read Maftir and Haftara on page \pageref{maft_parah}. For Shabbat Ha\d{H}odesh the Maftir and Haftara are on page \pageref{maft_hachodesh}.}
\newperek
\aliyacounter{ראשון}
\newparsha{תזריע}
\newseder{7}
\threeverse{\aliya{תזריע}\newline\vspace{-4pt}\newline\seder{ז}}%Leviticus12:1
{וַיְדַבֵּ֥ר יְהֹוָ֖ה אֶל\maqqaf מֹשֶׁ֥ה לֵּאמֹֽר׃}
{וּמַלֵּיל יְיָ עִם מֹשֶׁה לְמֵימַר׃}
{And the \lord\space spoke unto Moses, saying:}{\Roman{chap}}
\threeverse{\arabic{verse}}%Leviticus12:2
{דַּבֵּ֞ר אֶל\maqqaf בְּנֵ֤י יִשְׂרָאֵל֙ לֵאמֹ֔ר אִשָּׁה֙ כִּ֣י תַזְרִ֔יעַ וְיָלְדָ֖ה זָכָ֑ר וְטָֽמְאָה֙ שִׁבְעַ֣ת יָמִ֔ים כִּימֵ֛י נִדַּ֥ת דְּוֺתָ֖הּ תִּטְמָֽא׃}
{מַלֵּיל עִם בְּנֵי יִשְׂרָאֵל לְמֵימַר אִתְּתָא אֲרֵי תְעַדֵּי וּתְלִיד דְּכַר וּתְהֵי מְסָאֲבָא שִׁבְעָא יוֹמִין כְּיוֹמֵי רִיחוּק סְאוֹבְתַהּ תְּהֵי מְסָאֲבָא׃}
{Speak unto the children of Israel, saying: If a woman be delivered, and bear a man-child, then she shall be unclean seven days; as in the days of the impurity of her sickness shall she be unclean.}{\arabic{verse}}
\rashi{\rashiDH{אשה כי תזריע.} א״ר שמלאי (ויק״ר יד, א) כשם שיצירתו של אדם אחר כל בהמה חיה ועוף במעשה בראשית, כך תורתו נתפרשה אחר תורת בהמה חיה ועוף׃\quad \rashiDH{כי תזריע}. לרבות שאפי׳ ילדתו מחוי, שנמחה ונעשה כעין זרע, אמו טמאה לידה (נדה כז׃)׃\quad \rashiDH{כימי נדת דותה תטמא.} כסדר כל טומאה האמורה בנדה, מטמאה בטומאת לידה, ואפילו נפתח הקבר בלא דם׃\quad \rashiDH{דותה.} לשון דבר הזב מגופה. לשון אחר לשון מדוה וחולי, שאין אשה רואה דם שלא תחלה ראשה ואבריה כבדין עליה׃}
\threeverse{\arabic{verse}}%Leviticus12:3
{וּבַיּ֖וֹם הַשְּׁמִינִ֑י יִמּ֖וֹל בְּשַׂ֥ר עׇרְלָתֽוֹ׃}
{וּבְיוֹמָא תְּמִינָאָה יִתְגְּזַר בִּשְׂרָא דְּעוּרְלְתֵיהּ׃}
{And in the eighth day the flesh of his foreskin shall be circumcised.}{\arabic{verse}}
\threeverse{\arabic{verse}}%Leviticus12:4
{וּשְׁלֹשִׁ֥ים יוֹם֙ וּשְׁלֹ֣שֶׁת יָמִ֔ים תֵּשֵׁ֖ב בִּדְמֵ֣י טׇהֳרָ֑ה בְּכׇל\maqqaf קֹ֣דֶשׁ לֹֽא\maqqaf תִגָּ֗ע וְאֶל\maqqaf הַמִּקְדָּשׁ֙ לֹ֣א תָבֹ֔א עַד\maqqaf מְלֹ֖את יְמֵ֥י טׇהֳרָֽהּ׃}
{וּתְלָתִין וּתְלָתָא יוֹמִין תִּתֵּיב בְּדַם דְּכוּ בְּכָל קוּדְשָׁא לָא תִקְרַב וּלְמַקְדְּשָׁא לָא תֵיעוֹל עַד מִשְׁלַם יוֹמֵי דְּכוּתַהּ׃}
{And she shall continue in the blood of purification three and thirty days; she shall touch no hallowed thing, nor come into the sanctuary, until the days of her purification be fulfilled.}{\arabic{verse}}
\rashi{\rashiDH{תשב.} אין תשב אלא לשון עכבה, כמו וַתֵּשְׁבוּ בְקָדֵשׁ (דברים א, מו), וַיֵּשֶׁב בְּאֵלֹנֵי מַמְרֵא (בראשית יג, יח)׃\quad \rashiDH{בדמי טהרה.} אף על פי שרואה טהורה׃\quad \rashiDH{בדמי טהרה.} לא מפיק ה״א, והוא שם דבר, כמו טוהר׃\quad \rashiDH{ימי טהרה.} מפיק ה״א, ימי טוהר שלה׃\quad \rashiDH{לא תגע.} אזהרה לאוכל, כמו ששנויה ביבמות (דף עה.)׃\quad \rashiDH{בכל קדש וגו׳.} לרבות את התרומה (מכות יד׃ יבמות שם), לפי שזו טבולת יום ארוך שטבלה לסוף שבעה ואין שמשה מעריב לטהרה עד שקיעת החמה של יום ארבעים, שלמחר תביא את כפרת טהרתה׃}
\threeverse{\aliya{לוי}}%Leviticus12:5
{וְאִם\maqqaf נְקֵבָ֣ה תֵלֵ֔ד וְטָמְאָ֥ה שְׁבֻעַ֖יִם כְּנִדָּתָ֑הּ וְשִׁשִּׁ֥ים יוֹם֙ וְשֵׁ֣שֶׁת יָמִ֔ים תֵּשֵׁ֖ב עַל\maqqaf דְּמֵ֥י טׇהֳרָֽה׃}
{וְאִם נוּקְבְּתָא תְלִיד וּתְהֵי מְסָאֲבָא אַרְבְּעַת עֲשַׂר כְּרִיחוּקַהּ וְשִׁתִּין וְשִׁתָּא יוֹמִין תִּתֵּיב עַל דַּם דְּכוּ׃}
{But if she bear a maid-child, then she shall be unclean two weeks, as in her impurity; and she shall continue in the blood of purification threescore and six days.}{\arabic{verse}}
\threeverse{\arabic{verse}}%Leviticus12:6
{וּבִמְלֹ֣את \legarmeh  יְמֵ֣י טׇהֳרָ֗הּ לְבֵן֮ א֣וֹ לְבַת֒ תָּבִ֞יא כֶּ֤בֶשׂ בֶּן\maqqaf שְׁנָתוֹ֙ לְעֹלָ֔ה וּבֶן\maqqaf יוֹנָ֥ה אוֹ\maqqaf תֹ֖ר לְחַטָּ֑את אֶל\maqqaf פֶּ֥תַח אֹֽהֶל\maqqaf מוֹעֵ֖ד אֶל\maqqaf הַכֹּהֵֽן׃}
{וּבְמִשְׁלַם יוֹמֵי דְּכוּתַהּ לִבְרָא אוֹ לִבְרַתָּא תַּיְתֵי אִמַּר בַּר שַׁתֵּיהּ לַעֲלָתָא וּבַר יוֹנָה אוֹ שַׁפְנִינָא לְחַטָּתָא לִתְרַע מַשְׁכַּן זִמְנָא לְוָת כָּהֲנָא׃}
{And when the days of her purification are fulfilled, for a son, or for a daughter, she shall bring a lamb of the first year for a burnt-offering, and a young pigeon, or a turtle-dove, for a sin-offering, unto the door of the tent of meeting, unto the priest.}{\arabic{verse}}
\threeverse{\arabic{verse}}%Leviticus12:7
{וְהִקְרִיב֞וֹ לִפְנֵ֤י יְהֹוָה֙ וְכִפֶּ֣ר עָלֶ֔יהָ וְטָהֲרָ֖ה מִמְּקֹ֣ר דָּמֶ֑יהָ זֹ֤את תּוֹרַת֙ הַיֹּלֶ֔דֶת לַזָּכָ֖ר א֥וֹ לַנְּקֵבָֽה׃}
{וִיקָרְבִנֵּיהּ לִקְדָם יְיָ וִיכַפַּר עֲלַהּ וְתִדְכֵּי מִסּוֹאֲבָת דְּמַהָא דָּא אוֹרָיְתָא דְּיָלֵידְתָּא לִדְכַר אוֹ לְנוּקְבָּא׃}
{And he shall offer it before the \lord, and make atonement for her; and she shall be cleansed from the fountain of her blood. This is the law for her that beareth, whether a male or a female.}{\arabic{verse}}
\rashi{\rashiDH{והקריבו.} ללמדך, שאין מעכבה לאכול בקדשים אלא אחד מהם, ואי זה הוא, זה חטאת, שנאמר וכפר עליה הכהן וטהרה, מי שהוא בא לכפר, בו הטהרה תלויה׃\quad \rashiDH{וטהרה.} מכלל שעד כאן קרויה טמאה (זבחים יט׃ סנהדרין פג׃)׃}
\threeverse{\arabic{verse}}%Leviticus12:8
{וְאִם\maqqaf לֹ֨א תִמְצָ֣א יָדָהּ֮ דֵּ֣י שֶׂה֒ וְלָקְחָ֣ה שְׁתֵּֽי\maqqaf תֹרִ֗ים א֤וֹ שְׁנֵי֙ בְּנֵ֣י יוֹנָ֔ה אֶחָ֥ד לְעֹלָ֖ה וְאֶחָ֣ד לְחַטָּ֑את וְכִפֶּ֥ר עָלֶ֛יהָ הַכֹּהֵ֖ן וְטָהֵֽרָה׃ \petucha }
{וְאִם לָא תַשְׁכַּח יְדַהּ כְּמִסַּת אִמְּרָא וְתִסַּב ‏‏תְּרֵין שַׁפְנִינִין אוֹ תְּרֵין בְּנֵי יוֹנָה חַד לַעֲלָתָא וְחַד לְחַטָּתָא וִיכַפַּר עֲלַהּ כָּהֲנָא וְתִדְכֵּי׃}
{And if her means suffice not for a lamb, then she shall take two turtle-doves, or two young pigeons: the one for a burnt-offering, and the other for a sin-offering; and the priest shall make atonement for her, and she shall be clean.}{\arabic{verse}}
\rashi{\rashiDH{אחד לעולה ואחד לחטאת.} לא הקדימה הכתוב אלא למקראה, אבל להקרבה חטאת קודם לעולה, כך שנינו בזבחים (צ.) בפ׳ כל התדיר׃ 
}
\newperek
\threeverse{\aliya{ישראל}}%Leviticus13:1
{וַיְדַבֵּ֣ר יְהֹוָ֔ה אֶל\maqqaf מֹשֶׁ֥ה וְאֶֽל\maqqaf אַהֲרֹ֖ן לֵאמֹֽר׃}
{וּמַלֵּיל יְיָ עִם מֹשֶׁה וּלְאַהֲרֹן לְמֵימַר׃}
{And the \lord\space spoke unto Moses and unto Aaron, saying:}{\Roman{chap}}
\threeverse{\arabic{verse}}%Leviticus13:2
{אָדָ֗ם כִּֽי\maqqaf יִהְיֶ֤ה בְעוֹר\maqqaf בְּשָׂרוֹ֙ שְׂאֵ֤ת אֽוֹ\maqqaf סַפַּ֙חַת֙ א֣וֹ בַהֶ֔רֶת וְהָיָ֥ה בְעוֹר\maqqaf בְּשָׂר֖וֹ לְנֶ֣גַע צָרָ֑עַת וְהוּבָא֙ אֶל\maqqaf אַהֲרֹ֣ן הַכֹּהֵ֔ן א֛וֹ אֶל\maqqaf אַחַ֥ד מִבָּנָ֖יו הַכֹּהֲנִֽים׃}
{אֱנָשׁ אֲרֵי יְהֵי בִמְשַׁךְ בִּשְׂרֵיהּ עָמְקָא אוֹ עֶדְיָא אוֹ בַהֲרָא וִיהֵי בִמְשַׁךְ בִּסְרֵיהּ לְמַכְתָּשׁ סְגִירוּ וְיִתֵּיתֵי לְוָת אַהֲרֹן כָּהֲנָא אוֹ לְוָת חַד מִבְּנוֹהִי כָּהֲנַיָּא׃}
{When a man shall have in the skin of his flesh a rising, or a scab, or a bright spot, and it become in the skin of his flesh the plague of leprosy, then he shall be brought unto Aaron the priest, or unto one of his sons the priests.}{\arabic{verse}}
\rashi{\rashiDH{שאת או ספחת וגו׳.}שמות נגעים הם, ולבנות זו מזו (נגעים פ״א מ״א)׃\quad \rashiDH{בהרת.} חברבורות טייא״ר בלע״ז, וכן בָּהִיר הוּא בַּשְּׁחָקִים (איוב לז, כא)׃\quad \rashiDH{אל אהרן וגו׳.} גזירת הכתוב הוא, שאין טומאת נגעים וטהרתן אלא על פי כהן (ת״כ נגעים פרשתא א, ט)׃}
\threeverse{\arabic{verse}}%Leviticus13:3
{וְרָאָ֣ה הַכֹּהֵ֣ן אֶת\maqqaf הַנֶּ֣גַע בְּעֽוֹר\maqqaf הַ֠בָּשָׂ֠ר וְשֵׂעָ֨ר בַּנֶּ֜גַע הָפַ֣ךְ \legarmeh  לָבָ֗ן וּמַרְאֵ֤ה הַנֶּ֙גַע֙ עָמֹק֙ מֵע֣וֹר בְּשָׂר֔וֹ נֶ֥גַע צָרַ֖עַת ה֑וּא וְרָאָ֥הוּ הַכֹּהֵ֖ן וְטִמֵּ֥א אֹתֽוֹ׃}
{וְיִחְזֵי כָהֲנָא יָת מַכְתָּשָׁא בִּמְשַׁךְ בִּשְׂרֵיהּ וְשַׂעֲרָא בְמַכְתָּשָׁא אִתְהֲפֵיךְ לְמִחְוַר וּמִחְזֵי מַכְתָּשָׁא עַמִּיק מִמְּשַׁךְ בִּסְרֵיהּ מַכְתָּשׁ סְגִירוּתָא הוּא וְיִחְזֵינֵיהּ כָּהֲנָא וִיסַאֵיב יָתֵיהּ׃}
{And the priest shall look upon the plague in the skin of the flesh; and if the hair in the plague be turned white, and the appearance of the plague be deeper than the skin of his flesh, it is the plague of leprosy; and the priest shall look on him, and pronounce him unclean.}{\arabic{verse}}
\rashi{\rashiDH{ושער בנגע הפך לבן.} מתחלה שחור והפך ללבן בתוך הנגע. ומעוט שער שנים (ת״כ שם פרק ב, ג)׃\quad \rashiDH{עמוק מעור בשרו.} כל מראה לבן עמוק הוא, כמראה חמה, עמוקה מן הצל (שבועות ו׃)׃\quad \rashiDH{וטמא אותו.} יאמר לו טמא אתה, ששער לבן סימן טומאה, הוא גזירת הכתוב׃}
\threeverse{\arabic{verse}}%Leviticus13:4
{וְאִם\maqqaf בַּהֶ֩רֶת֩ לְבָנָ֨ה הִ֜וא בְּע֣וֹר בְּשָׂר֗וֹ וְעָמֹק֙ אֵין\maqqaf מַרְאֶ֣הָ מִן\maqqaf הָע֔וֹר וּשְׂעָרָ֖ה לֹא\maqqaf הָפַ֣ךְ לָבָ֑ן וְהִסְגִּ֧יר הַכֹּהֵ֛ן אֶת\maqqaf הַנֶּ֖גַע שִׁבְעַ֥ת יָמִֽים׃}
{וְאִם בַּהֲרָא חָוְרָא הִיא בִּמְשַׁךְ בִּסְרֵיהּ וְעַמִּיק לֵית מִחְזַהָא מִן מַשְׁכָּא וְשַׂעֲרָא לָא אִתְהֲפֵיךְ לְמִחְוַר וְיַסְגַּר כָּהֲנָא יָת מַכְתָּשָׁא שִׁבְעָא יוֹמִין׃}
{And if the bright spot be white in the skin of his flesh, and the appearance thereof be not deeper than the skin, and the hair thereof be not turned white, then the priest shall shut up him that hath the plague seven days.}{\arabic{verse}}
\rashi{\rashiDH{ועמוק אין מראה.} לא ידעתי פירושו׃\quad \rashiDH{והסגיר.} יסגירנו בבית אחד, ולא יראה עד סוף השבוע, ויוכיחו סימנים עליו׃}
\threeverse{\arabic{verse}}%Leviticus13:5
{וְרָאָ֣הוּ הַכֹּהֵן֮ בַּיּ֣וֹם הַשְּׁבִיעִי֒ וְהִנֵּ֤ה הַנֶּ֙גַע֙ עָמַ֣ד בְּעֵינָ֔יו לֹֽא\maqqaf פָשָׂ֥ה הַנֶּ֖גַע בָּע֑וֹר וְהִסְגִּיר֧וֹ הַכֹּהֵ֛ן שִׁבְעַ֥ת יָמִ֖ים שֵׁנִֽית׃}
{וְיִחְזֵינֵיהּ כָּהֲנָא בְּיוֹמָא שְׁבִיעָאָה וְהָא מַכְתָּשָׁא קָם כִּד הֲוָה לָא אוֹסֵיף מַכְתָּשָׁא בְמַשְׁכָּא וְיַסְגְּרִנֵּיהּ כָּהֲנָא שִׁבְעָא יוֹמִין תִּנְיָנוּת׃}
{And the priest shall look on him the seventh day; and, behold, if the plague stay in its appearance, and the plague be not spread in the skin, then the priest shall shut him up seven days more.}{\arabic{verse}}
\rashi{\rashiDH{בעיניו.} במראהו ובשיעורו הראשון׃ 
\quad \rashiDH{והסגירו, שנית.} הא אם פשה בשבוע ראשון טמא מוחלט׃}
\aliyacounter{שני}
\threeverse{\aliya{שני}}%Leviticus13:6
{וְרָאָה֩ הַכֹּהֵ֨ן אֹת֜וֹ בַּיּ֣וֹם הַשְּׁבִיעִי֮ שֵׁנִית֒ וְהִנֵּה֙ כֵּהָ֣ה הַנֶּ֔גַע וְלֹא\maqqaf פָשָׂ֥ה הַנֶּ֖גַע בָּע֑וֹר וְטִהֲר֤וֹ הַכֹּהֵן֙ מִסְפַּ֣חַת הִ֔וא וְכִבֶּ֥ס בְּגָדָ֖יו וְטָהֵֽר׃}
{וְיִחְזֵי כָהֲנָא יָתֵיהּ בְּיוֹמָא שְׁבִיעָאָה תִּנְיָנוּת וְהָא עֲמָא מַכְתָּשָׁא וְלָא אוֹסֵיף מַכְתָּשָׁא בְּמַשְׁכָּא וִידַכֵּינֵיהּ כָּהֲנָא עָדִיתָא הִיא וִיצַבַּע לְבוּשׁוֹהִי וְיִדְכֵּי׃}
{And the priest shall look on him again the seventh day; and, behold, if the plague be dim, and the plague be not spread in the skin, then the priest shall pronounce him clean: it is a scab; and he shall wash his clothes, and be clean.}{\arabic{verse}}
\rashi{\rashiDH{כהה.} הוכהה מראיתו, הא אם עמד במראיתו או פשה טמא׃\quad \rashiDH{מספחת.} שם נגע טהור׃\quad \rashiDH{וכבס בגדיו וטהר.} הואיל ונזקק להסגר נקרא טמא, וצריך טבילה׃}
\threeverse{\arabic{verse}}%Leviticus13:7
{וְאִם\maqqaf פָּשֹׂ֨ה תִפְשֶׂ֤ה הַמִּסְפַּ֙חַת֙ בָּע֔וֹר אַחֲרֵ֧י הֵרָאֹת֛וֹ אֶל\maqqaf הַכֹּהֵ֖ן לְטׇהֳרָת֑וֹ וְנִרְאָ֥ה שֵׁנִ֖ית אֶל\maqqaf הַכֹּהֵֽן׃}
{וְאִם אוֹסָפָא תוֹסֵיף עָדִיתָא בְּמַשְׁכָּא בָּתַר דְּאִתַּחְזִי לְכָהֲנָא לִדְכוּתֵיהּ וְיִתַּחְזֵי תִּנְיָנוּת לְכָהֲנָא׃}
{But if the scab spread abroad in the skin, after that he hath shown himself to the priest for his cleansing, he shall show himself to the priest again.}{\arabic{verse}}
\threeverse{\arabic{verse}}%Leviticus13:8
{וְרָאָה֙ הַכֹּהֵ֔ן וְהִנֵּ֛ה פָּשְׂתָ֥ה הַמִּסְפַּ֖חַת בָּע֑וֹר וְטִמְּא֥וֹ הַכֹּהֵ֖ן צָרַ֥עַת הִֽוא׃ \petucha }
{וְיִחְזֵי כָהֲנָא וְהָא אוֹסֵיפַת עָדִיתָא בְּמַשְׁכָּא וִיסַאֲבִנֵּיהּ כָּהֲנָא סְגִירוּתָא הִיא׃}
{And the priest shall look, and, behold, if the scab be spread in the skin, then the priest shall pronounce him unclean: it is leprosy.}{\arabic{verse}}
\rashi{\rashiDH{וטמאו הכהן.} ומשטמאו הרי הוא מוחלט, וזקוק לצפרים ולתגלחת ולקרבן האמור בפרשת זאת תהיה׃\quad \rashiDH{צרעת הוא.} המספחת הזאת׃\quad \rashiDH{צרעת.} לשון נקבה. נגע לשון זכר׃}
\threeverse{\arabic{verse}}%Leviticus13:9
{נֶ֣גַע צָרַ֔עַת כִּ֥י תִהְיֶ֖ה בְּאָדָ֑ם וְהוּבָ֖א אֶל\maqqaf הַכֹּהֵֽן׃}
{מַכְתָּשׁ סְגִירוּ אֲרֵי תְהֵי בַאֲנָשָׁא וְיִתֵּיתֵי לְוָת כָּהֲנָא׃}
{When the plague of leprosy is in a man, then he shall be brought unto the priest.}{\arabic{verse}}
\threeverse{\arabic{verse}}%Leviticus13:10
{וְרָאָ֣ה הַכֹּהֵ֗ן וְהִנֵּ֤ה שְׂאֵת\maqqaf לְבָנָה֙ בָּע֔וֹר וְהִ֕יא הָפְכָ֖ה שֵׂעָ֣ר לָבָ֑ן וּמִֽחְיַ֛ת בָּשָׂ֥ר חַ֖י בַּשְׂאֵֽת׃}
{וְיִחְזֵי כָהֲנָא וְהָא עָמְקָא חָוְרָא בְמַשְׁכָּא וְהִיא הֲפַכַת שַׂעֲרָא לְמִחְוַר וְרֹשֶׁם בְּשַׂר חַי בְּעַמִּיקְתָא׃}
{And the priest shall look, and, behold, if there be a white rising in the skin, and it have turned the hair white, and there be quick raw flesh in the rising,}{\arabic{verse}}
\rashi{\rashiDH{ומחית.} שנימינ״ט בלע״ז, שנהפך מקצת הלובן שבתוך השאת למראה בשר אף הוא סימן טומאה, שער לבן בלא מחיה, ומחיה בלא שער לבן, ואף על פי שלא נאמרה מחיה אלא בשאת, אף בכל המראות ותולדותיהן הוא סימן טומאה׃}
\threeverse{\arabic{verse}}%Leviticus13:11
{צָרַ֨עַת נוֹשֶׁ֤נֶת הִוא֙ בְּע֣וֹר בְּשָׂר֔וֹ וְטִמְּא֖וֹ הַכֹּהֵ֑ן לֹ֣א יַסְגִּרֶ֔נּוּ כִּ֥י טָמֵ֖א הֽוּא׃}
{סְגִירוּת עַתִּיקָא הִיא בִּמְשַׁךְ בִּשְׂרֵיהּ וִיסַאֲבִנֵּיהּ כָּהֲנָא לָא יַסְגְּרִנֵּיהּ אֲרֵי מְסָאַב הוּא׃}
{it is an old leprosy in the skin of his flesh, and the priest shall pronounce him unclean; he shall not shut him up; for he is unclean.}{\arabic{verse}}
\rashi{\rashiDH{צרעת נושנת היא.} מכה ישנה היא תחת המחיה, וחבורה זו נראית בריאה מלמעלה ותחתיה מלאה לחה, שלא תאמר הואיל ועלתה מחיה אטהרנה׃}
\threeverse{\arabic{verse}}%Leviticus13:12
{וְאִם\maqqaf פָּר֨וֹחַ תִּפְרַ֤ח הַצָּרַ֙עַת֙ בָּע֔וֹר וְכִסְּתָ֣ה הַצָּרַ֗עַת אֵ֚ת כׇּל\maqqaf ע֣וֹר הַנֶּ֔גַע מֵרֹאשׁ֖וֹ וְעַד\maqqaf רַגְלָ֑יו לְכׇל\maqqaf מַרְאֵ֖ה עֵינֵ֥י הַכֹּהֵֽן׃}
{וְאִם מִסְגָּא תִסְגֵּי סְגִירוּתָא בְמַשְׁכָּא וְתִחְפֵי סְגִירוּתָא יָת כָּל מְשַׁךְ מַכְתָּשָׁא מֵרֵישֵׁיהּ וְעַד רַגְלוֹהִי לְכָל חֵיזוּ עֵינֵי כָהֲנָא׃}
{And if the leprosy break out abroad in the skin, and the leprosy cover all the skin of him that hath the plague from his head even to his feet, as far as appeareth to the priest;}{\arabic{verse}}
\rashi{\rashiDH{מראשו.} של אדם ועד רגליו׃ 
\quad \rashiDH{לכל מראה עיני הכהן.} פרט לכהן שחשך מאורו׃}
\threeverse{\arabic{verse}}%Leviticus13:13
{וְרָאָ֣ה הַכֹּהֵ֗ן וְהִנֵּ֨ה כִסְּתָ֤ה הַצָּרַ֙עַת֙ אֶת\maqqaf כׇּל\maqqaf בְּשָׂר֔וֹ וְטִהַ֖ר אֶת\maqqaf הַנָּ֑גַע כֻּלּ֛וֹ הָפַ֥ךְ לָבָ֖ן טָה֥וֹר הֽוּא׃}
{וְיִחְזֵי כָהֲנָא וְהָא חֲפָת סְגִירוּתָא יָת כָּל בִּשְׂרֵיהּ וִידַכֵּי יָת מַכְתָּשָׁא כֻּלֵּיהּ אִתְהֲפֵיךְ לְמִחְוַר דְּכֵי הוּא׃}
{then the priest shall look; and, behold, if the leprosy have covered all his flesh, he shall pronounce him clean that hath the plague; it is all turned white: he is clean.}{\arabic{verse}}
\threeverse{\arabic{verse}}%Leviticus13:14
{וּבְי֨וֹם הֵרָא֥וֹת בּ֛וֹ בָּשָׂ֥ר חַ֖י יִטְמָֽא׃}
{וּבְיוֹמָא דְּיִתַּחְזֵי בֵיהּ בִּשְׂרָא חַיָּא יְהֵי מְסָאַב׃}
{But whensoever raw flesh appeareth in him, he shall be unclean.}{\arabic{verse}}
\rashi{\rashiDH{וביום הראות בו בשר חי.} אם צמחה בו מִחְיָה הרי כבר פירש שהמחיה סימן טומאה, אלא הרי שהיה הנגע בא׳ מעשרים וארבעה ראשי איברים שאין מטמאין משום מחיה, לפי שאין נראה הנגע כולו כאחד, ששופע אילך ואילך, וחזר ראש האבר ונתגלה שפועו ע״י שומן, כגון שהבריא ונעשה רחב ונראית בו המחיה, למדנו הכתוב שתטמא (שם פרק ה, א)׃\quad \rashiDH{וביום.} מה תלמוד לומר, ללמד יש יום שאתה רואה בו, ויש יום שאין אתה רואה בו, מכאן אמרו חתן נותנין לו כל שבעת ימי המשתה, לו ולאצטליתו ולכסותו ולביתו, וכן ברגל נותנין לו כל ימי הרגל׃}
\threeverse{\arabic{verse}}%Leviticus13:15
{וְרָאָ֧ה הַכֹּהֵ֛ן אֶת\maqqaf הַבָּשָׂ֥ר הַחַ֖י וְטִמְּא֑וֹ הַבָּשָׂ֥ר הַחַ֛י טָמֵ֥א ה֖וּא צָרַ֥עַת הֽוּא׃}
{וְיִחְזֵי כָהֲנָא יָת בִּשְׂרָא חַיָא וִיסַאֲבִנֵּיהּ בִּשְׂרָא חַיָּא מְסָאַב הוּא סְגִירוּתָא הוּא׃}
{And the priest shall look on the raw flesh, and pronounce him unclean; the raw flesh is unclean: it is leprosy.}{\arabic{verse}}
\rashi{\rashiDH{צרעת הוא.} הבשר ההוא, בשר לשון זכר׃}
\threeverse{\arabic{verse}}%Leviticus13:16
{א֣וֹ כִ֥י יָשׁ֛וּב הַבָּשָׂ֥ר הַחַ֖י וְנֶהְפַּ֣ךְ לְלָבָ֑ן וּבָ֖א אֶל\maqqaf הַכֹּהֵֽן׃}
{אוֹ אֲרֵי יְתוּב בִּשְׂרָא חַיָּא וְיִתְהֲפֵיךְ לְמִחְוַר וְיֵיתֵי לְוָת כָּהֲנָא׃}
{But if the raw flesh again be turned into white, then he shall come unto the priest;}{\arabic{verse}}
\threeverse{\arabic{verse}}%Leviticus13:17
{וְרָאָ֙הוּ֙ הַכֹּהֵ֔ן וְהִנֵּ֛ה נֶהְפַּ֥ךְ הַנֶּ֖גַע לְלָבָ֑ן וְטִהַ֧ר הַכֹּהֵ֛ן אֶת\maqqaf הַנֶּ֖גַע טָה֥וֹר הֽוּא׃ \petucha }
{וְיִחְזֵינֵיהּ כָּהֲנָא וְהָא אִתְהֲפֵיךְ מַכְתָּשָׁא לְמִחְוַר וִידַכֵּי כָהֲנָא יָת מַכְתָּשָׁא דְּכֵי הוּא׃}
{and the priest shall look on him; and, behold, if the plague be turned into white, then the priest shall pronounce him clean that hath the plague: he is clean.}{\arabic{verse}}
\aliyacounter{שלישי}
\threeverse{\aliya{שלישי}}%Leviticus13:18
{וּבָשָׂ֕ר כִּֽי\maqqaf יִהְיֶ֥ה בֽוֹ\maqqaf בְעֹר֖וֹ שְׁחִ֑ין וְנִרְפָּֽא׃}
{וֶאֱנָשׁ אֲרֵי יְהֵי בֵיהּ בְּמַשְׁכֵּיהּ שִׁחְנָא וְיִתַּסֵּי׃}
{And when the flesh hath in the skin thereof a boil, and it is healed,}{\arabic{verse}}
\rashi{\rashiDH{שחין.} לשון חמום, שנתחמם הבשר בלקוי הבא לו מחמת מכה שלא מחמת האור (חולין ח.)׃\quad \rashiDH{ונרפא.} השחין העלה ארוכה ובמקומו העלה נגע אחר׃}
\threeverse{\arabic{verse}}%Leviticus13:19
{וְהָיָ֞ה בִּמְק֤וֹם הַשְּׁחִין֙ שְׂאֵ֣ת לְבָנָ֔ה א֥וֹ בַהֶ֖רֶת לְבָנָ֣ה אֲדַמְדָּ֑מֶת וְנִרְאָ֖ה אֶל\maqqaf הַכֹּהֵֽן׃}
{וִיהֵי בַּאֲתַר שִׁחְנָא עָמְקָא חָוְרָא אוֹ בַהֲרָא חָוְרָא סָמְקָא וְיִתַּחְזֵי לְכָהֲנָא׃}
{and in the place of the boil there is a white rising, or a bright spot, reddish-white, then it shall be shown to the priest.}{\arabic{verse}}
\rashi{\rashiDH{או בהרת לבנה אדמדמת.} שאין הנגע לבן חלק, אלא פָּתוּךְ ומעורב בשתי מראות לובן ואודם׃}
\threeverse{\arabic{verse}}%Leviticus13:20
{וְרָאָ֣ה הַכֹּהֵ֗ן וְהִנֵּ֤ה מַרְאֶ֙הָ֙ שָׁפָ֣ל מִן\maqqaf הָע֔וֹר וּשְׂעָרָ֖הּ הָפַ֣ךְ לָבָ֑ן וְטִמְּא֧וֹ הַכֹּהֵ֛ן נֶֽגַע\maqqaf צָרַ֥עַת הִ֖וא בַּשְּׁחִ֥ין פָּרָֽחָה׃}
{וְיִחְזֵי כָהֲנָא וְהָא מִחְזַהָא מַכִּיךְ מִן מַשְׁכָּא וְשַׂעֲרַהּ אִתְהֲפֵיךְ לְמִחְוַר וִיסַאֲבִנֵּיהּ כָּהֲנָא מַכְתָּשׁ סְגִירוּתָא הִיא בְּשִׁחְנָא סְגִיאַת׃}
{And the priest shall look; and, behold, if the appearance thereof be lower than the skin, and the hair thereof be turned white, then the priest shall pronounce him unclean: it is the plague of leprosy, it hath broken out in the boil.}{\arabic{verse}}
\rashi{\rashiDH{מראה שפל.} ואין ממשו שפל, אלא מתוך לבנינותו הוא נראה שפל ועמוק, כמראה חמה עמוקה מן הצל׃ 
}
\threeverse{\arabic{verse}}%Leviticus13:21
{וְאִ֣ם \legarmeh  יִרְאֶ֣נָּה הַכֹּהֵ֗ן וְהִנֵּ֤ה אֵֽין\maqqaf בָּהּ֙ שֵׂעָ֣ר לָבָ֔ן וּשְׁפָלָ֥ה אֵינֶ֛נָּה מִן\maqqaf הָע֖וֹר וְהִ֣יא כֵהָ֑ה וְהִסְגִּיר֥וֹ הַכֹּהֵ֖ן שִׁבְעַ֥ת יָמִֽים׃}
{וְאִם יִחְזֵינַהּ כָּהֲנָא וְהָא לֵית בַּהּ שְׂעַר חִוָּר וּמַכִּיכָא לָיְתַהָא מִן מַשְׁכָּא וְהִיא עָמְיָא וְיַסְגְּרִנֵּיהּ כָּהֲנָא שִׁבְעָא יוֹמִין׃}
{But if the priest look on it, and, behold, there be no white hairs therein, and it be not lower than the skin, but be dim, then the priest shall shut him up seven days.}{\arabic{verse}}
\threeverse{\arabic{verse}}%Leviticus13:22
{וְאִם\maqqaf פָּשֹׂ֥ה תִפְשֶׂ֖ה בָּע֑וֹר וְטִמֵּ֧א הַכֹּהֵ֛ן אֹת֖וֹ נֶ֥גַע הִֽוא׃}
{וְאִם אוֹסָפָא תוֹסֵיף בְּמַשְׁכָּא וִיסַאֵיב כָּהֲנָא יָתֵיהּ מַכְתָּשָׁא הִיא׃}
{And if it spread abroad in the skin, then the priest shall pronounce him unclean: it is a plague.}{\arabic{verse}}
\rashi{\rashiDH{נגע הוא.} השאת הזאת או הבהרת׃}
\threeverse{\arabic{verse}}%Leviticus13:23
{וְאִם\maqqaf תַּחְתֶּ֜יהָ תַּעֲמֹ֤ד הַבַּהֶ֙רֶת֙ לֹ֣א פָשָׂ֔תָה צָרֶ֥בֶת הַשְּׁחִ֖ין הִ֑וא וְטִהֲר֖וֹ הַכֹּהֵֽן׃ \setuma }
{וְאִם בְּאַתְרַהּ קַמַת בַּהַרְתָּא לָא אוֹסֵיפַת רֹשֶׁם שִׁחְנָא הִיא וִידַכֵּינֵיהּ כָּהֲנָא׃}
{But if the bright spot stay in its place, and be not spread, it is the scar of the boil; and the priest shall pronounce him clean.}{\arabic{verse}}
\rashi{\rashiDH{תחתיה.} במקומה׃\quad \rashiDH{צרבת השחין.} כתרגומו רשֶׁם שִׁחֲנָא, אינו אלא רושם החמום הניכר בבשר. כל צרבת לשון רגיעת עור הנרגע מחמת חימום, כמו וְנִצְרְבוּ בָהּ כָּל פָּנִים (יחזקאל כא, ג), רייטרי״ד בלע״ז׃\quad \rashiDH{צרבת.} רייטרי״שמענט בלע״ז׃}
\aliyacounter{רביעי}
\threeverse{\aliya{רביעי\newline (שני)}}%Leviticus13:24
{א֣וֹ בָשָׂ֔ר כִּֽי\maqqaf יִהְיֶ֥ה בְעֹר֖וֹ מִכְוַת\maqqaf אֵ֑שׁ וְֽהָיְתָ֞ה מִֽחְיַ֣ת הַמִּכְוָ֗ה בַּהֶ֛רֶת לְבָנָ֥ה אֲדַמְדֶּ֖מֶת א֥וֹ לְבָנָֽה׃}
{אוֹ אֱנָשׁ אֲרֵי יְהֵי בְמַשְׁכֵּיהּ כְּוַאָה דְּנוּר וִיהֵי רֹשֶׁם כְּוַאָה בַּהֲרָא חָוְרָא סָמְקָא אוֹ חָוְרָא׃}
{Or when the flesh hath in the skin thereof a burning by fire, and the quick flesh of the burning become a bright spot, reddish-white, or white;}{\arabic{verse}}
\rashi{\rashiDH{מחית המכוה.} שנימני״ט בלע״ז, כשחיתה המכוה נהפכה לבהרת פתוכה או לבנה חלקה. וסימני מכוה וסימני שחין שוים הם, ולמה חלקן הכתוב, לומר שאין מצטרפין זה עם זה, נולד חצי גריס בשחין וחצי גריס במכוה, לא ידונו כגריס (חולין שם)׃ 
}
\threeverse{\arabic{verse}}%Leviticus13:25
{וְרָאָ֣ה אֹתָ֣הּ הַכֹּהֵ֡ן וְהִנֵּ֣ה נֶהְפַּךְ֩ שֵׂעָ֨ר לָבָ֜ן בַּבַּהֶ֗רֶת וּמַרְאֶ֙הָ֙ עָמֹ֣ק מִן\maqqaf הָע֔וֹר צָרַ֣עַת הִ֔וא בַּמִּכְוָ֖ה פָּרָ֑חָה וְטִמֵּ֤א אֹתוֹ֙ הַכֹּהֵ֔ן נֶ֥גַע צָרַ֖עַת הִֽוא׃}
{וְיִחְזֵי יָתַהּ כָּהֲנָא וְהָא אִתְהֲפֵיךְ שַׂעֲרָא לְמִחְוַר בְּבַהַרְתָּא וּמִחְזַהָא עַמִּיק מִן מַשְׁכָּא סְגִירוּתָא הִיא בִּכְוַאָה סְגִיאַת וִיסַאֵיב יָתֵיהּ כָּהֲנָא מַכְתָּשׁ סְגִירוּתָא הִיא׃}
{then the priest shall look upon it; and, behold, if the hair in the bright spot be turned white, and the appearance thereof be deeper than the skin, it is leprosy, it hath broken out in the burning; and the priest shall pronounce him unclean: it is the plague of leprosy.}{\arabic{verse}}
\threeverse{\arabic{verse}}%Leviticus13:26
{וְאִ֣ם \legarmeh  יִרְאֶ֣נָּה הַכֹּהֵ֗ן וְהִנֵּ֤ה אֵֽין\maqqaf בַּבַּהֶ֙רֶת֙ שֵׂעָ֣ר לָבָ֔ן וּשְׁפָלָ֥ה אֵינֶ֛נָּה מִן\maqqaf הָע֖וֹר וְהִ֣וא כֵהָ֑ה וְהִסְגִּיר֥וֹ הַכֹּהֵ֖ן שִׁבְעַ֥ת יָמִֽים׃}
{וְאִם יִחְזֵינַהּ כָּהֲנָא וְהָא לֵית בְּבַהַרְתָּא סְעַר חִיוָר וּמַכִּיכָא לָיְתַהָא מִן מַשְׁכָּא וְהִיא עָמְיָא וְיַסְגְּרִנֵּיהּ כָּהֲנָא שִׁבְעָא יוֹמִין׃}
{But if the priest look on it, and, behold, there be no white hair in the bright spot, and it be no lower than the skin, but be dim; then the priest shall shut him up seven days.}{\arabic{verse}}
\threeverse{\arabic{verse}}%Leviticus13:27
{וְרָאָ֥הוּ הַכֹּהֵ֖ן בַּיּ֣וֹם הַשְּׁבִיעִ֑י אִם\maqqaf פָּשֹׂ֤ה תִפְשֶׂה֙ בָּע֔וֹר וְטִמֵּ֤א הַכֹּהֵן֙ אֹת֔וֹ נֶ֥גַע צָרַ֖עַת הִֽוא׃}
{וְיִחְזֵינֵיהּ כָּהֲנָא בְּיוֹמָא שְׁבִיעָאָה אִם אוֹסָפָא תוֹסֵיף בְּמַשְׁכָּא וִיסַאֵיב כָּהֲנָא יָתֵיהּ מַכְתָּשׁ סְגִירוּתָא הִיא׃}
{And the priest shall look upon him the seventh day; if it spread abroad in the skin, then the priest shall pronounce him unclean: it is the plague of leprosy.}{\arabic{verse}}
\threeverse{\arabic{verse}}%Leviticus13:28
{וְאִם\maqqaf תַּחְתֶּ֩יהָ֩ תַעֲמֹ֨ד הַבַּהֶ֜רֶת לֹא\maqqaf פָשְׂתָ֤ה בָעוֹר֙ וְהִ֣וא כֵהָ֔ה שְׂאֵ֥ת הַמִּכְוָ֖ה הִ֑וא וְטִֽהֲרוֹ֙ הַכֹּהֵ֔ן כִּֽי\maqqaf צָרֶ֥בֶת הַמִּכְוָ֖ה הִֽוא׃ \petucha }
{וְאִם בְּאַתְרַהּ קַמַת בַּהַרְתָּא לָא אוֹסֵיפַת בְּמַשְׁכָּא וְהִיא עָמְיָא עוֹמֶק כְּוַאָה הִיא וִידַכֵּינֵיהּ כָּהֲנָא אֲרֵי רוֹשֶׁם כְּוַאָה הִיא׃}
{And if the bright spot stay in its place, and be not spread in the skin, but be dim, it is the rising of the burning, and the priest shall pronounce him clean; for it is the scar of the burning.}{\arabic{verse}}
\aliyacounter{חמישי}
\newseder{8}
\threeverse{\aliya{חמישי}\newline\vspace{-4pt}\newline\seder{ח}}%Leviticus13:29
{וְאִישׁ֙ א֣וֹ אִשָּׁ֔ה כִּֽי\maqqaf יִהְיֶ֥ה ב֖וֹ נָ֑גַע בְּרֹ֖אשׁ א֥וֹ בְזָקָֽן׃}
{וּגְבַר אוֹ אִתָּא אֲרֵי יְהֵי בֵיהּ מַכְתָּשָׁא בְּרֵישׁ אוֹ בִדְקַן׃}
{And when a man or woman hath a plague upon the head or upon the beard,}{\arabic{verse}}
\rashi{\rashiDH{בראש או בזקן.} בא הכתוב לחלק בין נגע שבמקום שער לנגע שבמקום בשר, שזה סימנו בשער לבן, וזה סימנו בשער צהוב (ת״כ פרשתא ה, ה)׃}
\threeverse{\arabic{verse}}%Leviticus13:30
{וְרָאָ֨ה הַכֹּהֵ֜ן אֶת\maqqaf הַנֶּ֗גַע וְהִנֵּ֤ה מַרְאֵ֙הוּ֙ עָמֹ֣ק מִן\maqqaf הָע֔וֹר וּב֛וֹ שֵׂעָ֥ר צָהֹ֖ב דָּ֑ק וְטִמֵּ֨א אֹת֤וֹ הַכֹּהֵן֙ נֶ֣תֶק ה֔וּא צָרַ֧עַת הָרֹ֛אשׁ א֥וֹ הַזָּקָ֖ן הֽוּא׃}
{וְיִחְזֵי כָהֲנָא יָת מַכְתָּשָׁא וְהָא מִחְזוֹהִי עַמִּיק מִן מַשְׁכָּא וּבֵיהּ שְׂעַר סוּמָּק דַּעְדַּק וִיסַאֵיב יָתֵיהּ כָּהֲנָא נִתְקָא הוּא סְגִירוּת רֵישָׁא אוֹ דִּקְנָא הוּא׃}
{then the priest shall look on the plague; and, behold, if the appearance thereof be deeper than the skin, and there be in it yellow thin hair, then the priest shall pronounce him unclean: it is a scall, it is leprosy of the head or of the beard.}{\arabic{verse}}
\rashi{\rashiDH{ובו שער צהוב.} שנהפך שער שחור שבו לצהוב׃\quad \rashiDH{נתק הוא.} כך שמו של נגע שבמקום שער׃}
\threeverse{\arabic{verse}}%Leviticus13:31
{וְכִֽי\maqqaf יִרְאֶ֨ה הַכֹּהֵ֜ן אֶת\maqqaf נֶ֣גַע הַנֶּ֗תֶק וְהִנֵּ֤ה אֵין\maqqaf מַרְאֵ֙הוּ֙ עָמֹ֣ק מִן\maqqaf הָע֔וֹר וְשֵׂעָ֥ר שָׁחֹ֖ר אֵ֣ין בּ֑וֹ וְהִסְגִּ֧יר הַכֹּהֵ֛ן אֶת\maqqaf נֶ֥גַע הַנֶּ֖תֶק שִׁבְעַ֥ת יָמִֽים׃}
{וַאֲרֵי יִחְזֵי כָהֲנָא יָת מַכְתָּשׁ נִתְקָא וְהָא לֵית מִחְזוֹהִי עַמִּיק מִן מַשְׁכָּא וּשְׂעַר אוּכָּם לֵית בֵּיהּ וְיַסְגַּר כָּהֲנָא יָת מַכְתָּשׁ נִתְקָא שִׁבְעָא יוֹמִין׃}
{And if the priest look on the plague of the scall, and, behold, the appearance thereof be not deeper than the skin, and there be no black hair in it, then the priest shall shut up him that hath the plague of the scall seven days.}{\arabic{verse}}
\rashi{\rashiDH{ושער שחור אין בו.} הא אם היה בו שער שחור טהור, ואין צריך להסגר, ששער שחור סימן טהרה הוא בנתקים, כמו שנאמר ושער שחור צמח בו וגו׳׃ 
}
\threeverse{\arabic{verse}}%Leviticus13:32
{וְרָאָ֨ה הַכֹּהֵ֣ן אֶת\maqqaf הַנֶּ֘גַע֮ בַּיּ֣וֹם הַשְּׁבִיעִי֒ וְהִנֵּה֙ לֹא\maqqaf פָשָׂ֣ה הַנֶּ֔תֶק וְלֹא\maqqaf הָ֥יָה ב֖וֹ שֵׂעָ֣ר צָהֹ֑ב וּמַרְאֵ֣ה הַנֶּ֔תֶק אֵ֥ין עָמֹ֖ק מִן\maqqaf הָעֽוֹר׃}
{וְיִחְזֵי כָהֲנָא יָת מַכְתָּשָׁא בְּיוֹמָא שְׁבִיעָאָה וְהָא לָא אוֹסֵיף נִתְקָא וְלָא הֲוָה בֵיהּ שְׂעַר סוּמָּק וּמִחְזֵי נִתְקָא לֵית עַמִּיק מִן מַשְׁכָּא׃}
{And in the seventh day the priest shall look on the plague; and, behold, if the scall be not spread, and there be in it no yellow hair, and the appearance of the scall be not deeper than the skin,}{\arabic{verse}}
\rashi{\rashiDH{והנה לא פשה וגו׳.} הא אם פשה, או היה בו שער צהוב, טמא׃}
\threeverse{\arabic{verse}}%Leviticus13:33
{וְהִ֨תְ\large גַּ\normalsize לָּ֔ח וְאֶת\maqqaf הַנֶּ֖תֶק לֹ֣א יְגַלֵּ֑חַ וְהִסְגִּ֨יר הַכֹּהֵ֧ן אֶת\maqqaf הַנֶּ֛תֶק שִׁבְעַ֥ת יָמִ֖ים שֵׁנִֽית׃}
{וִיגַלַּח סַחְרָנֵי נִתְקָא וּדְעִם נִתְקָא לָא יְגַלַּח וְיַסְגַּר כָּהֲנָא יָת נִתְקָא שִׁבְעָא יוֹמִין תִּנְיָנוּת׃}
{then he shall be shaven, but the scall shall he not shave; and the priest shall shut up him that hath the scall seven days more.}{\arabic{verse}}
\rashi{\rashiDH{והתגלח.} סביבות הנתק׃\quad \rashiDH{ואת הנתק לא יגלח.} מניח שתי שערות סמוך לו סביב, כדי שיהא ניכר אם פשה, שאם יפשה יעבור השערות ויצא למקום הגילוח׃ 
}
\threeverse{\arabic{verse}}%Leviticus13:34
{וְרָאָה֩ הַכֹּהֵ֨ן אֶת\maqqaf הַנֶּ֜תֶק בַּיּ֣וֹם הַשְּׁבִיעִ֗י וְ֠הִנֵּ֠ה לֹא\maqqaf פָשָׂ֤ה הַנֶּ֙תֶק֙ בָּע֔וֹר וּמַרְאֵ֕הוּ אֵינֶ֥נּוּ עָמֹ֖ק מִן\maqqaf הָע֑וֹר וְטִהַ֤ר אֹתוֹ֙ הַכֹּהֵ֔ן וְכִבֶּ֥ס בְּגָדָ֖יו וְטָהֵֽר׃}
{וְיִחְזֵי כָהֲנָא יָת נִתְקָא בְּיוֹמָא שְׁבִיעָאָה וְהָא לָא אוֹסֵיף נִתְקָא בְּמַשְׁכָּא וּמִחְזוֹהִי לָיְתוֹהִי עַמִּיק מִן מַשְׁכָּא וִידַכֵּי יָתֵיהּ כָּהֲנָא וִיצַבַּע לְבוּשׁוֹהִי וְיִדְכֵּי׃}
{And in the seventh day the priest shall look on the scall; and, behold, if the scall be not spread in the skin, and the appearance thereof be not deeper than the skin, then the priest shall pronounce him clean; and he shall wash his clothes, and be clean.}{\arabic{verse}}
\threeverse{\arabic{verse}}%Leviticus13:35
{וְאִם\maqqaf פָּשֹׂ֥ה יִפְשֶׂ֛ה הַנֶּ֖תֶק בָּע֑וֹר אַחֲרֵ֖י טׇהֳרָתֽוֹ׃}
{וְאִם אוֹסָפָא יוֹסֵיף נִתְקָא בְּמַשְׁכָּא בָּתַר דְּכוּתֵיהּ׃}
{But if the scall spread abroad in the skin after his cleansing,}{\arabic{verse}}
\rashi{\rashiDH{אחרי טהרתו.} אין לי אלא פושה לאחר הפטור, מנין אף בסוף שבוע ראשון ובסוף שבוע שני, תלמוד לומר פשה יפשה׃}
\threeverse{\arabic{verse}}%Leviticus13:36
{וְרָאָ֙הוּ֙ הַכֹּהֵ֔ן וְהִנֵּ֛ה פָּשָׂ֥ה הַנֶּ֖תֶק בָּע֑וֹר לֹֽא\maqqaf יְבַקֵּ֧ר הַכֹּהֵ֛ן לַשֵּׂעָ֥ר הַצָּהֹ֖ב טָמֵ֥א הֽוּא׃}
{וְיִחְזֵינֵיהּ כָּהֲנָא וְהָא אוֹסֵיף נִתְקָא בְמַשְׁכָּא לָא יְבַקַּר כָּהֲנָא לִשְׂעַר סוּמָּק מְסָאַב הוּא׃}
{then the priest shall look on him; and, behold, if the scall be spread in the skin, the priest shall not seek for the yellow hair: he is unclean.}{\arabic{verse}}
\threeverse{\arabic{verse}}%Leviticus13:37
{וְאִם\maqqaf בְּעֵינָיו֩ עָמַ֨ד הַנֶּ֜תֶק וְשֵׂעָ֨ר שָׁחֹ֧ר צָֽמַח\maqqaf בּ֛וֹ נִרְפָּ֥א הַנֶּ֖תֶק טָה֣וֹר ה֑וּא וְטִהֲר֖וֹ הַכֹּהֵֽן׃ \setuma }
{וְאִם כִּד הֲוָה קָם נִתְקָא וּסְעַר אוּכָּם צְמַח בֵּיהּ אִתַּסִּי נִתְקָא דְּכֵי הוּא וִידַכֵּינֵיהּ כָּהֲנָא׃}
{But if the scall stay in its appearance, and black hair be grown up therein; the scall is healed, he is clean; and the priest shall pronounce him clean.}{\arabic{verse}}
\rashi{\rashiDH{ושער שחר.} מנין אף הירוק והאדום שאינו צהוב, תלמוד לומר ושער. ולשון צהוב, דומה לתבנית הזהב. צהוב כמו זהוב, אור״בלא בלע״ז׃\quad \rashiDH{טהור הוא וטהרו הכהן.} הא טמא שטהרו הכהן לא טהור (מועד קטן ז׃)׃ 
}
\threeverse{\arabic{verse}}%Leviticus13:38
{וְאִישׁ֙ אֽוֹ\maqqaf אִשָּׁ֔ה כִּֽי\maqqaf יִהְיֶ֥ה בְעוֹר\maqqaf בְּשָׂרָ֖ם בֶּהָרֹ֑ת בֶּהָרֹ֖ת לְבָנֹֽת׃}
{וּגְבַר אוֹ אִתָּא אֲרֵי יְהֵי בִמְשַׁךְ בִּשְׂרְהוֹן בַּהֲרָן בַּהֲרָן חָוְרָן׃}
{And if a man or a woman have in the skin of their flesh bright spots, even white bright spots;}{\arabic{verse}}
\rashi{\rashiDH{בהרת.} חברבורות׃}
\threeverse{\arabic{verse}}%Leviticus13:39
{וְרָאָ֣ה הַכֹּהֵ֗ן וְהִנֵּ֧ה בְעוֹר\maqqaf בְּשָׂרָ֛ם בֶּהָרֹ֖ת כֵּה֣וֹת לְבָנֹ֑ת בֹּ֥הַק ה֛וּא פָּרַ֥ח בָּע֖וֹר טָה֥וֹר הֽוּא׃ \setuma }
{וְיִחְזֵי כָהֲנָא וְהָא בִמְשַׁךְ בִּשְׂרְהוֹן בַּהֲרָן עָמְיָן חָוְרָן בֻּהְקָא הוּא סְגִי בְמַשְׁכָּא דְּכֵי הוּא׃}
{then the priest shall look; and, behold, if the bright spots in the skin of their flesh be of a dull white, it is a tetter, it hath broken out in the skin: he is clean.}{\arabic{verse}}
\rashi{\rashiDH{כהות לבנות.} שאין לובן שלהן עז אלא כהה׃ 
\quad \rashiDH{בהק.} כמין לובן, הנראה בבשר אדם, אדום שקורין רוש״ו בין חברבורות אדמימותו קרויה בהק, כאיש עדשן שבין עדשה לעדשה מבהיק הבשר בלובן צח׃}
\aliyacounter{ששי}
\threeverse{\aliya{ששי\newline (שלישי)}}%Leviticus13:40
{וְאִ֕ישׁ כִּ֥י יִמָּרֵ֖ט רֹאשׁ֑וֹ קֵרֵ֥חַ ה֖וּא טָה֥וֹר הֽוּא׃}
{וּגְבַר אֲרֵי יִתַּר שְׂעַר רֵישֵׁיהּ קְרִיח הוּא דְּכֵי הוּא׃}
{And if a man’s hair be fallen off his head, he is bald; yet is he clean.}{\arabic{verse}}
\rashi{\rashiDH{קרח הוא טהור הוא.} טהור מטומאת נתקין, שאינו נדון בסימני ראש וזקן שהם מקום שער, אלא בסימני נגעי עור בשר, בשער לבן ומחיה ופשיון׃}
\threeverse{\arabic{verse}}%Leviticus13:41
{וְאִם֙ מִפְּאַ֣ת פָּנָ֔יו יִמָּרֵ֖ט רֹאשׁ֑וֹ גִּבֵּ֥חַ ה֖וּא טָה֥וֹר הֽוּא׃}
{וְאִם מִקֳּבֵיל אַפּוֹהִי יִתַּר שְׂעַר רֵישֵׁיהּ גְּלִישׁ הוּא דְּכֵי הוּא׃}
{And if his hair be fallen off from the front part of his head, he is forehead-bald; yet is he clean.}{\arabic{verse}}
\rashi{\rashiDH{ואם מפאת פניו.} משפוע קדקד כלפי פניו, קרוי גבחת, ואף הצדעין שמכאן ומכאן בכלל, ומשפוע קדקד כלפי אחוריו, קרוי קרחת׃}
\threeverse{\arabic{verse}}%Leviticus13:42
{וְכִֽי\maqqaf יִהְיֶ֤ה בַקָּרַ֙חַת֙ א֣וֹ בַגַּבַּ֔חַת נֶ֖גַע לָבָ֣ן אֲדַמְדָּ֑ם צָרַ֤עַת פֹּרַ֙חַת֙ הִ֔וא בְּקָרַחְתּ֖וֹ א֥וֹ בְגַבַּחְתּֽוֹ׃}
{וַאֲרֵי יְהֵי בְקַרְחוּתָא אוֹ בִגְלֵישׁוּתָא מַכְתָּשׁ חִיוָר סָמוֹק סְגִירוּת סָגְיָא הִיא בְּקַרְחוּתֵיהּ אוֹ בִגְלֵישׁוּתֵיהּ׃}
{But if there be in the bald head, or the bald forehead, a reddish-white plague, it is leprosy breaking out in his bald head, or his bald forehead.}{\arabic{verse}}
\rashi{\rashiDH{נגע לבן אדמדם.} פָּתוּךְ. מניין שאר המראות תלמוד לומר כמראה צרעת עור בשר, כמראה הצרעת האמור בפרשת עור בשר, אדם כי יהיה בעור בשרו, ומהו אמור בו, שמטמא בארבע מראות, ונדון בב׳ שבועות, ולא כמראה צרעת האמור בשחין ומכוה שהוא נדון בשבוע א׳, ולא כמראה נתקין של מקום שער שאין מטמאין בארבע מראות, שאת ותולדתה, בהרת ותולדתה׃}
\threeverse{\arabic{verse}}%Leviticus13:43
{וְרָאָ֨ה אֹת֜וֹ הַכֹּהֵ֗ן וְהִנֵּ֤ה שְׂאֵת\maqqaf הַנֶּ֙גַע֙ לְבָנָ֣ה אֲדַמְדֶּ֔מֶת בְּקָרַחְתּ֖וֹ א֣וֹ בְגַבַּחְתּ֑וֹ כְּמַרְאֵ֥ה צָרַ֖עַת ע֥וֹר בָּשָֽׂר׃}
{וְיִחְזֵי יָתֵיהּ כָּהֲנָא וְהָא עוֹמֶק מַכְתָּשָׁא חִיוָר סָמוֹק בְּקַרְחוּתֵיהּ אוֹ בִגְלֵישׁוּתֵיהּ כְּמִחְזֵי סְגִירוּת מְשַׁךְ בִּשְׂרָא׃}
{Then the priest shall look upon him; and, behold, if the rising of the plague be reddish-white in his bald head, or in his bald forehead, as the appearance of leprosy in the skin of the flesh,}{\arabic{verse}}
\threeverse{\arabic{verse}}%Leviticus13:44
{אִישׁ\maqqaf צָר֥וּעַ ה֖וּא טָמֵ֣א ה֑וּא טַמֵּ֧א יְטַמְּאֶ֛נּוּ הַכֹּהֵ֖ן בְּרֹאשׁ֥וֹ נִגְעֽוֹ׃}
{גְּבַר סְגִיר הוּא מְסָאַב הוּא סַאָבָא יְסַאֲבִנֵּיהּ כָּהֲנָא בְּרֵישֵׁיהּ מַכְתָּשֵׁיהּ׃}
{he is a leprous man, he is unclean; the priest shall surely pronounce him unclean: his plague is in his head. .}{\arabic{verse}}
\rashi{\rashiDH{בראשו נגעו.} אין לי אלא נתקין, מנין לרבות שאר המנוגעים, תלמוד לומר טמא יטמאנו, לרבות את כולן. על כולן הוא אומר בגדיו יהיו פרומים וגו׳׃ 
}
\threeverse{\arabic{verse}}%Leviticus13:45
{וְהַצָּר֜וּעַ אֲשֶׁר\maqqaf בּ֣וֹ הַנֶּ֗גַע בְּגָדָ֞יו יִהְי֤וּ פְרֻמִים֙ וְרֹאשׁוֹ֙ יִהְיֶ֣ה פָר֔וּעַ וְעַל\maqqaf שָׂפָ֖ם יַעְטֶ֑ה וְטָמֵ֥א \pasek  טָמֵ֖א יִקְרָֽא׃}
{וּסְגִירָא דְּבֵיהּ מַכְתָּשָׁא לְבוּשׁוֹהִי יְהוֹן מְבַזְּעִין וְרֵישֵׁיהּ יְהֵי פְרִיעַ וְעַל שָׂפָם כַּאֲבִילָא יִתְעַטַּף וְלָא תִסְתָּאֲבוּ וְלָא תִסְתָּאֲבוּ יִקְרֵי׃}
{And the leper in whom the plague is, his clothes shall be rent, and the hair of his head shall go loose, and he shall cover his upper lip, and shall cry: ‘Unclean, unclean.’}{\arabic{verse}}
\rashi{\rashiDH{פרומים.} קרועים (מ״ק טו.)׃\quad \rashiDH{פרוע.} מְגֻדָּל שֵׂעָר׃\quad \rashiDH{ועל שפם יעטה.} כאבל׃\quad \rashiDH{שפם.} שער השפתים, גרינו״ן בלע״ז׃\quad \rashiDH{וטמא טמא יקרא.} משמיע שהוא טמא ויפרשו ממנו (מ״ק ה.)׃ 
}
\threeverse{\arabic{verse}}%Leviticus13:46
{כׇּל\maqqaf יְמֵ֞י אֲשֶׁ֨ר הַנֶּ֥גַע בּ֛וֹ יִטְמָ֖א טָמֵ֣א ה֑וּא בָּדָ֣ד יֵשֵׁ֔ב מִח֥וּץ לַֽמַּחֲנֶ֖ה מוֹשָׁבֽוֹ׃ \setuma }
{כָּל יוֹמִין דְּמַכְתָּשָׁא בֵיהּ יְהֵי מְסָאַב מְסָאַב הוּא בִּלְחוֹדוֹהִי יִתֵּיב מִבַּרָא לְמַשְׁרִיתָא מוֹתְבֵיהּ׃}
{All the days wherein the plague is in him he shall be unclean; he is unclean; he shall dwell alone; without the camp shall his dwelling be.}{\arabic{verse}}
\rashi{\rashiDH{בדד ישב.} שלא יהיו שאר טמאים יושבים עמו. ואמרו רבותינו (ערכין טז׃), מה נשתנה משאר טמאים לישב בדד, הואיל והוא הבדיל בלשון הרע בין איש לאשתו ובין איש לרעהו, אף הוא יבדל׃\quad \rashiDH{מחוץ למחנה.} חוץ לשלש מחנות (פסחים סז.)׃}
\threeverse{\arabic{verse}}%Leviticus13:47
{וְהַבֶּ֕גֶד כִּֽי\maqqaf יִהְיֶ֥ה ב֖וֹ נֶ֣גַע צָרָ֑עַת בְּבֶ֣גֶד צֶ֔מֶר א֖וֹ בְּבֶ֥גֶד פִּשְׁתִּֽים׃}
{וּלְבוּשָׁא אֲרֵי יְהֵי בֵיהּ מַכְתָּשׁ סְגִירוּ בִּלְבוּשׁ עַמַּר אוֹ בִלְבוּשׁ כִּתָּן׃}
{And when the plague of leprosy is in a garment, whether it be a woolen garment, or a linen garment;}{\arabic{verse}}
\threeverse{\arabic{verse}}%Leviticus13:48
{א֤וֹ בִֽשְׁתִי֙ א֣וֹ בְעֵ֔רֶב לַפִּשְׁתִּ֖ים וְלַצָּ֑מֶר א֣וֹ בְע֔וֹר א֖וֹ בְּכׇל\maqqaf מְלֶ֥אכֶת עֽוֹר׃}
{אוֹ בְשִׁתְיָא אוֹ בְעִרְבָּא לְכִתָּנָא וּלְעַמְרָא אוֹ בְמַשְׁכָּא אוֹ בְכָל עֲבִידַת מְשַׁךְ׃}
{or in the warp, or in the woof, whether they be of linen, or of wool; or in a skin, or in any thing made of skin.}{\arabic{verse}}
\rashi{\rashiDH{לפשתים ולצמר.} של פשתים או של צמר׃\quad \rashiDH{או בעור.} זה עור שלא נעשה בו מלאכה׃\quad \rashiDH{או בכל מלאכת עור.} זה עור שנעשה בו מלאכה׃}
\threeverse{\arabic{verse}}%Leviticus13:49
{וְהָיָ֨ה הַנֶּ֜גַע יְרַקְרַ֣ק \legarmeh  א֣וֹ אֲדַמְדָּ֗ם בַּבֶּ֩גֶד֩ א֨וֹ בָע֜וֹר אֽוֹ\maqqaf בַשְּׁתִ֤י אוֹ\maqqaf בָעֵ֙רֶב֙ א֣וֹ בְכׇל\maqqaf כְּלִי\maqqaf ע֔וֹר נֶ֥גַע צָרַ֖עַת ה֑וּא וְהׇרְאָ֖ה אֶת\maqqaf הַכֹּהֵֽן׃}
{וִיהֵי מַכְתָּשָׁא יָרוֹק אוֹ סָמוֹק בִּלְבוּשָׁא אוֹ בְמַשְׁכָּא אוֹ בְשִׁתְיָא אוֹ בְעִרְבָּא אוֹ בְכָל מָן דִּמְשַׁךְ מַכְתָּשׁ סְגִירוּתָא הוּא וְיִתַּחְזֵי לְכָהֲנָא׃}
{If the plague be greenish or reddish in the garment, or in the skin, or in the warp, or in the woof, or in any thing of skin, it is the plague of leprosy, and shall be shown unto the priest.}{\arabic{verse}}
\rashi{\rashiDH{ירקרק.} ירוק שבירוקין׃\quad \rashiDH{אדמדם.} אדום שבאדומים׃ 
}
\threeverse{\arabic{verse}}%Leviticus13:50
{וְרָאָ֥ה הַכֹּהֵ֖ן אֶת\maqqaf הַנָּ֑גַע וְהִסְגִּ֥יר אֶת\maqqaf הַנֶּ֖גַע שִׁבְעַ֥ת יָמִֽים׃}
{וְיִחְזֵי כָהֲנָא יָת מַכְתָּשָׁא וְיַסְגַּר יָת מַכְתָּשָׁא שִׁבְעָא יוֹמִין׃}
{And the priest shall look upon the plague, and shut up that which hath the plague seven days.}{\arabic{verse}}
\threeverse{\arabic{verse}}%Leviticus13:51
{וְרָאָ֨ה אֶת\maqqaf הַנֶּ֜גַע בַּיּ֣וֹם הַשְּׁבִיעִ֗י כִּֽי\maqqaf פָשָׂ֤ה הַנֶּ֙גַע֙ בַּ֠בֶּ֠גֶד אֽוֹ\maqqaf בַשְּׁתִ֤י אֽוֹ\maqqaf בָעֵ֙רֶב֙ א֣וֹ בָע֔וֹר לְכֹ֛ל אֲשֶׁר\maqqaf יֵעָשֶׂ֥ה הָע֖וֹר לִמְלָאכָ֑ה צָרַ֧עַת מַמְאֶ֛רֶת הַנֶּ֖גַע טָמֵ֥א הֽוּא׃}
{וְיִחְזֵי יָת מַכְתָּשָׁא בְּיוֹמָא שְׁבִיעָאָה אֲרֵי אוֹסֵיף מַכְתָּשָׁא בִּלְבוּשָׁא אוֹ בְשִׁתְיָא אוֹ בְעִרְבָּא אוֹ בְמַשְׁכָּא לְכֹל דְּיִתְעֲבֵיד מַשְׁכָּא לַעֲבִידְתָא סְגִירוּת מְחַסְּרָא מַכְתָּשָׁא מְסָאַב הוּא׃}
{And he shall look on the plague on the seventh day: if the plague be spread in the garment, or in the warp, or in the woof, or in the skin, whatever service skin is used for, the plague is a malignant leprosy: it is unclean.}{\arabic{verse}}
\rashi{\rashiDH{צרעת ממארת.} לשון סִלּוֹן מַמְאִיר (יחזקאל כח, כד), פויי״נטש בלע״ז. ומדרשו תן בו מארה שלא תהנה הימנו׃}
\threeverse{\arabic{verse}}%Leviticus13:52
{וְשָׂרַ֨ף אֶת\maqqaf הַבֶּ֜גֶד א֥וֹ אֶֽת\maqqaf הַשְּׁתִ֣י \legarmeh  א֣וֹ אֶת\maqqaf הָעֵ֗רֶב בַּצֶּ֙מֶר֙ א֣וֹ בַפִּשְׁתִּ֔ים א֚וֹ אֶת\maqqaf כׇּל\maqqaf כְּלִ֣י הָע֔וֹר אֲשֶׁר\maqqaf יִהְיֶ֥ה ב֖וֹ הַנָּ֑גַע כִּֽי\maqqaf צָרַ֤עַת מַמְאֶ֙רֶת֙ הִ֔וא בָּאֵ֖שׁ תִּשָּׂרֵֽף׃}
{וְיוֹקֵיד יָת לְבוּשָׁא אוֹ יָת שִׁתְיָא אוֹ יָת עִרְבָּא בְּעַמְרָא אוֹ בְכִתָּנָא אוֹ יָת כָּל מָאן דִּמְשַׁךְ דִּיהֵי בֵיהּ מַכְתָּשָׁא אֲרֵי סְגִירוּת מְחַסְּרָא הִיא בְּנוּרָא תִּתּוֹקַד׃}
{And he shall burn the garment, or the warp, or the woof, whether it be of wool or of linen, or any thing of skin, wherein the plague is; for it is a malignant leprosy; it shall be burnt in the fire.}{\arabic{verse}}
\rashi{\rashiDH{בצמר או בפשתים.} של צמר או של פשתים, זהו פשוטו. ומדרשו יכול יביא גיזי צמר ואניצי פשתן וישרפם עמו, תלמוד לומר היא באש תשרף, אינה צריכה דבר אחר עמה, א״כ מה תלמוד לומר בצמר או בפשתים, להוציא את הָאִימְרִיוֹת שבו שהן ממין אחר אימריות לשון שפה, כמו אימרא׃}
\threeverse{\arabic{verse}}%Leviticus13:53
{וְאִם֮ יִרְאֶ֣ה הַכֹּהֵן֒ וְהִנֵּה֙ לֹא\maqqaf פָשָׂ֣ה הַנֶּ֔גַע בַּבֶּ֕גֶד א֥וֹ בַשְּׁתִ֖י א֣וֹ בָעֵ֑רֶב א֖וֹ בְּכׇל\maqqaf כְּלִי\maqqaf עֽוֹר׃}
{וְאִם יִחְזֵי כָהֲנָא וְהָא לָא אוֹסֵיף מַכְתָּשָׁא בִּלְבוּשָׁא אוֹ בְשִׁתְיָא אוֹ בְעִרְבָּא אוֹ בְכָל מָן דִּמְשַׁךְ׃}
{And if the priest shall look, and, behold, the plague be not spread in the garment, or in the warp, or in the woof, or in any thing of skin;}{\arabic{verse}}
\threeverse{\arabic{verse}}%Leviticus13:54
{וְצִוָּה֙ הַכֹּהֵ֔ן וְכִ֨בְּס֔וּ אֵ֥ת אֲשֶׁר\maqqaf בּ֖וֹ הַנָּ֑גַע וְהִסְגִּיר֥וֹ שִׁבְעַת\maqqaf יָמִ֖ים שֵׁנִֽית׃}
{וִיפַקֵּיד כָּהֲנָא וִיחַוְּרוּן יָת דְּבֵיהּ מַכְתָּשָׁא וְיַסְגְּרִנֵּיהּ שִׁבְעָא יוֹמִין תִּנְיָנוּת׃}
{then the priest shall command that they wash the thing wherein the plague is, and he shall shut it up seven days more.}{\arabic{verse}}
\rashi{\rashiDH{את אשר בו הנגע.} יכול מקום הנגע בלבד, תלמוד לומר את אשר בו הנגע, יכול כל הבגד כולו טעון כבוס, תלמוד לומר הנגע, הא כיצד, יכבס מן הבגד עמו׃}
\aliyacounter{שביעי}
\threeverse{\aliya{שביעי\newline (רביעי)}}%Leviticus13:55
{וְרָאָ֨ה הַכֹּהֵ֜ן אַחֲרֵ֣י \legarmeh  הֻכַּבֵּ֣ס אֶת\maqqaf הַנֶּ֗גַע וְ֠הִנֵּ֠ה לֹֽא\maqqaf הָפַ֨ךְ הַנֶּ֤גַע אֶת\maqqaf עֵינוֹ֙ וְהַנֶּ֣גַע לֹֽא\maqqaf פָשָׂ֔ה טָמֵ֣א ה֔וּא בָּאֵ֖שׁ תִּשְׂרְפֶ֑נּוּ פְּחֶ֣תֶת הִ֔וא בְּקָרַחְתּ֖וֹ א֥וֹ בְגַבַּחְתּֽוֹ׃}
{וְיִחְזֵי כָהֲנָא בָּתַר דְּחַוַּרוּ יָת מַכְתָּשָׁא וְהָא לָא שְׁנָא מַכְתָּשָׁא מִן כִּד הֲוָה וּמַכְתָּשָׁא לָא אוֹסֵיף מְסָאַב הוּא בְּנוּרָא תֵּיקְדִנֵּיהּ תָּבְרָא הִיא בִּשְׁחִיקוּתֵיהּ אוֹ בְחַדָּתוּתֵיהּ׃}
{And the priest shall look, after that the plague is washed; and, behold, if the plague have not changed its colour, and the plague be not spread, it is unclean; thou shalt burn it in the fire; it is a fret, whether the bareness be within or without.}{\arabic{verse}}
\rashi{\rashiDH{אחרי הכבס.} לשון הֵעָשׂוֹת׃\quad \rashiDH{לא הפך הנגע את עינו.} לא כהה ממראיתו׃ 
\quad \rashiDH{והנגע לא פשה.} שמענו שאם לא הפך ולא פשה טמא, ואין צריך לומר לא הפך ופשה, הפך ולא פשה איני יודע מה יעשה לו, תלמוד לומר והסגיר את הנגע מכל מקום, דברי רבי יהודה, וחכמים אומרים וכו׳ כדאיתא בתורת כהנים (פרק טו, ז), ורמזתיה כאן לישב המקרא על אופניו׃\quad \rashiDH{פחתת היא.} לשון גומא, כמו בְּאַחַת הַפְּחָתִים (שמואל־יז, ט), כלומר שפלה היא, נגע שמראיו שוקעין׃\quad \rashiDH{בקרחתו או בגבחתו.} כתרגומו בִּשְׁחִיקוּתֵהּ או בְּחַדְתוּתֵהּ׃\quad \rashiDH{קרחתו.} שחקים, ישנים. ומפני המדרש שהוצרך לגזרה שוה, מנין לפריחה בבגדים שהיא טהורה, נאמרה קרחת וגבחת באדם, ונאמרה קרחת וגבחת בבגדים מה להלן פרח בכולו טהור (סנהדרין פח.), אף כאן פרח בכולו טהור. לכך אחז הכתוב לשון קרחת וגבחת. ולענין פירושו ותרגומו זהו משמעו, קרחת לשון ישנים, וגבחת לשון חדשים, כאלו נכתב באחריתו או בקדמותו, שהקרחת לשון אחוריים, והגבחת לשון פנים, כמו שכתוב ואם מפאת פניו וגו׳, והקרחת כל ששופע ויורד מן הקדקד ולאחריו, כך מפורש בתורת כהנים (פרק טו, ט)׃}
\threeverse{\arabic{verse}}%Leviticus13:56
{וְאִם֮ רָאָ֣ה הַכֹּהֵן֒ וְהִנֵּה֙ כֵּהָ֣ה הַנֶּ֔גַע אַחֲרֵ֖י הֻכַּבֵּ֣ס אֹת֑וֹ וְקָרַ֣ע אֹת֗וֹ מִן\maqqaf הַבֶּ֙גֶד֙ א֣וֹ מִן\maqqaf הָע֔וֹר א֥וֹ מִן\maqqaf הַשְּׁתִ֖י א֥וֹ מִן\maqqaf הָעֵֽרֶב׃}
{וְאִם חֲזָא כָהֲנָא וְהָא עֲמָא מַכְתָּשָׁא בָּתַר דְּחַוַּרוּ יָתֵיהּ וִיבַזַּע יָתֵיהּ מִן לְבוּשָׁא אוֹ מִן מַשְׁכָּא אוֹ מִן שִׁתְיָא אוֹ מִן עִרְבָּא׃}
{And if the priest look, and, behold, the plague be dim after the washing thereof, then he shall rend it out of the garment, or out of the skin, or out of the warp, or out of the woof.}{\arabic{verse}}
\rashi{\rashiDH{וקרע אותו.} יקרע מקום הנגע מן הבגד, וישרפנו׃}
\threeverse{\aliya{מפטיר}}%Leviticus13:57
{וְאִם\maqqaf תֵּרָאֶ֨ה ע֜וֹד בַּ֠בֶּ֠גֶד אֽוֹ\maqqaf בַשְּׁתִ֤י אֽוֹ\maqqaf בָעֵ֙רֶב֙ א֣וֹ בְכׇל\maqqaf כְּלִי\maqqaf ע֔וֹר פֹּרַ֖חַת הִ֑וא בָּאֵ֣שׁ תִּשְׂרְפֶ֔נּוּ אֵ֥ת אֲשֶׁר\maqqaf בּ֖וֹ הַנָּֽגַע׃}
{וְאִם תִּתַּחְזֵי עוֹד בִּלְבוּשָׁא אוֹ בְשִׁתְיָא אוֹ בְעִרְבָּא אוֹ בְכָל מָן דִּמְשַׁךְ סָגְיָא הִיא בְּנוּרָא תֵּיקְדִנֵּיהּ יָת דְּבֵיהּ מַכְתָּשָׁא׃}
{And if it appear still in the garment, or in the warp, or in the woof, or in any thing of skin, it is breaking out, thou shalt burn that wherein the plague is with fire.}{\arabic{verse}}
\rashi{\rashiDH{פרחת הוא.} דבר החוזר וצומח׃\quad \rashiDH{באש תשרפנו.} את כל הבגד׃}
\threeverse{\arabic{verse}}%Leviticus13:58
{וְהַבֶּ֡גֶד אֽוֹ\maqqaf הַשְּׁתִ֨י אוֹ\maqqaf הָעֵ֜רֶב אֽוֹ\maqqaf כׇל\maqqaf כְּלִ֤י הָעוֹר֙ אֲשֶׁ֣ר תְּכַבֵּ֔ס וְסָ֥ר מֵהֶ֖ם הַנָּ֑גַע וְכֻבַּ֥ס שֵׁנִ֖ית וְטָהֵֽר׃}
{וּלְבוּשָׁא אוֹ שִׁתְיָא אוֹ עִרְבָּא אוֹ כָל מָאן דִּמְשַׁךְ דִּתְחַוַּר וְיִעְדֵּי מִנְּהוֹן מַכְתָּשָׁא וְיִצְטַבַּע תִּנְיָנוּת וְיִדְכֵּי׃}
{And the garment, or the warp, or the woof, or whatsoever thing of skin it be, which thou shalt wash, if the plague be departed from them, then it shall be washed the second time, and shall be clean.}{\arabic{verse}}
\rashi{\rashiDH{וסר מהם הנגע.} אם כשכבסוהו בתחלה על פי כהן סר ממנו הנגע לגמרי׃\quad \rashiDH{וכבס שנית.} לשון טבילה. תרגום של כבוסין שבפרשה זו לשון לבון, וְיִתְחַוַּר, חוץ מזה שאינו ללבון אלא לטבול, לכך תרגומו וְיִצְטַבַּע, וכן כל כבוסי בגדים שהן לטבילה מתורגמין ויצטבע׃}
\threeverse{\aliya{\Hebrewnumeral{67}}}%Leviticus13:59
{זֹ֠את תּוֹרַ֨ת נֶֽגַע\maqqaf צָרַ֜עַת בֶּ֥גֶד הַצֶּ֣מֶר \legarmeh  א֣וֹ הַפִּשְׁתִּ֗ים א֤וֹ הַשְּׁתִי֙ א֣וֹ הָעֵ֔רֶב א֖וֹ כׇּל\maqqaf כְּלִי\maqqaf ע֑וֹר לְטַהֲר֖וֹ א֥וֹ לְטַמְּאֽוֹ׃ \petucha }
{דָּא אוֹרָיְתָא דְּמַכְתָּשׁ סְגִירוּ לְבוּשׁ עַמַּר אוֹ כִתָּנָא אוֹ שִׁתְיָא אוֹ עִרְבָּא אוֹ כָל מָאן דִּמְשַׁךְ לְדַכָּאוּתֵיהּ אוֹ לְסַאָבוּתֵיהּ׃}
{This is the law of the plague of leprosy in a garment of wool or linen, or in the warp, or in the woof, or in any thing of skin, to pronounce it clean, or to pronounce it unclean.}{\arabic{verse}}

\engnote{The Haftarah is II Kings 4:42\verserangechar 5:19 on page \pageref{haft_27}. For Shabbat Ha\d{H}odesh the Maftir and Haftara are on page \pageref{maft_hachodesh}.}
\newperek
\aliyacounter{ראשון}
\newparsha{מצרע}
\newseder{9}
\threeverse{\aliya{מצרע}\newline\vspace{-4pt}\newline\seder{ט}}%Leviticus14:1
{וַיְדַבֵּ֥ר יְהֹוָ֖ה אֶל\maqqaf מֹשֶׁ֥ה לֵּאמֹֽר׃}
{וּמַלֵּיל יְיָ עִם מֹשֶׁה לְמֵימַר׃}
{And the \lord\space spoke unto Moses, saying:}{\Roman{chap}}
\threeverse{\arabic{verse}}%Leviticus14:2
{זֹ֤את תִּֽהְיֶה֙ תּוֹרַ֣ת הַמְּצֹרָ֔ע בְּי֖וֹם טׇהֳרָת֑וֹ וְהוּבָ֖א אֶל\maqqaf הַכֹּהֵֽן׃}
{דָּא תְהֵי אוֹרָיְתָא דִּסְגִירָא בְּיוֹמָא דִּדְכוּתֵיהּ וְיִתֵּיתֵי לְוָת כָּהֲנָא׃}
{This shall be the law of the leper in the day of his cleansing: he shall be brought unto the priest.}{\arabic{verse}}
\rashi{\rashiDH{זאת תהיה תורת וגו׳.} מלמד שאין מטהרין אותו בלילה (מגילה כא.)׃}
\threeverse{\arabic{verse}}%Leviticus14:3
{וְיָצָא֙ הַכֹּהֵ֔ן אֶל\maqqaf מִח֖וּץ לַֽמַּחֲנֶ֑ה וְרָאָה֙ הַכֹּהֵ֔ן וְהִנֵּ֛ה נִרְפָּ֥א נֶֽגַע\maqqaf הַצָּרַ֖עַת מִן\maqqaf הַצָּרֽוּעַ׃}
{וְיִפּוֹק כָּהֲנָא לְמִבַּרָא לְמַשְׁרִיתָא וְיִחְזֵי כָהֲנָא וְהָא אִתַּסִּי מַכְתָּשׁ סְגִירוּתָא מִן סְגִירָא׃}
{And the priest shall go forth out of the camp; and the priest shall look, and, behold, if the plague of leprosy be healed in the leper;}{\arabic{verse}}
\rashi{\rashiDH{אל מחוץ למחנה.} חוץ לשלש מחנות שנשתלח שם בימי חלוטו׃}
\threeverse{\arabic{verse}}%Leviticus14:4
{וְצִוָּה֙ הַכֹּהֵ֔ן וְלָקַ֧ח לַמִּטַּהֵ֛ר שְׁתֵּֽי\maqqaf צִפֳּרִ֥ים חַיּ֖וֹת טְהֹר֑וֹת וְעֵ֣ץ אֶ֔רֶז וּשְׁנִ֥י תוֹלַ֖עַת וְאֵזֹֽב׃}
{וִיפַקֵּיד כָּהֲנָא וְיִסַּב לִדְמִדַּכֵּי תַּרְתֵּין צִפְּרִין חַיִּין דָּכְיָין וְאָעָא דְּאַרְזָא וּצְבַע זְהוֹרִי וְאֵיזוֹבָא׃}
{then shall the priest command to take for him that is to be cleansed two living clean birds, and cedar-wood, and scarlet, and hyssop.}{\arabic{verse}}
\rashi{\rashiDH{חיות.} פרט לטרפות (חולין קמ.)׃\quad \rashiDH{טהרות.} פרט לעוף טמא (שם). לפי שהנגעים באים על לשון הרע, שהוא מעשה פִּטְפּוּטֵי דברים, לפיכך הוזקקו לטהרתו צפרים, שמפטפטין תמיד בצפצוף קול (ערכין טז׃)׃\quad \rashiDH{ועץ ארז.} לפי שהנגעים באין על גסות הרוח (שם)׃\quad \rashiDH{ושני תולעת ואזוב.} מה תקנתו ויתרפא, ישפיל עצמו מגאותו כתולעת וכאזוב׃\quad \rashiDH{עץ ארז.} מקל של ארז׃\quad \rashiDH{ושני תולעת.} לשון של צמר צבוע זהורית (ב״מ כא.)׃}
\threeverse{\arabic{verse}}%Leviticus14:5
{וְצִוָּה֙ הַכֹּהֵ֔ן וְשָׁחַ֖ט אֶת\maqqaf הַצִּפּ֣וֹר הָאֶחָ֑ת אֶל\maqqaf כְּלִי\maqqaf חֶ֖רֶשׂ עַל\maqqaf מַ֥יִם חַיִּֽים׃}
{וִיפַקֵּיד כָּהֲנָא וְיִכּוֹס יָת צִפְּרָא חֲדָא לְמָאן דַּחֲסַף עַל מֵי מַבּוּעַ׃}
{And the priest shall command to kill one of the birds in an earthen vessel over running water.}{\arabic{verse}}
\rashi{\rashiDH{על מים חיים.} נותן אותם תחלה בכלי, כדי שיהא דם צפור ניכר בהם, וכמה הם, רביעית (סוטה טז׃)׃}
\threeverse{\aliya{לוי}}%Leviticus14:6
{אֶת\maqqaf הַצִּפֹּ֤ר הַֽחַיָּה֙ יִקַּ֣ח אֹתָ֔הּ וְאֶת\maqqaf עֵ֥ץ הָאֶ֛רֶז וְאֶת\maqqaf שְׁנִ֥י הַתּוֹלַ֖עַת וְאֶת\maqqaf הָאֵזֹ֑ב וְטָבַ֨ל אוֹתָ֜ם וְאֵ֣ת \legarmeh  הַצִּפֹּ֣ר הַֽחַיָּ֗ה בְּדַם֙ הַצִּפֹּ֣ר הַשְּׁחֻטָ֔ה עַ֖ל הַמַּ֥יִם הַֽחַיִּֽים׃}
{יָת צִפְּרָא חַיְתָא יִסַּב יָתַהּ וְיָת אָעָא דְּאַרְזָא וְיָת צְבַע זְהוֹרִי וְיָת אֵיזוֹבָא וְיִטְבּוֹל יָתְהוֹן וְיָת צִפְּרָא חַיְתָא בִּדְמָא דְּצִפְּרָא דִּנְכִיסְתָא עַל מֵי מַבּוּעַ׃}
{As for the living bird, he shall take it, and the cedar-wood, and the scarlet, and the hyssop, and shall dip them and the living bird in the blood of the bird that was killed over the running water.}{\arabic{verse}}
\rashi{\rashiDH{את הצפור החיה יקח אותה.} מלמד שאינו אוגדה עמהם אלא מפרישה לעצמה, אבל העץ והאזוב כרוכים יחד בלשון הזהורית, כענין שנאמר ואת עץ הארז ואת שני התולעת ואת האזוב, קיחה אחת לשלשתן, יכול כשם שאינה בכלל אגודה כך לא תהא בכלל טבילה, תלמוד לומר וטבל אותם ואת הצפור החיה, החזיר את הצפור לכלל טבילה׃}
\threeverse{\arabic{verse}}%Leviticus14:7
{וְהִזָּ֗ה עַ֧ל הַמִּטַּהֵ֛ר מִן\maqqaf הַצָּרַ֖עַת שֶׁ֣בַע פְּעָמִ֑ים וְטִ֣הֲר֔וֹ וְשִׁלַּ֛ח אֶת\maqqaf הַצִּפֹּ֥ר הַֽחַיָּ֖ה עַל\maqqaf פְּנֵ֥י הַשָּׂדֶֽה׃}
{וְיַדֵּי עַל דְּמִדַּכֵּי מִן סְגִירוּתָא שְׁבַע זִמְנִין וִידַכֵּינֵיהּ וִישַׁלַּח יָת צִפְּרָא חַיְתָא עַל אַפֵּי חַקְלָא׃}
{And he shall sprinkle upon him that is to be cleansed from the leprosy seven times, and shall pronounce him clean, and shall let go the living bird into the open field.}{\arabic{verse}}
\threeverse{\arabic{verse}}%Leviticus14:8
{וְכִבֶּס֩ הַמִּטַּהֵ֨ר אֶת\maqqaf בְּגָדָ֜יו וְגִלַּ֣ח אֶת\maqqaf כׇּל\maqqaf שְׂעָר֗וֹ וְרָחַ֤ץ בַּמַּ֙יִם֙ וְטָהֵ֔ר וְאַחַ֖ר יָב֣וֹא אֶל\maqqaf הַֽמַּחֲנֶ֑ה וְיָשַׁ֛ב מִח֥וּץ לְאׇהֳל֖וֹ שִׁבְעַ֥ת יָמִֽים׃}
{וִיצַבַּע דְּמִדַּכֵּי יָת לְבוּשׁוֹהִי וִיגַלַּח יָת כָּל שַׂעֲרֵיהּ וְיִסְחֵי בְמַיָּא וְיִדְכֵּי וּבָתַר כֵּן יֵיעוֹל לְמַשְׁרִיתָא וְיִתֵּיב מִבַּרָא לְמַשְׁכְּנֵיהּ שִׁבְעָא יוֹמִין׃}
{And he that is to be cleansed shall wash his clothes, and shave off all his hair, and bathe himself in water, and he shall be clean; and after that he may come into the camp, but shall dwell outside his tent seven days.}{\arabic{verse}}
\rashi{\rashiDH{וישב מחוץ לאהלו.} מלמד שאסור בתשמיש המטה (ת״כ פרק א, יא.  חולין קמא.)׃}
\threeverse{\arabic{verse}}%Leviticus14:9
{וְהָיָה֩ בַיּ֨וֹם הַשְּׁבִיעִ֜י יְגַלַּ֣ח אֶת\maqqaf כׇּל\maqqaf שְׂעָר֗וֹ אֶת\maqqaf רֹאשׁ֤וֹ וְאֶת\maqqaf זְקָנוֹ֙ וְאֵת֙ גַּבֹּ֣ת עֵינָ֔יו וְאֶת\maqqaf כׇּל\maqqaf שְׂעָר֖וֹ יְגַלֵּ֑חַ וְכִבֶּ֣ס אֶת\maqqaf בְּגָדָ֗יו וְרָחַ֧ץ אֶת\maqqaf בְּשָׂר֛וֹ בַּמַּ֖יִם וְטָהֵֽר׃}
{וִיהֵי בְיוֹמָא שְׁבִיעָאָה יְגַלַּח יָת כָּל שַׂעֲרֵיהּ יָת רֵישֵׁיהּ וְיָת דִּקְנֵיהּ וְיָת גְּבִינֵי עֵינוֹהִי וְיָת כָּל שַׂעֲרֵיהּ יְגַלַּח וִיצַבַּע יָת לְבוּשׁוֹהִי וְיַסְחֵי יָת בִּשְׂרֵיהּ בְּמַיָּא וְיִדְכֵּי׃}
{And it shall be on the seventh day, that he shall shave all his hair off his head and his beard and his eyebrows, even all his hair he shall shave off; and he shall wash his clothes, and he shall bathe his flesh in water, and he shall be clean.}{\arabic{verse}}
\rashi{\rashiDH{את כל שערו וגו׳.} כלל ופרט וכלל, להביא כל מקום כנוס שער ונראה (סוטה טז.)׃}
\threeverse{\aliya{ישראל}}%Leviticus14:10
{וּבַיּ֣וֹם הַשְּׁמִינִ֗י יִקַּ֤ח שְׁנֵֽי\maqqaf כְבָשִׂים֙ תְּמִימִ֔ם וְכַבְשָׂ֥ה אַחַ֛ת בַּת\maqqaf שְׁנָתָ֖הּ תְּמִימָ֑ה וּשְׁלֹשָׁ֣ה עֶשְׂרֹנִ֗ים סֹ֤לֶת מִנְחָה֙ בְּלוּלָ֣ה בַשֶּׁ֔מֶן וְלֹ֥ג אֶחָ֖ד שָֽׁמֶן׃}
{וּבְיוֹמָא תְּמִינָאָה יִסַּב תְּרֵין אִמְּרִין שַׁלְמִין וְאִמַּרְתָּא חֲדָא בַּת שַׁתַּהּ שַׁלְמְתָא וּתְלָתָא עֶשְׂרוֹנִין סוּלְתָּא מִנְחָתָא דְּפִילָא בִמְשַׁח וְלוֹגָא חַד דְּמִשְׁחָא׃}
{And on the eighth day he shall take two he-lambs without blemish, and one ewe-lamb of the first year without blemish, and three tenth parts of an ephah of fine flour for a meal-offering, mingled with oil, and one log of oil.}{\arabic{verse}}
\rashi{\rashiDH{וכבשה אחת.} לחטאת׃\quad \rashiDH{ושלשה עשרונים.} לנסכי שלשה כבשים הללו, שחטאתו ואשמו של מצורע טעונין נסכים (מנחות צא.)׃\quad \rashiDH{ולוג אחד שמן.} להזות עליו שבע, וליתן ממנו ל תנוך אזנו ומתן בהונות׃}
\threeverse{\arabic{verse}}%Leviticus14:11
{וְהֶעֱמִ֞יד הַכֹּהֵ֣ן הַֽמְטַהֵ֗ר אֵ֛ת הָאִ֥ישׁ הַמִּטַּהֵ֖ר וְאֹתָ֑ם לִפְנֵ֣י יְהֹוָ֔ה פֶּ֖תַח אֹ֥הֶל מוֹעֵֽד׃}
{וִיקִים כָּהֲנָא דִּמְדַכֵּי יָת גּוּבְרָא דְּמִדַּכֵּי וְיָתְהוֹן קֳדָם יְיָ בִּתְרַע מַשְׁכַּן זִמְנָא׃}
{And the priest that cleanseth him shall set the man that is to be cleansed, and those things, before the \lord, at the door of the tent of meeting.}{\arabic{verse}}
\rashi{\rashiDH{לפני ה׳.} בשער נקנור, ולא בעזרה עצמה, לפי שהוא מחוסר כפורים (סוטה ז.)׃}
\threeverse{\arabic{verse}}%Leviticus14:12
{וְלָקַ֨ח הַכֹּהֵ֜ן אֶת\maqqaf הַכֶּ֣בֶשׂ הָאֶחָ֗ד וְהִקְרִ֥יב אֹת֛וֹ לְאָשָׁ֖ם וְאֶת\maqqaf לֹ֣ג הַשָּׁ֑מֶן וְהֵנִ֥יף אֹתָ֛ם תְּנוּפָ֖ה לִפְנֵ֥י יְהֹוָֽה׃}
{וְיִסַּב כָּהֲנָא יָת אִמְּרָא חַדָא וִיקָרֵיב יָתֵיהּ לַאֲשָׁמָא וְיָת לוֹגָא דְּמִשְׁחָא וִירִים יָתְהוֹן אֲרָמָא קֳדָם יְיָ׃}
{And the priest shall take one of the he-lambs, and offer him for a guilt-offering, and the log of oil, and wave them for a wave-offering before the \lord.}{\arabic{verse}}
\rashi{\rashiDH{והקריב אותו לאשם.} יקריבנו לתוך העזרה לשם אשם להניף, שהוא טעון תנופה חי׃\quad \rashiDH{והניף אותם.} את האשם ואת הלוג׃}
\aliyacounter{שני}
\threeverse{\aliya{שני}}%Leviticus14:13
{וְשָׁחַ֣ט אֶת\maqqaf הַכֶּ֗בֶשׂ בִּ֠מְק֠וֹם אֲשֶׁ֨ר יִשְׁחַ֧ט אֶת\maqqaf הַֽחַטָּ֛את וְאֶת\maqqaf הָעֹלָ֖ה בִּמְק֣וֹם הַקֹּ֑דֶשׁ כִּ֡י כַּ֠חַטָּ֠את הָאָשָׁ֥ם הוּא֙ לַכֹּהֵ֔ן קֹ֥דֶשׁ קׇֽדָשִׁ֖ים הֽוּא׃}
{וְיִכּוֹס יָת אִמְּרָא בְּאַתְרָא דְּיִכּוֹס יָת חַטָּתָא וְיָת עֲלָתָא בַּאֲתַר קוּדְשָׁא אֲרֵי כְּחַטָּתָא אֲשָׁמָא הוּא לְכָהֲנָא קוֹדֶשׁ קוּדְשִׁין הוּא׃}
{And he shall kill the he-lamb in the place where they kill the sin-offering and the burnt-offering, in the place of the sanctuary; for as the sin-offering is the priest’s, so is the guilt-offering; it is most holy.}{\arabic{verse}}
\rashi{\rashiDH{במקום אשר ישחט וגו׳.} על ירך המזבח בצפון, ומה תלמוד לומר, והלא כבר נאמר בתורת אשם בפרשת צו את אהרן שהאשם טעון שחיטה בצפון, לפי שיצא זה מכלל אשמות לידון בהעמדה, יכול תהא שחיטתו במקום העמדתו, לכך נאמר ושחט במקום אשר ישחט וגו׳ (זבחים מט.)׃\quad \rashiDH{כי כחטאת.} כי ככל החטאות׃ \rashiDH{האשם.} הזה׃ \rashiDH{הוא לכהן.} בכל עבודות התלוית בכהן השוה אשם זה לחטאת, שלא תאמר הואיל ויצא דמו מכלל שאר אשמות להנתן על תנוך ובהונות, לא יהא טעון מתן דמים ואימורים לגבי מזבח, לכך נאמר כי כחטאת האשם הוא לכהן, יכול יהא דמו ניתן למעלה כחטאת, תלמוד לומר וכו׳, בתורת כהנים (פרק ג, א)׃}
\threeverse{\arabic{verse}}%Leviticus14:14
{וְלָקַ֣ח הַכֹּהֵן֮ מִדַּ֣ם הָאָשָׁם֒ וְנָתַן֙ הַכֹּהֵ֔ן עַל\maqqaf תְּנ֛וּךְ אֹ֥זֶן הַמִּטַּהֵ֖ר הַיְמָנִ֑ית וְעַל\maqqaf בֹּ֤הֶן יָדוֹ֙ הַיְמָנִ֔ית וְעַל\maqqaf בֹּ֥הֶן רַגְל֖וֹ הַיְמָנִֽית׃}
{וְיִסַּב כָּהֲנָא מִדְּמָא דַּאֲשָׁמָא וְיִתֵּין כָּהֲנָא עַל רוּם אוּדְנָא דְּמִדַּכֵּי דְּיַמִּינָא וְעַל אִלְיוֹן יְדֵיהּ דְּיַמִּינָא וְעַל אִלְיוֹן רַגְלֵיהּ דְּיַמִּינָא׃}
{And the priest shall take of the blood of the guilt-offering, and the priest shall put it upon the tip of the right ear of him that is to be cleansed, and upon the thumb of his right hand, and upon the great toe of his right foot.}{\arabic{verse}}
\rashi{\rashiDH{תנוך.} גדר אמצעי שבאוזן, ולשון תנוך לא נודע לי, והפותרים קורים לו טנדרו״ס׃\quad \rashiDH{בהן.} גודל׃}
\threeverse{\arabic{verse}}%Leviticus14:15
{וְלָקַ֥ח הַכֹּהֵ֖ן מִלֹּ֣ג הַשָּׁ֑מֶן וְיָצַ֛ק עַל\maqqaf כַּ֥ף הַכֹּהֵ֖ן הַשְּׂמָאלִֽית׃}
{וְיִסַּב כָּהֲנָא מִלּוֹגָא דְּמִשְׁחָא וִירִיק עַל יְדָא דְּכָהֲנָא דִּשְׂמָאלָא׃}
{And the priest shall take of the log of oil, and pour it into the palm of his own left hand.}{\arabic{verse}}
\threeverse{\arabic{verse}}%Leviticus14:16
{וְטָבַ֤ל הַכֹּהֵן֙ אֶת\maqqaf אֶצְבָּע֣וֹ הַיְמָנִ֔ית מִן\maqqaf הַשֶּׁ֕מֶן אֲשֶׁ֥ר עַל\maqqaf כַּפּ֖וֹ הַשְּׂמָאלִ֑ית וְהִזָּ֨ה מִן\maqqaf הַשֶּׁ֧מֶן בְּאֶצְבָּע֛וֹ שֶׁ֥בַע פְּעָמִ֖ים לִפְנֵ֥י יְהֹוָֽה׃}
{וְיִטְבּוֹל כָּהֲנָא יָת אֶצְבְּעֵיהּ דְּיַמִּינָא מִן מִשְׁחָא דְּעַל יְדֵיהּ דִּשְׂמָאלָא וְיַדֵּי מִן מִשְׁחָא בְּאֶצְבְּעֵיהּ שְׁבַע זִמְנִין קֳדָם יְיָ׃}
{And the priest shall dip his right finger in the oil that is in his left hand, and shall sprinkle of the oil with his finger seven times before the \lord.}{\arabic{verse}}
\rashi{\rashiDH{לפני ה׳.} כנגד בית קדשי הקדשים (ת״כ שם ט)׃}
\threeverse{\arabic{verse}}%Leviticus14:17
{וּמִיֶּ֨תֶר הַשֶּׁ֜מֶן אֲשֶׁ֣ר עַל\maqqaf כַּפּ֗וֹ יִתֵּ֤ן הַכֹּהֵן֙ עַל\maqqaf תְּנ֞וּךְ אֹ֤זֶן הַמִּטַּהֵר֙ הַיְמָנִ֔ית וְעַל\maqqaf בֹּ֤הֶן יָדוֹ֙ הַיְמָנִ֔ית וְעַל\maqqaf בֹּ֥הֶן רַגְל֖וֹ הַיְמָנִ֑ית עַ֖ל דַּ֥ם הָאָשָֽׁם׃}
{וּמִשְּׁאָר מִשְׁחָא דְּעַל יְדֵיהּ יִתֵּין כָּהֲנָא עַל רוּם אוּדְנָא דְּמִדַּכֵּי דְּיַמִּינָא וְעַל אִלְיוֹן יְדֵיהּ דְּיַמִּינָא וְעַל אִלְיוֹן רַגְלֵיהּ דְּיַמִּינָא עַל דְּמָא דַּאֲשָׁמָא׃}
{And of the rest of the oil that is in his hand shall the priest put upon the tip of the right ear of him that is to be cleansed, and upon the thumb of his right hand, and upon the great toe of his right foot, upon the blood of the guilt-offering.}{\arabic{verse}}
\threeverse{\arabic{verse}}%Leviticus14:18
{וְהַנּוֹתָ֗ר בַּשֶּׁ֙מֶן֙ אֲשֶׁר֙ עַל\maqqaf כַּ֣ף הַכֹּהֵ֔ן יִתֵּ֖ן עַל\maqqaf רֹ֣אשׁ הַמִּטַּהֵ֑ר וְכִפֶּ֥ר עָלָ֛יו הַכֹּהֵ֖ן לִפְנֵ֥י יְהֹוָֽה׃}
{וּדְיִשְׁתְּאַר בְּמִשְׁחָא דְּעַל יְדָא דְּכָהֲנָא יִתֵּין עַל רֵישָׁא דְּמִדַּכֵּי וִיכַפַּר עֲלוֹהִי כָּהֲנָא קֳדָם יְיָ׃}
{And the rest of the oil that is in the priest’s hand he shall put upon the head of him that is to be cleansed; and the priest shall make atonement for him before the \lord.}{\arabic{verse}}
\threeverse{\arabic{verse}}%Leviticus14:19
{וְעָשָׂ֤ה הַכֹּהֵן֙ אֶת\maqqaf הַ֣חַטָּ֔את וְכִפֶּ֕ר עַל\maqqaf הַמִּטַּהֵ֖ר מִטֻּמְאָת֑וֹ וְאַחַ֖ר יִשְׁחַ֥ט אֶת\maqqaf הָעֹלָֽה׃}
{וְיַעֲבֵיד כָּהֲנָא יָת חַטָּתָא וִיכַפַּר עַל דְּמִדַּכֵּי מִסְּאוֹבְתֵיהּ וּבָתַר כֵּן יִכּוֹס יָת עֲלָתָא׃}
{And the priest shall offer the sin-offering, and make atonement for him that is to be cleansed because of his uncleanness; and afterward he shall kill the burnt-offering.}{\arabic{verse}}
\threeverse{\arabic{verse}}%Leviticus14:20
{וְהֶעֱלָ֧ה הַכֹּהֵ֛ן אֶת\maqqaf הָעֹלָ֥ה וְאֶת\maqqaf הַמִּנְחָ֖ה הַמִּזְבֵּ֑חָה וְכִפֶּ֥ר עָלָ֛יו הַכֹּהֵ֖ן וְטָהֵֽר׃ \setuma }
{וְיַסֵּיק כָּהֲנָא יָת עֲלָתָא וְיָת מִנְחָתָא לְמַדְבְּחָא וִיכַפַּר עֲלוֹהִי כָּהֲנָא וְיִדְכֵּי׃}
{And the priest shall offer the burnt-offering and the meal-offering upon the altar; and the priest shall make atonement for him, and he shall be clean.}{\arabic{verse}}
\rashi{\rashiDH{ואת המנחה.} מנחת נסכים של בהמה׃}
\aliyacounter{שלישי}
\threeverse{\aliya{שלישי\newline (חמישי)}}%Leviticus14:21
{וְאִם\maqqaf דַּ֣ל ה֗וּא וְאֵ֣ין יָדוֹ֮ מַשֶּׂ֒גֶת֒ וְ֠לָקַ֠ח כֶּ֣בֶשׂ אֶחָ֥ד אָשָׁ֛ם לִתְנוּפָ֖ה לְכַפֵּ֣ר עָלָ֑יו וְעִשָּׂר֨וֹן סֹ֜לֶת אֶחָ֨ד בָּל֥וּל בַּשֶּׁ֛מֶן לְמִנְחָ֖ה וְלֹ֥ג שָֽׁמֶן׃}
{וְאִם מִסְכֵּן הוּא וְלֵית יְדֵיהּ מַדְבְּקָא וְיִסַּב אִמַּר חַד אֲשָׁמָא לַאֲרָמָא לְכַפָּרָא עֲלוֹהִי וְעֶשְׂרוֹנָא סוּלְתָּא חַד דְּפִיל בִּמְשַׁח לְמִנְחָתָא וְלוֹגָא דְּמִשְׁחָא׃}
{And if he be poor, and his means suffice not, then he shall take one he-lamb for a guilt-offering to be waved, to make atonement for him, and one tenth part of an ephah of fine flour mingled with oil for a meal-offering, and a log of oil;}{\arabic{verse}}
\rashi{\rashiDH{ועשרון סלת אחד.} לכבש, זה שהוא אחד יביא עשרון אחד לנסכיו׃\quad \rashiDH{ולוג שמן.} לתת ממנו על הבהונות, ושמן של נסכי המנחה לא הוזקק הכתוב לפרש׃}
\threeverse{\arabic{verse}}%Leviticus14:22
{וּשְׁתֵּ֣י תֹרִ֗ים א֤וֹ שְׁנֵי֙ בְּנֵ֣י יוֹנָ֔ה אֲשֶׁ֥ר תַּשִּׂ֖יג יָד֑וֹ וְהָיָ֤ה אֶחָד֙ חַטָּ֔את וְהָאֶחָ֖ד עֹלָֽה׃}
{וּתְרֵין שַׁפְנִינִין אוֹ תְּרֵין בְּנֵי יוֹנָה דְּתַדְבֵּיק יְדֵיהּ וִיהֵי חַד חַטָּתָא וְחַד עֲלָתָא׃}
{and two turtle-doves, or two young pigeons, such as his means suffice for; and the one shall be a sin-offering, and the other a burnt-offering.}{\arabic{verse}}
\threeverse{\arabic{verse}}%Leviticus14:23
{וְהֵבִ֨יא אֹתָ֜ם בַּיּ֧וֹם הַשְּׁמִינִ֛י לְטׇהֳרָת֖וֹ אֶל\maqqaf הַכֹּהֵ֑ן אֶל\maqqaf פֶּ֥תַח אֹֽהֶל\maqqaf מוֹעֵ֖ד לִפְנֵ֥י יְהֹוָֽה׃}
{וְיַיְתֵי יָתְהוֹן בְּיוֹמָא תְּמִינָאָה לִדְכוּתֵיהּ לְוָת כָּהֲנָא לִתְרַע מַשְׁכַּן זִמְנָא לִקְדָם יְיָ׃}
{And on the eighth day he shall bring them for his cleansing unto the priest, unto the door of the tent of meeting, before the \lord.}{\arabic{verse}}
\rashi{\rashiDH{ביום השמיני לטהרתו.} שמיני לצפרים ולהזאת עץ ארז ואזוב ושני תולעת׃}
\threeverse{\arabic{verse}}%Leviticus14:24
{וְלָקַ֧ח הַכֹּהֵ֛ן אֶת\maqqaf כֶּ֥בֶשׂ הָאָשָׁ֖ם וְאֶת\maqqaf לֹ֣ג הַשָּׁ֑מֶן וְהֵנִ֨יף אֹתָ֧ם הַכֹּהֵ֛ן תְּנוּפָ֖ה לִפְנֵ֥י יְהֹוָֽה׃}
{וְיִסַּב כָּהֲנָא יָת אִמְּרָא דַּאֲשָׁמָא וְיָת לוֹגָא דְּמִשְׁחָא וִירִים יָתְהוֹן כָּהֲנָא אָרָמָא קֳדָם יְיָ׃}
{And the priest shall take the lamb of the guilt-offering, and the log of oil, and the priest shall wave them for a wave-offering before the \lord.}{\arabic{verse}}
\threeverse{\arabic{verse}}%Leviticus14:25
{וְשָׁחַט֮ אֶת\maqqaf כֶּ֣בֶשׂ הָֽאָשָׁם֒ וְלָקַ֤ח הַכֹּהֵן֙ מִדַּ֣ם הָֽאָשָׁ֔ם וְנָתַ֛ן עַל\maqqaf תְּנ֥וּךְ אֹֽזֶן\maqqaf הַמִּטַּהֵ֖ר הַיְמָנִ֑ית וְעַל\maqqaf בֹּ֤הֶן יָדוֹ֙ הַיְמָנִ֔ית וְעַל\maqqaf בֹּ֥הֶן רַגְל֖וֹ הַיְמָנִֽית׃}
{וְיִכּוֹס יָת אִמְּרָא דַּאֲשָׁמָא וְיִסַּב כָּהֲנָא מִדְּמָא דַּאֲשָׁמָא וְיִתֵּין עַל רוּם אוּדְנָא דְּמִדַּכֵּי דְּיַמִּינָא וְעַל אִלְיוֹן יְדֵיהּ דְּיַמִּינָא וְעַל אִלְיוֹן רַגְלֵיהּ דְּיַמִּינָא׃}
{And he shall kill the lamb of the guilt-offering, and the priest shall take of the blood of the guilt-offering, and put it upon the tip of the right ear of him that is to be cleansed, and upon the thumb of his right hand, and upon the great toe of his right foot.}{\arabic{verse}}
\threeverse{\arabic{verse}}%Leviticus14:26
{וּמִן\maqqaf הַשֶּׁ֖מֶן יִצֹ֣ק הַכֹּהֵ֑ן עַל\maqqaf כַּ֥ף הַכֹּהֵ֖ן הַשְּׂמָאלִֽית׃}
{וּמִן מִשְׁחָא יְרִיק כָּהֲנָא עַל יְדָא דְּכָהֲנָא דִּשְׂמָאלָא׃}
{And the priest shall pour of the oil into the palm of his own left hand.}{\arabic{verse}}
\threeverse{\arabic{verse}}%Leviticus14:27
{וְהִזָּ֤ה הַכֹּהֵן֙ בְּאֶצְבָּע֣וֹ הַיְמָנִ֔ית מִן\maqqaf הַשֶּׁ֕מֶן אֲשֶׁ֥ר עַל\maqqaf כַּפּ֖וֹ הַשְּׂמָאלִ֑ית שֶׁ֥בַע פְּעָמִ֖ים לִפְנֵ֥י יְהֹוָֽה׃}
{וְיַדֵּי כָּהֲנָא בְּאֶצְבְּעֵיהּ דְּיַמִּינָא מִן מִשְׁחָא דְּעַל יְדֵיהּ דִּשְׂמָאלָא שְׁבַע זִמְנִין קֳדָם יְיָ׃}
{And the priest shall sprinkle with his right finger some of the oil that is in his left hand seven times before the \lord.}{\arabic{verse}}
\threeverse{\arabic{verse}}%Leviticus14:28
{וְנָתַ֨ן הַכֹּהֵ֜ן מִן\maqqaf הַשֶּׁ֣מֶן \legarmeh  אֲשֶׁ֣ר עַל\maqqaf כַּפּ֗וֹ עַל\maqqaf תְּנ֞וּךְ אֹ֤זֶן הַמִּטַּהֵר֙ הַיְמָנִ֔ית וְעַל\maqqaf בֹּ֤הֶן יָדוֹ֙ הַיְמָנִ֔ית וְעַל\maqqaf בֹּ֥הֶן רַגְל֖וֹ הַיְמָנִ֑ית עַל\maqqaf מְק֖וֹם דַּ֥ם הָאָשָֽׁם׃}
{וְיִתֵּין כָּהֲנָא מִן מִשְׁחָא דְּעַל יְדֵיהּ עַל רוּם אוּדְנָא דְּמִדַּכֵּי דְּיַמִּינָא וְעַל אִלְיוֹן יְדֵיהּ דְּיַמִּינָא וְעַל אִלְיוֹן רַגְלֵיהּ דְּיַמִּינָא עַל אֲתַר דְּמָא דַּאֲשָׁמָא׃}
{And the priest shall put of the oil that is in his hand upon the tip of the right ear of him that is to be cleansed, and upon the thumb of his right hand, and upon the great toe of his right foot, upon the place of the blood of the guilt-offering.}{\arabic{verse}}
\rashi{\rashiDH{על מקום דם האשם.} אפי׳ נתקנח הדם, למד, שאין הדם גורם (מנחות י.), אלא המקום גורם׃}
\threeverse{\arabic{verse}}%Leviticus14:29
{וְהַנּוֹתָ֗ר מִן\maqqaf הַשֶּׁ֙מֶן֙ אֲשֶׁר֙ עַל\maqqaf כַּ֣ף הַכֹּהֵ֔ן יִתֵּ֖ן עַל\maqqaf רֹ֣אשׁ הַמִּטַּהֵ֑ר לְכַפֵּ֥ר עָלָ֖יו לִפְנֵ֥י יְהֹוָֽה׃}
{וּדְיִשְׁתְּאַר מִן מִשְׁחָא דְּעַל יְדָא דְּכָהֲנָא יִתֵּין עַל רֵישָׁא דְּמִדַּכֵּי לְכַפָּרָא עֲלוֹהִי קֳדָם יְיָ׃}
{And the rest of the oil that is in the priest’s hand he shall put upon the head of him that is to be cleansed, to make atonement for him before the \lord.}{\arabic{verse}}
\threeverse{\arabic{verse}}%Leviticus14:30
{וְעָשָׂ֤ה אֶת\maqqaf הָֽאֶחָד֙ מִן\maqqaf הַתֹּרִ֔ים א֖וֹ מִן\maqqaf בְּנֵ֣י הַיּוֹנָ֑ה מֵאֲשֶׁ֥ר תַּשִּׂ֖יג יָדֽוֹ׃}
{וְיַעֲבֵיד יָת חַד מִן שַׁפְנִינַיָּא אוֹ מִן בְּנֵי יוֹנָה מִדְּתַדְבֵּיק יְדֵיהּ׃}
{And he shall offer one of the turtle-doves, or of the young pigeons, such as his means suffice for;}{\arabic{verse}}
\threeverse{\arabic{verse}}%Leviticus14:31
{אֵ֣ת אֲשֶׁר\maqqaf תַּשִּׂ֞יג יָד֗וֹ אֶת\maqqaf הָאֶחָ֥ד חַטָּ֛את וְאֶת\maqqaf הָאֶחָ֥ד עֹלָ֖ה עַל\maqqaf הַמִּנְחָ֑ה וְכִפֶּ֧ר הַכֹּהֵ֛ן עַ֥ל הַמִּטַּהֵ֖ר לִפְנֵ֥י יְהֹוָֽה׃}
{יָת דְּתַדְבֵּיק יְדֵיהּ יָת חַד חַטָּתָא וְיָת חַד עֲלָתָא עַל מִנְחָתָא וִיכַפַּר כָּהֲנָא עַל דְּמִדַּכֵּי קֳדָם יְיָ׃}
{even such as his means suffice for, the one for a sin-offering, and the other for a burnt-offering, with the meal-offering; and the priest shall make atonement for him that is to be cleansed before the \lord.}{\arabic{verse}}
\threeverse{\arabic{verse}}%Leviticus14:32
{זֹ֣את תּוֹרַ֔ת אֲשֶׁר\maqqaf בּ֖וֹ נֶ֣גַע צָרָ֑עַת אֲשֶׁ֛ר לֹֽא\maqqaf תַשִּׂ֥יג יָד֖וֹ בְּטׇהֳרָתֽוֹ׃ \petucha }
{דָּא אוֹרָיְתָא דְּבֵיהּ מַכְתָּשׁ סְגִירוּ דְּלָא תַדְבֵּיק יְדֵיהּ בִּדְכוּתֵיהּ׃}
{This is the law of him in whom is the plague of leprosy, whose means suffice not for that which pertaineth to his cleansing.}{\arabic{verse}}
\aliyacounter{רביעי}
\newseder{10}
\threeverse{\aliya{רביעי\newline (ששי)}\newline\vspace{-4pt}\newline\seder{י}}%Leviticus14:33
{וַיְדַבֵּ֣ר יְהֹוָ֔ה אֶל\maqqaf מֹשֶׁ֥ה וְאֶֽל\maqqaf אַהֲרֹ֖ן לֵאמֹֽר׃}
{וּמַלֵּיל יְיָ עִם מֹשֶׁה וּלְאַהֲרֹן לְמֵימַר׃}
{And the \lord\space spoke unto Moses and unto Aaron, saying:}{\arabic{verse}}
\threeverse{\arabic{verse}}%Leviticus14:34
{כִּ֤י תָבֹ֙אוּ֙ אֶל\maqqaf אֶ֣רֶץ כְּנַ֔עַן אֲשֶׁ֥ר אֲנִ֛י נֹתֵ֥ן לָכֶ֖ם לַאֲחֻזָּ֑ה וְנָתַתִּי֙ נֶ֣גַע צָרַ֔עַת בְּבֵ֖ית אֶ֥רֶץ אֲחֻזַּתְכֶֽם׃}
{אֲרֵי תֵיעֲלוּן לְאַרְעָא דִּכְנַעַן דַּאֲנָא יָהֵיב לְכוֹן לְאַחְסָנָא וְאֶתֵּין מַכְתָּשׁ סְגִירוּ בְּבֵית אֲרַע אַחְסָנַתְכוֹן׃}
{When ye are come into the land of Canaan, which I give to you for a possession, and I put the plague of leprosy in a house of the land of your possession;}{\arabic{verse}}
\rashi{\rashiDH{ונתתי נגע צרעת.} בשורה היא להם שהנגעים באים עליהם, לפי שהטמינו אמוריים מטמוניות של זהב בקירות בתיהם כל ארבעים שנה שהיו ישראל במדבר, (ויק״ר יז, ו) וע״י הנגע נותץ הבית ומוצאן׃}
\threeverse{\arabic{verse}}%Leviticus14:35
{וּבָא֙ אֲשֶׁר\maqqaf ל֣וֹ הַבַּ֔יִת וְהִגִּ֥יד לַכֹּהֵ֖ן לֵאמֹ֑ר כְּנֶ֕גַע נִרְאָ֥ה לִ֖י בַּבָּֽיִת׃}
{וְיֵיתֵי דְּדִילֵיהּ בֵּיתָא וִיחַוֵּי לְכָהֲנָא לְמֵימַר כְּמַכְתָּשָׁא אִתַּחְזִי לִי בְּבֵיתָא׃}
{then he that owneth the house shall come and tell the priest, saying: ‘There seemeth to me to be as it were a plague in the house.’}{\arabic{verse}}
\rashi{\rashiDH{כנגע נראה לי בבית.} אפי׳ תלמיד חכם שיודע שהוא נגע ודאי, לא יפסוק דבר ברור לומר נגע נראה לי, אלא כנגע נראה לי (ת״כ פרשתא ה, י  נגעים פי״ב מ״ה)׃}
\threeverse{\arabic{verse}}%Leviticus14:36
{וְצִוָּ֨ה הַכֹּהֵ֜ן וּפִנּ֣וּ אֶת\maqqaf הַבַּ֗יִת בְּטֶ֨רֶם יָבֹ֤א הַכֹּהֵן֙ לִרְא֣וֹת אֶת\maqqaf הַנֶּ֔גַע וְלֹ֥א יִטְמָ֖א כׇּל\maqqaf אֲשֶׁ֣ר בַּבָּ֑יִת וְאַ֥חַר כֵּ֛ן יָבֹ֥א הַכֹּהֵ֖ן לִרְא֥וֹת אֶת\maqqaf הַבָּֽיִת׃}
{וִיפַקֵּיד כָּהֲנָא וִיפַנּוֹן יָת בֵּיתָא עַד לָא יֵיעוֹל כָּהֲנָא לְמִחְזֵי יָת מַכְתָּשָׁא וְלָא יִסְתָּאַב כָּל דִּבְבֵּיתָא וּבָתַר כֵּן יֵיעוֹל כָּהֲנָא לְמִחְזֵי יָת בֵּיתָא׃}
{And the priest shall command that they empty the house, before the priest go in to see the plague, that all that is in the house be not made unclean; and afterward the priest shall go in to see the house.}{\arabic{verse}}
\rashi{\rashiDH{בטרם יבא הכהן וגו׳.} שכל זמן שאין כהן נזקק לו, אין שם תורת טומאה׃\quad \rashiDH{ולא יטמא כל אשר בבית.} שאם לא יפנהו ויבא הכהן ויראה הנגע נזקק להסגר, וכל מה שבתוכו יטמא. ועל מה חסה תורה, אם על כלי שטף יטבילם ויטהרו, ואם על אוכלין ומשקין יאכלם בימי טומאתו, הא לא חסה התורה אלא על כלי חרס שאין להם טהרה במקוה (ת״כ שם יב)׃}
\threeverse{\arabic{verse}}%Leviticus14:37
{וְרָאָ֣ה אֶת\maqqaf הַנֶּ֗גַע וְהִנֵּ֤ה הַנֶּ֙גַע֙ בְּקִירֹ֣ת הַבַּ֔יִת שְׁקַֽעֲרוּרֹת֙ יְרַקְרַקֹּ֔ת א֖וֹ אֲדַמְדַּמֹּ֑ת וּמַרְאֵיהֶ֥ן שָׁפָ֖ל מִן\maqqaf הַקִּֽיר׃}
{וְיִחְזֵי יָת מַכְתָּשָׁא וְהָא מַכְתָּשָׁא בְּכוּתְלֵי בֵיתָא פַּחְתִּין יָרְקָן אוֹ סָמְקָן וּמִחְזֵיהוֹן מַכִּיךְ מִן כּוּתְלָא׃}
{And he shall look on the plague, and, behold, if the plague be in the walls of the house with hollow streaks, greenish or reddish, and the appearance thereof be lower than the wall;}{\arabic{verse}}
\rashi{\rashiDH{שקערורת.} שוקעות במראיהן (ת״כ פרשתא ו, ה)׃}
\threeverse{\arabic{verse}}%Leviticus14:38
{וְיָצָ֧א הַכֹּהֵ֛ן מִן\maqqaf הַבַּ֖יִת אֶל\maqqaf פֶּ֣תַח הַבָּ֑יִת וְהִסְגִּ֥יר אֶת\maqqaf הַבַּ֖יִת שִׁבְעַ֥ת יָמִֽים׃}
{וְיִפּוֹק כָּהֲנָא מִן בֵּיתָא לִתְרַע בֵּיתָא וְיַסְגַּר יָת בֵּיתָא שִׁבְעָא יוֹמִין׃}
{then the priest shall go out of the house to the door of the house, and shut up the house seven days.}{\arabic{verse}}
\threeverse{\arabic{verse}}%Leviticus14:39
{וְשָׁ֥ב הַכֹּהֵ֖ן בַּיּ֣וֹם הַשְּׁבִיעִ֑י וְרָאָ֕ה וְהִנֵּ֛ה פָּשָׂ֥ה הַנֶּ֖גַע בְּקִירֹ֥ת הַבָּֽיִת׃}
{וִיתוּב כָּהֲנָא בְּיוֹמָא שְׁבִיעָאָה וְיִחְזֵי וְהָא אוֹסֵיף מַכְתָּשָׁא בְּכוּתְלֵי בֵיתָא׃}
{And the priest shall come again the seventh day, and shall look; and, behold, if the plague be spread in the walls of the house;}{\arabic{verse}}
\threeverse{\arabic{verse}}%Leviticus14:40
{וְצִוָּה֙ הַכֹּהֵ֔ן וְחִלְּצוּ֙ אֶת\maqqaf הָ֣אֲבָנִ֔ים אֲשֶׁ֥ר בָּהֵ֖ן הַנָּ֑גַע וְהִשְׁלִ֤יכוּ אֶתְהֶן֙ אֶל\maqqaf מִח֣וּץ לָעִ֔יר אֶל\maqqaf מָק֖וֹם טָמֵֽא׃}
{וִיפַקֵּיד כָּהֲנָא וִישַׁלְּפוּן יָת אַבְנַיָּא דִּבְהוֹן מַכְתָּשָׁא וְיִרְמוֹן יָתְהוֹן לְמִבַּרָא לְקַרְתָּא לַאֲתַר מְסָאַב׃}
{then the priest shall command that they take out the stones in which the plague is, and cast them into an unclean place without the city.}{\arabic{verse}}
\rashi{\rashiDH{וחלצו את האבנים.} כתרגומו וִישַׁלְּפוּן, יטלום משם, כמו וחלצה נעלו (דברים כה, ט), לשון הסרה׃\quad \rashiDH{אל מקום טמא.} מקום שאין טהרות משתמשות שם, למדך הכתוב שהאבנים הללו מטמאות מקומן בעודן בו (ת״כ פרק ד, ד)׃}
\threeverse{\arabic{verse}}%Leviticus14:41
{וְאֶת\maqqaf הַבַּ֛יִת יַקְצִ֥עַ מִבַּ֖יִת סָבִ֑יב וְשָׁפְכ֗וּ אֶת\maqqaf הֶֽעָפָר֙ אֲשֶׁ֣ר הִקְצ֔וּ אֶל\maqqaf מִח֣וּץ לָעִ֔יר אֶל\maqqaf מָק֖וֹם טָמֵֽא׃}
{וְיָת בֵּיתָא יְקַלְּפוּן מִגָּיו סְחוֹר סְחוֹר וְיִרְמוֹן יָת עַפְרָא דְּקַלִּיפוּ לְמִבַּרָא לְקַרְתָּא לַאֲתַר מְסָאַב׃}
{And he shall cause the house to be scraped within round about, and they shall pour out the mortar that they scrape off without the city into an unclean place.}{\arabic{verse}}
\rashi{\rashiDH{יקצע.} (דריצי״ר בלע״ז), ובלשון משנה יש הרבה׃\quad \rashiDH{מבית.} מבפנים (שם ה)׃\quad \rashiDH{סביב.} סביבות הנגע, בת״כ נדרש כן, שיקלוף הטיח שסביב אבני הנגע׃\quad \rashiDH{הקצו.} לשון קצה, אשר קצעו בקצוע הנגע סביב׃}
\threeverse{\arabic{verse}}%Leviticus14:42
{וְלָקְחוּ֙ אֲבָנִ֣ים אֲחֵר֔וֹת וְהֵבִ֖יאוּ אֶל\maqqaf תַּ֣חַת הָאֲבָנִ֑ים וְעָפָ֥ר אַחֵ֛ר יִקַּ֖ח וְטָ֥ח אֶת\maqqaf הַבָּֽיִת׃}
{וְיִסְּבוּן אַבְנִין אָחֳרָנִין וְיַעֲלוּן בַּאֲתַר אַבְנַיָּא וַעֲפַר אָחֳרָן יִסַּב וִישׁוּעַ יָת בֵּיתָא׃}
{And they shall take other stones, and put them in the place of those stones; and he shall take other mortar, and shall plaster the house.}{\arabic{verse}}
\threeverse{\arabic{verse}}%Leviticus14:43
{וְאִם\maqqaf יָשׁ֤וּב הַנֶּ֙גַע֙ וּפָרַ֣ח בַּבַּ֔יִת אַחַ֖ר חִלֵּ֣ץ אֶת\maqqaf הָאֲבָנִ֑ים וְאַחֲרֵ֛י הִקְצ֥וֹת אֶת\maqqaf הַבַּ֖יִת וְאַחֲרֵ֥י הִטּֽוֹחַ׃}
{וְאִם יְתוּב מַכְתָּשָׁא וְיִסְגֵּי בְּבֵיתָא בָּתַר דְּשַׁלִּיפוּ יָת אַבְנַיָּא וּבָתַר דְּקַלִּיפוּ יָת בֵּיתָא וּבָתַר דְּאִתְּשָׁע׃}
{And if the plague come again, and break out in the house, after that the stones have been taken out, and after the house hath been scraped, and after it is plastered;}{\arabic{verse}}
\rashi{\rashiDH{הקצות.} לשון הֵעָשׂוֹת, וכן הטוח. אבל חלץ את האבנים, מוסב הלשון אל האדם שחלצן, והוא משקל לשון כבד, כמו כִּפֵּר, דִּבֵּר\quad \rashiDH{ואם ישוב הנגע וגו׳.} יכול חזר בו ביום יהא טמא, תלמוד לומר ושב הכהן, ואם ישוב, מה שיבה האמורה להלן, בסוף שבוע, אף שיבה האמורה כאן בסוף שבוע (שם פרשתא ז, ו)׃}
\threeverse{\arabic{verse}}%Leviticus14:44
{וּבָא֙ הַכֹּהֵ֔ן וְרָאָ֕ה וְהִנֵּ֛ה פָּשָׂ֥ה הַנֶּ֖גַע בַּבָּ֑יִת צָרַ֨עַת מַמְאֶ֥רֶת הִ֛וא בַּבַּ֖יִת טָמֵ֥א הֽוּא׃}
{וְיֵיעוֹל כָּהֲנָא וְיִחְזֵי וְהָא אוֹסֵיף מַכְתָּשָׁא בְּבֵיתָא סְגִירוּת מְחַסְּרָא הִיא בְּבֵיתָא מְסָאַב הוּא׃}
{then the priest shall come in and look; and, behold, if the plague be spread in the house, it is a malignant leprosy in the house: it is unclean.}{\arabic{verse}}
\rashi{\rashiDH{ובא הכהן וראה והנה פשה.} יכול לא יהא החוזר טמא אלא אם כן פשה, נאמר צרעת ממארת בבתים, ונאמר צרעת ממארת בבגדים, מה להלן טמא את החוזר אף על פי שאינו פושה (שם ה), אף כאן טמא את החוזר אף על פי שאינו פושה, אם כן מה תלמוד לומר והנה פשה, אין כאן מקומו של מקרא זה, אלא ונתץ את הבית, היה לו לכתוב אחר ואם ישוב הנגע, וראה והנה פשה, הא לא בא ללמד אלא על נגע העומד בעיניו בשבוע ראשון ובא בסוף שבוע שני ומצאו שפשה, שלא פירש בו הכתוב למעלה כלום בעומד בעיניו בשבוע ראשון, ולמדך כאן בפשיון זה שאינו מדבר אלא בעומד בראשון ופשה בשני, ומה יעשה לו, יכול יתצנו כמו שסמך לו ונתץ את הבית, תלמוד לומר ושב הכהן, ובא הכהן, נלמד ביאה משיבה, מה שיבה חולץ וקוצה וטח ונותן לו שבוע, אף ביאה חולץ וקוצה וטח ונותן לו שבוע, (שם ז) ואם חזר נותץ, לא חזר טהור. ומנין שאם עמד בזה ובזה חולץ וקוצה וטח ונותן לו שבוע, תלמוד לומר (ובא), ואם בא יבא, במה הכתוב מדבר, אם בפושה בראשון, הרי כבר אמור, אם בפושה בשני, הרי כבר אמור, הא אינו אמור (ובא) ואם בא יבא, אלא את שבא בסוף שבוע ראשון ובא בסוף שבוע שני וראה והנה לא פשה, זה העומד מה יעשה לו, יכול יפטר וילך, כמו שכתוב כאן וטהר את הבית, תלמוד לומר כי נרפא הנגע, לא טהרתי אלא את הרפוי, מה יעשה לו, ביאה אמורה למעלה וביאה אמורה למטה, מה בעליונה חולץ וקוצה וטח ונותן לו שבוע, דגמר לה זהו שיבה זהו ביאה, אף בתחתונה כך וכו׳, כדאיתא בתורת כהנים (פרשתא ז, י). גמרו של דבר, אין נתיצה אלא בנגע החוזר אחר חליצה וקצוי וטיחה, ואין החוזר צריך פשיון. וסדר המקראות כך הוא, ואם ישוב, ונתץ, והבא אל הבית, והאוכל בבית, ובא הכהן וראה והנה פשה, ודבר הכתוב בעומד בראשון שנותן לו שבוע שני להסגרו, ובסוף שבוע שני להסגרו בא וראהו שפשה, ומה יעשה לו, חולץ וקוצה וטח ונותן לו שבוע, חזר נותץ, לא חזר טעון צפרים, שאין בנגעים יותר מג׳ שבועות׃}
\threeverse{\arabic{verse}}%Leviticus14:45
{וְנָתַ֣ץ אֶת\maqqaf הַבַּ֗יִת אֶת\maqqaf אֲבָנָיו֙ וְאֶת\maqqaf עֵצָ֔יו וְאֵ֖ת כׇּל\maqqaf עֲפַ֣ר הַבָּ֑יִת וְהוֹצִיא֙ אֶל\maqqaf מִח֣וּץ לָעִ֔יר אֶל\maqqaf מָק֖וֹם טָמֵֽא׃}
{וִיתָרַע יָת בֵּיתָא יָת אַבְנוֹהִי וְיָת אָעוֹהִי וְיָת כָּל עֲפַר בֵּיתָא וְיַפֵּיק לְמִבַּרָא לְקַרְתָּא לַאֲתַר מְסָאַב׃}
{And he shall break down the house, the stones of it, and the timber thereof, and all the mortar of the house; and he shall carry them forth out of the city into an unclean place.}{\arabic{verse}}
\threeverse{\arabic{verse}}%Leviticus14:46
{וְהַבָּא֙ אֶל\maqqaf הַבַּ֔יִת כׇּל\maqqaf יְמֵ֖י הִסְגִּ֣יר אֹת֑וֹ יִטְמָ֖א עַד\maqqaf הָעָֽרֶב׃}
{וּדְיֵיעוֹל לְבֵיתָא כָּל יוֹמִין דְּיַסְגַּר יָתֵיהּ יְהֵי מְסָאַב עַד רַמְשָׁא׃}
{Moreover he that goeth into the house all the while that it is shut up shall be unclean until the even.}{\arabic{verse}}
\rashi{\rashiDH{כל ימי הסגיר אותו.} ולא ימים שקלף את נגעו, יכול שאני מוציא המוחלט שקלף את נגעו, תלמוד לומר כל ימי (ת״כ פרק ה, ד)׃\quad \rashiDH{יטמא עד הערב.} מלמד שאין מטמא בגדים, יכול אפי׳ שהה בכדי אכילת פרס, תלמוד לומר והאוכל בבית יכבס את בגדיו, אין לי אלא אוכל, שוכב מנין, תלמוד לומר והשוכב, אין לי אלא אוכל ושוכב, לא אוכל ולא שוכב מנין, תלמוד לומר יכבס, יכבס ריבה, אם כן למה נאמר אוכל ושוכב, ליתן שיעור לשוכב כדי אכילת פרס (ת״כ שם ה, ו, ז, ח  ברכות מא. עירובין פב׃)׃}
\threeverse{\arabic{verse}}%Leviticus14:47
{וְהַשֹּׁכֵ֣ב בַּבַּ֔יִת יְכַבֵּ֖ס אֶת\maqqaf בְּגָדָ֑יו וְהָאֹכֵ֣ל בַּבַּ֔יִת יְכַבֵּ֖ס אֶת\maqqaf בְּגָדָֽיו׃}
{וּדְיִשְׁכּוֹב בְּבֵיתָא יְצַבַּע יָת לְבוּשׁוֹהִי וּדְיֵיכוֹל בְּבֵיתָא יְצַבַּע יָת לְבוּשׁוֹהִי׃}
{And he that lieth in the house shall wash his clothes; and he that eateth in the house shall wash his clothes.}{\arabic{verse}}
\threeverse{\arabic{verse}}%Leviticus14:48
{וְאִם\maqqaf בֹּ֨א יָבֹ֜א הַכֹּהֵ֗ן וְרָאָה֙ וְ֠הִנֵּ֠ה לֹא\maqqaf פָשָׂ֤ה הַנֶּ֙גַע֙ בַּבַּ֔יִת אַחֲרֵ֖י הִטֹּ֣חַ אֶת\maqqaf הַבָּ֑יִת וְטִהַ֤ר הַכֹּהֵן֙ אֶת\maqqaf הַבַּ֔יִת כִּ֥י נִרְפָּ֖א הַנָּֽגַע׃}
{וְאִם מֵיעָל יֵיעוֹל כָּהֲנָא וְיִחְזֵי וְהָא לָא אוֹסֵיף מַכְתָּשָׁא בְּבֵיתָא בָּתַר דְּאִתְּשָׁע יָת בֵּיתָא וְיִדְכֵּי כָהֲנָא יָת בֵּיתָא אֲרֵי אִתַּסִּי מַכְתָּשָׁא׃}
{And if the priest shall come in, and look, and, behold, the plague hath not spread in the house, after the house was plastered; then the priest shall pronounce the house clean, because the plague is healed.}{\arabic{verse}}
\rashi{\rashiDH{ואם בא יבא.} לסוף שבוע שני׃\quad \rashiDH{וראה והנה לא פשה.} מקרא זה בא ללמד בעומד בעיניו בראשון ובשני מה יעשה לו, יכול יטהרנו כמשמעו של מקרא, וטהר הכהן את הבית, תלמוד לומר כי נרפא הנגע, לא טהרתי אלא את הרפוי, ואין רפוי אלא הבית שהוקצה והוטח ולא חזר הנגע, אבל זה טעון חליצה וקצוי וטיחה ושבוע שלישי, וכן המקרא נדרש ואם בא יבא בשני, וראה והנה לא פשה יטיחנו, ואין טיחה בלא חלוץ וקצוי, ואחרי הטוח את הבית, וטהר הכהן את הבית, אם לא חזר לסוף השבוע, כי נרפא הנגע, ואם חזר כבר פירש על החוזר שטעון נתיצה׃}
\threeverse{\arabic{verse}}%Leviticus14:49
{וְלָקַ֛ח לְחַטֵּ֥א אֶת\maqqaf הַבַּ֖יִת שְׁתֵּ֣י צִפֳּרִ֑ים וְעֵ֣ץ אֶ֔רֶז וּשְׁנִ֥י תוֹלַ֖עַת וְאֵזֹֽב׃}
{וְיִסַּב לְדַכָּאָה יָת בֵּיתָא תַּרְתֵּין צִפְּרִין וְאָעָא דְּאַרְזָא וּצְבַע זְהוֹרִי וְאֵיזוֹבָא׃}
{And he shall take to cleanse the house two birds, and cedar-wood, and scarlet, and hyssop.}{\arabic{verse}}
\threeverse{\arabic{verse}}%Leviticus14:50
{וְשָׁחַ֖ט אֶת\maqqaf הַצִּפֹּ֣ר הָאֶחָ֑ת אֶל\maqqaf כְּלִי\maqqaf חֶ֖רֶשׂ עַל\maqqaf מַ֥יִם חַיִּֽים׃}
{וְיִכּוֹס יָת צִפְּרָא חֲדָא לְמָאן דַּחֲסַף עַל מֵי מַבּוּעַ׃}
{And he shall kill one of the birds in an earthen vessel over running water.}{\arabic{verse}}
\threeverse{\arabic{verse}}%Leviticus14:51
{וְלָקַ֣ח אֶת\maqqaf עֵֽץ\maqqaf הָ֠אֶ֠רֶז וְאֶת\maqqaf הָ֨אֵזֹ֜ב וְאֵ֣ת \legarmeh  שְׁנִ֣י הַתּוֹלַ֗עַת וְאֵת֮ הַצִּפֹּ֣ר הַֽחַיָּה֒ וְטָבַ֣ל אֹתָ֗ם בְּדַם֙ הַצִּפֹּ֣ר הַשְּׁחוּטָ֔ה וּבַמַּ֖יִם הַֽחַיִּ֑ים וְהִזָּ֥ה אֶל\maqqaf הַבַּ֖יִת שֶׁ֥בַע פְּעָמִֽים׃}
{וְיִסַּב יָת אָעָא דְּאַרְזָא וְיָת אֵיזוֹבָא וְיָת צְבַע זְהוֹרִי וְיָת צִפְּרָא חַיְתָא וְיִטְבּוֹל יָתְהוֹן בִּדְמָא דְּצִפְּרָא דִּנְכִיסְתָא וּבְמֵי מַבּוּעַ וְיַדֵּי לְבֵיתָא שְׁבַע זִמְנִין׃}
{And he shall take the cedar-wood, and the hyssop, and the scarlet, and the living bird, and dip them in the blood of the slain bird, and in the running water, and sprinkle the house seven times.}{\arabic{verse}}
\threeverse{\arabic{verse}}%Leviticus14:52
{וְחִטֵּ֣א אֶת\maqqaf הַבַּ֔יִת בְּדַם֙ הַצִּפּ֔וֹר וּבַמַּ֖יִם הַֽחַיִּ֑ים וּבַצִּפֹּ֣ר הַחַיָּ֗ה וּבְעֵ֥ץ הָאֶ֛רֶז וּבָאֵזֹ֖ב וּבִשְׁנִ֥י הַתּוֹלָֽעַת׃}
{וִידַכֵּי יָת בֵּיתָא בִּדְמָא דְּצִפְּרָא וּבְמֵי מַבּוּעַ וּבְצִפְּרָא חַיְתָא וּבְאָעָא דְּאַרְזָא וּבְאֵיזוֹבָא וּבִצְבַע זְהוֹרִי׃}
{And he shall cleanse the house with the blood of the bird, and with the running water, and with the living bird, and with the cedar-wood, and with the hyssop, and with the scarlet.}{\arabic{verse}}
\threeverse{\arabic{verse}}%Leviticus14:53
{וְשִׁלַּ֞ח אֶת\maqqaf הַצִּפֹּ֧ר הַֽחַיָּ֛ה אֶל\maqqaf מִח֥וּץ לָעִ֖יר אֶל\maqqaf פְּנֵ֣י הַשָּׂדֶ֑ה וְכִפֶּ֥ר עַל\maqqaf הַבַּ֖יִת וְטָהֵֽר׃}
{וִישַׁלַּח יָת צִפְּרָא חַיְתָא לְמִבַּרָא לְקַרְתָּא לְאַפֵּי חַקְלָא וִיכַפַּר עַל בֵּיתָא וְיִדְכֵּי׃}
{But he shall let go the living bird out of the city into the open field; so shall he make atonement for the house; and it shall be clean.}{\arabic{verse}}
\aliyacounter{חמישי}
\threeverse{\aliya{חמישי}}%Leviticus14:54
{זֹ֖את הַתּוֹרָ֑ה לְכׇל\maqqaf נֶ֥גַע הַצָּרַ֖עַת וְלַנָּֽתֶק׃}
{דָּא אוֹרָיְתָא לְכָל מַכְתָּשׁ סְגִירוּתָא וּלְנִתְקָא׃}
{This is the law for all manner of plague of leprosy, and for a scall;}{\arabic{verse}}
\threeverse{\arabic{verse}}%Leviticus14:55
{וּלְצָרַ֥עַת הַבֶּ֖גֶד וְלַבָּֽיִת׃}
{וְלִסְגִירוּת לְבוּשָׁא וּלְבֵיתָא׃}
{and for the leprosy of a garment, and for a house;}{\arabic{verse}}
\threeverse{\arabic{verse}}%Leviticus14:56
{וְלַשְׂאֵ֥ת וְלַסַּפַּ֖חַת וְלַבֶּהָֽרֶת׃}
{וּלְעָמְקָא וּלְעַדְיָא וּלְבַהֲרָא׃}
{and for a rising, and for a scab, and for a bright spot;}{\arabic{verse}}
\threeverse{\arabic{verse}}%Leviticus14:57
{לְהוֹרֹ֕ת בְּי֥וֹם הַטָּמֵ֖א וּבְי֣וֹם הַטָּהֹ֑ר זֹ֥את תּוֹרַ֖ת הַצָּרָֽעַת׃ \petucha }
{לְאַלָּפָא בְּיוֹם מְסָאֲבָא וּבְיוֹם דָּכְיָא דָּא אוֹרָיְתָא דִּסְגִירוּתָא׃}
{to teach when it is unclean, and when it is clean; this is the law of leprosy.}{\arabic{verse}}
\rashi{\rashiDH{להורות ביום וגו׳.} איזה יום מטהרו, ואיזה יום מטמאו׃}
\newperek
\newseder{11}
\threeverse{\seder{יא}}%Leviticus15:1
{וַיְדַבֵּ֣ר יְהֹוָ֔ה אֶל\maqqaf מֹשֶׁ֥ה וְאֶֽל\maqqaf אַהֲרֹ֖ן לֵאמֹֽר׃}
{וּמַלֵּיל יְיָ עִם מֹשֶׁה וּלְאַהֲרֹן לְמֵימַר׃}
{And the \lord\space spoke unto Moses and to Aaron, saying:}{\Roman{chap}}
\threeverse{\arabic{verse}}%Leviticus15:2
{דַּבְּרוּ֙ אֶל\maqqaf בְּנֵ֣י יִשְׂרָאֵ֔ל וַאֲמַרְתֶּ֖ם אֲלֵהֶ֑ם אִ֣ישׁ אִ֗ישׁ כִּ֤י יִהְיֶה֙ זָ֣ב מִבְּשָׂר֔וֹ זוֹב֖וֹ טָמֵ֥א הֽוּא׃}
{מַלִּילוּ עִם בְּנֵי יִשְׂרָאֵל וְתֵימְרוּן לְהוֹן גְּבַר גְּבַר אֲרֵי יְהֵי דָאֵיב מִבִּשְׂרֵיהּ דּוֹבֵיהּ מְסָאַב הוּא׃}
{Speak unto the children of Israel, and say unto them: When any man hath an issue out of his flesh, his issue is unclean.}{\arabic{verse}}
\rashi{\rashiDH{כי יהיה זב.} יכול זב מכל מקום יהא טמא, תלמוד לומר מבשרו, ולא כל בשרו. (נדה מג.) אחר שחלק הכתוב בין בשר לבשר זכיתי לדין טימא בזב וטימא בזבה, מה זבה ממקום שהיא מְטַמְּאָה טומאה קלה, נדה, מטמאה טומאה חמורה, זיבה, אף הזב ממקום שמטמא טומאה קלה, קרי, מטמא טומאה חמורה זיבה (ת״כ זבים פרשתא א, ה)׃\quad \rashiDH{זובו טמא.} למד על הטפה שהיא מטמאה. זוב דומה לְמֵי בצק של שעורין ודחוי, ודומה ללובן ביצה המוזרת, שכבת זרע קשור כלובן ביצה שאינה מוזרת (נדה לה׃)׃}
\threeverse{\arabic{verse}}%Leviticus15:3
{וְזֹ֛את תִּהְיֶ֥ה טֻמְאָת֖וֹ בְּזוֹב֑וֹ רָ֣ר בְּשָׂר֞וֹ אֶת\maqqaf זוֹב֗וֹ אֽוֹ\maqqaf הֶחְתִּ֤ים בְּשָׂרוֹ֙ מִזּוֹב֔וֹ טֻמְאָת֖וֹ הִֽוא׃}
{וְדָא תְּהֵי סְאוֹבְתֵיהּ בְּדוֹבֵיהּ רִיר בִּסְרֵיהּ יָת דּוֹבֵיהּ אוֹ חֲתִים בִּסְרֵיהּ מִדּוֹבֵיהּ סְאוֹבְתֵיהּ הִיא׃}
{And this shall be his uncleanness in his issue: whether his flesh run with his issue, or his flesh be stopped from his issue, it is his uncleanness.}{\arabic{verse}}
\rashi{\rashiDH{רר.} לשון ריר שזב את בשרו׃ \rashiDH{את זובו.} כמו ריר שיוצא צלול׃\quad \rashiDH{או החתים.} שיוצא עב וסותם את פי האמה, ונסתם בשרו מטפת זובו, זהו פשוטו. ומדרשו (מגילה ח. נדה מג׃) מנה הכתוב הראשון ראיות שתים וקראו טמא, שנאמר זב מבשרו זובו טמא הוא, ומנה הכתוב השני ראיות שלש וקראו טמא, שנאמר טומאתו בזובו רר בשרו את זובו או החתים בשרו מזובו טומאתו היא, הא כיצד, שתים לטומאה, והשלישית מזקיקתו לקרבן׃}
\threeverse{\arabic{verse}}%Leviticus15:4
{כׇּל\maqqaf הַמִּשְׁכָּ֗ב אֲשֶׁ֨ר יִשְׁכַּ֥ב עָלָ֛יו הַזָּ֖ב יִטְמָ֑א וְכׇֽל\maqqaf הַכְּלִ֛י אֲשֶׁר\maqqaf יֵשֵׁ֥ב עָלָ֖יו יִטְמָֽא׃}
{כָּל מִשְׁכְּבָא דְּיִשְׁכּוֹב עֲלוֹהִי דּוֹבָנָא יְהֵי מְסָאַב וְכָל מָאנָא דְּיִתֵּיב עֲלוֹהִי יְהֵי מְסָאַב׃}
{Every bed whereon he that hath the issue lieth shall be unclean; and every thing whereon he sitteth shall be unclean. .}{\arabic{verse}}
\rashi{\rashiDH{כל המשכב.} הראוי למשכב, יכול אפילו מיוחד למלאכה אחרת, תלמוד לומר אשר ישכב, אשר שכב לא נאמר, אלא אשר ישכב, המיוחד תמיד לכך, יצא זה שאומרים לו עמוד ונעשה מלאכתנו׃\quad \rashiDH{אשר ישב.} יָשַׁב לא נאמר, אלא אשר יֵשֵׁב עליו, הזב, במיוחד תמיד לכך (שבת נט.)׃}
\threeverse{\arabic{verse}}%Leviticus15:5
{וְאִ֕ישׁ אֲשֶׁ֥ר יִגַּ֖ע בְּמִשְׁכָּב֑וֹ יְכַבֵּ֧ס בְּגָדָ֛יו וְרָחַ֥ץ בַּמַּ֖יִם וְטָמֵ֥א עַד\maqqaf הָעָֽרֶב׃}
{וּגְבַר דְּיִקְרַב בְּמִשְׁכְּבֵיהּ יְצַבַּע לְבוּשׁוֹהִי וְיַסְחֵי בְמַיָּא וִיהֵי מְסָאַב עַד רַמְשָׁא׃}
{And whosoever toucheth his bed shall wash his clothes, and bathe himself in water, and be unclean until the even.}{\arabic{verse}}
\rashi{\rashiDH{ואיש אשר יגע במשכבו.} לימד על המשכב שחמור מן המגע, שזה נעשה אב הטומאה לטמא אדם לטמא בגדים, והמגע שאינו משכב, אינו אלא ולד הטומאה, ואינו מטמא, אלא אוכלין ומשקין׃}
\threeverse{\arabic{verse}}%Leviticus15:6
{וְהַיֹּשֵׁב֙ עַֽל\maqqaf הַכְּלִ֔י אֲשֶׁר\maqqaf יֵשֵׁ֥ב עָלָ֖יו הַזָּ֑ב יְכַבֵּ֧ס בְּגָדָ֛יו וְרָחַ֥ץ בַּמַּ֖יִם וְטָמֵ֥א עַד\maqqaf הָעָֽרֶב׃}
{וּדְיִתֵּיב עַל מָאנָא דְּיִתֵּיב עֲלוֹהִי דּוֹבָנָא יְצַבַּע לְבוּשׁוֹהִי וְיַסְחֵי בְמַיָּא וִיהֵי מְסָאַב עַד רַמְשָׁא׃}
{And he that sitteth on any thing whereon he that hath the issue sat shall wash his clothes, and bathe himself in water, and be unclean until the even.}{\arabic{verse}}
\rashi{\rashiDH{והישב על הכלי.} אפילו לא נגע, אפילו עשרה כלים זה על זה, כולן מטמאין משום מושב, וכן במשכב (ת״כ פרק ג, א)׃}
\threeverse{\arabic{verse}}%Leviticus15:7
{וְהַנֹּגֵ֖עַ בִּבְשַׂ֣ר הַזָּ֑ב יְכַבֵּ֧ס בְּגָדָ֛יו וְרָחַ֥ץ בַּמַּ֖יִם וְטָמֵ֥א עַד\maqqaf הָעָֽרֶב׃}
{וּדְיִקְרַב בִּבְשַׂר דּוֹבָנָא יְצַבַּע לְבוּשׁוֹהִי וְיַסְחֵי בְמַיָּא וִיהֵי מְסָאַב עַד רַמְשָׁא׃}
{And he that toucheth the flesh of him that hath the issue shall wash his clothes, and bathe himself in water, and be unclean until the even.}{\arabic{verse}}
\threeverse{\arabic{verse}}%Leviticus15:8
{וְכִֽי\maqqaf יָרֹ֥ק הַזָּ֖ב בַּטָּה֑וֹר וְכִבֶּ֧ס בְּגָדָ֛יו וְרָחַ֥ץ בַּמַּ֖יִם וְטָמֵ֥א עַד\maqqaf הָעָֽרֶב׃}
{וַאֲרֵי יִרּוֹק דּוֹבָנָא בְדָכְיָא וִיצַבַּע לְבוּשׁוֹהִי וְיַסְחֵי בְמַיָּא וִיהֵי מְסָאַב עַד רַמְשָׁא׃}
{And if he that hath the issue spit upon him that is clean, then he shall wash his clothes, and bathe himself in water, and be unclean until the even.}{\arabic{verse}}
\rashi{\rashiDH{וכי ירק הזב בטהור.} ונגע בו (שם ח) או נשאו שהרוק מטמא במשא (נדה נה׃)׃}
\threeverse{\arabic{verse}}%Leviticus15:9
{וְכׇל\maqqaf הַמֶּרְכָּ֗ב אֲשֶׁ֨ר יִרְכַּ֥ב עָלָ֛יו הַזָּ֖ב יִטְמָֽא׃}
{וְכָל מֶרְכְּבָא דְּיִרְכוֹב עֲלוֹהִי דּוֹבָנָא יְהֵי מְסָאַב׃}
{And what saddle soever he that hath the issue rideth upon shall be unclean.}{\arabic{verse}}
\rashi{\rashiDH{וכל המרכב.} אף על פי שלא ישב עליו, כגון הַתְּפוּס של סַרְגָא שקורין ארצו״ן, טמא משום מרכב, והאוכף שקורין אליו״ש, טמא טומאת מושב (עירובין כז.)׃}
\threeverse{\arabic{verse}}%Leviticus15:10
{וְכׇל\maqqaf הַנֹּגֵ֗עַ בְּכֹל֙ אֲשֶׁ֣ר יִהְיֶ֣ה תַחְתָּ֔יו יִטְמָ֖א עַד\maqqaf הָעָ֑רֶב וְהַנּוֹשֵׂ֣א אוֹתָ֔ם יְכַבֵּ֧ס בְּגָדָ֛יו וְרָחַ֥ץ בַּמַּ֖יִם וְטָמֵ֥א עַד\maqqaf הָעָֽרֶב׃}
{וְכָל דְּיִקְרַב בְּכֹל דִּיהֵי תְּחוֹתוֹהִי יְהֵי מְסָאַב עַד רַמְשָׁא וּדְיִטּוֹל יָתְהוֹן יְצַבַּע לְבוּשׁוֹהִי וְיַסְחֵי בְמַיָּא וִיהֵי מְסָאַב עַד רַמְשָׁא׃}
{And whosoever toucheth any thing that was under him shall be unclean until the even; and he that beareth those things shall wash his clothes, and bathe himself in water, and be unclean until the even.}{\arabic{verse}}
\rashi{\rashiDH{וכל הנוגע בכל אשר יהיה תחתיו.} של זב (ת״כ פרק ד, א), בא ולימד על המרכב שיהא הנוגע בו טמא ואין טעון כבוס בגדים, והוא חומר במשכב מבמרכב׃\quad \rashiDH{והנושא אותם.} כל האמור בענין הזב, זובו ורוקו ושכבת זרעו ומימי רגליו והמשכב והמרכב, משאן מטמא אדם לטמא בגדים׃}
\threeverse{\arabic{verse}}%Leviticus15:11
{וְכֹ֨ל אֲשֶׁ֤ר יִגַּע\maqqaf בּוֹ֙ הַזָּ֔ב וְיָדָ֖יו לֹא\maqqaf שָׁטַ֣ף בַּמָּ֑יִם וְכִבֶּ֧ס בְּגָדָ֛יו וְרָחַ֥ץ בַּמַּ֖יִם וְטָמֵ֥א עַד\maqqaf הָעָֽרֶב׃}
{וְכֹל דְּיִקְרַב בֵּיהּ דּוֹבָנָא וִידוֹהִי לָא שְׁטֵיף בְּמַיָּא וִיצַבַּע לְבוּשׁוֹהִי וְיַסְחֵי בְמַיָּא וִיהֵי מְסָאַב עַד רַמְשָׁא׃}
{And whomsoever he that hath the issue toucheth, without having rinsed his hands in water, he shall wash his clothes, and bathe himself in water, and be unclean until the even.}{\arabic{verse}}
\rashi{\rashiDH{וידיו לא שטף במים.} בעוד שלא טבל מטומאתו, ואפילו פסק מזובו וספר שבעה ומחוסר טבילה, מטמא בכל טומאותיו. וזה שהוציא הכתוב טבילת גופו של זב בלשון שטיפת ידים, ללמדך שאין בית הסתרים טעון ביאת מים, אלא אבר הגלוי, כמו הידים (ת״כ שם ה  נדה סו׃)׃}
\threeverse{\arabic{verse}}%Leviticus15:12
{וּכְלִי\maqqaf חֶ֛רֶשׂ אֲשֶׁר\maqqaf יִגַּע\maqqaf בּ֥וֹ הַזָּ֖ב יִשָּׁבֵ֑ר וְכׇ֨ל\maqqaf כְּלִי\maqqaf עֵ֔ץ יִשָּׁטֵ֖ף בַּמָּֽיִם׃}
{וּמָאן דַּחֲסַף דְּיִקְרַב בֵּיהּ דּוֹבָנָא יִתְּבַר וְכָל מָאן דְּאָע יִשְׁתְּטֵיף בְּמַיָּא׃}
{And the earthen vessel, which he that hath the issue toucheth, shall be broken; and every vessel of wood shall be rinsed in water.}{\arabic{verse}}
\rashi{\rashiDH{וכלי חרש אשר יגע בו הזב.} יכול אפילו נגע בו מאחוריו וכו׳, כדאיתא בת״כ (פרשתא ג, א), עד איזהו מגעו שהוא ככולו, הוי אומר זה היסטו׃}
\threeverse{\arabic{verse}}%Leviticus15:13
{וְכִֽי\maqqaf יִטְהַ֤ר הַזָּב֙ מִזּוֹב֔וֹ וְסָ֨פַר ל֜וֹ שִׁבְעַ֥ת יָמִ֛ים לְטׇהֳרָת֖וֹ וְכִבֶּ֣ס בְּגָדָ֑יו וְרָחַ֧ץ בְּשָׂר֛וֹ בְּמַ֥יִם חַיִּ֖ים וְטָהֵֽר׃}
{וַאֲרֵי יִדְכֵּי דּוֹבָנָא מִדּוֹבֵיהּ וְיִמְנֵי לֵיהּ שִׁבְעָא יוֹמִין לִדְכוּתֵיהּ וִיצַבַּע לְבוּשׁוֹהִי וְיַסְחֵי בִּשְׂרֵיהּ בְּמֵי מַבּוּעַ וְיִדְכֵּי׃}
{And when he that hath an issue is cleansed of his issue, then he shall number to himself seven days for his cleansing, and wash his clothes; and he shall bathe his flesh in running water, and shall be clean.}{\arabic{verse}}
\rashi{\rashiDH{וכי יטהר.} כשיפסוק (מגילה ח.)׃\quad \rashiDH{שבעת ימים לטהרתו.} שבעת ימים טהורים מטומאת זיבה שלא יראה זוב, וכולן רצופין (נדה סח׃)׃}
\threeverse{\arabic{verse}}%Leviticus15:14
{וּבַיּ֣וֹם הַשְּׁמִינִ֗י יִֽקַּֽח\maqqaf לוֹ֙ שְׁתֵּ֣י תֹרִ֔ים א֥וֹ שְׁנֵ֖י בְּנֵ֣י יוֹנָ֑ה וּבָ֣א \legarmeh  לִפְנֵ֣י יְהֹוָ֗ה אֶל\maqqaf פֶּ֙תַח֙ אֹ֣הֶל מוֹעֵ֔ד וּנְתָנָ֖ם אֶל\maqqaf הַכֹּהֵֽן׃}
{וּבְיוֹמָא תְּמִינָאָה יִסַּב לֵיהּ תְּרֵין שַׁפְנִינִין אוֹ תְּרֵין בְּנֵי יוֹנָה וְיֵיתֵי לִקְדָם יְיָ לִתְרַע מַשְׁכַּן זִמְנָא וְיִתֵּינִנּוּן לְכָהֲנָא׃}
{And on the eighth day he shall take to him two turtle-doves, or two young pigeons, and come before the \lord\space unto the door of the tent of meeting, and give them unto the priest.}{\arabic{verse}}
\threeverse{\arabic{verse}}%Leviticus15:15
{וְעָשָׂ֤ה אֹתָם֙ הַכֹּהֵ֔ן אֶחָ֣ד חַטָּ֔את וְהָאֶחָ֖ד עֹלָ֑ה וְכִפֶּ֨ר עָלָ֧יו הַכֹּהֵ֛ן לִפְנֵ֥י יְהֹוָ֖ה מִזּוֹבֽוֹ׃ \setuma }
{וְיַעֲבֵיד יָתְהוֹן כָּהֲנָא חַד חַטָּתָא וְחַד עֲלָתָא וִיכַפַּר עֲלוֹהִי כָּהֲנָא קֳדָם יְיָ מִדּוֹבֵיהּ׃}
{And the priest shall offer them, the one for a sin-offering, and the other for a burnt-offering; and the priest shall make atonement for him before the \lord\space for his issue.}{\arabic{verse}}
\aliyacounter{ששי}
\threeverse{\aliya{ששי\newline (שביעי)}}%Leviticus15:16
{וְאִ֕ישׁ כִּֽי\maqqaf תֵצֵ֥א מִמֶּ֖נּוּ שִׁכְבַת\maqqaf זָ֑רַע וְרָחַ֥ץ בַּמַּ֛יִם אֶת\maqqaf כׇּל\maqqaf בְּשָׂר֖וֹ וְטָמֵ֥א עַד\maqqaf הָעָֽרֶב׃}
{וּגְבַר אֲרֵי תִפּוֹק מִנֵּיהּ שִׁכְבַת זַרְעָא וְיַסְחֵי בְמַיָּא יָת כָּל בִּסְרֵיהּ וִיהֵי מְסָאַב עַד רַמְשָׁא׃}
{And if the flow of seed go out from a man, then he shall bathe all his flesh in water, and be unclean until the even.}{\arabic{verse}}
\threeverse{\arabic{verse}}%Leviticus15:17
{וְכׇל\maqqaf בֶּ֣גֶד וְכׇל\maqqaf ע֔וֹר אֲשֶׁר\maqqaf יִהְיֶ֥ה עָלָ֖יו שִׁכְבַת\maqqaf זָ֑רַע וְכֻבַּ֥ס בַּמַּ֖יִם וְטָמֵ֥א עַד\maqqaf הָעָֽרֶב׃}
{וְכָל לְבוּשׁ וְכָל מְשַׁךְ דִּיהֵי עֲלוֹהִי שִׁכְבַת זַרְעָא וְיִצְטְבַע בְּמַיָּא וִיהֵי מְסָאַב עַד רַמְשָׁא׃}
{And every garment, and every skin, whereon is the flow of seed, shall be washed with water, and be unclean until the even.}{\arabic{verse}}
\threeverse{\arabic{verse}}%Leviticus15:18
{וְאִשָּׁ֕ה אֲשֶׁ֨ר יִשְׁכַּ֥ב אִ֛ישׁ אֹתָ֖הּ שִׁכְבַת\maqqaf זָ֑רַע וְרָחֲצ֣וּ בַמַּ֔יִם וְטָמְא֖וּ עַד\maqqaf הָעָֽרֶב׃ \petucha }
{וְאִתְּתָא דְּיִשְׁכּוֹב גְּבַר יָתַהּ שִׁכְבַת זַרְעָא וְיַסְחוֹן בְּמַיָּא וִיהוֹן מְסָאֲבִין עַד רַמְשָׁא׃}
{The woman also with whom a man shall lie carnally, they shall both bathe themselves in water, and be unclean until the even.}{\arabic{verse}}
\rashi{\rashiDH{ורחצו במים.} גזירת מלך היא שֶׁתִּטָּמֵא האשה בביאה, ואין הטעם משום נוגע בשכבת זרע, שהרי מגע בית הסתרים היא (נדה מא׃)׃}
\threeverse{\arabic{verse}}%Leviticus15:19
{וְאִשָּׁה֙ כִּֽי\maqqaf תִהְיֶ֣ה זָבָ֔ה דָּ֛ם יִהְיֶ֥ה זֹבָ֖הּ בִּבְשָׂרָ֑הּ שִׁבְעַ֤ת יָמִים֙ תִּהְיֶ֣ה בְנִדָּתָ֔הּ וְכׇל\maqqaf הַנֹּגֵ֥עַ בָּ֖הּ יִטְמָ֥א עַד\maqqaf הָעָֽרֶב׃}
{וְאִתְּתָא אֲרֵי תְהֵי דָּיְבָא דַּם יְהֵי דוֹבַהּ בְּבִשְׂרַהּ שִׁבְעָא יוֹמִין תְּהֵי בְּרִיחוּקַהּ וְכָל דְּיִקְרַב בַּהּ יְהֵי מְסָאַב עַד רַמְשָׁא׃}
{And if a woman have an issue, and her issue in her flesh be blood, she shall be in her impurity seven days; and whosoever toucheth her shall be unclean until the even.}{\arabic{verse}}
\rashi{\rashiDH{כי תהיה זבה.} (ת״כ) יכול מאחד מכל איבריה, תלמוד לומר והיא גלתה את מקור דמיה (ויקרא כ, יח), אין דם מטמא אלא הבא מן המקור (ת״כ פרשתא ד, ב)׃\quad \rashiDH{דם יהיה זבה בבשרה.} אין זובה קרוי זוב לטמא אלא אם כן הוא אדום (נדה יט.)׃\quad \rashiDH{בנדתה.} כמו וּמִתֵּבֵל יְנִדֻּהוּ (איוב יח, יח), שהיא מנודה ממגע כל אדם׃\quad \rashiDH{תהיה בנדתה.} אפילו לא ראתה אלא ראיה ראשונה׃}
\threeverse{\arabic{verse}}%Leviticus15:20
{וְכֹל֩ אֲשֶׁ֨ר תִּשְׁכַּ֥ב עָלָ֛יו בְּנִדָּתָ֖הּ יִטְמָ֑א וְכֹ֛ל אֲשֶׁר\maqqaf תֵּשֵׁ֥ב עָלָ֖יו יִטְמָֽא׃}
{וְכֹל דְּתִשְׁכוֹב עֲלוֹהִי בְּרִיחוּקַהּ יְהֵי מְסָאַב וְכֹל דְּתִתֵּיב עֲלוֹהִי יְהֵי מְסָאַב׃}
{And every thing that she lieth upon in her impurity shall be unclean; every thing also that she sitteth upon shall be unclean.}{\arabic{verse}}
\threeverse{\arabic{verse}}%Leviticus15:21
{וְכׇל\maqqaf הַנֹּגֵ֖עַ בְּמִשְׁכָּבָ֑הּ יְכַבֵּ֧ס בְּגָדָ֛יו וְרָחַ֥ץ בַּמַּ֖יִם וְטָמֵ֥א עַד\maqqaf הָעָֽרֶב׃}
{וְכָל דְּיִקְרַב בְּמִשְׁכְּבַהּ יְצַבַּע לְבוּשׁוֹהִי וְיַסְחֵי בְמַיָּא וִיהֵי מְסָאַב עַד רַמְשָׁא׃}
{And whosoever toucheth her bed shall wash his clothes, and bathe himself in water, and be unclean until the even.}{\arabic{verse}}
\threeverse{\arabic{verse}}%Leviticus15:22
{וְכׇ֨ל\maqqaf הַנֹּגֵ֔עַ בְּכׇל\maqqaf כְּלִ֖י אֲשֶׁר\maqqaf תֵּשֵׁ֣ב עָלָ֑יו יְכַבֵּ֧ס בְּגָדָ֛יו וְרָחַ֥ץ בַּמַּ֖יִם וְטָמֵ֥א עַד\maqqaf הָעָֽרֶב׃}
{וְכָל דְּיִקְרַב בְּכָל מָאנָא דְּתִתֵּיב עֲלוֹהִי יְצַבַּע לְבוּשׁוֹהִי וְיַסְחֵי בְמַיָּא וִיהֵי מְסָאַב עַד רַמְשָׁא׃}
{And whosoever toucheth any thing that she sitteth upon shall wash his clothes, and bathe himself in water, and be unclean until the even.}{\arabic{verse}}
\threeverse{\arabic{verse}}%Leviticus15:23
{וְאִ֨ם עַֽל\maqqaf הַמִּשְׁכָּ֜ב ה֗וּא א֧וֹ עַֽל\maqqaf הַכְּלִ֛י אֲשֶׁר\maqqaf הִ֥וא יֹשֶֽׁבֶת\maqqaf עָלָ֖יו בְּנׇגְעוֹ\maqqaf ב֑וֹ יִטְמָ֖א עַד\maqqaf הָעָֽרֶב׃}
{וְאִם עַל מִשְׁכְּבָא הוּא אוֹ עַל מָאנָא דְּהִיא יָתְבָא עֲלוֹהִי בְּמִקְרְבֵיהּ בֵּיהּ יְהֵי מְסָאַב עַד רַמְשָׁא׃}
{And if he be on the bed, or on any thing whereon she sitteth, when he toucheth it, he shall be unclean until the even.}{\arabic{verse}}
\rashi{\rashiDH{ואם על המשכב הוא.} השוכב, או היושב על משכבה, או על מושבה, אפי׳ לא נגע בה, אף הוא בדת טומאה האמורה במקרא העליון שטעון כבוס בגדים׃\quad \rashiDH{על הכלי.} לרבות את המרכב׃\quad \rashiDH{בנגעו בו יטמא.} אינו מדבר אלא על המרכב שנתרבה מעל הכלי׃\quad \rashiDH{בנגעו בו יטמא.} ואינו טעון כבוס בגדים, שהמרכב אין מגעו מטמא אדם לטמא בגדים׃}
\threeverse{\arabic{verse}}%Leviticus15:24
{וְאִ֡ם שָׁכֹב֩ יִשְׁכַּ֨ב אִ֜ישׁ אֹתָ֗הּ וּתְהִ֤י נִדָּתָהּ֙ עָלָ֔יו וְטָמֵ֖א שִׁבְעַ֣ת יָמִ֑ים וְכׇל\maqqaf הַמִּשְׁכָּ֛ב אֲשֶׁר\maqqaf יִשְׁכַּ֥ב עָלָ֖יו יִטְמָֽא׃ \setuma }
{וְאִם מִשְׁכָּב יִשְׁכּוֹב גְּבַר יָתַהּ וִיהֵי רִיחוּקַהּ עֲלוֹהִי וִיהֵי מְסָאַב שִׁבְעָא יוֹמִין וְכָל מִשְׁכְּבָא דְּיִשְׁכוֹב עֲלוֹהִי יְהֵי מְסָאַב׃}
{And if any man lie with her, and her impurity be upon him, he shall be unclean seven days; and every bed whereon he lieth shall be unclean. .}{\arabic{verse}}
\rashi{\rashiDH{ותהי נדתה עליו.} יכול יעלה לרגלה, שאם בא עליה בחמישי לנדתה לא יטמא אלא ג׳ ימים כמותה, תלמוד לומר וטמא שבעת ימים, ומה תלמוד לומר ותהי נדתה עליו, מה היא מטמאה אדם וכלי חרס, אף הוא מטמא אדם וכלי חרס (שם פרק ז, ג  נדה לג)׃}
\newseder{12}
\threeverse{\seder{יב}}%Leviticus15:25
{וְאִשָּׁ֡ה כִּֽי\maqqaf יָזוּב֩ ז֨וֹב דָּמָ֜הּ יָמִ֣ים רַבִּ֗ים בְּלֹא֙ עֶת\maqqaf נִדָּתָ֔הּ א֥וֹ כִֽי\maqqaf תָז֖וּב עַל\maqqaf נִדָּתָ֑הּ כׇּל\maqqaf יְמֵ֞י ז֣וֹב טֻמְאָתָ֗הּ כִּימֵ֧י נִדָּתָ֛הּ תִּהְיֶ֖ה טְמֵאָ֥ה הִֽוא׃}
{וְאִתְּתָא אֲרֵי יְדוּב דּוֹב דְּמַהּ יוֹמִין סַגִּיאִין בְּלָא עִדָּן רִיחוּקַהּ אוֹ אֲרֵי תְדוּב עַל רִיחוּקַהּ כָּל יוֹמֵי דּוֹב סְאוֹבְתַהּ כְּיוֹמֵי רִיחוּקַהּ תְּהֵי מְסָאֲבָא הִיא׃}
{And if a woman have an issue of her blood many days not in the time of her impurity, or if she have an issue beyond the time of her impurity; all the days of the issue of her uncleanness she shall be as in the days of her impurity: she is unclean.}{\arabic{verse}}
\rashi{\rashiDH{ימים רבים.} שלשה ימים׃\quad \rashiDH{בלא עת נדתה.} אחר שיצאו שבעת ימי נדתה׃\quad \rashiDH{או כי תזוב.} את ג׳ הימים הללו׃\quad \rashiDH{על נדתה.} מופלג מנדתה יום אחד זו היא זבה, ומשפטה חרוץ בפרשה זו, ולא כדת הנדה, שזו טעונה ספירת ז׳ נקיים וקרבן, והנדה אינה טעונה ספירת ז׳ נקיים, אלא שבעת ימים תהיה בנדתה, בין רואה בין שאינה רואה, ודרשו רבותינו (נדה עג.) בפרשה זו, י״א יום שבין סוף נדה לתחלת נדה, שכל שלשה רצופין שתראה באחד עשר יום הללו תהא זבה׃}
\threeverse{\arabic{verse}}%Leviticus15:26
{כׇּל\maqqaf הַמִּשְׁכָּ֞ב אֲשֶׁר\maqqaf תִּשְׁכַּ֤ב עָלָיו֙ כׇּל\maqqaf יְמֵ֣י זוֹבָ֔הּ כְּמִשְׁכַּ֥ב נִדָּתָ֖הּ יִֽהְיֶה\maqqaf לָּ֑הּ וְכׇֽל\maqqaf הַכְּלִי֙ אֲשֶׁ֣ר תֵּשֵׁ֣ב עָלָ֔יו טָמֵ֣א יִהְיֶ֔ה כְּטֻמְאַ֖ת נִדָּתָֽהּ׃}
{כָּל מִשְׁכְּבָא דְּתִשְׁכוֹב עֲלוֹהִי כָּל יוֹמֵי דּוֹבַהּ כְּמִשְׁכַּב רִיחוּקַהּ יְהֵי לַהּ וְכָל מָאנָא דְּתִתֵּיב עֲלוֹהִי מְסָאַב יְהֵי כְּסוֹאֲבָת רִיחוּקַהּ׃}
{Every bed whereon she lieth all the days of her issue shall be unto her as the bed of her impurity; and every thing whereon she sitteth shall be unclean, as the uncleanness of her impurity.}{\arabic{verse}}
\threeverse{\arabic{verse}}%Leviticus15:27
{וְכׇל\maqqaf הַנּוֹגֵ֥עַ בָּ֖ם יִטְמָ֑א וְכִבֶּ֧ס בְּגָדָ֛יו וְרָחַ֥ץ בַּמַּ֖יִם וְטָמֵ֥א עַד\maqqaf הָעָֽרֶב׃}
{וְכָל דְּיִקְרַב בְּהוֹן יְהֵי מְסָאַב וִיצַבַּע לְבוּשׁוֹהִי וְיַסְחֵי בְּמַיָּא וִיהֵי מְסָאַב עַד רַמְשָׁא׃}
{And whosoever toucheth those things shall be unclean, and shall wash his clothes, and bathe himself in water, and be unclean until the even.}{\arabic{verse}}
\threeverse{\arabic{verse}}%Leviticus15:28
{וְאִֽם\maqqaf טָהֲרָ֖ה מִזּוֹבָ֑הּ וְסָ֥פְרָה לָּ֛הּ שִׁבְעַ֥ת יָמִ֖ים וְאַחַ֥ר תִּטְהָֽר׃}
{וְאִם דְּכִיאַת מִדּוֹבַהּ וְתִמְנֵי לַהּ שִׁבְעָא יוֹמִין וּבָתַר כֵּן תִּדְכֵּי׃}
{But if she be cleansed of her issue, then she shall number to herself seven days, and after that she shall be clean.}{\arabic{verse}}
\aliyacounter{שביעי}
\threeverse{\aliya{שביעי}}%Leviticus15:29
{וּבַיּ֣וֹם הַשְּׁמִינִ֗י תִּֽקַּֽח\maqqaf לָהּ֙ שְׁתֵּ֣י תֹרִ֔ים א֥וֹ שְׁנֵ֖י בְּנֵ֣י יוֹנָ֑ה וְהֵבִיאָ֤ה אוֹתָם֙ אֶל\maqqaf הַכֹּהֵ֔ן אֶל\maqqaf פֶּ֖תַח אֹ֥הֶל מוֹעֵֽד׃}
{וּבְיוֹמָא תְּמִינָאָה תִּסַּב לַהּ תְּרֵין שַׁפְנִינִין אוֹ תְּרֵין בְּנֵי יוֹנָה וְתַיְתֵי יָתְהוֹן לְוָת כָּהֲנָא לִתְרַע מַשְׁכַּן זִמְנָא׃}
{And on the eighth day she shall take unto her two turtle-doves, or two young pigeons, and bring them unto the priest, to the door of the tent of meeting.}{\arabic{verse}}
\threeverse{\arabic{verse}}%Leviticus15:30
{וְעָשָׂ֤ה הַכֹּהֵן֙ אֶת\maqqaf הָאֶחָ֣ד חַטָּ֔את וְאֶת\maqqaf הָאֶחָ֖ד עֹלָ֑ה וְכִפֶּ֨ר עָלֶ֤יהָ הַכֹּהֵן֙ לִפְנֵ֣י יְהֹוָ֔ה מִזּ֖וֹב טֻמְאָתָֽהּ׃}
{וְיַעֲבֵיד כָּהֲנָא יָת חַד חַטָּתָא וְיָת חַד עֲלָתָא וִיכַפַּר עֲלַהּ כָּהֲנָא קֳדָם יְיָ מִדּוֹב סְאוֹבְתַהּ׃}
{And the priest shall offer the one for a sin-offering, and the other for a burnt-offering; and the priest shall make atonement for her before the \lord\space for the issue of her uncleanness.}{\arabic{verse}}
\threeverse{\aliya{מפטיר}}%Leviticus15:31
{וְהִזַּרְתֶּ֥ם אֶת\maqqaf בְּנֵי\maqqaf יִשְׂרָאֵ֖ל מִטֻּמְאָתָ֑ם וְלֹ֤א יָמֻ֙תוּ֙ בְּטֻמְאָתָ֔ם בְּטַמְּאָ֥ם אֶת\maqqaf מִשְׁכָּנִ֖י אֲשֶׁ֥ר בְּתוֹכָֽם׃}
{וְתַפְרְשׁוּן יָת בְּנֵי יִשְׂרָאֵל מִסּוֹאֲבָתְהוֹן וְלָא יְמוּתוּן בְּסוֹאֲבָתְהוֹן בְּסַאוֹבֵיהוֹן יָת מַשְׁכְּנִי דְּבֵינֵיהוֹן׃}
{Thus shall ye separate the children of Israel from their uncleanness; that they die not in their uncleanness, when they defile My tabernacle that is in the midst of them.}{\arabic{verse}}
\rashi{\rashiDH{והזרתם.} אין נזירה אלא פרישה, וכן נָזֹרוּ אָחוֹר (ישעי׳ א, ד), וכן נזיר אחיו (בראשית מט, כו)׃\quad \rashiDH{ולא ימתו בטמאתם.} הרי הכרת של מטמא מקדש קרוי מיתה׃}
\threeverse{\arabic{verse}}%Leviticus15:32
{זֹ֥את תּוֹרַ֖ת הַזָּ֑ב וַאֲשֶׁ֨ר תֵּצֵ֥א מִמֶּ֛נּוּ שִׁכְבַת\maqqaf זֶ֖רַע לְטׇמְאָה\maqqaf בָֽהּ׃}
{דָּא אוֹרָיְתָא דְּדוֹבָנָא וּדְתִפּוֹק מִנֵּיהּ שִׁכְבַת זַרְעָא לְאִסְתַּאָבָא בַהּ׃}
{This is the law of him that hath an issue, and of him from whom the flow of seed goeth out, so that he is unclean thereby;}{\arabic{verse}}
\rashi{\rashiDH{זאת תורת הזב.} בעל ראיה אחת, ומהו תורתו׃\quad \rashiDH{ואשר תצא ממנו שכבת זרע.} הרי הוא כבעל קרי, טמא טומאת ערב׃}
\threeverse{\aliya{\Hebrewnumeral{90}}}%Leviticus15:33
{וְהַדָּוָה֙ בְּנִדָּתָ֔הּ וְהַזָּב֙ אֶת\maqqaf זוֹב֔וֹ לַזָּכָ֖ר וְלַנְּקֵבָ֑ה וּלְאִ֕ישׁ אֲשֶׁ֥ר יִשְׁכַּ֖ב עִם\maqqaf טְמֵאָֽה׃ \petucha }
{וְלִדְסְאוֹבְתַהּ בְּרִיחוּקַהּ וְלִדְדָּאִיב יָת דּוֹבֵיהּ לִדְכַר וּלְנוּקְבָּא וְלִגְבַר דְּיִשְׁכּוֹב עִם מְסָאַבְתָּא׃}
{and of her that is sick with her impurity, and of them that have an issue, whether it be a man, or a woman; and of him that lieth with her that is unclean.}{\arabic{verse}}
\rashi{\rashiDH{והזב את זובו.} בעל שתי ראיות ובעל שלש ראיות, שתורתן מפורשת למעלה׃}
\engnote{The Haftarah is II Kings 7:3\verserangechar 7:20 on page \pageref{haft_28}. On the Shabbat before Pesa\d{h}, read the Haftara on page \pageref{haft_hagadol}. On Rosh \d{H}odesh, read the Maftir and Haftara on page \pageref{maft_ShabRCh}. On Erev Rosh \d{H}odesh, read the Haftara on page \pageref{haft_macharchodesh}.}
\newperek
\aliyacounter{ראשון}
\newparsha{אחרי מות}
\threeverse{\aliya{אחרי מות}}%Leviticus16:1
{וַיְדַבֵּ֤ר יְהֹוָה֙ אֶל\maqqaf מֹשֶׁ֔ה אַחֲרֵ֣י מ֔וֹת שְׁנֵ֖י בְּנֵ֣י אַהֲרֹ֑ן בְּקׇרְבָתָ֥ם לִפְנֵי\maqqaf יְהֹוָ֖ה וַיָּמֻֽתוּ׃}
{וּמַלֵּיל יְיָ עִם מֹשֶׁה בָּתַר דְּמִיתוּ תְּרֵין בְּנֵי אַהֲרֹן בְּקָרוֹבֵיהוֹן אִישָׁתָא נוּכְרֵיתָא קֳדָם יְיָ וּמִיתוּ׃}
{And the \lord\space spoke unto Moses, after the death of the two sons of Aaron, when they drew near before the \lord, and died;}{\Roman{chap}}
\rashi{\rashiDH{וידבר ה׳ אל משה אחרי מות שני בני אהרן וגו׳.} מה תלמוד לומר, היה רבי אלעזר בן עזריה מושלו, משל לחולה שנכנס אצלו רופא, אמר לו אל תאכל צונן ואל תשכב בטחב, בא אחר ואמר לו אל תאכל צונן ואל תשכב בטחב שלא תמות כדרך שמת פלוני, זה זרזו יותר מן הראשון, לכך נאמר אחרי מות שני בני אהרן׃ 
}
\threeverse{\arabic{verse}}%Leviticus16:2
{וַיֹּ֨אמֶר יְהֹוָ֜ה אֶל\maqqaf מֹשֶׁ֗ה דַּבֵּר֮ אֶל\maqqaf אַהֲרֹ֣ן אָחִ֒יךָ֒ וְאַל\maqqaf יָבֹ֤א בְכׇל\maqqaf עֵת֙ אֶל\maqqaf הַקֹּ֔דֶשׁ מִבֵּ֖ית לַפָּרֹ֑כֶת אֶל\maqqaf פְּנֵ֨י הַכַּפֹּ֜רֶת אֲשֶׁ֤ר עַל\maqqaf הָאָרֹן֙ וְלֹ֣א יָמ֔וּת כִּ֚י בֶּֽעָנָ֔ן אֵרָאֶ֖ה עַל\maqqaf הַכַּפֹּֽרֶת׃}
{וַאֲמַר יְיָ לְמֹשֶׁה מַלֵּיל עִם אַהֲרֹן אֲחוּךְ וְלָא יְהֵי עָלֵיל בְּכָל עִדָּן לְקוּדְשָׁא מִגָּיו לְפָרוּכְתָּא לִקְדָם כָּפוּרְתָּא דְּעַל אֲרוֹנָא וְלָא יְמוּת אֲרֵי בַעֲנָנָא אֲנָא מִתְגְּלֵי עַל בֵּית כַּפּוֹרֵי׃}
{and the \lord\space said unto Moses: ‘Speak unto Aaron thy brother, that he come not at all times into the holy place within the veil, before the ark-cover which is upon the ark; that he die not; for I appear in the cloud upon the ark-cover.}{\arabic{verse}}
\rashi{\rashiDH{ויאמר ה׳ אל משה דבר אל אהרן אחיך ואל יבא.} שלא ימות כדרך שמתו בניו׃\quad \rashiDH{ולא ימות.} שאם בא הוא מת׃ 
\quad \rashiDH{כי בענן אראה.} כי תמיד אני נראה שם עם עמוד ענני, ולפי שגלוי שכינתי שם יזהר שלא ירגיל לבא, זהו פשוטו. ומדרשו לא יבא כי אם בענן הקטרת ביום הכיפורים (יומא נג.)׃}
\threeverse{\arabic{verse}}%Leviticus16:3
{בְּזֹ֛את יָבֹ֥א אַהֲרֹ֖ן אֶל\maqqaf הַקֹּ֑דֶשׁ בְּפַ֧ר בֶּן\maqqaf בָּקָ֛ר לְחַטָּ֖את וְאַ֥יִל לְעֹלָֽה׃}
{בְּדָא יְהֵי עָלֵיל אַהֲרֹן לְקוּדְשָׁא בְּתוֹר בַּר תּוֹרֵי לְחַטָּתָא וּדְכַר לַעֲלָתָא׃}
{Herewith shall Aaron come into the holy place: with a young bullock for a sin-offering, and a ram for a burnt-offering.}{\arabic{verse}}
\rashi{\rashiDH{בזאת.} גימטריא שלו ד׳ מאות ועשר, רמז לבית ראשון׃ 
\quad \rashiDH{בזאת יבא אהרן וגו׳.} ואף זו לא בכל עת כי אם ביום הכיפורים, כמו שמפורש בסוף הפרשה בחדש השביעי בעשור לחודש׃}
\threeverse{\arabic{verse}}%Leviticus16:4
{כְּתֹֽנֶת\maqqaf בַּ֨ד קֹ֜דֶשׁ יִלְבָּ֗שׁ וּמִֽכְנְסֵי\maqqaf בַד֮ יִהְי֣וּ עַל\maqqaf בְּשָׂרוֹ֒ וּבְאַבְנֵ֥ט בַּד֙ יַחְגֹּ֔ר וּבְמִצְנֶ֥פֶת בַּ֖ד יִצְנֹ֑ף בִּגְדֵי\maqqaf קֹ֣דֶשׁ הֵ֔ם וְרָחַ֥ץ בַּמַּ֛יִם אֶת\maqqaf בְּשָׂר֖וֹ וּלְבֵשָֽׁם׃}
{כִּתּוּנָא דְּבוּצָא דְּקוּדְשָׁא יִלְבַּשׁ וּמִכְנְסִין דְּבוּץ יְהוֹן עַל בִּשְׂרֵיהּ וְהִמְיָנָא דְּבוּצָא יֵיסַר וּמַצְנַפְתָּא דְּבוּצָא יַחֵית בְּרֵישֵׁיהּ לְבוּשֵׁי קוּדְשָׁא אִנּוּן וְיַסְחֵי בְּמַיָּא יָת בִּשְׂרֵיהּ וְיִלְבְּשִׁנּוּן׃}
{He shall put on the holy linen tunic, and he shall have the linen breeches upon his flesh, and shall be girded with the linen girdle, and with the linen mitre shall he be attired; they are the holy garments; and he shall bathe his flesh in water, and put them on.}{\arabic{verse}}
\rashi{\rashiDH{כתנת בד וגו׳.} מגיד שאינו משמש לפנים בשמונה בגדים שהוא משמש בהם בחוץ שיש בהם זהב, לפי שאין קטיגור נעשה סניגור, (ר״ה כו.) אלא בד׳ ככהן הדיוט, וכולן של בוץ׃\quad \rashiDH{קדש ילבש.} שיהיו משל הקדש (ת״כ פרק א, י)׃\quad \rashiDH{יצנף.} כתרגומו יָחֵת בְּרֵישֵׁהּ, יניח בראשו, כמו ותנח בגדו (בראשית לט, טז), וְאַחְתְּתֵיהּ׃\quad \rashiDH{ורחץ במים.} אותו היום טעון טבילה בכל חליפותיו. וחמש פעמים היה מחליף מעבודת פנים לעבודת חוץ, ומחוץ לפנים, ומשנה מבגדי זהב לבגדי לבן, ומבגדי לבן לבגדי זהב, ובכל חליפה טעון טבילה ושני קדושי ידים ורגלים מן הכיור (יומא לב.)׃ 
}
\threeverse{\arabic{verse}}%Leviticus16:5
{וּמֵאֵ֗ת עֲדַת֙ בְּנֵ֣י יִשְׂרָאֵ֔ל יִקַּ֛ח שְׁנֵֽי\maqqaf שְׂעִירֵ֥י עִזִּ֖ים לְחַטָּ֑את וְאַ֥יִל אֶחָ֖ד לְעֹלָֽה׃}
{וּמִן כְּנִשְׁתָּא דִּבְנֵי יִשְׂרָאֵל יִסַּב תְּרֵין צְפִירֵי עִזֵּי לְחַטָּתָא וּדְכַר חַד לַעֲלָתָא׃}
{And he shall take of the congregation of the children of Israel two he-goats for a sin-offering, and one ram for a burnt-offering.}{\arabic{verse}}
\threeverse{\arabic{verse}}%Leviticus16:6
{וְהִקְרִ֧יב אַהֲרֹ֛ן אֶת\maqqaf פַּ֥ר הַחַטָּ֖את אֲשֶׁר\maqqaf ל֑וֹ וְכִפֶּ֥ר בַּעֲד֖וֹ וּבְעַ֥ד בֵּיתֽוֹ׃}
{וִיקָרֵיב אַהֲרֹן יָת תּוֹרָא דְּחַטָּתָא דִּילֵיהּ וִיכַפַּר עֲלוֹהִי וְעַל אֱנָשׁ בֵּיתֵיהּ׃}
{And Aaron shall present the bullock of the sin-offering, which is for himself, and make atonement for himself, and for his house.}{\arabic{verse}}
\rashi{\rashiDH{את פר החטאת אשר לו.} האמור למעלה, ולמדך כאן שמשלו הוא בא ולא משל צבור (שם ג׃)׃\quad \rashiDH{וכפר בעדו ובעד ביתו.} מתודה עליו עונותיו ועונות ביתו (שם לו׃  ת״כ פרשתא ב, ג)׃ 
}
\threeverse{\aliya{לוי}}%Leviticus16:7
{וְלָקַ֖ח אֶת\maqqaf שְׁנֵ֣י הַשְּׂעִירִ֑ם וְהֶעֱמִ֤יד אֹתָם֙ לִפְנֵ֣י יְהֹוָ֔ה פֶּ֖תַח אֹ֥הֶל מוֹעֵֽד׃}
{וְיִסַּב יָת תְּרֵין צְפִירִין וִיקִים יָתְהוֹן קֳדָם יְיָ בִּתְרַע מַשְׁכַּן זִמְנָא׃}
{And he shall take the two goats, and set them before the \lord\space at the door of the tent of meeting.}{\arabic{verse}}
\threeverse{\arabic{verse}}%Leviticus16:8
{וְנָתַ֧ן אַהֲרֹ֛ן עַל\maqqaf שְׁנֵ֥י הַשְּׂעִירִ֖ם גֹּרָל֑וֹת גּוֹרָ֤ל אֶחָד֙ לַיהֹוָ֔ה וְגוֹרָ֥ל אֶחָ֖ד לַעֲזָאזֵֽל׃}
{וְיִתֵּין אַהֲרֹן עַל תְּרֵין צְפִירִין עַדְבִּין עַדְבָּא חַד לִשְׁמָא דַּייָ וְעַדְבָּא חַד לַעֲזָאזֵל׃}
{And Aaron shall cast lots upon the two goats: one lot for the \lord, and the other lot for Azazel.}{\arabic{verse}}
\rashi{\rashiDH{ונתן אהרן על שני השעירים גרלות.} מעמיד אחד לימין ואחד לשמאל, ונותן ב׳ ידיו בקלפי, ונוטל גורל בימין וחברו בשמאל, ונותן עליהם, את שכתוב בו לשם, הוא לשם ואת שכתוב בו לעזאזל, משתלח לעזאזל (יומא לט.)׃ 
\quad \rashiDH{עזאזל.} הוא הר עז וקשה, צוק גבוה (ת״כ פרק ב, ח  יומא סז׃), שנאמר אֶרֶץ גְּזֵרָה, חתוכה׃}
\threeverse{\arabic{verse}}%Leviticus16:9
{וְהִקְרִ֤יב אַהֲרֹן֙ אֶת\maqqaf הַשָּׂעִ֔יר אֲשֶׁ֨ר עָלָ֥ה עָלָ֛יו הַגּוֹרָ֖ל לַיהֹוָ֑ה וְעָשָׂ֖הוּ חַטָּֽאת׃}
{וִיקָרֵיב אַהֲרֹן יָת צְפִירָא דִּסְלֵיק עֲלוֹהִי עַדְבָּא לִשְׁמָא דַּייָ וְיַעְבְּדִנֵּיהּ חַטָּתָא׃}
{And Aaron shall present the goat upon which the lot fell for the \lord, and offer him for a sin-offering.}{\arabic{verse}}
\rashi{\rashiDH{ועשהו חטאת.} כשמניח הגורל עליו קורא לו שם ואומר לה׳ חטאת (ת״כ שם ה)׃}
\threeverse{\arabic{verse}}%Leviticus16:10
{וְהַשָּׂעִ֗יר אֲשֶׁר֩ עָלָ֨ה עָלָ֤יו הַגּוֹרָל֙ לַעֲזָאזֵ֔ל יׇֽעֳמַד\maqqaf חַ֛י לִפְנֵ֥י יְהֹוָ֖ה לְכַפֵּ֣ר עָלָ֑יו לְשַׁלַּ֥ח אֹת֛וֹ לַעֲזָאזֵ֖ל הַמִּדְבָּֽרָה׃}
{וּצְפִירָא דִּסְלֵיק עֲלוֹהִי עַדְבָּא לַעֲזָאזֵל יִתָּקַם כִּד חַי קֳדָם יְיָ לְכַפָּרָא עֲלוֹהִי לְשַׁלָּחָא יָתֵיהּ לַעֲזָאזֵל לְמַדְבְּרָא׃}
{But the goat, on which the lot fell for Azazel, shall be set alive before the \lord, to make atonement over him, to send him away for Azazel into the wilderness.}{\arabic{verse}}
\rashi{\rashiDH{יעמד חי.} כמו יועמד חי, על ידי אחרים, ותרגומו יִתָּקַם כַּד חַי. מה תלמוד לומר, לפי שנאמר לשלח אותו לעזאזל, ואיני יודע שילוחו אם למיתה אם לחיים, לכך נאמר יעמד חי, עמידתו חי עד שישתלח, מכאן ששליחותו למיתה (שם ו)׃\quad \rashiDH{לכפר עליו.} שיתודה עליו כדכתיב וְהִתְוַדָּה עָלָיו וגו׳׃}
\threeverse{\arabic{verse}}%Leviticus16:11
{וְהִקְרִ֨יב אַהֲרֹ֜ן אֶת\maqqaf פַּ֤ר הַֽחַטָּאת֙ אֲשֶׁר\maqqaf ל֔וֹ וְכִפֶּ֥ר בַּֽעֲד֖וֹ וּבְעַ֣ד בֵּית֑וֹ וְשָׁחַ֛ט אֶת\maqqaf פַּ֥ר הַֽחַטָּ֖את אֲשֶׁר\maqqaf לֽוֹ׃}
{וִיקָרֵיב אַהֲרֹן יָת תּוֹרָא דְּחַטָּתָא דִּילֵיהּ וִיכַפַּר עֲלוֹהִי וְעַל אֱנָשׁ בֵּיתֵיהּ וְיִכּוֹס יָת תּוֹרָא דְּחַטָּתָא דִּילֵיהּ׃}
{And Aaron shall present the bullock of the sin-offering, which is for himself, and shall make atonement for himself, and for his house, and shall kill the bullock of the sin-offering which is for himself.}{\arabic{verse}}
\rashi{\rashiDH{וכפר בעדו וגו׳.} וידוי שני עליו ועל אחיו הכהנים, שהם כלם קרוים ביתו (שם פרשתא ג, א), שנאמר בֵּית אַהֲרֹן בָּרְכוּ אֶת ה׳ וגו׳ (תהלים קלה, יט), מכאן שהכהנים מתכפרים בו, וכל כפרתן אינה אלא על טומאת מקדש וקדשיו, כמו שנאמר, וְכִפֶּר עַל הַקֹּדֶשׁ מִטֻּמְאֹת וגו׳ (ת״כ פרק ד, ב)׃}
\threeverse{\aliya{ישראל}}%Leviticus16:12
{וְלָקַ֣ח מְלֹֽא\maqqaf הַ֠מַּחְתָּ֠ה גַּֽחֲלֵי\maqqaf אֵ֞שׁ מֵעַ֤ל הַמִּזְבֵּ֙חַ֙ מִלִּפְנֵ֣י יְהֹוָ֔ה וּמְלֹ֣א חׇפְנָ֔יו קְטֹ֥רֶת סַמִּ֖ים דַּקָּ֑ה וְהֵבִ֖יא מִבֵּ֥ית לַפָּרֹֽכֶת׃}
{וְיִסַּב מְלֵי מַחְתִּיתָא גּוּמְרִין דְּאִישָׁא מֵעִלָּוֵי מַדְבְּחָא מִן קֳדָם יְיָ וּמְלֵי חוּפְנוֹהִי קְטֹרֶת בּוּסְמִין דַּקִיקִין וְיַעֵיל מִגָּיו לְפָרוּכְתָּא׃}
{And he shall take a censer full of coals of fire from off the altar before the \lord, and his hands full of sweet incense beaten small, and bring it within the veil.}{\arabic{verse}}
\rashi{\rashiDH{מעל המזבח.} החיצון (יומא מה׃)׃\quad \rashiDH{מלפני ה׳.} מצד שלפני הפתח, והוא צד מערבי (שם)׃\quad \rashiDH{דקה.} מה ת״ל דקה, והלא כל הקטורת דקה היא, שנאמר וְשָׁחַקְתָּ מִמֶּנָּה הָדֵק (שמות ל, לו), אלא שתהא דקה מן הדקה (יומא מג׃), שמערב יום הכפורים היה מחזירה למכתשת׃}
\threeverse{\arabic{verse}}%Leviticus16:13
{וְנָתַ֧ן אֶֽת\maqqaf הַקְּטֹ֛רֶת עַל\maqqaf הָאֵ֖שׁ לִפְנֵ֣י יְהֹוָ֑ה וְכִסָּ֣ה \legarmeh  עֲנַ֣ן הַקְּטֹ֗רֶת אֶת\maqqaf הַכַּפֹּ֛רֶת אֲשֶׁ֥ר עַל\maqqaf הָעֵד֖וּת וְלֹ֥א יָמֽוּת׃}
{וְיִתֵּין יָת קְטֹרֶת בּוּסְמַיָּא עַל אִישָׁתָא קֳדָם יְיָ וְיִחְפֵי עֲנַן קָטוּרְתָּא יָת כָּפוּרְתָּא דְּעַל סָהֲדוּתָא וְלָא יְמוּת׃}
{And he shall put the incense upon the fire before the \lord, that the cloud of the incense may cover the ark-cover that is upon the testimony, that he die not.}{\arabic{verse}}
\rashi{\rashiDH{על האש.} שבתוך המחתה׃\quad \rashiDH{ולא ימות.} הא אם לא עשאה כתקנה חייב מיתה (יומא נג.)׃ 
}
\threeverse{\arabic{verse}}%Leviticus16:14
{וְלָקַח֙ מִדַּ֣ם הַפָּ֔ר וְהִזָּ֧ה בְאֶצְבָּע֛וֹ עַל\maqqaf פְּנֵ֥י הַכַּפֹּ֖רֶת קֵ֑דְמָה וְלִפְנֵ֣י הַכַּפֹּ֗רֶת יַזֶּ֧ה שֶֽׁבַע\maqqaf פְּעָמִ֛ים מִן\maqqaf הַדָּ֖ם בְּאֶצְבָּעֽוֹ׃}
{וְיִסַּב מִדְּמָא דְּתוֹרָא וְיַדֵּי בְאֶצְבְּעֵיהּ עַל אַפֵּי כָּפוּרְתָּא קִדּוּמָא וּקְדָם כָּפוּרְתָּא יַדֵּי שְׁבַע זִמְנִין מִן דְּמָא בְאֶצְבְּעֵיהּ׃}
{And he shall take of the blood of the bullock, and sprinkle it with his finger upon the ark-cover on the east; and before the ark-cover shall he sprinkle of the blood with his finger seven times.}{\arabic{verse}}
\rashi{\rashiDH{והזה באצבעו.} הזאה אחת במשמע׃\quad \rashiDH{ולפני הכפרת יזה שבע.} הרי אחת למעלה ושבע למטה (יומא נה.)׃}
\threeverse{\arabic{verse}}%Leviticus16:15
{וְשָׁחַ֞ט אֶת\maqqaf שְׂעִ֤יר הַֽחַטָּאת֙ אֲשֶׁ֣ר לָעָ֔ם וְהֵבִיא֙ אֶת\maqqaf דָּמ֔וֹ אֶל\maqqaf מִבֵּ֖ית לַפָּרֹ֑כֶת וְעָשָׂ֣ה אֶת\maqqaf דָּמ֗וֹ כַּאֲשֶׁ֤ר עָשָׂה֙ לְדַ֣ם הַפָּ֔ר וְהִזָּ֥ה אֹת֛וֹ עַל\maqqaf הַכַּפֹּ֖רֶת וְלִפְנֵ֥י הַכַּפֹּֽרֶת׃}
{וְיִכּוֹס יָת צְפִירָא דְּחַטָּתָא דִּלְעַמָּא וְיַעֵיל יָת דְּמֵיהּ לְמִגָּיו לְפָרוּכְתָּא וְיַעֲבֵיד לִדְמֵיהּ כְּמָא דַּעֲבַד לִדְמָא דְּתוֹרָא וְיַדֵּי יָתֵיהּ עַל כָּפוּרְתָּא וּקְדָם כָּפוּרְתָּא׃}
{Then shall he kill the goat of the sin-offering, that is for the people, and bring his blood within the veil, and do with his blood as he did with the blood of the bullock, and sprinkle it upon the ark-cover, and before the ark-cover.}{\arabic{verse}}
\rashi{\rashiDH{אשר לעם.} מה שהפר מכפר על הכהנים מכפר השעיר על ישראל, והוא השעיר שעלה עליו הגורל לשם (שם סא.)׃\quad \rashiDH{כאשר עשה לדם הפר.} אחת למעלה ושבע למטה (שם נה.)׃}
\threeverse{\arabic{verse}}%Leviticus16:16
{וְכִפֶּ֣ר עַל\maqqaf הַקֹּ֗דֶשׁ מִטֻּמְאֹת֙ בְּנֵ֣י יִשְׂרָאֵ֔ל וּמִפִּשְׁעֵיהֶ֖ם לְכׇל\maqqaf חַטֹּאתָ֑ם וְכֵ֤ן יַעֲשֶׂה֙ לְאֹ֣הֶל מוֹעֵ֔ד הַשֹּׁכֵ֣ן אִתָּ֔ם בְּת֖וֹךְ טֻמְאֹתָֽם׃}
{וִיכַפַּר עַל קוּדְשָׁא מִסּוֹאֲבָת בְּנֵי יִשְׂרָאֵל וּמִמִּרְדֵּיהוֹן לְכָל חֲטָאֵיהוֹן וְכֵן יַעֲבֵיד לְמַשְׁכַּן זִמְנָא דְּשָׁרֵי עִמְּהוֹן בְּגוֹ סוֹאֲבָתְהוֹן׃}
{And he shall make atonement for the holy place, because of the uncleannesses of the children of Israel, and because of their transgressions, even all their sins; and so shall he do for the tent of meeting, that dwelleth with them in the midst of their uncleannesses.}{\arabic{verse}}
\rashi{\rashiDH{מטמאת בני ישראל.} על הנכנסין למקדש בטומאה ולא נודע להם בסוף, שנאמר לכל חטאתם, וחטאת היא שוגג (שבועות יז׃  ת״כ פרק ד, ב)׃\quad \rashiDH{ומפשעיהם.} אף הנכנסין מזיד בטומאה (שם ג.)׃\quad \rashiDH{וכן יעשה לאהל מועד.} כשם שהזה משניהם בפנים, אחת למעלה ושבע למטה, כך מזה על הפרוכת מבחוץ משניהם, אחת למעלה ושבע למטה (יומא נו׃)׃\quad \rashiDH{השכן אתם בתוך טומאותם.} אף על פי שהם טמאים, שכינה ביניהם׃}
\threeverse{\arabic{verse}}%Leviticus16:17
{וְכׇל\maqqaf אָדָ֞ם לֹא\maqqaf יִהְיֶ֣ה \legarmeh  בְּאֹ֣הֶל מוֹעֵ֗ד בְּבֹא֛וֹ לְכַפֵּ֥ר בַּקֹּ֖דֶשׁ עַד\maqqaf צֵאת֑וֹ וְכִפֶּ֤ר בַּעֲדוֹ֙ וּבְעַ֣ד בֵּית֔וֹ וּבְעַ֖ד כׇּל\maqqaf קְהַ֥ל יִשְׂרָאֵֽל׃}
{וְכָל אֱנָשׁ לָא יְהֵי בְּמַשְׁכַּן זִמְנָא בְּמֵיעֲלֵיהּ לְכַפָּרָא בְּקוּדְשָׁא עַד מִפְּקֵיהּ וִיכַפַּר עֲלוֹהִי וְעַל אֱנָשׁ בֵּיתֵיהּ וְעַל כָּל קְהָלָא דְּיִשְׂרָאֵל׃}
{And there shall be no man in the tent of meeting when he goeth in to make atonement in the holy place, until he come out, and have made atonement for himself, and for his household, and for all the assembly of Israel.}{\arabic{verse}}
\aliyacounter{שני}
\threeverse{\aliya{שני}}%Leviticus16:18
{וְיָצָ֗א אֶל\maqqaf הַמִּזְבֵּ֛חַ אֲשֶׁ֥ר לִפְנֵֽי\maqqaf יְהֹוָ֖ה וְכִפֶּ֣ר עָלָ֑יו וְלָקַ֞ח מִדַּ֤ם הַפָּר֙ וּמִדַּ֣ם הַשָּׂעִ֔יר וְנָתַ֛ן עַל\maqqaf קַרְנ֥וֹת הַמִּזְבֵּ֖חַ סָבִֽיב׃}
{וְיִפּוֹק לְמַדְבְּחָא דִּקְדָם יְיָ וִיכַפַּר עֲלוֹהִי וְיִסַּב מִדְּמָא דְּתוֹרָא וּמִדְּמָא דִּצְפִירָא וְיִתֵּין עַל קַרְנָת מַדְבְּחָא סְחוֹר סְחוֹר׃}
{And he shall go out unto the altar that is before the \lord, and make atonement for it; and shall take of the blood of the bullock, and of the blood of the goat, and put it upon the horns of the altar round about.}{\arabic{verse}}
\rashi{\rashiDH{אל המזבח אשר לפני ה׳.} זה מזבח הזהב, שהוא לפני ה׳ בהיכל. ומה תלמוד לומר ויצא, לפי שהזה ההזאות על הפרוכת, ועמד מן המזבח ולפנים והזה, ובמתנות המזבח הזקיקו לצאת מן המזבח ולחוץ, ויתחיל מקרן מזרחית צפונית (יומא נח׃)׃\quad \rashiDH{וכפר עליו.} ומה היא כפרתו, ולקח מדם הפר ומדם השעיר, מעורבין זה לתוך זה (יומא נז׃)׃}
\threeverse{\arabic{verse}}%Leviticus16:19
{וְהִזָּ֨ה עָלָ֧יו מִן\maqqaf הַדָּ֛ם בְּאֶצְבָּע֖וֹ שֶׁ֣בַע פְּעָמִ֑ים וְטִהֲר֣וֹ וְקִדְּשׁ֔וֹ מִטֻּמְאֹ֖ת בְּנֵ֥י יִשְׂרָאֵֽל׃}
{וְיַדֵּי עֲלוֹהִי מִן דְּמָא בְּאֶצְבְּעֵיהּ שְׁבַע זִמְנִין וִידַכֵּינֵיהּ וִיקַדְּשִׁנֵּיהּ מִסּוֹאֲבָת בְּנֵי יִשְׂרָאֵל׃}
{And he shall sprinkle of the blood upon it with his finger seven times, and cleanse it, and hallow it from the uncleannesses of the children of Israel.}{\arabic{verse}}
\rashi{\rashiDH{והזה עליו מן הדם.} אחר שנתן מתנות באצבעו על קרנותיו, מַזֶּה ז׳ הזאות על גגו׃\quad \rashiDH{וטהרו.} ממה שעבר׃\quad \rashiDH{וקדשו.} לעתיד לבא (ת״כ פרק ד, יג)׃}
\threeverse{\arabic{verse}}%Leviticus16:20
{וְכִלָּה֙ מִכַּפֵּ֣ר אֶת\maqqaf הַקֹּ֔דֶשׁ וְאֶת\maqqaf אֹ֥הֶל מוֹעֵ֖ד וְאֶת\maqqaf הַמִּזְבֵּ֑חַ וְהִקְרִ֖יב אֶת\maqqaf הַשָּׂעִ֥יר הֶחָֽי׃}
{וִישֵׁיצֵי מִלְּכַפָּרָא עַל קוּדְשָׁא וְעַל מַשְׁכַּן זִמְנָא וְעַל מַדְבְּחָא וִיקָרֵיב יָת צְפִירָא חַיָּא׃}
{And when he hath made an end of atoning for the holy place, and the tent of meeting, and the altar, he shall present the live goat.}{\arabic{verse}}
\threeverse{\arabic{verse}}%Leviticus16:21
{וְסָמַ֨ךְ אַהֲרֹ֜ן אֶת\maqqaf שְׁתֵּ֣י יָדָ֗ו עַ֣ל רֹ֣אשׁ הַשָּׂעִיר֮ הַחַי֒ וְהִתְוַדָּ֣ה עָלָ֗יו אֶת\maqqaf כׇּל\maqqaf עֲוֺנֹת֙ בְּנֵ֣י יִשְׂרָאֵ֔ל וְאֶת\maqqaf כׇּל\maqqaf פִּשְׁעֵיהֶ֖ם לְכׇל\maqqaf חַטֹּאתָ֑ם וְנָתַ֤ן אֹתָם֙ עַל\maqqaf רֹ֣אשׁ הַשָּׂעִ֔יר וְשִׁלַּ֛ח בְּיַד\maqqaf אִ֥ישׁ עִתִּ֖י הַמִּדְבָּֽרָה׃}
{וְיִסְמוֹךְ אַהֲרֹן יָת תַּרְתֵּין יְדוֹהִי עַל רֵישׁ צְפִירָא חַיָּא וִיוַדֵּי עֲלוֹהִי יָת כָּל עֲוָיָת בְּנֵי יִשְׂרָאֵל וְיָת כָּל מִרְדֵּיהוֹן לְכָל חֲטָאֵיהוֹן וְיִתֵּין יָתְהוֹן עַל רֵישׁ צְפִירָא וִישַׁלַּח בְּיַד גְּבַר דִּזְמִין לִמְהָךְ לְמַדְבְּרָא׃}
{And Aaron shall lay both his hands upon the head of the live goat, and confess over him all the iniquities of the children of Israel, and all their transgressions, even all their sins; and he shall put them upon the head of the goat, and shall send him away by the hand of an appointed man into the wilderness.}{\arabic{verse}}
\rashi{\rashiDH{איש עתי.} המוכן לכך מיום אתמול (יומא סו׃)׃}
\threeverse{\arabic{verse}}%Leviticus16:22
{וְנָשָׂ֨א הַשָּׂעִ֥יר עָלָ֛יו אֶת\maqqaf כׇּל\maqqaf עֲוֺנֹתָ֖ם אֶל\maqqaf אֶ֣רֶץ גְּזֵרָ֑ה וְשִׁלַּ֥ח אֶת\maqqaf הַשָּׂעִ֖יר בַּמִּדְבָּֽר׃}
{וְיִטּוֹל צְפִירָא עֲלוֹהִי יָת כָּל חוֹבֵיהוֹן לַאֲרַע דְּלָא יָתְבָא וִישַׁלַּח יָת צְפִירָא בְּמַדְבְּרָא׃}
{And the goat shall bear upon him all their iniquities unto a land which is cut off; and he shall let go the goat in the wilderness.}{\arabic{verse}}
\threeverse{\arabic{verse}}%Leviticus16:23
{וּבָ֤א אַהֲרֹן֙ אֶל\maqqaf אֹ֣הֶל מוֹעֵ֔ד וּפָשַׁט֙ אֶת\maqqaf בִּגְדֵ֣י הַבָּ֔ד אֲשֶׁ֥ר לָבַ֖שׁ בְּבֹא֣וֹ אֶל\maqqaf הַקֹּ֑דֶשׁ וְהִנִּיחָ֖ם שָֽׁם׃}
{וְיֵיעוֹל אַהֲרֹן לְמַשְׁכַּן זִמְנָא וְיַשְׁלַח יָת לְבוּשֵׁי בוּצָא דִּלְבַשׁ בְּמֵיעֲלֵיהּ לְקוּדְשָׁא וְיַצְנְעִנּוּן תַּמָּן׃}
{And Aaron shall come into the tent of meeting, and shall put off the linen garments, which he put on when he went into the holy place, and shall leave them there.}{\arabic{verse}}
\rashi{\rashiDH{ובא אהרן אל אהל מועד.} אמרו רבותינו, שאין זה מקומו של מקרא זה, ונתנו טעם לדבריהם במסכת יומא (דף לב.), ואמרו כל הפרשה כולה אמור על הסדר חוץ מביאה זו, שהיא אחר עשיית עולתו ועולת העם והקטרת אימורי פר ושעיר שנעשים בחוץ בבגדי זהב, וטובל ומקדש ופושטן ולובש בגדי לבן. \rashiDH{ובא אל אהל מועד.} להוציא את הכף ואת המחתה שהקטיר בה הקטרת לפני ולפנים׃\quad \rashiDH{ופשט את בגדי הבד.} אחר שהוציאם ולובש בגדי זהב לתמיד של בין הערבים. וזהו סדר העבודות, תמיד של שחר בבגדי זהב, ועבודת פר ושעיר הפנימים וקטרת של מחתה בבגדי לבן, ואילו ואיל העם ומקצת המוספין בבגדי זהב, והוצאת כף ומחתה בבגדי לבן, ושירי המוספין ותמיד של בין הערבים וקטורת ההיכל שעל מזבח הפנימי בבגדי זהב, וסדר המקראות לפי סדר העבודות כך הוא, וְשִׁלַּח אֶת הַשָּׂעִיר בַּמִּדְבָּר, וְרָחַץ אֶת בְּשָׂרוֹ בַמַּיִם וגו׳, וְיָצָא וְעָשָׂה אֶת עֹלָתוֹ וגו׳, וְאֶת חֵלֶב הַחַטָּאת וגו׳, וכל הפרשה עד וְאַחֲרֵי כֵן יָבוֹא אֶל הַמַּחֲנֶה, ואחר כך וּבָא אַהֲרֹן׃\quad \rashiDH{והניחם שם.} מלמד שטעונין גניזה, ולא ישתמש באותן ארבעה בגדים ליום כפורים אחר (יומא יב׃)׃}
\threeverse{\arabic{verse}}%Leviticus16:24
{וְרָחַ֨ץ אֶת\maqqaf בְּשָׂר֤וֹ בַמַּ֙יִם֙ בְּמָק֣וֹם קָד֔וֹשׁ וְלָבַ֖שׁ אֶת\maqqaf בְּגָדָ֑יו וְיָצָ֗א וְעָשָׂ֤ה אֶת\maqqaf עֹֽלָתוֹ֙ וְאֶת\maqqaf עֹלַ֣ת הָעָ֔ם וְכִפֶּ֥ר בַּעֲד֖וֹ וּבְעַ֥ד הָעָֽם׃}
{וְיַסְחֵי יָת בִּשְׂרֵיהּ בְּמַיָּא בַּאֲתַר קַדִּישׁ וְיִלְבַּשׁ יָת לְבוּשׁוֹהִי וְיִפּוֹק וְיַעֲבֵיד יָת עֲלָתֵיהּ וְיָת עֲלַת עַמָּא וִיכַפַּר עֲלוֹהִי וְעַל עַמָּא׃}
{And he shall bathe his flesh in water in a holy place and put on his other vestments, and come forth, and offer his burnt-offering and the burnt-offering of the people, and make atonement for himself and for the people.}{\arabic{verse}}
\rashi{\rashiDH{ורחץ את בשרו וגו׳.} למעלה למדנו מורחץ את בשרו ולבשם שכשהוא מְשַׁנֶּה מבגדי זהב לבגדי לבן טעון טבילה, שבאותה טבילה פשט בגדי זהב שעבד בהן עבודת תמיד של שחר ולבש בגדי לבן לעבודת היום וכאן למדנו שכשהוא מְשַׁנֶּה מבגדי לבן לבגדי זהב טעון טבילה (שם לב.)׃\quad \rashiDH{במקום קדוש.} המקודש בקדושת עזרה, והיא היתה בגג בית הַפַּרְוָה, וכן ד׳ טבילות הבאות חובה ליום, אבל הראשונה היתה בחיל (שם ל.  ת״כ פרק י, ח)׃\quad \rashiDH{ולבש את בגדיו.} שמנה בגדים שהוא עובד בהן כל ימות השנה׃\quad \rashiDH{ויצא.} מן ההיכל אל החצר שמזבח העולה שם׃\quad \rashiDH{ועשה את עולתו.} איל לעולה האמור למעלה בְּזֹאת יָבֹא אַהֲרֹן וגו׳׃\quad \rashiDH{ואת עולת העם.} ואיל לעולה האמור למעלה וּמֵאֵת עֲדַת בְּנֵי יִשְׂרָאֵל וגו׳׃}
\aliyacounter{שלישי}
\threeverse{\aliya{שלישי\newline (שני)}}%Leviticus16:25
{וְאֵ֛ת חֵ֥לֶב הַֽחַטָּ֖את יַקְטִ֥יר הַמִּזְבֵּֽחָה׃}
{וְיָת תַּרְבֵּי חֲטָוָותָא יַסֵּיק לְמַדְבְּחָא׃}
{And the fat of the sin-offering shall he make smoke upon the altar.}{\arabic{verse}}
\rashi{\rashiDH{ואת חלב החטאת.} אימורי פר ושעיר׃\quad \rashiDH{יקטיר המזבחה.} על מזבח החיצון, דאלו בפנימי כתיב לֹא תַעֲלוּ עָלָיו קְטֹרֶת זָרָה וְעֹלָה וּמִנְחָה (שמות ל.)׃}
\threeverse{\arabic{verse}}%Leviticus16:26
{וְהַֽמְשַׁלֵּ֤חַ אֶת\maqqaf הַשָּׂעִיר֙ לַֽעֲזָאזֵ֔ל יְכַבֵּ֣ס בְּגָדָ֔יו וְרָחַ֥ץ אֶת\maqqaf בְּשָׂר֖וֹ בַּמָּ֑יִם וְאַחֲרֵי\maqqaf כֵ֖ן יָב֥וֹא אֶל\maqqaf הַֽמַּחֲנֶֽה׃}
{וּדְמוֹבֵיל יָת צְפִירָא לַעֲזָאזֵל יְצַבַּע לְבוּשׁוֹהִי וְיַסְחֵי יָת בִּשְׂרֵיהּ בְּמַיָּא וּבָתַר כֵּן יֵיעוֹל לְמַשְׁרִיתָא׃}
{And he that letteth go the goat for Azazel shall wash his clothes, and bathe his flesh in water, and afterward he may come into the camp.}{\arabic{verse}}
\threeverse{\arabic{verse}}%Leviticus16:27
{וְאֵת֩ פַּ֨ר הַֽחַטָּ֜את וְאֵ֣ת \legarmeh  שְׂעִ֣יר הַֽחַטָּ֗את אֲשֶׁ֨ר הוּבָ֤א אֶת\maqqaf דָּמָם֙ לְכַפֵּ֣ר בַּקֹּ֔דֶשׁ יוֹצִ֖יא אֶל\maqqaf מִח֣וּץ לַֽמַּחֲנֶ֑ה וְשָׂרְפ֣וּ בָאֵ֔שׁ אֶת\maqqaf עֹרֹתָ֥ם וְאֶת\maqqaf בְּשָׂרָ֖ם וְאֶת\maqqaf פִּרְשָֽׁם׃}
{וְיָת תּוֹרָא דְּחַטָּתָא וְיָת צְפִירָא דְּחַטָּתָא דְּאִתָּעַל מִדַּמְהוֹן לְכַפָּרָא בְּקוּדְשָׁא יִתַּפְקוּן לְמִבַּרָא לְמַשְׁרִיתָא וְיוֹקְדוּן בְּנוּרָא יָת מַשְׁכְּהוֹן וְיָת בִּשְׂרְהוֹן וְיָת אוּכְלְהוֹן׃}
{And the bullock of the sin-offering, and the goat of the sin-offering, whose blood was brought in to make atonement in the holy place, shall be carried forth without the camp; and they shall burn in the fire their skins, and their flesh, and their dung.}{\arabic{verse}}
\rashi{\rashiDH{אשר הובא את דמם.} להיכל לִפְנַי ולפנים׃}
\threeverse{\arabic{verse}}%Leviticus16:28
{וְהַשֹּׂרֵ֣ף אֹתָ֔ם יְכַבֵּ֣ס בְּגָדָ֔יו וְרָחַ֥ץ אֶת\maqqaf בְּשָׂר֖וֹ בַּמָּ֑יִם וְאַחֲרֵי\maqqaf כֵ֖ן יָב֥וֹא אֶל\maqqaf הַֽמַּחֲנֶֽה׃}
{וּדְמוֹקֵיד יָתְהוֹן יְצַבַּע לְבוּשׁוֹהִי וְיַסְחֵי יָת בִּשְׂרֵיהּ בְּמַיָּא וּבָתַר כֵּן יֵיעוֹל לְמַשְׁרִיתָא׃}
{And he that burneth them shall wash his clothes, and bathe his flesh in water, and afterward he may come into the camp.}{\arabic{verse}}
\threeverse{\arabic{verse}}%Leviticus16:29
{וְהָיְתָ֥ה לָכֶ֖ם לְחֻקַּ֣ת עוֹלָ֑ם בַּחֹ֣דֶשׁ הַ֠שְּׁבִיעִ֠י בֶּֽעָשׂ֨וֹר לַחֹ֜דֶשׁ תְּעַנּ֣וּ אֶת\maqqaf נַפְשֹֽׁתֵיכֶ֗ם וְכׇל\maqqaf מְלָאכָה֙ לֹ֣א תַעֲשׂ֔וּ הָֽאֶזְרָ֔ח וְהַגֵּ֖ר הַגָּ֥ר בְּתוֹכְכֶֽם׃}
{וּתְהֵי לְכוֹן לִקְיָם עָלַם בְּיַרְחָא שְׁבִיעָאָה בְּעַשְׂרָא לְיַרְחָא תְּעַנּוֹן יָת נַפְשָׁתְכוֹן וְכָל עֲבִידָא לָא תַעְבְּדוּן יַצִּיבַיָּא וְגִיּוֹרַיָּא דְּיִתְגַּיְרוּן בֵּינֵיכוֹן׃}
{And it shall be a statute for ever unto you: in the seventh month, on the tenth day of the month, ye shall afflict your souls, and shall do no manner of work, the home-born, or the stranger that sojourneth among you.}{\arabic{verse}}
\threeverse{\arabic{verse}}%Leviticus16:30
{כִּֽי\maqqaf בַיּ֥וֹם הַזֶּ֛ה יְכַפֵּ֥ר עֲלֵיכֶ֖ם לְטַהֵ֣ר אֶתְכֶ֑ם מִכֹּל֙ חַטֹּ֣אתֵיכֶ֔ם לִפְנֵ֥י יְהֹוָ֖ה תִּטְהָֽרוּ׃}
{אֲרֵי בְּיוֹמָא הָדֵין יְכַפַּר עֲלֵיכוֹן לְדַכָּאָה יָתְכוֹן מִכֹּל חוֹבֵיכוֹן קֳדָם יְיָ תִּדְכּוֹן׃}
{For on this day shall atonement be made for you, to cleanse you; from all your sins shall ye be clean before the \lord.}{\arabic{verse}}
\threeverse{\arabic{verse}}%Leviticus16:31
{שַׁבַּ֨ת שַׁבָּת֥וֹן הִיא֙ לָכֶ֔ם וְעִנִּיתֶ֖ם אֶת\maqqaf נַפְשֹׁתֵיכֶ֑ם חֻקַּ֖ת עוֹלָֽם׃}
{שַׁבָּא שַׁבָּתָא הִיא לְכוֹן וּתְעַנּוֹן יָת נַפְשָׁתְכוֹן קְיָם עָלַם׃}
{It is a sabbath of solemn rest unto you, and ye shall afflict your souls; it is a statute for ever.}{\arabic{verse}}
\threeverse{\arabic{verse}}%Leviticus16:32
{וְכִפֶּ֨ר הַכֹּהֵ֜ן אֲשֶׁר\maqqaf יִמְשַׁ֣ח אֹת֗וֹ וַאֲשֶׁ֤ר יְמַלֵּא֙ אֶת\maqqaf יָד֔וֹ לְכַהֵ֖ן תַּ֣חַת אָבִ֑יו וְלָבַ֛שׁ אֶת\maqqaf בִּגְדֵ֥י הַבָּ֖ד בִּגְדֵ֥י הַקֹּֽדֶשׁ׃}
{וִיכַפַּר כָּהֲנָא דִּירַבֵּי יָתֵיהּ וְדִיקָרֵיב יָת קוּרְבָּנֵיהּ לְשַׁמָּשָׁא תְּחוֹת אֲבוּהִי וְיִלְבַּשׁ יָת לְבוּשֵׁי בוּצָא לְבוּשֵׁי קוּדְשָׁא׃}
{And the priest, who shall be anointed and who shall be consecrated to be priest in his father’s stead, shall make the atonement, and shall put on the linen garments, even the holy garments.}{\arabic{verse}}
\rashi{\rashiDH{וכפר הכהן אשר ימשח וגו׳.} כפרה זו של יום הכיפורים אינה כשרה אלא בכהן גדול (יומא עג.  ת״כ פרק ח, ד), לפי שנאמרה כל הפרשה באהרן הוצרך לומר בכהן גדול הבא אחריו שיהא כמוהו׃\quad \rashiDH{ואשר ימלא את ידו.} אין לי אלא המשוח בשמן המשחה, מרובה בגדים מנין, תלמוד לומר ואשר ימלא את ידו וגו׳ (ת״כ שם), והם כל הכהנים גדולים שעמדו מיאשיהו ואילך, שבימיו נגנזה צלוחית של שמן המשחה (יומא נב׃)׃\quad \rashiDH{לכהן תחת אביו.} ללמד שאם בנו ממלא את מקומו, הוא קודם לכל אדם (ת״כ שם ה)׃}
\threeverse{\arabic{verse}}%Leviticus16:33
{וְכִפֶּר֙ אֶת\maqqaf מִקְדַּ֣שׁ הַקֹּ֔דֶשׁ וְאֶת\maqqaf אֹ֧הֶל מוֹעֵ֛ד וְאֶת\maqqaf הַמִּזְבֵּ֖חַ יְכַפֵּ֑ר וְעַ֧ל הַכֹּהֲנִ֛ים וְעַל\maqqaf כׇּל\maqqaf עַ֥ם הַקָּהָ֖ל יְכַפֵּֽר׃}
{וִיכַפַּר עַל מַקְדַּשׁ קוּדְשָׁא וְעַל מַשְׁכַּן זִמְנָא וְעַל מַדְבְּחָא יְכַפַּר וְעַל כָּהֲנַיָּא וְעַל כָּל עַמָּא דִּקְהָלָא יְכַפַּר׃}
{And he shall make atonement for the most holy place, and he shall make atonement for the tent of meeting and for the altar; and he shall make atonement for the priests and for all the people of the assembly.}{\arabic{verse}}
\threeverse{\arabic{verse}}%Leviticus16:34
{וְהָֽיְתָה\maqqaf זֹּ֨את לָכֶ֜ם לְחֻקַּ֣ת עוֹלָ֗ם לְכַפֵּ֞ר עַל\maqqaf בְּנֵ֤י יִשְׂרָאֵל֙ מִכׇּל\maqqaf חַטֹּאתָ֔ם אַחַ֖ת בַּשָּׁנָ֑ה וַיַּ֕עַשׂ כַּאֲשֶׁ֛ר צִוָּ֥ה יְהֹוָ֖ה אֶת\maqqaf מֹשֶֽׁה׃ \petucha }
{וּתְהֵי דָּא לְכוֹן לִקְיָם עָלַם לְכַפָּרָא עַל בְּנֵי יִשְׂרָאֵל מִכָּל חוֹבֵיהוֹן חֲדָא בְּשַׁתָּא וַעֲבַד כְּמָא דְּפַקֵּיד יְיָ יָת מֹשֶׁה׃}
{And this shall be an everlasting statute unto you, to make atonement for the children of Israel because of all their sins once in the year.’ And he did as the \lord\space commanded Moses.}{\arabic{verse}}
\rashi{\rashiDH{ויעש כאשר צוה ה׳ וגו׳.} כשהגיע יום הכפורים עשה כסדר הזה, ולהגיד שבחו של אהרן שלא היה לובשן לגדולתו אלא כמקיים גזירת המלך׃}
\newperek
\aliyacounter{רביעי}
\newseder{13}
\threeverse{\aliya{רביעי}\newline\vspace{-4pt}\newline\seder{יג}}%Leviticus17:1
{וַיְדַבֵּ֥ר יְהֹוָ֖ה אֶל\maqqaf מֹשֶׁ֥ה לֵּאמֹֽר׃}
{וּמַלֵּיל יְיָ עִם מֹשֶׁה לְמֵימַר׃}
{And the \lord\space spoke unto Moses, saying:}{\Roman{chap}}
\threeverse{\arabic{verse}}%Leviticus17:2
{דַּבֵּ֨ר אֶֽל\maqqaf אַהֲרֹ֜ן וְאֶל\maqqaf בָּנָ֗יו וְאֶל֙ כׇּל\maqqaf בְּנֵ֣י יִשְׂרָאֵ֔ל וְאָמַרְתָּ֖ אֲלֵיהֶ֑ם זֶ֣ה הַדָּבָ֔ר אֲשֶׁר\maqqaf צִוָּ֥ה יְהֹוָ֖ה לֵאמֹֽר׃}
{מַלֵּיל עִם אַהֲרֹן וְעִם בְּנוֹהִי וְעִם כָּל בְּנֵי יִשְׂרָאֵל וְתֵימַר לְהוֹן דֵּין פִּתְגָמָא דְּפַקֵּיד יְיָ לְמֵימַר׃}
{Speak unto Aaron, and unto his sons, and unto all the children of Israel, and say unto them: This is the thing which the \lord\space hath commanded, saying:}{\arabic{verse}}
\threeverse{\arabic{verse}}%Leviticus17:3
{אִ֥ישׁ אִישׁ֙ מִבֵּ֣ית יִשְׂרָאֵ֔ל אֲשֶׁ֨ר יִשְׁחַ֜ט שׁ֥וֹר אוֹ\maqqaf כֶ֛שֶׂב אוֹ\maqqaf עֵ֖ז בַּֽמַּחֲנֶ֑ה א֚וֹ אֲשֶׁ֣ר יִשְׁחַ֔ט מִח֖וּץ לַֽמַּחֲנֶֽה׃}
{גְּבַר גְּבַר מִבֵּית יִשְׂרָאֵל דְּיִכּוֹס תּוֹר אוֹ אִמַּר אוֹ עֵז בְּמַשְׁרִיתָא אוֹ דְּיִכּוֹס מִבַּרָא לְמַשְׁרִיתָא׃}
{What man soever there be of the house of Israel, that killeth an ox, or lamb, or goat, in the camp, or that killeth it without the camp,}{\arabic{verse}}
\rashi{\rashiDH{אשר ישחט שור או כשב.} במוקדשין הכתוב מדבר, שנאמר להקריב קרבן׃ 
\quad \rashiDH{במחנה.} חוץ לעזרה (זבחים קז׃)׃ 
}
\threeverse{\arabic{verse}}%Leviticus17:4
{וְאֶל\maqqaf פֶּ֜תַח אֹ֣הֶל מוֹעֵד֮ לֹ֣א הֱבִיאוֹ֒ לְהַקְרִ֤יב קׇרְבָּן֙ לַֽיהֹוָ֔ה לִפְנֵ֖י מִשְׁכַּ֣ן יְהֹוָ֑ה דָּ֣ם יֵחָשֵׁ֞ב לָאִ֤ישׁ הַהוּא֙ דָּ֣ם שָׁפָ֔ךְ וְנִכְרַ֛ת הָאִ֥ישׁ הַה֖וּא מִקֶּ֥רֶב עַמּֽוֹ׃}
{וְלִתְרַע מַשְׁכַּן זִמְנָא לָא אַיְתְיֵהּ לְקָרָבָא קוּרְבָּנָא קֳדָם יְיָ קֳדָם מַשְׁכְּנָא דַּייָ דְּמָא יִתְחֲשֵׁיב לְגוּבְרָא הַהוּא דְּמָא אֲשַׁד וְיִשְׁתֵּיצֵי אֲנָשָׁא הַהוּא מִגּוֹ עַמֵּיהּ׃}
{and hath not brought it unto the door of the tent of meeting, to present it as an offering unto the \lord\space before the tabernacle of the \lord, blood shall be imputed unto that man; he hath shed blood; and that man shall be cut off from among his people.}{\arabic{verse}}
\rashi{\rashiDH{דם יחשב.} כשופך דם האדם שמתחייב בנפשו׃\quad \rashiDH{דם שפך.} לרבות את הזורק דמים בחוץ (שם)׃}
\threeverse{\arabic{verse}}%Leviticus17:5
{לְמַ֩עַן֩ אֲשֶׁ֨ר יָבִ֜יאוּ בְּנֵ֣י יִשְׂרָאֵ֗ל אֶֽת\maqqaf זִבְחֵיהֶם֮ אֲשֶׁ֣ר הֵ֣ם זֹבְחִים֮ עַל\maqqaf פְּנֵ֣י הַשָּׂדֶה֒ וֶֽהֱבִיאֻ֣ם לַֽיהֹוָ֗ה אֶל\maqqaf פֶּ֛תַח אֹ֥הֶל מוֹעֵ֖ד אֶל\maqqaf הַכֹּהֵ֑ן וְזָ֨בְח֜וּ זִבְחֵ֧י שְׁלָמִ֛ים לַֽיהֹוָ֖ה אוֹתָֽם׃}
{בְּדִיל דְּיַיְתוֹן בְּנֵי יִשְׂרָאֵל יָת דִּבְחֵיהוֹן דְּאִנּוּן דָּבְחִין עַל אַפֵּי חַקְלָא וְיַיְתוּנוּן לִקְדָם יְיָ לִתְרַע מַשְׁכַּן זִמְנָא לְוָת כָּהֲנָא וְיִכְּסוּן נִכְסַת קוּדְשִׁין קֳדָם יְיָ יָתְהוֹן׃}
{To the end that the children of Israel may bring their sacrifices, which they sacrifice in the open field, even that they may bring them unto the \lord, unto the door of the tent of meeting, unto the priest, and sacrifice them for sacrifices of peace-offerings unto the \lord.}{\arabic{verse}}
\rashi{\rashiDH{אשר הם זבחים.} אשר הם רגילים לזבוח׃}
\threeverse{\arabic{verse}}%Leviticus17:6
{וְזָרַ֨ק הַכֹּהֵ֤ן אֶת\maqqaf הַדָּם֙ עַל\maqqaf מִזְבַּ֣ח יְהֹוָ֔ה פֶּ֖תַח אֹ֣הֶל מוֹעֵ֑ד וְהִקְטִ֣יר הַחֵ֔לֶב לְרֵ֥יחַ נִיחֹ֖חַ לַיהֹוָֽה׃}
{וְיִזְרוֹק כָּהֲנָא יָת דְּמָא עַל מַדְבְּחָא דַּייָ בִּתְרַע מַשְׁכַּן זִמְנָא וְיַסֵּיק תַּרְבָּא לְאִתְקַבָּלָא בְּרַעֲוָא קֳדָם יְיָ׃}
{And the priest shall dash the blood against the altar of the \lord\space at the door of the tent of meeting, and make the fat smoke for a sweet savour unto the \lord.}{\arabic{verse}}
\threeverse{\arabic{verse}}%Leviticus17:7
{וְלֹא\maqqaf יִזְבְּח֥וּ עוֹד֙ אֶת\maqqaf זִבְחֵיהֶ֔ם לַשְּׂעִירִ֕ם אֲשֶׁ֛ר הֵ֥ם זֹנִ֖ים אַחֲרֵיהֶ֑ם חֻקַּ֥ת עוֹלָ֛ם תִּֽהְיֶה\maqqaf זֹּ֥את לָהֶ֖ם לְדֹרֹתָֽם׃}
{וְלָא יְדַבְּחוּן עוֹד יָת דִּבְחֵיהוֹן לְשֵׁידִין דְּאִנּוּן טָעַן בָּתְרֵיהוֹן קְיָם עָלַם תְּהֵי דָּא לְהוֹן לְדָרֵיהוֹן׃}
{And they shall no more sacrifice their sacrifices unto the satyrs, after whom they go astray. This shall be a statute for ever unto them throughout their generations. .}{\arabic{verse}}
\rashi{\rashiDH{לשעירם.} לשדים, כמו וּשְׂעִירִים יְרַקְּדוּ שָׁם (ישעיה יג, כא)׃}
\aliyacounter{חמישי}
\threeverse{\aliya{חמישי\newline (שלישי)}}%Leviticus17:8
{וַאֲלֵהֶ֣ם תֹּאמַ֔ר אִ֥ישׁ אִישׁ֙ מִבֵּ֣ית יִשְׂרָאֵ֔ל וּמִן\maqqaf הַגֵּ֖ר אֲשֶׁר\maqqaf יָג֣וּר בְּתוֹכָ֑ם אֲשֶׁר\maqqaf יַעֲלֶ֥ה עֹלָ֖ה אוֹ\maqqaf זָֽבַח׃}
{וּלְהוֹן תֵּימַר גְּבַר גְּבַר מִבֵּית יִשְׂרָאֵל וּמִן גִּיּוֹרַיָּא דְּיִתְגַּיְּירוּן בֵּינֵיהוֹן דְּיַסֵּיק עֲלָתָא אוֹ נִכְסַת קוּדְשַׁיָּא׃}
{And thou shalt say unto them: Whatsoever man there be of the house of Israel, or of the strangers that sojourn among them, that offereth a burnt-offering or sacrifice,}{\arabic{verse}}
\rashi{\rashiDH{אשר יעלה עלה.} לחייב על המקטיר איברים בחוץ, (עי׳ ברא״ם) כשוחט בחוץ, שאם שחט אחד והעלה חבירו, שניהם חייבין (ת״כ פרק י, ו  חולין כט׃)׃}
\threeverse{\arabic{verse}}%Leviticus17:9
{וְאֶל\maqqaf פֶּ֜תַח אֹ֤הֶל מוֹעֵד֙ לֹ֣א יְבִיאֶ֔נּוּ לַעֲשׂ֥וֹת אֹת֖וֹ לַיהֹוָ֑ה וְנִכְרַ֛ת הָאִ֥ישׁ הַה֖וּא מֵעַמָּֽיו׃}
{וְלִתְרַע מַשְׁכַּן זִמְנָא לָא יַיְתֵינֵיהּ לְמֶעֱבַד יָתֵיהּ קֳדָם יְיָ וְיִשְׁתֵּיצֵי אֲנָשָׁא הַהוּא מֵעַמֵּיהּ׃}
{and bringeth it not unto the door of the tent of meeting, to sacrifice it unto the \lord, even that man shall be cut off from his people.}{\arabic{verse}}
\rashi{\rashiDH{ונכרת.} זרעו נכרת, וימיו נכרתין.}
\threeverse{\arabic{verse}}%Leviticus17:10
{וְאִ֨ישׁ אִ֜ישׁ מִבֵּ֣ית יִשְׂרָאֵ֗ל וּמִן\maqqaf הַגֵּר֙ הַגָּ֣ר בְּתוֹכָ֔ם אֲשֶׁ֥ר יֹאכַ֖ל כׇּל\maqqaf דָּ֑ם וְנָתַתִּ֣י פָנַ֗י בַּנֶּ֙פֶשׁ֙ הָאֹכֶ֣לֶת אֶת\maqqaf הַדָּ֔ם וְהִכְרַתִּ֥י אֹתָ֖הּ מִקֶּ֥רֶב עַמָּֽהּ׃}
{וּגְבַר גְּבַר מִבֵּית יִשְׂרָאֵל וּמִן גִּיּוֹרַיָּא דְּיִתְגַּיְּרוּן בֵּינֵיהוֹן דְּיֵיכוֹל כָּל דַּם וְאֶתֵּין רוּגְזִי בַּאֲנָשָׁא דְּיֵיכוֹל יָת דְּמָא וַאֲשֵׁיצֵי יָתֵיהּ מִגּוֹ עַמֵּיהּ׃}
{And whatsoever man there be of the house of Israel, or of the strangers that sojourn among them, that eateth any manner of blood, I will set My face against that soul that eateth blood, and will cut him off from among his people.}{\arabic{verse}}
\rashi{\rashiDH{כל דם.} לפי שנאמר בנפש יכפר, יכול לא יהא חייב אלא על דם המוקדשים, תלמוד לומר כל דם (כריתות ד׃)׃ 
\quad \rashiDH{ונתתי פני.} פנאי שלי, פונה אני מכל עסקי ועוסק בו׃}
\threeverse{\arabic{verse}}%Leviticus17:11
{כִּ֣י נֶ֣פֶשׁ הַבָּשָׂר֮ בַּדָּ֣ם הִוא֒ וַאֲנִ֞י נְתַתִּ֤יו לָכֶם֙ עַל\maqqaf הַמִּזְבֵּ֔חַ לְכַפֵּ֖ר עַל\maqqaf נַפְשֹׁתֵיכֶ֑ם כִּֽי\maqqaf הַדָּ֥ם ה֖וּא בַּנֶּ֥פֶשׁ יְכַפֵּֽר׃}
{אֲרֵי נְפַשׁ בִּשְׂרָא בִּדְמָא הִיא וַאֲנָא יְהַבְתֵּיהּ לְכוֹן עַל מַדְבְּחָא לְכַפָּרָא עַל נַפְשָׁתְכוֹן אֲרֵי דְּמָא הוּא עַל נַפְשָׁא מְכַפַּר׃}
{For the life of the flesh is in the blood; and I have given it to you upon the altar to make atonement for your souls; for it is the blood that maketh atonement by reason of the life.}{\arabic{verse}}
\rashi{\rashiDH{כי נפש הבשר.} של כל בריה בדם היא תלויה, ולפיכך נתתיו על המזבח לכפר על נפש האדם, תבוא נפש ותכפר על הנפש׃ 
}
\threeverse{\arabic{verse}}%Leviticus17:12
{עַל\maqqaf כֵּ֤ן אָמַ֙רְתִּי֙ לִבְנֵ֣י יִשְׂרָאֵ֔ל כׇּל\maqqaf נֶ֥פֶשׁ מִכֶּ֖ם לֹא\maqqaf תֹ֣אכַל דָּ֑ם וְהַגֵּ֛ר הַגָּ֥ר בְּתוֹכְכֶ֖ם לֹא\maqqaf יֹ֥אכַל דָּֽם׃}
{עַל כֵּן אֲמַרִית לִבְנֵי יִשְׂרָאֵל כָּל אֱנָשׁ מִנְּכוֹן לָא יֵיכוֹל דַּם וְגִיּוֹרַיָּא דְּיִתְגַּיְּירוּן בֵּינֵיכוֹן לָא יֵיכְלוּן דַּם׃}
{Therefore I said unto the children of Israel: No soul of you shall eat blood, neither shall any stranger that sojourneth among you eat blood.}{\arabic{verse}}
\rashi{\rashiDH{כל נפש מכם.} להזהיר גדולים על הקטנים (יבמות קיד.)׃}
\threeverse{\arabic{verse}}%Leviticus17:13
{וְאִ֨ישׁ אִ֜ישׁ מִבְּנֵ֣י יִשְׂרָאֵ֗ל וּמִן\maqqaf הַגֵּר֙ הַגָּ֣ר בְּתוֹכָ֔ם אֲשֶׁ֨ר יָצ֜וּד צֵ֥יד חַיָּ֛ה אוֹ\maqqaf ע֖וֹף אֲשֶׁ֣ר יֵאָכֵ֑ל וְשָׁפַךְ֙ אֶת\maqqaf דָּמ֔וֹ וְכִסָּ֖הוּ בֶּעָפָֽר׃}
{וּגְבַר גְּבַר מִבְּנֵי יִשְׂרָאֵל וּמִן גִּיּוֹרַיָּא דְּיִתְגַּיְּירוּן בֵּינֵיהוֹן דִּיצוּד צֵידָא חַיְתָא אוֹ עוֹפָא דְּמִתְאֲכִיל וְיֵישׁוֹד יָת דְּמֵיהּ וִיכַסֵּינֵיהּ בְּעַפְרָא׃}
{And whatsoever man there be of the children of Israel, or of the strangers that sojourn among them, that taketh in hunting any beast or fowl that may be eaten, he shall pour out the blood thereof, and cover it with dust.}{\arabic{verse}}
\rashi{\rashiDH{אשר יצוד.} אין לי אלא ציד, אווזין ותרנגולין מנין, תלמוד לומר ציד מכל מקום, אם כן למה נאמר אשר יצוד, שלא יאכל בשר אלא בהזמנה זאת (חולין פד.  ת״כ פרק יא, ב)׃\quad \rashiDH{אשר יאכל.} פרט לטמאים׃}
\threeverse{\arabic{verse}}%Leviticus17:14
{כִּֽי\maqqaf נֶ֣פֶשׁ כׇּל\maqqaf בָּשָׂ֗ר דָּמ֣וֹ בְנַפְשׁוֹ֮ הוּא֒ וָֽאֹמַר֙ לִבְנֵ֣י יִשְׂרָאֵ֔ל דַּ֥ם כׇּל\maqqaf בָּשָׂ֖ר לֹ֣א תֹאכֵ֑לוּ כִּ֣י נֶ֤פֶשׁ כׇּל\maqqaf בָּשָׂר֙ דָּמ֣וֹ הִ֔וא כׇּל\maqqaf אֹכְלָ֖יו יִכָּרֵֽת׃}
{אֲרֵי נְפַשׁ כָּל בִּשְׂרָא דְּמֵיהּ בְּנַפְשֵׁיהּ הוּא וַאֲמַרִית לִבְנֵי יִשְׂרָאֵל דַּם כָּל בִּשְׂרָא לָא תֵיכְלוּן אֲרֵי נְפַשׁ כָּל בִּשְׂרָא דְּמֵיהּ הִיא כָּל דְּיֵיכְלִנֵּיהּ יִשְׁתֵּיצֵי׃}
{For as to the life of all flesh, the blood thereof is all one with the life thereof; therefore I said unto the children of Israel: Ye shall eat the blood of no manner of flesh; for the life of all flesh is the blood thereof; whosoever eateth it shall be cut off.}{\arabic{verse}}
\rashi{\rashiDH{דמו בנפשו הוא.} דמו הוא לו במקום הנפש שהנפש תלויה בו׃\quad \rashiDH{כי נפש כל בשר דמו הוא.} הנפש היא הדם. דם ובשר לשון זכר, נפש לשון נקבה׃ 
}
\threeverse{\arabic{verse}}%Leviticus17:15
{וְכׇל\maqqaf נֶ֗פֶשׁ אֲשֶׁ֨ר תֹּאכַ֤ל נְבֵלָה֙ וּטְרֵפָ֔ה בָּאֶזְרָ֖ח וּבַגֵּ֑ר וְכִבֶּ֨ס בְּגָדָ֜יו וְרָחַ֥ץ בַּמַּ֛יִם וְטָמֵ֥א עַד\maqqaf הָעֶ֖רֶב וְטָהֵֽר׃}
{וְכָל אֱנָשׁ דְּיֵיכוֹל נְבִילָא וּתְבִירָא בְּיַצִּיבַיָּא וּבְגִיּוֹרַיָּא וִיצַבַּע לְבוּשׁוֹהִי וְיִסְחֵי בְמַיָּא וִיהֵי מְסָאַב עַד רַמְשָׁא וְיִדְכֵּי׃}
{And every soul that eateth that which dieth of itself, or that which is torn of beasts, whether he be home-born or a stranger, he shall wash his clothes, and bathe himself in water, and be unclean until the even; then shall he be clean.}{\arabic{verse}}
\rashi{\rashiDH{אשר תאכל נבלה וטרפה.} בנבלת עוף טהור דבר הכתוב, שאין לה טומאה אלא בשעה שנבלעת בבית הבליעה, ולמדך כאן שמטמאה באכילתה, ואינה מטמאה במגע, וטרפה האמורה כאן לא נכתבה אלא לדרוש, וכן שנינו יכול תהא נבלת עוף טמא מטמאה בבית הבליעה, תלמוד לומר טרפה, מי שיש במינו טרפה, יצא עוף טמא שאין במינו טרפה׃}
\threeverse{\arabic{verse}}%Leviticus17:16
{וְאִם֙ לֹ֣א יְכַבֵּ֔ס וּבְשָׂר֖וֹ לֹ֣א יִרְחָ֑ץ וְנָשָׂ֖א עֲוֺנֽוֹ׃ \petucha }
{וְאִם לָא יְצַבַּע וּבִשְׂרֵיהּ לָא יַסְחֵי וִיקַבֵּיל חוֹבֵיהּ׃}
{But if he wash them not, nor bathe his flesh, then he shall bear his iniquity.}{\arabic{verse}}
\rashi{\rashiDH{ונשא עונו.} אם יאכל קדש, או יכנס למקדש, חייב על טומאה זו ככל שאר טומאות׃\quad \rashiDH{ובשרו לא ירחץ ונשא עונו.} על רחיצת גופו ענוש כרת, ועל כבוס בגדים במלקות׃ 
}
\newperek
\newseder{14}
\threeverse{\seder{יד}}%Leviticus18:1
{וַיְדַבֵּ֥ר יְהֹוָ֖ה אֶל\maqqaf מֹשֶׁ֥ה לֵּאמֹֽר׃}
{וּמַלֵּיל יְיָ עִם מֹשֶׁה לְמֵימַר׃}
{And the \lord\space spoke unto Moses, saying:}{\Roman{chap}}
\threeverse{\arabic{verse}}%Leviticus18:2
{דַּבֵּר֙ אֶל\maqqaf בְּנֵ֣י יִשְׂרָאֵ֔ל וְאָמַרְתָּ֖ אֲלֵהֶ֑ם אֲנִ֖י יְהֹוָ֥ה אֱלֹהֵיכֶֽם׃}
{מַלֵּיל עִם בְּנֵי יִשְׂרָאֵל וְתֵימַר לְהוֹן אֲנָא יְיָ אֱלָהֲכוֹן׃}
{Speak unto the children of Israel, and say unto them: I am the \lord\space your God.}{\arabic{verse}}
\rashi{\rashiDH{אני ה׳ אלהיכם.} אני הוא שאמרתי בסיני, אנכי ה׳ אלהיך (שמות כ, ב), וקבלתם עליכם מלכותי, מעתה קבלו גזרותי. רבי אומר גלוי וידוע לפניו שסופן לִנָּתֵק בעריות בימי עזרא, לפיכך בא עליהם בגזירה אני ה׳ אלהיכם, דעו מי גוזר עליכם, דיין להפרע ונאמן לשלם שכר׃ 
}
\threeverse{\arabic{verse}}%Leviticus18:3
{כְּמַעֲשֵׂ֧ה אֶֽרֶץ\maqqaf מִצְרַ֛יִם אֲשֶׁ֥ר יְשַׁבְתֶּם\maqqaf בָּ֖הּ לֹ֣א תַעֲשׂ֑וּ וּכְמַעֲשֵׂ֣ה אֶֽרֶץ\maqqaf כְּנַ֡עַן אֲשֶׁ֣ר אֲנִי֩ מֵבִ֨יא אֶתְכֶ֥ם שָׁ֙מָּה֙ לֹ֣א תַעֲשׂ֔וּ וּבְחֻקֹּתֵיהֶ֖ם לֹ֥א תֵלֵֽכוּ׃}
{כְּעוּבָדֵי עַמָּא דְּאַרְעָא דְּמִצְרַיִם דִּיתֵיבְתּוּן בַּהּ לָא תַעְבְּדוּן וּכְעוּבָדֵי עַמָּא דְּאַרְעָא דִּכְנַעַן דַּאֲנָא מַעֵיל יָתְכוֹן לְתַמָּן לָא תַעְבְּדוּן וּבְנִמּוֹסֵיהוֹן לָא תְהָכוּן׃}
{After the doings of the land of Egypt, wherein ye dwelt, shall ye not do; and after the doings of the land of Canaan, whither I bring you, shall ye not do; neither shall ye walk in their statutes.}{\arabic{verse}}
\rashi{\rashiDH{כמעשה ארץ מצרים.} מגיד שמעשיהם של מצריים ושל כנעניים מקולקלים מכל האומות, ואותו מקום שישבו בו ישראל מקולקל מן הכל (ת״כ פרק יג, ה)׃\quad \rashiDH{אשר אני מביא אתכם שמה.} מגיד שאותן עממין שכבשו ישראל מקולקלים יותר מכולם׃\quad \rashiDH{ובחקתיהם לא תלכו.} מה הניח הכתוב שלא אמר, אלא אלו נמוסות שלהן, דברים החקוקין להם, כגון טַרְטִיָּאוֹת וְאִצְטַדְּיָאוֹת, ר׳ מאיר אומר אלו דרכי האמורי שמנו חכמים׃}
\threeverse{\arabic{verse}}%Leviticus18:4
{אֶת\maqqaf מִשְׁפָּטַ֧י תַּעֲשׂ֛וּ וְאֶת\maqqaf חֻקֹּתַ֥י תִּשְׁמְר֖וּ לָלֶ֣כֶת בָּהֶ֑ם אֲנִ֖י יְהֹוָ֥ה אֱלֹהֵיכֶֽם׃}
{יָת דִּינַי תַּעְבְּדוּן וְיָת קְיָמַי תִּטְּרוּן לְהַלָּכָא בְהוֹן אַנָא יְיָ אֱלָהֲכוֹן׃}
{Mine ordinances shall ye do, and My statutes shall ye keep, to walk therein: I am the \lord\space your God.}{\arabic{verse}}
\rashi{\rashiDH{את משפטי תעשו.} אלו דברים האמורים בתורה במשפט, שֶׁאִלּוּ לא נאמרו היו כדאי לאומרן׃\quad \rashiDH{ואת חקתי תשמרו.} דברים שהם גזירת המלך, שיצר הרע משיב עליהם למה לנו לשומרן, ואומות העולם ע״א משיבין עליהן, כגון אכילת חזיר, ולבישת שעטנז, וטהרת מי חטאת, לכך נאמר אני ה׳, גזרתי עליכם, אי אתם רשאים להפטר׃\quad \rashiDH{ללכת בהם.} אל תפטר מתוכם, שלא תאמר למדתי חכמת ישראל אלך ואלמד חכמת המצריים והכשדיים׃}
\threeverse{\arabic{verse}}%Leviticus18:5
{וּשְׁמַרְתֶּ֤ם אֶת\maqqaf חֻקֹּתַי֙ וְאֶת\maqqaf מִשְׁפָּטַ֔י אֲשֶׁ֨ר יַעֲשֶׂ֥ה אֹתָ֛ם הָאָדָ֖ם וָחַ֣י בָּהֶ֑ם אֲנִ֖י יְהֹוָֽה׃ \setuma }
{וְתִטְּרוּן יָת קְיָמַי וְיָת דִּינַי דְּאִם יַעֲבֵיד יָתְהוֹן אֲנָשָׁא יֵיחֵי בְהוֹן בְּחַיֵּי עָלְמָא אֲנָא יְיָ׃}
{Ye shall therefore keep My statutes, and Mine ordinances, which if a man do, he shall live by them: I am the \lord.}{\arabic{verse}}
\rashi{\rashiDH{ושמרתם את חקותי.} לרבות שאר דקדוקי הפרשה שלא פרט הכתוב בהם (ת״כ שם יא). דבר אחר ליתן שמירה ועשייה לחוקים, ושמירה ועשייה למשפטים (שם פרשתא ט, י), לפי שלא נתן אלא עשייה למשפטים ושמירה לחוקים׃\quad \rashiDH{וחי בהם.} לעולם הבא, שאם תאמר בעולם הזה, והלא סופו הוא מת׃\quad \rashiDH{אני ה׳.} נאמן לשלם שכר׃}
\aliyacounter{ששי}
\threeverse{\aliya{ששי}}%Leviticus18:6
{אִ֥ישׁ אִישׁ֙ אֶל\maqqaf כׇּל\maqqaf שְׁאֵ֣ר בְּשָׂר֔וֹ לֹ֥א תִקְרְב֖וּ לְגַלּ֣וֹת עֶרְוָ֑ה אֲנִ֖י יְהֹוָֽה׃ \setuma }
{גְּבַר גְּבַר לְכָל קָרִיב בִּשְׂרֵיהּ לָא תִקְרְבוּן לְגַלָּאָה עֶרְיָא אֲנָא יְיָ׃}
{None of you shall approach to any that is near of kin to him, to uncover their nakedness. I am the \lord.}{\arabic{verse}}
\rashi{\rashiDH{לא תקרבו.} להזהיר הנקבה כזכר, לכך נאמר לשון רבים׃ 
\quad \rashiDH{אני ה׳.} נאמן לשלם שכר׃}
\threeverse{\arabic{verse}}%Leviticus18:7
{עֶרְוַ֥ת אָבִ֛יךָ וְעֶרְוַ֥ת אִמְּךָ֖ לֹ֣א תְגַלֵּ֑ה אִמְּךָ֣ הִ֔וא לֹ֥א תְגַלֶּ֖ה עֶרְוָתָֽהּ׃ \setuma }
{עֶרְיַת אֲבוּךְ וְעֶרְיַת אִמָּךְ לָא תְגַלֵּי אִמָּךְ הִיא לָא תְגַלֵּי עֶרְיְתַהּ׃}
{The nakedness of thy father, and the nakedness of thy mother, shalt thou not uncover: she is thy mother; thou shalt not uncover her nakedness.}{\arabic{verse}}
\rashi{\rashiDH{ערות אביך.} זו אשת אביך (סנהדרין נד.), או אינו אלא כמשמעו, נאמר כאן ערות אביך, ונאמר להלן עֶרְוַת אָבִיו גִּלָּה (ויקרא כ, יא), מה להלן אשת אביו, אף כאן אשת אביו׃\quad \rashiDH{וערות אמך.} להביא אמו שאינה אשת אביו׃}
\threeverse{\arabic{verse}}%Leviticus18:8
{עֶרְוַ֥ת אֵֽשֶׁת\maqqaf אָבִ֖יךָ לֹ֣א תְגַלֵּ֑ה עֶרְוַ֥ת אָבִ֖יךָ הִֽוא׃ \setuma }
{עֶרְיַת אִתַּת אֲבוּךְ לָא תְגַלֵּי עֶרְיְתָא דַּאֲבוּךְ הִיא׃}
{The nakedness of thy father’s wife shalt thou not uncover: it is thy father’s nakedness.}{\arabic{verse}}
\rashi{\rashiDH{ערות אשת אביך.} לרבות לאחר מיתה׃ 
}
\threeverse{\arabic{verse}}%Leviticus18:9
{עֶרְוַ֨ת אֲחֽוֹתְךָ֤ בַת\maqqaf אָבִ֙יךָ֙ א֣וֹ בַת\maqqaf אִמֶּ֔ךָ מוֹלֶ֣דֶת בַּ֔יִת א֖וֹ מוֹלֶ֣דֶת ח֑וּץ לֹ֥א תְגַלֶּ֖ה עֶרְוָתָֽן׃ \setuma }
{עֶרְיַת אֲחָתָךְ בַּת אֲבוּךְ אוֹ בַת אִמָּךְ דִּילִידָא מִן אֲבוּךְ מִן אִתָּא אוּחְרִי אוֹ מִן אִמָּךְ מִן גְּבַר אָחֳרָן לָא תְגַלֵּי עֶרְיַתְהוֹן׃}
{The nakedness of thy sister, the daughter of thy father, or the daughter of thy mother, whether born at home, or born abroad, even their nakedness thou shalt not uncover. .}{\arabic{verse}}
\rashi{\rashiDH{בת אביך.} אף בת אנוסה במשמע׃\quad \rashiDH{מולדת בית או מולדת חוץ.} בין שאומרים לו לאביך קיים את אמה, ובין שאומרים לו הוצא את אמה (יבמות כג.), כגון ממזרת או נתינה׃}
\threeverse{\arabic{verse}}%Leviticus18:10
{עֶרְוַ֤ת בַּת\maqqaf בִּנְךָ֙ א֣וֹ בַֽת\maqqaf בִּתְּךָ֔ לֹ֥א תְגַלֶּ֖ה עֶרְוָתָ֑ן כִּ֥י עֶרְוָתְךָ֖ הֵֽנָּה׃ \setuma }
{עֶרְיַת בַּת בְּרָךְ אוֹ בַת בְּרַתָּךְ לָא תְגַלֵּי עֶרְיַתְהוֹן אֲרֵי עֶרְיְתָךְ אִנִּין׃}
{The nakedness of thy son’s daughter, or of thy daughter’s daughter, even their nakedness thou shalt not uncover; for theirs is thine own nakedness.}{\arabic{verse}}
\rashi{\rashiDH{ערות בת בנך וגו׳.} בבתו מאנוסתו הכתוב מדבר (סנהדרין עו.), ובתו ובת בתו מאשתו אנו למדין מערות אשה ובתה שנאמר בהן לא תגלה, בין שהיא ממנו בין שהיא מאיש אחר׃\quad \rashiDH{ערות בת בנך.} קל וחומר לבתך, אלא לפי שאין מזהירין מן הדין למדוה מגזרה שוה, (במסכת יבמות ג.)׃}
\threeverse{\arabic{verse}}%Leviticus18:11
{עֶרְוַ֨ת בַּת\maqqaf אֵ֤שֶׁת אָבִ֙יךָ֙ מוֹלֶ֣דֶת אָבִ֔יךָ אֲחוֹתְךָ֖ הִ֑וא לֹ֥א תְגַלֶּ֖ה עֶרְוָתָֽהּ׃ \setuma }
{עֶרְיַת בַּת אִתַּת אֲבוּךְ דִּילִידָא מִן אֲבוּךְ אֲחָתָךְ הִיא לָא תְגַלֵּי עֶרְיְתַהּ׃}
{The nakedness of thy father’s wife’s daughter, begotten of thy father, she is thy sister, thou shalt not uncover her nakedness.}{\arabic{verse}}
\rashi{\rashiDH{ערות בת אשת אביך.} לימד שאינו חייב על אחותו מִשִּׁפְחָה וְנָכְרִית, לכך נאמר בת אשת אביך, בראויה לקידושין (יבמות כג.)׃}
\threeverse{\arabic{verse}}%Leviticus18:12
{עֶרְוַ֥ת אֲחוֹת\maqqaf אָבִ֖יךָ לֹ֣א תְגַלֵּ֑ה שְׁאֵ֥ר אָבִ֖יךָ הִֽוא׃ \setuma }
{עֶרְיַת אֲחָת אֲבוּךְ לָא תְגַלֵּי קָרִיבַת אֲבוּךְ הִיא׃}
{Thou shalt not uncover the nakedness of thy father’s sister: she is thy father’s near kinswoman.}{\arabic{verse}}
\threeverse{\arabic{verse}}%Leviticus18:13
{עֶרְוַ֥ת אֲחֽוֹת\maqqaf אִמְּךָ֖ לֹ֣א תְגַלֵּ֑ה כִּֽי\maqqaf שְׁאֵ֥ר אִמְּךָ֖ הִֽוא׃ \setuma }
{עֶרְיַת אֲחָת אִמָּךְ לָא תְגַלֵּי אֲרֵי קָרִיבַת אִמָּךְ הִיא׃}
{Thou shalt not uncover the nakedness of thy mother’s sister; for she is thy mother’s near kinswoman.}{\arabic{verse}}
\threeverse{\arabic{verse}}%Leviticus18:14
{עֶרְוַ֥ת אֲחִֽי\maqqaf אָבִ֖יךָ לֹ֣א תְגַלֵּ֑ה אֶל\maqqaf אִשְׁתּוֹ֙ לֹ֣א תִקְרָ֔ב דֹּדָֽתְךָ֖ הִֽוא׃ \setuma }
{עֶרְיַת אַחְבּוּךְ לָא תְגַלֵּי לְאִתְּתֵיהּ לָא תִקְרַב אִתַּת אַחְבּוּךְ הִיא׃}
{Thou shalt not uncover the nakedness of thy fathers brother, thou shalt not approach to his wife: she is thine aunt.}{\arabic{verse}}
\rashi{\rashiDH{ערות אחי אביך לא תגלה.} ומה היא ערותו, אל אשתו לא תקרב׃}
\threeverse{\arabic{verse}}%Leviticus18:15
{עֶרְוַ֥ת כַּלָּֽתְךָ֖ לֹ֣א תְגַלֵּ֑ה אֵ֤שֶׁת בִּנְךָ֙ הִ֔וא לֹ֥א תְגַלֶּ֖ה עֶרְוָתָֽהּ׃ \setuma }
{עֶרְיַת כַּלְּתָךְ לָא תְגַלֵּי אִתַּת בְּרָךְ הִיא לָא תְגַלֵּי עֶרְיְתַהּ׃}
{Thou shalt not uncover the nakedness of thy daughter-in-law: she is thy son’wife; thou shalt not uncover her nakedness.}{\arabic{verse}}
\rashi{\rashiDH{אשת בנך היא.} לא אמרתי אלא בשיש לבנך אישות בה, פרט לאנוסה ושפחה ונכרית׃}
\threeverse{\arabic{verse}}%Leviticus18:16
{עֶרְוַ֥ת אֵֽשֶׁת\maqqaf אָחִ֖יךָ לֹ֣א תְגַלֵּ֑ה עֶרְוַ֥ת אָחִ֖יךָ הִֽוא׃ \setuma }
{עֶרְיַת אִתַּת אֲחוּךְ לָא תְגַלֵּי עֶרְיְתָא דַּאֲחוּךְ הִיא׃}
{Thou shalt not uncover the nakedness of thy brother’s wife: it is thy brother’s nakedness.}{\arabic{verse}}
\threeverse{\arabic{verse}}%Leviticus18:17
{עֶרְוַ֥ת אִשָּׁ֛ה וּבִתָּ֖הּ לֹ֣א תְגַלֵּ֑ה אֶֽת\maqqaf בַּת\maqqaf בְּנָ֞הּ וְאֶת\maqqaf בַּת\maqqaf בִּתָּ֗הּ לֹ֤א תִקַּח֙ לְגַלּ֣וֹת עֶרְוָתָ֔הּ שַׁאֲרָ֥ה הֵ֖נָּה זִמָּ֥ה הִֽוא׃}
{עֶרְיַת אִתְּתָא וּבְרַתַּהּ לָא תְגַלֵּי יָת בַּת בְּרַהּ וְיָת בַּת בְּרַתַּהּ לָא תִסַּב לְגַלָּאָה עֶרְיְתַהּ קָרִיבָן אִנִּין עֵיצַת חֲטִאין הִיא׃}
{Thou shalt not uncover the nakedness of a woman and her daughter; thou shalt not take her son’s daughter, or her daughter’s daughter, to uncover her nakedness: they are near kinswomen; it is lewdness.}{\arabic{verse}}
\rashi{\rashiDH{ערות אשה ובתה.} לא אסר הכתוב אלא ע״י נשואי הראשונה (יבמות צז.), לכך נאמר לא תקח, לשון קיחה, וכן לענין העונש אֲשֶׁר יִקַּח אֶת אִשָּׁה וְאֶת אִמָּהּ (ויקרא כ, יד), לשון קיחה, אבל אנס אשה מותר לישא בתה׃\quad \rashiDH{שארה הנה.} קרובות הן זו לזו׃\quad \rashiDH{זמה.} עצה, כתרגומו עֲצַת חֶטְאִין, שיצרך יועצך לחטוא׃ 
}
\threeverse{\arabic{verse}}%Leviticus18:18
{וְאִשָּׁ֥ה אֶל\maqqaf אֲחֹתָ֖הּ לֹ֣א תִקָּ֑ח לִצְרֹ֗ר לְגַלּ֧וֹת עֶרְוָתָ֛הּ עָלֶ֖יהָ בְּחַיֶּֽיהָ׃}
{וְאִתְּתָא עִם אֲחָתַהּ לָא תִסַּב לְאַעָקָא לַהּ לְגַלָּאָה עֶרְיְתַהּ עֲלַהּ בְּחַיַּיהָא׃}
{And thou shalt not take a woman to her sister, to be a rival to her, to uncover her nakedness, beside the other in her lifetime.}{\arabic{verse}}
\rashi{\rashiDH{אל אחותה.} שתיהן כאחת (קידושין נ׃)׃\quad \rashiDH{לצרר.} לשון צרה, לעשות את זו צרה לזו׃\quad \rashiDH{בחייה.} למדך, שאם גרשה לא ישא את אחותה, כל זמן שהיא בחיים (יבמות ח׃)׃}
\threeverse{\arabic{verse}}%Leviticus18:19
{וְאֶל\maqqaf אִשָּׁ֖ה בְּנִדַּ֣ת טֻמְאָתָ֑הּ לֹ֣א תִקְרַ֔ב לְגַלּ֖וֹת עֶרְוָתָֽהּ׃}
{וּלְאִתְּתָא בְּרִיחוּק סְאוֹבְתַהּ לָא תִקְרַב לְגַלָּאָה עֶרְיְתַהּ׃}
{And thou shalt not approach unto a woman to uncover her nakedness, as long as she is impure by her uncleanness.}{\arabic{verse}}
\threeverse{\arabic{verse}}%Leviticus18:20
{וְאֶל\maqqaf אֵ֙שֶׁת֙ עֲמִֽיתְךָ֔ לֹא\maqqaf תִתֵּ֥ן שְׁכׇבְתְּךָ֖ לְזָ֑רַע לְטׇמְאָה\maqqaf בָֽהּ׃}
{וּבְאִתַּת חַבְרָךְ לָא תִתֵּין שְׁכוּבְתָּךְ לְזַרְעָא לְאִסְתַּאָבָא בַהּ׃}
{And thou shalt not lie carnally with thy neighbour’s wife, to defile thyself with her.}{\arabic{verse}}
\threeverse{\arabic{verse}}%Leviticus18:21
{וּמִֽזַּרְעֲךָ֥ לֹא\maqqaf תִתֵּ֖ן לְהַעֲבִ֣יר לַמֹּ֑לֶךְ וְלֹ֧א תְחַלֵּ֛ל אֶת\maqqaf שֵׁ֥ם אֱלֹהֶ֖יךָ אֲנִ֥י יְהֹוָֽה׃}
{וּמִזַּרְעָךְ לָא תִתֵּין לְאַעְבָּרָא לְמוֹלֶךְ וְלָא תַחֵיל יָת שְׁמָא דֶּאֱלָהָךְ אֲנָא יְיָ׃}
{And thou shalt not give any of thy seed to set them apart to Molech, neither shalt thou profane the name of thy God: I am the \lord.}{\arabic{verse}}
\rashi{\rashiDH{למלך.} עבודת אלילים היא ששמה מולך, וזו היא עבודתה, שמוסר בנו לכומרים ועושין שתי מדורות גדולות ומעבירין את הבן ברגליו בין שתי מדורות האש (סנהדרין סד׃)׃\quad \rashiDH{לא תתן.} זו היא מסירתו לכומרים׃\quad \rashiDH{להעביר למלך.} זו העברת האש׃}
\aliyacounter{שביעי}
\threeverse{\aliya{שביעי\newline (רביעי)}}%Leviticus18:22
{וְאֶ֨ת\maqqaf זָכָ֔ר לֹ֥א תִשְׁכַּ֖ב מִשְׁכְּבֵ֣י אִשָּׁ֑ה תּוֹעֵבָ֖ה הִֽוא׃}
{וְיָת דְּכוּרָא לָא תִשְׁכּוֹב מִשְׁכְּבֵי אִתָּא תּוֹעֵיבָא הִיא׃}
{Thou shalt not lie with mankind, as with womankind; it is abomination.}{\arabic{verse}}
\threeverse{\arabic{verse}}%Leviticus18:23
{וּבְכׇל\maqqaf בְּהֵמָ֛ה לֹא\maqqaf תִתֵּ֥ן שְׁכׇבְתְּךָ֖ לְטׇמְאָה\maqqaf בָ֑הּ וְאִשָּׁ֗ה לֹֽא\maqqaf תַעֲמֹ֞ד לִפְנֵ֧י בְהֵמָ֛ה לְרִבְעָ֖הּ תֶּ֥בֶל הֽוּא׃}
{וּבְכָל בְּעִירָא לָא תִתֵּין שְׁכוּבְתָּךְ לְאִסְתַּאָבָא בַהּ וְאִתְּתָא לָא תְקוּם קֳדָם בְּעִירָא לְמִשְׁלַט בַּהּ תָּבְלָא הוּא׃}
{And thou shalt not lie with any beast to defile thyself therewith; neither shall any woman stand before a beast, to lie down thereto; it is perversion.}{\arabic{verse}}
\rashi{\rashiDH{תבל הוא.} לשון קָדֵשׁ וערוה וניאוף, וכן וְאַפִּי עַל תַּבְלִיתָם (ישעי׳ י, כה). דבר אחר תבל הוא, לשון בלילה וערבוב, זרע אדם וזרע בהמה׃ 
}
\threeverse{\arabic{verse}}%Leviticus18:24
{אַל\maqqaf תִּֽטַּמְּא֖וּ בְּכׇל\maqqaf אֵ֑לֶּה כִּ֤י בְכׇל\maqqaf אֵ֙לֶּה֙ נִטְמְא֣וּ הַגּוֹיִ֔ם אֲשֶׁר\maqqaf אֲנִ֥י מְשַׁלֵּ֖חַ מִפְּנֵיכֶֽם׃}
{לָא תִסְתָּאֲבוּן בְּכָל אִלֵּין אֲרֵי בְכָל אִלֵּין אִסְתָאֲבוּ עַמְמַיָּא דַּאֲנָא מַגְלֵי מִן קֳדָמֵיכוֹן׃}
{Defile not ye yourselves in any of these things; for in all these the nations are defiled, which I cast out from before you.}{\arabic{verse}}
\threeverse{\arabic{verse}}%Leviticus18:25
{וַתִּטְמָ֣א הָאָ֔רֶץ וָאֶפְקֹ֥ד עֲוֺנָ֖הּ עָלֶ֑יהָ וַתָּקִ֥א הָאָ֖רֶץ אֶת\maqqaf יֹשְׁבֶֽיהָ׃}
{וְאִסְתָּאַבַת אַרְעָא וְאַסְעַרִית חוֹבַהּ עֲלַהּ וְרוֹקֵינַת אַרְעָא יָת יָתְבַהָא׃}
{And the land was defiled, therefore I did visit the iniquity thereof upon it, and the land vomited out her inhabitants.}{\arabic{verse}}
\threeverse{\arabic{verse}}%Leviticus18:26
{וּשְׁמַרְתֶּ֣ם אַתֶּ֗ם אֶת\maqqaf חֻקֹּתַי֙ וְאֶת\maqqaf מִשְׁפָּטַ֔י וְלֹ֣א תַעֲשׂ֔וּ מִכֹּ֥ל הַתּוֹעֵבֹ֖ת הָאֵ֑לֶּה הָֽאֶזְרָ֔ח וְהַגֵּ֖ר הַגָּ֥ר בְּתוֹכְכֶֽם׃}
{וְתִטְּרוּן אַתּוּן יָת קְיָמַי וְיָת דִּינַי וְלָא תַעְבְּדוּן מִכֹּל תּוֹעֵיבָתָא הָאִלֵּין יַצִּיבַיָּא וְגִיּוֹרַיָּא דְּיִתְגַּיְּרוּן בֵּינֵיכוֹן׃}
{Ye therefore shall keep My statutes and Mine ordinances, and shall not do any of these abominations; neither the home-born, nor the stranger that sojourneth among you—}{\arabic{verse}}
\threeverse{\arabic{verse}}%Leviticus18:27
{כִּ֚י אֶת\maqqaf כׇּל\maqqaf הַתּוֹעֵבֹ֣ת הָאֵ֔ל עָשׂ֥וּ אַנְשֵֽׁי\maqqaf הָאָ֖רֶץ אֲשֶׁ֣ר לִפְנֵיכֶ֑ם וַתִּטְמָ֖א הָאָֽרֶץ׃}
{אֲרֵי יָת כָּל תּוֹעֵיבָתָא הָאִלֵּין עֲבַדוּ אֱנָשֵׁי אַרְעָא דִּקְדָמֵיכוֹן וְאִסְתָּאַבַת אַרְעָא׃}
{for all these abominations have the men of the land done, that were before you, and the land is defiled—}{\arabic{verse}}
\threeverse{\aliya{מפטיר}}%Leviticus18:28
{וְלֹֽא\maqqaf תָקִ֤יא הָאָ֙רֶץ֙ אֶתְכֶ֔ם בְּטַֽמַּאֲכֶ֖ם אֹתָ֑הּ כַּאֲשֶׁ֥ר קָאָ֛ה אֶת\maqqaf הַגּ֖וֹי אֲשֶׁ֥ר לִפְנֵיכֶֽם׃}
{וְלָא תְרוֹקֵין אַרְעָא יָתְכוֹן בְּסַאוֹבֵיכוֹן יָתַהּ כְּמָא דְּרוֹקֵינַת יָת עַמְמַיָּא דִּקְדָמֵיכוֹן׃}
{that the land vomit not you out also, when ye defile it, as it vomited out the nation that was before you.}{\arabic{verse}}
\rashi{\rashiDH{ולא תקיא הארץ אתכם.} משל לבן מלך שהאכילוהו דבר מאוס שאין עומד במעיו אלא מקיאו, כך ארץ ישראל אינה מקיימת עוברי עבירה. ותרגומו וְלָא תְרוֹקֵן, לשון ריקון, מריקה עצמה מהם׃}
\threeverse{\arabic{verse}}%Leviticus18:29
{כִּ֚י כׇּל\maqqaf אֲשֶׁ֣ר יַעֲשֶׂ֔ה מִכֹּ֥ל הַתּוֹעֵבֹ֖ת הָאֵ֑לֶּה וְנִכְרְת֛וּ הַנְּפָשׁ֥וֹת הָעֹשֹׂ֖ת מִקֶּ֥רֶב עַמָּֽם׃}
{אֲרֵי כָל דְּיַעֲבֵיד מִכֹּל תּוֹעֵיבָתָא הָאִלֵּין וְיִשְׁתֵּיצוֹן נַפְשָׁתָא דְּיַעְבְּדָן מִגּוֹ עַמְּהוֹן׃}
{For whosoever shall do any of these abominations, even the souls that do them shall be cut off from among their people.}{\arabic{verse}}
\rashi{\rashiDH{הנפשות העושות.} הזכר והנקבה במשמע (ב״ק לב.)׃}
\threeverse{\aliya{\Hebrewnumeral{80}}}%Leviticus18:30
{וּשְׁמַרְתֶּ֣ם אֶת\maqqaf מִשְׁמַרְתִּ֗י לְבִלְתִּ֨י עֲשׂ֜וֹת מֵחֻקּ֤וֹת הַתּֽוֹעֵבֹת֙ אֲשֶׁ֣ר נַעֲשׂ֣וּ לִפְנֵיכֶ֔ם וְלֹ֥א תִֽטַּמְּא֖וּ בָּהֶ֑ם אֲנִ֖י יְהֹוָ֥ה אֱלֹהֵיכֶֽם׃ \petucha }
{וְתִטְּרוּן יָת מַטְּרַת מֵימְרִי בְּדִיל דְּלָא לְמֶעֱבַד מִנִּמּוֹסֵי תּוֹעֵיבָתָא דְּאִתְעֲבִידָא קֳדָמֵיכוֹן וְלָא תִסְתָּאֲבוּן בְּהוֹן אֲנָא יְיָ אֱלָהֲכוֹן׃}
{Therefore shall ye keep My charge, that ye do not any of these abominable customs, which were done before you, and that ye defile not yourselves therein: I am the \lord\space your God.}{\arabic{verse}}
\rashi{\rashiDH{ושמרתם את משמרתי.} להזהיר בית דין על כך׃\quad \rashiDH{ולא תטמאו בהם אני ה׳ אלהיכם.} הא אם תטמאו איני אלהיכם, ואתם נפסלים מאחרי, ומה הנאה יש לי בכם, ואתם מתחייבים כלייה, לכך נאמר אני ה׳ אלהיכם׃ 
}
\engnote{The Haftarah is Ezekiel 22:1\verserangechar 22:19 on page \pageref{haft_29}. Sepharadim read Ezekiel 22:1\verserangechar 22:16. On the Shabbat before Pesa\d{h}, read the Haftara on page \pageref{haft_hagadol}.}
\newperek
\aliyacounter{ראשון}
\newparsha{קדשים}
\newseder{15}
\threeverse{\aliya{קדשים}\newline\vspace{-4pt}\newline\seder{טו}}%Leviticus19:1
{וַיְדַבֵּ֥ר יְהֹוָ֖ה אֶל\maqqaf מֹשֶׁ֥ה לֵּאמֹֽר׃}
{וּמַלֵּיל יְיָ עִם מֹשֶׁה לְמֵימַר׃}
{And the \lord\space spoke unto Moses, saying:}{\Roman{chap}}
\threeverse{\arabic{verse}}%Leviticus19:2
{דַּבֵּ֞ר אֶל\maqqaf כׇּל\maqqaf עֲדַ֧ת בְּנֵי\maqqaf יִשְׂרָאֵ֛ל וְאָמַרְתָּ֥ אֲלֵהֶ֖ם קְדֹשִׁ֣ים תִּהְי֑וּ כִּ֣י קָד֔וֹשׁ אֲנִ֖י יְהֹוָ֥ה אֱלֹהֵיכֶֽם׃}
{מַלֵּיל עִם כָּל כְּנִשְׁתָּא דִּבְנֵי יִשְׂרָאֵל וְתֵימַר לְהוֹן קַדִּישִׁין תְּהוֹן אֲרֵי קַדִּישׁ אֲנָא יְיָ אֱלָהֲכוֹן׃}
{Speak unto all the congregation of the children of Israel, and say unto them: Ye shall be holy; for I the \lord\space your God am holy.}{\arabic{verse}}
\rashi{\rashiDH{דבר אל כל עדת בני ישראל.} מלמד שנאמרה פרשה זו בהקהל, מפני שרוב גופי תורה תלויין בה (ויק״ר כד, ה)׃\quad \rashiDH{קדושים תהיו.} הוו פרושים מן העריות ומן העבירה, שכל מקום שאתה מוצא גדר ערוה אתה מוצא קדושה (ויק״ר שם ו), אִשָּׁה זֹנָה וַחֲלָלָה וגו׳ (ויקרא כא, ז), אֲנִי ה׳ מְקַדִּשְׁכֶם. (ת״כ פרשתא א, א)) וְלֹא יְחַלֵּל זַרְעוֹ אֲנִי ה׳ מְקַדְּשׁוֹ (שם פסוק טו). קְדשִׁים יִהְיוּ (שם פסוק ו), אִשָּׁה זֹנָה וַחֲלָלָה וגו׳ (שם פסוק ז)׃}
\threeverse{\arabic{verse}}%Leviticus19:3
{אִ֣ישׁ אִמּ֤וֹ וְאָבִיו֙ תִּירָ֔אוּ וְאֶת\maqqaf שַׁבְּתֹתַ֖י תִּשְׁמֹ֑רוּ אֲנִ֖י יְהֹוָ֥ה אֱלֹהֵיכֶֽם׃}
{גְּבַר מִן אִמֵּיהּ וּמִן אֲבוּהִי תְּהוֹן דָּחֲלִין וְיָת יוֹמֵי שַׁבַּיָּא דִּילִי תִטְּרוּן אֲנָא יְיָ אֱלָהֲכוֹן׃}
{Ye shall fear every man his mother, and his father, and ye shall keep My sabbaths: I am the \lord\space your God.}{\arabic{verse}}
\rashi{\rashiDH{איש אמו ואביו תיראו.} כל אחד מכם תיראו אביו ואמו, זהו פשוטו. ומדרשו (ת״כ שם ג  קידושין ל׃) אין לי אלא איש, אשה מנין, כשהוא אומר תיראו הרי כאן שנים, א״כ למה נאמר איש, שהאיש סיפק בידו לעשות, אבל אשה רשות אחרים עליה׃\quad \rashiDH{אמו ואביו תיראו.} כאן הקדים אֵם לאב, לפי שגלוי לפניו שהבן ירא את אביו יותר מאמו, וּבְכָבוֹד הקדים אב לאם, לפי שגלוי לפניו שהבן מכבד את אמו יותר מאביו, מפני שמשדלתו בדברים (קידושין לא.)׃\quad \rashiDH{ואת שבתתי תשמרו.} סמך שמירת שבת למורא אב, לומר אף על פי שהזהרתיך על מורא אב, אם יאמר לך חלל את השבת, אל תשמע לו, וכן בשאר כל המצות (ב״מ לב)׃\quad \rashiDH{אני ה׳ אלהיכם.} אתה ואביך חייבים בכבודי (יבמות ה׃), לפיכך לא תשמע לו לבטל את דברי (ב״מ לב.). איזהו מורא, לא ישב במקומו, ולא ידבר במקומו, ולא יסתור את דבריו. ואיזהו כָּבוֹד, מאכיל ומשקה, מלביש ומנעיל, מכניס ומוציא (קידושין לא׃)׃}
\threeverse{\arabic{verse}}%Leviticus19:4
{אַל\maqqaf תִּפְנוּ֙ אֶל\maqqaf הָ֣אֱלִילִ֔ם וֵֽאלֹהֵי֙ מַסֵּכָ֔ה לֹ֥א תַעֲשׂ֖וּ לָכֶ֑ם אֲנִ֖י יְהֹוָ֥ה אֱלֹהֵיכֶֽם׃}
{לָא תִתְפְּנוֹן בָּתַר טָעֲוָן וְדַחְלָן דְּמַתְּכָא לָא תַעְבְּדוּן לְכוֹן אֲנָא יְיָ אֱלָהֲכוֹן׃}
{Turn ye not unto the idols, nor make to yourselves molten gods: I am the \lord\space your God.}{\arabic{verse}}
\rashi{\rashiDH{אל תפנו אל האלילים.} לעבדם. (ת״כ שם י) אלילים לשון אַל, כלא הוא חשוב׃\quad \rashiDH{ואלהי מסכה.} תחילתן אלילים הם, ואם אתה פונה אחריהם סופך לעשותן אלהות׃\quad \rashiDH{לא תעשו לכם.} לא תעשו לאחרים, ולא אחרים לכם. ואם תאמר לא תעשו לעצמכם אבל אחרים עושין לכם, הרי כבר נאמר לֹא יִהְיֶה לְךָ (שמות כ, ג), לא שלך ולא של אחרים׃ 
}
\threeverse{\aliya{לוי}}%Leviticus19:5
{וְכִ֧י תִזְבְּח֛וּ זֶ֥בַח שְׁלָמִ֖ים לַיהֹוָ֑ה לִֽרְצֹנְכֶ֖ם תִּזְבָּחֻֽהוּ׃}
{וַאֲרֵי תִכְּסוּן נִכְסַת קוּדְשִׁין קֳדָם יְיָ לְרַעֲוָא לְכוֹן תִּכְּסוּנֵּיהּ׃}
{And when ye offer a sacrifice of peace-offerings unto the \lord, ye shall offer it that ye may be accepted.}{\arabic{verse}}
\rashi{\rashiDH{וכי תזבחו וגו׳.} לא נאמרה פרשה זו, אלא ללמד, שלא תהא זביחתן אלא על מנת להאכל בתוך הזמן הזה, שאם לקבוע להם זמן אכילה, הרי כבר נאמר וְאִם נֵדֶר אוֹ נְדָבָה זֶבַח קָרְבָּנוֹ וגו׳ (ויקרא ז, טז)׃\quad \rashiDH{לרצנכם תזבחהו.} תחלת זביחתו תהא על מנת נחת רוח שיהא לכם לרצון, שאם תחשבו עליו מחשבת פסול לא ירצה עליכם לפני׃\quad \rashiDH{לרצנכם.} אנפיי״צימנטו, זהו לפי פשוטו. ורבותינו למדו (חולין יג׃) מכאן למתעסק בקדשים שפסול, שצריך שיתכוין לשחוט׃ 
}
\threeverse{\arabic{verse}}%Leviticus19:6
{בְּי֧וֹם זִבְחֲכֶ֛ם יֵאָכֵ֖ל וּמִֽמׇּחֳרָ֑ת וְהַנּוֹתָר֙ עַד\maqqaf י֣וֹם הַשְּׁלִישִׁ֔י בָּאֵ֖שׁ יִשָּׂרֵֽף׃}
{בְּיוֹמָא דְּיִתְנְכֵיס יִתְאֲכִיל וּבְיוֹמָא דְּבָתְרוֹהִי וּדְיִשְׁתְּאַר עַד יוֹמָא תְּלִיתָאָה בְּנוּרָא יִתּוֹקַד׃}
{It shall be eaten the same day ye offer it, and on the morrow; and if aught remain until the third day, it shall be burnt with fire.}{\arabic{verse}}
\rashi{\rashiDH{ביום זבחכם יאכל.} כשתזבחוהו, תשחטוהו על מנת זמן זה שקבעתי לכם כבר׃}
\threeverse{\arabic{verse}}%Leviticus19:7
{וְאִ֛ם הֵאָכֹ֥ל יֵאָכֵ֖ל בַּיּ֣וֹם הַשְּׁלִישִׁ֑י פִּגּ֥וּל ה֖וּא לֹ֥א יֵרָצֶֽה׃}
{וְאִם אִתְאֲכָלָא יִתְאֲכִיל בְּיוֹמָא תְּלִיתָאָה מְרַחַק הוּא לָא יְהֵי לְרַעֲוָא׃}
{And if it be eaten at all on the third day, it is a vile thing; it shall not be accepted.}{\arabic{verse}}
\rashi{\rashiDH{ואם האכל יאכל וגו׳.} אם אינו ענין לחוץ, לזמנו, שהרי כבר נאמר וְאִם הֵאָכֹל יֵאָכֵל מִבְּשַׂר זֶבַח שְׁלָמָיו וגו׳ (ויקרא שם, יח), תנהו ענין לחוץ למקומו, יכול יהיו חייבין כרת על אכילתו, תלמוד לומר וְהַנֶּפֶשׁ הָאֹכֶלֶת מִמֶּנּוּ עֲוֹנָהּ תִּשָּׂא (שם), ממנו ולא מחבירו, יצא הנשחט במחשבת חוץ למקומו (זבחים כט.)׃\quad \rashiDH{פגול.} מתועב, כמו וּמְרַק פִּגּוּלִים כְּלֵיהֶם (ישעי׳ סה, ד)׃ 
}
\threeverse{\arabic{verse}}%Leviticus19:8
{וְאֹֽכְלָיו֙ עֲוֺנ֣וֹ יִשָּׂ֔א כִּֽי\maqqaf אֶת\maqqaf קֹ֥דֶשׁ יְהֹוָ֖ה חִלֵּ֑ל וְנִכְרְתָ֛ה הַנֶּ֥פֶשׁ הַהִ֖וא מֵעַמֶּֽיהָ׃}
{וּדְיֵיכְלִנֵּיהּ חוֹבֵיהּ יְקַבֵּיל אֲרֵי יָת קוּדְשָׁא דַּייָ אַחֵיל וְיִשְׁתֵּיצֵי אֲנָשָׁא הַהוּא מֵעַמֵּיהּ׃}
{But every one that eateth it shall bear his iniquity, because he hath profaned the holy thing of the \lord; and that soul shall be cut off from his people.}{\arabic{verse}}
\rashi{\rashiDH{ואכליו עונו ישא.} בנותר גמור הכתוב מדבר ואינו ענוש כרת על הנשחט חוץ למקומו, שכבר מיעטו הכתוב, וזהו בנותר גמור מדבר (זבחים כח׃), ובמסכת כריתות (ה.) למדוהו מגזרה שוה׃}
\threeverse{\arabic{verse}}%Leviticus19:9
{וּֽבְקֻצְרְכֶם֙ אֶת\maqqaf קְצִ֣יר אַרְצְכֶ֔ם לֹ֧א תְכַלֶּ֛ה פְּאַ֥ת שָׂדְךָ֖ לִקְצֹ֑ר וְלֶ֥קֶט קְצִֽירְךָ֖ לֹ֥א תְלַקֵּֽט׃}
{וּבְמִחְצַדְכוֹן יָת חֲצָדָא דַּאֲרַעְכוֹן לָא תְשֵׁיצֵי פָּתָא דְּחַקְלָךְ לְמִחְצַד וּלְקָטָא לָא תְלַקֵּיט׃}
{And when ye reap the harvest of your land, thou shalt not wholly reap the corner of thy field, neither shalt thou gather the gleaning of thy harvest.}{\arabic{verse}}
\rashi{\rashiDH{לא תכלה פאת שדך.} שיניח פאה בסוף שדהו (ת״כ פרק א, ט)׃\quad \rashiDH{ולקט קצירך.} שִׁבֳּלִים הנושרים בשעת קצירה אחת או שתים, אבל שלש אינן לקט (פאה פ״ו, מ״ה)׃}
\threeverse{\arabic{verse}}%Leviticus19:10
{וְכַרְמְךָ֙ לֹ֣א תְעוֹלֵ֔ל וּפֶ֥רֶט כַּרְמְךָ֖ לֹ֣א תְלַקֵּ֑ט לֶֽעָנִ֤י וְלַגֵּר֙ תַּעֲזֹ֣ב אֹתָ֔ם אֲנִ֖י יְהֹוָ֥ה אֱלֹהֵיכֶֽם׃}
{וְכַרְמָךְ לָא תְעָלֵיל וְנִתְרָא דְּכַרְמָךְ לָא תְלַקֵּיט לְעַנְיֵי וּלְגִיּוֹרֵי תִּשְׁבּוֹק יָתְהוֹן אֲנָא יְיָ אֱלָהֲכוֹן׃}
{And thou shalt not glean thy vineyard, neither shalt thou gather the fallen fruit of thy vineyard; thou shalt leave them for the poor and for the stranger: I am the \lord\space your God.}{\arabic{verse}}
\rashi{\rashiDH{לא תעולל.} לא תטול עוללות שבה והן ניכרות. איזהו עוללות, כל שאין לה לא כָּתֵף ולא נָטֵף (פאה פ״ז, מ״ד)׃\quad \rashiDH{ופרט כרמך.} גרגרי ענבים הנושרים בשעת בצירה׃\quad \rashiDH{אני ה׳ אלהיכם.} דיין להפרע, ואיני גובה מכם אלא נפשות, שנאמר אַל תִּגְזָל דָּל וגו׳, כִּי ה׳ יָרִיב רִיבָם וגו׳ (משלי כב, כבכג)׃}
\threeverse{\aliya{ישראל}}%Leviticus19:11
{לֹ֖א תִּגְנֹ֑בוּ וְלֹא\maqqaf תְכַחֲשׁ֥וּ וְלֹֽא\maqqaf תְשַׁקְּר֖וּ אִ֥ישׁ בַּעֲמִיתֽוֹ׃}
{לָא תִגְנְבוּן וְלָא תְכַדְּבוּן וְלָא תְשַׁקְּרוּן אֱנָשׁ בְּחַבְרֵיהּ׃}
{Ye shall not steal; neither shall ye deal falsely, nor lie one to another.}{\arabic{verse}}
\rashi{\rashiDH{לא תגנבו.} אזהרה לגונב ממון, אבל לא תגנוב שבעשרת הדברות אזהרה לגונב נפשות, דבר הלמד מענינו, דבר שחייבין עליו מיתת ב״ד (סנהדרין פו.)׃\quad \rashiDH{ולא תכחשו}. לפי שנאמר וְכִחֶשׁ בָּהּ (ויקרא ה, כב), משלם קרן וחומש, למדנו עונש, אזהרה מנין, תלמוד לומר ולא תכחשו׃ 
\quad \rashiDH{ולא תשקרו.} לפי שנאמר וְנִשְׁבַּע עַל שָׁקֶר (שם), ישלם קרן וחומש, למדנו עונש, אזהרה מנין, תלמוד לומר ולא תשקרו׃\quad \rashiDH{לא תגנבו ולא תכחשו ולא תשקרו ולא תשבעו.} אם גנבת סופך לכחש סופך לשקר סופך להשבע לשקר׃}
\threeverse{\arabic{verse}}%Leviticus19:12
{וְלֹֽא\maqqaf תִשָּׁבְע֥וּ בִשְׁמִ֖י לַשָּׁ֑קֶר וְחִלַּלְתָּ֛ אֶת\maqqaf שֵׁ֥ם אֱלֹהֶ֖יךָ אֲנִ֥י יְהֹוָֽה׃}
{וְלָא תִשְׁתַּבְעוּן בִּשְׁמִי לְשִׁקְרָא וְתַחֵיל יָת שְׁמָא דֶּאֱלָהָךְ אֲנָא יְיָ׃}
{And ye shall not swear by My name falsely, so that thou profane the name of thy God: I am the \lord.}{\arabic{verse}}
\rashi{\rashiDH{ולא תשבעו בשמי.} למה נאמר, לפי שנאמר לֹא תִשָּׂא אֶת שֵׁם ה׳ אֱלֹהֶיךָ לַשָּׁוְא (שמות כ, ז), יכול לא יהא חייב אלא על שם המיוחד, מנין לרבות כל הכנויין, ת״ל ולא תשבעו בשמי לשקר, כל שם שיש לי׃}
\threeverse{\arabic{verse}}%Leviticus19:13
{לֹֽא\maqqaf תַעֲשֹׁ֥ק אֶת\maqqaf רֵֽעֲךָ֖ וְלֹ֣א תִגְזֹ֑ל לֹֽא\maqqaf תָלִ֞ין פְּעֻלַּ֥ת שָׂכִ֛יר אִתְּךָ֖ עַד\maqqaf בֹּֽקֶר׃}
{לָא תַעְשׁוֹק יָת חַבְרָךְ וְלָא תֵינוֹס לָא תְבִית אַגְרָא דַּאֲגִירָא לְוָתָךְ עַד צַפְרָא׃}
{Thou shalt not oppress thy neighbour, nor rob him; the wages of a hired servant shall not abide with thee all night until the morning.}{\arabic{verse}}
\rashi{\rashiDH{לא תעשק.} זה הכובש שכר שכיר (ת״כ פרשתא ב)׃\quad \rashiDH{לא תלין.} לשון נקבה, מוסב על הפעולה׃\quad \rashiDH{עד בקר.} בשכיר יום הכתוב מדבר, שיציאתו מששקעה חמה, לפיכך זמן גִּבּוּי שכרו כל הלילה, ובמקום אחר הוא אומר וְלֹא תָבוֹא עָלָיו הַשֶּׁמֶשׁ (דברים כד, טו), מדבר בשכיר לילה (בבא מציעא קי׃), שהשלמת פעולתו משיעלה עמוד השחר, לפיכך זמן גִּבּוּי שכרו כל היום, לפי שנתנה תורה זמן לבעל הבית עונה, לבקש מעות׃}
\threeverse{\arabic{verse}}%Leviticus19:14
{לֹא\maqqaf תְקַלֵּ֣ל חֵרֵ֔שׁ וְלִפְנֵ֣י עִוֵּ֔ר לֹ֥א תִתֵּ֖ן מִכְשֹׁ֑ל וְיָרֵ֥אתָ מֵּאֱלֹהֶ֖יךָ אֲנִ֥י יְהֹוָֽה׃}
{לָא תְלוּט דְּלָא שָׁמַע וּקְדָם דְּלָא חָזֵי לָא תְשִׂים תַּקְלָא וְתִדְחַל מֵאֱלָהָךְ אֲנָא יְיָ׃}
{Thou shalt not curse the deaf, nor put a stumbling-block before the blind, but thou shalt fear thy God: I am the \lord.}{\arabic{verse}}
\rashi{\rashiDH{לא תקלל חרש.} אין לי אלא חרש, מנין לרבות כל אדם, תלמוד לומר בְעַמְּךָ לֹא תָאֹר (שמות כב, כז), אם כן למה נאמר חרש, מה חרש מיוחד שהוא בחיים אף כל שהוא בחיים, יצא המת שאינו בחיים (ת״כ שם יג)׃\quad \rashiDH{ולפני עור לא תתן מכשול.} לפני הסומא בדבר לא תתן עצה שאינה הוגנת לו, אל תאמר מכור שדך וקח לך חמור, ואתה עוקף עליו ונוטלה הימנו (שם יד)׃\quad \rashiDH{ויראת מאלהיך.} לפי שהדבר הזה אינו מסור לבריות לידע אם דעתו של זה לטובה או לרעה, ויכול להשמט ולומר לטובה נתכוונתי, לפיכך נאמר בו ויראת מאלהיך, המכיר מחשבותיך. וכן כל דבר המסור ללבו של אדם העושהו, ואין שאר הבריות מכירות בו, נאמר בו ויראת מאלהיך׃ 
}
\aliyacounter{שני}
\threeverse{\aliya{שני\newline (חמישי)}}%Leviticus19:15
{לֹא\maqqaf תַעֲשׂ֥וּ עָ֙וֶל֙ בַּמִּשְׁפָּ֔ט לֹא\maqqaf תִשָּׂ֣א פְנֵי\maqqaf דָ֔ל וְלֹ֥א תֶהְדַּ֖ר פְּנֵ֣י גָד֑וֹל בְּצֶ֖דֶק תִּשְׁפֹּ֥ט עֲמִיתֶֽךָ׃}
{לָא תַעְבְּדוּן שְׁקַר בְּדִין לָא תִּסַּב אַפֵּי מִסְכֵּינָא וְלָא תֶהְדַּר אַפֵּי רַבָּא בְּקוּשְׁטָא תְּדִינֵיהּ לְחַבְרָךְ׃}
{Ye shall do no unrighteousness in judgment; thou shalt not respect the person of the poor, nor favour the person of the mighty; but in righteousness shalt thou judge thy neighbour.}{\arabic{verse}}
\rashi{\rashiDH{לא תעשו עול במשפט.} מלמד שהדיין המקלקל את הדין קרוי עול, שנאוי, ומשוקץ, חרם, ותועבה. שהעול קרוי תועבה, שנאמר כִּי תוֹעֲבַת ה׳ וגו׳ כֹּל עֹשֵׂה עָוֶל (דברים כה, טז), והתועבה קרויה שקץ וחרם, שנאמר וְלֹא תָבִיא תֹועֵבָה אֶל בֵּיתֶךָ וְהָיִיתָ חֵרֶם כָּמֹהוּ שַׁקֵּץ תְשַׁקְּצֶנוּ וגו׳ (שם ז, כו)׃\quad \rashiDH{לא תשא פני דל.} שלא תאמר עני הוא זה והעשיר חייב לפרנסו אזכנו בדין ונמצא מתפרנס בנקיות (ת״כ פרק ד, ב)׃\quad \rashiDH{ולא תהדר פני גדול.} שלא תאמר עשיר הוא זה, בן גדולים הוא זה, היאך אביישנו ואראה בבושתו, עונש יש בדבר, לכך נאמר, ולא תהדר פני גדול׃\quad \rashiDH{בצדק תשפוט עמיתך.} כמשמעו. דבר אחר הוי דן את חבירך לכף זכות׃}
\threeverse{\arabic{verse}}%Leviticus19:16
{לֹא\maqqaf תֵלֵ֤ךְ רָכִיל֙ בְּעַמֶּ֔יךָ לֹ֥א תַעֲמֹ֖ד עַל\maqqaf דַּ֣ם רֵעֶ֑ךָ אֲנִ֖י יְהֹוָֽה׃}
{לָא תֵיכוֹל קוּרְצִין בְּעַמָּךְ לָא תְקוּם עַל דְּמָא דְּחַבְרָךְ אֲנָא יְיָ׃}
{Thou shalt not go up and down as a talebearer among thy people; neither shalt thou stand idly by the blood of thy neighbour: I am the \lord.}{\arabic{verse}}
\rashi{\rashiDH{לא תלך רכיל.} אני אומר על שם שכל משלחי מדנים ומספרי לשון הרע הולכים בבתי רעיהם לְרַגֵּל מה יראו רע או מה ישמעו רע לספר בשוק, נקראים הולכי רכיל, הולכי רגילה, אשפיי״מנט בלע״ז. וראיה לדברי שלא מצינו רכילות שאין כתוב בלשון הליכה. לא תלך רכיל, הֹלְכֵי רָכִיל נְחשֶׁת וּבַרְזֶל (ירמיה ו, כח), ושאר לשון הרע אין כתוב בו הליכה, מְלָשְׁנִי בַסֵּתֶר רֵעֵהוּ (תהלים קא, ה), לָשׁוֹן רְמִיָּה (שם קכ, ב), לָשׁוֹן מְדַבֶּרֶת גְּדֹלוֹת (שם יב, ד), לכך אני אומר, שהלשון הולך ומרגל, שהכ״ף נחלפת בגימ״ל, שכל האותיות שמוצאיהם ממקום אחד מתחלפות זו בזו, בי״ת בפ״א, וגימ״ל בכ״ף וקו״ף, ונו״ן בלמ״ד, וזי״ן בצד״י, וכן וַיְרַגֵּל בְּעַבְדְּךָ (שמואל־ ב יט, כח), רגל במרמה לאמר עלי רעה, וכן לֹא רָגַל עַל לְשֹׁנוֹ (תהלים טו, ג), וכן רוכל הסוחר ומרגל אחר כל סחורה, וכל המוכר בשמים להתקשט בהם הנשים, על שם שֶׁמְחַזֵר תמיד בעיירות נקרא רוכל, לשון רוגל. ותרגומו לָא תֵיכוּל קוּרְצִין, כמו וַאֲכַלוּ קַרְצֵיהוֹן דִּי יְהוּדָאֵי (דניאל ג, ח), אָכַל בְּהוּ קוּרְצָא בֵּי מַלְכָּא (ברכות נח.), נראה בעיני שהיה משפטם לאכול בבית המקבל דבריהם שום הלעטה, והוא גמר חזוק שדבריו מקויימים ומעמידם על האמת, ואותה הלעטה נקראת אכילת קורצין, לשון קֹרֵץ בְּעֵינָיו (משלי ו, יג), שכן דרך כל הולכי רכיל לקרוץ בעיניהם ולרמוז דברי רכילותן שלא יבינו שאר השומעים׃\quad \rashiDH{לא תעמוד על דם רעך.} לראות במיתתו, ואתה יכול להצילו, כגון טובע בנהר, וחיה או לסטים באים עליו (ת״כ פרק ד, ח  סנהדרין עג.)׃\quad \rashiDH{אני ה׳.} נאמן לשלם שכר, ונאמן להפרע׃ 
}
\threeverse{\arabic{verse}}%Leviticus19:17
{לֹֽא\maqqaf תִשְׂנָ֥א אֶת\maqqaf אָחִ֖יךָ בִּלְבָבֶ֑ךָ הוֹכֵ֤חַ תּוֹכִ֙יחַ֙ אֶת\maqqaf עֲמִיתֶ֔ךָ וְלֹא\maqqaf תִשָּׂ֥א עָלָ֖יו חֵֽטְא׃}
{לָא תִשְׂנֵי יָת אֲחוּךְ בְּלִבָּךְ אוֹכָחָא תּוֹכַח יָת חַבְרָךְ וְלָא תְקַבֵּיל עַל דִּילֵיהּ חוֹבָא׃}
{Thou shalt not hate thy brother in thy heart; thou shalt surely rebuke thy neighbour, and not bear sin because of him.}{\arabic{verse}}
\rashi{\rashiDH{ולא תשא עליו חטא.} לא תלבין את פניו ברבים (ערכין טז׃)׃}
\threeverse{\arabic{verse}}%Leviticus19:18
{לֹֽא\maqqaf תִקֹּ֤ם וְלֹֽא\maqqaf תִטֹּר֙ אֶת\maqqaf בְּנֵ֣י עַמֶּ֔ךָ וְאָֽהַבְתָּ֥ לְרֵעֲךָ֖ כָּמ֑וֹךָ אֲנִ֖י יְהֹוָֽה׃}
{לָא תִקּוֹם וְלָא תִטַּר דְּבָבוּ לִבְנֵי עַמָּךְ וְתִרְחֲמֵיהּ לְחַבְרָךְ כְּוָתָךְ אֲנָא יְיָ׃}
{Thou shalt not take vengeance, nor bear any grudge against the children of thy people, but thou shalt love thy neighbour as thyself: I am the \lord.}{\arabic{verse}}
\rashi{\rashiDH{לא תקום.} אמר לו השאילני מַגְלָךְ, אמר לו לאו, למחר אמר לו השאילני קרדומך, אמר לו איני משאילך כדרך שלא השאלתני, זו היא נקימה. ואיזו היא נטירה, אמר לו השאילני קרדומך, אמר לו לאו, למחר אמר לו השאילני מגלך, אמר לו הא לך ואיני כמותך שלא השאלתני, זו היא נטירה, שנוטר האיבה בלבו אף על פי שאינו נוקם (יומא כג.)׃\quad \rashiDH{ואהבת לרעך כמוך.} אמר ר׳ עקיבא זה כלל גדול בתורה (ת״כ שם יב)׃}
\threeverse{\arabic{verse}}%Leviticus19:19
{אֶֽת\maqqaf חֻקֹּתַי֮ תִּשְׁמֹ֒רוּ֒ בְּהֶמְתְּךָ֙ לֹא\maqqaf תַרְבִּ֣יעַ כִּלְאַ֔יִם שָׂדְךָ֖ לֹא\maqqaf תִזְרַ֣ע כִּלְאָ֑יִם וּבֶ֤גֶד כִּלְאַ֙יִם֙ שַֽׁעַטְנֵ֔ז לֹ֥א יַעֲלֶ֖ה עָלֶֽיךָ׃}
{יָת קְיָמַי תִּטְּרוּן בְּעִירָךְ לָא תַרְכֵּיב עֵירוּבִין חַקְלָךְ לָא תִזְרַע עֵירוּבִין וּלְבוּשׁ עֵירוּבִין שַׁעַטְנֵיזָא לָא יִסַּק עֲלָךְ׃}
{Ye shall keep My statutes. Thou shalt not let thy cattle gender with a diverse kind; thou shalt not sow thy field with two kinds of seed; neither shall there come upon thee a garment of two kinds of stuff mingled together.}{\arabic{verse}}
\rashi{\rashiDH{את חקתי תשמרו.} ואלו הן בהמתך לא תרביע כלאים וגו׳, חקים אלו גזרות מלך שאין טעם לדבר׃\quad \rashiDH{ובגד כלאים.} למה נאמר, לפי שנאמר לֹא תִלְבַּשׁ שַׁעַטְנֵז צֶמֶר וּפִשְׁתִּים יַחְדָּו (דברים כב, יא), יכול לא ילבש גיזי צמר ואניצי פשתן, תלמוד לומר בגד, מנין לרבות הלבדים, למוד לומר שעטנז, דבר שהוא שׁוּעַ טָווּי וְנוּז, ואומר אני, נוז לשון דבר הנמלל ושזור זה עם זה לחברו, מישטי״ר בלע״ז, כמו חַזְיָין לְנַאזִי דְאִית בְּהוֹן (מועד קטן יב׃), שאנו מפרשין לשון כמוש, פלישטר״א, ולשון שעטנז פירש מנחם, מחברת צמר ופשתים׃ 
}
\threeverse{\arabic{verse}}%Leviticus19:20
{וְ֠אִ֠ישׁ כִּֽי\maqqaf יִשְׁכַּ֨ב אֶת\maqqaf אִשָּׁ֜ה שִׁכְבַת\maqqaf זֶ֗רַע וְהִ֤וא שִׁפְחָה֙ נֶחֱרֶ֣פֶת לְאִ֔ישׁ וְהׇפְדֵּה֙ לֹ֣א נִפְדָּ֔תָה א֥וֹ חֻפְשָׁ֖ה לֹ֣א נִתַּן\maqqaf לָ֑הּ בִּקֹּ֧רֶת תִּהְיֶ֛ה לֹ֥א יוּמְת֖וּ כִּי\maqqaf לֹ֥א חֻפָּֽשָׁה׃}
{וּגְבַר אֲרֵי יִשְׁכּוֹב יָת אִתְּתָא שִׁכְבַת זַרְעָא וְהִיא אָמָא אֲחִידָא לִגְבַר וְאִתְפְּרָקָא לָא אִתְפְּרֵיקַת בְּכַסְפָּא אוֹ חֵירוּתָא לָא אִתְיְהֵיבַת לַהּ בִּשְׁטָר בִּקּוּרְתָּא תְּהֵי בַהּ לָא יְמוּתוּן אֲרֵי לָא אִתְחָרַרַת׃}
{And whosoever lieth carnally with a woman, that is a bondmaid, designated for a man, and not at all redeemed, nor was freedom given her; there shall be inquisition; they shall not be put to death, because she was not free.}{\arabic{verse}}
\rashi{\rashiDH{נחרפת לאיש.} מיועדת ומיוחדת לאיש, ואיני יודע לו דמיון במקרא. ובשפחה כנענית שחציה שפחה וחציה בת חורין המאורסת לעבד עברי שמותר בשפחה הכתוב מדבר (כריתות יא.)׃\quad \rashiDH{והפדה לא נפדתה.} פדויה ואינה פדויה, וסתם פדיון בכסף (ת״כ פרק ה, ג  גיטין לט.)׃\quad \rashiDH{או חפשה.} בשטר׃\quad \rashiDH{בקרת תהיה.} היא לוקה ולא הוא. יש על בית דין לבקר את הדבר שלא לחייבו מיתה, כי לא חפשה, ואין קידושיה קידושין גמורין. ורבותינו (כריתות יא.) למדו מכאן שמי שהוא במלקות יהא בקריאה, שהדיינים המלקין קורין על הלוקה, אִם לֹא תִשְׁמֹר לַעֲשׂוֹת וגו׳ (דברים כח, נח), וְהִפְלָא ה׳ אֶת מַכֹּתְךָ וגו׳ (שם נט)׃\quad \rashiDH{כי לא חפשה.} לפיכך אין חייב עליה מיתה, שאין קידושיה קידושין, הא אם חופשה, קידושיה קידושין וחייב מיתה (גיטין מג׃)׃ 
}
\threeverse{\arabic{verse}}%Leviticus19:21
{וְהֵבִ֤יא אֶת\maqqaf אֲשָׁמוֹ֙ לַֽיהֹוָ֔ה אֶל\maqqaf פֶּ֖תַח אֹ֣הֶל מוֹעֵ֑ד אֵ֖יל אָשָֽׁם׃}
{וְיַיְתֵי יָת אֲשָׁמֵיהּ לִקְדָם יְיָ לִתְרַע מַשְׁכַּן זִמְנָא דִּכְרָא לַאֲשָׁמָא׃}
{And he shall bring his forfeit unto the \lord, unto the door of the tent of meeting, even a ram for a guilt-offering.}{\arabic{verse}}
\threeverse{\arabic{verse}}%Leviticus19:22
{וְכִפֶּר֩ עָלָ֨יו הַכֹּהֵ֜ן בְּאֵ֤יל הָֽאָשָׁם֙ לִפְנֵ֣י יְהֹוָ֔ה עַל\maqqaf חַטָּאת֖וֹ אֲשֶׁ֣ר חָטָ֑א וְנִסְלַ֣ח ל֔וֹ מֵחַטָּאת֖וֹ אֲשֶׁ֥ר חָטָֽא׃ \petucha }
{וִיכַפַּר עֲלוֹהִי כָּהֲנָא בְּדִכְרָא דַּאֲשָׁמָא קֳדָם יְיָ עַל חוֹבְתֵיהּ דְּחָב וְיִשְׁתְּבֵיק לֵיהּ מֵחוֹבְתֵיהּ דְּחָב׃}
{And the priest shall make atonement for him with the ram of the guilt-offering before the \lord\space for his sin which he hath sinned; and he shall be forgiven for his sin which he hath sinned.}{\arabic{verse}}
\rashi{\rashiDH{ונסלח לו מחטאתו אשר חטא.} לרבות את המזיד כשוגג (כריתות ט.)׃}
\aliyacounter{שלישי}
\newseder{16}
\threeverse{\aliya{שלישי}\newline\vspace{-4pt}\newline\seder{טז}}%Leviticus19:23
{וְכִי\maqqaf תָבֹ֣אוּ אֶל\maqqaf הָאָ֗רֶץ וּנְטַעְתֶּם֙ כׇּל\maqqaf עֵ֣ץ מַאֲכָ֔ל וַעֲרַלְתֶּ֥ם עׇרְלָת֖וֹ אֶת\maqqaf פִּרְי֑וֹ שָׁלֹ֣שׁ שָׁנִ֗ים יִהְיֶ֥ה לָכֶ֛ם עֲרֵלִ֖ים לֹ֥א יֵאָכֵֽל׃}
{וַאֲרֵי תֵיעֲלוּן לְאַרְעָא וְתִצְּבוּן כָּל אִילָן דְּמֵיכַל וּתְרַחֲקוּן רַחָקָא יָת אִבֵּיהּ תְּלָת שְׁנִין יְהֵי לְכוֹן מְרַחַק לְאַבָּדָא לָא יִתְאֲכִיל׃}
{And when ye shall come into the land, and shall have planted all manner of trees for food, then ye shall count the fruit thereof as forbidden; three years shall it be as forbidden unto you; it shall not be eaten.}{\arabic{verse}}
\rashi{\rashiDH{וערלתם ערלתו.} ואטמתם אטימתו, יהא אטום ונסתם מליהנות ממנו׃\quad \rashiDH{שלש שנים יהיה לכם ערלים.} מאימתי מונה לו, משעת נטיעתו (ת״כ פרשתא ג. ג), יכול אם הצניעו לאחר שלש שנים יהא מותר, תלמוד לומר יהיה, בהוייתו יהא (שם ד)׃}
\threeverse{\arabic{verse}}%Leviticus19:24
{וּבַשָּׁנָה֙ הָרְבִיעִ֔ת יִהְיֶ֖ה כׇּל\maqqaf פִּרְי֑וֹ קֹ֥דֶשׁ הִלּוּלִ֖ים לַיהֹוָֽה׃}
{וּבְשַׁתָּא רְבִיעֵיתָא יְהֵי כָּל אִבֵּיהּ קוֹדֶשׁ תּוּשְׁבְּחָן קֳדָם יְיָ׃}
{And in the fourth year all the fruit thereof shall be holy, for giving praise unto the \lord.}{\arabic{verse}}
\rashi{\rashiDH{יהיה כל פריו קדש.} כמעשר שני (קידושין נד׃) שכתוב בו וכל מעשר הארץ וגו׳ קדש לה׳ (ויקרא כז, ל), מה מעשר שני אינו נאכל חוץ לחומת ירושלים, אלא בפדיון, אף זה כן. ודבר זה הלולים לה׳ הוא, שנושאו שם לשבח ולהלל לשמים׃}
\threeverse{\arabic{verse}}%Leviticus19:25
{וּבַשָּׁנָ֣ה הַחֲמִישִׁ֗ת תֹּֽאכְלוּ֙ אֶת\maqqaf פִּרְי֔וֹ לְהוֹסִ֥יף לָכֶ֖ם תְּבוּאָת֑וֹ אֲנִ֖י יְהֹוָ֥ה אֱלֹהֵיכֶֽם׃}
{וּבְשַׁתָּא חֲמִישֵׁיתָא תֵּיכְלוּן יָת אִבֵּיהּ לְאוֹסָפָא לְכוֹן עֲלַלְתֵּיהּ אֲנָא יְיָ אֱלָהֲכוֹן׃}
{But in the fifth year may ye eat of the fruit thereof, that it may yield unto you more richly the increase thereof: I am the \lord\space your God.}{\arabic{verse}}
\rashi{\rashiDH{להוסיף לכם תבואתו.} המצוה הזאת שתשמרו תהיה להוסיף לכם תבואתו, שבשכרה אני מברך לכם פירות הנטיעות, היה ר׳ עקיבא אומר דברה תורה כנגד יצר הרע, שלא יאמר אדם הרי ארבע שנים אני מצטער בו חנם, לפיכך נאמר להוסיף לכם תבואתו׃\quad \rashiDH{אני ה׳.} אני ה׳ המבטיח על כן, ונאמן לשמור הבטחתי׃}
\threeverse{\arabic{verse}}%Leviticus19:26
{לֹ֥א תֹאכְל֖וּ עַל\maqqaf הַדָּ֑ם לֹ֥א תְנַחֲשׁ֖וּ וְלֹ֥א תְעוֹנֵֽנוּ׃}
{לָא תֵיכְלוּן עַל דְּמָא לָא תְנַחֲשׁוּן וְלָא תְעָנוֹן׃}
{Ye shall not eat with the blood; neither shall ye practise divination nor soothsaying.}{\arabic{verse}}
\rashi{\rashiDH{לא תאכלו על הדם.} להרבה פנים נדרש בסנהדרין, (סג.) אזהרה שלא יאכל מבשר קדשים לפני זריקת דמים, ואזהרה לאוכל מבהמת חולין טרם שתצא נפשה, ועוד הרבה׃ 
\quad \rashiDH{לא תנחשו.} כגון אלו המנחשין בחולדה ובעופות, פתו נפלה מפיו, צבי הפסיקו בדרך (סנהדרין סה׃)׃\quad \rashiDH{לא תעוננו.} לשון עונות ושעות, שאומר יום פלוני יפה להתחיל מלאכה, שעה פלונית קשה לצאת (סנהדרין סו.)׃}
\threeverse{\arabic{verse}}%Leviticus19:27
{לֹ֣א תַקִּ֔פוּ פְּאַ֖ת רֹאשְׁכֶ֑ם וְלֹ֣א תַשְׁחִ֔ית אֵ֖ת פְּאַ֥ת זְקָנֶֽךָ׃}
{לָא תַקְּפוּן פָּתָא דְּרֵישְׁכוֹן וְלָא תְחַבֵּיל יָת פָּתָא דְּדִקְנָךְ׃}
{Ye shall not round the corners of your heads, neither shalt thou mar the corners of thy beard.}{\arabic{verse}}
\rashi{\rashiDH{לא תקיפו פאת ראשכם.} זה המשוה צדעיו לאחורי אזנו ולפדחתו (מכות כ׃), ונמצא הקף ראשו עגול סביב, שעל אחורי אזניו עקרי שערו למעלה מצדעיו הרבה׃ 
\quad \rashiDH{פאת זקנך.} סוף הזקן וגבוליו, והן חמש, שתים בכל לחי, ולחי למעלה אצל הראש שהוא רחב ויש בו שתי פאות, ואחת למטה בסנטרו מקום חבור שני הלחיים יחד (שם כ.)׃ 
}
\threeverse{\arabic{verse}}%Leviticus19:28
{וְשֶׂ֣רֶט לָנֶ֗פֶשׁ לֹ֤א תִתְּנוּ֙ בִּבְשַׂרְכֶ֔ם וּכְתֹ֣בֶת קַֽעֲקַ֔ע לֹ֥א תִתְּנ֖וּ בָּכֶ֑ם אֲנִ֖י יְהֹוָֽה׃}
{וְחִבּוּל עַל מִית לָא תִתְּנוּן בִּבְשַׂרְכוֹן וְרוּשְׁמִין חֲרִיתִין לָא תִתְּנוּן בְּכוֹן אֲנָא יְיָ׃}
{Ye shall not make any cuttings in your flesh for the dead, nor imprint any marks upon you: I am the \lord.}{\arabic{verse}}
\rashi{\rashiDH{ושרט לנפש.} כן דרכן של אמוריים, להיות משרטין בשרם כשמת להם מת׃\quad \rashiDH{וכתבת קעקע.} כתב המחוקה ושקוע שאינו נמחק לעולם, שמקעקעו במחט והוא משחיר לעולם (שם כא.)׃\quad \rashiDH{קעקע.} לשון וְהוֹקַע אוֹתָם (במדבר כה, ד), וְהוֹקַעְנוּם (שמואל־ב כא, ו), תוחבין עץ בארץ ותולין אותם עליהם, ונמצאו מחוקין ותחובין בקרקע, פורפו״יינט בלע״ז׃}
\threeverse{\arabic{verse}}%Leviticus19:29
{אַל\maqqaf תְּחַלֵּ֥ל אֶֽת\maqqaf בִּתְּךָ֖ לְהַזְנוֹתָ֑הּ וְלֹא\maqqaf תִזְנֶ֣ה הָאָ֔רֶץ וּמָלְאָ֥ה הָאָ֖רֶץ זִמָּֽה׃}
{לָא תַחֵיל יָת בְּרַתָּךְ לְאַטְעָיוּתַהּ וְלָא תִטְעֵי אַרְעָא וְתִתְמְלֵי אַרְעָא עֵיצַת חֲטִאין׃}
{Profane not thy daughter, to make her a harlot, lest the land fall into harlotry, and the land become full of lewdness.}{\arabic{verse}}
\rashi{\rashiDH{אל תחלל את בתך להזנותה.} במוסר בתו פנויה לביאה שלא לשם קידושין (סנהדרין עו.)׃ 
\quad \rashiDH{ולא תזנה הארץ.} אם אתה עושה כן הארץ מזנה את פירותיה לעשותן במקום אחר ולא בארצכם, וכן הוא אומר וַיִּמָּנְעוּ רְבִיבִים וגו׳ (ירמיה ג, ג)׃}
\threeverse{\arabic{verse}}%Leviticus19:30
{אֶת\maqqaf שַׁבְּתֹתַ֣י תִּשְׁמֹ֔רוּ וּמִקְדָּשִׁ֖י תִּירָ֑אוּ אֲנִ֖י יְהֹוָֽה׃}
{יָת יוֹמֵי שַׁבַּיָּא דִּילִי תִּטְּרוּן וּלְבֵית מַקְדְּשִׁי תְּהוֹן דָּחֲלִין אֲנָא יְיָ׃}
{Ye shall keep My sabbaths, and reverence My sanctuary: I am the \lord.}{\arabic{verse}}
\rashi{\rashiDH{ומקדשי תיראו.} לא יכנס לא במקלו ולא במנעלו ובאפונדתו ובאבק שעל רגליו (יבמות ו׃). ואף על פי שאני מזהירכם על המקדש את שבתותי תשמורו, אין בנין בית המקדש דוחה שבת (יבמות ו.)׃ 
}
\threeverse{\arabic{verse}}%Leviticus19:31
{אַל\maqqaf תִּפְנ֤וּ אֶל\maqqaf הָאֹבֹת֙ וְאֶל\maqqaf הַיִּדְּעֹנִ֔ים אַל\maqqaf תְּבַקְשׁ֖וּ לְטׇמְאָ֣ה בָהֶ֑ם אֲנִ֖י יְהֹוָ֥ה אֱלֹהֵיכֶֽם׃}
{לָא תִתְפְּנוֹן בָּתַר בִּדִּין וּזְכוּרוּ לָא תִתְבְּעוּן לְאִסְתַּאָבָא בְּהוֹן אֲנָא יְיָ אֱלָהֲכוֹן׃}
{Turn ye not unto the ghosts, nor unto familiar spirits; seek them not out, to be defiled by them: I am the \lord\space your God.}{\arabic{verse}}
\rashi{\rashiDH{אל תפנו אל האובות.} אזהרה לבעל אוב וידעוני (סנהדרין סה.). בעל אוב זה פיתום המדבר משחיו, וידעוני מכניס עצם חיה ששמה ידוע לתוך פיו והעצם מדבר׃\quad \rashiDH{אל תבקשו.} להיות עסוקים בם, שאם תעסקו בם אתם מִטַּמְּאִין לפני ואני מתעב אתכם׃\quad \rashiDH{אני ה׳ אלהיכם.} דעו את מי אתם מחליפין במי׃}
\threeverse{\arabic{verse}}%Leviticus19:32
{מִפְּנֵ֤י שֵׂיבָה֙ תָּק֔וּם וְהָדַרְתָּ֖ פְּנֵ֣י זָקֵ֑ן וְיָרֵ֥אתָ מֵּאֱלֹהֶ֖יךָ אֲנִ֥י יְהֹוָֽה׃ \setuma }
{מִן קֳדָם דְּסָבַר בְּאוֹרָיְתָא תְּקוּם וְתֶהְדַּר אַפֵּי סָבָא וְתִדְחַל מֵאֱלָהָךְ אֲנָא יְיָ׃}
{Thou shalt rise up before the hoary head, and honour the face of the old man, and thou shalt fear thy God: I am the \lord.}{\arabic{verse}}
\rashi{\rashiDH{מפני שיבה תקום.} יכול זקן אשמאי, תלמוד לומר זקן, אין זקן אלא שקנה חכמה (ת״כ פרק ז. יב  קידושין לב׃)׃\quad \rashiDH{והדרת פני זקן.} איזהו הדור, לא ישב במקומו, ולא יסתור את דבריו. יכול יעצים עיניו כמי שלא ראהו, לכך נאמר ויראת מאלהיך, שהרי דבר זה מסור ללבו של עושהו שאין מכיר בו אלא הוא, וכל דבר המסור ללב נאמר בו ויראת מאלהיך (שם)׃}
\aliyacounter{רביעי}
\threeverse{\aliya{רביעי\newline (ששי)}}%Leviticus19:33
{וְכִֽי\maqqaf יָג֧וּר אִתְּךָ֛ גֵּ֖ר בְּאַרְצְכֶ֑ם לֹ֥א תוֹנ֖וּ אֹתֽוֹ׃}
{וַאֲרֵי יִתְגַּיַּיר עִמְּכוֹן גִּיּוֹרָא בַּאֲרַעְכוֹן לָא תוֹנוֹן יָתֵיהּ׃}
{And if a stranger sojourn with thee in your land, ye shall not do him wrong.}{\arabic{verse}}
\rashi{\rashiDH{לא תונו.} אונאת דברים, לא תאמר לו אמש היית עובד עבודת כוכבים, ועכשיו אתה בא ללמוד תורה שנתנה מפי הגבורה׃}
\threeverse{\arabic{verse}}%Leviticus19:34
{כְּאֶזְרָ֣ח מִכֶּם֩ יִהְיֶ֨ה לָכֶ֜ם הַגֵּ֣ר \legarmeh  הַגָּ֣ר אִתְּכֶ֗ם וְאָהַבְתָּ֥ לוֹ֙ כָּמ֔וֹךָ כִּֽי\maqqaf גֵרִ֥ים הֱיִיתֶ֖ם בְּאֶ֣רֶץ מִצְרָ֑יִם אֲנִ֖י יְהֹוָ֥ה אֱלֹהֵיכֶֽם׃}
{כְּיַצִּיבָא מִנְּכוֹן יְהֵי לְכוֹן גִּיּוֹרָא דְּיִתְגַּיַּיר עִמְּכוֹן וְתִרְחַם לֵיהּ כְּוָתָךְ אֲרֵי דַּיָּירִין הֲוֵיתוֹן בְּאַרְעָא דְּמִצְרָיִם אֲנָא יְיָ אֱלָהֲכוֹן׃}
{The stranger that sojourneth with you shall be unto you as the home-born among you, and thou shalt love him as thyself; for ye were strangers in the land of Egypt: I am the \lord\space your God.}{\arabic{verse}}
\rashi{\rashiDH{כי גרים הייתם.} מום שבך אל תאמר לחברך׃\quad \rashiDH{אני ה׳ אלהיכם.} אלהיך ואלהיו אני׃}
\threeverse{\arabic{verse}}%Leviticus19:35
{לֹא\maqqaf תַעֲשׂ֥וּ עָ֖וֶל בַּמִּשְׁפָּ֑ט בַּמִּדָּ֕ה בַּמִּשְׁקָ֖ל וּבַמְּשׂוּרָֽה׃}
{לָא תַעְבְּדוּן שְׁקַר בְּדִין בְּמִשְׁחֲתָא בְּמַתְקָלָא וּבִמְכִילְתָא׃}
{Ye shall do no unrighteousness in judgment, in meteyard, in weight, or in measure.}{\arabic{verse}}
\rashi{\rashiDH{לא תעשו עול במשפט.} אם לדין הרי כבר נאמר לֹא תַעֲשׂוּ עָוֶל בַּמִּשְׁפָּט (פסוק טו), ומהו משפט השנוי כאן, היא המדה והמשקל והמשורה. מלמד שהמודד נקרא דיין, שאם שיקר במדה הרי הוא כמקלקל את הדין, וקרוי עול שנאוי ומשוקץ חרם ותועבה, וגורם לחמשה דברים האמורים בדיין, מטמא את הארץ, ומחלל את השם, ומסלק את השכינה, ומפיל את ישראל בחרב, ומגלה אותם מארצם׃\quad \rashiDH{במדה.} זו מדת הארץ (ב״מ סא׃ בבא בתרא פט׃)׃ 
\quad \rashiDH{במשקל.} כמשמעו׃\quad \rashiDH{ובמשורה.} היא מדת הלח (והיבש)׃}
\threeverse{\arabic{verse}}%Leviticus19:36
{מֹ֧אזְנֵי צֶ֣דֶק אַבְנֵי\maqqaf צֶ֗דֶק אֵ֥יפַת צֶ֛דֶק וְהִ֥ין צֶ֖דֶק יִהְיֶ֣ה לָכֶ֑ם אֲנִי֙ יְהֹוָ֣ה אֱלֹֽהֵיכֶ֔ם אֲשֶׁר\maqqaf הוֹצֵ֥אתִי אֶתְכֶ֖ם מֵאֶ֥רֶץ מִצְרָֽיִם׃}
{מוֹזְנָוָאן דִּקְשׁוֹט מַתְקָלִין דִּקְשׁוֹט מְכִילָן דִּקְשׁוֹט וְהִינִין דִּקְשׁוֹט יְהוֹן לְכוֹן אֲנָא יְיָ אֱלָהֲכוֹן דְּאַפֵּיקִית יָתְכוֹן מֵאַרְעָא דְּמִצְרָיִם׃}
{Just balances, just weights, a just ephah, and a just hin, shall ye have: I am the \lord\space your God, who brought you out of the land of Egypt.}{\arabic{verse}}
\rashi{\rashiDH{אבני צדק.} הם המשקולות ששוקלין כנגדן׃\quad \rashiDH{איפת.} היא מדת היבש׃\quad \rashiDH{הין.} זו היא מדת הלח׃\quad \rashiDH{אשר הוצאתי אתכם.} על מנת כן. דבר אחר אני הבחנתי במצרים בין טפה של בכור לטפה שאינה של בכור, ואני הנאמן להפרע ממי שטומן משקלותיו במלח להונות את הבריות שאין מכירים בהם (ב״מ סא׃)׃}
\threeverse{\arabic{verse}}%Leviticus19:37
{וּשְׁמַרְתֶּ֤ם אֶת\maqqaf כׇּל\maqqaf חֻקֹּתַי֙ וְאֶת\maqqaf כׇּל\maqqaf מִשְׁפָּטַ֔י וַעֲשִׂיתֶ֖ם אֹתָ֑ם אֲנִ֖י יְהֹוָֽה׃ \petucha }
{וְתִטְּרוּן יָת כָּל קְיָמַי וְיָת כָּל דִּינַי וְתַעְבְּדוּן יָתְהוֹן אֲנָא יְיָ׃}
{And ye shall observe all My statutes, and all Mine ordinances, and do them: I am the \lord.}{\arabic{verse}}

\newperek
\aliyacounter{חמישי}
\threeverse{\aliya{חמישי}}%Leviticus20:1
{וַיְדַבֵּ֥ר יְהֹוָ֖ה אֶל\maqqaf מֹשֶׁ֥ה לֵּאמֹֽר׃}
{וּמַלֵּיל יְיָ עִם מֹשֶׁה לְמֵימַר׃}
{And the \lord\space spoke unto Moses, saying:}{\Roman{chap}}
\threeverse{\arabic{verse}}%Leviticus20:2
{וְאֶל\maqqaf בְּנֵ֣י יִשְׂרָאֵל֮ תֹּאמַר֒ אִ֣ישׁ אִישׁ֩ מִבְּנֵ֨י יִשְׂרָאֵ֜ל וּמִן\maqqaf הַגֵּ֣ר \legarmeh  הַגָּ֣ר בְּיִשְׂרָאֵ֗ל אֲשֶׁ֨ר יִתֵּ֧ן מִזַּרְע֛וֹ לַמֹּ֖לֶךְ מ֣וֹת יוּמָ֑ת עַ֥ם הָאָ֖רֶץ יִרְגְּמֻ֥הוּ בָאָֽבֶן׃}
{וְעִם בְּנֵי יִשְׂרָאֵל תֵּימַר גְּבַר גְּבַר מִבְּנֵי יִשְׂרָאֵל וּמִן גִּיּוֹרַיָּא דְּיִיתְגַּיְּרוּן בְּיִשְׂרָאֵל דְּיִתֵּין מִזַּרְעֵיהּ לְמֹלֶךְ אִתְקְטָלָא יִתְקְטִיל עַמָּא בֵּית יִשְׂרָאֵל יִרְגְּמוּנֵּיהּ בְּאַבְנָא׃}
{Moreover, thou shalt say to the children of Israel: Whosoever he be of the children of Israel, or of the strangers that sojourn in Israel, that giveth of his seed unto Molech; he shall surely be put to death; the people of the land shall stone him with stones.}{\arabic{verse}}
\rashi{\rashiDH{ואל בני ישראל תאמר.} עונשין על האזהרות׃\quad \rashiDH{מות יומת.} בבית דין, ואם אין כח לבית דין, עם הארץ מסייעין אותן (ת״כ פרשתא ד, ד)׃\quad \rashiDH{עם הארץ.} עם שבגינו נבראת הארץ, דבר אחר עם שעתידין לירש את הארץ על ידי מצות הללו (שם)׃}
\threeverse{\arabic{verse}}%Leviticus20:3
{וַאֲנִ֞י אֶתֵּ֤ן אֶת\maqqaf פָּנַי֙ בָּאִ֣ישׁ הַה֔וּא וְהִכְרַתִּ֥י אֹת֖וֹ מִקֶּ֣רֶב עַמּ֑וֹ כִּ֤י מִזַּרְעוֹ֙ נָתַ֣ן לַמֹּ֔לֶךְ לְמַ֗עַן טַמֵּא֙ אֶת\maqqaf מִקְדָּשִׁ֔י וּלְחַלֵּ֖ל אֶת\maqqaf שֵׁ֥ם קׇדְשִֽׁי׃}
{וַאֲנָא אֶתֵּין יָת רוּגְזִי בַּאֲנָשָׁא הַהוּא וַאֲשֵׁיצֵי יָתֵיהּ מִגּוֹ עַמֵּיהּ אֲרֵי מִזַּרְעֵיהּ יְהַב לְמוֹלֶךְ בְּדִיל לְסַאָבָא יָת מַקְדְּשִׁי וּלְאַחָלָא יָת שְׁמָא דְּקוּדְשִׁי׃}
{I also will set My face against that man, and will cut him off from among his people, because he hath given of his seed unto Molech, to defile My sanctuary, and to profane My holy name.}{\arabic{verse}}
\rashi{\rashiDH{אתן את פני.} פנאי שלי, פונה אני מכל עסקי ועוסק בו (שם יב)׃\quad \rashiDH{באיש.} ולא בצבור, שאין כל הצבור נכרתין׃\quad \rashiDH{כי מזרעו נתן למלך.} לפי שנאמר מעביר בנו ובתו באש (דברים יח, י), בן בנו ובן בתו מנין, תלמוד לומר כי מזרעו נתן למולך (ת״כ פרשתא ד, ו), זרע פסול מנין, ת״ל בתתו מזרעו למולך (שם ז.  סנהדרין סד׃)׃\quad \rashiDH{למען טמא את מקדשי.} את כנסת ישראל שהיא מקודשת לי, כלשון וְלֹא יְחַלְּלוּ אֶת מִקְדָּשַׁי (ויקרא כא, כג)׃}
\threeverse{\arabic{verse}}%Leviticus20:4
{וְאִ֡ם הַעְלֵ֣ם יַעְלִ֩ימֽוּ֩ עַ֨ם הָאָ֜רֶץ אֶת\maqqaf עֵֽינֵיהֶם֙ מִן\maqqaf הָאִ֣ישׁ הַה֔וּא בְּתִתּ֥וֹ מִזַּרְע֖וֹ לַמֹּ֑לֶךְ לְבִלְתִּ֖י הָמִ֥ית אֹתֽוֹ׃}
{וְאִם מִכְבָּשׁ יִכְבְּשׁוּן עַמָּא בֵּית יִשְׂרָאֵל יָת עֵינֵיהוֹן מִן גּוּבְרָא הַהוּא בְּדִיהַב מִזַּרְעֵיהּ לְמֹלֶךְ בְּדִיל דְּלָא לְמִקְטַל יָתֵיהּ׃}
{And if the people of the land do at all hide their eyes from that man, when he giveth of his seed unto Molech, and put him not to death;}{\arabic{verse}}
\rashi{\rashiDH{ואם העלם יעלימו.} אם העלימו בדבר אחד סוף שיעלימו בדברים הרבה, אם העלימו סנהדרי קטנה סוף שיעלימו סנהדרי גדולה׃}
\threeverse{\arabic{verse}}%Leviticus20:5
{וְשַׂמְתִּ֨י אֲנִ֧י אֶת\maqqaf פָּנַ֛י בָּאִ֥ישׁ הַה֖וּא וּבְמִשְׁפַּחְתּ֑וֹ וְהִכְרַתִּ֨י אֹת֜וֹ וְאֵ֣ת \legarmeh  כׇּל\maqqaf הַזֹּנִ֣ים אַחֲרָ֗יו לִזְנ֛וֹת אַחֲרֵ֥י הַמֹּ֖לֶךְ מִקֶּ֥רֶב עַמָּֽם׃}
{וַאֲשַׁוֵּי אֲנָא יָת רוּגְזִי בְּגוּבְרָא הַהוּא וּבְסָעֲדוֹהִי וַאֲשֵׁיצֵי יָתֵיהּ וְיָת כָּל דְּטָעַן בָּתְרוֹהִי לְמִטְעֵי בָּתַר מוֹלֶךְ מִגּוֹ עַמְּהוֹן׃}
{then I will set My face against that man, and against his family, and will cut him off, and all that go astray after him, to go astray after Molech, from among their people.}{\arabic{verse}}
\rashi{\rashiDH{ובמשפחתו.} אמר ר׳ שמעון וכי משפחה מה חטאה, אלא ללמדך שאין לך משפחה שיש בה מוכס שאין כולם מוכסין, שכולם מחפין עליו (שבועות לט.)׃\quad \rashiDH{והכרתי אותו.} למה נאמר, לפי שנאמר ובמשפחתו, יכול יהיו כל המשפחה בהכרת, תלמוד לומר אותו. אותו בהכרת ולא כל המשפחה בהכרת אלא ביסורין׃\quad \rashiDH{לזנות אחרי המלך.} לרבות שאר עבודת אלילים שעבדה בכך (ת״כ שם טו), ואפילו אין זו עבודתה (סנהדרין סד׃)׃}
\threeverse{\arabic{verse}}%Leviticus20:6
{וְהַנֶּ֗פֶשׁ אֲשֶׁ֨ר תִּפְנֶ֤ה אֶל\maqqaf הָֽאֹבֹת֙ וְאֶל\maqqaf הַיִּדְּעֹנִ֔ים לִזְנֹ֖ת אַחֲרֵיהֶ֑ם וְנָתַתִּ֤י אֶת\maqqaf פָּנַי֙ בַּנֶּ֣פֶשׁ הַהִ֔וא וְהִכְרַתִּ֥י אֹת֖וֹ מִקֶּ֥רֶב עַמּֽוֹ׃}
{וֶאֱנָשׁ דְּיִתְפְּנֵי בָּתַר בִּדִּין וּזְכוּרוּ לְמִטְעֵי בָּתְרֵיהוֹן וְאֶתֵּין יָת רוֹגְזִי בַּאֲנָשָׁא הַהוּא וַאֲשֵׁיצֵי יָתֵיהּ מִגּוֹ עַמֵּיהּ׃}
{And the soul that turneth unto the ghosts, and unto the familiar spirits, to go astray after them, I will even set My face against that soul, and will cut him off from among his people.}{\arabic{verse}}
\threeverse{\arabic{verse}}%Leviticus20:7
{וְהִ֨תְקַדִּשְׁתֶּ֔ם וִהְיִיתֶ֖ם קְדֹשִׁ֑ים כִּ֛י אֲנִ֥י יְהֹוָ֖ה אֱלֹהֵיכֶֽם׃}
{וְתִתְקַדְּשׁוּן וּתְהוֹן קַדִּישִׁין אֲרֵי אֲנָא יְיָ אֱלָהֲכוֹן׃}
{Sanctify yourselves therefore, and be ye holy; for I am the \lord\space your God.}{\arabic{verse}}
\rashi{\rashiDH{והתקדשתם.} זו פרישות עבודת אלילים׃}
\aliyacounter{ששי}
\threeverse{\aliya{ששי\newline (שביעי)}}%Leviticus20:8
{וּשְׁמַרְתֶּם֙ אֶת\maqqaf חֻקֹּתַ֔י וַעֲשִׂיתֶ֖ם אֹתָ֑ם אֲנִ֥י יְהֹוָ֖ה מְקַדִּשְׁכֶֽם׃}
{וְתִטְּרוּן יָת קְיָמַי וְתַעְבְּדוּן יָתְהוֹן אֲנָא יְיָ מְקַדִּשְׁכוֹן׃}
{And keep ye My statutes, and do them: I am the \lord\space who sanctify you.}{\arabic{verse}}
\threeverse{\arabic{verse}}%Leviticus20:9
{כִּֽי\maqqaf אִ֣ישׁ אִ֗ישׁ אֲשֶׁ֨ר יְקַלֵּ֧ל אֶת\maqqaf אָבִ֛יו וְאֶת\maqqaf אִמּ֖וֹ מ֣וֹת יוּמָ֑ת אָבִ֧יו וְאִמּ֛וֹ קִלֵּ֖ל דָּמָ֥יו בּֽוֹ׃}
{אֲרֵי גְּבַר גְּבַר דִּילוּט יָת אֲבוּהִי וְיָת אִמֵּיהּ אִתְקְטָלָא יִתְקְטִיל אֲבוּהִי וְאִמֵּיהּ לָט קַטְלָא חַיָּיב׃}
{For whatsoever man there be that curseth his father or his mother shall surely be put to death; he hath cursed his father or his mother; his blood shall be upon him.}{\arabic{verse}}
\rashi{\rashiDH{אביו ואמו קלל.} לרבות לאחר מיתה (שם פרק ט, ג.  סנהדרין פה׃)׃ 
\quad \rashiDH{דמיו בו.} זו סקילה, וכן כל מקום שנאמר דמיו בו, דמיהם בם, ולמדנו מאוב וידעוני שנאמר בהם באבן ירגמו אותם דמיהם בם (שם סו.), ופשוטו של מקרא, כמו דָּמוֹ בְרֹאשׁוֹ (יהושע ב, יט), אין נענש על מיתתו, אלא הוא, שהוא גרם לעצמו שֶׁיֵּהָרֵג׃}
\threeverse{\arabic{verse}}%Leviticus20:10
{וְאִ֗ישׁ אֲשֶׁ֤ר יִנְאַף֙ אֶת\maqqaf אֵ֣שֶׁת אִ֔ישׁ אֲשֶׁ֥ר יִנְאַ֖ף אֶת\maqqaf אֵ֣שֶׁת רֵעֵ֑הוּ מֽוֹת\maqqaf יוּמַ֥ת הַנֹּאֵ֖ף וְהַנֹּאָֽפֶת׃}
{וּגְבַר דִּיגוּף יָת אִתַּת גְּבַר דִּיגוּף יָת אִתַּת חַבְרֵיהּ אִתְקְטָלָא יִתְקְטִיל גַּיָּיפָא וְגַיָּיפְתָא׃}
{And the man that committeth adultery with another man’s wife, even he that committeth adultery with his neighbour’s wife, both the adulterer and the adulteress shall surely be put to death.}{\arabic{verse}}
\rashi{\rashiDH{ואיש.} פרט לקטן׃\quad \rashiDH{אשר ינאף את אשת איש.} רט לאשת קטן (קידושין יט. סנהדרין נב׃), למדנו שאין לקטן קידושין, ועל איזו אשת איש חייבתי לך׃\quad \rashiDH{אשר ינאף את אשת רעהו.} פרט לאשת עובדי גילולים, למדנו שאין קידושין לעובדי גילולים׃ 
\quad \rashiDH{מות יומת הנואף והנואפת.} כל מיתה האמורה בתורה סתם, אינה אלא חנק׃}
\threeverse{\arabic{verse}}%Leviticus20:11
{וְאִ֗ישׁ אֲשֶׁ֤ר יִשְׁכַּב֙ אֶת\maqqaf אֵ֣שֶׁת אָבִ֔יו עֶרְוַ֥ת אָבִ֖יו גִּלָּ֑ה מֽוֹת\maqqaf יוּמְת֥וּ שְׁנֵיהֶ֖ם דְּמֵיהֶ֥ם בָּֽם׃}
{וּגְבַר דְּיִשְׁכּוֹב יָת אִתַּת אֲבוּהִי עֶרְיְתָא דַּאֲבוּהִי גַּלִּי אִתְקְטָלָא יִתְקַטְלוּן תַּרְוֵיהוֹן קַטְלָא חַיָּיבִין׃}
{And the man that lieth with his father’s wife—he hath uncovered his father’s nakedness—both of them shall surely be put to death; their blood shall be upon them.}{\arabic{verse}}
\threeverse{\arabic{verse}}%Leviticus20:12
{וְאִ֗ישׁ אֲשֶׁ֤ר יִשְׁכַּב֙ אֶת\maqqaf כַּלָּת֔וֹ מ֥וֹת יוּמְת֖וּ שְׁנֵיהֶ֑ם תֶּ֥בֶל עָשׂ֖וּ דְּמֵיהֶ֥ם בָּֽם׃}
{וּגְבַר דְּיִשְׁכּוֹב יָת כַּלְּתֵיהּ אִתְקְטָלָא יִתְקַטְלוּן תַּרְוֵיהוֹן תָּבְלָא עֲבַדוּ קַטְלָא חַיָּיבִין׃}
{And if a man lie with his daughter-in-law, both of them shall surely be put to death; they have wrought corruption; their blood shall be upon them.}{\arabic{verse}}
\rashi{\rashiDH{תבל עשו.} גנאי. לישנא אחרינא מבלבלין זרע האב בזרע הבן׃}
\threeverse{\arabic{verse}}%Leviticus20:13
{וְאִ֗ישׁ אֲשֶׁ֨ר יִשְׁכַּ֤ב אֶת\maqqaf זָכָר֙ מִשְׁכְּבֵ֣י אִשָּׁ֔ה תּוֹעֵבָ֥ה עָשׂ֖וּ שְׁנֵיהֶ֑ם מ֥וֹת יוּמָ֖תוּ דְּמֵיהֶ֥ם בָּֽם׃}
{וּגְבַר דְּיִשְׁכּוֹב יָת דְּכוּרָא מִשְׁכְּבֵי אִתָּא תּוֹעֵיבָא עֲבַדוּ תַּרְוֵיהוֹן אִתְקְטָלָא יִתְקַטְלוּן קַטְלָא חַיָּיבִין׃}
{And if a man lie with mankind, as with womankind, both of them have committed abomination: they shall surely be put to death; their blood shall be upon them.}{\arabic{verse}}
\rashi{\rashiDH{משכבי אשה.} מכניס כמכחול בשפופרת (עי׳ סנהדרין נה.)׃}
\threeverse{\arabic{verse}}%Leviticus20:14
{וְאִ֗ישׁ אֲשֶׁ֨ר יִקַּ֧ח אֶת\maqqaf אִשָּׁ֛ה וְאֶת\maqqaf אִמָּ֖הּ זִמָּ֣ה הִ֑וא בָּאֵ֞שׁ יִשְׂרְפ֤וּ אֹתוֹ֙ וְאֶתְהֶ֔ן וְלֹא\maqqaf תִהְיֶ֥ה זִמָּ֖ה בְּתוֹכְכֶֽם׃}
{וּגְבַר דְּיִסַּב יָת אִתְּתָא וְיָת אִמַּהּ עֵיצַת חֲטִאין הִיא בְּנוּרָא יוֹקְדוּן יָתֵיהּ וְיָתְהוֹן וְלָא תְהֵי עֵיצַת חֲטִאין בֵּינֵיכוֹן׃}
{And if a man take with his wife also her mother, it is wickedness: they shall be burnt with fire, both he and they; that there be no wickedness among you.}{\arabic{verse}}
\rashi{\rashiDH{ישרפו אתו ואתהן.} אי אתה יכול לומר אשתו הראשונה ישרפו, שהרי נשאה בהיתר ולא נאסרה עליו, אלא אשה ואמה הכתובין כאן שתיהן לאיסור, שנשא את חמותו ואמה (סנהדרין עו׃). ויש מרבותינו שאומרים (שם) אין כאן אלא חמותו, ומהו אתהן, את אחת מהן, ולשון יוני הוא הן אחת׃}
\threeverse{\arabic{verse}}%Leviticus20:15
{וְאִ֗ישׁ אֲשֶׁ֨ר יִתֵּ֧ן שְׁכׇבְתּ֛וֹ בִּבְהֵמָ֖ה מ֣וֹת יוּמָ֑ת וְאֶת\maqqaf הַבְּהֵמָ֖ה תַּהֲרֹֽגוּ׃}
{וּגְבַר דְּיִתֵּין שְׁכוּבְתֵּיהּ בִּבְעִירָא אִתְקְטָלָא יִתְקְטִיל וְיָת בְּעִירָא תִּקְטְלוּן׃}
{And if a man lie with a beast, he shall surely be put to death; and ye shall slay the beast.}{\arabic{verse}}
\rashi{\rashiDH{ואת הבהמה תהרוגו.} אם אדם חטא בהמה מה חטאה, אלא מפני שבאה לאדם תקלה על ידה, לפיכך אמר הכתוב תסקל. קל וחומר לאדם שיודע להבחין בין טוב לרע וגורם רעה לחבירו לעבור עבירה. כיוצא בדבר אתה אומר אַבֵּד תְּאַבְּדוּן אֶת כָּל הַמְּקֹמוֹת (דברים יב, ב), הרי דברים קל וחומר ומה אילנות שאינן רואין ואינן שומעין על שבאת תקלה על ידם אמרה תורה הַשְׁחֵת, שְׂרֹף וְכַלֵּה, המטה את חבירו מדרך חיים לדרכי מיתה, על אחת כמה וכמה׃}
\threeverse{\arabic{verse}}%Leviticus20:16
{וְאִשָּׁ֗ה אֲשֶׁ֨ר תִּקְרַ֤ב אֶל\maqqaf כׇּל\maqqaf בְּהֵמָה֙ לְרִבְעָ֣ה אֹתָ֔הּ וְהָרַגְתָּ֥ אֶת\maqqaf הָאִשָּׁ֖ה וְאֶת\maqqaf הַבְּהֵמָ֑ה מ֥וֹת יוּמָ֖תוּ דְּמֵיהֶ֥ם בָּֽם׃}
{וְאִתְּתָא דְּתִקְרַב לְוָת כָּל בְּעִירָא לְמִשְׁלַט בַּהּ וְתִקְטוֹל יָת אִתְּתָא וְיָת בְּעִירָא אִתְקְטָלָא יִתְקַטְלוּן קַטְלָא חַיָּיבִין׃}
{And if a woman approach unto any beast, and lie down thereto, thou shalt kill the woman, and the beast: they shall surely be put to death; their blood shall be upon them.}{\arabic{verse}}
\threeverse{\arabic{verse}}%Leviticus20:17
{וְאִ֣ישׁ אֲשֶׁר\maqqaf יִקַּ֣ח אֶת\maqqaf אֲחֹת֡וֹ בַּת\maqqaf אָבִ֣יו א֣וֹ בַת\maqqaf אִ֠מּ֠וֹ וְרָאָ֨ה אֶת\maqqaf עֶרְוָתָ֜הּ וְהִֽיא\maqqaf תִרְאֶ֤ה אֶת\maqqaf עֶרְוָתוֹ֙ חֶ֣סֶד ה֔וּא וְנִ֨כְרְת֔וּ לְעֵינֵ֖י בְּנֵ֣י עַמָּ֑ם עֶרְוַ֧ת אֲחֹת֛וֹ גִּלָּ֖ה עֲוֺנ֥וֹ יִשָּֽׂא׃}
{וּגְבַר דְּיִסַּב יָת אֲחָתֵיהּ בַּת אֲבוּהִי אוֹ בַת אִמֵּיהּ וְיִחְזֵי יָת עֶרְיְתַהּ וְהִיא תִחְזֵי יָת עֶרְיְתֵיהּ קְלָנָא הוּא וְיִשְׁתֵּיצוֹן לְעֵינֵי בְּנֵי עַמְּהוֹן עֶרְיַת אֲחָתֵיהּ גַּלִּי חוֹבֵיהּ יְקַבֵּיל׃}
{And if a man shall take his sister, his father’s daughter, or his mother’s daughter, and see her nakedness, and she see his nakedness: it is a shameful thing; and they shall be cut off in the sight of the children of their people: he hath uncovered his sister’s nakedness; he shall bear his iniquity.}{\arabic{verse}}
\rashi{\rashiDH{חסד הוא.} לשון ארמי חֶרְפָּה (בראשית לד, יד), חִסּוּדָא. ומדרשו (סנהדרין נח׃) אם תאמר קין נשא אחותו, חסד עשה המקום לבנות עולמו ממנו, שנאמר עוֹלָם חֶסֶד יִבָּנֶה (תהלים פט, ג)׃}
\threeverse{\arabic{verse}}%Leviticus20:18
{וְ֠אִ֠ישׁ אֲשֶׁר\maqqaf יִשְׁכַּ֨ב אֶת\maqqaf אִשָּׁ֜ה דָּוָ֗ה וְגִלָּ֤ה אֶת\maqqaf עֶרְוָתָהּ֙ אֶת\maqqaf מְקֹרָ֣הּ הֶֽעֱרָ֔ה וְהִ֕וא גִּלְּתָ֖ה אֶת\maqqaf מְק֣וֹר דָּמֶ֑יהָ וְנִכְרְת֥וּ שְׁנֵיהֶ֖ם מִקֶּ֥רֶב עַמָּֽם׃}
{וּגְבַר דְּיִשְׁכּוֹב יָת אִתָּא טוּמְאָה וִיגַלֵּי יָת עֶרְיְתַהּ יָת קְלָנַהּ גַּלִּי וְהִיא תְּגַלֵּי יָת סוֹאֲבָת דְּמַהָא וְיִשְׁתֵּיצוֹן תַּרְוֵיהוֹן מִגּוֹ עַמְּהוֹן׃}
{And if a man shall lie with a woman having her sickness, and shall uncover her nakedness—he hath made naked her fountain, and she hath uncovered the fountain of her blood—both of them shall be cut off from among their people.}{\arabic{verse}}
\rashi{\rashiDH{הערה.} גִּלָּה, וכן כל לשון ערוה גלוי הוא, והוי״ו יורדת בתיבה לשם דבר, כמו זעוה, מגזרת וְלֹא קָם וְלֹא זָע (אסתר ה, ט), וכן אחוה מגזרת אח. והעראה זו נחלקו בה רבותינו, (יבמות נה׃) יש אומרים זו נשיקת שמש, ויש אומרים זו הכנסת עטרה׃ 
}
\threeverse{\arabic{verse}}%Leviticus20:19
{וְעֶרְוַ֨ת אֲח֧וֹת אִמְּךָ֛ וַאֲח֥וֹת אָבִ֖יךָ לֹ֣א תְגַלֵּ֑ה כִּ֧י אֶת\maqqaf שְׁאֵר֛וֹ הֶעֱרָ֖ה עֲוֺנָ֥ם יִשָּֽׂאוּ׃}
{וְעֶרְיַת אֲחָת אִמָּךְ וַאֲחָת אֲבוּךְ לָא תְגַלֵּי אֲרֵי יָת קָרִיבְתֵיהּ גַּלִּי חוֹבְהוֹן יְקַבְּלוּן׃}
{And thou shalt not uncover the nakedness of thy mother’s sister, nor of thy father’s sister; for he hath made naked his near kin; they shall bear their iniquity.}{\arabic{verse}}
\rashi{\rashiDH{וערות אחות אמך.} שנה הכתוב באזהרתן, לומר, שהוזהר עליהן בין על אחות אביו ואמו מן האב, בין על אחיותיהן מן האם (שם נד׃), אבל ערות אשת אחי אביו לא הוזהר אלא על אשת אחי אביו מן האב׃}
\threeverse{\arabic{verse}}%Leviticus20:20
{וְאִ֗ישׁ אֲשֶׁ֤ר יִשְׁכַּב֙ אֶת\maqqaf דֹּ֣דָת֔וֹ עֶרְוַ֥ת דֹּד֖וֹ גִּלָּ֑ה חֶטְאָ֥ם יִשָּׂ֖אוּ עֲרִירִ֥ים יָמֻֽתוּ׃}
{וּגְבַר דְּיִשְׁכּוֹב יָת אִתַּת אַחְבּוּהִי עֶרְיְתָא דְּאַחְבּוּהִי גַּלִּי חוֹבְהוֹן יְקַבְּלוּן דְּלָא וְלַד יְמוּתוּן׃}
{And if a man shall lie with his uncle’s wife—he hath uncovered his uncle’s nakedness—they shall bear their sin; they shall die childless.}{\arabic{verse}}
\rashi{\rashiDH{אשר ישכב את דדתו.} המקרא הזה בא ללמד על כרת האמור למעלה, שהוא בעונש הליכת ערירי׃ 
\quad \rashiDH{ערירים.} כתרגומו בְּלָא וְלָד, ודומה לו וְאָנֹכִי הוֹלֵךְ עֲרִירִי (בראשית טו, ב), יש לו בנים קוברן, אין לו נים מת בלא בנים, לכך שִׁנָּה בשני מקראות אלו, ערירים ימותו, ערירים יהיו, ערירים ימותו אם יהיו לו בשעת עבירה, לא יהיו לו כשימות, לפי שקוברן בחייו, ערירים יהיו, שאם אין לו בשעת עבירה יהיה כל ימיו כמו שהוא עכשיו (יבמות נה.)׃}
\threeverse{\arabic{verse}}%Leviticus20:21
{וְאִ֗ישׁ אֲשֶׁ֥ר יִקַּ֛ח אֶת\maqqaf אֵ֥שֶׁת אָחִ֖יו נִדָּ֣ה הִ֑וא עֶרְוַ֥ת אָחִ֛יו גִּלָּ֖ה עֲרִירִ֥ים יִהְיֽוּ׃}
{וּגְבַר דְּיִסַּב יָת אִתַּת אֲחוּהִי מְרַחֲקָא הִיא עֶרְיְתָא דַּאֲחוּהִי גַּלִּי דְּלָא וְלַד יְהוֹן׃}
{And if a man shall take his brother’s wife, it is impurity: he hath uncovered his brother’s nakedness; they shall be childless.}{\arabic{verse}}
\rashi{\rashiDH{נדה הוא.} השכיבה הזאת מנודה היא ומאוסה. ורבותינו דרשו (שם נד׃) לאסור העראה בה כנדה שהעראה מפורשת בה את מקורה הערה׃}
\threeverse{\arabic{verse}}%Leviticus20:22
{וּשְׁמַרְתֶּ֤ם אֶת\maqqaf כׇּל\maqqaf חֻקֹּתַי֙ וְאֶת\maqqaf כׇּל\maqqaf מִשְׁפָּטַ֔י וַעֲשִׂיתֶ֖ם אֹתָ֑ם וְלֹא\maqqaf תָקִ֤יא אֶתְכֶם֙ הָאָ֔רֶץ אֲשֶׁ֨ר אֲנִ֜י מֵבִ֥יא אֶתְכֶ֛ם שָׁ֖מָּה לָשֶׁ֥בֶת בָּֽהּ׃}
{וְתִטְּרוּן יָת כָּל קְיָמַי וְיָת כָּל דִּינַי וְתַעְבְּדוּן יָתְהוֹן וְלָא תְרוֹקֵין יָתְכוֹן אַרְעָא דַּאֲנָא מַעֵיל יָתְכוֹן לְתַמָּן לְמִתַּב בַּהּ׃}
{Ye shall therefore keep all My statutes, and all Mine ordinances, and do them, that the land, whither I bring you to dwell therein, vomit you not out.}{\arabic{verse}}
\aliyacounter{שביעי}
\threeverse{\aliya{שביעי}}%Leviticus20:23
{וְלֹ֤א תֵֽלְכוּ֙ בְּחֻקֹּ֣ת הַגּ֔וֹי אֲשֶׁר\maqqaf אֲנִ֥י מְשַׁלֵּ֖חַ מִפְּנֵיכֶ֑ם כִּ֤י אֶת\maqqaf כׇּל\maqqaf אֵ֙לֶּה֙ עָשׂ֔וּ וָאָקֻ֖ץ בָּֽם׃}
{וְלָא תְהָכוּן בְּנִמּוֹסֵי עַמְמַיָּא דַּאֲנָא מַגְלֵי מִן קֳדָמֵיכוֹן אֲרֵי יָת כָּל אִלֵּין עֲבַדוּ וְרַחֵיק מֵימְרִי יָתְהוֹן׃}
{And ye shall not walk in the customs of the nation, which I am casting out before you; for they did all these things, and therefore I abhorred them.}{\arabic{verse}}
\rashi{\rashiDH{ואקץ.} לשון מיאוס, כמו קַצְתִּי בְּחַיַּי (בראשית כז, מו), כאדם שהוא קץ במזונו׃ 
}
\threeverse{\arabic{verse}}%Leviticus20:24
{וָאֹמַ֣ר לָכֶ֗ם אַתֶּם֮ תִּֽירְשׁ֣וּ אֶת\maqqaf אַדְמָתָם֒ וַאֲנִ֞י אֶתְּנֶ֤נָּה לָכֶם֙ לָרֶ֣שֶׁת אֹתָ֔הּ אֶ֛רֶץ זָבַ֥ת חָלָ֖ב וּדְבָ֑שׁ אֲנִי֙ יְהֹוָ֣ה אֱלֹֽהֵיכֶ֔ם אֲשֶׁר\maqqaf הִבְדַּ֥לְתִּי אֶתְכֶ֖ם מִן\maqqaf הָֽעַמִּֽים׃}
{וַאֲמַרִית לְכוֹן אַתּוּן תֵּירְתוּן יָת אֲרַעְהוֹן וַאֲנָא אֶתְּנִנַּהּ לְכוֹן לְמֵירַת יָתַהּ אֲרַע עָבְדָא חֲלָב וּדְבָשׁ אֲנָא יְיָ אֱלָהֲכוֹן דְּאַפְרֵישִׁית יָתְכוֹן מִן עַמְמַיָּא׃}
{But I have said unto you: ‘Ye shall inherit their land, and I will give it unto you to possess it, a land flowing with milk and honey.’ I am the \lord\space your God, who have set you apart from the peoples.}{\arabic{verse}}
\threeverse{\aliya{מפטיר}}%Leviticus20:25
{וְהִבְדַּלְתֶּ֞ם בֵּֽין\maqqaf הַבְּהֵמָ֤ה הַטְּהֹרָה֙ לַטְּמֵאָ֔ה וּבֵין\maqqaf הָע֥וֹף הַטָּמֵ֖א לַטָּהֹ֑ר וְלֹֽא\maqqaf תְשַׁקְּצ֨וּ אֶת\maqqaf נַפְשֹֽׁתֵיכֶ֜ם בַּבְּהֵמָ֣ה וּבָע֗וֹף וּבְכֹל֙ אֲשֶׁ֣ר תִּרְמֹ֣שׂ הָֽאֲדָמָ֔ה אֲשֶׁר\maqqaf הִבְדַּ֥לְתִּי לָכֶ֖ם לְטַמֵּֽא׃}
{וְתַפְרְשׁוּן בֵּין בְּעִירָא דָּכְיָא לִמְסָאֲבָא וּבֵין עוֹפָא מְסָאֲבָא לְדָכְיָא וְלָא תְשַׁקְּצוּן יָת נַפְשָׁתְכוֹן בִּבְעִירָא וּבְעוֹפָא וּבְכֹל דְּתַרְחֵישׁ אַרְעָא דְּאַפְרֵישִׁית לְכוֹן לְסַאָבָא׃}
{Ye shall therefore separate between the clean beast and the unclean, and between the unclean fowl and the clean; and ye shall not make your souls detestable by beast, or by fowl, or by any thing wherewith the ground teemeth, which I have set apart for you to hold unclean.}{\arabic{verse}}
\rashi{\rashiDH{והבדלתם בין הבהמה הטהורה לטמאה.} אין צריך לומר בין פרה לחמור, שהרי מובדלין ונכרין הם, אלא בין טהורה לך לטמאה בין שנשחט רובו של סימן, לנשחט חציו, וכמה בין רובו לחציו מלא שערה׃\quad \rashiDH{אשר הבדלתי לכם לטמא.} לאסור׃}
\threeverse{\arabic{verse}}%Leviticus20:26
{וִהְיִ֤יתֶם לִי֙ קְדֹשִׁ֔ים כִּ֥י קָד֖וֹשׁ אֲנִ֣י יְהֹוָ֑ה וָאַבְדִּ֥ל אֶתְכֶ֛ם מִן\maqqaf הָֽעַמִּ֖ים לִהְי֥וֹת לִֽי׃}
{וּתְהוֹן קֳדָמַי קַדִּישִׁין אֲרֵי קַדִּישׁ אֲנָא יְיָ וְאַפְרֵישִׁית יָתְכוֹן מִן עַמְמַיָּא לְמִהְוֵי פָלְחִין קֳדָמָי׃}
{And ye shall be holy unto Me; for I the \lord\space am holy, and have set you apart from the peoples, that ye should be Mine.}{\arabic{verse}}
\rashi{\rashiDH{ואבדל אתכם מן העמים להיות לי.} אם אתם מובדלים מהם הרי אתם שלי, ואם לאו, אתם של נבוכדנצר וחביריו, רבי אלעזר בן עזריה אומר מנין שלא יאמר אדם, נפשי קצה בבשר חזיר, אי אפשי ללבוש כלאים, אבל יאמר אפשי, ומה אעשה ואבי שבשמים גזר עלי, תלמוד לומר ואבדיל אתכם מן העמים להיות לי, שתהא הבדלתכם מהם לשמי, פורש מן העבירה ומקבל עליו עול מלכות שמים׃}
\threeverse{\aliya{\Hebrewnumeral{64}}}%Leviticus20:27
{וְאִ֣ישׁ אֽוֹ\maqqaf אִשָּׁ֗ה כִּֽי\maqqaf יִהְיֶ֨ה בָהֶ֥ם א֛וֹב א֥וֹ יִדְּעֹנִ֖י מ֣וֹת יוּמָ֑תוּ בָּאֶ֛בֶן יִרְגְּמ֥וּ אֹתָ֖ם דְּמֵיהֶ֥ם בָּֽם׃ \petucha }
{וּגְבַר אוֹ אִתָּא אֲרֵי יְהֵי בְהוֹן בִּדִּין אוֹ זְכוּרוּ אִתְקְטָלָא יִתְקַטְלוּן בְּאַבְנָא יִרְגְּמוּן יָתְהוֹן קַטְלָא חַיָּיבִין׃}
{A man also or a woman that divineth by a ghost or a familiar spirit, shall surely be put to death; they shall stone them with stones; their blood shall be upon them.}{\arabic{verse}}
\rashi{\rashiDH{כי יהיה בהם אוב וגו׳.} כאן נאמר בהם מיתה, ולמעלה כרת, עדים והתראה בסקילה, מזיד בלא התראה בהכרת, ושגגתם חטאת, וכן בכל חייבי מיתות שנאמר בהם כרת׃}
\engnote{The Haftarah is Amos 9:7\verserangechar 9:15 on page \pageref{haft_30}. Sepharadim read Ezekiel 20:2\verserangechar 20:20. }
\newperek
\aliyacounter{ראשון}
\newparsha{אמור}
\newseder{17}
\threeverse{\aliya{אמור}\newline\vspace{-4pt}\newline\seder{יז}}%Leviticus21:1
{וַיֹּ֤אמֶר יְהֹוָה֙ אֶל\maqqaf מֹשֶׁ֔ה אֱמֹ֥ר אֶל\maqqaf הַכֹּהֲנִ֖ים בְּנֵ֣י אַהֲרֹ֑ן וְאָמַרְתָּ֣ אֲלֵהֶ֔ם לְנֶ֥פֶשׁ לֹֽא\maqqaf יִטַּמָּ֖א בְּעַמָּֽיו׃}
{וַאֲמַר יְיָ לְמֹשֶׁה אֵימַר לְכָהֲנַיָּא בְּנֵי אַהֲרֹן וְתֵימַר לְהוֹן עַל מִית לָא יִסְתָּאַב בְּעַמֵּיהּ׃}
{And the \lord\space said unto Moses: Speak unto the priests the sons of Aaron, and say unto them: There shall none defile himself for the dead among his people;}{\Roman{chap}}
\rashi{\rashiDH{אמור אל הכהנים.} אמור, ואמרת, להזהיר גדולים על הקטנים (יבמות קיד.  ת״כ פרשתא א, א)׃\quad \rashiDH{בני אהרן.} יכול חללים, תלמוד לומר הכהנים׃\quad \rashiDH{בני אהרן.} אף בעלי מומין במשמע׃\quad \rashiDH{בני אהרן.} ולא בנות אהרן (קידושין לה׃)׃ 
\quad \rashiDH{לא יטמא בעמיו.} בעוד שהמת בתוך עמיו, יצא מת מצוה (ת״כ שם ג)׃ 
}
\threeverse{\arabic{verse}}%Leviticus21:2
{כִּ֚י אִם\maqqaf לִשְׁאֵר֔וֹ הַקָּרֹ֖ב אֵלָ֑יו לְאִמּ֣וֹ וּלְאָבִ֔יו וְלִבְנ֥וֹ וּלְבִתּ֖וֹ וּלְאָחִֽיו׃}
{אֱלָהֵין לְקָרִיבֵיהּ דְּקָרִיב לֵיהּ לְאִמֵּיהּ וְלַאֲבוּהִי וְלִבְרֵיהּ וְלִבְרַתֵּיהּ וּלְאֲחוּהִי׃}
{except for his kin, that is near unto him, for his mother, and for his father, and for his son, and for his daughter, and for his brother;}{\arabic{verse}}
\rashi{\rashiDH{כי אם לשארו.} אין שארו אלא אשתו (שם ד)׃}
\threeverse{\arabic{verse}}%Leviticus21:3
{וְלַאֲחֹת֤וֹ הַבְּתוּלָה֙ הַקְּרוֹבָ֣ה אֵלָ֔יו אֲשֶׁ֥ר לֹֽא\maqqaf הָיְתָ֖ה לְאִ֑ישׁ לָ֖הּ יִטַּמָּֽא׃}
{וְלַאֲחָתֵיהּ בְּתוּלְתָא דְּקָרִיבָא לֵיהּ דְּלָא הֲוָת לִגְבַר לַהּ יִסְתָּאַב׃}
{and for his sister a virgin, that is near unto him, that hath had no husband, for her may he defile himself.}{\arabic{verse}}
\rashi{\rashiDH{הקרובה.} לרבות את הארוסה (שם יב. יבמות ס.)׃\quad \rashiDH{אשר לא היתה לאיש.} למשכב׃ 
\quad \rashiDH{לה יטמא.} מצוה (ת״כ שם)׃}
\threeverse{\arabic{verse}}%Leviticus21:4
{לֹ֥א יִטַּמָּ֖א בַּ֣עַל בְּעַמָּ֑יו לְהֵ֖חַלּֽוֹ׃}
{לָא יִסְתָּאַב בְּרַבָּא בְּעַמֵּיהּ לְאַחָלוּתֵיהּ׃}
{He shall not defile himself, being a chief man among his people, to profane himself.}{\arabic{verse}}
\rashi{\rashiDH{לא יטמא בעל בעמיו להחלו.} לא יטמא לאשתו פסולה שהוא מחולל בה בעודה עמו (שם טו. יבמות כב׃), וכן פשוטו של מקרא לא יטמא בעל בשארו, בעוד שהיא בתוך עמיו שיש לה קוברין, שאינה מת מצוה, ובאיזה שאר אמרתי, באותו שהיא להחלו, להתחלל הוא מכהונתו׃}
\threeverse{\arabic{verse}}%Leviticus21:5
{לֹֽא\maqqaf \qk{יִקְרְח֤וּ}{יקרחה} קׇרְחָה֙ בְּרֹאשָׁ֔ם וּפְאַ֥ת זְקָנָ֖ם לֹ֣א יְגַלֵּ֑חוּ וּבִ֨בְשָׂרָ֔ם לֹ֥א יִשְׂרְט֖וּ שָׂרָֽטֶת׃}
{לָא יִמְרְטוּן מְרַט בְּרֵישְׁהוֹן וּפָתָא דְּדִקְנְהוֹן לָא יְגַלְּחוּן וּבִבְשַׂרְהוֹן לָא יְחַבְּלוּן חִבּוּל׃}
{They shall not make baldness upon their head, neither shall they shave off the corners of their beard, nor make any cuttings in their flesh.}{\arabic{verse}}
\rashi{\rashiDH{לא יקרחה קרחה.} על מת, והלא אף ישראל הוזהרו על כך, אלא לפי שנאמר בישראל בין עיניכם, (דברים יד, א) יכול לא יהא חייב על כל הראש, תלמוד לומר בראשם, וילמדו ישראל מהכהנים בגזרה שוה, נאמר כאן קרחה ונאמר להלן בישראל קרחה, מה כאן כל הראש אף להלן כל הראש במשמע, כל מקום שיקרח בראש, ומה להלן על מת, אף כאן על מת (ת״כ פרק א, ג.  קידושין לו. מכות כ.)׃\quad \rashiDH{ופאת זקנם לא יגלחו.} לפי שנאמר בישראל ולא תשחית (ויקרא יט, כז), יכול לקטו בְּמַלְקֵט וּרְהִיטָנִי יהא חייב, לכך נאמר לא יגלחו, שאינו חייב אלא על דבר הקרוי גלוח ויש בו השחתה וזהו תער׃\quad \rashiDH{ובבשרם לא ישרטו שרטת.} לפי שנאמר בישראל (מכות כא.) וְשֶׂרֶט לָנֶפֶשׁ לֹא תִתְּנוּ (שם כח.), יכול שרט חמש שריטות לא יהא חייב אלא אחת, תלמוד לומר לא ישרטו שרטת, לחייב על כל שריטה ושריטה, שתיבה זו יתירה היא לדרוש, שהיה לו לכתוב לא ישרטו ואני יודע שהיא שרטת׃}
\threeverse{\arabic{verse}}%Leviticus21:6
{קְדֹשִׁ֤ים יִהְיוּ֙ לֵאלֹ֣הֵיהֶ֔ם וְלֹ֣א יְחַלְּל֔וּ שֵׁ֖ם אֱלֹהֵיהֶ֑ם כִּי֩ אֶת\maqqaf אִשֵּׁ֨י יְהֹוָ֜ה לֶ֧חֶם אֱלֹהֵיהֶ֛ם הֵ֥ם מַקְרִיבִ֖ם וְהָ֥יוּ קֹֽדֶשׁ׃}
{קַדִּישִׁין יְהוֹן קֳדָם אֱלָהֲהוֹן וְלָא יַחֲלוּן שְׁמָא דֶּאֱלָהֲהוֹן אֲרֵי יָת קוּרְבָּנַיָּא דַּייָ קוּרְבַּן אֱלָהֲהוֹן אִנּוּן מְקָרְבִין וִיהוֹן קַדִּישִׁין׃}
{They shall be holy unto their God, and not profane the name of their God; for the offerings of the \lord\space made by fire, the bread of their God, they do offer; therefore they shall be holy.}{\arabic{verse}}
\rashi{\rashiDH{קדושים יהיו.} על כרחם יקדישום בית דין בכך (ת״כ פרק א, ו)׃}
\threeverse{\aliya{לוי}}%Leviticus21:7
{אִשָּׁ֨ה זֹנָ֤ה וַחֲלָלָה֙ לֹ֣א יִקָּ֔חוּ וְאִשָּׁ֛ה גְּרוּשָׁ֥ה מֵאִישָׁ֖הּ לֹ֣א יִקָּ֑חוּ כִּֽי\maqqaf קָדֹ֥שׁ ה֖וּא לֵאלֹהָֽיו׃}
{אִתְּתָא מַטְעֲיָא וּמַחֲלָא לָא יִסְּבוּן וְאִתְּתָא דִּמְתָרְכָא מִבַּעְלַהּ לָא יִסְּבוּן אֲרֵי קַדִּישׁ הוּא קֳדָם אֱלָהֵיהּ׃}
{They shall not take a woman that is a harlot, or profaned; neither shall they take a woman put away from her husband; for he is holy unto his God.}{\arabic{verse}}
\rashi{\rashiDH{זונה.} שנבעלה בעילת ישראל האסור לה, כגון חייבי כריתות או נתין או ממזר (יבמות סא׃)׃\quad \rashiDH{חללה.} שנולדה מן הפסולים שבכהונה, כגון בת אלמנה מכהן גדול, או בת גרושה וחלוצה מכהן הדיוט, וכן שנתחללה מן הכהונה על ידי ביאת אחד מן הפסולים לכהונה (קידושין עז.)׃}
\threeverse{\arabic{verse}}%Leviticus21:8
{וְקִ֨דַּשְׁתּ֔וֹ כִּֽי\maqqaf אֶת\maqqaf לֶ֥חֶם אֱלֹהֶ֖יךָ ה֣וּא מַקְרִ֑יב קָדֹשׁ֙ יִֽהְיֶה\maqqaf לָּ֔ךְ כִּ֣י קָד֔וֹשׁ אֲנִ֥י יְהֹוָ֖ה מְקַדִּשְׁכֶֽם׃}
{וּתְקַדְּשִׁנֵּיהּ אֲרֵי יָת קוּרְבַּן אֱלָהָךְ הוּא מְקָרֵיב קַדִּישׁ יְהֵי לָךְ אֲרֵי קַדִּישׁ אֲנָא יְיָ מְקַדִּשְׁכוֹן׃}
{Thou shalt sanctify him therefore; for he offereth the bread of thy God; he shall be holy unto thee; for I the \lord, who sanctify you, am holy.}{\arabic{verse}}
\rashi{\rashiDH{וקדשתו.} על כרחו, שאם לא רצה לגרש, הלקהו ויסרהו עד שיגרש (יבמות פח׃)׃\quad \rashiDH{קדוש יהיה לך.} נהוג בו קדושה לפתוח ראשון בכל דבר, ולברך ראשון בסעודה (גיטין נט׃  ת״כ)׃}
\threeverse{\arabic{verse}}%Leviticus21:9
{וּבַת֙ אִ֣ישׁ כֹּהֵ֔ן כִּ֥י תֵחֵ֖ל לִזְנ֑וֹת אֶת\maqqaf אָבִ֙יהָ֙ הִ֣יא מְחַלֶּ֔לֶת בָּאֵ֖שׁ תִּשָּׂרֵֽף׃ \setuma }
{וּבַת גְּבַר כָּהִין אֲרֵי תִתַּחַל לְמִטְעֵי מִקְּדוּשַּׁת אֲבוּהָא הִיא מִתַּחֲלָא בְּנוּרָא תִּתּוֹקַד׃}
{And the daughter of any priest, if she profane herself by playing the harlot, she profaneth her father: she shall be burnt with fire.}{\arabic{verse}}
\rashi{\rashiDH{כי תחל לזנות.} כשתחלל על ידי זנות, שהיתה בה זיקת בעל, וזנתה או מן האירוסין או מן הנשואין, ורבותינו נחלקו בדבר (סנהדרין נא.), והכל מודים שלא דבר הכתוב בפנויה׃\quad \rashiDH{את אביה היא מחללת.} חללה ובזתה את כבודו, שאומרים עליו ארור שזו ילד, ארור שזו גדל (שם נב.)׃ 
}
\threeverse{\arabic{verse}}%Leviticus21:10
{וְהַכֹּהֵן֩ הַגָּד֨וֹל מֵאֶחָ֜יו אֲֽשֶׁר\maqqaf יוּצַ֥ק עַל\maqqaf רֹאשׁ֣וֹ \legarmeh  שֶׁ֤מֶן הַמִּשְׁחָה֙ וּמִלֵּ֣א אֶת\maqqaf יָד֔וֹ לִלְבֹּ֖שׁ אֶת\maqqaf הַבְּגָדִ֑ים אֶת\maqqaf רֹאשׁוֹ֙ לֹ֣א יִפְרָ֔ע וּבְגָדָ֖יו לֹ֥א יִפְרֹֽם׃}
{וְכָהֲנָא דְּיִתְרַבַּא מֵאֲחוֹהִי דְּיִתָּרַק עַל רֵישֵׁיהּ מִשְׁחָא דִּרְבוּתָא וְדִיקָרֵיב יָת קוּרְבָּנֵיהּ לְמִלְבַּשׁ יָת לְבוּשַׁיָּא יָת רֵישֵׁיהּ לָא יְרַבֵּי פֵירוּעַ וּלְבוּשׁוֹהִי לָא יְבַזַּע׃}
{And the priest that is highest among his brethren, upon whose head the anointing oil is poured, and that is consecrated to put on the garments, shall not let the hair of his head go loose, nor rend his clothes;}{\arabic{verse}}
\rashi{\rashiDH{לא יפרע.} לא יגדל פרע על אבל (ת״כ פרשתא ב, ג), ואיזהו גידול פרע יותר משלשים יום (סנהדרין כב׃)׃}
\threeverse{\arabic{verse}}%Leviticus21:11
{וְעַ֛ל כׇּל\maqqaf נַפְשֹׁ֥ת מֵ֖ת לֹ֣א יָבֹ֑א לְאָבִ֥יו וּלְאִמּ֖וֹ לֹ֥א יִטַּמָּֽא׃}
{וְעַל כָּל נַפְשָׁת מִיתָא לָא יֵיעוֹל לַאֲבוּהִי וּלְאִמֵּיהּ לָא יִסְתָּאַב׃}
{neither shall he go in to any dead body, nor defile himself for his father, or for his mother;}{\arabic{verse}}
\rashi{\rashiDH{ועל כל נפשת מת.} באהל המת׃\quad \rashiDH{נפשת מת.} להביא רביעית דם מן המת שמטמא באהל (נזיר לח.)׃\quad \rashiDH{לאביו ולאמו לא יטמא.} לא בא אלא להתיר לו מת מצוה (שם מז׃  ת״כ שם ד)׃}
\threeverse{\arabic{verse}}%Leviticus21:12
{וּמִן\maqqaf הַמִּקְדָּשׁ֙ לֹ֣א יֵצֵ֔א וְלֹ֣א יְחַלֵּ֔ל אֵ֖ת מִקְדַּ֣שׁ אֱלֹהָ֑יו כִּ֡י נֵ֠זֶר שֶׁ֣מֶן מִשְׁחַ֧ת אֱלֹהָ֛יו עָלָ֖יו אֲנִ֥י יְהֹוָֽה׃}
{וּמִן מַקְדְּשָׁא לָא יִפּוֹק וְלָא יַחֵיל יָת מַקְדְּשָׁא דֶּאֱלָהֵיהּ אֲרֵי כְּלִיל מְשַׁח רְבוּתָא דֶּאֱלָהֵיהּ עֲלוֹהִי אֲנָא יְיָ׃}
{neither shall he go out of the sanctuary, nor profane the sanctuary of his God; for the consecration of the anointing oil of his God is upon him: I am the \lord.}{\arabic{verse}}
\rashi{\rashiDH{ומן המקדש לא יצא.} אינו הולך אחר המטה (ת״כ שם ה.  סנהדרין יח). ועוד מכאן למדו רבותינו (שם פד.) שכהן גדול מקריב אונן, וכן משמעו, אף אם מתו אביו ואמו אינו צריך לצאת מן המקדש אלא עובד עבודה׃\quad \rashiDH{ולא יחלל את מקדש.} שאינו מחלל בכך את העבודה, שהתיר לו הכתוב, הא כהן הדיוט שעבד אונן חלל׃}
\threeverse{\aliya{ישראל}}%Leviticus21:13
{וְה֕וּא אִשָּׁ֥ה בִבְתוּלֶ֖יהָ יִקָּֽח׃}
{וְהוּא אִתְּתָא בִּבְתוּלַהָא יִסַּב׃}
{And he shall take a wife in her virginity.}{\arabic{verse}}
\threeverse{\arabic{verse}}%Leviticus21:14
{אַלְמָנָ֤ה וּגְרוּשָׁה֙ וַחֲלָלָ֣ה זֹנָ֔ה אֶת\maqqaf אֵ֖לֶּה לֹ֣א יִקָּ֑ח כִּ֛י אִם\maqqaf בְּתוּלָ֥ה מֵעַמָּ֖יו יִקַּ֥ח אִשָּֽׁה׃}
{אַרְמְלָא וּמְתָרְכָא וַחֲלִילָא מַטְעֲיָא יָת אִלֵּין לָא יִסַּב אֱלָהֵין בְּתוּלְתָא מֵעַמֵּיהּ יִסַּב אִתְּתָא׃}
{A widow, or one divorced, or a profaned woman, or a harlot, these shall he not take; but a virgin of his own people shall he take to wife.}{\arabic{verse}}
\rashi{\rashiDH{וחללה.} שנולדה מפסולי כהונה׃}
\threeverse{\arabic{verse}}%Leviticus21:15
{וְלֹֽא\maqqaf יְחַלֵּ֥ל זַרְע֖וֹ בְּעַמָּ֑יו כִּ֛י אֲנִ֥י יְהֹוָ֖ה מְקַדְּשֽׁוֹ׃ \setuma }
{וְלָא יַחֵיל זַרְעֵיהּ בְּעַמֵּיהּ אֲרֵי אֲנָא יְיָ מְקַדְּשֵׁיהּ׃}
{And he shall not profane his seed among his people; for I am the \lord\space who sanctify him.}{\arabic{verse}}
\rashi{\rashiDH{ולא יחלל זרעו.} הא אם נשא אחת מן הפסולות, זרעו הימנה חלל מדין קדושת כהונה׃}
\aliyacounter{שני}
\threeverse{\aliya{שני}}%Leviticus21:16
{וַיְדַבֵּ֥ר יְהֹוָ֖ה אֶל\maqqaf מֹשֶׁ֥ה לֵּאמֹֽר׃}
{וּמַלֵּיל יְיָ עִם מֹשֶׁה לְמֵימַר׃}
{And the \lord\space spoke unto Moses, saying:}{\arabic{verse}}
\threeverse{\arabic{verse}}%Leviticus21:17
{דַּבֵּ֥ר אֶֽל\maqqaf אַהֲרֹ֖ן לֵאמֹ֑ר אִ֣ישׁ מִֽזַּרְעֲךָ֞ לְדֹרֹתָ֗ם אֲשֶׁ֨ר יִהְיֶ֥ה בוֹ֙ מ֔וּם לֹ֣א יִקְרַ֔ב לְהַקְרִ֖יב לֶ֥חֶם אֱלֹהָֽיו׃}
{מַלֵּיל עִם אַהֲרֹן לְמֵימַר גְּבַר מִבְּנָךְ לְדָרֵיהוֹן דִּיהֵי בֵּיהּ מוּמָא לָא יִקְרַב לְקָרָבָא קוּרְבָּנָא קֳדָם אֱלָהֵיהּ׃}
{Speak unto Aaron, saying: Whosoever he be of thy seed throughout their generations that hath a blemish, let him not approach to offer the bread of his God.}{\arabic{verse}}
\rashi{\rashiDH{לחם אלהיו.} מאכל אלהיו, כל סעודה קרויה לחם, כמו עֲבַד לְחֶם רַב (דניאל ה, א)׃}
\threeverse{\arabic{verse}}%Leviticus21:18
{כִּ֥י כׇל\maqqaf אִ֛ישׁ אֲשֶׁר\maqqaf בּ֥וֹ מ֖וּם לֹ֣א יִקְרָ֑ב אִ֤ישׁ עִוֵּר֙ א֣וֹ פִסֵּ֔חַ א֥וֹ חָרֻ֖ם א֥וֹ שָׂרֽוּעַ׃}
{אֲרֵי כָל גְּבַר דְּבֵיהּ מוּמָא לָא יִקְרַב גְּבַר עֲוִיר אוֹ חֲגִיר אוֹ חֲרִים אוֹ סְרִיעַ׃}
{For whatsoever man he be that hath a blemish, he shall not approach: a blind man, or a lame, or he that hath any thing maimed, or anything too long,}{\arabic{verse}}
\rashi{\rashiDH{כי כל איש אשר בו מום לא יקרב.} אינו דין שיקרב, כמו הַקְרִיבֵהוּ נָא לְפֶחָתֶךָ (מלאכי א, ח)׃\quad \rashiDH{חרם.} שחוטמו שקוע בין שתי העינים, שכוחל שתי עיניו כאחת (בכורות מג׃) ì ׃\quad \rashiDH{שרוע.} שאחד מאיבריו גדול מחבירו, עינו אחת גדולה ועינו אחת קטנה, או שוקו אחת ארוכה מחברתה (שם מ׃)׃ 
}
\threeverse{\arabic{verse}}%Leviticus21:19
{א֣וֹ אִ֔ישׁ אֲשֶׁר\maqqaf יִהְיֶ֥ה ב֖וֹ שֶׁ֣בֶר רָ֑גֶל א֖וֹ שֶׁ֥בֶר יָֽד׃}
{אוֹ גְּבַר דִּיהֵי בֵיהּ תְּבָר רִגְלָא אוֹ תְּבָר יְדָא׃}
{or a man that is broken-footed, or broken-handed,}{\arabic{verse}}
\threeverse{\arabic{verse}}%Leviticus21:20
{אֽוֹ\maqqaf גִבֵּ֣ן אוֹ\maqqaf דַ֔ק א֖וֹ תְּבַלֻּ֣ל בְּעֵינ֑וֹ א֤וֹ גָרָב֙ א֣וֹ יַלֶּ֔פֶת א֖וֹ מְר֥וֹחַ אָֽשֶׁךְ׃}
{אוֹ גְבִין אוֹ דוּקָּא אוֹ חִלִּיז בְּעֵינוֹהִי אוֹ גַרְבָּן אוֹ חֲזָזָן אוֹ מְרִיס פַּחְדִּין׃}
{or crook-backed, or a dwarf, or that hath his eye overspread, or is scabbed, or scurvy, or hath his stones crushed;}{\arabic{verse}}
\rashi{\rashiDH{או גבן.} שוריציול״ש בלע״ז, שגביני עיניו שערן ארוך ושוכב (שם מג׃)׃\quad \rashiDH{או דק.} שיש לו בעיניו דוק שקורין טיל״א (שם לח.), כמו הַנּוֹטֶה כַדֹּק (ישעיה מ, כב)׃\quad \rashiDH{או תבלל.} דבר המבלבל את העין, כגון חוט לבן הנמשך מן הלבן ופוסק בסירא, שהוא עוגל המקיף את השחור שקוראים פרוניל״א, והחוט הזה פוסק את העוגל ונכנס בשחור. ותרגום תבלול חִילִיז, לשון חלזון, שהוא דומה לתולעת אותו החוט, וכן כינוהו חכמי ישראל במומי הבכור חלזון נחש עֵינָב (שם)׃\quad \rashiDH{גרב וילפת.} מיני שחין הם׃ \rashiDH{גרב.} זו החרס, שחין היבש מבפנים ומבחוץ׃ \rashiDH{ילפת.} היא חזזית המצרית, ולמה נקראת ילפת, שמלפפת והולכת עד יום המיתה, והוא לח מבחוץ ויבש מבפנים. ובמקום אחר קורא לגרב שחין הלח מבחוץ ויבש מבפנים, שנאמר וּבַגָּרָב וּבֶחָרֶס (דברים כח, כז), כשסמוך גרב אצל חרס קורא לילפת גרב, וכשהוא סמוך אצל ילפת קורא לחרס גרב, כך מפורש בבכורות (בכורות מא.)׃\quad \rashiDH{מרוח אשך.} לפי התרגום מְרִיס פַחֲדִין, שפחדיו מרוססים, שביצים שלו כתותין. פחדין כמו גִידֵי פַחֲדָו יְשֹׂרָגוּ (איוב מ, יז)׃}
\threeverse{\arabic{verse}}%Leviticus21:21
{כׇּל\maqqaf אִ֞ישׁ אֲשֶׁר\maqqaf בּ֣וֹ מ֗וּם מִזֶּ֙רַע֙ אַהֲרֹ֣ן הַכֹּהֵ֔ן לֹ֣א יִגַּ֔שׁ לְהַקְרִ֖יב אֶת\maqqaf אִשֵּׁ֣י יְהֹוָ֑ה מ֣וּם בּ֔וֹ אֵ֚ת לֶ֣חֶם אֱלֹהָ֔יו לֹ֥א יִגַּ֖שׁ לְהַקְרִֽיב׃}
{כָּל גְּבַר דְּבֵיהּ מוּמָא מִזַּרְעָא דְּאַהֲרֹן כָּהֲנָא לָא יִקְרַב לְקָרָבָא יָת קוּרְבָּנַיָּא דַּייָ מוּמָא בֵיהּ יָת קוּרְבַּן אֱלָהֵיהּ לָא יִקְרַב לְקָרָבָא׃}
{no man of the seed of Aaron the priest, that hath a blemish, shall come nigh to offer the offerings of the \lord\space made by fire; he hath a blemish; he shall not come nigh to offer the bread of his God.}{\arabic{verse}}
\rashi{\rashiDH{כל איש אשר בו מום.} לרבות שאר מומין (ת״כ פרק ג, א)׃\quad \rashiDH{מום בו.} בעוד מומו בו פסול, הא אם עבר מומו כשר׃\quad \rashiDH{לחם אלהיו.} כל מאכל קרוי לחם׃}
\threeverse{\arabic{verse}}%Leviticus21:22
{לֶ֣חֶם אֱלֹהָ֔יו מִקׇּדְשֵׁ֖י הַקֳּדָשִׁ֑ים וּמִן\maqqaf הַקֳּדָשִׁ֖ים יֹאכֵֽל׃}
{קוּרְבַּן אֱלָהֵיהּ מִקּוּדְשֵׁי קוּדְשַׁיָּא וּמִן קוּדְשַׁיָּא יֵיכוֹל׃}
{He may eat the bread of his God, both of the most holy, and of the holy.}{\arabic{verse}}
\rashi{\rashiDH{מקדשי הקדשים.} אלו קדשי הקדשים׃\quad \rashiDH{ומן הקדשים יאכל.} אלו קדשים קלים, ואם נאמרו קדשי הקדשים, למה נאמרו קדשים קלים, אם לא נאמר הייתי אומר בקדשי הקדשים יאכל בעל מום, שמצינו שהותרו לזר, שאכל משה בשר המלואים, אבל בחזה ושוק של קדשים קלים לא יאכל, שלא מצינו זר חולק בהן, לכך נאמרו קדשים קלים (ת״כ פרק ג, ח), כך מפורש בזבחים (קא׃)׃}
\threeverse{\arabic{verse}}%Leviticus21:23
{אַ֣ךְ אֶל\maqqaf הַפָּרֹ֜כֶת לֹ֣א יָבֹ֗א וְאֶל\maqqaf הַמִּזְבֵּ֛חַ לֹ֥א יִגַּ֖שׁ כִּֽי\maqqaf מ֣וּם בּ֑וֹ וְלֹ֤א יְחַלֵּל֙ אֶת\maqqaf מִקְדָּשַׁ֔י כִּ֛י אֲנִ֥י יְהֹוָ֖ה מְקַדְּשָֽׁם׃}
{בְּרַם לְפָרוּכְתָּא לָא יֵיעוֹל וּלְמַדְבְּחָא לָא יִקְרַב אֲרֵי מוּמָא בֵיהּ וְלָא יַחֵיל יָת מַקְדָּשַׁי אֲרֵי אֲנָא יְיָ מְקַדִּשְׁהוֹן׃}
{Only he shall not go in unto the veil, nor come nigh unto the altar, because he hath a blemish; that he profane not My holy places; for I am the \lord\space who sanctify them.}{\arabic{verse}}
\rashi{\rashiDH{אך אל הפרכת.} להזות שבע הזאות שעל הפרכת׃\quad \rashiDH{ואל המזבח.} החיצון. ושניהם הוצרכו להכתב, ומפורש בת״כ (שם י.)׃\quad \rashiDH{ולא יחלל את מקדשי.} שאם עבד, עבודתו מחוללת להפסל׃}
\threeverse{\arabic{verse}}%Leviticus21:24
{וַיְדַבֵּ֣ר מֹשֶׁ֔ה אֶֽל\maqqaf אַהֲרֹ֖ן וְאֶל\maqqaf בָּנָ֑יו וְאֶֽל\maqqaf כׇּל\maqqaf בְּנֵ֖י יִשְׂרָאֵֽל׃ \petucha }
{וּמַלֵּיל מֹשֶׁה עִם אַהֲרֹן וְעִם בְּנוֹהִי וְעִם כָּל בְּנֵי יִשְׂרָאֵל׃}
{So Moses spoke unto Aaron, and to his sons, and unto all the children of Israel.}{\arabic{verse}}
\rashi{\rashiDH{וידבר משה.} המצוה הזאת׃\quad \rashiDH{אל אהרן וגו׳ ואל כל בני ישראל.} להזהיר בית דין על הכהנים (שם יב.).}
\newperek
\threeverse{\Roman{chap}}%Leviticus22:1
{וַיְדַבֵּ֥ר יְהֹוָ֖ה אֶל\maqqaf מֹשֶׁ֥ה לֵּאמֹֽר׃}
{וּמַלֵּיל יְיָ עִם מֹשֶׁה לְמֵימַר׃}
{And the \lord\space spoke unto Moses, saying:}{\Roman{chap}}
\threeverse{\arabic{verse}}%Leviticus22:2
{דַּבֵּ֨ר אֶֽל\maqqaf אַהֲרֹ֜ן וְאֶל\maqqaf בָּנָ֗יו וְיִנָּֽזְרוּ֙ מִקׇּדְשֵׁ֣י בְנֵֽי\maqqaf יִשְׂרָאֵ֔ל וְלֹ֥א יְחַלְּל֖וּ אֶת\maqqaf שֵׁ֣ם קׇדְשִׁ֑י אֲשֶׁ֨ר הֵ֧ם מַקְדִּשִׁ֛ים לִ֖י אֲנִ֥י יְהֹוָֽה׃}
{מַלֵּיל עִם אַהֲרֹן וְעִם בְּנוֹהִי וְיִפְרְשׁוּן מִקּוּדְשַׁיָּא דִּבְנֵי יִשְׂרָאֵל וְלָא יַחֲלוּן יָת שְׁמָא דְּקוּדְשִׁי דְּאִנּוּן מַקְדְּשִׁין קֳדָמַי אֲנָא יְיָ׃}
{Speak unto Aaron and to his sons, that they separate themselves from the holy things of the children of Israel, which they hallow unto Me, and that they profane not My holy name: I am the \lord.}{\arabic{verse}}
\rashi{\rashiDH{וינזרו.} אין נזירה אלא פרישה, וכן הוא אומר וַיִּנָּזֵר מֵאַחֲרַי (יחזקאל יד, ז), נָזֹרוּ אָחוֹר (ישעיה א, ד), יפרשו מן הקדשים בימי טומאתן. דבר אחר וינזרו מקדשי בני ישראל, אשר הם מקדישים לי ולא יחללו את שם קדשי, סרס המקרא ודרשהו׃ 
\quad \rashiDH{אשר הם מקדישים לי.} לרבות קדשי כהנים עצמן׃}
\threeverse{\arabic{verse}}%Leviticus22:3
{אֱמֹ֣ר אֲלֵהֶ֗ם לְדֹרֹ֨תֵיכֶ֜ם כׇּל\maqqaf אִ֣ישׁ \legarmeh  אֲשֶׁר\maqqaf יִקְרַ֣ב מִכׇּל\maqqaf זַרְעֲכֶ֗ם אֶל\maqqaf הַקֳּדָשִׁים֙ אֲשֶׁ֨ר יַקְדִּ֤ישׁוּ בְנֵֽי\maqqaf יִשְׂרָאֵל֙ לַֽיהֹוָ֔ה וְטֻמְאָת֖וֹ עָלָ֑יו וְנִכְרְתָ֞ה הַנֶּ֧פֶשׁ הַהִ֛וא מִלְּפָנַ֖י אֲנִ֥י יְהֹוָֽה׃}
{אֵימַר לְהוֹן לְדָרֵיכוֹן כָּל גְּבַר דְּיִקְרַב מִכָּל זַרְעֲכוֹן לְקוּדְשַׁיָּא דְּיַקְדְּשׁוּן בְּנֵי יִשְׂרָאֵל קֳדָם יְיָ וּסְאוֹבְתֵיהּ עֲלוֹהִי וְיִשְׁתֵּיצֵי אֲנָשָׁא הַהוּא מִן קֳדָמַי אֲנָא יְיָ׃}
{Say unto them: Whosoever he be of all your seed throughout your generations, that approacheth unto the holy things, which the children of Israel hallow unto the \lord, having his uncleanness upon him, that soul shall be cut off from before Me: I am the \lord.}{\arabic{verse}}
\rashi{\rashiDH{כל איש אשר יקרב.} אין קריבה זו אלא אכילה, וכן מצינו שנאמרה אזהרת אכילת קדשים בטומאה בלשון נגיעה, בכל קדש לא תגע, (ויקרא יב, ד) אזהרה לאוכל, ולמדוה רבותינו (זבחים לג׃) מגזירה שוה. ואי אפשר לומר שחייב על הנגיעה, שהרי נאמר כרת על האכילה בצו את אהרן שתי כריתות זו אצל זו, ואם על הנגיעה חייב לא הוצרך לחייבו על האכילה, וכן נדרש בת״כ (פרשתא ד, ז), וכי יש נוגע חייב, אם כן מה תלמוד לומר יקרב, משיכשר להקרב, שאין חייבין עליו משום טומאה, אלא אם כן קרבו מתיריו, ואם תאמר שלש כריתות בטומאת כהנים למה, כבר נדרשו במס׳ שבועות (ז.) אחת לכלל, ואחת לפרט וכו׳׃\quad \rashiDH{וטמאתו עליו.} וטומאת האדם עליו. יכול בבשר הכתוב מדבר וטומאתו של בשר עליו, ובטהור שאכל את הטמא הכתוב מדבר, על כרחך ממשמעו אתה למד, במי שטומאתו פורחת ממנו הכתוב מדבר, וזהו האדם שיש לו טהרה בטבילה (זבחים מג׃)׃\quad \rashiDH{ונכרתה וגו׳.} יכול מצד זה לצד זה, יכרת ממקומו ויתיישב במקום אחר, תלמוד לומר אני ה׳, בכל מקום אני׃}
\threeverse{\arabic{verse}}%Leviticus22:4
{אִ֣ישׁ אִ֞ישׁ מִזֶּ֣רַע אַהֲרֹ֗ן וְה֤וּא צָר֙וּעַ֙ א֣וֹ זָ֔ב בַּקֳּדָשִׁים֙ לֹ֣א יֹאכַ֔ל עַ֖ד אֲשֶׁ֣ר יִטְהָ֑ר וְהַנֹּגֵ֙עַ֙ בְּכׇל\maqqaf טְמֵא\maqqaf נֶ֔פֶשׁ א֣וֹ אִ֔ישׁ אֲשֶׁר\maqqaf תֵּצֵ֥א מִמֶּ֖נּוּ שִׁכְבַת\maqqaf זָֽרַע׃}
{גְּבַר גְּבַר מִזַּרְעָא דְּאַהֲרֹן וְהוּא סְגִיר אוֹ דָאִיב בְּקוּדְשַׁיָּא לָא יֵיכוֹל עַד דְּיִדְכֵּי וּדְיִקְרַב בְּכָל טְמֵי נַפְשָׁא אוֹ גְּבַר דְּתִפּוֹק מִנֵּיהּ שִׁכְבַת זַרְעָא׃}
{What man soever of the seed of Aaron is a leper, or hath an issue, he shall not eat of the holy things, until he be clean. And whoso toucheth any one that is unclean by the dead; or from whomsoever the flow of seed goeth out;}{\arabic{verse}}
\rashi{\rashiDH{בכל טמא נפש.} במי שנטמא במת׃}
\threeverse{\arabic{verse}}%Leviticus22:5
{אוֹ\maqqaf אִישׁ֙ אֲשֶׁ֣ר יִגַּ֔ע בְּכׇל\maqqaf שֶׁ֖רֶץ אֲשֶׁ֣ר יִטְמָא\maqqaf ל֑וֹ א֤וֹ בְאָדָם֙ אֲשֶׁ֣ר יִטְמָא\maqqaf ל֔וֹ לְכֹ֖ל טֻמְאָתֽוֹ׃}
{אוֹ גְּבַר דְּיִקְרַב בְּכָל רִחְשָׁא דְּיִסְתָּאַב לֵיהּ אוֹ בַאֲנָשָׁא דְּיִסְתָּאַב לֵיהּ לְכֹל סְאוֹבְתֵיהּ׃}
{or whosoever toucheth any swarming thing, whereby he may be made unclean, or a man of whom he may take uncleanness, whatsoever uncleanness he hath;}{\arabic{verse}}
\rashi{\rashiDH{בכל שרץ אשר יטמא לו.} בשיעור הראוי לטמא (ת״כ פרק ד, ד), בכעדשה (חגיגה יא.  ת״כ)׃\quad \rashiDH{או באדם.} במת׃\quad \rashiDH{אשר יטמא לו.} כשיעורו לטמא, וזהו כזית׃\quad \rashiDH{לכל טומאתו.} לרבות נוגע בזב, וזבה, נדה, ויולדת׃}
\threeverse{\arabic{verse}}%Leviticus22:6
{נֶ֚פֶשׁ אֲשֶׁ֣ר תִּגַּע\maqqaf בּ֔וֹ וְטָמְאָ֖ה עַד\maqqaf הָעָ֑רֶב וְלֹ֤א יֹאכַל֙ מִן\maqqaf הַקֳּדָשִׁ֔ים כִּ֛י אִם\maqqaf רָחַ֥ץ בְּשָׂר֖וֹ בַּמָּֽיִם׃}
{אֱנָשׁ דְּיִקְרַב בֵּיהּ וִיהֵי מְסָאַב עַד רַמְשָׁא וְלָא יֵיכוֹל מִן קוּדְשַׁיָּא אֱלָהֵין אַסְחִי בִּשְׂרֵיהּ בְּמַיָּא׃}
{the soul that toucheth any such shall be unclean until the even, and shall not eat of the holy things, unless he bathe his flesh in water.}{\arabic{verse}}
\rashi{\rashiDH{נפש אשר תגע בו.} באחד מן הטמאים הללו׃}
\threeverse{\arabic{verse}}%Leviticus22:7
{וּבָ֥א הַשֶּׁ֖מֶשׁ וְטָהֵ֑ר וְאַחַר֙ יֹאכַ֣ל מִן\maqqaf הַקֳּדָשִׁ֔ים כִּ֥י לַחְמ֖וֹ הֽוּא׃}
{וּכְמֵיעַל שִׁמְשָׁא וְיִדְכֵּי וּבָתַר כֵּן יֵיכוֹל מִן קוּדְשַׁיָּא אֲרֵי לַחְמֵיהּ הוּא׃}
{And when the sun is down, he shall be clean; and afterward he may eat of the holy things, because it is his bread.}{\arabic{verse}}
\rashi{\rashiDH{ואחר יאכל מן הקדשים.} נדרש ביבמות (עד׃) בתרומה, שמותר לאכלה בהערב השמש׃\quad \rashiDH{מן הקדשים} ולא כל הקדשים׃}
\threeverse{\arabic{verse}}%Leviticus22:8
{נְבֵלָ֧ה וּטְרֵפָ֛ה לֹ֥א יֹאכַ֖ל לְטׇמְאָה\maqqaf בָ֑הּ אֲנִ֖י יְהֹוָֽה׃}
{נְבִילָא וּתְבִירָא לָא יֵיכוֹל לְאִסְתַּאָבָא בַהּ אֲנָא יְיָ׃}
{That which dieth of itself, or is torn of beasts, he shall not eat to defile himself therewith: I am the \lord.}{\arabic{verse}}
\rashi{\rashiDH{נבלה וטרפה לא יאכל לטמאה בה.} לענין הטומאה הזהיר כאן, שאם אכל נבלת עוף טהור שאין לה טומאת מגע ומשא, אלא טומאת אכילה בבית הבליעה, אסור לאכול בקדשים, וצריך לומר וטרפה, מי שיש במינו טרפה, יצא נבלת עוף טמא שאין במינו טרפה׃}
\threeverse{\arabic{verse}}%Leviticus22:9
{וְשָׁמְר֣וּ אֶת\maqqaf מִשְׁמַרְתִּ֗י וְלֹֽא\maqqaf יִשְׂא֤וּ עָלָיו֙ חֵ֔טְא וּמֵ֥תוּ ב֖וֹ כִּ֣י יְחַלְּלֻ֑הוּ אֲנִ֥י יְהֹוָ֖ה מְקַדְּשָֽׁם׃}
{וְיִטְּרוּן יָת מַטְּרַת מֵימְרִי וְלָא יְקַבְּלוּן עֲלוֹהִי חוֹבָא וִימוּתוּן בֵּיהּ אֲרֵי יַחֲלוּנֵּיהּ אֲנָא יְיָ מְקַדִּשְׁהוֹן׃}
{They shall therefore keep My charge, lest they bear sin for it, and die therein, if they profane it: I am the \lord\space who sanctify them.}{\arabic{verse}}
\rashi{\rashiDH{ושמרו את משמרתי.} מלאכול תרומה בטומאת הגוף׃\quad \rashiDH{ומתו בו.} למדנו שהיא מיתה בידי שמים (סנהדרין פג.)׃ 
}
\threeverse{\arabic{verse}}%Leviticus22:10
{וְכׇל\maqqaf זָ֖ר לֹא\maqqaf יֹ֣אכַל קֹ֑דֶשׁ תּוֹשַׁ֥ב כֹּהֵ֛ן וְשָׂכִ֖יר לֹא\maqqaf יֹ֥אכַל קֹֽדֶשׁ׃}
{וְכָל חִילוֹנַי לָא יֵיכוֹל קוּדְשָׁא תּוֹתָבָא דְּכָהֲנָא וַאֲגִירָא לָא יֵיכוֹל קוּדְשָׁא׃}
{There shall no acommon man eat of the holy thing; a tenant of a priest, or a hired servant, shall not eat of the holy thing.}{\arabic{verse}}
\rashi{\rashiDH{לא יאכל קדש.} בתרומה הכתוב מדבר, שכל הענין דבר בה (שם פג׃)׃\quad \rashiDH{תושב כהן ושכיר.} תושבו של כהן ושכירו, לפיכך תושב זה נקוד פתח, לפי שהוא דבוק, ואיזהו תושב, זה נרצע שהוא קנוי לו עד היובל, ואיזהו שכיר, זה קנוי קנין שנים שיוצא בשש, בא הכתוב ולמדך כאן, שאין גופו קנוי לאדוניו לאכול בתרומתו (יבמות ע.)׃}
\threeverse{\arabic{verse}}%Leviticus22:11
{וְכֹהֵ֗ן כִּֽי\maqqaf יִקְנֶ֥ה נֶ֙פֶשׁ֙ קִנְיַ֣ן כַּסְפּ֔וֹ ה֖וּא יֹ֣אכַל בּ֑וֹ וִילִ֣יד בֵּית֔וֹ הֵ֖ם יֹאכְל֥וּ בְלַחְמֽוֹ׃}
{וְכָהִין אֲרֵי יִקְנֵי נְפַשׁ קִנְיַן כַּסְפֵּיהּ הוּא יֵיכוֹל בֵּיהּ וִילִידֵי בֵיתֵיהּ אִנּוּן יֵיכְלוּן בְּלַחְמֵיהּ׃}
{But if a priest buy any soul, the purchase of his money, he may eat of it; and such as are born in his house, they may eat of his bread.}{\arabic{verse}}
\rashi{\rashiDH{וכהן כי יקנה נפש.} עבד כנעני שקנוי לגופו׃\quad \rashiDH{ויליד ביתו.} אלו בני השפחות, ואשת כהן אוכלת בתרומה מן המקרא הזה, שאף היא קנין כספו (כתובות נז׃), ועוד למד ממקרא אחר כל טהור בביתך וגו׳ בספרי (קרח יז)׃}
\threeverse{\arabic{verse}}%Leviticus22:12
{וּבַ֨ת\maqqaf כֹּהֵ֔ן כִּ֥י תִהְיֶ֖ה לְאִ֣ישׁ זָ֑ר הִ֕וא בִּתְרוּמַ֥ת הַקֳּדָשִׁ֖ים לֹ֥א תֹאכֵֽל׃}
{וּבַת כָּהִין אֲרֵי תְהֵי לִגְבַר חִילוֹנַי הִיא בְּאַפְרָשׁוּת קוּדְשַׁיָּא לָא תֵיכוֹל׃}
{And if a priest’s daughter be married unto a common man, she shall not eat of that which is set apart from the holy things.}{\arabic{verse}}
\rashi{\rashiDH{לאיש זר.} ללוי וישראל׃}
\threeverse{\arabic{verse}}%Leviticus22:13
{וּבַת\maqqaf כֹּהֵן֩ כִּ֨י תִהְיֶ֜ה אַלְמָנָ֣ה וּגְרוּשָׁ֗ה וְזֶ֘רַע֮ אֵ֣ין לָהּ֒ וְשָׁבָ֞ה אֶל\maqqaf בֵּ֤ית אָבִ֙יהָ֙ כִּנְעוּרֶ֔יהָ מִלֶּ֥חֶם אָבִ֖יהָ תֹּאכֵ֑ל וְכׇל\maqqaf זָ֖ר לֹא\maqqaf יֹ֥אכַל בּֽוֹ׃}
{וּבַת כָּהִין אֲרֵי תְהֵי אַרְמְלָא וּמְתָרְכָא וּבַר לֵית לַהּ וּתְתוּב לְבֵית אֲבוּהָא כְּרַבְיוּתַהּ מִלַּחְמָא דַּאֲבוּהָא תֵּיכוֹל וְכָל חִילוֹנַי לָא יֵיכוֹל בֵּיהּ׃}
{But if a priest’s daughter be a widow, or divorced, and have no child, and is returned unto her father’s house, as in her youth, she may eat of her father’s bread; but there shall no common man}{\arabic{verse}}
\rashi{\rashiDH{אלמנה וגרושה.} מן האיש הזר׃\quad \rashiDH{וזרע אין לה.} ממנו׃\quad \rashiDH{ושבה.} הא אם יש לה זרע ממנו, אסורה בתרומה כל זמן שהזרע קיים (יבמות פז.)׃\quad \rashiDH{וכל זר לא יאכל בו.} לא בא אלא להוציא את האונן שמותר בתרומה, זרות אמרתי לך, ולא אנינות (שם סח׃)׃ 
}
\threeverse{\arabic{verse}}%Leviticus22:14
{וְאִ֕ישׁ כִּֽי\maqqaf יֹאכַ֥ל קֹ֖דֶשׁ בִּשְׁגָגָ֑ה וְיָסַ֤ף חֲמִֽשִׁיתוֹ֙ עָלָ֔יו וְנָתַ֥ן לַכֹּהֵ֖ן אֶת\maqqaf הַקֹּֽדֶשׁ׃}
{וּגְבַר אֲרֵי יֵיכוֹל קוּדְשָׁא בְּשָׁלוּ וְיוֹסֵיף חוּמְשֵׁיהּ עֲלוֹהִי וְיִתֵּין לְכָהֲנָא יָת קוּדְשָׁא׃}
{And if a man eat of the holy thing through error, then he shall put the fifth part thereof unto it, and shall give unto the priest the holy thing.}{\arabic{verse}}
\rashi{\rashiDH{כי יאכל קדש.} תרומה׃\quad \rashiDH{ונתן לכהן את הקדש.} דבר הראוי להיות קדש, שאינו פורע לו מעות, אלא פירות של חולין, והן נעשין תרומה (פסחים לב.)׃}
\threeverse{\arabic{verse}}%Leviticus22:15
{וְלֹ֣א יְחַלְּל֔וּ אֶת\maqqaf קׇדְשֵׁ֖י בְּנֵ֣י יִשְׂרָאֵ֑ל אֵ֥ת אֲשֶׁר\maqqaf יָרִ֖ימוּ לַיהֹוָֽה׃}
{וְלָא יַחֲלוּן יָת קוּדְשַׁיָּא דִּבְנֵי יִשְׂרָאֵל יָת דְּיַפְרְשׁוּן קֳדָם יְיָ׃}
{And they shall not profane the holy things of the children of Israel, which they set apart unto the \lord;}{\arabic{verse}}
\rashi{\rashiDH{ולא יחללו וגו׳.} להאכילם לזרים׃}
\threeverse{\arabic{verse}}%Leviticus22:16
{וְהִשִּׂ֤יאוּ אוֹתָם֙ עֲוֺ֣ן אַשְׁמָ֔ה בְּאׇכְלָ֖ם אֶת\maqqaf קׇדְשֵׁיהֶ֑ם כִּ֛י אֲנִ֥י יְהֹוָ֖ה מְקַדְּשָֽׁם׃ \petucha }
{וִיקַבְּלוּן עֲלֵיהוֹן עֲוָיָין וְחוֹבִין בְּמֵיכַלְהוֹן בְּסוֹאֲבָא יָת קוּדְשֵׁיהוֹן אֲרֵי אֲנָא יְיָ מְקַדִּשְׁהוֹן׃}
{and so cause them to bear the iniquity that bringeth guilt, when they eat their holy things; for I am the \lord\space who sanctify them.}{\arabic{verse}}
\rashi{\rashiDH{והשיאו אותם.} את עצמם יטענו עון באכלם את קדשיהם, שהובדלו לשם תרומה וקדשו, ונאסרו עליהם. ואונקלוס שתרגם בְּמֵיכָלְהוֹן בְּסוֹאֲבָא, שלא לצורך תרגמו כן׃\quad \rashiDH{והשיאו אותם.} זה אחד מג׳ אתים שהיה רבי ישמעאל דורש בתורה שמדברים באדם עצמו, וכן בְּיוֹם מְלֹאת יְמֵי נִזְרוֹ יָבִיא אֹתוֹ (במדבר ו, יג), הוא יביא את עצמו, וכן וַיִּקְבֹּר אֹתוֹ בַגַּיא (דברים לד, ו), הוא קבר את עצמו, כך נדרש בספרי (נשא לב)׃}
\aliyacounter{שלישי}
\newseder{18}
\threeverse{\aliya{שלישי}\newline\vspace{-4pt}\newline\seder{יח}}%Leviticus22:17
{וַיְדַבֵּ֥ר יְהֹוָ֖ה אֶל\maqqaf מֹשֶׁ֥ה לֵּאמֹֽר׃}
{וּמַלֵּיל יְיָ עִם מֹשֶׁה לְמֵימַר׃}
{And the \lord\space spoke unto Moses, saying:}{\arabic{verse}}
\threeverse{\arabic{verse}}%Leviticus22:18
{דַּבֵּ֨ר אֶֽל\maqqaf אַהֲרֹ֜ן וְאֶל\maqqaf בָּנָ֗יו וְאֶל֙ כׇּל\maqqaf בְּנֵ֣י יִשְׂרָאֵ֔ל וְאָמַרְתָּ֖ אֲלֵהֶ֑ם אִ֣ישׁ אִישׁ֩ מִבֵּ֨ית יִשְׂרָאֵ֜ל וּמִן\maqqaf הַגֵּ֣ר בְּיִשְׂרָאֵ֗ל אֲשֶׁ֨ר יַקְרִ֤יב קׇרְבָּנוֹ֙ לְכׇל\maqqaf נִדְרֵיהֶם֙ וּלְכׇל\maqqaf נִדְבוֹתָ֔ם אֲשֶׁר\maqqaf יַקְרִ֥יבוּ לַיהֹוָ֖ה לְעֹלָֽה׃}
{מַלֵּיל עִם אַהֲרֹן וְעִם בְּנוֹהִי וְעִם כָּל בְּנֵי יִשְׂרָאֵל וְתֵימַר לְהוֹן גְּבַר גְּבַר מִבֵּית יִשְׂרָאֵל וּמִן גִּיּוֹרַיָּא בְּיִשְׂרָאֵל דִּיקָרֵיב קוּרְבָּנֵיהּ לְכָל נִדְרֵיהוֹן וּלְכָל נִדְבָתְהוֹן דִּיקָרְבוּן קֳדָם יְיָ לַעֲלָתָא׃}
{Speak unto Aaron, and to his sons, and unto all the children of Israel, and say unto them: Whosoever he be of the house of Israel, or of the strangers in Israel, that bringeth his offering, whether it be any of their vows, or any of their free-will-offerings, which are brought unto the \lord\space for a burnt-offering;}{\arabic{verse}}
\rashi{\rashiDH{נדריהם.} הרי עלי׃\quad \rashiDH{נדבותם.} הרי זו (מגילה ח.)׃}
\threeverse{\arabic{verse}}%Leviticus22:19
{לִֽרְצֹנְכֶ֑ם תָּמִ֣ים זָכָ֔ר בַּבָּקָ֕ר בַּכְּשָׂבִ֖ים וּבָֽעִזִּֽים׃}
{לְרַעֲוָא לְכוֹן שְׁלִים דְּכַר בְּתוֹרַיָּא בְּאִמְּרַיָּא וּבְעִזַּיָּא׃}
{that ye may be accepted, ye shall offer a male without blemish, of the beeves, of the sheep, or of the goats.}{\arabic{verse}}
\rashi{\rashiDH{לרצונכם.} הביאו דבר הראוי לרצות אתכם לפני, שיהא לכם לרצון, אפיישמנ״ט בלע״ז. ואיזהו הראוי לרצון׃\quad \rashiDH{תמים זכר בבקר בכשבים ובעזים.} אבל בעולת העוף אין צריך תמות וזכרות, ואינו נפסל במום, אלא בחסרון אבר׃}
\threeverse{\arabic{verse}}%Leviticus22:20
{כֹּ֛ל אֲשֶׁר\maqqaf בּ֥וֹ מ֖וּם לֹ֣א תַקְרִ֑יבוּ כִּי\maqqaf לֹ֥א לְרָצ֖וֹן יִהְיֶ֥ה לָכֶֽם׃}
{כֹּל דְּבֵיהּ מוּמָא לָא תְקָרְבוּן אֲרֵי לָא לְרַעֲוָא יְהֵי לְכוֹן׃}
{But whatsoever hath a blemish, that shall ye not bring; for it shall not be acceptable for you.}{\arabic{verse}}
\threeverse{\arabic{verse}}%Leviticus22:21
{וְאִ֗ישׁ כִּֽי\maqqaf יַקְרִ֤יב זֶֽבַח\maqqaf שְׁלָמִים֙ לַיהֹוָ֔ה לְפַלֵּא\maqqaf נֶ֙דֶר֙ א֣וֹ לִנְדָבָ֔ה בַּבָּקָ֖ר א֣וֹ בַצֹּ֑אן תָּמִ֤ים יִֽהְיֶה֙ לְרָצ֔וֹן כׇּל\maqqaf מ֖וּם לֹ֥א יִהְיֶה\maqqaf בּֽוֹ׃}
{וּגְבַר אֲרֵי יְקָרֵיב נִכְסַת קוּדְשַׁיָּא קֳדָם יְיָ לְפָרָשָׁא נִדְרָא אוֹ לִנְדַבְתָּא בְּתוֹרֵי אוֹ בְעָנָא שְׁלִים יְהֵי לְרַעֲוָא כָּל מוּם לָא יְהֵי בֵיהּ׃}
{And whosoever bringeth a sacrifice of peace-offerings unto the \lord\space in fulfilment of a vow clearly uttered, or for a freewill-offering, of the herd or of the flock, it shall be perfect to be accepted; there shall be no blemish therein.}{\arabic{verse}}
\rashi{\rashiDH{לפלא נדר.} להפריש בדיבורו׃}
\threeverse{\arabic{verse}}%Leviticus22:22
{עַוֶּ֩רֶת֩ א֨וֹ שָׁב֜וּר אוֹ\maqqaf חָר֣וּץ אֽוֹ\maqqaf יַבֶּ֗לֶת א֤וֹ גָרָב֙ א֣וֹ יַלֶּ֔פֶת לֹא\maqqaf תַקְרִ֥יבוּ אֵ֖לֶּה לַיהֹוָ֑ה וְאִשֶּׁ֗ה לֹא\maqqaf תִתְּנ֥וּ מֵהֶ֛ם עַל\maqqaf הַמִּזְבֵּ֖חַ לַיהֹוָֽה׃}
{עֲוִיר אוֹ תְּבִיר אוֹ פְסִיק אוֹ יַבְלָן אוֹ גַּרְבָּן אוֹ חֲזָזָן לָא תְקָרְבוּן אִלֵּין קֳדָם יְיָ וְקוּרְבָּנָא לָא תִתְּנוּן מִנְּהוֹן עַל מַדְבְּחָא קֳדָם יְיָ׃}
{Blind, or broken, or maimed, or having a wen, or scabbed, or scurvy, ye shall not offer these unto the \lord, nor make an offering by fire of them upon the altar unto the \lord.}{\arabic{verse}}
\rashi{\rashiDH{עורת.} שם דבר של מום, עִוָּרוֹן בלשון נקבה, שלא יהא בו מום של עורת׃\quad \rashiDH{או שבור.} לא יהיה׃\quad \rashiDH{חרוץ.} ריס של עין שנסדק או שנפגם (בכורות לח.), וכן שפתו שנסדקה או נפגמה (שם לט.)׃\quad \rashiDH{יבלת.} ורוא״ה בלע״ז׃\quad \rashiDH{גרב.} מין חזזית, וכן ילפת ולשון ילפת כמו וַיִּלְפֹּת שִׁמְשׁוֹן (שופטים טז, כט), שאחוזה בו עד יום מיתה, שאין לה רפואה (בכורות מא.)׃\quad \rashiDH{לא תקריבו.} שלש פעמים, להזהיר על הקדשתן, ועל שחיטתן, ועל זריקת דמן (תמורה ו׃)׃\quad \rashiDH{ואשה לא תתנו.} אזהרת הקטרתן׃}
\threeverse{\arabic{verse}}%Leviticus22:23
{וְשׁ֥וֹר וָשֶׂ֖ה שָׂר֣וּעַ וְקָל֑וּט נְדָבָה֙ תַּעֲשֶׂ֣ה אֹת֔וֹ וּלְנֵ֖דֶר לֹ֥א יֵרָצֶֽה׃}
{וְתוֹר וְאִמַּר יַתִּיר וְחַסִּיר נְדַבְתָּא תַּעֲבֵיד יָתֵיהּ וּלְנִדְרָא לָא יִתְרְעֵי׃}
{Either a bullock or a lamb that hath any thing too long or too short, that mayest thou offer for a freewill-offering; but for a vow it shall not be accepted.}{\arabic{verse}}
\rashi{\rashiDH{שרוע.} אבר גדול מחבירו (בכורות מ.)׃\quad \rashiDH{וקלוט.} פרסותיו קלוטות׃\quad \rashiDH{נדבה תעשה אתו.} לבדק הבית׃\quad \rashiDH{ולנדר.} למזבח׃\quad \rashiDH{לא ירצה.} איזה הקדש בא לרצות, הוי אומר זה הקדש המזבח (ת״כ פרק ז, ו)׃}
\threeverse{\arabic{verse}}%Leviticus22:24
{וּמָע֤וּךְ וְכָתוּת֙ וְנָת֣וּק וְכָר֔וּת לֹ֥א תַקְרִ֖יבוּ לַֽיהֹוָ֑ה וּֽבְאַרְצְכֶ֖ם לֹ֥א תַעֲשֽׂוּ׃}
{וְדִמְרִיס וְדִרְסִיס וְדִשְׁלִיף וְדִגְזִיר לָא תְקָרְבוּן קֳדָם יְיָ וּבַאֲרַעְכוֹן לָא תַעְבְּדוּן׃}
{That which hath its stones bruised, or crushed, or torn, or cut, ye shall not offer unto the \lord; neither shall ye do thus in your land.}{\arabic{verse}}
\rashi{\rashiDH{ומעוך וכתות ונתוק וכרות.} בביצים או בגיד׃\quad \rashiDH{מעוך.} ביציו מעוכין ביד׃\quad \rashiDH{כתות.} כתושים יותר ממעוך (בכורות לט׃)׃\quad \rashiDH{נתוק.} תלושין ביד עד שנפסקו חוטים שתלויים בהן, אבל נתונים הם בתוך הכיס, והכיס לא נתלש׃\quad \rashiDH{וכרות.} כרותין בכלי ועודן בכיס׃\quad \rashiDH{ומעוך.} תרגומו וְדִמְרִיס, זה לשונו בארמית, לשון כתישה׃\quad \rashiDH{וכתות.} תרגומו וְדִרְסִיס, כמו הַבַּיִת הַגָּדוֹל רְסִיסִים (עמוס ו, יא), בְּקִיעוֹת דַּקּוֹת, וכן קנה המרוסס (שבת פ׃)׃\quad \rashiDH{ובארצכם לא תעשו.} דבר זה לסרס שום בהמה וחיה ואפילו טמאה, לכך נאמר בארצכם, לרבות כל אשר בארצכם, שאי אפשר לומר לא נצטוו על הסרוס אלא בארץ, שהרי סרוס חובת הגוף הוא, וכל חובת הגוף נוהגת בין בארץ בין בחוצה לארץ (קידושין לו׃)׃}
\threeverse{\arabic{verse}}%Leviticus22:25
{וּמִיַּ֣ד בֶּן\maqqaf נֵכָ֗ר לֹ֥א תַקְרִ֛יבוּ אֶת\maqqaf לֶ֥חֶם אֱלֹהֵיכֶ֖ם מִכׇּל\maqqaf אֵ֑לֶּה כִּ֣י מׇשְׁחָתָ֤ם בָּהֶם֙ מ֣וּם בָּ֔ם לֹ֥א יֵרָצ֖וּ לָכֶֽם׃ \setuma }
{וּמִיַּד בַּר עַמְמִין לָא תְקָרְבוּן יָת קוּרְבַּן אֱלָהֲכוֹן מִכָּל אִלֵּין אֲרֵי חִבֻוּלְהוֹן בְּהוֹן מוּמָא בְּהוֹן לָא לְרַעֲוָא יְהוֹן לְכוֹן׃}
{Neither from the hand of a foreigner shall ye offer the bread of your God of any of these, because their corruption is in them, there is a blemish in them; they shall not be accepted for you.}{\arabic{verse}}
\rashi{\rashiDH{ומיד בן נכר.} נכרי שהביא קרבן ביד כהן להקריבו לשמים. \rashiDH{לא תקריבו.} לו בעל מום, ואף על פי שלא נאסרו בעלי מומים לקרבן בני נח אלא אם כן מחוסרי אבר, זאת נוהגת בבמה שבשדות, (תמורה ז.) אבל על המזבח שבמשכן לא תקריבוה, אבל תמימה תקבלו מהם, לכך נאמר למעלה (פסוק יח) איש איש, לרבות את הנכרים שנודרים נדרים ונדבות כישראל (חולין יג׃)׃\quad \rashiDH{משחתם.} חִבּוּלְהוֹן׃\quad \rashiDH{לא ירצו לכם.} לכפר עליכם׃}
\threeverse{\arabic{verse}}%Leviticus22:26
{וַיְדַבֵּ֥ר יְהֹוָ֖ה אֶל\maqqaf מֹשֶׁ֥ה לֵּאמֹֽר׃}
{וּמַלֵּיל יְיָ עִם מֹשֶׁה לְמֵימַר׃}
{And the \lord\space spoke unto Moses, saying:}{\arabic{verse}}
\threeverse{\arabic{verse}}%Leviticus22:27
{שׁ֣וֹר אוֹ\maqqaf כֶ֤שֶׂב אוֹ\maqqaf עֵז֙ כִּ֣י יִוָּלֵ֔ד וְהָיָ֛ה שִׁבְעַ֥ת יָמִ֖ים תַּ֣חַת אִמּ֑וֹ וּמִיּ֤וֹם הַשְּׁמִינִי֙ וָהָ֔לְאָה יֵרָצֶ֕ה לְקׇרְבַּ֥ן אִשֶּׁ֖ה לַיהֹוָֽה׃}
{תּוֹר אוֹ אִמַּר אוֹ עֵז אֲרֵי יִתְיְלֵיד וִיהֵי שִׁבְעָא יוֹמִין בָּתַר אִמֵּיהּ וּמִיּוֹמָא תְּמִינָאָה וּלְהַלְאָה יִתְרְעֵי לְקָרָבָא קוּרְבָּנָא קֳדָם יְיָ׃}
{When a bullock, or a sheep, or a goat, is brought forth, then it shall be seven days under the dam; but from the eighth day and thenceforth it may be accepted for an offering made by fire unto the \lord.}{\arabic{verse}}
\rashi{\rashiDH{כי יולד.} פרט ליוצא דופן (שם לח׃)׃ 
}
\threeverse{\arabic{verse}}%Leviticus22:28
{וְשׁ֖וֹר אוֹ\maqqaf שֶׂ֑ה אֹת֣וֹ וְאֶת\maqqaf בְּנ֔וֹ לֹ֥א תִשְׁחֲט֖וּ בְּי֥וֹם אֶחָֽד׃}
{וְתוֹרְתָא אוֹ שִׂיתָא לַהּ וְלִבְרַהּ לָא תִכְּסוּן בְּיוֹמָא חַד׃}
{And whether it be cow or ewe, ye shall not kill it and its young both in one day.}{\arabic{verse}}
\rashi{\rashiDH{אתו ואת בנו.} נוהג בנקבה, שאסור לשחוט האם והבן או הבת, ואינו נוהג בזכרים, ומותר לשחוט האב והבן (ת״כ פרק ח, א  חולין עח׃)׃\quad \rashiDH{אתו ואת בנו.} אף בנו ואותו במשמע (חולין פב.)׃}
\threeverse{\arabic{verse}}%Leviticus22:29
{וְכִֽי\maqqaf תִזְבְּח֥וּ זֶֽבַח\maqqaf תּוֹדָ֖ה לַיהֹוָ֑ה לִֽרְצֹנְכֶ֖ם תִּזְבָּֽחוּ׃}
{וַאֲרֵי תִכְּסוּן נִכְסַת תּוֹדְתָא קֳדָם יְיָ לְרַעֲוָא לְכוֹן תִּכְּסוּן׃}
{And when ye sacrifice a sacrifice of thanksgiving unto the \lord, ye shall sacrifice it that ye may be accepted.}{\arabic{verse}}
\rashi{\rashiDH{לרצנכם תזבחו.} תחלת זביחתכם הזהרו שתהא לרצון לכם, ומהו הרצון׃ 
}
\threeverse{\arabic{verse}}%Leviticus22:30
{בַּיּ֤וֹם הַהוּא֙ יֵאָכֵ֔ל לֹֽא\maqqaf תוֹתִ֥ירוּ מִמֶּ֖נּוּ עַד\maqqaf בֹּ֑קֶר אֲנִ֖י יְהֹוָֽה׃}
{בְּיוֹמָא הַהוּא יִתְאֲכִיל לָא תַשְׁאֲרוּן מִנֵּיהּ עַד צַפְרָא אֲנָא יְיָ׃}
{On the same day it shall be eaten; ye shall leave none of it until the morning: I am the \lord.}{\arabic{verse}}
\rashi{\rashiDH{ביום ההוא יאכל.} לא בא להזהיר אלא שתהא שחיטה על מנת כן, אל תשחוטוהו על מנת לאכלו למחר, שאם תחשבו בו מחשבת פסול לא יהא לכם לרצון. דבר אחר לרצונכם, לדעתכם, מכאן למתעסק שפסול בשחיטת קדשים (חולין יג.), ואף על פי שפרט בנאכלים לשני ימים, חזר ופרט בנאכלין ליום אחד, שתהא זביחתן על מנת לאכלן בזמנן׃ \rashiDH{ביום ההוא יאכל.} לא בא להזהיר אלא שתהא שחיטה על מנת כן, שאם לקבוע לה זמן אכילה, כבר כתיב וּבְשַׂר זֶבַח תּוֹדַת שְׁלָמָיו וגו׳ (ויקרא ז, טו)׃\quad \rashiDH{אני ה׳.} דע מי גזר על הדבר ואל יקל בעיניך׃}
\threeverse{\arabic{verse}}%Leviticus22:31
{וּשְׁמַרְתֶּם֙ מִצְוֺתַ֔י וַעֲשִׂיתֶ֖ם אֹתָ֑ם אֲנִ֖י יְהֹוָֽה׃}
{וְתִטְּרוּן פִּקּוֹדַי וְתַעְבְּדוּן יָתְהוֹן אֲנָא יְיָ׃}
{And ye shall keep My commandments, and do them: I am the \lord.}{\arabic{verse}}
\rashi{\rashiDH{ושמרתם.} זו המשנה (ת״כ פרק ט, ג)׃\quad \rashiDH{ועשיתם.} זה המעשה׃}
\threeverse{\arabic{verse}}%Leviticus22:32
{וְלֹ֤א תְחַלְּלוּ֙ אֶת\maqqaf שֵׁ֣ם קׇדְשִׁ֔י וְנִ֨קְדַּשְׁתִּ֔י בְּת֖וֹךְ בְּנֵ֣י יִשְׂרָאֵ֑ל אֲנִ֥י יְהֹוָ֖ה מְקַדִּשְׁכֶֽם׃}
{וְלָא תַחֲלוּן יָת שְׁמָא דְּקוּדְשִׁי וְאֶתְקַדַּשׁ בְּגוֹ בְּנֵי יִשְׂרָאֵל אֲנָא יְיָ מְקַדִּשְׁכוֹן׃}
{And ye shall not profane My holy name; but I will be hallowed among the children of Israel: I am the \lord\space who hallow you,}{\arabic{verse}}
\rashi{\rashiDH{ולא תחללו.} לעבור על דְּבָרַי מְזִידִין. ממשמע שנאמר ולא תחללו, מה תלמוד לומר ונקדשתי, מסור עצמך וקדש שמי, יכול ביחיד, תלמוד לומר בתוך בני ישראל, וכשהוא מוסר עצמו ימסור עצמו על מנת למות, שכל המוסר עצמו על מנת הנס, אין עושין לו נס, שכן מצינו בחנניה מישאל ועזריה שלא מסרו עצמן על מנת הנס שנאמר וְהֵן לָא יְדִיעַ לֶהֱוֵא לָךְ מַלְכָּא וגו׳ (דניאל ג, יח), מציל ולא מציל ידיע להוי לך וגו׳׃}
\threeverse{\arabic{verse}}%Leviticus22:33
{הַמּוֹצִ֤יא אֶתְכֶם֙ מֵאֶ֣רֶץ מִצְרַ֔יִם לִהְי֥וֹת לָכֶ֖ם לֵאלֹהִ֑ים אֲנִ֖י יְהֹוָֽה׃ \petucha }
{דְּאַפֵּיק יָתְכוֹן מֵאַרְעָא דְּמִצְרַיִם לְמִהְוֵי לְכוֹן לֶאֱלָהּ אֲנָא יְיָ׃}
{that brought you out of the land of Egypt, to be your God: I am the \lord.}{\arabic{verse}}
\rashi{\rashiDH{המוציא אתכם.} על מנת כן׃\quad \rashiDH{אני ה׳.} נאמן לשלם שכר׃}
\newperek
\aliyacounter{רביעי}
\threeverse{\aliya{רביעי}}%Leviticus23:1
{וַיְדַבֵּ֥ר יְהֹוָ֖ה אֶל\maqqaf מֹשֶׁ֥ה לֵּאמֹֽר׃}
{וּמַלֵּיל יְיָ עִם מֹשֶׁה לְמֵימַר׃}
{And the \lord\space spoke unto Moses, saying:}{\Roman{chap}}
\threeverse{\arabic{verse}}%Leviticus23:2
{דַּבֵּ֞ר אֶל\maqqaf בְּנֵ֤י יִשְׂרָאֵל֙ וְאָמַרְתָּ֣ אֲלֵהֶ֔ם מוֹעֲדֵ֣י יְהֹוָ֔ה אֲשֶׁר\maqqaf תִּקְרְא֥וּ אֹתָ֖ם מִקְרָאֵ֣י קֹ֑דֶשׁ אֵ֥לֶּה הֵ֖ם מוֹעֲדָֽי׃}
{מַלֵּיל עִם בְּנֵי יִשְׂרָאֵל וְתֵימַר לְהוֹן מוֹעֲדַיָּא דַּייָ דִּתְעָרְעוּן יָתְהוֹן מְעָרְעֵי קַדִּישׁ אִלֵּין אִנּוּן מוֹעֲדָי׃}
{Speak unto the children of Israel, and say unto them: The appointed seasons of the \lord, which ye shall proclaim to be holy convocations, even these are My appointed seasons.}{\arabic{verse}}
\rashi{\rashiDH{דבר אל בני ישראל וגו׳ מועדי ה׳.} עשה מועדות שיהיו ישראל מלומדין בהם, שמעברים את השנה על גליות שנעקרו ממקומם לעלות לרגל ועדיין לא הגיעו לירושלים׃ 
}
\threeverse{\arabic{verse}}%Leviticus23:3
{שֵׁ֣שֶׁת יָמִים֮ תֵּעָשֶׂ֣ה מְלָאכָה֒ וּבַיּ֣וֹם הַשְּׁבִיעִ֗י שַׁבַּ֤ת שַׁבָּתוֹן֙ מִקְרָא\maqqaf קֹ֔דֶשׁ כׇּל\maqqaf מְלָאכָ֖ה לֹ֣א תַעֲשׂ֑וּ שַׁבָּ֥ת הִוא֙ לַֽיהֹוָ֔ה בְּכֹ֖ל מוֹשְׁבֹֽתֵיכֶֽם׃ \petucha }
{שִׁתָּא יוֹמִין תִּתְעֲבֵיד עֲבִידְתָא וּבְיוֹמָא שְׁבִיעָאָה שַׁבָּא שַׁבָּתָא מְעָרַע קַדִּישׁ כָּל עֲבִידָא לָא תַעְבְּדוּן שַׁבְּתָא הִיא קֳדָם יְיָ בְּכֹל מוֹתְבָנֵיכוֹן׃}
{Six days shall work be done; but on the seventh day is a sabbath of solemn rest, a holy convocation; ye shall do no manner of work; it is a sabbath unto the \lord\space in all your dwellings.}{\arabic{verse}}
\rashi{\rashiDH{ששת ימים.} מה ענין שבת אצל מועדות, ללמדך שכל המחלל את המועדות, מעלין עליו כאילו חלל את השבתות, וכל המקיים את המועדות, מעלין עליו כאלו קיים את השבתות׃}
\threeverse{\arabic{verse}}%Leviticus23:4
{אֵ֚לֶּה מוֹעֲדֵ֣י יְהֹוָ֔ה מִקְרָאֵ֖י קֹ֑דֶשׁ אֲשֶׁר\maqqaf תִּקְרְא֥וּ אֹתָ֖ם בְּמוֹעֲדָֽם׃}
{אִלֵּין מוֹעֲדַיָּא דַּייָ מְעָרְעֵי קַדִּישׁ דִּתְעָרְעוּן יָתְהוֹן בְּזִמְנֵיהוֹן׃}
{These are the appointed seasons of the \lord, even holy convocations, which ye shall proclaim in their appointed season.}{\arabic{verse}}
\rashi{\rashiDH{אלה מועדי ה׳.} למעלה מדבר בְּעִבּוּר שָׁנָה וכאן מדבר בקדוש החדש׃}
\threeverse{\arabic{verse}}%Leviticus23:5
{בַּחֹ֣דֶשׁ הָרִאשׁ֗וֹן בְּאַרְבָּעָ֥ה עָשָׂ֛ר לַחֹ֖דֶשׁ בֵּ֣ין הָעַרְבָּ֑יִם פֶּ֖סַח לַיהֹוָֽה׃}
{בְּיַרְחָא קַדְמָאָה בְּאַרְבְּעַת עַשְׂרָא לְיַרְחָא בֵּין שִׁמְשַׁיָּא פִּסְחָא קֳדָם יְיָ׃}
{In the first month, on the fourteenth day of the month at dusk, is the \lord’S passover.}{\arabic{verse}}
\rashi{\rashiDH{בין הערבים.} משש שעות ולמעלה׃\quad \rashiDH{פסח לה׳.} הקרבת קרבן ששמו פסח׃ 
}
\threeverse{\arabic{verse}}%Leviticus23:6
{וּבַחֲמִשָּׁ֨ה עָשָׂ֥ר יוֹם֙ לַחֹ֣דֶשׁ הַזֶּ֔ה חַ֥ג הַמַּצּ֖וֹת לַיהֹוָ֑ה שִׁבְעַ֥ת יָמִ֖ים מַצּ֥וֹת תֹּאכֵֽלוּ׃}
{וּבַחֲמֵישְׁתְּ עַשְׂרָא יוֹמָא לְיַרְחָא הָדֵין חַגָּא דְּפַטִּירַיָּא קֳדָם יְיָ שִׁבְעָא יוֹמִין פַּטִּירָא תֵּיכְלוּן׃}
{And on the fifteenth day of the same month is the feast of unleavened bread unto the \lord; seven days ye shall eat unleavened bread.}{\arabic{verse}}
\threeverse{\arabic{verse}}%Leviticus23:7
{בַּיּוֹם֙ הָֽרִאשׁ֔וֹן מִקְרָא\maqqaf קֹ֖דֶשׁ יִהְיֶ֣ה לָכֶ֑ם כׇּל\maqqaf מְלֶ֥אכֶת עֲבֹדָ֖ה לֹ֥א תַעֲשֽׂוּ׃}
{בְּיוֹמָא קַדְמָאָה מְעָרַע קַדִּישׁ יְהֵי לְכוֹן כָּל עֲבִידַת פּוּלְחַן לָא תַעְבְּדוּן׃}
{In the first day ye shall have a holy convocation; ye shall do no manner of servile work.}{\arabic{verse}}
\threeverse{\arabic{verse}}%Leviticus23:8
{וְהִקְרַבְתֶּ֥ם אִשֶּׁ֛ה לַיהֹוָ֖ה שִׁבְעַ֣ת יָמִ֑ים בַּיּ֤וֹם הַשְּׁבִיעִי֙ מִקְרָא\maqqaf קֹ֔דֶשׁ כׇּל\maqqaf מְלֶ֥אכֶת עֲבֹדָ֖ה לֹ֥א תַעֲשֽׂוּ׃ \petucha }
{וּתְקָרְבוּן קוּרְבָּנָא קֳדָם יְיָ שִׁבְעָא יוֹמִין בְּיוֹמָא שְׁבִיעָאָה מְעָרַע קַדִּישׁ כָּל עֲבִידַת פּוּלְחַן לָא תַעְבְּדוּן׃}
{And ye shall bring an offering made by fire unto the \lord\space seven days; in the seventh day is a holy convocation; ye shall do no manner of servile work.}{\arabic{verse}}
\rashi{\rashiDH{והקרבתם אשה וגו׳.} הם המוספין האמורים בפרשת פנחס, ולמה נאמרו כאן, לומר לך שאין המוספין מעכבין זה את זה (מנחות מט.)׃\quad \rashiDH{והקרבתם אשה לה׳.} מכל מקום, אם אין פרים, הָבֵא אילים, ואם אין פרים ואילים, הָבֵא כבשים׃\quad \rashiDH{שבעת ימים.} כל מקום שנאמר שבעת, שם דבר הוא שבוע של ימים, שטיינ״א בלע״ז, וכן כל לשון שמונת, ששת, חמשת, שלשת׃\quad \rashiDH{מלאכת עבדה.} אפילו מלאכות החשובות לכם עבודה וצורך שיש חסרון כיס בבטלה שלהן, כגון דבר האבד, כך הבנתי מתורת כהנים (פרשתא יב, ח), דקתני יכול אף חולו של מועד יהא אסור במלאכת עבודה וכו׳׃}
\newseder{19}
\threeverse{\seder{[יט]}}%Leviticus23:9
{וַיְדַבֵּ֥ר יְהֹוָ֖ה אֶל\maqqaf מֹשֶׁ֥ה לֵּאמֹֽר׃}
{וּמַלֵּיל יְיָ עִם מֹשֶׁה לְמֵימַר׃}
{And the \lord\space spoke unto Moses saying:}{\arabic{verse}}
\threeverse{\arabic{verse}}%Leviticus23:10
{דַּבֵּ֞ר אֶל\maqqaf בְּנֵ֤י יִשְׂרָאֵל֙ וְאָמַרְתָּ֣ אֲלֵהֶ֔ם כִּֽי\maqqaf תָבֹ֣אוּ אֶל\maqqaf הָאָ֗רֶץ אֲשֶׁ֤ר אֲנִי֙ נֹתֵ֣ן לָכֶ֔ם וּקְצַרְתֶּ֖ם אֶת\maqqaf קְצִירָ֑הּ וַהֲבֵאתֶ֥ם אֶת\maqqaf עֹ֛מֶר רֵאשִׁ֥ית קְצִירְכֶ֖ם אֶל\maqqaf הַכֹּהֵֽן׃}
{מַלֵּיל עִם בְּנֵי יִשְׂרָאֵל וְתֵימַר לְהוֹן אֲרֵי תֵיעֲלוּן לְאַרְעָא דַּאֲנָא יָהֵיב לְכוֹן וְתִחְצְדוּן יָת חֲצָדַהּ וְתַיְתוֹן יָת עוֹמֶר רֵישׁ חֲצָדְכוֹן לְוָת כָּהֲנָא׃}
{Speak unto the children of Israel, and say unto them: When ye are come into the land which I give unto you, and shall reap the harvest thereof, then ye shall bring the sheaf of the first-fruits of your harvest unto the priest.}{\arabic{verse}}
\rashi{\rashiDH{ראשית קצירכם.} שתהא ראשונה לקציר (מנחות עא.)׃\quad \rashiDH{עומר.} עשירית האיפה, כך היתה שמה, כמו וַיָּמֹדּוּ בָעֹמֶר (שמות טז, יח)׃}
\threeverse{\arabic{verse}}%Leviticus23:11
{וְהֵנִ֧יף אֶת\maqqaf הָעֹ֛מֶר לִפְנֵ֥י יְהֹוָ֖ה לִֽרְצֹנְכֶ֑ם מִֽמׇּחֳרַת֙ הַשַּׁבָּ֔ת יְנִיפֶ֖נּוּ הַכֹּהֵֽן׃}
{וִירִים יָת עוּמְרָא קֳדָם יְיָ לְרַעֲוָא לְכוֹן מִבָּתַר יוֹמָא טָבָא יְרִימִנֵּיהּ כָּהֲנָא׃}
{And he shall wave the sheaf before the \lord, to be accepted for you; on the morrow after the sabbath the priest shall wave it.}{\arabic{verse}}
\rashi{\rashiDH{והניף.} כל תנופה מוליך ומביא מעלה ומוריד, מוליך ומביא לעצור רוחות רעות, מעלה ומוריד לעצור טללים רעים (מנחות סב.)׃\quad \rashiDH{לרצנכם.} אם תקריבו כמשפט זה יהיה לרצון לכם׃\quad \rashiDH{ממחרת השבת.} ממחרת יום טוב הראשון של פסח, שאם אתה אומר שבת בראשית אי אתה יודע איזהו (שם סו.)׃}
\threeverse{\arabic{verse}}%Leviticus23:12
{וַעֲשִׂיתֶ֕ם בְּי֥וֹם הֲנִֽיפְכֶ֖ם אֶת\maqqaf הָעֹ֑מֶר כֶּ֣בֶשׂ תָּמִ֧ים בֶּן\maqqaf שְׁנָת֛וֹ לְעֹלָ֖ה לַיהֹוָֽה׃}
{וְתַעְבְּדוּן בְּיוֹם אֲרָמוּתְכוֹן יָת עוּמְרָא אִמַּר שְׁלִים בַּר שַׁתֵּיהּ לַעֲלָתָא קֳדָם יְיָ׃}
{And in the day when ye wave the sheaf, ye shall offer a he-lamb without blemish of the first year for a burnt-offering unto the \lord.}{\arabic{verse}}
\rashi{\rashiDH{ועשיתם. כבש.} חובה לעומר הוא בא׃}
\threeverse{\arabic{verse}}%Leviticus23:13
{וּמִנְחָתוֹ֩ שְׁנֵ֨י עֶשְׂרֹנִ֜ים סֹ֣לֶת בְּלוּלָ֥ה בַשֶּׁ֛מֶן אִשֶּׁ֥ה לַיהֹוָ֖ה רֵ֣יחַ נִיחֹ֑חַ וְנִסְכֹּ֥ה יַ֖יִן רְבִיעִ֥ת הַהִֽין׃}
{וּמִנְחָתֵיהּ תְּרֵין עֶשְׂרוֹנִין סוּלְתָּא דְּפִילָא בִּמְשַׁח קוּרְבָּנָא קֳדָם יְיָ לְאִתְקַבָּלָא בְרַעֲוָא וְנִסְכֵּיהּ חַמְרָא רַבְעוּת הִינָא׃}
{And the meal-offering thereof shall be two tenth parts of an ephah of fine flour mingled with oil, an offering made by fire unto the \lord\space for a sweet savour; and the drink-offering thereof shall be of wine, the fourth part of a hin.}{\arabic{verse}}
\rashi{\rashiDH{ומנחתו.} מנחת נסכיו׃\quad \rashiDH{שני עשרנים.} כפולה היתה׃\quad \rashiDH{ונסכו יין רביעית ההין.} אף על פי שמנחתו כפולה, אין נסכיו כפולים (מנחות פט׃)׃}
\threeverse{\arabic{verse}}%Leviticus23:14
{וְלֶ֩חֶם֩ וְקָלִ֨י וְכַרְמֶ֜ל לֹ֣א תֹֽאכְל֗וּ עַד\maqqaf עֶ֙צֶם֙ הַיּ֣וֹם הַזֶּ֔ה עַ֚ד הֲבִ֣יאֲכֶ֔ם אֶת\maqqaf קׇרְבַּ֖ן אֱלֹהֵיכֶ֑ם חֻקַּ֤ת עוֹלָם֙ לְדֹרֹ֣תֵיכֶ֔ם בְּכֹ֖ל מֹשְׁבֹֽתֵיכֶֽם׃ \setuma }
{וּלְחֵים וּקְלֵי וּפֵירוּכָן לָא תֵיכְלוּן עַד כְּרַן יוֹמָא הָדֵין עַד אַיְתוֹאֵיכוֹן יָת קוּרְבָּנָא דֶּאֱלָהֲכוֹן קְיָם עָלַם לְדָרֵיכוֹן בְּכֹל מוֹתְבָנֵיכוֹן׃}
{And ye shall eat neither bread, nor parched corn, nor fresh ears, until this selfsame day, until ye have brought the offering of your God; it is a statute for ever throughout your generations in all your dwellings.}{\arabic{verse}}
\rashi{\rashiDH{וקלי.} קמח עשוי מכרמל רך שמייבשין אותו בתנור׃\quad \rashiDH{וכרמל.} הן קליות שקורין גרניילי״ש׃\quad \rashiDH{בכל משבתיכם.} נחלקו בו חכמי ישראל (קידושין לז.), יש שלמדו מכאן, שהחדש נוהג בחוצה לארץ, ויש אומרים, לא בא אלא ללמד שלא נצטוו על החדש אלא לאחר ירושה וישיבה משכבשו וחלקו׃}
\threeverse{\seder{(יט)}}%Leviticus23:15
{וּסְפַרְתֶּ֤ם לָכֶם֙ מִמׇּחֳרַ֣ת הַשַּׁבָּ֔ת מִיּוֹם֙ הֲבִ֣יאֲכֶ֔ם אֶת\maqqaf עֹ֖מֶר הַתְּנוּפָ֑ה שֶׁ֥בַע שַׁבָּת֖וֹת תְּמִימֹ֥ת תִּהְיֶֽינָה׃}
{וְתִמְנוֹן לְכוֹן מִבָּתַר יוֹמָא טָבָא מִיּוֹם אַיְתוֹאֵיכוֹן יָת עוּמְרָא דַּאֲרָמוּתָא שְׁבַע שָׁבוּעָן שַׁלְמָן יִהְוְיָן׃}
{And ye shall count unto you from the morrow after the day of rest, from the day that ye brought the sheaf of the waving; seven weeks shall there be complete;}{\arabic{verse}}
\rashi{\rashiDH{ממחרת השבת.} ממחרת יום טוב (מנחות סה׃)׃\quad \rashiDH{תמימות תהיינה.} מלמד שמתחיל ומונה מבערב, שאם לא כן אינן תמימות (שם סו.)׃}
\threeverse{\arabic{verse}}%Leviticus23:16
{עַ֣ד מִֽמׇּחֳרַ֤ת הַשַּׁבָּת֙ הַשְּׁבִיעִ֔ת תִּסְפְּר֖וּ חֲמִשִּׁ֣ים י֑וֹם וְהִקְרַבְתֶּ֛ם מִנְחָ֥ה חֲדָשָׁ֖ה לַיהֹוָֽה׃}
{עַד מִבָּתַר שְׁבוּעֲתָא שְׁבִיעֵיתָא תִּמְנוֹן חַמְשִׁין יוֹמִין וּתְקָרְבוּן מִנְחָתָא חֲדַתָּא קֳדָם יְיָ׃}
{even unto the morrow after the seventh week shall ye number fifty days; and ye shall present a new meal-offering unto the \lord.}{\arabic{verse}}
\rashi{\rashiDH{השבת השביעת.} כתרגומו שְׁבוּעֲתָא שְׁבִיעֵתָא׃\quad \rashiDH{עד ממחרת השבת השביעת תספרו.} ולא עד בכלל, והן ארבעים ותשעה יום׃\quad \rashiDH{חמשים יום והקרבתם מנחה חדשה לה׳.} ביום החמשים תקריבוה. ואומר אני זהו מדרשו, אבל פשוטו עד ממחרת השבת השביעית שהוא יום חמשים תספרו, ומקרא מסורס הוא׃\quad \rashiDH{מנחה חדשה.} היא המנחה הראשונה שהובאה מן החדש, ואם תאמר הרי קרבה מנחת העומר, אינה כשאר כל המנחות שהיא באה מן השעורים׃}
\threeverse{\arabic{verse}}%Leviticus23:17
{מִמּוֹשְׁבֹ֨תֵיכֶ֜ם תָּבִ֣יאּוּ \legarmeh  לֶ֣חֶם תְּנוּפָ֗ה שְׁ֚תַּיִם שְׁנֵ֣י עֶשְׂרֹנִ֔ים סֹ֣לֶת תִּהְיֶ֔ינָה חָמֵ֖ץ תֵּאָפֶ֑ינָה בִּכּוּרִ֖ים לַֽיהֹוָֽה׃}
{מִמּוֹתְבָנֵיכוֹן תַּיְתוֹן לְחֵים אֲרָמוּתָא תַּרְתֵּין גְּרִיצָן תְּרֵין עֶשְׂרוֹנִין סוּלְתָּא יִהְוְיָן חֲמִיעַ יִתְאַפְיָן בִּכּוּרִין קֳדָם יְיָ׃}
{Ye shall bring out of your dwellings two wave-loaves of two tenth parts of an ephah; they shall be of fine flour, they shall be baked with leaven, for first-fruits unto the \lord.}{\arabic{verse}}
\rashi{\rashiDH{ממושבותיכם.} ולא מחוצה לארץ (מנחות פג׃)׃\quad \rashiDH{לחם תנופה.} לחם תרומה המורם לשם גבוה, וזו היא המנחה החדשה האמורה למעלה׃\quad \rashiDH{בכורים.} ראשונה לכל המנחות, אף למנחת קנאות הבאה מן השעורים לא תקרב מן החדש קודם לשתי הלחם (שם פד׃)׃}
\threeverse{\arabic{verse}}%Leviticus23:18
{וְהִקְרַבְתֶּ֣ם עַל\maqqaf הַלֶּ֗חֶם שִׁבְעַ֨ת כְּבָשִׂ֤ים תְּמִימִם֙ בְּנֵ֣י שָׁנָ֔ה וּפַ֧ר בֶּן\maqqaf בָּקָ֛ר אֶחָ֖ד וְאֵילִ֣ם שְׁנָ֑יִם יִהְי֤וּ עֹלָה֙ לַֽיהֹוָ֔ה וּמִנְחָתָם֙ וְנִסְכֵּיהֶ֔ם אִשֵּׁ֥ה רֵֽיחַ\maqqaf נִיחֹ֖חַ לַיהֹוָֽה׃}
{וּתְקָרְבוּן עַל לַחְמָא שִׁבְעָא אִמְּרִין שַׁלְמִין בְּנֵי שְׁנָא וְתוֹר בַּר תּוֹרֵי חַד וְדִכְרִין תְּרֵין יְהוֹן עֲלָתָא קֳדָם יְיָ וּמִנְחָתְהוֹן וְנִסְכֵּיהוֹן קוּרְבַּן דְּמִתְקַבַּל בְּרַעֲוָא קֳדָם יְיָ׃}
{And ye shall present with the bread seven lambs without blemish of the first year, and one young bullock, and two rams; they shall be a burnt-offering unto the \lord, with their meal-offering, and their drink-offerings, even an offering made by fire, of a sweet savour unto the \lord.}{\arabic{verse}}
\rashi{\rashiDH{על הלחם.} בגלל הלחם, חובה ללחם׃\quad \rashiDH{ומנחתם ונסכיהם.} כמשפט מנחה ונסכים המפורשים בכל בהמה בפרשת נסכים, ג׳ עשרונים לפר, וב׳ עשרונים לאיל, ועשרון לכבש, זו היא המנחה. והנסכים, חצי ההין לפר, ושלישית ההין לאיל, ורביעית ההין לכבש (במדבר טו, ד־ז ט־י)׃}
\threeverse{\arabic{verse}}%Leviticus23:19
{וַעֲשִׂיתֶ֛ם שְׂעִיר\maqqaf עִזִּ֥ים אֶחָ֖ד לְחַטָּ֑את וּשְׁנֵ֧י כְבָשִׂ֛ים בְּנֵ֥י שָׁנָ֖ה לְזֶ֥בַח שְׁלָמִֽים׃}
{וְתַעְבְּדוּן צְפִיר בַּר עִזִּין חַד לְחַטָּתָא וּתְרֵין אִמְּרִין בְּנֵי שְׁנָא לְנִכְסַת קוּדְשַׁיָּא׃}
{And ye shall offer one he-goat for a sin-offering, and two he-lambs of the first year for a sacrifice of peace-offerings.}{\arabic{verse}}
\rashi{\rashiDH{ועשיתם שעיר עזים.} יכול ז׳ כבשים והשעיר האמורים כאן הם ז׳ הכבשים והשעיר האמורים בחומש הפקודים, כשאתה מגיע אצל פרים ואילים אינן הם, אמור מעתה אלו לעצמן ואלו לעצמן, אלו קרבו בגלל הלחם ואלו למוספין (ת״כ פרק יג, ו  מנחות מה׃)׃}
\threeverse{\arabic{verse}}%Leviticus23:20
{וְהֵנִ֣יף הַכֹּהֵ֣ן \pasek  אֹתָ֡ם עַל֩ לֶ֨חֶם הַבִּכֻּרִ֤ים תְּנוּפָה֙ לִפְנֵ֣י יְהֹוָ֔ה עַל\maqqaf שְׁנֵ֖י כְּבָשִׂ֑ים קֹ֛דֶשׁ יִהְי֥וּ לַיהֹוָ֖ה לַכֹּהֵֽן׃}
{וִירִים כָּהֲנָא יָתְהוֹן עַל לַחְמָא דְּבִכּוּרַיָּא אֲרָמָא קֳדָם יְיָ עַל תְּרֵין אִמְּרִין קוּדְשָׁא יְהוֹן קֳדָם יְיָ לְכָהֲנָא׃}
{And the priest shall wave them with the bread of the first-fruits for a wave-offering before the \lord, with the two lambs; they shall be holy to the \lord\space for the priest.}{\arabic{verse}}
\rashi{\rashiDH{והניף הכהן אותם תנופה.} מלמד שטעונין תנופה מחיים (מנחות סב.), יכול כולם, תלמוד לומר על שני כבשים׃\quad \rashiDH{קדש יהיו.} לפי ששלמי יחיד קדשים קלים, הוזקק לומר בשלמי צבור שהם קדשי קדשים׃}
\threeverse{\arabic{verse}}%Leviticus23:21
{וּקְרָאתֶ֞ם בְּעֶ֣צֶם \legarmeh  הַיּ֣וֹם הַזֶּ֗ה מִֽקְרָא\maqqaf קֹ֙דֶשׁ֙ יִהְיֶ֣ה לָכֶ֔ם כׇּל\maqqaf מְלֶ֥אכֶת עֲבֹדָ֖ה לֹ֣א תַעֲשׂ֑וּ חֻקַּ֥ת עוֹלָ֛ם בְּכׇל\maqqaf מוֹשְׁבֹ֥תֵיכֶ֖ם לְדֹרֹֽתֵיכֶֽם׃}
{וּתְעָרְעוּן בִּכְרַן יוֹמָא הָדֵין מְעָרַע קַדִּישׁ יְהֵי לְכוֹן כָּל עֲבִידַת פּוּלְחַן לָא תַעְבְּדוּן קְיָם עָלַם בְּכָל מוֹתְבָנֵיכוֹן לְדָרֵיכוֹן׃}
{And ye shall make proclamation on the selfsame day; there shall be a holy convocation unto you; ye shall do no manner of servile work; it is a statute for ever in all your dwellings throughout your generations.}{\arabic{verse}}
\threeverse{\arabic{verse}}%Leviticus23:22
{וּֽבְקֻצְרְכֶ֞ם אֶת\maqqaf קְצִ֣יר אַרְצְכֶ֗ם לֹֽא\maqqaf תְכַלֶּ֞ה פְּאַ֤ת שָֽׂדְךָ֙ בְּקֻצְרֶ֔ךָ וְלֶ֥קֶט קְצִירְךָ֖ לֹ֣א תְלַקֵּ֑ט לֶֽעָנִ֤י וְלַגֵּר֙ תַּעֲזֹ֣ב אֹתָ֔ם אֲנִ֖י יְהֹוָ֥ה אֱלֹהֵיכֶֽם׃ \petucha }
{וּבְמִחְצַדְכוֹן יָת חֲצָדָא דַּאֲרַעְכוֹן לָא תְשֵׁיצֵי פָּתָא דְּחַקְלָךְ בִּחְצָדָךְ וּלְקָטָא דִּחְצָדָךְ לָא תְלַקֵּיט לְעַנְיֵי וּלְגִיּוֹרֵי תִּשְׁבּוֹק יָתְהוֹן אֲנָא יְיָ אֱלָהֲכוֹן׃}
{And when ye reap the harvest of your land, thou shalt not wholly reap the corner of thy field, neither shalt thou gather the gleaning of thy harvest; thou shalt leave them for the poor, and for the stranger: I am the \lord\space your God.}{\arabic{verse}}
\rashi{\rashiDH{ובקצרכם.} חזר ושנה, לעבור עליהם בשני לאוין. אמר רבי אבדימי ברבי יוסף, מה ראה הכתוב ליתנם באמצע הרגלים, פסח ועצרת מכאן, וראש השנה ויום הכפורים וחג מכאן, ללמדך שכל הנותן לקט שכחה ופאה לעני כראוי, מעלין עליו כאילו בנה בית המקדש והקריב קרבנותיו בתוכו׃\quad \rashiDH{תעזב.} הנח לפניהם, והם ילקטו ואין לך לסייע לאחד מהם׃\quad \rashiDH{אני ה׳ אלהיכם.} נאמן לשלם שכר׃ 
}
\aliyacounter{חמישי}
\threeverse{\aliya{חמישי}}%Leviticus23:23
{וַיְדַבֵּ֥ר יְהֹוָ֖ה אֶל\maqqaf מֹשֶׁ֥ה לֵּאמֹֽר׃}
{וּמַלֵּיל יְיָ עִם מֹשֶׁה לְמֵימַר׃}
{And the \lord\space spoke unto Moses, saying:}{\arabic{verse}}
\threeverse{\arabic{verse}}%Leviticus23:24
{דַּבֵּ֛ר אֶל\maqqaf בְּנֵ֥י יִשְׂרָאֵ֖ל לֵאמֹ֑ר בַּחֹ֨דֶשׁ הַשְּׁבִיעִ֜י בְּאֶחָ֣ד לַחֹ֗דֶשׁ יִהְיֶ֤ה לָכֶם֙ שַׁבָּת֔וֹן זִכְר֥וֹן תְּרוּעָ֖ה מִקְרָא\maqqaf קֹֽדֶשׁ׃}
{מַלֵּיל עִם בְּנֵי יִשְׂרָאֵל לְמֵימַר בְּיַרְחָא שְׁבִיעָאָה בְּחַד לְיַרְחָא יְהֵי לְכוֹן נְיָחָא דּוּכְרַן יַבָּבָא מְעָרַע קַדִּישׁ׃}
{Speak unto the children of Israel, saying: In the seventh month, in the first day of the month, shall be a solemn rest unto you, a memorial proclaimed with the blast of horns, a holy convocation.}{\arabic{verse}}
\rashi{\rashiDH{זכרון תרועה.} זכרון פסוקי זכרונות ופסוקי שופרות (ר״ה לב.), לזכור לכם עקידת יצחק שקרב תחתיו איל׃}
\threeverse{\arabic{verse}}%Leviticus23:25
{כׇּל\maqqaf מְלֶ֥אכֶת עֲבֹדָ֖ה לֹ֣א תַעֲשׂ֑וּ וְהִקְרַבְתֶּ֥ם אִשֶּׁ֖ה לַיהֹוָֽה׃ \setuma }
{כָּל עֲבִידַת פּוּלְחַן לָא תַעְבְּדוּן וּתְקָרְבוּן קוּרְבָּנָא קֳדָם יְיָ׃}
{Ye shall do no manner of servile work; and ye shall bring an offering made by fire unto the \lord.}{\arabic{verse}}
\rashi{\rashiDH{והקרבתם אשה.} המוספים האמורים בחומש הפקודים׃ 
}
\threeverse{\arabic{verse}}%Leviticus23:26
{וַיְדַבֵּ֥ר יְהֹוָ֖ה אֶל\maqqaf מֹשֶׁ֥ה לֵּאמֹֽר׃}
{וּמַלֵּיל יְיָ עִם מֹשֶׁה לְמֵימַר׃}
{And the \lord\space spoke unto Moses, saying:}{\arabic{verse}}
\threeverse{\arabic{verse}}%Leviticus23:27
{אַ֡ךְ בֶּעָשׂ֣וֹר לַחֹ֩דֶשׁ֩ הַשְּׁבִיעִ֨י הַזֶּ֜ה י֧וֹם הַכִּפֻּרִ֣ים ה֗וּא מִֽקְרָא\maqqaf קֹ֙דֶשׁ֙ יִהְיֶ֣ה לָכֶ֔ם וְעִנִּיתֶ֖ם אֶת\maqqaf נַפְשֹׁתֵיכֶ֑ם וְהִקְרַבְתֶּ֥ם אִשֶּׁ֖ה לַיהֹוָֽה׃}
{בְּרַם בְּעַשְׂרָא לְיַרְחָא שְׁבִיעָאָה הָדֵין יוֹמָא דְּכִפּוּרַיָּא הוּא מְעָרַע קַדִּישׁ יְהֵי לְכוֹן וּתְעַנּוֹן יָת נַפְשָׁתְכוֹן וּתְקָרְבוּן קוּרְבָּנָא קֳדָם יְיָ׃}
{Howbeit on the tenth day of this seventh month is the day of atonement; there shall be a holy convocation unto you, and ye shall afflict your souls; and ye shall bring an offering made by fire unto the \lord.}{\arabic{verse}}
\rashi{\rashiDH{אך.} כל אכין ורקין שבתורה מיעוטין, מכפר הוא לשבים ואינו מכפר לשאינם שבים (שבועות יג.)׃}
\threeverse{\arabic{verse}}%Leviticus23:28
{וְכׇל\maqqaf מְלָאכָה֙ לֹ֣א תַעֲשׂ֔וּ בְּעֶ֖צֶם הַיּ֣וֹם הַזֶּ֑ה כִּ֣י י֤וֹם כִּפֻּרִים֙ ה֔וּא לְכַפֵּ֣ר עֲלֵיכֶ֔ם לִפְנֵ֖י יְהֹוָ֥ה אֱלֹהֵיכֶֽם׃}
{וְכָל עֲבִידָא לָא תַעְבְּדוּן בִּכְרַן יוֹמָא הָדֵין אֲרֵי יוֹמָא דְּכִפּוּרַיָּא הוּא לְכַפָּרָא עֲלֵיכוֹן קֳדָם יְיָ אֱלָהֲכוֹן׃}
{And ye shall do no manner of work in that same day; for it is a day of atonement, to make atonement for you before the \lord\space your God.}{\arabic{verse}}
\threeverse{\arabic{verse}}%Leviticus23:29
{כִּ֤י כׇל\maqqaf הַנֶּ֙פֶשׁ֙ אֲשֶׁ֣ר לֹֽא\maqqaf תְעֻנֶּ֔ה בְּעֶ֖צֶם הַיּ֣וֹם הַזֶּ֑ה וְנִכְרְתָ֖ה מֵֽעַמֶּֽיהָ׃}
{אֲרֵי כָל אֱנָשׁ דְּלָא יִתְעַנֵּי בִּכְרַן יוֹמָא הָדֵין וְיִשְׁתֵּיצֵי מֵעַמֵּיהּ׃}
{For whatsoever soul it be that shall not be afflicted in that same day, he shall be cut off from his people.}{\arabic{verse}}
\threeverse{\arabic{verse}}%Leviticus23:30
{וְכׇל\maqqaf הַנֶּ֗פֶשׁ אֲשֶׁ֤ר תַּעֲשֶׂה֙ כׇּל\maqqaf מְלָאכָ֔ה בְּעֶ֖צֶם הַיּ֣וֹם הַזֶּ֑ה וְהַֽאֲבַדְתִּ֛י אֶת\maqqaf הַנֶּ֥פֶשׁ הַהִ֖וא מִקֶּ֥רֶב עַמָּֽהּ׃}
{וְכָל אֱנָשׁ דְּיַעֲבֵיד כָּל עֲבִידָא בִּכְרַן יוֹמָא הָדֵין וְאוֹבֵיד יָת אֲנָשָׁא הַהוּא מִגּוֹ עַמֵּיהּ׃}
{And whatsoever soul it be that doeth any manner of work in that same day, that soul will I destroy from among his people.}{\arabic{verse}}
\rashi{\rashiDH{והאבדתי.} לפי שהוא אומר כרת בכל מקום ואיני יודע מה הוא, כשהוא אומר והאבדתי, למד על הכרת שאינו אלא אבדן׃ 
}
\threeverse{\arabic{verse}}%Leviticus23:31
{כׇּל\maqqaf מְלָאכָ֖ה לֹ֣א תַעֲשׂ֑וּ חֻקַּ֤ת עוֹלָם֙ לְדֹרֹ֣תֵיכֶ֔ם בְּכֹ֖ל מֹשְׁבֹֽתֵיכֶֽם׃}
{כָּל עֲבִידָא לָא תַעְבְּדוּן קְיָם עָלַם לְדָרֵיכוֹן בְּכֹל מוֹתְבָנֵיכוֹן׃}
{Ye shall do no manner of work; it is a statute for ever throughout your generations in all your dwellings.}{\arabic{verse}}
\rashi{\rashiDH{כל מלאכה וגו׳.} לעבור עליו בלאוין הרבה, או להזהיר על מלאכת לילה כמלאכת יום (יומא פא.)׃}
\threeverse{\arabic{verse}}%Leviticus23:32
{שַׁבַּ֨ת שַׁבָּת֥וֹן הוּא֙ לָכֶ֔ם וְעִנִּיתֶ֖ם אֶת\maqqaf נַפְשֹׁתֵיכֶ֑ם בְּתִשְׁעָ֤ה לַחֹ֙דֶשׁ֙ בָּעֶ֔רֶב מֵעֶ֣רֶב עַד\maqqaf עֶ֔רֶב תִּשְׁבְּת֖וּ שַׁבַּתְּכֶֽם׃ \petucha }
{שַׁבָּא שַׁבָּתָא הוּא לְכוֹן וּתְעַנּוֹן יָת נַפְשָׁתְכוֹן בְּתִשְׁעָא לְיַרְחָא בְּרַמְשָׁא מֵרַמְשָׁא עַד רַמְשָׁא תְּנוּחוּן נְיָחֲכוֹן׃}
{It shall be unto you a sabbath of solemn rest, and ye shall afflict your souls; in the ninth day of the month at even, from even unto even, shall ye keep your sabbath.}{\arabic{verse}}
\aliyacounter{ששי}
\threeverse{\aliya{ששי}}%Leviticus23:33
{וַיְדַבֵּ֥ר יְהֹוָ֖ה אֶל\maqqaf מֹשֶׁ֥ה לֵּאמֹֽר׃}
{וּמַלֵּיל יְיָ עִם מֹשֶׁה לְמֵימַר׃}
{And the \lord\space spoke unto Moses, saying:}{\arabic{verse}}
\threeverse{\arabic{verse}}%Leviticus23:34
{דַּבֵּ֛ר אֶל\maqqaf בְּנֵ֥י יִשְׂרָאֵ֖ל לֵאמֹ֑ר בַּחֲמִשָּׁ֨ה עָשָׂ֜ר י֗וֹם לַחֹ֤דֶשׁ הַשְּׁבִיעִי֙ הַזֶּ֔ה חַ֧ג הַסֻּכּ֛וֹת שִׁבְעַ֥ת יָמִ֖ים לַיהֹוָֽה׃}
{מַלֵּיל עִם בְּנֵי יִשְׂרָאֵל לְמֵימַר בַּחֲמֵישְׁתְּ עַשְׂרָא יוֹמָא לְיַרְחָא שְׁבִיעָאָה הָדֵין חַגָּא דִּמְטַלַּיָּא שִׁבְעָא יוֹמִין קֳדָם יְיָ׃}
{Speak unto the children of Israel, saying: On the fifteenth day of this seventh month is the feast of tabernacles for seven days unto the \lord.}{\arabic{verse}}
\threeverse{\arabic{verse}}%Leviticus23:35
{בַּיּ֥וֹם הָרִאשׁ֖וֹן מִקְרָא\maqqaf קֹ֑דֶשׁ כׇּל\maqqaf מְלֶ֥אכֶת עֲבֹדָ֖ה לֹ֥א תַעֲשֽׂוּ׃}
{בְּיוֹמָא קַדְמָאָה מְעָרַע קַדִּישׁ כָּל עֲבִידַת פּוּלְחַן לָא תַעְבְּדוּן׃}
{On the first day shall be a holy convocation; ye shall do no manner of servile work.}{\arabic{verse}}
\rashi{\rashiDH{מקרא קדש.} [ביוה״כ] קדשהו בכסות נקיה ובתפלה, ובשאר ימים טובים, במאכל ובמשתה ובכסות נקיה ובתפלה׃ 
}
\threeverse{\arabic{verse}}%Leviticus23:36
{שִׁבְעַ֣ת יָמִ֔ים תַּקְרִ֥יבוּ אִשֶּׁ֖ה לַיהֹוָ֑ה בַּיּ֣וֹם הַשְּׁמִינִ֡י מִקְרָא\maqqaf קֹ֩דֶשׁ֩ יִהְיֶ֨ה לָכֶ֜ם וְהִקְרַבְתֶּ֨ם אִשֶּׁ֤ה לַֽיהֹוָה֙ עֲצֶ֣רֶת הִ֔וא כׇּל\maqqaf מְלֶ֥אכֶת עֲבֹדָ֖ה לֹ֥א תַעֲשֽׂוּ׃}
{שִׁבְעָא יוֹמִין תְּקָרְבוּן קוּרְבָּנָא קֳדָם יְיָ בְּיוֹמָא תְּמִינָאָה מְעָרַע קַדִּישׁ יְהֵי לְכוֹן וּתְקָרְבוּן קוּרְבָּנָא קֳדָם יְיָ כְּנִישִׁין תְּהוֹן כָּל עֲבִידַת פּוּלְחַן לָא תַעְבְּדוּן׃}
{Seven days ye shall bring an offering made by fire unto the \lord; on the eighth day shall be a holy convocation unto you; and ye shall bring an offering made by fire unto the \lord; it is a day of solemn assembly; ye shall do no manner of servile work.}{\arabic{verse}}
\rashi{\rashiDH{עצרת הוא.} עצרתי אתכם אצלי, כמלך שזימן את בניו לסעודה לכך וכך ימים, כיון שהגיע זמנן להפטר אמר, בני, בבקשה מכם עכבו עמי עוד יום אחד, קשה עלי פרידתכם׃\quad \rashiDH{כל מלאכת עבדה.} אפילו מלאכה שהיא עבודה לכם, שאם לא תעשוה יש חסרון כיס בדבר׃\quad \rashiDH{לא תעשו.} יכול אף חולו של מועד יהא אסור במלאכת עבודה, תלמוד לומר, היא׃ 
}
\threeverse{\arabic{verse}}%Leviticus23:37
{אֵ֚לֶּה מוֹעֲדֵ֣י יְהֹוָ֔ה אֲשֶׁר\maqqaf תִּקְרְא֥וּ אֹתָ֖ם מִקְרָאֵ֣י קֹ֑דֶשׁ לְהַקְרִ֨יב אִשֶּׁ֜ה לַיהֹוָ֗ה עֹלָ֧ה וּמִנְחָ֛ה זֶ֥בַח וּנְסָכִ֖ים דְּבַר\maqqaf י֥וֹם בְּיוֹמֽוֹ׃}
{אִלֵּין מוֹעֲדַיָּא דַּייָ דִּתְעָרְעוּן יָתְהוֹן מְעָרְעֵי קַדִּישׁ לְקָרָבָא קוּרְבָּנָא קֳדָם יְיָ עֲלָתָא וּמִנְחָתָא נִכְסַת קוּדְשִׁין וְנִסּוּכִין פִּתְגָם יוֹם בְּיוֹמֵיהּ׃}
{These are the appointed seasons of the \lord, which ye shall proclaim to be holy convocations, to bring an offering made by fire unto the \lord, a burnt-offering, and a meal-offering, a sacrifice, and drink-offerings, each on its own day;}{\arabic{verse}}
\rashi{\rashiDH{עלה ומנחה.} מנחת נסכים הקריבה עם העולה (מנחות מד׃)׃\quad \rashiDH{דבר יום ביומו.} חוק הקצוב בחומש הפקודים׃\quad \rashiDH{דבר יום ביומו.} הא אם עבר יומו בטל קרבנו׃}
\threeverse{\arabic{verse}}%Leviticus23:38
{מִלְּבַ֖ד שַׁבְּתֹ֣ת יְהֹוָ֑ה וּמִלְּבַ֣ד מַתְּנֽוֹתֵיכֶ֗ם וּמִלְּבַ֤ד כׇּל\maqqaf נִדְרֵיכֶם֙ וּמִלְּבַד֙ כׇּל\maqqaf נִדְבֹ֣תֵיכֶ֔ם אֲשֶׁ֥ר תִּתְּנ֖וּ לַיהֹוָֽה׃}
{בָּר מִשַּׁבַּיָּא דַּייָ וּבָר מִמַּתְּנָתְכוֹן וּבָר מִכָּל נִדְרֵיכוֹן וּבָר מִכָּל נִדְבָתְכוֹן דְּתִתְּנוּן קֳדָם יְיָ׃}
{beside the sabbaths of the \lord, and beside your gifts, and beside all your vows, and beside all your freewill-offerings, which ye give unto the \lord.}{\arabic{verse}}
\threeverse{\arabic{verse}}%Leviticus23:39
{אַ֡ךְ בַּחֲמִשָּׁה֩ עָשָׂ֨ר י֜וֹם לַחֹ֣דֶשׁ הַשְּׁבִיעִ֗י בְּאׇסְפְּכֶם֙ אֶת\maqqaf תְּבוּאַ֣ת הָאָ֔רֶץ תָּחֹ֥גּוּ אֶת\maqqaf חַג\maqqaf יְהֹוָ֖ה שִׁבְעַ֣ת יָמִ֑ים בַּיּ֤וֹם הָֽרִאשׁוֹן֙ שַׁבָּת֔וֹן וּבַיּ֥וֹם הַשְּׁמִינִ֖י שַׁבָּתֽוֹן׃}
{בְּרַם בַּחֲמֵישְׁתְּ עַשְׂרָא יוֹמָא לְיַרְחָא שְׁבִיעָאָה בְּמִכְנַשְׁכוֹן יָת עֲלַלְתָּא דְּאַרְעָא תֵּיחֲגוּן יָת חַגָּא קֳדָם יְיָ שִׁבְעָא יוֹמִין בְּיוֹמָא קַדְמָאָה נְיָחָא וּבְיוֹמָא תְּמִינָאָה נְיָחָא׃}
{Howbeit on the fifteenth day of the seventh month, when ye have gathered in the fruits of the land, ye shall keep the feast of the \lord\space seven days; on the first day shall be a solemn rest, and on the eighth day shall be a solemn rest.}{\arabic{verse}}
\rashi{\rashiDH{אך בחמשה עשר יום תחגו.} קרבן שלמים לחגיגה, יכול תדחה את השבת, תלמוד לומר, אך, הואיל ויש לה תשלומין כל שבעה׃\quad \rashiDH{באספכם את תבואת הארץ.} שיהא חדש שביעי זה בא בזמן אסיפה, מכאן שנצטוו לְעַבֵּר את השנים שאם אין העבור, פעמים שהוא בא באמצע הקיץ או החורף׃\quad \rashiDH{תחגו.} שלמי חגיגה׃\quad \rashiDH{שבעת ימים.} אם לא הביא בזה יביא בזה, יכול יהא מביאן כל שבעה, תלמוד לומר וחגותם אותו, יום אחד במשמע, ולא יותר, ולמה נאמר שבעה, לתשלומין (חגיגה ט.)׃}
\threeverse{\arabic{verse}}%Leviticus23:40
{וּלְקַחְתֶּ֨ם לָכֶ֜ם בַּיּ֣וֹם הָרִאשׁ֗וֹן פְּרִ֨י עֵ֤ץ הָדָר֙ כַּפֹּ֣ת תְּמָרִ֔ים וַעֲנַ֥ף עֵץ\maqqaf עָבֹ֖ת וְעַרְבֵי\maqqaf נָ֑חַל וּשְׂמַחְתֶּ֗ם לִפְנֵ֛י יְהֹוָ֥ה אֱלֹהֵיכֶ֖ם שִׁבְעַ֥ת יָמִֽים׃}
{וְתִסְּבוּן לְכוֹן בְּיוֹמָא קַדְמָאָה פֵּירֵי אִילָנָא אֶתְּרוֹגִין לוֹלַבִּין וַהֲדַסִּין וְעַרְבִין דִּנְחַל וְתִחְדוֹן קֳדָם יְיָ אֱלָהֲכוֹן שִׁבְעָא יוֹמִין׃}
{And ye shall take you on the first day the fruit of goodly trees, branches of palm-trees, and boughs of thick trees, and willows of the brook, and ye shall rejoice before the \lord\space your God seven days.}{\arabic{verse}}
\rashi{\rashiDH{פרי עץ הדר.} עץ, שטעם עצו ופריו שוה (סוכה לה.)׃\quad \rashiDH{הדר.} הַדָּר באילנו משנה לשנה, וזהו אתרוג (שם)׃\quad \rashiDH{כפת תמרים.} חסר וי״ו, למד שאינה אלא אחת (שם לב.)׃\quad \rashiDH{וענף עץ עבת.} שענפיו קלועים כעבותות וכחבלים, וזהו הדס, העשוי כמין קליעה (שם לב׃)׃ 
}
\threeverse{\arabic{verse}}%Leviticus23:41
{וְחַגֹּתֶ֤ם אֹתוֹ֙ חַ֣ג לַֽיהֹוָ֔ה שִׁבְעַ֥ת יָמִ֖ים בַּשָּׁנָ֑ה חֻקַּ֤ת עוֹלָם֙ לְדֹרֹ֣תֵיכֶ֔ם בַּחֹ֥דֶשׁ הַשְּׁבִיעִ֖י תָּחֹ֥גּוּ אֹתֽוֹ׃}
{וְתֵיחֲגוּן יָתֵיהּ חַגָּא קֳדָם יְיָ שִׁבְעָא יוֹמִין בְּשַׁתָּא קְיָם עָלַם לְדָרֵיכוֹן בְּיַרְחָא שְׁבִיעָאָה תֵּיחֲגוּן יָתֵיהּ׃}
{And ye shall keep it a feast unto the \lord\space seven days in the year; it is a statute for ever in your generations; ye shall keep it in the seventh month.}{\arabic{verse}}
\threeverse{\arabic{verse}}%Leviticus23:42
{בַּסֻּכֹּ֥ת תֵּשְׁב֖וּ שִׁבְעַ֣ת יָמִ֑ים כׇּל\maqqaf הָֽאֶזְרָח֙ בְּיִשְׂרָאֵ֔ל יֵשְׁב֖וּ בַּסֻּכֹּֽת׃}
{בִּמְטַלַּיָּא תִּתְּבוּן שִׁבְעָא יוֹמִין כָּל יַצִּיבָא בְּיִשְׂרָאֵל יִתְּבוּן בִּמְטַלַּיָּא׃}
{Ye shall dwell in booths seven days; all that are home-born in Israel shall dwell in booths;}{\arabic{verse}}
\rashi{\rashiDH{האזרח.} זה אזרח׃ \rashiDH{בישראל.} לרבות את הגרים (שם כח׃  ת״כ פרק יז, ט)׃}
\threeverse{\arabic{verse}}%Leviticus23:43
{לְמַ֘עַן֮ יֵדְע֣וּ דֹרֹֽתֵיכֶם֒ כִּ֣י בַסֻּכּ֗וֹת הוֹשַׁ֙בְתִּי֙ אֶת\maqqaf בְּנֵ֣י יִשְׂרָאֵ֔ל בְּהוֹצִיאִ֥י אוֹתָ֖ם מֵאֶ֣רֶץ מִצְרָ֑יִם אֲנִ֖י יְהֹוָ֥ה אֱלֹהֵיכֶֽם׃}
{בְּדִיל דְּיִדְּעוּן דָּרֵיכוֹן אֲרֵי בִּמְטַלַּת עֲנָנִי אוֹתֵיבִית יָת בְּנֵי יִשְׂרָאֵל בְּאַפָּקוּתִי יָתְהוֹן מֵאַרְעָא דְּמִצְרָיִם אֲנָא יְיָ אֱלָהֲכוֹן׃}
{that your generations may know that I made the children of Israel to dwell in booths, when I brought them out of the land of Egypt: I am the \lord\space your God.}{\arabic{verse}}
\rashi{\rashiDH{כי בסכות הושבתי.} ענני כבוד (סוכה יא׃)׃
}
\threeverse{\arabic{verse}}%Leviticus23:44
{וַיְדַבֵּ֣ר מֹשֶׁ֔ה אֶת\maqqaf מֹעֲדֵ֖י יְהֹוָ֑ה אֶל\maqqaf בְּנֵ֖י יִשְׂרָאֵֽל׃ \petucha }
{וּמַלֵּיל מֹשֶׁה יָת סְדַר מוֹעֲדַיָּא דַּייָ וְאַלֵּיפִנּוּן לִבְנֵי יִשְׂרָאֵל׃}
{And Moses declared unto the children of Israel the appointed seasons of the \lord.}{\arabic{verse}}

\newperek
\aliyacounter{שביעי}
\threeverse{\aliya{שביעי}\newline\vspace{-4pt}\newline\seder{יט*}}%Leviticus24:1
{וַיְדַבֵּ֥ר יְהֹוָ֖ה אֶל\maqqaf מֹשֶׁ֥ה לֵּאמֹֽר׃}
{וּמַלֵּיל יְיָ עִם מֹשֶׁה לְמֵימַר׃}
{And the \lord\space spoke unto Moses, saying:}{\Roman{chap}}
\threeverse{\arabic{verse}}%Leviticus24:2
{צַ֞ו אֶת\maqqaf בְּנֵ֣י יִשְׂרָאֵ֗ל וְיִקְח֨וּ אֵלֶ֜יךָ שֶׁ֣מֶן זַ֥יִת זָ֛ךְ כָּתִ֖ית לַמָּא֑וֹר לְהַעֲלֹ֥ת נֵ֖ר תָּמִֽיד׃}
{פַּקֵּיד יָת בְּנֵי יִשְׂרָאֵל וְיִסְּבוּן לָךְ מְשַׁח זֵיתָא דָּכְיָא כָּתִישָׁא לְאַנְהָרָא לְאַדְלָקָא בּוֹצִינַיָּא תְּדִירָא׃}
{‘Command the children of Israel, that they bring unto thee pure olive oil beaten for the light, to cause a lamp to burn continually.}{\arabic{verse}}
\rashi{\rashiDH{צו את בני ישראל.} זו פרשת מצות הנרות, ופרשת ואתה תצוה לא נאמרה אלא על סדר מלאכת המשכן, לפרש צורך המנורה, וכן משמע ואתה סופך לצוות את בני ישראל על כך׃\quad \rashiDH{שמן זית זך.} שלשה שמנים יוצאים מן הזית, הראשון קרוי זך, והן מפורשים במנחות (פו.) ובת״כ (פרשתא יג, א)׃\quad \rashiDH{תמיד.} מלילה ללילה, כמו עולת תמיד (במדבר כח ו), שאינה אלא מיום ליום׃}
\threeverse{\arabic{verse}}%Leviticus24:3
{מִחוּץ֩ לְפָרֹ֨כֶת הָעֵדֻ֜ת בְּאֹ֣הֶל מוֹעֵ֗ד יַעֲרֹךְ֩ אֹת֨וֹ אַהֲרֹ֜ן מֵעֶ֧רֶב עַד\maqqaf בֹּ֛קֶר לִפְנֵ֥י יְהֹוָ֖ה תָּמִ֑יד חֻקַּ֥ת עוֹלָ֖ם לְדֹרֹֽתֵיכֶֽם׃}
{מִבַּרָא לְפָרוּכְתָּא דְּסָהֲדוּתָא בְּמַשְׁכַּן זִמְנָא יַסְדַּר יָתֵיהּ אַהֲרֹן מֵרַמְשָׁא עַד צַפְרָא קֳדָם יְיָ תְּדִירָא קְיָם עָלַם לְדָרֵיכוֹן׃}
{Without the veil of the testimony, in the tent of meeting, shall Aaron order it from evening to morning before the \lord\space continually; it shall be a statute for ever throughout your generations.}{\arabic{verse}}
\rashi{\rashiDH{לפרכת העדת.} שלפני הארון, שהוא קרוי עדות. ורבותינו דרשו (שבת כב.  ת״כ שם ט), על נר מערבי, שהוא עדות לכל באי עולם, שהשכינה שורה בישראל, שנותן בה שמן כמדת חברותיה, וממנה היה מתחיל, ובה היה מסיים׃\quad \rashiDH{יערוך אתו אהרן מערב עד בוקר.} יערוך אותו, עריכה הראויה למדת כל הלילה (ת״כ שם יא), ושיערו חכמים, חצי לוג לכל נר ונר, והן כדאי אף ללילי תקופת טבת, ומדה זו הוקבעה להם (מנחות פט.)׃}
\threeverse{\arabic{verse}}%Leviticus24:4
{עַ֚ל הַמְּנֹרָ֣ה הַטְּהֹרָ֔ה יַעֲרֹ֖ךְ אֶת\maqqaf הַנֵּר֑וֹת לִפְנֵ֥י יְהֹוָ֖ה תָּמִֽיד׃ \petucha }
{עַל מְנָרְתָא דָּכִיתָא יַסְדַּר יָת בּוֹצִינַיָּא קֳדָם יְיָ תְּדִירָא׃}
{He shall order the lamps upon the pure candlestick before the \lord\space continually.}{\arabic{verse}}
\rashi{\rashiDH{המנורה הטהרה.} שהיא זהב טהור. דבר אחר על טהרה של מנורה, שמטהרה ומדשנה תחלה מן האפר׃}
\threeverse{\arabic{verse}}%Leviticus24:5
{וְלָקַחְתָּ֣ סֹ֔לֶת וְאָפִיתָ֣ אֹתָ֔הּ שְׁתֵּ֥ים עֶשְׂרֵ֖ה חַלּ֑וֹת שְׁנֵי֙ עֶשְׂרֹנִ֔ים יִהְיֶ֖ה הַֽחַלָּ֥ה הָאֶחָֽת׃}
{וְתִסַּב סוּלְתָּא וְתֵיפֵי יָתַהּ תַּרְתַּא עֱשְׂרֵי גְּרִיצָן תְּרֵין עֶשְׂרוֹנִין תְּהֵי הָוְיָא גְּרִיצְתָא חֲדָא׃}
{And thou shalt take fine flour, and bake twelve cakes thereof: two tenth parts of an ephah shall be in one cake.}{\arabic{verse}}
\threeverse{\arabic{verse}}%Leviticus24:6
{וְשַׂמְתָּ֥ אוֹתָ֛ם שְׁתַּ֥יִם מַֽעֲרָכ֖וֹת שֵׁ֣שׁ הַֽמַּעֲרָ֑כֶת עַ֛ל הַשֻּׁלְחָ֥ן הַטָּהֹ֖ר לִפְנֵ֥י יְהֹוָֽה׃}
{וּתְשַׁוֵּי יָתְהוֹן תַּרְתֵּין סִדְרִין שֵׁית סִדְרָא עַל פָּתוּרָא דָּכְיָא קֳדָם יְיָ׃}
{And thou shalt set them in two rows, six in a row, upon the pure table before the \lord.}{\arabic{verse}}
\rashi{\rashiDH{שש המערכת.} שש חלות המערכה האחת׃\quad \rashiDH{השלחן הטהר.} של זהב טהור. דבר אחר על טהרו של שלחן, שלא יהיו הסניפין מגביהין את הלחם מעל גבי השלחן (ת״כ פרק יח, ד)׃}
\threeverse{\arabic{verse}}%Leviticus24:7
{וְנָתַתָּ֥ עַל\maqqaf הַֽמַּעֲרֶ֖כֶת לְבֹנָ֣ה זַכָּ֑ה וְהָיְתָ֤ה לַלֶּ֙חֶם֙ לְאַזְכָּרָ֔ה אִשֶּׁ֖ה לַֽיהֹוָֽה׃}
{וְתִתֵּין עַל סִדְרָא לְבוֹנְתָא דָּכִיתָא וּתְהֵי לִלְחֵים לְאַדְכָרָא קוּרְבָּנָא קֳדָם יְיָ׃}
{And thou shalt put pure frankincense with each row, that it may be to the bread for a memorial-part, even an offering made by fire unto the \lord.}{\arabic{verse}}
\rashi{\rashiDH{ונתת על המערכת.} על כל אחת משתי המערכות, היו שני בזיכי לבונה, מלא קומץ לכל אחת׃\quad \rashiDH{והיתה.} הלבונה הזאת׃\quad \rashiDH{ללחם לאזכרה.} שאין מן הלחם לגבוה כלום, אלא הלבונה נקטרת כשמסלקין אותו בכל שבת ושבת, והיא לזכרון ללחם, שעל ידה הוא נזכר למעלה כקומץ, שהוא אזכרה למנחה׃ 
}
\threeverse{\arabic{verse}}%Leviticus24:8
{בְּי֨וֹם הַשַּׁבָּ֜ת בְּי֣וֹם הַשַּׁבָּ֗ת יַֽעַרְכֶ֛נּוּ לִפְנֵ֥י יְהֹוָ֖ה תָּמִ֑יד מֵאֵ֥ת בְּנֵֽי\maqqaf יִשְׂרָאֵ֖ל בְּרִ֥ית עוֹלָֽם׃}
{בְּיוֹמָא דְּשַׁבְּתָא בְּיוֹמָא דְּשַׁבְּתָא יַסְדְּרִנֵּיהּ קֳדָם יְיָ תְּדִירָא מִן בְּנֵי יִשְׂרָאֵל קְיָם עָלַם׃}
{Every sabbath day he shall set it in order before the \lord\space continually; it is from the children of Israel, an everlasting covenant.}{\arabic{verse}}
\threeverse{\arabic{verse}}%Leviticus24:9
{וְהָֽיְתָה֙ לְאַהֲרֹ֣ן וּלְבָנָ֔יו וַאֲכָלֻ֖הוּ בְּמָק֣וֹם קָדֹ֑שׁ כִּ֡י קֹ֩דֶשׁ֩ קׇֽדָשִׁ֨ים ה֥וּא ל֛וֹ מֵאִשֵּׁ֥י יְהֹוָ֖ה חׇק\maqqaf עוֹלָֽם׃ \setuma }
{וּתְהֵי לְאַהֲרֹן וְלִבְנוֹהִי וְיֵיכְלוּנֵּיהּ בַּאֲתַר קַדִּישׁ אֲרֵי קֹדֶשׁ קוּדְשִׁין הוּא לֵיהּ מִקּוּרְבָּנַיָּא דַּייָ קְיָם עָלַם׃}
{And it shall be for Aaron and his sons; and they shall eat it in a holy place; for it is most holy unto him of the offerings of the \lord\space made by fire, a perpetual due.’}{\arabic{verse}}
\rashi{\rashiDH{והיתה.} המנחה הזאת, שכל דבר הבא מן התבואה, בכלל מנחה היא׃\quad \rashiDH{ואכלהו.} מוסב על הלחם, שהוא לשון זכר׃}
\threeverse{\arabic{verse}}%Leviticus24:10
{וַיֵּצֵא֙ בֶּן\maqqaf אִשָּׁ֣ה יִשְׂרְאֵלִ֔ית וְהוּא֙ בֶּן\maqqaf אִ֣ישׁ מִצְרִ֔י בְּת֖וֹךְ בְּנֵ֣י יִשְׂרָאֵ֑ל וַיִּנָּצוּ֙ בַּֽמַּחֲנֶ֔ה בֶּ֚ן הַיִּשְׂרְאֵלִ֔ית וְאִ֖ישׁ הַיִּשְׂרְאֵלִֽי׃}
{וּנְפַק בַּר אִתְּתָא בַת יִשְׂרָאֵל וְהוּא בַּר גְּבַר מִצְרַאי בְּגוֹ בְּנֵי יִשְׂרָאֵל וְאִתְנְצִיאוּ בְּמַשְׁרִיתָא בַּר אִתְּתָא בַת יִשְׂרָאֵל וְגוּבְרָא בַּר יִשְׂרָאֵל׃}
{And the son of an Israelitish woman, whose father was an Egyptian, went out among the children of Israel; and the son of the Israelitish woman and a man of Israel strove together in the camp.}{\arabic{verse}}
\rashi{\rashiDH{ויצא בן אשה ישראלית.} מהיכן יצא, רבי לוי אומר מעולמו יצא, רבי ברכיה אומר מפרשה שלמעלה יצא, לגלג ואמר, ביום השבת יערכנו, דרך המלך לאכול פת חמה בכל יום שמא פת צוננת של תשעה ימים, בתמיה. ומתניתא אמרה (ת״כ פרשתא יד, א), מבית דינו של משה יצא מחוייב, בא ליטע אהלו בתוך מחנה דן, אמרו לו מה טיבך לכאן, אמר להם מבני דן אני, אמרו לו אִישׁ עַל דִּגְלוֹ בְאֹתֹת לְבֵית אֲבֹתָם (במדבר ב, ב) כתיב, נכנס לבית דינו של משה ויצא מחוייב עמד וגדף׃\quad \rashiDH{בן איש מצרי.} הוא המצרי שהרגו משה׃\quad \rashiDH{בתוך בני ישראל.} מלמד שנתגייר׃\quad \rashiDH{וינצו במחנה.} על עסקי המחנה׃\quad \rashiDH{ואיש הישראלי.} זה שכנגדו שמיחה בו מִטַּע אהלו׃}
\threeverse{\arabic{verse}}%Leviticus24:11
{וַ֠יִּקֹּ֠ב בֶּן\maqqaf הָֽאִשָּׁ֨ה הַיִּשְׂרְאֵלִ֤ית אֶת\maqqaf הַשֵּׁם֙ וַיְקַלֵּ֔ל וַיָּבִ֥יאוּ אֹת֖וֹ אֶל\maqqaf מֹשֶׁ֑ה וְשֵׁ֥ם אִמּ֛וֹ שְׁלֹמִ֥ית בַּת\maqqaf דִּבְרִ֖י לְמַטֵּה\maqqaf דָֽן׃}
{וּפָרֵישׁ בַּר אִתְּתָא בַת יִשְׂרָאֵל יָת שְׁמָא וְאַרְגֵּיז וְאֵיתִיאוּ יָתֵיהּ לְוָת מֹשֶׁה וְשׁוֹם אִמֵּיהּ שְׁלוֹמִית בַּת דִּבְרִי לְשִׁבְטָא דְּדָן׃}
{And the son of the Israelitish woman blasphemed the Name, and cursed; and they brought him unto Moses. And his mother’s name was Shelomith, the daughter of Dibri, of the tribe of Dan.}{\arabic{verse}}
\rashi{\rashiDH{ויקב.} כתרגומו, וּפָרֵישׁ שנקב שם המיוחד וגדף (סנהדרין נו.), והוא שם המפורש ששמע מסיני׃\quad \rashiDH{ושם אמו שלומית בת דברי.} שבחן של ישראל שפרסמה הכתוב לזו לומר, שהיא לבדה היתה זונה׃\quad \rashiDH{שלמית.} דַּהֲוַת פטפטה שלם עלך, שלם עלך, שלם עליכון, מפטפטת בדברים, שואלת בשלום הכל׃\quad \rashiDH{בת דברי.} דברנית היתה, מדברת עם כל אדם לפיכך קלקלה׃\quad \rashiDH{למטה דן.} מגיד שהרשע גורם גנאי לו, גנאי לאביו, גנאי לשבטו, כיוצא בו אהליאב בן אחיסמך למטה דן (שמות לא, ו), שבח לו שבח לאביו שבח לשבטו׃}
\threeverse{\arabic{verse}}%Leviticus24:12
{וַיַּנִּיחֻ֖הוּ בַּמִּשְׁמָ֑ר לִפְרֹ֥שׁ לָהֶ֖ם עַל\maqqaf פִּ֥י יְהֹוָֽה׃ \petucha }
{וְאַסְרוּהִי בְּבֵית מַטְּרָא עַד דְּיִתְפָּרַשׁ לְהוֹן עַל גְּזֵירַת מֵימְרָא דַּייָ׃}
{And they put him in ward, that it might be declared unto them at the mouth of the \lord.}{\arabic{verse}}
\rashi{\rashiDH{וינחהו.} לבדו, ולא הניחו מקושש עמו, ששניהם היו בפרק אחד, ויודעים היו שהמקושש במיתה, (סנהדרין עח׃) שנאמר מְחַלְּלֶיהָ מוֹת יוּמָת (שמות שם יד), אבל לא פורש להם באיזו מיתה, לכך נאמר כִּי לֹא פֹרַשׁ מַה יֵּעָשֶׂה לוֹ (במדבר טו, לד), אבל במקלל הוא אומר, לפרוש להם, שלא היו יודעים אם חייב מיתה אם לאו׃}
\threeverse{\arabic{verse}}%Leviticus24:13
{וַיְדַבֵּ֥ר יְהֹוָ֖ה אֶל\maqqaf מֹשֶׁ֥ה לֵּאמֹֽר׃}
{וּמַלֵּיל יְיָ עִם מֹשֶׁה לְמֵימַר׃}
{And the \lord\space spoke unto Moses, saying:}{\arabic{verse}}
\threeverse{\arabic{verse}}%Leviticus24:14
{הוֹצֵ֣א אֶת\maqqaf הַֽמְקַלֵּ֗ל אֶל\maqqaf מִחוּץ֙ לַֽמַּחֲנֶ֔ה וְסָמְכ֧וּ כׇֽל\maqqaf הַשֹּׁמְעִ֛ים אֶת\maqqaf יְדֵיהֶ֖ם עַל\maqqaf רֹאשׁ֑וֹ וְרָגְמ֥וּ אֹת֖וֹ כׇּל\maqqaf הָעֵדָֽה׃}
{אַפֵּיק יָת דְּאַרְגֵּיז לְמִבַּרָא לְמַשְׁרִיתָא וְיִסְמְכוּן כָּל דִּשְׁמַעוּ יָת יְדֵיהוֹן עַל רֵישֵׁיהּ וְיִרְגְּמוּן יָתֵיהּ כָּל כְּנִשְׁתָּא׃}
{‘Bring forth him that hath cursed without the camp; and let all that heard him lay their hands upon his head, and let all the congregation stone him.}{\arabic{verse}}
\rashi{\rashiDH{השמעים.} אלו העדים׃\quad \rashiDH{כל.} להביא את הדיינים׃ 
\quad \rashiDH{את ידיהם.} אומרים לו דמך בראשך, ואין אנו נענשים במיתתך, שאתה גרמת לך׃\quad \rashiDH{כל העדה.} במעמד כל העדה (ת״כ פרק יט, ג), מכאן, ששלוחו של אדם כמותו׃}
\threeverse{\arabic{verse}}%Leviticus24:15
{וְאֶל\maqqaf בְּנֵ֥י יִשְׂרָאֵ֖ל תְּדַבֵּ֣ר לֵאמֹ֑ר אִ֥ישׁ אִ֛ישׁ כִּֽי\maqqaf יְקַלֵּ֥ל אֱלֹהָ֖יו וְנָשָׂ֥א חֶטְאֽוֹ׃}
{וְעִם בְּנֵי יִשְׂרָאֵל תְּמַלֵּיל לְמֵימַר גְּבַר גְּבַר אֲרֵי יַרְגֵּיז קֳדָם אֱלָהֵיהּ וִיקַבֵּיל חוֹבֵיהּ׃}
{And thou shalt speak unto the children of Israel, saying: Whosoever curseth his God shall bear his sin.}{\arabic{verse}}
\rashi{\rashiDH{ונשא חטאו.} בכרת, כשאין התראה׃}
\threeverse{\arabic{verse}}%Leviticus24:16
{וְנֹקֵ֤ב שֵׁם\maqqaf יְהֹוָה֙ מ֣וֹת יוּמָ֔ת רָג֥וֹם יִרְגְּמוּ\maqqaf ב֖וֹ כׇּל\maqqaf הָעֵדָ֑ה כַּגֵּר֙ כָּֽאֶזְרָ֔ח בְּנׇקְבוֹ\maqqaf שֵׁ֖ם יוּמָֽת׃}
{וְדִיפָרֵישׁ שְׁמָא דַּייָ אִתְקְטָלָא יִתְקְטִיל מִרְגָּם יִרְגְּמוּן בֵּיהּ כָּל כְּנִשְׁתָּא גִּיּוֹרָא כְּיַצִּיבָא בְּפָרָשׁוּתֵיהּ שְׁמָא יִתְקְטִיל׃}
{And he that blasphemeth the name of the \lord, he shall surely be put to death; all the congregation shall certainly stone him; as well the stranger, as the home-born, when he blasphemeth the Name, shall be put to death.}{\arabic{verse}}
\rashi{\rashiDH{ונקב שם.} אינו חייב עד שיפרש את השם, ולא המקלל בכינוי׃\quad \rashiDH{ונקב.} לשון קללה, כמו מָה אֶקֹּב (במדבר כג, ח. סנהדרין נו)׃}
\threeverse{\arabic{verse}}%Leviticus24:17
{וְאִ֕ישׁ כִּ֥י יַכֶּ֖ה כׇּל\maqqaf נֶ֣פֶשׁ אָדָ֑ם מ֖וֹת יוּמָֽת׃}
{וּגְבַר אֲרֵי יִקְטוֹל כָּל נַפְשָׁא דַּאֲנָשָׁא אִתְקְטָלָא יִתְקְטִיל׃}
{And he that smiteth any man mortally shall surely be put to death.}{\arabic{verse}}
\rashi{\rashiDH{ואיש כי יכה.} לפי שנאמר מכה איש וגו׳ (שמות כא, יב), אין לי אלא שהרג את האיש, אשה וקטן מנין, תלמוד לומר כל נפש אדם׃ 
}
\threeverse{\arabic{verse}}%Leviticus24:18
{וּמַכֵּ֥ה נֶֽפֶשׁ\maqqaf בְּהֵמָ֖ה יְשַׁלְּמֶ֑נָּה נֶ֖פֶשׁ תַּ֥חַת נָֽפֶשׁ׃}
{וּדְיִקְטוֹל נְפַשׁ בְּעִירָא יְשַׁלְּמִנַּהּ נַפְשָׁא חֲלָף נַפְשָׁא׃}
{And he that smiteth a beast mortally shall make it good: life for life.}{\arabic{verse}}
\threeverse{\arabic{verse}}%Leviticus24:19
{וְאִ֕ישׁ כִּֽי\maqqaf יִתֵּ֥ן מ֖וּם בַּעֲמִית֑וֹ כַּאֲשֶׁ֣ר עָשָׂ֔ה כֵּ֖ן יֵעָ֥שֶׂה לּֽוֹ׃}
{וּגְבַר אֲרֵי יִתֵּין מוּמָא בְּחַבְרֵיהּ כְּמָא דַּעֲבַד כֵּן יִתְעֲבֵיד לֵיהּ׃}
{And if a man maim his neighbour; as he hath done, so shall it be done to him:}{\arabic{verse}}
\threeverse{\arabic{verse}}%Leviticus24:20
{שֶׁ֚בֶר תַּ֣חַת שֶׁ֔בֶר עַ֚יִן תַּ֣חַת עַ֔יִן שֵׁ֖ן תַּ֣חַת שֵׁ֑ן כַּאֲשֶׁ֨ר יִתֵּ֥ן מוּם֙ בָּֽאָדָ֔ם כֵּ֖ן יִנָּ֥תֶן בּֽוֹ׃}
{תְּבָרָא חֲלָף תְּבָרָא עֵינָא חֲלָף עֵינָא שִׁנָּא חֲלָף שִׁנָּא כְּמָא דִּיהַב מוּמָא בַּאֲנָשָׁא כֵּן יִתְיְהֵיב בֵּיהּ׃}
{breach for breach, eye for eye, tooth for tooth; as he hath maimed a man, so shall it be rendered unto him.}{\arabic{verse}}
\rashi{\rashiDH{כן ינתן בו.} פירשו רבותינו (ב״ק פד.) שאינו נתינת מום ממש, אלא תשלומי ממון שמין אותו כעבד, לכך כתוב בו לשון נתינה, דבר הנתון מיד ליד׃}
\threeverse{\aliya{מפטיר}}%Leviticus24:21
{וּמַכֵּ֥ה בְהֵמָ֖ה יְשַׁלְּמֶ֑נָּה וּמַכֵּ֥ה אָדָ֖ם יוּמָֽת׃}
{וּדְיִקְטוֹל בְּעִירָא יְשַׁלְּמִנַּהּ וּדְיִקְטוֹל אֲנָשָׁא יִתְקְטִיל׃}
{And he that killeth a beast shall make it good; and he that killeth a man shall be put to death.}{\arabic{verse}}
\rashi{\rashiDH{ומכה בהמה ישלמנה.} למעלה דִּבֵּר בהורג בהמה, וכאן דבר בעושה בה חבורה׃\quad \rashiDH{ומכה אדם יומת.} אפילו לא הרגו, אלא עשה בו חבורה, שלא נאמר כאן נפש, ובמכה אביו ואמו דבר הכתוב, ובא להקישו למכה בהמה, מה מכה בהמה מחיים, אף מכה אביו ואמו מחיים, פרט למכה לאחר מיתה, לפי שמצינו שהמקללו לאחר מיתה חייב, הוצרך לומר במכה שפטור, ומה בבהמה בחבלה, שאם אין חבלה אין תשלומין, אף מכה אביו ואמו אינו חייב עד שיעשה בהם חבורה׃}
\threeverse{\arabic{verse}}%Leviticus24:22
{מִשְׁפַּ֤ט אֶחָד֙ יִהְיֶ֣ה לָכֶ֔ם כַּגֵּ֥ר כָּאֶזְרָ֖ח יִהְיֶ֑ה כִּ֛י אֲנִ֥י יְהֹוָ֖ה אֱלֹהֵיכֶֽם׃}
{דִּינָא חַד יְהֵי לְכוֹן גִּיּוֹרָא כְּיַצִּיבָא יְהֵי אֲרֵי אֲנָא יְיָ אֱלָהֲכוֹן׃}
{Ye shall have one manner of law, as well for the stranger, as for the home-born; for I am the \lord\space your God.’}{\arabic{verse}}
\rashi{\rashiDH{אני ה׳ אלהיכם.} אלהי כולכם, כשם שאני מיחד שמי עליכם, כך אני מיחד שמי על הגרים׃}
\threeverse{\aliya{\Hebrewnumeral{124}}}%Leviticus24:23
{וַיְדַבֵּ֣ר מֹשֶׁה֮ אֶל\maqqaf בְּנֵ֣י יִשְׂרָאֵל֒ וַיּוֹצִ֣יאוּ אֶת\maqqaf הַֽמְקַלֵּ֗ל אֶל\maqqaf מִחוּץ֙ לַֽמַּחֲנֶ֔ה וַיִּרְגְּמ֥וּ אֹת֖וֹ אָ֑בֶן וּבְנֵֽי\maqqaf יִשְׂרָאֵ֣ל עָשׂ֔וּ כַּֽאֲשֶׁ֛ר צִוָּ֥ה יְהֹוָ֖ה אֶת\maqqaf מֹשֶֽׁה׃ \petucha }
{וּמַלֵּיל מֹשֶׁה עִם בְּנֵי יִשְׂרָאֵל וְאַפִּיקוּ יָת דְּאַרְגֵּיז לְמִבַּרָא לְמַשְׁרִיתָא וּרְגַמוּ יָתֵיהּ בְּאַבְנָא וּבְנֵי יִשְׂרָאֵל עֲבַדוּ כְּמָא דְּפַקֵּיד יְיָ יָת מֹשֶׁה׃}
{And Moses spoke to the children of Israel, and they brought forth him that had cursed out of the camp, and stoned him with stones. And the children of Israel did as the \lord\space commanded Moses.}{\arabic{verse}}
\rashi{\rashiDH{ובני ישראל עשו.} כל המצוה האמורה בסקילה במקום אחר דחייה, רגימה, ותלייה׃ 
}
\engnote{The Haftarah is Ezekiel 44:15\verserangechar 44:31 on page \pageref{haft_31}. }
\newperek
\aliyacounter{ראשון}
\newparsha{בהר}
\threeverse{\aliya{בהר}}%Leviticus25:1
{וַיְדַבֵּ֤ר יְהֹוָה֙ אֶל\maqqaf מֹשֶׁ֔ה בְּהַ֥ר סִינַ֖י לֵאמֹֽר׃}
{וּמַלֵּיל יְיָ עִם מֹשֶׁה בְּטוּרָא דְּסִינַי לְמֵימַר׃}
{And the \lord\space spoke unto Moses in mount Sinai, saying:}{\Roman{chap}}
\rashi{\rashiDH{בהר סיני.} מה ענין שמיטה אצל הר סיני, והלא כל המצות נאמרו מסיני, אלא מה שמיטה נאמרו כללותיה (ופרטותיה) ודקדוקיה מסיני, אף כולן נאמרו כללותיהן ודקדוקיהן מסיני, כך שנויה בת״כ (פרשתא א, א). ונראה לי שכך פירושה, לפי שלא מצינו שמיטת קרקעות שנשנית בערבות מואב במשנה תורה, למדנו שכללותיה ופרטותיה כולן נאמרו מסיני, ובא הכתוב ולמד כאן על כל דבור שנדבר למשה שמסיני היו כולם כללותיהן ודקדוקיהן, וחזרו ונשנו בערבות מואב׃}
\threeverse{\arabic{verse}}%Leviticus25:2
{דַּבֵּ֞ר אֶל\maqqaf בְּנֵ֤י יִשְׂרָאֵל֙ וְאָמַרְתָּ֣ אֲלֵהֶ֔ם כִּ֤י תָבֹ֙אוּ֙ אֶל\maqqaf הָאָ֔רֶץ אֲשֶׁ֥ר אֲנִ֖י נֹתֵ֣ן לָכֶ֑ם וְשָׁבְתָ֣ה הָאָ֔רֶץ שַׁבָּ֖ת לַיהֹוָֽה׃}
{מַלֵּיל עִם בְּנֵי יִשְׂרָאֵל וְתֵימַר לְהוֹן אֲרֵי תֵיעֲלוּן לְאַרְעָא דַּאֲנָא יָהֵיב לְכוֹן וְתַשְׁמֵיט אַרְעָא שְׁמִטְּתָא קֳדָם יְיָ׃}
{Speak unto the children of Israel, and say unto them: When ye come into the land which I give you, then shall the land keep a sabbath unto the \lord.}{\arabic{verse}}
\rashi{\rashiDH{שבת לה׳.} לשם ה׳ כשם שנאמר בשבת בראשית׃}
\threeverse{\arabic{verse}}%Leviticus25:3
{שֵׁ֤שׁ שָׁנִים֙ תִּזְרַ֣ע שָׂדֶ֔ךָ וְשֵׁ֥שׁ שָׁנִ֖ים תִּזְמֹ֣ר כַּרְמֶ֑ךָ וְאָסַפְתָּ֖ אֶת\maqqaf תְּבוּאָתָֽהּ׃}
{שֵׁית שְׁנִין תִּזְרַע חַקְלָךְ וְשֵׁית שְׁנִין תִּכְסַח כַּרְמָךְ וְתִכְנוֹשׁ יָת עֲלַלְתַּהּ׃}
{Six years thou shalt sow thy field, and six years thou shalt prune thy vineyard, and gather in the produce thereof.}{\arabic{verse}}
\threeverse{\aliya{לוי}}%Leviticus25:4
{וּבַשָּׁנָ֣ה הַשְּׁבִיעִ֗ת שַׁבַּ֤ת שַׁבָּתוֹן֙ יִהְיֶ֣ה לָאָ֔רֶץ שַׁבָּ֖ת לַיהֹוָ֑ה שָֽׂדְךָ֙ לֹ֣א תִזְרָ֔ע וְכַרְמְךָ֖ לֹ֥א תִזְמֹֽר׃}
{וּבְשַׁתָּא שְׁבִיעֵיתָא נְיָח שְׁמִטְּתָא יְהֵי לְאַרְעָא דְּתַשְׁמֵיט קֳדָם יְיָ חַקְלָךְ לָא תִזְרַע וְכַרְמָךְ לָא תִכְסַח׃}
{But in the seventh year shall be a sabbath of solemn rest for the land, a sabbath unto the \lord; thou shalt neither sow thy field, nor prune thy vineyard.}{\arabic{verse}}
\rashi{\rashiDH{יהיה לארץ.} לשדות ולכרמים׃\quad \rashiDH{לא תזמור.} שקוצצין זמורותיה, ותרגומו לָא תִכְסָח, ודומה לו קוֹצִים כְּסוּחִים (ישעיה לג, יב), שְׂרֻפָה בָאֵשׁ כְּסוּחָה (תהלים פ, יז)׃}
\threeverse{\arabic{verse}}%Leviticus25:5
{אֵ֣ת סְפִ֤יחַ קְצִֽירְךָ֙ לֹ֣א תִקְצ֔וֹר וְאֶת\maqqaf עִנְּבֵ֥י נְזִירֶ֖ךָ לֹ֣א תִבְצֹ֑ר שְׁנַ֥ת שַׁבָּת֖וֹן יִהְיֶ֥ה לָאָֽרֶץ׃}
{יָת כָּתֵי חֲצָדָךְ לָא תִחְצוּד וְיָת עִנְּבֵי שִׁבְקָךְ לָא תִקְטוּף שְׁנַת שְׁמִטְּתָא יְהֵי לְאַרְעָא׃}
{That which groweth of itself of thy harvest thou shalt not reap, and the grapes of thy undressed vine thou shalt not gather; it shall be a year of solemn rest for the land.}{\arabic{verse}}
\rashi{\rashiDH{את ספיח קצירך.} אפילו לא זרעתה, והיא צמחה מן הזרע שנפל בה בעת הקציר, הוא קרוי ספיח׃\quad \rashiDH{לא תקצור.} להיות מחזיק בו כשאר קציר אלא הפקר יהיה לכל׃\quad \rashiDH{נזירך.} שהנזרת והפרשת בני אדם מהם ולא הפקרתם׃\quad \rashiDH{לא תבצר.} אותם אינך בוצר, אלא מן המופקר׃}
\threeverse{\arabic{verse}}%Leviticus25:6
{וְ֠הָיְתָ֠ה שַׁבַּ֨ת הָאָ֤רֶץ לָכֶם֙ לְאׇכְלָ֔ה לְךָ֖ וּלְעַבְדְּךָ֣ וְלַאֲמָתֶ֑ךָ וְלִשְׂכִֽירְךָ֙ וּלְתוֹשָׁ֣בְךָ֔ הַגָּרִ֖ים עִמָּֽךְ׃}
{וּתְהֵי שְׁמִטַּת אַרְעָא לְכוֹן לְמֵיכַל לָךְ וּלְעַבְדָךְ וּלְאַמְתָּךְ וְלַאֲגִירָךְ וּלְתוֹתָבָךְ דְּדָיְירִין עִמָּךְ׃}
{And the sabbath-produce of the land shall be for food for you: for thee, and for thy servant and for thy maid, and for thy hired servant and for the settler by thy side that sojourn with thee;}{\arabic{verse}}
\rashi{\rashiDH{והיתה שבת הארץ וגו׳.} אע״פ שאסרתים עליך לא באכילה ולא בהנאה אסרתים אלא שלא תנהוג בהם כבעל הבית אלא הכל יהיו שוים בה אתה ושכירך ותושבך׃\quad \rashiDH{שבת הארץ לכם לאכלה.} מן הַשְּׁבוּת אתה אוכל, ואי אתה אוכל מן השמור (ת״כ פרק א, ג)׃\quad \rashiDH{לך ולעבדך ולאמתך.} לפי שנאמר וְאָכְלוּ אֶבְיֹנֵי עַמֶּךָ (שמות כג, יא), יכול יהיו אסורים באכילה לעשירים, תלמוד לומר לך ולעבדך ולאמתך, הרי בעלים ועבדים ושפחות אמורים כאן (ת״כ שם ו)׃\quad \rashiDH{ולשכירך ולתושבך.} אף הגוים (שם ז)׃}
\threeverse{\arabic{verse}}%Leviticus25:7
{וְלִ֨בְהֶמְתְּךָ֔ וְלַֽחַיָּ֖ה אֲשֶׁ֣ר בְּאַרְצֶ֑ךָ תִּהְיֶ֥ה כׇל\maqqaf תְּבוּאָתָ֖הּ לֶאֱכֹֽל׃ \setuma }
{וְלִבְעִירָךְ וּלְחַיְתָא דִּבְאַרְעָךְ תְּהֵי כָל עֲלַלְתַּהּ לְמֵיכַל׃}
{and for thy cattle, and for the beasts that are in thy land, shall all the increase thereof be for food.}{\arabic{verse}}
\rashi{\rashiDH{ולבהמתך ולחיה.} אם חיה אוכלת, בהמה לא כל שכן, שמזונותיה עליך, מה תלמוד לומר ולבהמתך, מקיש בהמה לחיה, כל זמן שחיה אוכלת מן השדה האכל לבהמתך מן הבית, כָּלָה לחיה מן השדה כַּלֵּה לבהמתך מן הבית (ת״כ. שם ח.  תענית ו׃)׃}
\threeverse{\aliya{ישראל}}%Leviticus25:8
{וְסָפַרְתָּ֣ לְךָ֗ שֶׁ֚בַע שַׁבְּתֹ֣ת שָׁנִ֔ים שֶׁ֥בַע שָׁנִ֖ים שֶׁ֣בַע פְּעָמִ֑ים וְהָי֣וּ לְךָ֗ יְמֵי֙ שֶׁ֚בַע שַׁבְּתֹ֣ת הַשָּׁנִ֔ים תֵּ֥שַׁע וְאַרְבָּעִ֖ים שָׁנָֽה׃}
{וְתִמְנֵי לָךְ שְׁבַע שְׁמִטָּן דִּשְׁנִין שְׁבַע שְׁנִין שְׁבַע זִמְנִין וִיהוֹן לָךְ יוֹמֵי שְׁבַע שְׁמִטָּן דִּשְׁנִין אַרְבְּעִין וּתְשַׁע שְׁנִין׃}
{And thou shalt number seven sabbaths of years unto thee, seven times seven years; and there shall be unto thee the days of seven sabbaths of years, even forty and nine years.}{\arabic{verse}}
\rashi{\rashiDH{שבתת שנים.} שמטות שנים, יכול יעשה שבע שנים רצופות שמטה ויעשה יובל אחריהם, תלמוד לומר שבע שנים שבע פעמים, הוי אומר כל שמטה ושמטה בזמנה (ת״כ פרשתא ב, א)׃\quad \rashiDH{והיו לך ימי שבע וגו׳.} מגיד לך שאף על פי שלא עשית שמטות עשה יובל לסוף מ״ט שנה. ופשוטו של מקרא יעלה לך חשבון שנות השמטות למספר מ״ט׃}
\threeverse{\arabic{verse}}%Leviticus25:9
{וְהַֽעֲבַרְתָּ֞ שׁוֹפַ֤ר תְּרוּעָה֙ בַּחֹ֣דֶשׁ הַשְּׁבִעִ֔י בֶּעָשׂ֖וֹר לַחֹ֑דֶשׁ בְּיוֹם֙ הַכִּפֻּרִ֔ים תַּעֲבִ֥ירוּ שׁוֹפָ֖ר בְּכׇל\maqqaf אַרְצְכֶֽם׃}
{וְתַעְבַּר שׁוֹפַר יַבָּבָא בְּיַרְחָא שְׁבִיעָאָה בְּעֶשְׂרָא לְיַרְחָא בְּיוֹמָא דְּכִפּוּרַיָּא תַּעְבְּרוּן שׁוֹפָרָא בְּכָל אֲרַעְכוֹן׃}
{Then shalt thou make proclamation with the blast of the horn on the tenth day of the seventh month; in the day of atonement shall ye make proclamation with the horn throughout all your land.}{\arabic{verse}}
\rashi{\rashiDH{והעברת.} לשון וַיַּעֲבִירוּ קוֹל בַּמַּחֲנֶה (שמות לו, ו), לשון הכרזה (ר״ה לד.)׃\quad \rashiDH{ביום הכפורים.} ממשמע שנאמר ביום הכפורים, איני יודע שהוא בעשור לחדש, אם כן למה נאמר בעשור לחדש, אלא לומר לך, תקיעת עשור לחדש דוחה שבת בכל ארצכם, ואין תקיעת ראש השנה דוחה שבת בכל ארצכם, אלא בבית דין בלבד (ת״כ שם ה)׃}
\threeverse{\arabic{verse}}%Leviticus25:10
{וְקִדַּשְׁתֶּ֗ם אֵ֣ת שְׁנַ֤ת הַחֲמִשִּׁים֙ שָׁנָ֔ה וּקְרָאתֶ֥ם דְּר֛וֹר בָּאָ֖רֶץ לְכׇל\maqqaf יֹשְׁבֶ֑יהָ יוֹבֵ֥ל הִוא֙ תִּהְיֶ֣ה לָכֶ֔ם וְשַׁבְתֶּ֗ם אִ֚ישׁ אֶל\maqqaf אֲחֻזָּת֔וֹ וְאִ֥ישׁ אֶל\maqqaf מִשְׁפַּחְתּ֖וֹ תָּשֻֽׁבוּ׃}
{וּתְקַדְּשׁוּן יָת שְׁנַת חַמְשִׁין שְׁנִין וְתִקְרוֹן חֵירוּתָא בְּאַרְעָא לְכָל יָתְבַֽהָא יוֹבֵילָא הִיא תְּהֵי לְכוֹן וּתְתוּבוּן גְּבַר לְאַחְסָנְתֵיהּ וּגְבַר לְזַרְעִיתֵיהּ תְּתוּבוּן׃}
{And ye shall hallow the fiftieth year, and proclaim liberty throughout the land unto all the inhabitants thereof; it shall be a jubilee unto you; and ye shall return every man unto his possession, and ye shall return every man unto his family.}{\arabic{verse}}
\rashi{\rashiDH{וקדשתם.} (ת״כ פרק ב, א) בכניסתה מקדשין אותה בבית דין, ואומרים מקודשת השנה (ר״ה ח׃)׃\quad \rashiDH{וקראתם דרור.} לעבדים, בין נרצע, בין שלא כָּלוּ לו שש שנים משנמכר, אמר ר׳ יהודה, מהו לשון דרור, כמדייר בי דיירא וכו׳ (ראש השנה ט׃), שדר בכל מקום שהוא רוצה ואינו ברשות אחרים׃\quad \rashiDH{יובל הוא.} שנה זאת מובדלת משאר שנים בנקיבת שם לה לבדה, ומה שמה, יובל שמה, על שם תקיעת שופר׃\quad \rashiDH{ושבתם איש אל אחזתו.} שהשדות חוזרות לבעליהן׃\quad \rashiDH{ואיש אל משפחתו תשבו.} לרבות את הנרצע (קידושין טו.)׃}
\threeverse{\arabic{verse}}%Leviticus25:11
{יוֹבֵ֣ל הִ֗וא שְׁנַ֛ת הַחֲמִשִּׁ֥ים שָׁנָ֖ה תִּהְיֶ֣ה לָכֶ֑ם לֹ֣א תִזְרָ֔עוּ וְלֹ֤א תִקְצְרוּ֙ אֶת\maqqaf סְפִיחֶ֔יהָ וְלֹ֥א תִבְצְר֖וּ אֶת\maqqaf נְזִרֶֽיהָ׃}
{יוֹבֵילָא הִיא שְׁנַת חַמְשִׁין שְׁנִין תְּהֵי לְכוֹן לָא תִזְרְעוּן וְלָא תִחְצְדוּן יָת כָּתַֽהָא וְלָא תִקְטְפוּן יָת שִׁבְקַֽהָא׃}
{A jubilee shall that fiftieth year be unto you; ye shall not sow, neither reap that which groweth of itself in it, nor gather the grapes in it of the undressed vines.}{\arabic{verse}}
\rashi{\rashiDH{יובל הוא שנת החמשים שנה.} מה תלמוד לומר, לפי שנאמר וקדשתם וגו׳, כדאיתא בראש השנה (ח׃), ובת״כ (פרק ג, א)׃\quad \rashiDH{את נזריה.} את הענבים המשומרים, אבל בוצר אתה מן המופקרים. כשם שנאמר בשביעית, כך נאמר ביובל, נמצאו שתי שנים קדושות סמוכות זו לזו, שנת מ״ט שמטה, ושנת החמשים יובל׃}
\threeverse{\arabic{verse}}%Leviticus25:12
{כִּ֚י יוֹבֵ֣ל הִ֔וא קֹ֖דֶשׁ תִּהְיֶ֣ה לָכֶ֑ם מִ֨ן\maqqaf הַשָּׂדֶ֔ה תֹּאכְל֖וּ אֶת\maqqaf תְּבוּאָתָֽהּ׃}
{אֲרֵי יוֹבֵילָא הִיא קוּדְשָׁא תְּהֵי לְכוֹן מִן חַקְלָא תֵּיכְלוּן יָת עֲלַלְתַּהּ׃}
{For it is a jubilee; it shall be holy unto you; ye shall eat the increase thereof out of the field.}{\arabic{verse}}
\rashi{\rashiDH{קדש תהיה לכם}.תופסת דמיה כהקדש, יכול תצא היא לחולין, תלמוד לומר תהיה, בהוייתה תהא (סוכה מ׃  ת״כ פרק ג, ג)׃\quad \rashiDH{מן השדה תאכלו.} על ידי השדה אתה אוכל מן הבית, שאם כָּלָה לחיה מן השדה, אתה צריך לבער מן הבית (שם ד). כשם שנאמר בשביעית, כך נאמר ביובל׃}
\threeverse{\arabic{verse}}%Leviticus25:13
{בִּשְׁנַ֥ת הַיּוֹבֵ֖ל הַזֹּ֑את תָּשֻׁ֕בוּ אִ֖ישׁ אֶל\maqqaf אֲחֻזָּתֽוֹ׃}
{בְּשַׁתָּא דְּיוֹבֵלָא הָדָא תְּתוּבוּן גְּבַר לְאַחְסָנְתֵיהּ׃}
{In this year of jubilee ye shall return every man unto his possession.}{\arabic{verse}}
\rashi{\rashiDH{תשובו איש אל אחזתו.} והרי כבר נאמר וְשַׁבְתֶּם אִישׁ אֶל אֲחֻזָּתוֹ, אלא לרבות, המוכר שדהו ועמד בנו וגאלה, שחוזרת לאביו ביובל׃}
\aliyacounter{שני}
\newseder{20}
\threeverse{\aliya{שני}\newline\vspace{-4pt}\newline\seder{כ}}%Leviticus25:14
{וְכִֽי\maqqaf תִמְכְּר֤וּ מִמְכָּר֙ לַעֲמִיתֶ֔ךָ א֥וֹ קָנֹ֖ה מִיַּ֣ד עֲמִיתֶ֑ךָ אַל\maqqaf תּוֹנ֖וּ אִ֥ישׁ אֶת\maqqaf אָחִֽיו׃}
{וַאֲרֵי תְזַבֵּין זְבִינִין לְחַבְרָךְ אוֹ תִזְבּוֹן מִיַּד חַבְרָךְ לָא תוֹנוֹן גְּבַר יָת אֲחוּהִי׃}
{And if thou sell aught unto thy neighbour, or buy of thy neighbour’s hand, ye shall not wrong one another.}{\arabic{verse}}
\rashi{\rashiDH{וכי תמכרו וגו׳.} לפי פשוטו, כמשמעו. ועוד יש דרשה, מנין כשאתה מוכר מכור לישראל חברך, תלמוד לומר וכי תמכרו ממכר לעמיתך, מכור, ומנין שאם באת לקנות, קנה מישראל חברך, תלמוד לומר או קנה מיד עמיתך׃\quad \rashiDH{אל תונו.} זו אונאת ממון (שם פרשתא ג, ד  ב״מ נח׃)׃}
\threeverse{\arabic{verse}}%Leviticus25:15
{בְּמִסְפַּ֤ר שָׁנִים֙ אַחַ֣ר הַיּוֹבֵ֔ל תִּקְנֶ֖ה מֵאֵ֣ת עֲמִיתֶ֑ךָ בְּמִסְפַּ֥ר שְׁנֵֽי\maqqaf תְבוּאֹ֖ת יִמְכׇּר\maqqaf לָֽךְ׃}
{בְּמִנְיַן שְׁנַיָּא בָּתַר יוֹבֵילָא תִּזְבּוֹן מִן חַבְרָךְ בְּמִנְיַן שְׁנֵי עֲלַלְתָּא יְזַבֵּין לָךְ׃}
{According to the number of years after the jubilee thou shalt buy of thy neighbour, and according unto the number of years of the crops he shall sell unto thee.}{\arabic{verse}}
\rashi{\rashiDH{במספר שנים אחר היובל תקנה.} זהו פשוטו, ליישב המקרא על אופניו, על האונאה בא להזהיר, כשתמכור או תקנה קרקע דע כמה שנים יש עד היובל, ולפי השנים ותבואות השדה שהיא ראויה לעשות ימכור המוכר ויקנה הקונה, שהרי סופו להחזירה לו בשנת היובל, ואם יש שנים מועטות וזה מוכרה בדמים יקרים, הרי נתאנה לוקח, ואם יש שנים מרובות ואכל ממנה תבואות הרבה, הרי נתאנה מוכר, לפיכך צריך לקנותה לפי הזמן, וזהו שנאמר, במספר שני תבואות ימכר לך, לפי מנין שני התבואות שתהא עומדת ביד הלוקח תמכור לו. ורבותינו דרשו מכאן (ערכין כט׃), שהמוכר שדהו אינו רשאי לגאול פחות משתי שנים, שתעמוד שתי שנים ביד הלוקחו מיום ליום, ואפילו יש שלש תבואות באותן שתי שנים, כגון, שמכרה לו בקמותיה, ושני, אינו יוצא מפשוטו, כלומר, מספר שנים של תבואות, ולא של שדפון, ומיעוט שָׁנִים שְׁנַיִם׃}
\threeverse{\arabic{verse}}%Leviticus25:16
{לְפִ֣י \legarmeh  רֹ֣ב הַשָּׁנִ֗ים תַּרְבֶּה֙ מִקְנָת֔וֹ וּלְפִי֙ מְעֹ֣ט הַשָּׁנִ֔ים תַּמְעִ֖יט מִקְנָת֑וֹ כִּ֚י מִסְפַּ֣ר תְּבוּאֹ֔ת ה֥וּא מֹכֵ֖ר לָֽךְ׃}
{לְפוֹם סַגְיוּת שְׁנַיָּא תַּסְגֵּי זְבִינוֹהִי וּלְפוֹם זְעֵירוּת שְׁנַיָּא תַּזְעַר זְבִינוֹהִי אֲרֵי מִנְיַן עֲלַלְתָּא הוּא מְזַבֵּין לָךְ׃}
{According to the multitude of the years thou shalt increase the price thereof, and according to the fewness of the years thou shalt diminish the price of it; for the number of crops doth he sell unto thee.}{\arabic{verse}}
\rashi{\rashiDH{תרבה מקנתו.} תמכרנה ביוקר׃\quad \rashiDH{תמעיט מקנתו.} תמעיט בדמיה׃}
\threeverse{\arabic{verse}}%Leviticus25:17
{וְלֹ֤א תוֹנוּ֙ אִ֣ישׁ אֶת\maqqaf עֲמִית֔וֹ וְיָרֵ֖אתָ מֵֽאֱלֹהֶ֑יךָ כִּ֛י אֲנִ֥י יְהֹוָ֖ה אֱלֹהֵיכֶֽם׃}
{וְלָא תּוֹנוֹן גְּבַר יָת חַבְרֵיהּ וְתִדְחַל מֵאֱלָהָךְ אֲרֵי אֲנָא יְיָ אֱלָהֲכוֹן׃}
{And ye shall not wrong one another; but thou shalt fear thy God; for I am the \lord\space your God.}{\arabic{verse}}
\rashi{\rashiDH{ולא תונו איש את עמיתו.} כאן הזהיר על אונאת דברים (ת״כ פרק ד, א), שלא יקניט איש את חבירו, ולא ישיאנו עצה שאינה הוגנת לו, לפי דרכו והנאתו של יועץ, ואם תאמר מי יודע אם נתכוונתי לרעה, לכך נאמר ויראת מאלהיך, היודע מחשבות הוא יודע. כל דבר המסור ללב, שאין מכיר אלא מי שהמחשבה בלבו, נאמר בו ויראת מאלהיך (ב״מ נח׃)׃}
\threeverse{\arabic{verse}}%Leviticus25:18
{וַעֲשִׂיתֶם֙ אֶת\maqqaf חֻקֹּתַ֔י וְאֶת\maqqaf מִשְׁפָּטַ֥י תִּשְׁמְר֖וּ וַעֲשִׂיתֶ֣ם אֹתָ֑ם וִֽישַׁבְתֶּ֥ם עַל\maqqaf הָאָ֖רֶץ לָבֶֽטַח׃}
{וְתַעְבְּדוּן יָת קְיָמַי וְיָת דִּינַי תִּטְּרוּן וְתַעְבְּדוּן יָתְהוֹן וְתִתְּבוּן עַל אַרְעָא לְרֻחְצָן׃}
{Wherefore ye shall do My statutes, and keep Mine ordinances and do them; and ye shall dwell in the land in safety.}{\arabic{verse}}
\rashi{\rashiDH{וישבתם על הארץ לבטח.} שֶׁבְּעָוֹן שמטה ישראל גולים, שנאמר אָז תִּרְצֶה הָאָרֶץ אֶת שַׁבְּתֹתֶיהָ, וְהִרְצָת אֶת שַׁבְּתֹתֶיהָ (ויקרא כו, לד  שבת לג.), ושבעים שנה של גלות בבל כנגד שבעים שמטות שבטלו היו׃}
\aliyacounter{שלישי}
\threeverse{\aliya{שלישי\newline (שני)}}%Leviticus25:19
{וְנָתְנָ֤ה הָאָ֙רֶץ֙ פִּרְיָ֔הּ וַאֲכַלְתֶּ֖ם לָשֹׂ֑בַע וִֽישַׁבְתֶּ֥ם לָבֶ֖טַח עָלֶֽיהָ׃}
{וְתִתֵּין אַרְעָא אִבַּהּ וְתֵיכְלוּן לְמִשְׂבַּע וְתִתְּבוּן לְרֻחְצָן עֲלַהּ׃}
{And the land shall yield her fruit, and ye shall eat until ye have enough, and dwell therein in safety.}{\arabic{verse}}
\rashi{\rashiDH{ונתנה הארץ וגו׳ וישבתם לבטח עליה.} שלא תדאגו משנת בצורת׃\quad \rashiDH{ואכלתם לשבע.} אף בתוך המעים תהא בו ברכה׃}
\threeverse{\arabic{verse}}%Leviticus25:20
{וְכִ֣י תֹאמְר֔וּ מַה\maqqaf נֹּאכַ֖ל בַּשָּׁנָ֣ה הַשְּׁבִיעִ֑ת הֵ֚ן לֹ֣א נִזְרָ֔ע וְלֹ֥א נֶאֱסֹ֖ף אֶת\maqqaf תְּבוּאָתֵֽנוּ׃}
{וַאֲרֵי תֵימְרוּן מָא נֵיכוֹל בְּשַׁתָּא שְׁבִיעֵיתָא הָא לָא נִזְרַע וְלָא נִכְנוֹשׁ יָת עֲלַלְתַּֽנָא׃}
{And if ye shall say: ‘What shall we eat the seventh year? behold, we may not sow, nor gather in our increase’;}{\arabic{verse}}
\rashi{\rashiDH{ולא נאסף.} אל הבית׃\quad \rashiDH{את תבואתנו.} כגון יין ופירות האילן, וספיחין הבאים מאליהם (פסחים נא׃)׃}
\threeverse{\arabic{verse}}%Leviticus25:21
{וְצִוִּ֤יתִי אֶת\maqqaf בִּרְכָתִי֙ לָכֶ֔ם בַּשָּׁנָ֖ה הַשִּׁשִּׁ֑ית וְעָשָׂת֙ אֶת\maqqaf הַתְּבוּאָ֔ה לִשְׁלֹ֖שׁ הַשָּׁנִֽים׃}
{וַאֲפַקֵּיד יָת בִּרְכְתִי לְכוֹן בְּשַׁתָּא שְׁתִיתֵיתָא וְתַעֲבֵיד יָת עֲלַלְתָּא לִתְלָת שְׁנִין׃}
{then I will command My blessing upon you in the sixth year, and it shall bring forth produce for the three years.}{\arabic{verse}}
\rashi{\rashiDH{לשלש השנים.} למקצת הששית מניסן ועד ראש השנה, ולשביעית, ולשמינית, שיזרעו בשמינית במרחשון ויקצרו בניסן׃}
\threeverse{\arabic{verse}}%Leviticus25:22
{וּזְרַעְתֶּ֗ם אֵ֚ת הַשָּׁנָ֣ה הַשְּׁמִינִ֔ת וַאֲכַלְתֶּ֖ם מִן\maqqaf הַתְּבוּאָ֣ה יָשָׁ֑ן עַ֣ד \legarmeh  הַשָּׁנָ֣ה הַתְּשִׁיעִ֗ת עַד\maqqaf בּוֹא֙ תְּב֣וּאָתָ֔הּ תֹּאכְל֖וּ יָשָֽׁן׃}
{וְתִזְרְעוּן יָת שַׁתָּא תְּמִינֵיתָא וְתֵיכְלוּן מִן עֲלַלְתָּא עַתִּיקָא עַד שַׁתָּא תְּשִׁיעֵיתָא עַד מֵיעַל עֲלַלְתַּהּ תֵּיכְלוּן עַתִּיקָא׃}
{And ye shall sow the eighth year, and eat of the produce, the old store; until the ninth year, until her produce come in, ye shall eat the old store.}{\arabic{verse}}
\rashi{\rashiDH{עד השנה התשיעת.} עד חג הסכות של תשיעית שהיא עת בוא תבואתה של שמינית לתוך הבית, שכל ימות הקיץ היו בשדה בגרנות, ובתשרי הוא עת האסיף לבית. ופעמים שהיתה צריכה לעשות לארבע שנים, בששית שלפני השמטה השביעית, שהן בטלין מעבודת קרקע שתי שנים רצופות השביעית והיובל, ומקרא זה נאמר בשאר השמטות כולן׃}
\threeverse{\arabic{verse}}%Leviticus25:23
{וְהָאָ֗רֶץ לֹ֤א תִמָּכֵר֙ לִצְמִתֻ֔ת כִּי\maqqaf לִ֖י הָאָ֑רֶץ כִּֽי\maqqaf גֵרִ֧ים וְתוֹשָׁבִ֛ים אַתֶּ֖ם עִמָּדִֽי׃}
{וְאַרְעָא לָא תִזְדַּבַּן לַחְלוּטִין אֲרֵי דִּילִי אַרְעָא אֲרֵי דַּיָּירִין וְתוֹתָבִין אַתּוּן קֳדָמָי׃}
{And the land shall not be sold in perpetuity; for the land is Mine; for ye are strangers and settlers with Me.}{\arabic{verse}}
\rashi{\rashiDH{והארץ לא תמכר.} ליתן לאו על חזרת שדות לבעלים ביובל, שלא יהא הלוקח כובשה (ת״כ שם ח)׃\quad \rashiDH{לצמתת.} לפסיקה, למכירה פסוקה עולמית׃\quad \rashiDH{כי לי הארץ.} (ת״כ) אל תרע עינך בה (שם), שאינה שלך׃}
\threeverse{\arabic{verse}}%Leviticus25:24
{וּבְכֹ֖ל אֶ֣רֶץ אֲחֻזַּתְכֶ֑ם גְּאֻלָּ֖ה תִּתְּנ֥וּ לָאָֽרֶץ׃ \setuma }
{וּבְכֹל אֲרַע אַחְסָנַתְכוֹן פּוּרְקָנָא תִּתְּנוּן לְאַרְעָא׃}
{And in all the land of your possession ye shall grant a redemption for the land.}{\arabic{verse}}
\rashi{\rashiDH{ובכל ארץ אחזתכם.} (ת״כ) לרבות בתים, ועבד עברי (שם ט), ודבר זה מפורש בקידושין בפרק א׳ (דף כא.). ולפי פשוטו, סמוך לפרשה שלאחריו, שהמוכר אחוזתו רשאי לגאלה לאחר שתי שנים, או הוא, או קרובו, ואין הלוקח יכול לעכב׃}
\aliyacounter{רביעי}
\threeverse{\aliya{רביעי}}%Leviticus25:25
{כִּֽי\maqqaf יָמ֣וּךְ אָחִ֔יךָ וּמָכַ֖ר מֵאֲחֻזָּת֑וֹ וּבָ֤א גֹֽאֲלוֹ֙ הַקָּרֹ֣ב אֵלָ֔יו וְגָאַ֕ל אֵ֖ת מִמְכַּ֥ר אָחִֽיו׃}
{אֲרֵי יִתְמַסְכַּן אֲחוּךְ וִיזַבֵּין מֵאַחְסָנְתֵיהּ וְיֵיתֵי פָרִיקֵיהּ דְּקָרִיב לֵיהּ וְיִפְרוֹק יָת זְבִינֵי אֲחוּהִי׃}
{If thy brother be waxen poor, and sell some of his possession, then shall his kinsman that is next unto him come, and shall redeem that which his brother hath sold.}{\arabic{verse}}
\rashi{\rashiDH{כי ימוך אחיך ומכר.} מלמד שאין אדם רשאי למכור שדהו אלא מחמת דוחק עוני (ת״כ פרק ה, א)׃\quad \rashiDH{מאחזתו.} ולא כולה, למדה תורה דרך ארץ, שישייר שדה לעצמו׃\quad \rashiDH{וגאל את ממכר אחיו.} ואין הלוקח יכול לעכב׃}
\threeverse{\arabic{verse}}%Leviticus25:26
{וְאִ֕ישׁ כִּ֛י לֹ֥א יִֽהְיֶה\maqqaf לּ֖וֹ גֹּאֵ֑ל וְהִשִּׂ֣יגָה יָד֔וֹ וּמָצָ֖א כְּדֵ֥י גְאֻלָּתֽוֹ׃}
{וּגְבַר אֲרֵי לָא יְהֵי לֵיהּ פָּרִיק וְתַדְבֵּיק יְדֵיהּ וְיַשְׁכַּח כְּמִסַּת פּוּרְקָנֵיהּ׃}
{And if a man have no one to redeem it, and he be waxen rich and find sufficient means to redeem it;}{\arabic{verse}}
\rashi{\rashiDH{ואיש כי לא יהיה לו גאל.} וכי יש לך אדם בישראל שאין לו גואלים, אלא גואל שיוכל לגאול ממכרו (קידושין שם  ת״כ פרק ה, ב)׃}
\threeverse{\arabic{verse}}%Leviticus25:27
{וְחִשַּׁב֙ אֶת\maqqaf שְׁנֵ֣י מִמְכָּר֔וֹ וְהֵשִׁיב֙ אֶת\maqqaf הָ֣עֹדֵ֔ף לָאִ֖ישׁ אֲשֶׁ֣ר מָֽכַר\maqqaf ל֑וֹ וְשָׁ֖ב לַאֲחֻזָּתֽוֹ׃}
{וִיחַשֵּׁיב יָת שְׁנֵי זְבִינוֹהִי וְיָתִיב יָת מוֹתָרָא לִגְבַר דְּזַבֵּין לֵיהּ וִיתוּב לְאַחְסָנְתֵיהּ׃}
{then let him count the years of the sale thereof, and restore the overplus unto the man to whom he sold it; and he shall return unto his possession.}{\arabic{verse}}
\rashi{\rashiDH{וחשב את שני ממכרו.} כמה שנים היו עד היובל, כך וכך, ובכמה מכרתיה לך, בכך וכך, עתיד היית להחזירה ביובל, נמצאת קונה מספר התבואות כפי חשבון של כל שנה, אכלת אותה שלש שנים או ארבע, הוצא את דמיהן מן החשבון, וטול את השאר, וזהו׃ \rashiDH{והשיב את העודף.} בדמי המקח על האכילה שאכל, ויתנם ללוקח׃\quad \rashiDH{לאיש אשר מכר לו.} המוכר הזה, שבא לגאלה (ערכין ל.)׃}
\threeverse{\arabic{verse}}%Leviticus25:28
{וְאִ֨ם לֹֽא\maqqaf מָצְאָ֜ה יָד֗וֹ דֵּי֮ הָשִׁ֣יב לוֹ֒ וְהָיָ֣ה מִמְכָּר֗וֹ בְּיַד֙ הַקֹּנֶ֣ה אֹת֔וֹ עַ֖ד שְׁנַ֣ת הַיּוֹבֵ֑ל וְיָצָא֙ בַּיֹּבֵ֔ל וְשָׁ֖ב לַאֲחֻזָּתֽוֹ׃ \setuma }
{וְאִם לָא אַשְׁכַּחַת יְדֵיהּ כְּמִסַּת דְּיָתִיב לֵיהּ וִיהֵי זְבִינוֹהִי בְּיַד דִּזְבַן יָתֵיהּ עַד שַׁתָּא דְּיוֹבֵילָא וְיִפּוֹק בְּיוֹבֵילָא וִיתוּב לְאַחְסָנְתֵיהּ׃}
{But if he have not sufficient means to get it back for himself, then that which he hath sold shall remain in the hand of him that hath bought it until the year of jubilee; and in the jubilee it shall go out, and he shall return unto his possession.}{\arabic{verse}}
\rashi{\rashiDH{די השיב לו.} מכאן שאינו גואל לחצאין׃\quad \rashiDH{עד שנת היובל.} שלא יכנס לתוך אותה שנה כלום, שהיובל משמט בתחלתו (ת״כ שם ז)׃}
\aliyacounter{חמישי}
\threeverse{\aliya{חמישי\newline (שלישי)}}%Leviticus25:29
{וְאִ֗ישׁ כִּֽי\maqqaf יִמְכֹּ֤ר בֵּית\maqqaf מוֹשַׁב֙ עִ֣יר חוֹמָ֔ה וְהָיְתָה֙ גְּאֻלָּת֔וֹ עַד\maqqaf תֹּ֖ם שְׁנַ֣ת מִמְכָּר֑וֹ יָמִ֖ים תִּהְיֶ֥ה גְאֻלָּתֽוֹ׃}
{וּגְבַר אֲרֵי יְזַבֵּין בֵּית מוֹתַב קַרְתָּא מַקְּפָא שׁוּר וִיהֵי פוּרְקָנֵיהּ עַד מִשְׁלַם שַׁתָּא דִּזְבִינוֹהִי עִדָּן בְּעִדָּן יְהֵי פוּרְקָנֵיהּ׃}
{And if a man sell a dwelling-house in a walled city, then he may redeem it within a whole year after it is sold; for a full year shall he have the right of redemption.}{\arabic{verse}}
\rashi{\rashiDH{בית מושב עיר חומה.} בית בתוך עיר המוקפת חומה מימות יהושע בן נון (שם פרשתא ד, א)׃\quad \rashiDH{והיתה גאלתו.} לפי שנאמר בשדה שיכול לגאלה משתי שנים ואילך כל זמן שירצה, ובתוך שתי שנים הראשונים אינו יכול לגאלה, הוצרך לפרש בזה שהוא חלוף, שאם רצה לגאול בשנה ראשונה גואלה, ולאחר מכאן אינו גואלה׃\quad \rashiDH{והיתה גאלתו.} של בית׃\quad \rashiDH{ימים.} ימי שנה שלימה קרוים ימים, וכן תֵּשֵׁב הַנַּעֲרָה אִתָּנוּ יָמִים (בראשית כד, נה)׃}
\threeverse{\arabic{verse}}%Leviticus25:30
{וְאִ֣ם לֹֽא\maqqaf יִגָּאֵ֗ל עַד\maqqaf מְלֹ֣את לוֹ֮ שָׁנָ֣ה תְמִימָה֒ וְ֠קָ֠ם הַבַּ֨יִת אֲשֶׁר\maqqaf בָּעִ֜יר אֲשֶׁר\maqqaf \qk{ל֣וֹ}{לא} חֹמָ֗ה לַצְּמִיתֻ֛ת לַקֹּנֶ֥ה אֹת֖וֹ לְדֹרֹתָ֑יו לֹ֥א יֵצֵ֖א בַּיֹּבֵֽל׃}
{וְאִם לָא יִתְפְּרֵיק עַד מִשְׁלַם לֵיהּ שַׁתָּא שַׁלְמְתָא וִיקוּם בֵּיתָא דִּבְקַרְתָּא דְּלֵיהּ שׁוּרָא לַחְלוּטִין לְדִזְבַן יָתֵיהּ לְדָרוֹהִי לָא יִפּוֹק בְּיוֹבֵילָא׃}
{And if it be not redeemed within the space of a full year, then the house that is in the walled city shall be made sure in perpetuity to him that bought it, throughout his generations; it shall not go out in the jubilee.}{\arabic{verse}}
\rashi{\rashiDH{וקם הבית וגו׳ לצמיתת.} יצא מכחו של מוכר ועומד בכחו של קונה׃\quad \rashiDH{(אשר לא חמה.} לו קרינן, אמרו רז״ל (ערכין לב.) אע״פ שאין לו עכשיו, הואיל והיתה לו קודם לכן. ועיר נקבה היא, והוצרך לכתוב לה, אלא מתוך שצריך לכתוב לא בפנים, תקנו לו במסורת זה נופל על זה)׃\quad \rashiDH{לא יצא ביבל.} אמר רב ספרא (אף) אם פגע בו יובל בתוך שנתו לא יצא (שם לא׃)׃}
\threeverse{\arabic{verse}}%Leviticus25:31
{וּבָתֵּ֣י הַחֲצֵרִ֗ים אֲשֶׁ֨ר אֵין\maqqaf לָהֶ֤ם חֹמָה֙ סָבִ֔יב עַל\maqqaf שְׂדֵ֥ה הָאָ֖רֶץ יֵחָשֵׁ֑ב גְּאֻלָּה֙ תִּהְיֶה\maqqaf לּ֔וֹ וּבַיֹּבֵ֖ל יֵצֵֽא׃}
{וּבָתֵּי פַּצְחַיָּא דְּלֵית לְהוֹן שׁוּר מַקַּף סְחוֹר סְחוֹר עַל חֲקַל אַרְעָא יִתְחַשְׁבוּן פּוּרְקָנָא יְהֵי לְהוֹן וּבְיוֹבֵילָא יִפְּקוּן׃}
{But the houses of the villages which have no wall round about them shall be reckoned with the fields of the country; they may be redeemed, and they shall go out in the jubilee.}{\arabic{verse}}
\rashi{\rashiDH{ובתי החצרים.} כתרגומו פַּצְחַיָּא, עיירות פתוחות מאין חומה, ויש הרבה בספר יהושע הֶעָרִים וְחַצְרֵיהֶם (יהושע יג, כח), בְּחַצְרֵיהֶם וּבְטִירֹתָם (בראשית כה, טז)׃\quad \rashiDH{על שדה הארץ יחשב.} הרי הן כשדות הנגאלים עד היובל, ויוצאין ביובל לבעלים, אם לא נגאלו׃\quad \rashiDH{גאלה תהיה לו.} מיד אם ירצה, ובזה יפה כחו מכח שדות, שהשדות אין נגאלות עד שתי שנים (ת״כ פרק ו, ב  ערכין לג.)׃\quad וביובל יצא. בחנם׃}
\threeverse{\arabic{verse}}%Leviticus25:32
{וְעָרֵי֙ הַלְוִיִּ֔ם בָּתֵּ֖י עָרֵ֣י אֲחֻזָּתָ֑ם גְּאֻלַּ֥ת עוֹלָ֖ם תִּהְיֶ֥ה לַלְוִיִּֽם׃}
{וְקִרְוֵי לֵיוָאֵי בָּתֵּי קִרְוֵי אַחְסָנַתְהוֹן פּוּרְקַן עָלַם יְהֵי לְלֵיוָאֵי׃}
{But as for the cities of the Levites, the houses of the cities of their possession, the Levites shall have a perpetual right of redemption.}{\arabic{verse}}
\rashi{\rashiDH{וערי הלוים.} ארבעים ושמנה עיר שנתנו להם׃\quad \rashiDH{גאלת עולם.} גואל מיד אפילו לפני שתי שנים אם מכרו שדה משדותיהם הנתונות להם באלפים אמה סביבות הערים, או אם מכרו בית בעיר חומה גואלין לעולם, ואינו חלוט לסוף שנה (שם׃)׃}
\threeverse{\arabic{verse}}%Leviticus25:33
{וַאֲשֶׁ֤ר יִגְאַל֙ מִן\maqqaf הַלְוִיִּ֔ם וְיָצָ֧א מִמְכַּר\maqqaf בַּ֛יִת וְעִ֥יר אֲחֻזָּת֖וֹ בַּיֹּבֵ֑ל כִּ֣י בָתֵּ֞י עָרֵ֣י הַלְוִיִּ֗ם הִ֚וא אֲחֻזָּתָ֔ם בְּת֖וֹךְ בְּנֵ֥י יִשְׂרָאֵֽל׃}
{וּדְיִפְרוֹק מִן לֵיוָאֵי וְיִפְּקוּן זְבִינֵי בֵיתָא וְקִרְוֵי אַחְסָנְתֵיהּ בְּיוֹבֵילָא אֲרֵי בָתֵּי קִרְוֵי לֵיוָאֵי אִנִּין אַחְסָנַתְהוֹן בְּגוֹ בְּנֵי יִשְׂרָאֵל׃}
{And if a man purchase of the Levites, then the house that was sold in the city of his possession, shall go out in the jubilee; for the houses of the cities of the Levites are their possession among the children of Israel.}{\arabic{verse}}
\rashi{\rashiDH{ואשר יגאל מן הלוים.} ואם יקנה בית או עיר מהם. ויצא ביבל. אותו ממכר של בית, או של עיר, וישוב ללוי שמכרו, ולא יהיה חלוט כשאר בתי ערי חומה של ישראל, וגאולה זו, לשון מכירה. דבר אחר לפי שנאמר גאולת עולם תהיה ללוים, יכול לא דבר הכתוב אלא בלוקח ישראל שקנה בית בערי הלוים, אבל לוי שקנה מלוי יהיה חלוט, ת״ל ואשר יגאל מן הלוים, אף הגואל מיד לוי גואל גאולת עולם׃\quad \rashiDH{ויצא ממכר בית.} הרי זו מצוה אחרת, ואם לא גאלה, יוצאה ביובל ואינו נחלט לסוף שנה כבית של ישראל׃\quad \rashiDH{כי בתי ערי הלוים הוא אחזתם.} לא היה להם נחלת שדות וכרמים, אלא ערים לשבת ומגרשיהם, לפיכך הם להם במקום שדות, ויש להם גאולה כשדות, כדי שלא יופקע נחלתם מהם׃}
\threeverse{\arabic{verse}}%Leviticus25:34
{וּֽשְׂדֵ֛ה מִגְרַ֥שׁ עָרֵיהֶ֖ם לֹ֣א יִמָּכֵ֑ר כִּֽי\maqqaf אֲחֻזַּ֥ת עוֹלָ֛ם ה֖וּא לָהֶֽם׃ \setuma }
{וַחֲקַל רְוַח קִרְוֵיהוֹן לָא יִזְדַּבַּן אֲרֵי אַחְסָנַת עָלַם הוּא לְהוֹן׃}
{But the fields of the open land about their cities may not be sold; for that is their perpetual possession.}{\arabic{verse}}
\rashi{\rashiDH{ושדה מגרש עריהם לא ימכר.}מֶכֶר גזבר, שאם הקדיש בן לוי את שדהו ולא גאלה ומכרה גזבר אינה יוצאה לכהנים ביובל, כמו שנאמר בישראל וְאִם מָכַר אֶת הַשָּׂדֶה לְאִישׁ אַחֵר לֹא יִגָּאֵל עוֹד (ויקרא כז, כ), אבל בן לוי גואל לעולם׃}
\newseder{21}
\threeverse{\seder{כא}}%Leviticus25:35
{וְכִֽי\maqqaf יָמ֣וּךְ אָחִ֔יךָ וּמָ֥טָה יָד֖וֹ עִמָּ֑ךְ וְהֶֽחֱזַ֣קְתָּ בּ֔וֹ גֵּ֧ר וְתוֹשָׁ֛ב וָחַ֖י עִמָּֽךְ׃}
{וַאֲרֵי יִתְמַסְכַּן אֲחוּךְ וּתְמוּט יְדֵיהּ עִמָּךְ וְתַתְקֵיף בֵּיהּ יְדוּר וְיִתּוֹתַב וְיֵיחֵי עִמָּךְ׃}
{And if thy brother be waxen poor, and his means fail with thee; then thou shalt uphold him: as a stranger and a settler shall he live with thee.}{\arabic{verse}}
\rashi{\rashiDH{והחזקת בו.} אל תניחהו שירד ויפול, ויהיה קשה להקימו, אלא חזקהו משעת מוטת היד. למה זה דומה, למשאוי שעל החמור, עודהו על החמור, אחד תופס בו ומעמידו, נפל לארץ, חמשה אין מעמידין אותו׃\quad \rashiDH{גר ותושב.} אף אם הוא גר או תושב, ואיזהו תושב כל שקבל עליו שלא לעבוד עבודת אלילים ואוכל נבלות׃}
\threeverse{\arabic{verse}}%Leviticus25:36
{אַל\maqqaf תִּקַּ֤ח מֵֽאִתּוֹ֙ נֶ֣שֶׁךְ וְתַרְבִּ֔ית וְיָרֵ֖אתָ מֵֽאֱלֹהֶ֑יךָ וְחֵ֥י אָחִ֖יךָ עִמָּֽךְ׃}
{לָא תִסַּב מִנֵּיהּ חִיבוּלְיָא וְרִבִּיתָא וְתִדְחַל מֵאֱלָהָךְ וְיֵיחֵי אֲחוּךְ עִמָּךְ׃}
{Take thou no interest of him or increase; but fear thy God; that thy brother may live with thee.}{\arabic{verse}}
\rashi{\rashiDH{נשך ותרבית.} חד שווינהו רבנן, ולעבור עליו בשני לאוין (ב״מ ס׃)׃\quad \rashiDH{ויראת מאלהיך.} לפי שדעתו של אדם נמשכת אחר הרבית, וקשה לפרוש הימנו, ומורה לעצמו היתר בשביל מעותיו שהיו בטלות אצלו, הוצרך לומר ויראת מאלהיך, או התולה מעותיו בנכרי כדי להלוותם לישראל ברבית, הרי זה דבר המסור ללבו של אדם ומחשבתו, לכך הוצרך לומר, ויראת מאלהיך (ב״מ נח׃)׃}
\threeverse{\arabic{verse}}%Leviticus25:37
{אֶ֨ת\maqqaf כַּסְפְּךָ֔ לֹֽא\maqqaf תִתֵּ֥ן ל֖וֹ בְּנֶ֑שֶׁךְ וּבְמַרְבִּ֖ית לֹא\maqqaf תִתֵּ֥ן אׇכְלֶֽךָ׃}
{יָת כַּסְפָּךְ לָא תִתֵּין לֵיהּ בְּחִיבוּלְיָא וּבְרִבִּיתָא לָא תִתֵּין מֵיכְלָךְ׃}
{Thou shalt not give him thy money upon interest, nor give him thy victuals for increase.}{\arabic{verse}}
\threeverse{\arabic{verse}}%Leviticus25:38
{אֲנִ֗י יְהֹוָה֙ אֱלֹ֣הֵיכֶ֔ם אֲשֶׁר\maqqaf הוֹצֵ֥אתִי אֶתְכֶ֖ם מֵאֶ֣רֶץ מִצְרָ֑יִם לָתֵ֤ת לָכֶם֙ אֶת\maqqaf אֶ֣רֶץ כְּנַ֔עַן לִהְי֥וֹת לָכֶ֖ם לֵאלֹהִֽים׃ \setuma }
{אֲנָא יְיָ אֱלָהֲכוֹן דְּאַפֵּיקִית יָתְכוֹן מֵאַרְעָא דְּמִצְרָֽיִם לְמִתַּן לְכוֹן יָת אַרְעָא דִּכְנַֽעַן לְמִהְוֵי לְכוֹן לֶאֱלָהּ׃}
{I am the \lord\space your God, who brought you forth out of the land of Egypt, to give you the land of Canaan, to be your God.}{\arabic{verse}}
\rashi{\rashiDH{אשר הוצאתי וגו׳.} והבחנתי בין בכור לשאינו בכור, אף אני יודע ונפרע מן המלוה מעות לישראל ברבית, ואומר של נכרי הם. דבר אחר אשר הוצאתי אתכם מארץ מצרים על מנת שתקבלו עליכם מצותי, אפילו הן כבדות עליכם׃\quad \rashiDH{לתת לכם את ארץ כנען.} בשכר שתקבלו מצותי׃\quad \rashiDH{להיות לכם לאלהים.} שכל הַדָּר בארץ ישראל, אני לו לאלהים, וכל היוצא ממנה, כעובד עבודת אלילים (ת״כ פרשתא ה, ד  כתובות קי׃)׃}
\aliyacounter{ששי}
\threeverse{\aliya{ששי\newline (רביעי)}}%Leviticus25:39
{וְכִֽי\maqqaf יָמ֥וּךְ אָחִ֛יךָ עִמָּ֖ךְ וְנִמְכַּר\maqqaf לָ֑ךְ לֹא\maqqaf תַעֲבֹ֥ד בּ֖וֹ עֲבֹ֥דַת עָֽבֶד׃}
{וַאֲרֵי יִתְמַסְכַּן אֲחוּךְ עִמָּךְ וְיִזְדַּבַּן לָךְ לָא תִפְלַח בֵּיהּ פּוּלְחַן עַבְדִּין׃}
{And if thy brother be waxen poor with thee, and sell himself unto thee, thou shalt not make him to serve as a bondservant.}{\arabic{verse}}
\rashi{\rashiDH{עבודת עבד.} עבודה של גנאי שיהא ניכר בה כעבד, שלא יוליך כליו אחריו לבית המרחץ, ולא ינעול לו מנעליו׃}
\threeverse{\arabic{verse}}%Leviticus25:40
{כְּשָׂכִ֥יר כְּתוֹשָׁ֖ב יִהְיֶ֣ה עִמָּ֑ךְ עַד\maqqaf שְׁנַ֥ת הַיֹּבֵ֖ל יַעֲבֹ֥ד עִמָּֽךְ׃}
{כַּאֲגִירָא כְּתוֹתָבָא יְהֵי עִמָּךְ עַד שַׁתָּא דְּיוֹבֵילָא יִפְלַח עִמָּךְ׃}
{As a hired servant, and as a settler, he shall be with thee; he shall serve with thee unto the year of jubilee.}{\arabic{verse}}
\rashi{\rashiDH{כשכיר כתושב.} עבודת קרקע, ומלאכת אומנות, כשאר שכירים התנהג בו׃\quad \rashiDH{עד שנת היבל.} אם פגע בו יובל לפני שש שנים היובל מוציאו׃}
\threeverse{\arabic{verse}}%Leviticus25:41
{וְיָצָא֙ מֵֽעִמָּ֔ךְ ה֖וּא וּבָנָ֣יו עִמּ֑וֹ וְשָׁב֙ אֶל\maqqaf מִשְׁפַּחְתּ֔וֹ וְאֶל\maqqaf אֲחֻזַּ֥ת אֲבֹתָ֖יו יָשֽׁוּב׃}
{וְיִפּוֹק מֵעִמָּךְ הוּא וּבְנוֹהִי עִמֵּיהּ וִיתוּב לְזַרְעִיתֵיהּ וּלְאַחְסָנַת אֲבָהָתוֹהִי יְתוּב׃}
{Then shall he go out from thee, he and his children with him, and shall return unto his own family, and unto the possession of his fathers shall he return.}{\arabic{verse}}
\rashi{\rashiDH{הוא ובניו עמו.} אמר רבי שמעון אם הוא נמכר בניו מי מכרן, אלא מכאן, שרבו חייב במזונות בניו (ת״כ פרק ז, ג  קידושין כב.)׃\quad \rashiDH{ואל אחזת אבותיו.}אל כבוד אבותיו, ואין לזלזלו בכך (מכות יג.)׃\quad \rashiDH{אחזת.} חזקת׃}
\threeverse{\arabic{verse}}%Leviticus25:42
{כִּֽי\maqqaf עֲבָדַ֣י הֵ֔ם אֲשֶׁר\maqqaf הוֹצֵ֥אתִי אֹתָ֖ם מֵאֶ֣רֶץ מִצְרָ֑יִם לֹ֥א יִמָּכְר֖וּ מִמְכֶּ֥רֶת עָֽבֶד׃}
{אֲרֵי עַבְדַּי אִנּוּן דְּאַפֵּיקִית יָתְהוֹן מֵאַרְעָא דְּמִצְרָֽיִם לָא יִזְדַּבְּנוּן זִבּוּן עַבְדִּין׃}
{For they are My servants, whom I brought forth out of the land of Egypt; they shall not be sold as bondmen.}{\arabic{verse}}
\rashi{\rashiDH{כי עבדי הם.} שטרי קודם (ת״כ פרשתא ו, א)׃\quad \rashiDH{לא ימכרו ממכרת עבד.} בהכרזה, כאן יש עבד למכור, ולא יעמידנו על אבן הַלֶּקַח׃}
\threeverse{\arabic{verse}}%Leviticus25:43
{לֹא\maqqaf תִרְדֶּ֥ה ב֖וֹ בְּפָ֑רֶךְ וְיָרֵ֖אתָ מֵאֱלֹהֶֽיךָ׃}
{לָא תִפְלַח בֵּיהּ בְּקַשְׁיוּ וְתִדְחַל מֵאֱלָהָךְ׃}
{Thou shalt not rule over him with rigour; but shalt fear thy God.}{\arabic{verse}}
\rashi{\rashiDH{לא תרדה בו בפרך.} מלאכה שלא לצורך, כדי לענותו, אל תאמר לו הָחֵם לי את הכוס הזה, והוא אינו צריך, עֲדֹר תחת הגפן, עד שאבוא, שמא תאמר אין מכיר בדבר אם לצורך אם לאו ואומר אני לו שהוא לצורך, הרי הדבר הזה מסור ללב, לכך נאמר ויראת׃}
\threeverse{\arabic{verse}}%Leviticus25:44
{וְעַבְדְּךָ֥ וַאֲמָתְךָ֖ אֲשֶׁ֣ר יִהְיוּ\maqqaf לָ֑ךְ מֵאֵ֣ת הַגּוֹיִ֗ם אֲשֶׁר֙ סְבִיבֹ֣תֵיכֶ֔ם מֵהֶ֥ם תִּקְנ֖וּ עֶ֥בֶד וְאָמָֽה׃}
{וְעַבְדָּךְ וְאַמְתָּךְ דִּיהוֹן לָךְ מִן עַמְמַיָּא דִּבְסַחְרָנֵיכוֹן מִנְּהוֹן תִּקְנוֹן עַבְדִּין וְאַמְהָן׃}
{And as for thy bondmen, and thy bondmaids, whom thou mayest have: of the nations that are round about you, of them shall ye buy bondmen and bondmaids.}{\arabic{verse}}
\rashi{\rashiDH{ועבדך ואמתך אשר יהיו לך.} אם תאמר אם כן במה אשתמש, בעבדי איני מושל, בז׳ אומות איני נוחל, שהרי הזהרתני לֹא תְחַיֶּה כָּל נְשָׁמָה (דברים כ, טז), אלא מי ישמשני׃\quad \rashiDH{מאת הגוים.} הם יהיו לך לעבדים׃\quad \rashiDH{אשר סביבתיכם.} ולא שבתוך גבול ארצכם, שהרי בהם אמרתי, לא תחיה כל נשמה׃}
\threeverse{\arabic{verse}}%Leviticus25:45
{וְ֠גַ֠ם מִבְּנֵ֨י הַתּוֹשָׁבִ֜ים הַגָּרִ֤ים עִמָּכֶם֙ מֵהֶ֣ם תִּקְנ֔וּ וּמִמִּשְׁפַּחְתָּם֙ אֲשֶׁ֣ר עִמָּכֶ֔ם אֲשֶׁ֥ר הוֹלִ֖ידוּ בְּאַרְצְכֶ֑ם וְהָי֥וּ לָכֶ֖ם לַֽאֲחֻזָּֽה׃}
{וְאַף מִבְּנֵי תּוֹתָבַיָּא עַרְלַיָּא דְּדָיְירִין עִמְּכוֹן מִנְּהוֹן תִּקְנוֹן וּמִזַּרְעִיתְהוֹן דְּעִמְּכוֹן דְּאִתְיַלַּדוּ בַּאֲרַעְכוֹן וִיהוֹן לְכוֹן לְאַחְסָנָא׃}
{Moreover of the children of the strangers that do sojourn among you, of them may ye buy, and of their families that are with you, which they have begotten in your land; and they may be your possession.}{\arabic{verse}}
\rashi{\rashiDH{וגם מבני התושבים.} שבאו מסביבותיכם לישא נשים בארצכם וילדו להם, הבן הולך אחר האב, ואינו בכלל לא תחיה, אלא אתה מותר לקנותו בעבד (קידושין סז׃)׃\quad \rashiDH{מהם תקנו.} אותם תקנו׃}
\threeverse{\arabic{verse}}%Leviticus25:46
{וְהִתְנַחַלְתֶּ֨ם אֹתָ֜ם לִבְנֵיכֶ֤ם אַחֲרֵיכֶם֙ לָרֶ֣שֶׁת אֲחֻזָּ֔ה לְעֹלָ֖ם בָּהֶ֣ם תַּעֲבֹ֑דוּ וּבְאַ֨חֵיכֶ֤ם בְּנֵֽי\maqqaf יִשְׂרָאֵל֙ אִ֣ישׁ בְּאָחִ֔יו לֹא\maqqaf תִרְדֶּ֥ה ב֖וֹ בְּפָֽרֶךְ׃ \setuma }
{וְתַחְסְנוּן יָתְהוֹן לִבְנֵיכוֹן בָּתְרֵיכוֹן לִירוּתַּת אַחְסָנָא לְעָלַם בְּהוֹן תִּפְלְחוּן וּבַאֲחֵיכוֹן בְּנֵי יִשְׂרָאֵל גְּבַר בַּאֲחוּהִי לָא תִפְלַח בֵּיהּ בְּקַשְׁיוּ׃}
{And ye may make them an inheritance for your children after you, to hold for a possession: of them may ye take your bondmen for ever; but over your brethren the children of Israel ye shall not rule, one over another, with rigour.}{\arabic{verse}}
\rashi{\rashiDH{והתנחלתם אתם לבניכם.} החזיקו בהם לנחלה לצורך בניכם אחריכם ולא יתכן לפרש הנחילום לבניכם, שאם כן היה לו לכתוב והנחלתם אותם לבניכם׃ \rashiDH{והתנחלתם.} כמו והתחזקתם׃\quad \rashiDH{איש באחיו.} להביא נשיא בעמו ומלך במשרתיו שלא לרדות בפרך׃}
\aliyacounter{שביעי}
\threeverse{\aliya{שביעי}}%Leviticus25:47
{וְכִ֣י תַשִּׂ֗יג יַ֣ד גֵּ֤ר וְתוֹשָׁב֙ עִמָּ֔ךְ וּמָ֥ךְ אָחִ֖יךָ עִמּ֑וֹ וְנִמְכַּ֗ר לְגֵ֤ר תּוֹשָׁב֙ עִמָּ֔ךְ א֥וֹ לְעֵ֖קֶר מִשְׁפַּ֥חַת גֵּֽר׃}
{וַאֲרֵי תַדְבֵּיק יַד עֲרַל וְתוֹתָב עִמָּךְ וְיִתְמַסְכַּן אֲחוּךְ עִמֵּיהּ וְיִזְדַּבַּן לַעֲרַל תּוֹתָב דְּעִמָּךְ אוֹ לְאַרְמַאי מִזַּרְעִית גִּיּוֹרָא׃}
{And if a stranger who is a settler with thee be waxen rich, and thy brother be waxen poor beside him, and sell himself unto the stranger who is a settler with thee, or to the offshoot of a stranger’s family,}{\arabic{verse}}
\rashi{\rashiDH{יד גר ותושב.} גר והוא תושב, כתרגומו עֲרַל תוֹתָב, וסופו מוכיח ונמכר לגר תושב׃\quad \rashiDH{וכי תשיג יד גר ותושב עמך.} מי גרם לו שיעשיר, דבוקו עמך׃\quad \rashiDH{ומך אחיך עמו.} מי גרם לו שימוך, דבוקו עמו, על ידי שלמד ממעשיו׃\quad \rashiDH{משפחת גר.} זהו עכו״ם, כשהוא אומר לעקר, זה הנמכר לעבודת אלילים עצמה (שם כ. ב״מ עא.), להיות לה שמש, ולא לאלהות, אלא לחטוב עצים, ולשאוב מים׃}
\threeverse{\arabic{verse}}%Leviticus25:48
{אַחֲרֵ֣י נִמְכַּ֔ר גְּאֻלָּ֖ה תִּהְיֶה\maqqaf לּ֑וֹ אֶחָ֥ד מֵאֶחָ֖יו יִגְאָלֶֽנּוּ׃}
{בָּתַר דְּאִזְדַּבַּן פּוּרְקָנָא יְהֵי לֵיהּ חַד מֵאֲחוֹהִי יִפְרְקִנֵּיהּ׃}
{after that he is sold he may be redeemed; one of his brethren may redeem him;}{\arabic{verse}}
\rashi{\rashiDH{גאולה תהי׳ לו.} מיד, אל תניחהו שֶׁיִּטָּמַע׃}
\threeverse{\arabic{verse}}%Leviticus25:49
{אוֹ\maqqaf דֹד֞וֹ א֤וֹ בֶן\maqqaf דֹּדוֹ֙ יִגְאָלֶ֔נּוּ אֽוֹ\maqqaf מִשְּׁאֵ֧ר בְּשָׂר֛וֹ מִמִּשְׁפַּחְתּ֖וֹ יִגְאָלֶ֑נּוּ אֽוֹ\maqqaf הִשִּׂ֥יגָה יָד֖וֹ וְנִגְאָֽל׃}
{אוֹ אַחְבּוּהִי אוֹ בַר אַחְבּוּהִי יִפְרְקִנֵּיהּ אוֹ מִקָּרִיב בִּשְׂרֵיהּ מִזַּרְעִיתֵיהּ יִפְרְקִנֵּיהּ אוֹ דְּתַדְבֵּיק יְדֵיהּ וְיִתְפְּרֵיק׃}
{or his uncle, or his uncle’s son, may redeem him, or any that is nigh of kin unto him of his family may redeem him; or if he be waxen rich, he may redeem himself.}{\arabic{verse}}
\threeverse{\arabic{verse}}%Leviticus25:50
{וְחִשַּׁב֙ עִם\maqqaf קֹנֵ֔הוּ מִשְּׁנַת֙ הִמָּ֣כְרוֹ ל֔וֹ עַ֖ד שְׁנַ֣ת הַיֹּבֵ֑ל וְהָיָ֞ה כֶּ֤סֶף מִמְכָּרוֹ֙ בְּמִסְפַּ֣ר שָׁנִ֔ים כִּימֵ֥י שָׂכִ֖יר יִהְיֶ֥ה עִמּֽוֹ׃}
{וִיחַשֵּׁיב עִם זָבְנֵיהּ מִשַּׁתָּא דְּאִזְדַּבַּן לֵיהּ עַד שַׁתָּא דְּיוֹבֵילָא וִיהֵי כְּסַף זְבִינוֹהִי בְּמִנְיַן שְׁנַיָּא כְּיוֹמֵי אֲגִירָא יְהֵי עִמֵּיהּ׃}
{And he shall reckon with him that bought him from the year that he sold himself to him unto the year of jubilee; and the price of his sale shall be according unto the number of years; according to the time of a hired servant shall he be with him.}{\arabic{verse}}
\rashi{\rashiDH{עד שנת היובל.} שהרי כל עצמו לא קנאו אלא לעובדו עד היובל, שהרי ביובל יֵצֵא, כמו שנאמר למטה, וְיָצָא בִּשְׁנַת הַיֹּבֵל, ובנכרי שתחת ידך הכתוב מדבר, ואף על פי כן לא תבא עליו בעקיפין, מפני חלול השם (ב״ק קיג.), אלא כשבא ליגאל ידקדק בחשבון לפי המגיע בכל שנה ושנה ינכה לו הנכרי מן דמיו, אם היו עשרים שנה משנמכר עד היובל וקנאו בעשרים מנה, נמצא שקנה הנכרי עבודת שנה במנה, ואם שהה זה אצלו חמש שנים ובא ליגאל ינכה לו חמשה מנים, ויתן לו העבד ט״ו מנים, וזהו והיה כסף ממכרו במספר שנים׃\quad \rashiDH{כימי שכיר יהיה עמו.} חשבון המגיע לכל שנה ושנה יחשוב כאלו נשכר עמו כל שנה במנה, וינכה לו׃}
\threeverse{\arabic{verse}}%Leviticus25:51
{אִם\maqqaf ע֥וֹד רַבּ֖וֹת בַּשָּׁנִ֑ים לְפִיהֶן֙ יָשִׁ֣יב גְּאֻלָּת֔וֹ מִכֶּ֖סֶף מִקְנָתֽוֹ׃}
{אִם עוֹד סַגְיוּת בִּשְׁנַיָּא לְפוֹמְהוֹן יָתִיב פּוּרְקָנֵיהּ מִכְּסַף זְבִינוֹהִי׃}
{If there be yet many years, according unto them he shall give back the price of his redemption out of the money that he was bought for.}{\arabic{verse}}
\rashi{\rashiDH{אם עוד רבות בשנים.} עד היובל׃\quad \rashiDH{לפיהן.} הכל כמו שפירשתי׃}
\threeverse{\arabic{verse}}%Leviticus25:52
{וְאִם\maqqaf מְעַ֞ט נִשְׁאַ֧ר בַּשָּׁנִ֛ים עַד\maqqaf שְׁנַ֥ת הַיֹּבֵ֖ל וְחִשַּׁב\maqqaf ל֑וֹ כְּפִ֣י שָׁנָ֔יו יָשִׁ֖יב אֶת\maqqaf גְּאֻלָּתֽוֹ׃}
{וְאִם זְעֵיר אִשְׁתְּאַר בִּשְׁנַיָּא עַד שַׁתָּא דְּיוֹבֵילָא וִיחַשֵּׁיב לֵיהּ כְּפוֹם שְׁנוֹהִי יָתִיב יָת פּוּרְקָנֵיהּ׃}
{And if there remain but few years unto the year of jubilee, then he shall reckon with him; according unto his years shall he give back the price of his redemption.}{\arabic{verse}}
\threeverse{\arabic{verse}}%Leviticus25:53
{כִּשְׂכִ֥יר שָׁנָ֛ה בְּשָׁנָ֖ה יִהְיֶ֣ה עִמּ֑וֹ לֹֽא\maqqaf יִרְדֶּ֥נּֽוּ בְּפֶ֖רֶךְ לְעֵינֶֽיךָ׃}
{כַּאֲגִיר שְׁנָא בִּשְׁנָא יְהֵי עִמֵּיהּ לָא יִפְלַח בֵּיהּ בְּקַשְׁיוּ לְעֵינָךְ׃}
{As a servant hired year by year shall he be with him; he shall not rule with rigour over him in thy sight.}{\arabic{verse}}
\rashi{\rashiDH{לא ירדנו בפרך לעיניך.} כלומר ואתה רואה׃}
\threeverse{\arabic{verse}}%Leviticus25:54
{וְאִם\maqqaf לֹ֥א יִגָּאֵ֖ל בְּאֵ֑לֶּה וְיָצָא֙ בִּשְׁנַ֣ת הַיֹּבֵ֔ל ה֖וּא וּבָנָ֥יו עִמּֽוֹ׃}
{וְאִם לָא יִתְפְּרֵיק בְּאִלֵּין וְיִפּוֹק בְּשַׁתָּא דְּיוֹבֵילָא הוּא וּבְנוֹהִי עִמֵּיהּ׃}
{And if he be not redeemed by any of these means, then he shall go out in the year of jubilee, he, and his children with him.}{\arabic{verse}}
\rashi{\rashiDH{ואם לא יגאל באלה.} באלה הוא נגאל, ואינו נגאל בשש (קידושין טו׃)׃\quad \rashiDH{(הוא ובניו עמו.} הנכרי חייב במזונות בניו (קידושין כב.))}
\threeverse{\aliya{מפטיר}}%Leviticus25:55
{כִּֽי\maqqaf לִ֤י בְנֵֽי\maqqaf יִשְׂרָאֵל֙ עֲבָדִ֔ים עֲבָדַ֣י הֵ֔ם אֲשֶׁר\maqqaf הוֹצֵ֥אתִי אוֹתָ֖ם מֵאֶ֣רֶץ מִצְרָ֑יִם אֲנִ֖י יְהֹוָ֥ה אֱלֹהֵיכֶֽם׃}
{אֲרֵי דִּילִי בְנֵי יִשְׂרָאֵל עַבְדִּין עַבְדַּי אִנּוּן דְּאַפֵּיקִית יָתְהוֹן מֵאַרְעָא דְּמִצְרָֽיִם אֲנָא יְיָ אֱלָהֲכוֹן׃}
{For unto Me the children of Israel are servants; they are My servants whom I brought forth out of the land of Egypt: I am the \lord\space your God.}{\arabic{verse}}
\rashi{\rashiDH{כי לי בני ישראל עבדים.} שטרי קודם׃\quad \rashiDH{אני ה׳ אלהיכם.} כל המשעבדן מלמטה, כאלו משעבדן מלמעלה׃}
\newperek
\threeverse{\Roman{chap}}%Leviticus26:1
{לֹֽא\maqqaf תַעֲשׂ֨וּ לָכֶ֜ם אֱלִילִ֗ם וּפֶ֤סֶל וּמַצֵּבָה֙ לֹֽא\maqqaf תָקִ֣ימוּ לָכֶ֔ם וְאֶ֣בֶן מַשְׂכִּ֗ית לֹ֤א תִתְּנוּ֙ בְּאַרְצְכֶ֔ם לְהִֽשְׁתַּחֲוֺ֖ת עָלֶ֑יהָ כִּ֛י אֲנִ֥י יְהֹוָ֖ה אֱלֹהֵיכֶֽם׃}
{לָא תַעְבְּדוּן לְכוֹן טָעֲוָן וְצֵילַם וְקָמָא לָא תְקִימוּן לְכוֹן וְאֶבֶן סִגְדָּא לָא תִתְּנוּן בַּאֲרַעְכוֹן לְמִסְגַּד עֲלַהּ אֲרֵי אֲנָא יְיָ אֱלָהֲכוֹן׃}
{Ye shall make you no idols, neither shall ye rear you up a graven image, or a pillar, neither shall ye place any figured stone in your land, to bow down unto it; for I am the \lord\space your God.}{\Roman{chap}}
\rashi{\rashiDH{לא תעשו לכם אלילם.} כנגד זה הנמכר לנכרי, שלא יאמר הואיל ורבי מגלה עריות אף אני כמותו, הואיל ורבי עובד עבודת אלילים אף אני כמותו, הואיל ורבי מחלל שבת אף אני כמותו, לכך נאמרו מקראות הללו (ת״כ פרק ט, ו). ואף הפרשיות הללו נאמרו על הסדר, בתחלה הזהיר על השביעית, ואם חמד ממון ונחשד על השביעית, סופו למכור מטלטליו, לכך סמך לה, וכי תמכור ממכר, (מה כתיב ביה או קנה מיד וגו׳ דבר הנקנה מיד ליד)לא חזר בו, סוף מוכר אחוזתו, לא חזר בו, סוף מוכר את ביתו, לא חזר בו, סוף לוה ברבית, כל אלו האחרונות קשות מן הראשונות, לא חזר בו, סוף מוכר את עצמו, לא חזר בו, לא דיו לישראל אלא אפילו לנכרי (קידושין כ.)׃\quad \rashiDH{ואבן משכית.} לשון כסוי, כמו וְשַׂכֹּתִי כַפִּי (שמות לג, כב), שמכסין הקרקע ברצפת אבנים׃\quad \rashiDH{להשתחות עליה.} אפילו לשמים, לפי שהשתחואה בפשוט ידים ורגלים היא, ואסרה תורה לעשות כן, חוץ מן המקדש (מגילה כב.)׃}
\threeverse{\aliya{\Hebrewnumeral{57}}}%Leviticus26:2
{אֶת\maqqaf שַׁבְּתֹתַ֣י תִּשְׁמֹ֔רוּ וּמִקְדָּשִׁ֖י תִּירָ֑אוּ אֲנִ֖י יְהֹוָֽה׃ \petucha }
{יָת יוֹמֵי שַׁבַּיָּא דִּילִי תִטְּרוּן וּלְבֵית מַקְדְּשִׁי תְּהוֹן דָּחֲלִין אֲנָא יְיָ׃}
{Ye shall keep My sabbaths, and reverence My sanctuary: I am the \lord.}{\arabic{verse}}
\rashi{\rashiDH{אני ה׳.} נאמן לשלם שכר׃}
\engnote{The Haftarah is Jeremiah 32:6\verserangechar 32:27 on page \pageref{haft_32}. }
\aliyacounter{ראשון}
\newparsha{בחקתי}
\newseder{22}
\threeverse{\aliya{בחקתי}\newline\vspace{-4pt}\newline\seder{כב}}%Leviticus26:3
{אִם\maqqaf בְּחֻקֹּתַ֖י תֵּלֵ֑כוּ וְאֶת\maqqaf מִצְוֺתַ֣י תִּשְׁמְר֔וּ וַעֲשִׂיתֶ֖ם אֹתָֽם׃}
{אִם בִּקְיָמַי תְּהָכוּן וְיָת פִּקּוֹדַי תִּטְּרוּן וְתַעְבְּדְוּן יָתְהוֹן׃}
{If ye walk in My statutes, and keep My commandments, and do them;}{\arabic{verse}}
\rashi{\rashiDH{אם בחקתי תלכו.} יכול זה קיום המצות, כשהוא אומר ואת מצותי תשמרו הרי קיום המצות אמור, הא מה אני מקיים אם בחקותי, תלכו שתהיו עמלים בתורה׃\quad \rashiDH{ואת מצותי תשמרו.} הוו עמלים בתורה על מנת לשמור ולקיים, כמו שנאמר (דברים ה, א) וּלְמַדְתֶּם אֹתָם וּשְׁמַרְתֶּם לַעֲשׂתָם׃}
\threeverse{\arabic{verse}}%Leviticus26:4
{וְנָתַתִּ֥י גִשְׁמֵיכֶ֖ם בְּעִתָּ֑ם וְנָתְנָ֤ה הָאָ֙רֶץ֙ יְבוּלָ֔הּ וְעֵ֥ץ הַשָּׂדֶ֖ה יִתֵּ֥ן פִּרְיֽוֹ׃}
{וְאֶתֵּין מִטְרֵיכוֹן בְּעִדָּנְהוֹן וְתִתֵּין אַרְעָא עֲלַלְתַּהּ וְאִילָן חַקְלָא יִתֵּין אִבֵּיהּ׃}
{then I will give your rains in their season, and the land shall yield her produce, and the trees of the field shall yield their fruit.}{\arabic{verse}}
\rashi{\rashiDH{בעתם.} בשעה שאין דרך בני אדם לצאת, כגון (בלילי רביעית. רש״י ישן) בלילי שבתות (תענית כג.  ת״כ פרק א, א)׃\quad \rashiDH{ועץ השדה.} הן אילני סרק, ועתידין לעשות פירות (ת״כ שם ו)׃}
\threeverse{\arabic{verse}}%Leviticus26:5
{וְהִשִּׂ֨יג לָכֶ֥ם דַּ֙יִשׁ֙ אֶת\maqqaf בָּצִ֔יר וּבָצִ֖יר יַשִּׂ֣יג אֶת\maqqaf זָ֑רַע וַאֲכַלְתֶּ֤ם לַחְמְכֶם֙ לָשֹׂ֔בַע וִֽישַׁבְתֶּ֥ם לָבֶ֖טַח בְּאַרְצְכֶֽם׃}
{וִיעָרַע לְכוֹן דְּיָשָׁא לִקְטָפָא וּקְטָפָא יְעָרַע לְאַפּוֹקֵי בַר זַרְעָא וְתֵיכְלוּן לַחְמְכוֹן לְמִשְׂבַּע וְתִתְּבוּן לְרוּחְצָן בַּאֲרַעְכוֹן׃}
{And your threshing shall reach unto the vintage, and the vintage shall reach unto the sowing time; and ye shall eat your bread until ye have enough, and dwell in your land safely.}{\arabic{verse}}
\rashi{\rashiDH{והשיג לכם דיש את בציר.} שיהא הדיש מרובה, ואתם עסוקים בו עד הבציר, ובבציר תעסקו עד שעת הזרע׃\quad \rashiDH{ואכלתם לחמכם לשבע.} אוכל קמעא, והוא מתברך במעיו (שם ז)׃}
\aliyacounter{שני}
\threeverse{\aliya{שני}}%Leviticus26:6
{וְנָתַתִּ֤י שָׁלוֹם֙ בָּאָ֔רֶץ וּשְׁכַבְתֶּ֖ם וְאֵ֣ין מַחֲרִ֑יד וְהִשְׁבַּתִּ֞י חַיָּ֤ה רָעָה֙ מִן\maqqaf הָאָ֔רֶץ וְחֶ֖רֶב לֹא\maqqaf תַעֲבֹ֥ר בְּאַרְצְכֶֽם׃}
{וְאֶתֵּין שְׁלָמָא בְּאַרְעָא וְתִשְׁרוֹן וְלֵית דְּמַנִּיד וַאֲבַטֵּיל חַיְתָא בִּשְׁתָּא מִן אַרְעָא וּדְקָטְלִין בְּחַרְבָּא לָא יִעְדּוֹן בַּאֲרַעְכוֹן׃}
{And I will give peace in the land, and ye shall lie down, and none shall make you afraid; and I will cause evil beasts to cease out of the land, neither shall the sword go through your land.}{\arabic{verse}}
\rashi{\rashiDH{ונתתי שלום.} שמא תאמרו הרי מאכל והרי משתה, אם אין שלום אין כלום, תלמוד לומר אחר כל זאת ונתתי שלום בארץ, מכאן שהשלום שקול כנגד הכל, וכן הוא אומר עושה שלום ובורא את הכל׃\quad \rashiDH{וחרב לא תעבר בארצכם.} אין צריך לומר שלא יבאו למלחמה, אלא אפילו לעבור דרך ארצכם ממדינה למדינה (ת״כ פרק ב, ג)׃}
\threeverse{\arabic{verse}}%Leviticus26:7
{וּרְדַפְתֶּ֖ם אֶת\maqqaf אֹיְבֵיכֶ֑ם וְנָפְל֥וּ לִפְנֵיכֶ֖ם לֶחָֽרֶב׃}
{וְתִרְדְּפוּן יָת בַּעֲלֵי דְּבָבֵיכוֹן וְיִפְּלוּן קֳדָמֵיכוֹן לְחַרְבָּא׃}
{And ye shall chase your enemies, and they shall fall before you by the sword.}{\arabic{verse}}
\rashi{\rashiDH{לפניכם לחרב.} איש בחרב רעהו (שם)׃}
\threeverse{\arabic{verse}}%Leviticus26:8
{וְרָדְפ֨וּ מִכֶּ֤ם חֲמִשָּׁה֙ מֵאָ֔ה וּמֵאָ֥ה מִכֶּ֖ם רְבָבָ֣ה יִרְדֹּ֑פוּ וְנָפְל֧וּ אֹיְבֵיכֶ֛ם לִפְנֵיכֶ֖ם לֶחָֽרֶב׃}
{וְיִרְדְּפוּן מִנְּכוֹן חַמְשָׁא לִמְאָה וּמְאָה מִנְּכוֹן לְרִבּוֹתָא יְעָרְקוּן וְיִפְּלוּן בַּעֲלֵי דְּבָבֵיכוֹן קֳדָמֵיכוֹן לְחַרְבָּא׃}
{And five of you shall chase a hundred, and a hundred of you shall chase ten thousand; and your enemies shall fall before you by the sword.}{\arabic{verse}}
\rashi{\rashiDH{ורדפו מכם.} מן החלשים שבכם, ולא מן הגבורים שבכם (שם ד)׃\quad \rashiDH{חמשה מאה ומאה מכם רבבה.} וכי כך הוא החשבון, והלא לא היה צריך לומר אלא ומאה מכם שני אלפים ירדופו, (יש מדקדקים בלשון רש״י שהוא כמו כפל ואריכות וכי כך הוא החשבון והלא לא היה צריך לומר, וכמו כן מדקדקים כל גדולי המפרשים וחשוביהם בלשון רש״י בפרשת בראשית, לא היה צריך להתחיל, ומה טעם פתח, שגם כן כפל ועיין שם בישובם, וכאן נראה לפרש למורי הגאון המופלג מוהר״ר משה חריף נר״ו, שמלת רבבה סובל ב׳ פירושים, פירוש א׳ עשרת אלפים, ועל זה מקשה רש״י וכי כך הוא החשבון, ופירוש ב׳, רבבה מספר מרובה, על שם ריבוי, וזה הפירוש שולל רש״י. באומרו והלא לא היה צריך לומר וכו׳. ודו״ק) אלא אינו דומה מועטין העושין את התורה, למרובים העושין את התורה (שם)׃\quad \rashiDH{ונפלו איביכם וגו׳.} שיהיו נופלין לפניכם, שלא כדרך הארץ׃}
\threeverse{\arabic{verse}}%Leviticus26:9
{וּפָנִ֣יתִי אֲלֵיכֶ֔ם וְהִפְרֵיתִ֣י אֶתְכֶ֔ם וְהִרְבֵּיתִ֖י אֶתְכֶ֑ם וַהֲקִימֹתִ֥י אֶת\maqqaf בְּרִיתִ֖י אִתְּכֶֽם׃}
{וְאֶתְפְּנֵי בְּמֵימְרִי לְאֵיטָבָא לְכוֹן וְאַפֵּישׁ יָתְכוֹן וְאַסְגֵּי יָתְכוֹן וְאַקֵּים יָת קְיָמִי עִמְּכוֹן׃}
{And I will have respect unto you, and make you fruitful, and multiply you; and will establish My covenant with you.}{\arabic{verse}}
\rashi{\rashiDH{ופניתי אליכם.} אפנה מכל עסקי לשלם שכרכם. משל למה הדבר דומה, למלך ששכר פועלים וכו׳, כדאיתא בת״כ (פרק ב, ה)׃\quad \rashiDH{והפריתי אתכם.} בפריה ורביה (שם ה)׃\quad \rashiDH{והרביתי אתכם.} בקומה זקופה (שם)׃\quad \rashiDH{והקימתי את בריתי אתכם.} ברית חדשה, לא כברית הראשונה שהפרתם אותה, אלא ברית חדשה שלא תופר, שנאמר וְכָרַתִּי אֶת בֵּית יִשְׂרָאֵל וְאֶת בֵּית יְהוּדָה בְּרִית חֲדָשָׁה, (ירמיה לא, ל־לא), לֹא כַבְּרִית וגו׳ (ת״כ שם)׃}
\aliyacounter{שלישי}
\threeverse{\aliya{שלישי\newline (חמישי)}}%Leviticus26:10
{וַאֲכַלְתֶּ֥ם יָשָׁ֖ן נוֹשָׁ֑ן וְיָשָׁ֕ן מִפְּנֵ֥י חָדָ֖שׁ תּוֹצִֽיאוּ׃}
{וְתֵיכְלוּן עַתִּיקָא דְּעַתִּיק וְעַתִּיקָא מִן קֳדָם חֲדַתָּא תְּפַנּוֹן׃}
{And ye shall eat old store long kept, and ye shall bring forth the old from before the new.}{\arabic{verse}}
\rashi{\rashiDH{ואכלתם ישן נושן.} הפירות יהיו משתמרין וטובים להתיישן, שיהא ישן הנושן של שלש שנים יפה לאכול משל אשתקד (שם פרק ג, א  ב״ב צא׃)׃\quad \rashiDH{וישן מפני חדש תוציאו.} שיהיו הגרנות מלאות חדש והאוצרות מלאות ישן, וצריכים אתם לפנות האוצרות למקום אחר, לתת החדש לתוכן׃}
\threeverse{\arabic{verse}}%Leviticus26:11
{וְנָתַתִּ֥י מִשְׁכָּנִ֖י בְּתוֹכְכֶ֑ם וְלֹֽא\maqqaf תִגְעַ֥ל נַפְשִׁ֖י אֶתְכֶֽם׃}
{וְאֶתֵּין מַשְׁכְּנִי בֵּינֵיכוֹן וְלָא יְרַחֵיק מֵימְרִי יָתְכוֹן׃}
{And I will set My tabernacle among you, and My soul shall not abhor you.}{\arabic{verse}}
\rashi{\rashiDH{ונתתי משכני}. זה בית המקדש׃\quad \rashiDH{ולא תגעל נפשי.} אין רוחי קצה בכם. כל געילה לשון פליטת דבר הבלוע בדבר, כמו כִּי שָׁם נִגְעַל מָגֵן גִּבּוֹרִים (שמואל־ב א, כא), לא קבל המשיחה, שמושחין מגן של עור בְּחֵלֶב מבושל כדי להחליק מעליו מכת חץ או חנית שלא יקוב העור׃}
\threeverse{\arabic{verse}}%Leviticus26:12
{וְהִתְהַלַּכְתִּי֙ בְּת֣וֹכְכֶ֔ם וְהָיִ֥יתִי לָכֶ֖ם לֵֽאלֹהִ֑ים וְאַתֶּ֖ם תִּהְיוּ\maqqaf לִ֥י לְעָֽם׃}
{וְאַשְׁרֵי שְׁכִינְתִי בֵּינֵיכוֹן וְאֶהְוֵי לְכוֹן לֶאֱלָהּ וְאַתּוּן תְּהוֹן קֳדָמַי לְעַם׃}
{And I will walk among you, and will be your God, and ye shall be My people.}{\arabic{verse}}
\rashi{\rashiDH{והתהלכתי בתוככם.} אטייל עמכם בגן עדן כאחד מכם, ולא תהיו מזדעזעים ממני, יכול לא תיראו ממני, תלמוד לומר והייתי לכם לאלהים׃}
\threeverse{\aliya{ע״כ בחול}}%Leviticus26:13
{אֲנִ֞י יְהֹוָ֣ה אֱלֹֽהֵיכֶ֗ם אֲשֶׁ֨ר הוֹצֵ֤אתִי אֶתְכֶם֙ מֵאֶ֣רֶץ מִצְרַ֔יִם מִֽהְיֹ֥ת לָהֶ֖ם עֲבָדִ֑ים וָאֶשְׁבֹּר֙ מֹטֹ֣ת עֻלְּכֶ֔ם וָאוֹלֵ֥ךְ אֶתְכֶ֖ם קֽוֹמְמִיּֽוּת׃ \petucha }
{אֲנָא יְיָ אֱלָהֲכוֹן דְּאַפֵּיקִית יָתְכוֹן מֵאַרְעָא דְּמִצְרַיִם מִלְּמִהְוֵי לְהוֹן עַבְדִּין וְתַבַּרִית נִיר עַמְמַיָּא מִנְּכוֹן וְדַבַּרִית יָתְכוֹן בְּחֵירוּתָא׃}
{I am the \lord\space your God, who brought you forth out of the land of Egypt, that ye should not be their bondmen; and I have broken the bars of your yoke, and made you go upright.}{\arabic{verse}}
\rashi{\rashiDH{אני ה׳ אלהיכם.} כדאי אני שתאמינו בי שאני יכול לעשות כל אלה, שהרי הוצאתי אתכם מארץ מצרים, ועשיתי לכם נסים גדולים׃\quad \rashiDH{מטת.} כמין יתד בשני ראשי העול המעכבים המוסרה שלא תצא מראש השור ויתיר הקשר, כמו עֲשֵׂה לְךָ מוֹסֵרוֹת וּמֹטוֹת (ירמיה כז, ב), קביליי״א בלע״ז׃\quad \rashiDH{קוממיות.} בקומה זקופה׃}
\threeverse{\arabic{verse}}%Leviticus26:14
{וְאִם\maqqaf לֹ֥א תִשְׁמְע֖וּ לִ֑י וְלֹ֣א תַעֲשׂ֔וּ אֵ֥ת כׇּל\maqqaf הַמִּצְוֺ֖ת הָאֵֽלֶּה׃}
{וְאִם לָא תְקַבְּלוּן לְמֵימְרִי וְלָא תַעְבְּדוּן יָת כָּל פִּקּוֹדַיָּא הָאִלֵּין׃}
{But if ye will not hearken unto Me, and will not do all these commandments;}{\arabic{verse}}
\rashi{\rashiDH{ואם לא תשמעו לי.} להיות עמלים בתורה, ולדעת מדרש חכמים. יכול לקיום המצות, כשהוא אומר ולא תעשו וגו׳ הרי קיום מצות אמור, הא מה אני מקיים ואם לא תשמעו לי, להיות עמלים בתורה. ומה תלמוד לומר לי, אין לי אלא זה המכיר את רבונו ומתכוין למרוד בו, וכן בנמרוד גִּבֹּר צַיִד לִפְנֵי ה׳ (בראשית י, ט), שמכירו ומתכוין למרוד בו, וכן באנשי סדום רָעִים וְחַטָּאִים לַה׳ מְאֹד (שם יג, יג), מכירים את רבונם ומתכוונים למרוד בו׃\quad \rashiDH{ולא תעשו.} משלא תלמדו, לא תעשו, הרי שתי עבירות (ת״כ פרשתא ב, ג)׃}
\threeverse{\arabic{verse}}%Leviticus26:15
{וְאִם\maqqaf בְּחֻקֹּתַ֣י תִּמְאָ֔סוּ וְאִ֥ם אֶת\maqqaf מִשְׁפָּטַ֖י תִּגְעַ֣ל נַפְשְׁכֶ֑ם לְבִלְתִּ֤י עֲשׂוֹת֙ אֶת\maqqaf כׇּל\maqqaf מִצְוֺתַ֔י לְהַפְרְכֶ֖ם אֶת\maqqaf בְּרִיתִֽי׃}
{וְאִם בִּקְיָמַי תְּקוּצוּן וְאִם יָת דִּינַי תְּרַחֵיק נַפְשְׁכוֹן בְּדִיל דְּלָא לְמֶעֱבַד יָת כָּל פִּקּוֹדַי לְאַשְׁנָיוּתְכוֹן יָת קְיָמִי׃}
{and if ye shall reject My statutes, and if your soul abhor Mine ordinances, so that ye will not do all My commandments, but break My covenant;}{\arabic{verse}}
\rashi{\rashiDH{ואם בחקתי תמאסו.} מואס באחרים העושים׃\quad \rashiDH{משפטי תגעל נפשכם.} שונא החכמים׃\quad \rashiDH{לבלתי עשות.} מונע את אחרים מעשות׃\quad \rashiDH{את כל מצותי.} כופר שלא צויתים, לכך נאמר את כל מצותי, ולא נאמר את כל המצות׃\quad \rashiDH{להפרכם את בריתי.} כופר בעיקר. הרי שבע עבירות, הראשונה גוררת השניה, וכן עד השביעית, ואלו הן, לא למד, ולא עשה, מואס באחרים העושים, שונא את החכמים, מונע את האחרים, כופר במצות, כופר בעיקר׃}
\threeverse{\arabic{verse}}%Leviticus26:16
{אַף\maqqaf אֲנִ֞י אֶֽעֱשֶׂה\maqqaf זֹּ֣את לָכֶ֗ם וְהִפְקַדְתִּ֨י עֲלֵיכֶ֤ם בֶּֽהָלָה֙ אֶת\maqqaf הַשַּׁחֶ֣פֶת וְאֶת\maqqaf הַקַּדַּ֔חַת מְכַלּ֥וֹת עֵינַ֖יִם וּמְדִיבֹ֣ת נָ֑פֶשׁ וּזְרַעְתֶּ֤ם לָרִיק֙ זַרְעֲכֶ֔ם וַאֲכָלֻ֖הוּ אֹיְבֵיכֶֽם׃}
{אַף אֲנָא אַעֲבֵיד דָּא לְכוֹן וְאַסְעַר עֲלֵיכוֹן בִּיהוּלְתָּא שַׁחֶפְתָּא וְקַדַּחְתָּא מַחְשְׁכָן עַיְנִין וּמַפְּחָן נְפַשׁ וְתִזְרְעוּן לְרֵיקָנוּ זַרְעֲכוֹן וְיֵיכְלוּנֵּיהּ בַּעֲלֵי דְּבָבֵיכוֹן׃}
{I also will do this unto you: I will appoint terror over you, even consumption and fever, that shall make the eyes to fail, and the soul to languish; and ye shall sow your seed in vain, for your enemies shall eat it.}{\arabic{verse}}
\rashi{\rashiDH{והפקדתי עליכם.} וצויתי עליכם׃\quad \rashiDH{שחפת.} חולי שמשחף את הבשר, אנפולי״ש בלע״ז, (געשוואלען) דומה לנפוח שהוקלה נפיחתו ומראית פניו זעופה׃\quad \rashiDH{קדחת.} חולי שמקדיח את הגוף ומחממו ומבעירו, כמו כִּי אֵשׁ קָדְחָה בְאַפִּי (דברים לב, כב)׃\quad \rashiDH{מכלות עינים ומדיבת נפש.} העינים צופות וכלות לראות שֶׁיֵּקַל וְיֵרָפֵא, וסוף שלא ירפא, וידאבו הנפשות של משפחתו במותו. כל תאוה שאינה באה ותוחלת ממושכה, קרויה כליון עינים׃\quad \rashiDH{וזרעתם לריק.} תזרעו ולא תצמח, ואם תצמח, ואכלוהו אויביכם׃}
\threeverse{\arabic{verse}}%Leviticus26:17
{וְנָתַתִּ֤י פָנַי֙ בָּכֶ֔ם וְנִגַּפְתֶּ֖ם לִפְנֵ֣י אֹיְבֵיכֶ֑ם וְרָד֤וּ בָכֶם֙ שֹֽׂנְאֵיכֶ֔ם וְנַסְתֶּ֖ם וְאֵין\maqqaf רֹדֵ֥ף אֶתְכֶֽם׃}
{וְאֶתֵּין רוּגְזִי בְּכוֹן וְתִתַּבְרוּן קֳדָם בַּעֲלֵי דְּבָבֵיכוֹן וְיִרְדּוֹן בְּכוֹן שָׂנְאֵיכוֹן וְתִעְרְקוּן וְלֵית דְּרָדֵיף יָתְכוֹן׃}
{And I will set My face against you, and ye shall be smitten before your enemies; they that hate you shall rule over you; and ye shall flee when none pursueth you.}{\arabic{verse}}
\rashi{\rashiDH{ונתתי פני.} פנאי שלי, פונה אני מכל עסקי להרע לכם׃\quad \rashiDH{ורדו בכם שנאיכם.}כמשמעו, ישלטו בכם. אגדת ת״כ מפרשה זו (פרק ד)׃\quad \rashiDH{אף אני אעשה זאת.} איני מדבר אלא באף, וכן אַף אֲנִי אֵלֵךְ עִמָּם בְּקֶרִי (פסוק מא)׃\quad \rashiDH{והפקדתי עליכם.} שיהיו המכות פוקדות אתכם מזו לזו, עד שהראשונה פקודה אצלכם, אביא אחרת ואסמכנה לה׃\quad \rashiDH{בהלה.} מכה המבהלת את הבריות, ואיזו, זו מכת מותן׃\quad \rashiDH{את השחפת.} יש לך אדם שהוא חולה ומוטל במטה אבל בשרו שמור עליו, תלמוד לומר שחפת, שהוא נשחף, או עתים שהוא נשחף אבל נוח ואינו מקדיח, תלמוד לומר ואת הקדחת, מלמד שהוא מקדיח, או עתים שהוא מקדיח וסבור הוא בעצמו שיחיה, תלמוד לומר מכלות עינים, או הוא אינו סבור בעצמו שיחיה אבל אחרים סבורים שיחיה, תלמוד לומר ומדיבות נפש׃\quad \rashiDH{וזרעתם לריק זרעכם.} זורעה ואינה מצמחת, ומעתה מה אויביכם באים ואוכלים, ומה ת״ל ואכלוהו אויביכם, הא כיצד, זורעה שנה ראשונה, ואינה מצמחת, שנה שניה מצמחת ואויבים באים ומוצאים תבואה לימי המצור, ושבפנים מתים ברעב שלא לקטו תבואה אשתקד. ד״א וזרעתם לריק זרעכם, כנגד הבנים והבנות הכתוב מדבר, שאתה עמל בהם ומגדלן, והחטא בא ומכלה אותם, שנאמר אֲשֶׁר טִפַּחְתִּי וְרִבִּיתִי אֹיְבִי כִלָּם (איכה ב, כב)׃\quad \rashiDH{ונתתי פני בכם.} כמו שנאמר בטובה ופניתי אליכם, כך נאמר ברעה ונתתי פני. משלו משל למלך שאמר לעבדיו פונה אני מכל עסקי ועוסק אני עמכם לרעה׃\quad \rashiDH{ונגפתם לפני איביכם.} שיהא המות הורג אתכם מבפנים, ובעלי דְבָבֵיכוֹן מקיפין אתכם מבחוץ (ת״כ פרק ד, ה)׃\quad \rashiDH{ורדו בכם שנאיכם.} שאיני מעמיד שונאים אלא מכם ובכם, שבשעה שאומות העולם עובדי אלילים עומדים על ישראל אינם מבקשים אלא מה שבגלוי, שנאמר וְהָיָה אִם זָרַע יִשְׂרָאֵל וְעָלָה מִדְיָן וַעֲמָלֵק וּבְנֵי קֶדֶם וגו׳ (שופטים ו, ג), וַיַּחֲנוּ עֲלֵיהֶם וַיַּשְׁחִיתוּ אֶת יְבוּל הָאָרֶץ (שם ד), אבל בשעה שאעמיד עליכם מכם ובכם הם מחפשים אחר המטמוניות שלכם, וכן הוא אומר וַאֲשֶׁר אָכְלוּ שְׁאֵר עַמִּי וְעוֹרָם מֵעֲלֵיהֶם הִפְשִׁיטוּ וגו׳ (מיכה ג, ג. ת״כ שם)׃\quad \rashiDH{ונסתם.} מפני אימה׃\quad \rashiDH{ואין רודף אתכם.} מבלי כח׃}
\threeverse{\arabic{verse}}%Leviticus26:18
{וְאִ֨ם\maqqaf עַד\maqqaf אֵ֔לֶּה לֹ֥א תִשְׁמְע֖וּ לִ֑י וְיָסַפְתִּי֙ לְיַסְּרָ֣ה אֶתְכֶ֔ם שֶׁ֖בַע עַל\maqqaf חַטֹּאתֵיכֶֽם׃}
{וְאִם עַד אִלֵּין לָא תְקַבְּלוּן לְמֵימְרִי וְאוֹסֵיף לְמִרְדֵּי יָתְכוֹן שְׁבַע עַל חוֹבֵיכוֹן׃}
{And if ye will not yet for these things hearken unto Me, then I will chastise you seven times more for your sins.}{\arabic{verse}}
\rashi{\rashiDH{ואם עד אלה.} ואם בעוד אלה לא תשמעו׃\quad \rashiDH{ויספתי.} עוד יסורין אחרים׃\quad \rashiDH{שבע על חטאתיכם.} שבע פורעניות על ז׳ העבירות האמורות למעלה (ת״כ פרק ה, א)׃}
\threeverse{\arabic{verse}}%Leviticus26:19
{וְשָׁבַרְתִּ֖י אֶת\maqqaf גְּא֣וֹן עֻזְּכֶ֑ם וְנָתַתִּ֤י אֶת\maqqaf שְׁמֵיכֶם֙ כַּבַּרְזֶ֔ל וְאֶֽת\maqqaf אַרְצְכֶ֖ם כַּנְּחֻשָֽׁה׃}
{וְאֶתְבַּר יָת יְקָר תּוּקְפְכוֹן וְאֶתֵּין יָת שְׁמַיָּא דְּעִלָּוֵיכוֹן תַּקִּיפִין כְּבַרְזְלָא מִלְּאַחָתָא מִטְרָא וְאַרְעָא דִּתְחוֹתֵיכוֹן חַסִּינָא כִּנְחָשָׁא מִלְּמֶעֱבַד פֵּירִין׃}
{And I will break the pride of your power; and I will make your heaven as iron, and your earth as brass.}{\arabic{verse}}
\rashi{\rashiDH{ושברתי את גאון עזכם.} זה בית המקדש, וכן הוא אומר הנני מחלל את מקדשי גאון עוזכם (יחזקאל כד, כא)׃\quad \rashiDH{ונתתי את שמיכם כברזל ואת ארצכם כנחשה.} זו קשה משל משה ששם הוא אומר וְהָיוּ שָׁמֶיךָ אֲשֶׁר עַל רֹאשְׁךָ נְחשֶׁת וגו׳ (דברים כח, כג), שיהיו השמים מזיעין כדרך שהנחשת מזיעה, והארץ אינה מזיעה כדרך שאין הברזל מזיע, והיא משמרת פירותיה, אבל כאן השמים לא יהיו מזיעין כדרך שאין הברזל מזיע, ויהא חורב בעולם, והארץ תהא מזיעה כדרך שהנחשת מזיעה, והיא מאבדת פירותיה׃}
\threeverse{\arabic{verse}}%Leviticus26:20
{וְתַ֥ם לָרִ֖יק כֹּחֲכֶ֑ם וְלֹֽא\maqqaf תִתֵּ֤ן אַרְצְכֶם֙ אֶת\maqqaf יְבוּלָ֔הּ וְעֵ֣ץ הָאָ֔רֶץ לֹ֥א יִתֵּ֖ן פִּרְיֽוֹ׃}
{וִיסוּף לְרֵיקָנוּ חֵילְכוֹן וְלָא תִתֵּין אֲרַעְכוֹן יָת עֲלַלְתַּהּ וְאִילָן חַקְלָא לָא יִתֵּין אִבֵּיהּ׃}
{And your strength shall be spent in vain; for your land shall not yield her produce, neither shall the trees of the land yield their fruit.}{\arabic{verse}}
\rashi{\rashiDH{ותם לריק כחכם.} הרי אדם שלא עמל שלא חרש שלא זרע שלא נכש שלא כסח שלא עדר ובשעת הקציר בא שדפון ומלקה אותו אין בכך כלום, אבל אדם שעמל וחרש וזרע ונכש וכסח ועדר ובא שדפון ומלקה אותו, הרי שניו של זה קהות (ת״כ שם ד)׃\quad \rashiDH{ולא תתן ארצכם את יבולה.} אף מה שאתה מוביל לה בשעת הזרע (שם)׃\quad \rashiDH{ועץ הארץ.} אפילו מן הארץ יהא לקוי, שלא יחניט פירותיו בשעת החנטה (שם)׃\quad \rashiDH{לא יתן.} משמש למעלה ולמטה, אעץ ואפרי׃\quad \rashiDH{לא יתן פריו.} כשהוא מפרה מֵשִׁיר פירותיו, הרי שתי קללות, ויש אן שבע פורעניות׃}
\threeverse{\arabic{verse}}%Leviticus26:21
{וְאִם\maqqaf תֵּֽלְכ֤וּ עִמִּי֙ קֶ֔רִי וְלֹ֥א תֹאב֖וּ לִשְׁמֹ֣עַֽ לִ֑י וְיָסַפְתִּ֤י עֲלֵיכֶם֙ מַכָּ֔ה שֶׁ֖בַע כְּחַטֹּאתֵיכֶֽם׃}
{וְאִם תְּהָכוּן קֳדָמַי בְּקַשְׁיוּ וְלָא תֵיבוֹן לְקַבָּלָא לְמֵימְרִי וְאוֹסֵיף לְאֵיתָאָה עֲלֵיכוֹן מַחָא שְׁבַע כְּחוֹבֵיכוֹן׃}
{And if ye walk contrary unto Me, and will not hearken unto Me; I will bring seven times more plagues upon you according to your sins.}{\arabic{verse}}
\rashi{\rashiDH{ואם תלכו עמי קרי.} רבותינו אמרו עראי, במקרה, שאינו אלא לפרקים, כן תלכו עראי במצות. ומנחם פירש לשון מניעה, וכן הֹקֵר רַגְלְךָ (משלי כה, יז), וכן יְקַר רוּחַ (משלי יז, כז), וקרוב לשון זה לתרגומו של אונקלוס, לשון קושי, שמקשים לבם להמנע מהתקרב אלי׃\quad \rashiDH{שבע כחטאתיכם.} שבע פורעניות אחרים במספר שבע כחטאתיכם׃}
\threeverse{\arabic{verse}}%Leviticus26:22
{וְהִשְׁלַחְתִּ֨י בָכֶ֜ם אֶת\maqqaf חַיַּ֤ת הַשָּׂדֶה֙ וְשִׁכְּלָ֣ה אֶתְכֶ֔ם וְהִכְרִ֙יתָה֙ אֶת\maqqaf בְּהֶמְתְּכֶ֔ם וְהִמְעִ֖יטָה אֶתְכֶ֑ם וְנָשַׁ֖מּוּ דַּרְכֵיכֶֽם׃}
{וַאֲשַׁלַּח בְּכוֹן יָת חַיַּת בָּרָא וְתַתְכֵּיל יָתְכוֹן וּתְשֵׁיצֵי יָת בְּעִירְכוֹן וְתַזְעַר יָתְכוֹן וְיִצְדְּיָן אוֹרְחָתְכוֹן׃}
{And I will send the beast of the field among you, which shall rob you of your children, and destroy your cattle, and make you few in number; and your ways shall become desolate.}{\arabic{verse}}
\rashi{\rashiDH{והשלחתי.} לשון גירוי׃\quad \rashiDH{ושכלה אתכם.} אין לי אלא חיה משכלת שדרכה בכך, בהמה שאין דרכה בכך מנין, תלמוד לומר וְשֶׁן בְּהֵמֹת אֲשַׁלַּח ָּם (דברים לב, כד), הרי שתים, ומנין שתהא ממיתה בנשיכתה, תלמוד לומר עִם חֲמַת זֹחֲלֵי עָפָר (שם), מה אלו נושכין וממיתין, אף אלו נושכין וממיתין כבר היו שנים בארץ ישראל, חמור נושך וממית, ערוד נושך וממית׃\quad \rashiDH{ושכלה אתכם}. אלו הקטנים׃\quad \rashiDH{והכריתה את בהמתכם.} מבחוץ׃\quad \rashiDH{והמעיטה אתכם.} מבפנים׃\quad \rashiDH{ונשמו דרכיכם.} שבילים גדולים ושבילים קטנים, הרי שבע פורעניות, שן בהמה, ושן חיה, חמת זוחלי עפר, ושכלה, והכריתה, והמעיטה, ונשמו׃}
\threeverse{\arabic{verse}}%Leviticus26:23
{וְאִ֨ם\maqqaf בְּאֵ֔לֶּה לֹ֥א תִוָּסְר֖וּ לִ֑י וַהֲלַכְתֶּ֥ם עִמִּ֖י קֶֽרִי׃}
{וְאִם בְּאִלֵּין לָא תִתְרְדוֹן לְמֵימְרִי וּתְהָכוּן קֳדָמַי בְּקַשְׁיוּ׃}
{And if in spite of these things ye will not be corrected unto Me, but will walk contrary unto Me;}{\arabic{verse}}
\rashi{\rashiDH{לא תוסרו לי.} לשוב אלי׃}
\threeverse{\arabic{verse}}%Leviticus26:24
{וְהָלַכְתִּ֧י אַף\maqqaf אֲנִ֛י עִמָּכֶ֖ם בְּקֶ֑רִי וְהִכֵּיתִ֤י אֶתְכֶם֙ גַּם\maqqaf אָ֔נִי שֶׁ֖בַע עַל\maqqaf חַטֹּאתֵיכֶֽם׃}
{וַאֲהָךְ אַף אֲנָא עִמְּכוֹן בְּקַשְׁיוּ וְאַלְקֵי יָתְכוֹן אַף אֲנָא שְׁבַע עַל חוֹבֵיכוֹן׃}
{then will I also walk contrary unto you; and I will smite you, even I, seven times for your sins.}{\arabic{verse}}
\threeverse{\arabic{verse}}%Leviticus26:25
{וְהֵבֵאתִ֨י עֲלֵיכֶ֜ם חֶ֗רֶב נֹקֶ֙מֶת֙ נְקַם\maqqaf בְּרִ֔ית וְנֶאֱסַפְתֶּ֖ם אֶל\maqqaf עָרֵיכֶ֑ם וְשִׁלַּ֤חְתִּי דֶ֙בֶר֙ בְּת֣וֹכְכֶ֔ם וְנִתַּתֶּ֖ם בְּיַד\maqqaf אוֹיֵֽב׃}
{וְאַיְתִי עֲלֵיכוֹן דְּקָטְלִין בְּחַרְבָּא וְיִתְפַּרְעוּן מִנְּכוֹן פּוּרְעָנוּתָא עַל דַּעֲבַרְתּוּן עַל אוֹרָיְתָא וְתִתְכַּנְשׁוּן לְקִרְוֵיכוֹן וַאֲגָרֵי מוֹתָנָא בֵּינֵיכוֹן וְתִתְמַסְרוּן בְּיַד שָׂנְאָה׃}
{And I will bring a sword upon you, that shall execute the vengeance of the covenant; and ye shall be gathered together within your cities; and I will send the pestilence among you; and ye shall be delivered into the hand of the enemy.}{\arabic{verse}}
\rashi{\rashiDH{נקם ברית.} ויש נקם שאינו בברית, כדרך שאר נקמות, וזהו סמוי עיניו של צדקיהו. דבר אחר נקם ברית, נקמת בריתי אשר עברתם. כל הבאת חרב שבמקרא, היא מלחמת חיילות אויבים׃\quad \rashiDH{ונאספתם.} מן החוץ אל תוך הערים מפני המצור׃\quad \rashiDH{ושלחתי דבר בתוככם.} וע״י הַדֶּבֶר, ונתתם ביד האויבים הצרים עליכם, לפי שאין מלינים את המת בירושלים, וכשהם מוציאים את המת לקברו נתנים ביד אויב׃}
\threeverse{\arabic{verse}}%Leviticus26:26
{בְּשִׁבְרִ֣י לָכֶם֮ מַטֵּה\maqqaf לֶ֒חֶם֒ וְ֠אָפ֠וּ עֶ֣שֶׂר נָשִׁ֤ים לַחְמְכֶם֙ בְּתַנּ֣וּר אֶחָ֔ד וְהֵשִׁ֥יבוּ לַחְמְכֶ֖ם בַּמִּשְׁקָ֑ל וַאֲכַלְתֶּ֖ם וְלֹ֥א תִשְׂבָּֽעוּ׃ \setuma }
{בִּדְאֶתְבַּר לְכוֹן סְעֵיד מֵיכְלָא וְיֵיפְיָן עֲשַׂר נְשִׁין לַחְמְכוֹן בְּתַנּוּרָא חַד וְיָתִיבוּן לַחְמְכוֹן בְּמַתְקָלָא וְתֵיכְלוּן וְלָא תִשְׂבְּעוּן׃}
{When I break your staff of bread, ten women shall bake your bread in one oven, and they shall deliver your bread again by weight; and ye shall eat, and not be satisfied.}{\arabic{verse}}
\rashi{\rashiDH{מטה לחם.} לשון משען, כמו מַטֵּה עֹז (ירמיה מח, יז)׃\quad \rashiDH{בשברי לכם מטה לחם.} אשבור לכם כל מסעד אוכל, והם חצי רעב׃\quad \rashiDH{ואפו עשר נשים לחמכם בתנור אחד.} מחוסר עצים׃\quad \rashiDH{והשיבו לחמכם במשקל.} שתהא התבואה נרקבת ונעשית פת נפולה ומשתברת בתנור, והן יושבות ושוקלות את השברים לחלקם ביניהם׃\quad \rashiDH{ואכלתם ולא תשבעו.} זה מארה בתוך המעים בלחם. הרי ז׳ פורעניות, חרב, מצור, דבר, שבר מטה לחם, חוסר עצים, פת נפולה, מארה במעים. ונתתם אינה מן המנין שהיא החרב׃}
\threeverse{\arabic{verse}}%Leviticus26:27
{וְאִ֨ם\maqqaf בְּזֹ֔את לֹ֥א תִשְׁמְע֖וּ לִ֑י וַהֲלַכְתֶּ֥ם עִמִּ֖י בְּקֶֽרִי׃}
{וְאִם בְּדָא לָא תְקַבְּלוּן לְמֵימְרִי וּתְהָכוּן קֳדָמַי בְּקַשְׁיוּ׃}
{And if ye will not for all this hearken unto Me, but walk contrary unto Me;}{\arabic{verse}}
\threeverse{\arabic{verse}}%Leviticus26:28
{וְהָלַכְתִּ֥י עִמָּכֶ֖ם בַּחֲמַת\maqqaf קֶ֑רִי וְיִסַּרְתִּ֤י אֶתְכֶם֙ אַף\maqqaf אָ֔נִי שֶׁ֖בַע עַל\maqqaf חַטֹּאתֵיכֶֽם׃}
{וַאֲהָךְ עִמְּכוֹן בִּתְקוֹף רְגַז וְאֶרְדֵּי יָתְכוֹן אַף אֲנָא שְׁבַע עַל חוֹבֵיכוֹן׃}
{then I will walk contrary unto you in fury; and I also will chastise you seven times for your sins.}{\arabic{verse}}
\threeverse{\arabic{verse}}%Leviticus26:29
{וַאֲכַלְתֶּ֖ם בְּשַׂ֣ר בְּנֵיכֶ֑ם וּבְשַׂ֥ר בְּנֹתֵיכֶ֖ם תֹּאכֵֽלוּ׃}
{וְתֵיכְלוּן בְּסַר בְּנֵיכוֹן וּבְסַר בְּנָתְכוֹן תֵּיכְלוּן׃}
{And ye shall eat the flesh of your sons, and the flesh of your daughters shall ye eat.}{\arabic{verse}}
\threeverse{\arabic{verse}}%Leviticus26:30
{וְהִשְׁמַדְתִּ֞י אֶת\maqqaf בָּמֹֽתֵיכֶ֗ם וְהִכְרַתִּי֙ אֶת\maqqaf חַמָּ֣נֵיכֶ֔ם וְנָֽתַתִּי֙ אֶת\maqqaf פִּגְרֵיכֶ֔ם עַל\maqqaf פִּגְרֵ֖י גִּלּוּלֵיכֶ֑ם וְגָעֲלָ֥ה נַפְשִׁ֖י אֶתְכֶֽם׃}
{וַאֲשֵׁיצֵי יָת בָּמָתְכוֹן וַאֲקַצֵּיץ יָת חֲנִסְנְסֵיכוֹן וְאֶתֵּין יָת פִּגְרֵיכוֹן עַל פִּגּוּר טָעֲוָתְכוֹן וִירַחֵיק מֵימְרִי יָתְכוֹן׃}
{And I will destroy your high places, and cut down your sun-pillars, and cast your carcasses upon the carcasses of your idols; and My soul shall abhor you.}{\arabic{verse}}
\rashi{\rashiDH{במתיכם.} מִגְדָּלִים וּבִרָנִיוֹת׃\quad \rashiDH{חמניכם.} מין עבודת אלילים שמעמידין על הגגות, ועל שם שמעמידין בחמה קרויין חמנים׃\quad \rashiDH{ונתתי את פגריכם.} תְּפוּחֵי רעב היו, ומוציאים יראתם מחיקם ומנשקים אותם, וכרסו נבקעת ונופל עליה׃\quad \rashiDH{וגעלה נפשי אתכם.} זה סילוק שכינה (ת״כ פרק ו, ד)׃}
\threeverse{\arabic{verse}}%Leviticus26:31
{וְנָתַתִּ֤י אֶת\maqqaf עָֽרֵיכֶם֙ חׇרְבָּ֔ה וַהֲשִׁמּוֹתִ֖י אֶת\maqqaf מִקְדְּשֵׁיכֶ֑ם וְלֹ֣א אָרִ֔יחַ בְּרֵ֖יחַ נִיחֹֽחֲכֶֽם׃}
{וְאֶתֵּין יָת קִרְוֵיכוֹן חָרְבָּא וַאֲצַדֵּי יָת מַקְדְּשֵׁיכוֹן וְלָא אֲקַבֵּיל בְּרַעֲוָא קוּרְבַּן כְּנֵישָׁתְכוֹן׃}
{And I will make your cities a waste, and will bring your sanctuaries unto desolation, and I will not smell the savour of your sweet odours.}{\arabic{verse}}
\rashi{\rashiDH{ונתתי את עריכם חרבה.} יכול מאדם, כשהוא אומר והשימותי אני את הארץ, הרי אדם אמור, הא מה אני מקיים חרבה, מעובר ושב׃\quad \rashiDH{והשימותי את מקדשיכם.} יכול מן הקרבנות, כשהוא אומר ולא אריח הרי קרבנות אמורים, הא מה אני מקיים והשימותי את מקדשיכם מן הַגְּדוּדִיוֹת, שיירות של ישראל שהיו מתקדשות ונועדות לבא שם. הרי שבע פורעניות, אכילת בשר בנים ובנות, והשמדת במות הרי שתים, כריתת חמנים אין כאן פורענות אלא על ידי השמדת הבירניות יפלו החמנים שבראשי הגגות ויכרתו, ונתתי את פגריכם וגו׳ הרי שלש, סלוק שכינה ארבע, חרבן ערים, שממון מקדש מן הגדודיות, ולא אריח קרבנות, הרי שבע׃}
\threeverse{\arabic{verse}}%Leviticus26:32
{וַהֲשִׁמֹּתִ֥י אֲנִ֖י אֶת\maqqaf הָאָ֑רֶץ וְשָֽׁמְמ֤וּ עָלֶ֙יהָ֙ אֹֽיְבֵיכֶ֔ם הַיֹּשְׁבִ֖ים בָּֽהּ׃}
{וַאֲצַדֵּי אֲנָא יָת אַרְעָא וְיִצְדּוֹן עֲלַהּ בַּעֲלֵי דְּבָבֵיכוֹן דְּיָתְבִין בַּהּ׃}
{And I will bring the land into desolation; and your enemies that dwell therein shall be astonished at it.}{\arabic{verse}}
\rashi{\rashiDH{והשמתי אני את הארץ.} זו מדה טובה לישראל שלא ימצאו האויבים נחת רוח בארצם, שתהא שוממה מיושביה׃}
\threeverse{\arabic{verse}}%Leviticus26:33
{וְאֶתְכֶם֙ אֱזָרֶ֣ה בַגּוֹיִ֔ם וַהֲרִיקֹתִ֥י אַחֲרֵיכֶ֖ם חָ֑רֶב וְהָיְתָ֤ה אַרְצְכֶם֙ שְׁמָמָ֔ה וְעָרֵיכֶ֖ם יִהְי֥וּ חׇרְבָּֽה׃}
{וְיָתְכוֹן אֲבַדַּר בֵּינֵי עַמְמַיָּא וַאֲגָרֵי בָּתְרֵיכוֹן דְּקָטְלִין בְּחַרְבָּא וּתְהֵי אֲרַעְכוֹן צָדְיָא וְקִרְוֵיכוֹן יִהְוְיָן חָרְבָּא׃}
{And you will I scatter among the nations, and I will draw out the sword after you; and your land shall be a desolation, and your cities shall be a waste.}{\arabic{verse}}
\rashi{\rashiDH{ואתכם אזרה בגוים.} זו מדה קשה, שבשעה שבני מדינה גולים למקום אחד רואים זה את זה ומתנחמין, וישראל נזרו כבמזרה, כאדם הזורה שעורים בנפה ואין אחת מהן דבוקה בחבירתה׃\quad \rashiDH{והריקתי.} כששולף החרב מתרוקן הנדן. ומדרשו חרב הנשמטת אחריכם אינה חוזרת מהר, כאדם שמריק את המים ואין סופן לחזור׃\quad \rashiDH{והיתה ארצכם שממה.} שלא תמהרו לשוב לתוכה, ומתוך כך עריכם יהיו חרבה, נראות לכם חרבות, שבשעה שאדם גולה מביתו ומכרמו ומעירו סופו לחזור כאילו אין כרמו וביתו חרבים, כך שנויה בת״כ (פרק ז, א)׃}
\threeverse{\arabic{verse}}%Leviticus26:34
{אָז֩ תִּרְצֶ֨ה הָאָ֜רֶץ אֶת\maqqaf שַׁבְּתֹתֶ֗יהָ כֹּ֚ל יְמֵ֣י הׇשַּׁמָּ֔ה וְאַתֶּ֖ם בְּאֶ֣רֶץ אֹיְבֵיכֶ֑ם אָ֚ז תִּשְׁבַּ֣ת הָאָ֔רֶץ וְהִרְצָ֖ת אֶת\maqqaf שַׁבְּתֹתֶֽיהָ׃}
{בְּכֵין תִּרְעֵי אַרְעָא יָת שְׁמִטַּהָא כֹּל יוֹמִין דִּצְדִּיאַת וְאַתּוּן בַּאֲרַע בַּעֲלֵי דְּבָבֵיכוֹן בְּכֵין תַּשְׁמֵיט אַרְעָא וְתִרְעֵי יָת שְׁמִטַּהָא׃}
{Then shall the land be paid her sabbaths, as long as it lieth desolate, and ye are in your enemies’ land; even then shall the land rest, and repay her sabbaths.}{\arabic{verse}}
\rashi{\rashiDH{אז תרצה.} תפייס את כעס המקום, שכעס על שמטותיה׃\quad \rashiDH{והרצת.} למלך את שבתותיה׃}
\threeverse{\arabic{verse}}%Leviticus26:35
{כׇּל\maqqaf יְמֵ֥י הׇשַּׁמָּ֖ה תִּשְׁבֹּ֑ת אֵ֣ת אֲשֶׁ֧ר לֹֽא\maqqaf שָׁבְתָ֛ה בְּשַׁבְּתֹתֵיכֶ֖ם בְּשִׁבְתְּכֶ֥ם עָלֶֽיהָ׃}
{כָּל יוֹמִין דִּצְדִּיאַת תַּשְׁמֵיט יָת דְּלָא שְׁמַטַת בִּשְׁמִטֵּיכוֹן כַּד הֲוֵיתוֹן יָתְבִין עֲלַהּ׃}
{As long as it lieth desolate it shall have rest; even the rest which it had not in your sabbaths, when ye dwelt upon it.}{\arabic{verse}}
\rashi{\rashiDH{כל ימי השמה.} לשון הֵעָשׂוֹת, ומ״ם דגש במקום כפל שממה׃\quad \rashiDH{את אשר לא שבתה.} שבעים שנה של גלות בבל הן היו כנגד ע׳ שנות השמטה ויובל שהיו בשנים שהכעיסו ישראל בארצם לפני המקום ארבע מאות ושלשים שנה. שלש מאות ותשעים היו שני עונם משנכנסו לארץ עד שגלו עשרת השבטים, ובני יהודה הכעיסו לפניו מ׳ שנה משגלו עשרת השבטים עד חרבות ירושלים, הוא שנאמר ביחזקאל וְאַתָּה שְׁכַב עַל צִדְּךָ הַשְּׂמָאלִי וגו׳ (יחזקאל ד, ד), וְכִלִּיתָ אֶת אֵלֶּה וְשָׁכַבְתָּ עַל צִדְּךָ הַיְמָנִי שֵׁנִית אַרְבָּעִים יוֹם וְנָשָׂאתָ אֶת עֲוֹן בֵּית יְהוּדָה (שם ו), ונבואה זו נאמרה ליחזקאל בשנה החמישית לגלות המלך יהויכין, ועוד עשו שש שנים עד גלות צדקיהו, הרי ארבעים ושש. ואם תאמר שנות מנשה חמשים וחמש היו, מנשה עשה תשובה שלשים ושלש שנה, וכל שנות רשעו עשרים ושתים, כמו שאמרו באגדת חלק (סנהדרין קג.), ושל אמון שתים, ואחת עשרה ליהויקים, וכנגדן לצדקיהו. צא וחשוב לארבע מאות ושלשים ושש שנה שמיטין ויובלות שבהם, והם שש עשרה למאה י״ד שמיטין וב׳ יובלות הרי לארבע מאות שנה ששים וארבע, לשלשים ושש שנה חמש שמיטות, הרי שבעים חסר אחת, ועוד שנה יתירה שנכנסה בשמטה המשלמת לשבעים. (נ״א ואותו יובל שגלו שלא נגמר בעונם נחשב להם) ועליהם נגזר שבעים שנה שלמים, וכן הוא אומר בדברי הימים עַד רָצְתָה הָאָרֶץ אֶת שַׁבְּתוֹתֶיהָ וגו׳, לְמַלְּאוֹת שִׁבְעִים שָׁנָה (דברים הימים־ב לו, כא)׃}
\threeverse{\arabic{verse}}%Leviticus26:36
{וְהַנִּשְׁאָרִ֣ים בָּכֶ֔ם וְהֵבֵ֤אתִי מֹ֙רֶךְ֙ בִּלְבָבָ֔ם בְּאַרְצֹ֖ת אֹיְבֵיהֶ֑ם וְרָדַ֣ף אֹתָ֗ם ק֚וֹל עָלֶ֣ה נִדָּ֔ף וְנָס֧וּ מְנֻֽסַת\maqqaf חֶ֛רֶב וְנָפְל֖וּ וְאֵ֥ין רֹדֵֽף׃}
{וּדְיִשְׁתְּאַרוּן בְּכוֹן וְאַעֵיל תְּבָרָא בְּלִבְּהוֹן בְּאַרְעָתָא דְּשָׂנְאֵיהוֹן וְיִרְדּוֹף יָתְהוֹן קָל טַרְפָא דְּשָׁקֵיף וְיִעְרְקוֹן מִעְרָק כִּד מִן קֳדָם דְּקָטְלִין בְּחַרְבָּא וְיִפְּלוּן וְלֵית דְּרָדֵיף׃}
{And as for them that are left of you, I will send a faintness into their heart in the lands of their enemies; and the sound of a driven leaf shall chase them; and they shall flee, as one fleeth from the sword; and they shall fall when none pursueth.}{\arabic{verse}}
\rashi{\rashiDH{והבאתי מרך.} פחד ורך לבב מ״ם של מרך יסוד נופל הוא, כמו מ״ם של מועד ושל מוקש׃\quad \rashiDH{ונסו מנסת חרב.} כאילו רודפים הורגים אותם׃\quad \rashiDH{עלה נדף.} שהרוח דוחפו ומכהו על עלה אחר ומקשקש ומוציא קול, וכן תרגומו קַל טַרְפָּא דְשַׁקִיף, לשון חבטה, שְׁדוּפֹת קָּדִים (בראשית מא, ו), שְׁקִיפָן קִדּוּם, לשון משקוף, מקום חבטת הדלת וכן תרגומו של חבורה (שמות כא, כה), מַשְׁקוֹפֵי׃}
\threeverse{\arabic{verse}}%Leviticus26:37
{וְכָשְׁל֧וּ אִישׁ\maqqaf בְּאָחִ֛יו כְּמִפְּנֵי\maqqaf חֶ֖רֶב וְרֹדֵ֣ף אָ֑יִן וְלֹא\maqqaf תִֽהְיֶ֤ה לָכֶם֙ תְּקוּמָ֔ה לִפְנֵ֖י אֹֽיְבֵיכֶֽם׃}
{וְיִתַּקְלוּן גְּבַר בַּאֲחוּהִי כִּד מִן קֳדָם דְּקָטְלִין בְּחַרְבָּא וְרָדֵיף לָיִת וְלָא תְהֵי לְכוֹן תְּקוּמָה קֳדָם בַּעֲלֵי דְּבָבֵיכוֹן׃}
{And they shall stumble one upon another, as it were before the sword, when none pursueth; and ye shall have no power to stand before your enemies.}{\arabic{verse}}
\rashi{\rashiDH{וכשלו איש באחיו.} כשירצו לנוס יכשלו זה בזה, כי יבהלו לרוץ׃\quad \rashiDH{כמפני חרב.} כאילו בורחים מלפני הורגים, שיהא בלבבם פחד וכל שעה סבורים שאדם רודפם. ומדרשו (ת״כ פרק ז, ה), וכשלו איש באחיו, זה נכשל בעונו של זה, שכל ישראל ערבין זה לזה (שבועות לט.)׃}
\threeverse{\arabic{verse}}%Leviticus26:38
{וַאֲבַדְתֶּ֖ם בַּגּוֹיִ֑ם וְאָכְלָ֣ה אֶתְכֶ֔ם אֶ֖רֶץ אֹיְבֵיכֶֽם׃}
{וְתֵיבְדוּן בֵּינֵי עַמְמַיָּא וּתְגַּמַּר יָתְכוֹן אֲרַע בַּעֲלֵי דְּבָבֵיכוֹן׃}
{And ye shall perish among the nations, and the land of your enemies shall eat you up.}{\arabic{verse}}
\rashi{\rashiDH{ואבדתם בגוים}. כשתהיו פזורים תהיו אבודים זה מזה׃\quad \rashiDH{ואכלה אתכם.} אלו המתים בגולה׃}
\threeverse{\arabic{verse}}%Leviticus26:39
{וְהַנִּשְׁאָרִ֣ים בָּכֶ֗ם יִמַּ֙קּוּ֙ בַּֽעֲוֺנָ֔ם בְּאַרְצֹ֖ת אֹיְבֵיכֶ֑ם וְאַ֛ף בַּעֲוֺנֹ֥ת אֲבֹתָ֖ם אִתָּ֥ם יִמָּֽקּוּ׃}
{וּדְיִשְׁתְּאַרוּן בְּכוֹן יִתְמְסוֹן בְּחוֹבֵיהוֹן בְּאַרְעָתָא דְּסָנְאֵיכוֹן וְאַף בְּחוֹבֵי אֲבָהָתְהוֹן בִּישַׁיָּא דַּאֲחִידִין בְּיַדְהוֹן יִתְמְסוֹן׃}
{And they that are left of you shall pine away in their iniquity in your enemies’ lands; and also in the iniquities of their fathers shall they pine away with them.}{\arabic{verse}}
\rashi{\rashiDH{בעונת אבותם אתם.} כשעונות אבותם אתם כשאוחזים מעשה אבותיהם בידיהם (שם פרק ח, ב  סנהדרין כז׃)׃\quad \rashiDH{ימקו.} לשון המסה, כמו ימסו, וכמוהו תִּמַּקְנָה בְחֹרֵיהֶן (זכרי׳ יד, יב), נָמַקּוּ חַבּוּרֹתָי (תהלים לח, ו)׃}
\threeverse{\arabic{verse}}%Leviticus26:40
{וְהִתְוַדּ֤וּ אֶת\maqqaf עֲוֺנָם֙ וְאֶת\maqqaf עֲוֺ֣ן אֲבֹתָ֔ם בְּמַעֲלָ֖ם אֲשֶׁ֣ר מָֽעֲלוּ\maqqaf בִ֑י וְאַ֕ף אֲשֶׁר\maqqaf הָֽלְכ֥וּ עִמִּ֖י בְּקֶֽרִי׃}
{וִיוַדּוֹן יָת חוֹבֵיהוֹן וְיָת חוֹבֵי אֲבָהָתְהוֹן בְּשִׁקְרְהוֹן דְּשַׁקַּרוּ בְּמֵימְרִי וְאַף דְּהַלִּיכוּ קֳדָמַי בְּקַשְׁיוּ׃}
{And they shall confess their iniquity, and the iniquity of their fathers, in their treachery which they committed against Me, and also that they have walked contrary unto Me.}{\arabic{verse}}
\threeverse{\arabic{verse}}%Leviticus26:41
{אַף\maqqaf אֲנִ֗י אֵלֵ֤ךְ עִמָּם֙ בְּקֶ֔רִי וְהֵבֵאתִ֣י אֹתָ֔ם בְּאֶ֖רֶץ אֹיְבֵיהֶ֑ם אוֹ\maqqaf אָ֣ז יִכָּנַ֗ע לְבָבָם֙ הֶֽעָרֵ֔ל וְאָ֖ז יִרְצ֥וּ אֶת\maqqaf עֲוֺנָֽם׃}
{אַף אֲנָא אֲהָךְ עִמְּהוֹן בְּקַשְׁיוּ וְאַעֵיל יָתְהוֹן בַּאֲרַע בַּעֲלֵי דְּבָבֵיהוֹן אוֹ בְכֵין יִתְּבַר לִבְּהוֹן טַפְשָׁא וּבְכֵין יִרְעוֹן יָת חוֹבֵיהוֹן׃}
{I also will walk contrary unto them, and bring them into the land of their enemies; if then perchance their uncircumcised heart be humbled, and they then be paid the punishment of their iniquity;}{\arabic{verse}}
\rashi{\rashiDH{והבאתי אתם.} אני בעצמי אביאם. זו מדה טובה לישראל, שלא יהיו אומרים הואיל וגלינו בין האומות עובדי אלילים נעשה כמעשיהם, אני איני מניחם, אלא מעמיד אני את נביאי ומחזירן לתחת כנפי, שנאמר וְהָעֹלָה עַל רוּחֲכֶם הָיוֹ לֹא תִהְיֶה וגו׳ (יחזקאל כ, לב), חַי אָנִי וגו׳ אִם לֹא בְּיָד חֲזָקָה וגו׳ (שם לג. ת״כ שם ה)׃\quad \rashiDH{או אז יכנע.} כמו אוֹ נוֹדַע כִּי שׁוֹר נַגָּח הוּא (שמות כא, לו), אם אז יכנע. לשון אחר אולי, שמא אז יכנע לבבם וגו׳׃\quad \rashiDH{ואז ירצו את עונם.} יכפרו על עונם ביסוריהם׃}
\threeverse{\arabic{verse}}%Leviticus26:42
{וְזָכַרְתִּ֖י אֶת\maqqaf בְּרִיתִ֣י יַעֲק֑וֹב וְאַף֩ אֶת\maqqaf בְּרִיתִ֨י יִצְחָ֜ק וְאַ֨ף אֶת\maqqaf בְּרִיתִ֧י אַבְרָהָ֛ם אֶזְכֹּ֖ר וְהָאָ֥רֶץ אֶזְכֹּֽר׃}
{וּדְכִירְנָא יָת קְיָמִי דְּעִם יַעֲקוֹב וְאַף יָת קְיָמִי דְּעִם יִצְחָק וְאַף יָת קְיָמִי דְּעִם אַבְרָהָם אֲנָא דְּכִיר וְאַרְעָא אֲנָא דְּכִיר׃}
{then will I remember My covenant with Jacob, and also My covenant with Isaac, and also My covenant with Abraham will I remember; and I will remember the land.}{\arabic{verse}}
\rashi{\rashiDH{וזכרתי את בריתי יעקוב.} בחמשה מקומות נכתב מלא, ואליהו חסר בחמשה מקומות, יעקב נטל אות משמו של אליהו ערבון שיבוא ויבשר גאולת בניו׃\quad \rashiDH{וזכרתי את בריתי יעקוב.} למה נמנו אחרונית, כלומר, כדאי הוא יעקב הקטן לכך, ואם אינו כדאי, הרי יצחק עמו, ואם אינו כדאי הרי אברהם עמו, שהוא כדאי. ולמה לא נאמרה זכירה ביצחק, אלא אפרו של יצחק נראה לפני צבור ומונח על המזבח׃}
\threeverse{\arabic{verse}}%Leviticus26:43
{וְהָאָ֩רֶץ֩ תֵּעָזֵ֨ב מֵהֶ֜ם וְתִ֣רֶץ אֶת\maqqaf שַׁבְּתֹתֶ֗יהָ בׇּהְשַׁמָּה֙ מֵהֶ֔ם וְהֵ֖ם יִרְצ֣וּ אֶת\maqqaf עֲוֺנָ֑ם יַ֣עַן וּבְיַ֔עַן בְּמִשְׁפָּטַ֣י מָאָ֔סוּ וְאֶת\maqqaf חֻקֹּתַ֖י גָּעֲלָ֥ה נַפְשָֽׁם׃}
{וְאַרְעָא תִּתְרְטֵישׁ מִנְּהוֹן וְתִרְעֵי יָת שְׁמִטַּהָא בְּדִצְדִּיאַת מִנְּהוֹן וְאִנּוּן יִרְעוֹן יָת חוֹבֵיהוֹן לְוָטִין חֲלָף בִּרְכָאן אַיְתִי עֲלֵיהוֹן דִּבְדִינַי קַצוּ וְיָת קְיָמַי רַחֵיקַת נַפְשְׁהוֹן׃}
{For the land shall lie forsaken without them, and shall be paid her sabbaths, while she lieth desolate without them; and they shall be paid the punishment of their iniquity; because, even because they rejected Mine ordinances, and their soul abhorred My statutes.}{\arabic{verse}}
\rashi{\rashiDH{יען וביען.} גמול, ובגמול אשר במשפטי מאסו׃}
\threeverse{\arabic{verse}}%Leviticus26:44
{וְאַף\maqqaf גַּם\maqqaf זֹ֠את בִּֽהְיוֹתָ֞ם בְּאֶ֣רֶץ אֹֽיְבֵיהֶ֗ם לֹֽא\maqqaf מְאַסְתִּ֤ים וְלֹֽא\maqqaf גְעַלְתִּים֙ לְכַלֹּתָ֔ם לְהָפֵ֥ר בְּרִיתִ֖י אִתָּ֑ם כִּ֛י אֲנִ֥י יְהֹוָ֖ה אֱלֹהֵיהֶֽם׃}
{וְאַף בְּרַם דָּא בְּמִהְוֵיהוֹן בַּאֲרַע בַּעֲלֵי דְּבָבֵיהוֹן לָא אַרְטוֹשִׁנּוּן וְלָא אֲרַחֵיקִנּוּן לְשֵׁיצָיוּתְהוֹן לְאַשְׁנָאָה קְיָמִי עִמְּהוֹן אֲרֵי אֲנָא יְיָ אֱלָהֲהוֹן׃}
{And yet for all that, when they are in the land of their enemies, I will not reject them, neither will I abhor them, to destroy them utterly, and to break My covenant with them; for I am the \lord\space their God.}{\arabic{verse}}
\rashi{\rashiDH{ואף גם זאת.} ואף אפילו אני עושה עמהם זאת, הפורענות אשר אמרתי, בהיותם בארץ אויביהם, לא מאסתים לכלותם ולהפר בריתי אשר אתם׃}
\threeverse{\arabic{verse}}%Leviticus26:45
{וְזָכַרְתִּ֥י לָהֶ֖ם בְּרִ֣ית רִאשֹׁנִ֑ים אֲשֶׁ֣ר הוֹצֵֽאתִי\maqqaf אֹתָם֩ מֵאֶ֨רֶץ מִצְרַ֜יִם לְעֵינֵ֣י הַגּוֹיִ֗ם לִהְי֥וֹת לָהֶ֛ם לֵאלֹהִ֖ים אֲנִ֥י יְהֹוָֽה׃}
{וּדְכִירְנָא לְהוֹן קְיָם קַדְמָאֵי דְּאַפֵּיקִית יָתְהוֹן מֵאַרְעָא דְּמִצְרַיִם לְעֵינֵי עַמְמַיָּא לְמִהְוֵי לְהוֹן לֶאֱלָהּ אֲנָא יְיָ׃}
{But I will for their sakes remember the covenant of their ancestors, whom I brought forth out of the land of Egypt in the sight of the nations, that I might be their God: I am the \lord.}{\arabic{verse}}
\rashi{\rashiDH{ברית ראשונים.} של שבטים׃}
\threeverse{\arabic{verse}}%Leviticus26:46
{אֵ֠לֶּה הַֽחֻקִּ֣ים וְהַמִּשְׁפָּטִים֮ וְהַתּוֹרֹת֒ אֲשֶׁר֙ נָתַ֣ן יְהֹוָ֔ה בֵּינ֕וֹ וּבֵ֖ין בְּנֵ֣י יִשְׂרָאֵ֑ל בְּהַ֥ר סִינַ֖י בְּיַד\maqqaf מֹשֶֽׁה׃ \petucha }
{אִלֵּין קְיָמַיָּא וְדִינַיָּא וְאוֹרָיָתָא דִּיהַב יְיָ בֵּין מֵימְרֵיהּ וּבֵין בְּנֵי יִשְׂרָאֵל בְּטוּרָא דְּסִינַי בִּידָא דְּמֹשֶׁה׃}
{These are the statutes and ordinances and laws, which the \lord\space made between Him and the children of Israel in mount Sinai by the hand of Moses.}{\arabic{verse}}
\rashi{\rashiDH{והתורת.} אחת בכתב ואחת בעל פה, מגיד שכולם נתנו למשה בסיני׃}
\newperek
\aliyacounter{רביעי}
\newseder{23}
\threeverse{\aliya{רביעי\newline (ששי)}\newline\vspace{-4pt}\newline\seder{כג}}%Leviticus27:1
{וַיְדַבֵּ֥ר יְהֹוָ֖ה אֶל\maqqaf מֹשֶׁ֥ה לֵּאמֹֽר׃}
{וּמַלֵּיל יְיָ עִם מֹשֶׁה לְמֵימַר׃}
{And the \lord\space spoke unto Moses, saying:}{\Roman{chap}}
\threeverse{\arabic{verse}}%Leviticus27:2
{דַּבֵּ֞ר אֶל\maqqaf בְּנֵ֤י יִשְׂרָאֵל֙ וְאָמַרְתָּ֣ אֲלֵהֶ֔ם אִ֕ישׁ כִּ֥י יַפְלִ֖א נֶ֑דֶר בְּעֶרְכְּךָ֥ נְפָשֹׁ֖ת לַֽיהֹוָֽה׃}
{מַלֵּיל עִם בְּנֵי יִשְׂרָאֵל וְתֵימַר לְהוֹן גְּבַר אֲרֵי יְפָרֵישׁ נְדַר בְּפוּרְסַן נַפְשָׁתָא קֳדָם יְיָ׃}
{Speak unto the children of Israel, and say unto them: When a man shall clearly utter a vow of persons unto the \lord, according to thy valuation,}{\arabic{verse}}
\rashi{\rashiDH{כי יפלא.} יפריש בפיו׃\quad \rashiDH{בערכך נפשת.} ליתן ערך נפשו, לומר ערך דבר שנפשו תלויה בו, עלי׃}
\threeverse{\arabic{verse}}%Leviticus27:3
{וְהָיָ֤ה עֶרְכְּךָ֙ הַזָּכָ֔ר מִבֶּן֙ עֶשְׂרִ֣ים שָׁנָ֔ה וְעַ֖ד בֶּן\maqqaf שִׁשִּׁ֣ים שָׁנָ֑ה וְהָיָ֣ה עֶרְכְּךָ֗ חֲמִשִּׁ֛ים שֶׁ֥קֶל כֶּ֖סֶף בְּשֶׁ֥קֶל הַקֹּֽדֶשׁ׃}
{וִיהֵי פּוּרְסָנֵיהּ דְּדִכְרָא מִבַּר עֶשְׂרִין שְׁנִין וְעַד בַּר שִׁתִּין שְׁנִין וִיהֵי פּוּרְסָנֵיהּ חַמְשִׁין סִלְעִין דִּכְסַף בְּסִלְעֵי קוּדְשָׁא׃}
{then thy valuation shall be for the male from twenty years old even unto sixty years old, even thy valuation shall be fifty shekels of silver, after the shekel of the sanctuary.}{\arabic{verse}}
\rashi{\rashiDH{והיה ערכך וגו׳.} אין ערך זה לשון דמים, אלא בין שהוא יוקר בין שהוא זול, כפי שניו, הוא הערך הקצוב עליו בפרשה זו׃\quad \rashiDH{ערכך.} כמו ערך, וכפל הכפי״ן, לא ידעתי מאיזה לשון הוא׃}
\threeverse{\arabic{verse}}%Leviticus27:4
{וְאִם\maqqaf נְקֵבָ֖ה הִ֑וא וְהָיָ֥ה עֶרְכְּךָ֖ שְׁלֹשִׁ֥ים שָֽׁקֶל׃}
{וְאִם נוּקְבְּתָא הִיא וִיהֵי פֻּרְסָנַהּ תְּלָתִין סִלְעִין׃}
{And if it be a female, then thy valuation shall be thirty shekels.}{\arabic{verse}}
\threeverse{\arabic{verse}}%Leviticus27:5
{וְאִ֨ם מִבֶּן\maqqaf חָמֵ֜שׁ שָׁנִ֗ים וְעַד֙ בֶּן\maqqaf עֶשְׂרִ֣ים שָׁנָ֔ה וְהָיָ֧ה עֶרְכְּךָ֛ הַזָּכָ֖ר עֶשְׂרִ֣ים שְׁקָלִ֑ים וְלַנְּקֵבָ֖ה עֲשֶׂ֥רֶת שְׁקָלִֽים׃}
{וְאִם מִבַּר חֲמֵישׁ שְׁנִין וְעַד בַּר עֶשְׂרִין שְׁנִין וִיהֵי פּוּרְסָנֵיהּ דִּדְכוּרָא עַסְרִין סִלְעִין וּלְנוּקְבְּתָא עֲשַׂר סִלְעִין׃}
{And if it be from five years old even unto twenty years old, then thy valuation shall be for the male twenty shekels, and for the female ten shekels.}{\arabic{verse}}
\rashi{\rashiDH{ואם מבן חמש שנים.} לא שיהא הנודר קטן שאין בדברי קטן כלום, אלא גדול שאמר ערך קטן הזה שהוא בן חמש שנים עלי׃}
\threeverse{\arabic{verse}}%Leviticus27:6
{וְאִ֣ם מִבֶּן\maqqaf חֹ֗דֶשׁ וְעַד֙ בֶּן\maqqaf חָמֵ֣שׁ שָׁנִ֔ים וְהָיָ֤ה עֶרְכְּךָ֙ הַזָּכָ֔ר חֲמִשָּׁ֥ה שְׁקָלִ֖ים כָּ֑סֶף וְלַנְּקֵבָ֣ה עֶרְכְּךָ֔ שְׁלֹ֥שֶׁת שְׁקָלִ֖ים כָּֽסֶף׃}
{וְאִם מִבַּר יַרְחָא וְעַד בַּר חֲמֵישׁ שְׁנִין וִיהֵי פּוּרְסָנֵיהּ דִּדְכוּרָא חֲמֵישׁ סִלְעִין דִּכְסַף וּלְנוּקְבְּתָא פּוּרְסָנַהּ תְּלָת סִלְעִין דִּכְסַף׃}
{And if it be from a month old even unto five years old, then thy valuation shall be for the male five shekels of silver, and for the female thy valuation shall be three shekels of silver.}{\arabic{verse}}
\threeverse{\arabic{verse}}%Leviticus27:7
{וְ֠אִ֠ם מִבֶּן\maqqaf שִׁשִּׁ֨ים שָׁנָ֤ה וָמַ֙עְלָה֙ אִם\maqqaf זָכָ֔ר וְהָיָ֣ה עֶרְכְּךָ֔ חֲמִשָּׁ֥ה עָשָׂ֖ר שָׁ֑קֶל וְלַנְּקֵבָ֖ה עֲשָׂרָ֥ה שְׁקָלִֽים׃}
{וְאִם מִבַּר שִׁתִּין שְׁנִין וּלְעֵילָא אִם דְּכוּרָא וִיהֵי פּוּרְסָנֵיהּ חֲמֵישׁ עֲשַׂר סִלְעִין וּלְנוּקְבְּתָא עֲשַׂר סִלְעִין׃}
{And if it be from sixty years old and upward: if it be a male, then thy valuation shall be fifteen shekels, and for the female ten shekels.}{\arabic{verse}}
\rashi{\rashiDH{ואם מבן ששים שנה וגו׳.} כשמגיע לימי הזקנה האשה קרובה להחשב כאיש, לפיכך האיש פוחת בהזדקנו יותר משליש בערכו, והאשה אינה פוחתת אלא שליש בערכה, דאמרי אינשי (ערכין יט.) סָבָא בְּבֵיתָא פַּחָא בְּבֵיתָא, סַבְתָּא בְּבֵיתָא סִימָא בְּבֵיתָא וְסִימָנָא טָבָא בְּבֵיתָא׃}
\threeverse{\arabic{verse}}%Leviticus27:8
{וְאִם\maqqaf מָ֥ךְ הוּא֙ מֵֽעֶרְכֶּ֔ךָ וְהֶֽעֱמִידוֹ֙ לִפְנֵ֣י הַכֹּהֵ֔ן וְהֶעֱרִ֥יךְ אֹת֖וֹ הַכֹּהֵ֑ן עַל\maqqaf פִּ֗י אֲשֶׁ֤ר תַּשִּׂיג֙ יַ֣ד הַנֹּדֵ֔ר יַעֲרִיכֶ֖נּוּ הַכֹּהֵֽן׃ \setuma }
{וְאִם מִסְכֵּן הוּא מִפּוּרְסָנֵיהּ וִיקִימִנֵּיהּ קֳדָם כָּהֲנָא וְיִפְרוֹס יָתֵיהּ כָּהֲנָא עַל פּוֹם דְּתַדְבֵּיק יַד נָדְרָא יִפְרְסִנֵּיהּ כָּהֲנָא׃}
{But if he be too poor for thy valuation, then he shall be set before the priest, and the priest shall value him; according to the means of him that vowed shall the priest value him.}{\arabic{verse}}
\rashi{\rashiDH{ואם מך הוא.} שאין ידו משגת ליתן הערך הזה׃\quad \rashiDH{והעמידו.} לנערך לפני הכהן ויעריכנו לפי השגת ידו של מעריך׃\quad \rashiDH{על פי אשר תשיג.} לפי מה שיש לו יסדרנו, וישאיר לו כדי חייו, מטה כר וכסת וכלי אומנות, אם היה חמר משאיר לו חמורו (ערכין כג׃)׃}
\threeverse{\arabic{verse}}%Leviticus27:9
{וְאִ֨ם\maqqaf בְּהֵמָ֔ה אֲשֶׁ֨ר יַקְרִ֧יבוּ מִמֶּ֛נָּה קׇרְבָּ֖ן לַֽיהֹוָ֑ה כֹּל֩ אֲשֶׁ֨ר יִתֵּ֥ן מִמֶּ֛נּוּ לַיהֹוָ֖ה יִֽהְיֶה\maqqaf קֹּֽדֶשׁ׃}
{וְאִם בְּעִירָא דִּיקָרְבוּן מִנַּהּ קוּרְבָּנָא קֳדָם יְיָ כֹּל דְּיִתֵּין מִנֵּיהּ קֳדָם יְיָ יְהֵי קוּדְשָׁא׃}
{And if it be a beast, whereof men bring an offering unto the \lord, all that any man giveth of such unto the \lord\space shall be holy.}{\arabic{verse}}
\rashi{\rashiDH{כל אשר יתן ממנו.} אמר רגלה של זו עולה, דבריו קיימין, ותמכר לצרכי עולה, ודמיה חולין, חוץ מדמי אותו האבר׃}
\threeverse{\arabic{verse}}%Leviticus27:10
{לֹ֣א יַחֲלִיפֶ֗נּוּ וְלֹֽא\maqqaf יָמִ֥יר אֹת֛וֹ ט֥וֹב בְּרָ֖ע אוֹ\maqqaf רַ֣ע בְּט֑וֹב וְאִם\maqqaf הָמֵ֨ר יָמִ֤יר בְּהֵמָה֙ בִּבְהֵמָ֔ה וְהָֽיָה\maqqaf ה֥וּא וּתְמוּרָת֖וֹ יִֽהְיֶה\maqqaf קֹּֽדֶשׁ׃}
{לָא יְחַלְּפִנֵּיהּ וְלָא יַעְבַּר יָתֵיהּ טָב בְּבִישׁ אוֹ בִישׁ בְּטָב וְאִם חַלָּפָא יְחַלֵּיף בְּעִירָא בִּבְעִירָא וִיהֵי הוּא וְחִלּוּפֵיהּ יְהֵי קוּדְשָׁא׃}
{He shall not alter it, nor change it, a good for a bad, or a bad for a good; and if he shall at all change beast for beast, then both it and that for which it is changed shall be holy.}{\arabic{verse}}
\rashi{\rashiDH{טוב ברע.} תם בבעל מום (ת״כ פרק ט, ו)׃\quad \rashiDH{או רע בטוב.}וכל שכן טוב בטוב, ורע ברע (תמורה ט.)׃}
\threeverse{\arabic{verse}}%Leviticus27:11
{וְאִם֙ כׇּל\maqqaf בְּהֵמָ֣ה טְמֵאָ֔ה אֲ֠שֶׁ֠ר לֹא\maqqaf יַקְרִ֧יבוּ מִמֶּ֛נָּה קׇרְבָּ֖ן לַֽיהֹוָ֑ה וְהֶֽעֱמִ֥יד אֶת\maqqaf הַבְּהֵמָ֖ה לִפְנֵ֥י הַכֹּהֵֽן׃}
{וְאִם כָּל בְּעִירָא מְסָאֲבָא דְּלָא יְקָרְבוּן מִנַּהּ קוּרְבָּנָא קֳדָם יְיָ וִיקִים יָת בְּעִירָא קֳדָם כָּהֲנָא׃}
{And if it be any unclean beast, of which they may not bring an offering unto the \lord, then he shall set the beast before the priest.}{\arabic{verse}}
\rashi{\rashiDH{ואם כל בהמה טמאה.} בבעלת מום הכתוב מדבר שהיא טמאה להקרבה, ולמדך הכתוב שאין קדשים תמימים יוצאין לחולין בפדיון, אלא א״כ הוממו (שם לב׃  ת״כ פרשתא ד, א)׃}
\threeverse{\arabic{verse}}%Leviticus27:12
{וְהֶעֱרִ֤יךְ הַכֹּהֵן֙ אֹתָ֔הּ בֵּ֥ין ט֖וֹב וּבֵ֣ין רָ֑ע כְּעֶרְכְּךָ֥ הַכֹּהֵ֖ן כֵּ֥ן יִהְיֶֽה׃}
{וְיִפְרוֹס כָּהֲנָא יָתַהּ בֵּין טָב וּבֵין בִּישׁ כְּפוּרְסַן כָּהֲנָא כֵּן יְהֵי׃}
{And the priest shall value it, whether it be good or bad; as thou the priest valuest it, so shall it be.}{\arabic{verse}}
\rashi{\rashiDH{כערכך הכהן כן יהיה.} לשאר כל אדם הבא לקנותה מיד הקדש׃}
\threeverse{\arabic{verse}}%Leviticus27:13
{וְאִם\maqqaf גָּאֹ֖ל יִגְאָלֶ֑נָּה וְיָסַ֥ף חֲמִישִׁת֖וֹ עַל\maqqaf עֶרְכֶּֽךָ׃}
{וְאִם מִפְרָק יִפְרְקִנַּהּ וְיוֹסֵיף חוּמְשֵׁיהּ עַל פּוּרְסָנֵיהּ׃}
{But if he will indeed redeem it, then he shall add the fifth part thereof unto thy valuation.}{\arabic{verse}}
\rashi{\rashiDH{ואם גאל יגאלנה.} בבעלים החמיר הכתוב, להוסיף חומש, וכן במקדיש בית, וכן במקדיש את השדה, וכן בפדיון מעשר שני, הבעלים מוסיפין חומש, ולא שאר כל אדם (ת״כ שם ז)׃}
\threeverse{\arabic{verse}}%Leviticus27:14
{וְאִ֗ישׁ כִּֽי\maqqaf יַקְדִּ֨שׁ אֶת\maqqaf בֵּית֥וֹ קֹ֙דֶשׁ֙ לַֽיהֹוָ֔ה וְהֶעֱרִיכוֹ֙ הַכֹּהֵ֔ן בֵּ֥ין ט֖וֹב וּבֵ֣ין רָ֑ע כַּאֲשֶׁ֨ר יַעֲרִ֥יךְ אֹת֛וֹ הַכֹּהֵ֖ן כֵּ֥ן יָקֽוּם׃}
{וּגְבַר אֲרֵי יַקְדֵּישׁ יָת בֵּיתֵיהּ קוּדְשָׁא קֳדָם יְיָ וְיִפְרְסִנֵּיהּ כָּהֲנָא בֵּין טָב וּבֵין בִּישׁ כְּמָא דְּיִפְרוֹס יָתֵיהּ כָּהֲנָא כֵּן יְקוּם׃}
{And when a man shall sanctify his house to be holy unto the \lord, then the priest shall value it, whether it be good or bad; as the priest shall value it, so shall it stand.}{\arabic{verse}}
\threeverse{\arabic{verse}}%Leviticus27:15
{וְאִ֨ם\maqqaf הַמַּקְדִּ֔ישׁ יִגְאַ֖ל אֶת\maqqaf בֵּית֑וֹ וְ֠יָסַ֠ף חֲמִישִׁ֧ית כֶּֽסֶף\maqqaf עֶרְכְּךָ֛ עָלָ֖יו וְהָ֥יָה לֽוֹ׃}
{וְאִם דְּאַקְדֵּישׁ יִפְרוֹק יָת בֵּיתֵיהּ וְיוֹסֵיף חוֹמֶשׁ כְּסַף פּוּרְסָנֵיהּ עֲלוֹהִי וִיהֵי לֵיהּ׃}
{And if he that sanctified it will redeem his house, then he shall add the fifth part of the money of thy valuation unto it, and it shall be his.}{\arabic{verse}}
\aliyacounter{חמישי}
\threeverse{\aliya{חמישי\newline (שביעי)}}%Leviticus27:16
{וְאִ֣ם \legarmeh  מִשְּׂדֵ֣ה אֲחֻזָּת֗וֹ יַקְדִּ֥ישׁ אִישׁ֙ לַֽיהֹוָ֔ה וְהָיָ֥ה עֶרְכְּךָ֖ לְפִ֣י זַרְע֑וֹ זֶ֚רַע חֹ֣מֶר שְׂעֹרִ֔ים בַּחֲמִשִּׁ֖ים שֶׁ֥קֶל כָּֽסֶף׃}
{וְאִם מֵחֲקַל אַחְסָנְתֵיהּ יַקְדֵּישׁ גְּבַר קֳדָם יְיָ וִיהֵי פּוּרְסָנֵיהּ לְפוֹם זַרְעֵיהּ בֵּית זְרַע כּוֹר שְׂעָרִין בְּחַמְשִׁין סִלְעִין דִּכְסַף׃}
{And if a man shall sanctify unto the \lord\space part of the field of his possession, then thy valuation shall be according to the sowing thereof; the sowing of a homer of barley shall be valued at fifty shekels of silver.}{\arabic{verse}}
\rashi{\rashiDH{והיה ערכך לפי זרעו.} ולא כפי שוויה, אחת שדה טובה ואחת שדה רעה פדיון הקדשן שוים, בית כור שעורים בחמשים שקלים, כך גזירת הכתוב, והוא שבא לגאלה בתחלת היובל, ואם בא לגאלה באמצעו נותן לפי החשבון, סלע וּפוּנְדְיוֹן לשנה (ערכין כה.), לפי שאינה הקדש אלא למנין שני היובל, שאם נגאלה הרי טוב, ואם לאו הגזבר מוכרה בדמים הללו לאחר ועומדת ביד הלוקח עד היובל כשאר כל השדות המכורות, וכשהיא יוצאה מידו חוזרת לכהנים של אותו משמר שהיובל פוגע בו ומתחלקת ביניהם (שם כח׃), זהו המשפט האמור במקדיש שדה, ועכשיו אפרשנו על סדר המקראות׃}
\threeverse{\arabic{verse}}%Leviticus27:17
{אִם\maqqaf מִשְּׁנַ֥ת הַיֹּבֵ֖ל יַקְדִּ֣ישׁ שָׂדֵ֑הוּ כְּעֶרְכְּךָ֖ יָקֽוּם׃}
{אִם מִשַּׁתָּא דְּיוֹבֵילָא יַקְדֵישׁ חַקְלֵיהּ כְּפוּרְסָנֵיהּ יְקוּם׃}
{If he sanctify his field from the year of jubilee, according to thy valuation it shall stand.}{\arabic{verse}}
\rashi{\rashiDH{אם משנת היובל יקדיש וגו׳.} אם משעברה שנת היובל מיד הקדישה ובא זה לגאלה מיד׃\quad \rashiDH{כערכך יקום.} כערך הזה האמור יהיה, חמשים כסף יתן׃}
\threeverse{\arabic{verse}}%Leviticus27:18
{וְאִם\maqqaf אַחַ֣ר הַיֹּבֵל֮ יַקְדִּ֣ישׁ שָׂדֵ֒הוּ֒ וְחִשַּׁב\maqqaf ל֨וֹ הַכֹּהֵ֜ן אֶת\maqqaf הַכֶּ֗סֶף עַל\maqqaf פִּ֤י הַשָּׁנִים֙ הַנּ֣וֹתָרֹ֔ת עַ֖ד שְׁנַ֣ת הַיֹּבֵ֑ל וְנִגְרַ֖ע מֵֽעֶרְכֶּֽךָ׃}
{וְאִם בָּתַר יוֹבֵילָא יַקְדֵּישׁ חַקְלֵיהּ וִיחַשֵּׁיב לֵיהּ כָּהֲנָא יָת כַּסְפָּא עַל פּוֹם שְׁנַיָּא דְּאִשְׁתְּאַרָא עַד שַׁתָּא דְּיוֹבֵילָא וְיִתְמְנַע מִפּוּרְסָנֵיהּ׃}
{But if he sanctify his field after the jubilee, then the priest shall reckon unto him the money according to the years that remain unto the year of jubilee, and an abatement shall be made from thy valuation.}{\arabic{verse}}
\rashi{\rashiDH{ואם אחר היבל יקדיש.}וכן אם הקדישה משנת היובל ונשתהה ביד גזבר ובא זה לגאלה אחר היובל׃\quad \rashiDH{וחשב לו הכהן את הכסף על פי השנים הנותרות.} כפי חשבון. כיצד הרי קצב דמיה של ארבעים ותשע שנים חמשים שקל, הרי שקל לכל שנה ושקל יתר על כולן, והשקל ארבעים ושמנה פונדיונין, הרי סלע ופונדיון לשנה אלא שחסר פונדיון אחד לכולן, ואמרו רבותינו (בכורות נ.) שאותו פונדיון קַלְבּוֹן לִפְרוֹטְרוֹט, והבא לגאול יתן סלע ופונדיון לכל שנה לשנים הנותרות עד שנת היובל׃\quad \rashiDH{ונגרע מערכך.} מִנְיַן השנים שמשנת היובל, עד שנת הפדיון׃}
\threeverse{\arabic{verse}}%Leviticus27:19
{וְאִם\maqqaf גָּאֹ֤ל יִגְאַל֙ אֶת\maqqaf הַשָּׂדֶ֔ה הַמַּקְדִּ֖ישׁ אֹת֑וֹ וְ֠יָסַ֠ף חֲמִשִׁ֧ית כֶּֽסֶף\maqqaf עֶרְכְּךָ֛ עָלָ֖יו וְקָ֥ם לֽוֹ׃}
{וְאִם מִפְרָק יִפְרוֹק יָת חַקְלָא דְּאַקְדֵּישׁ יָתֵיהּ וְיוֹסֵיף חוֹמֶשׁ כְּסַף פּוּרְסָנֵיהּ עֲלוֹהִי וִיקוּם לֵיהּ׃}
{And if he that sanctified the field will indeed redeem it, then he shall add the fifth part of the money of thy valuation unto it, and it shall be assured to him.}{\arabic{verse}}
\rashi{\rashiDH{ואם גאל יגאל.} המקדיש אותו יוסיף חומש על הקצבה הזאת׃}
\threeverse{\arabic{verse}}%Leviticus27:20
{וְאִם\maqqaf לֹ֤א יִגְאַל֙ אֶת\maqqaf הַשָּׂדֶ֔ה וְאִם\maqqaf מָכַ֥ר אֶת\maqqaf הַשָּׂדֶ֖ה לְאִ֣ישׁ אַחֵ֑ר לֹ֥א יִגָּאֵ֖ל עֽוֹד׃}
{וְאִם לָא יִפְרוֹק יָת חַקְלָא וְאִם זַבֵּין יָת חַקְלָא לִגְבַר אָחֳרָן לָא יִתְפְּרֵיק עוֹד׃}
{And if he will not redeem the field, or if he have sold the field to another man, it shall not be redeemed any more.}{\arabic{verse}}
\rashi{\rashiDH{ואם לא יגאל את השדה.} המקדיש׃\quad ואם מכר. הגזבר (ערכין כה׃)׃\quad \rashiDH{את השדה לאיש אחר לא יגאל עוד.} לשוב ביד המקדיש׃}
\threeverse{\arabic{verse}}%Leviticus27:21
{וְהָיָ֨ה הַשָּׂדֶ֜ה בְּצֵאת֣וֹ בַיֹּבֵ֗ל קֹ֛דֶשׁ לַֽיהֹוָ֖ה כִּשְׂדֵ֣ה הַחֵ֑רֶם לַכֹּהֵ֖ן תִּהְיֶ֥ה אֲחֻזָּתֽוֹ׃}
{וִיהֵי חַקְלָא בְּמִפְּקֵיהּ בְּיוֹבֵילָא קוּדְשָׁא קֳדָם יְיָ כַּחֲקַל חֶרְמָא לְכָהֲנָא תְּהֵי אַחְסָנְתֵיהּ׃}
{But the field, when it goeth out in the jubilee, shall be holy unto the \lord, as a field devoted; the possession thereof shall be the priest’s.}{\arabic{verse}}
\rashi{\rashiDH{והיה השדה בצאתו ביבל.} מיד הלוקחו מן הגזבר, כדרך שאר שדות היוצאות מיד לוקחיהם ביובל׃\quad \rashiDH{קדש לה׳.} לא שישוב להקדש בדק הבית ליד הגזבר, אלא כשדה החרם הנתון לכהנים, שנאמר כָּל חֵרֶם בְּיִשְׂרָאֵל לְךָ יִהְיֶה (במדבר יח, יד), אף זו תתחלק לכהנים של אותו משמר שיום הכפורים של יובל פוגע בו (ערכין כח׃)׃}
\aliyacounter{ששי}
\threeverse{\aliya{ששי}}%Leviticus27:22
{וְאִם֙ אֶת\maqqaf שְׂדֵ֣ה מִקְנָת֔וֹ אֲשֶׁ֕ר לֹ֖א מִשְּׂדֵ֣ה אֲחֻזָּת֑וֹ יַקְדִּ֖ישׁ לַֽיהֹוָֽה׃}
{וְאִם יָת חֲקַל זְבִינוֹהִי דְּלָא מֵחֲקַל אַחְסָנְתֵיהּ יַקְדֵּישׁ קֳדָם יְיָ׃}
{And if he sanctify unto the \lord\space a field which he hath bought, which is not of the field of his possession;}{\arabic{verse}}
\rashi{\rashiDH{ואם את שדה מקנתו וגו׳.} חלוק יש בין שדה מקנה לשדה אחוזה, ששדה מקנה לא תתחלק לכהנים ביובל, לפי שאינו יכול להקדישה אלא עד היובל, שהרי ביובל היתה עתידה לצאת מידו ולשוב לבעלים, לפיכך, אם בא לגאלה, יגאל בדמים הללו הקצובים לשדה אחוזה, ואם לא יגאל, וימכרנה גזבר לאחר, או אם לא יגאל הוא, בשנת היובל ישוב השדה לאשר קנהו מאתו, אותו שהקדישה, ופן תאמר לאשר קנהו הלוקח הזה האחרון מאתו, וזהו הגזבר, לכך הוצרך לומר לאשר לו אחוזת הארץ, מירושת אבות, וזהו בעלים הראשונים שמכרוה למקדיש (שם כו׃)׃}
\threeverse{\arabic{verse}}%Leviticus27:23
{וְחִשַּׁב\maqqaf ל֣וֹ הַכֹּהֵ֗ן אֵ֚ת מִכְסַ֣ת הָֽעֶרְכְּךָ֔ עַ֖ד שְׁנַ֣ת הַיֹּבֵ֑ל וְנָתַ֤ן אֶת\maqqaf הָעֶרְכְּךָ֙ בַּיּ֣וֹם הַה֔וּא קֹ֖דֶשׁ לַיהֹוָֽה׃}
{וִיחַשֵּׁיב לֵיהּ כָּהֲנָא יָת נְסִיב פּוּרְסָנֵיהּ עַד שַׁתָּא דְּיוֹבֵילָא וְיִתֵּין יָת פּוּרְסָנֵיהּ בְּיוֹמָא הַהוּא קוּדְשָׁא קֳדָם יְיָ׃}
{then the priest shall reckon unto him the worth of thy valuation unto the year of jubilee; and he shall give thy valuation in that day, as a holy thing unto the \lord.}{\arabic{verse}}
\threeverse{\arabic{verse}}%Leviticus27:24
{בִּשְׁנַ֤ת הַיּוֹבֵל֙ יָשׁ֣וּב הַשָּׂדֶ֔ה לַאֲשֶׁ֥ר קָנָ֖הוּ מֵאִתּ֑וֹ לַאֲשֶׁר\maqqaf ל֖וֹ אֲחֻזַּ֥ת הָאָֽרֶץ׃}
{בְּשַׁתָּא דְּיוֹבֵילָא יְתוּב חַקְלָא לִדְזַבְנֵיהּ מִנֵּיהּ לִדְדִּילֵיהּ אַחְסָנַת אַרְעָא׃}
{In the year of jubilee the field shall return unto him of whom it was bought, even to him to whom the possession of the land belongeth.}{\arabic{verse}}
\threeverse{\arabic{verse}}%Leviticus27:25
{וְכׇ֨ל\maqqaf עֶרְכְּךָ֔ יִהְיֶ֖ה בְּשֶׁ֣קֶל הַקֹּ֑דֶשׁ עֶשְׂרִ֥ים גֵּרָ֖ה יִהְיֶ֥ה הַשָּֽׁקֶל׃}
{וְכָל פּוּרְסָנֵיהּ יְהֵי בְּסִלְעֵי קוּדְשָׁא עַסְרִין מָעִין יְהֵי סִלְעָא׃}
{And all thy valuations shall be according to the shekel of the sanctuary; twenty gerahs shall be the shekel.}{\arabic{verse}}
\rashi{\rashiDH{וכל ערכך יהיה בשקל הקדש.} כל ערכך שכתוב בו שקלים יהיה בשקל הקדש׃\quad \rashiDH{עשרים גרה.} עשרים מעות, כך היו מתחלה, ולאחר מכאן הוסיפו שתות, ואמרו רבותינו (בכורות נ.) שש מעה כסף דינר, עשרים וארבע מעות לסלע׃}
\threeverse{\arabic{verse}}%Leviticus27:26
{אַךְ\maqqaf בְּכ֞וֹר אֲשֶׁר\maqqaf יְבֻכַּ֤ר לַֽיהֹוָה֙ בִּבְהֵמָ֔ה לֹֽא\maqqaf יַקְדִּ֥ישׁ אִ֖ישׁ אֹת֑וֹ אִם\maqqaf שׁ֣וֹר אִם\maqqaf שֶׂ֔ה לַֽיהֹוָ֖ה הֽוּא׃}
{בְּרַם בּוּכְרָא דְּיִתְבַּכַּר קֳדָם יְיָ בִּבְעִירָא לָא יַקְדֵּישׁ גְּבַר יָתֵיהּ אִם תּוֹר אִם אִמַּר דַּייָ הוּא׃}
{Howbeit the firstling among beasts, which is born as a firstling to the \lord, no man shall sanctify it; whether it be ox or sheep, it is the \lord’S.}{\arabic{verse}}
\rashi{\rashiDH{לא יקדיש איש אתו.} לשם קרבן אחר, לפי שאינו שלו׃}
\threeverse{\arabic{verse}}%Leviticus27:27
{וְאִ֨ם בַּבְּהֵמָ֤ה הַטְּמֵאָה֙ וּפָדָ֣ה בְעֶרְכֶּ֔ךָ וְיָסַ֥ף חֲמִשִׁת֖וֹ עָלָ֑יו וְאִם\maqqaf לֹ֥א יִגָּאֵ֖ל וְנִמְכַּ֥ר בְּעֶרְכֶּֽךָ׃}
{וְאִם בִּבְעִירָא מְסָאֲבָא וְיִפְרוֹק בְּפוּרְסָנֵיהּ וְיוֹסֵיף חוּמְשֵׁיהּ עֲלוֹהִי וְאִם לָא יִתְפְּרֵיק וְיִזְדַּבַּן בְּפוּרְסָנֵיהּ׃}
{And if it be of an unclean beast, then he shall ransom it according to thy valuation, and shall add unto it the fifth part thereof; or if it be not redeemed, then it shall be sold according to thy valuation.}{\arabic{verse}}
\rashi{\rashiDH{ואם בבהמה הטמאה וגו׳.} אין המקרא הזה מוסב על הבכור, שאין לומר בבכור בהמה טמאה ופדה בערכך, וחמור אין זה, שהרי אין פדיון פטר חמור אלא טלה, והוא מתנה לכהן ואינו להקדש, אלא הכתוב מוסב על ההקדש, שהכתוב שלמעלה דבר בפדיון בהמה טהורה שהוממה, וכאן דבר במקדיש בהמה טמאה לבדק הבית (מנחות קא.)׃\quad \rashiDH{ופדה בערכך.} כפי מה שיעריכנה הכהן׃\quad \rashiDH{ואם לא יגאל.} ע״י בעלים (ת״כ פרק ב, ב)׃\quad \rashiDH{ונמכר בערכך.}לאחרים׃}
\threeverse{\arabic{verse}}%Leviticus27:28
{אַךְ\maqqaf כׇּל\maqqaf חֵ֡רֶם אֲשֶׁ֣ר יַחֲרִם֩ אִ֨ישׁ לַֽיהֹוָ֜ה מִכׇּל\maqqaf אֲשֶׁר\maqqaf ל֗וֹ מֵאָדָ֤ם וּבְהֵמָה֙ וּמִשְּׂדֵ֣ה אֲחֻזָּת֔וֹ לֹ֥א יִמָּכֵ֖ר וְלֹ֣א יִגָּאֵ֑ל כׇּל\maqqaf חֵ֕רֶם קֹֽדֶשׁ\maqqaf קׇדָשִׁ֥ים ה֖וּא לַיהֹוָֽה׃}
{בְּרַם כָּל חֶרְמָא דְּיַחְרֵים גְּבַר קֳדָם יְיָ מִכָּל דְּלֵיהּ מֵאֲנָשָׁא וּבְעִירָא וּמֵחֲקַל אַחְסָנְתֵיהּ לָא יִזְדַּבַּן וְלָא יִתְפְּרֵיק כָּל חֶרְמָא קוֹדֶשׁ קוּדְשִׁין הוּא קֳדָם יְיָ׃}
{Notwithstanding, no devoted thing, that a man may devote unto the \lord\space of all that he hath, whether of man or beast, or of the field of his possession, shall be sold or redeemed; every devoted thing is most holy unto the \lord.}{\arabic{verse}}
\rashi{\rashiDH{אך כל חרם וגו׳.} נחלקו רבותינו בדבר (ערכין כח׃), יש אומרים סתם חרמים להקדש (שם ה), ומה אני מקיים כל חרם בישראל לך יהיה (במדבר יח, יד), בחרמי כהנים, שפירש ואמר הרי זה חרם לכהן, ויש שאמרו סתם חרמים לכהנים׃\quad \rashiDH{לא ימכר ולא יגאל.} אלא ינתן לכהן, לדברי האומר סתם חרמים לכהנים, מפרש מקרא זה בסתם חרמים, והאומר סתם חרמים לבדק הבית, מפרש מקרא זה, בחרמי כהנים, שהכל מודים שחרמי כהנים אין להם פדיון (ערכין כח׃), עד שיבואו ליד כהן, וחרמי גבוה נפדים׃\quad \rashiDH{כל חרם קדש קדשים הוא.} האומר סתם חרמים לבדק הבית, מביא ראיה מכאן, והאומר סתם חרמים לכהנים, מפרש כל חרם קדש קדשים הוא לה׳, ללמד שחרמי כהנים חלים על קדשי קדשים, ועל קדשים קלים, ונותן לכהן כמו ששנינו במסכת ערכין (כח׃) אם נדר, נותן דמיהם, ואם נדבה, נותן את טובתה׃\quad \rashiDH{מאדם.}כגון שהחרים עבדיו ושפחותיו הכנענים (שם כח.)׃}
\aliyacounter{שביעי}
\threeverse{\aliya{שביעי}}%Leviticus27:29
{כׇּל\maqqaf חֵ֗רֶם אֲשֶׁ֧ר יׇחֳרַ֛ם מִן\maqqaf הָאָדָ֖ם לֹ֣א יִפָּדֶ֑ה מ֖וֹת יוּמָֽת׃}
{כָּל חֶרְמָא דְּיִתַּחְרַם מִן אֲנָשָׁא לָא יִתְפְּרֵיק אִתְקְטָלָא יִתְקְטִיל׃}
{None devoted, that may be devoted of men, shall be ransomed; he shall surely be put to death.}{\arabic{verse}}
\rashi{\rashiDH{כל חרם אשר יחרם וגו׳.} היוצא ליהרג ואמר אחד ערכו עלי, לא אמר כלום (שם ו.  ת״כ שם ז)׃\quad \rashiDH{מות יומת.} הרי הולך למות, לפיכך לא יפדה, אין לו לא דמים, ולא ערך׃}
\threeverse{\arabic{verse}}%Leviticus27:30
{וְכׇל\maqqaf מַעְשַׂ֨ר הָאָ֜רֶץ מִזֶּ֤רַע הָאָ֙רֶץ֙ מִפְּרִ֣י הָעֵ֔ץ לַיהֹוָ֖ה ה֑וּא קֹ֖דֶשׁ לַֽיהֹוָֽה׃}
{וְכָל מַעְשַׂר אַרְעָא מִזַּרְעָא דְּאַרְעָא מִפֵּירֵי אִילָנָא דַּייָ הוּא קוּדְשָׁא קֳדָם יְיָ׃}
{And all the tithe of the land, whether of the seed of the land, or of the fruit of the tree, is the \lord’S; it is holy unto the \lord.}{\arabic{verse}}
\rashi{\rashiDH{וכל מעשר הארץ.} במעשר שני הכתוב מדבר׃\quad \rashiDH{מזרע הארץ.} דגן׃\quad \rashiDH{מפרי העץ.} תירוש ויצהר׃\quad \rashiDH{לה׳ הוא.} קנאו השם ומשולחנו צוה לך לעלות ולאכול בירושלים, כמו שנאמר וְאָכַלְתָּ לִפְנֵי ה׳ אֱלֹהֶיךָ מַעְשַׂר דְּגָנְךָ תִּירשְׁךָ וגו׳ (דברים יד, כג  קידושין נג.)׃}
\threeverse{\arabic{verse}}%Leviticus27:31
{וְאִם\maqqaf גָּאֹ֥ל יִגְאַ֛ל אִ֖ישׁ מִמַּֽעַשְׂר֑וֹ חֲמִשִׁית֖וֹ יֹסֵ֥ף עָלָֽיו׃}
{וְאִם מִפְרָק יִפְרוֹק גְּבַר מִמַּעַשְׂרֵיהּ חוּמְשֵׁיהּ יוֹסֵיף עֲלוֹהִי׃}
{And if a man will redeem aught of his tithe, he shall add unto it the fifth part thereof.}{\arabic{verse}}
\rashi{\rashiDH{ממעשרו.} ולא ממעשר חבירו, הפודה מעשר של חבירו אין מוסיף חומש (שם כד). ומה היא גאולתו, כדי להתירו באכילה בכל מקום, והמעות יעלה ויאכל בירושלים, כמו שכתוב וְנָתַתָּה בַּכָּסֶף וגו׳׃}
\threeverse{\aliya{מפטיר}}%Leviticus27:32
{וְכׇל\maqqaf מַעְשַׂ֤ר בָּקָר֙ וָצֹ֔אן כֹּ֥ל אֲשֶׁר\maqqaf יַעֲבֹ֖ר תַּ֣חַת הַשָּׁ֑בֶט הָֽעֲשִׂירִ֕י יִֽהְיֶה\maqqaf קֹּ֖דֶשׁ לַֽיהֹוָֽה׃}
{וְכָל מַעְשַׂר תּוֹרִין וְעָאן כֹּל דְּיִעְבַּר תְּחוֹת חוּטְרָא עֲשִׂירָאָה יְהֵי קוּדְשָׁא קֳדָם יְיָ׃}
{And all the tithe of the herd or the flock, whatsoever passeth under the rod, the tenth shall be holy unto the \lord.}{\arabic{verse}}
\rashi{\rashiDH{תחת השבט.} כשבא לעשרן מוציאן בפתח זה אחר זה, והעשירי מכה בשבט צבועה בסקרא להיות ניכר שהוא מעשר, כן עושה לטלאים ועגלים של כל שנה ושנה (בכורות נח׃)׃\quad \rashiDH{יהיה קדש.}ליקרב למזבח דמו ואמוריו, והבשר נאכל לבעלים, שהרי לא נמנה עם שאר מתנות כהונה, ולא מצינו שיהא בשרו ניתן לכהנים׃}
\threeverse{\arabic{verse}}%Leviticus27:33
{לֹ֧א יְבַקֵּ֛ר בֵּֽין\maqqaf ט֥וֹב לָרַ֖ע וְלֹ֣א יְמִירֶ֑נּוּ וְאִם\maqqaf הָמֵ֣ר יְמִירֶ֔נּוּ וְהָֽיָה\maqqaf ה֧וּא וּתְמוּרָת֛וֹ יִֽהְיֶה\maqqaf קֹּ֖דֶשׁ לֹ֥א יִגָּאֵֽל׃}
{לָא יְבַקַּר בֵּין טָב לְבִישׁ וְלָא יְחַלְּפִנֵּיהּ וְאִם חַלָּפָא יְחַלְּפִנֵּיהּ וִיהֵי הוּא וְחִלּוּפֵיהּ יְהֵי קוּדְשָׁא לָא יִתְפְּרֵיק׃}
{He shall not inquire whether it be good or bad, neither shall he change it; and if he change it at all, then both it and that for which it is changed shall be holy; it shall not be redeemed.}{\arabic{verse}}
\rashi{\rashiDH{לא יבקר וגו׳.} לפי שנאמר וְכֹל מִבְחַר נִדְרֵיכֶם (שם יב, יא), יכול יהא בורר ומוציא את היפה, תלמוד לומר לא יבקר בין טוב לרע, בין תם בין בעל מום, חלה עליו קדושה, ולא שיקריב בעל מום, אלא יאכל בתורת מעשר, ואסור ליגזז וליעבד (בכורות יד׃)׃}
\threeverse{\aliya{\Hebrewnumeral{78}}}%Leviticus27:34
{אֵ֣לֶּה הַמִּצְוֺ֗ת אֲשֶׁ֨ר צִוָּ֧ה יְהֹוָ֛ה אֶת\maqqaf מֹשֶׁ֖ה אֶל\maqqaf בְּנֵ֣י יִשְׂרָאֵ֑ל בְּהַ֖ר סִינָֽי׃}
{אִלֵּין פִּקּוֹדַיָּא דְּפַקֵּיד יְיָ יָת מֹשֶׁה לְוָת בְּנֵי יִשְׂרָאֵל בְּטוּרָא דְּסִינָי׃}
{These are the commandments, which the \lord\space commanded Moses for the children of Israel in mount Sinai.}{\arabic{verse}}

\engnote{The Haftarah is Jeremiah 16:19\verserangechar 17:14 on page \pageref{haft_33}. }
\newperek
