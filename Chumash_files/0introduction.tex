\frontmatter
\pagenumbering{roman}

\title{\texttitle\\
\engtext{\engtitle}
}

\author{\engtext{Formatting by Nathan Kasimer}}

\date{\engtext{With the Targum, 1917 JPS Translation, and Rashi}}

\maketitle

\begin{minipage}[b][\textheight][b]{\textwidth}\englishfont	
	\begin{english}
		\vfill
		©Nathan Kasimer, 2021 (5782). This text may be re-used under the terms of the Creative Commons Sharealike 2.0 license (CC-BY-SA), the terms of which are available here: https://creativecommons.org/licenses/by-sa/2.0/. This book includes the following texts:
		\begin{itemize}
			\item Miqra According to the Masora
			\begin{itemize}
				\item License: CC-BY-SA
				\item Source: https://he.wikisource.org/wiki/\hspace{-0.4em}\begin{hebrew}מקרא\textunderscore על\textunderscore פי\textunderscore מסורה\end{hebrew}
			\end{itemize}
			\item Targum Onkelos, vocalized according to the Yemenite Taj
			\begin{itemize}
				\item License: CC-BY-SA
				\item Source: https://he.wikisource.org/wiki/\hspace{-0.4em}\begin{hebrew}תרגום\textunderscore אונקלוס\end{hebrew}
			\end{itemize}
			\item Rashi Chumash, Metsudah Publications, 2009
			\begin{itemize}
				\item License: CC-BY
			\end{itemize}
		\end{itemize}
		All these texts were retreived from Sefaria.  It was typeset and formatted using \LaTeX , using the Shlomo font by Shlomo Orbach from https://sites.google.com/site/orlaeinayim/download and the EB Garamond font by Georg Duffner from http://www.georgduffner.at/ebgaramond/index.html. Both of these were used under the terms of the SIL Open Font License.
		\clearpage
		
	\end{english}
\end{minipage}

\tableofcontents
\clearpage
\addcontentsline{toc}{chapter}{Introduction}	
\begin{minipage}{\textwidth}\englishfont
	
\begin{english}
\vspace{16pt}
\Large Introduction\vspace{12pt}\\
\normalsize
This \d{h}umash is intended primarily for learning \textit{Shnayim Mikra veEchad Targum}, but to be versatile enough to be usable in a synagogue. To that end, it includes multiple texts used as the ``Targum"\textemdash Targum Onkelos itself, the commentary of Rashi, and a translation into English.  For Haftarot, special Maftir portions, and Shabbat Min\d{h}a readings, only Hebrew and English are printed, since these sections are included for ease of use in synagogues rather than for study use.\\

The text of the Torah itself is from the \textit{Mikra al pi Masorah} project. The text was selected for its open licensing, extensive source documentation, and accuracy in presenting the masoretic text.  The text of Targum is the Wikisource Targum, which is based on Yemenite texts (particularly the 1901 edition of the Taj). It was selected for its accurate vocalization.
%I hope to review and tweak the text in accordance with printed scholarly editions which are based on Yemenite texts, such as Bar Ilan, Mossad HaRav Kook \textit{Torat \d{H}ayyim}, and Sperber's critical edition of the Targum.
The English translation is the 1917 JPS translation, and text of Rashi is from the Metzudah edition.  All these texts were retrieved from Sefaria. Data for aliya divisions, haftarot, and lengths of parshiyot was pulled from Hebcal. Information on which special \textit{Maftir} and \textit{Haftarot} can occur on which Sabbaths is from the Koren Shabbat Chumash.\\

This text has minimal notations about various textual differences in the Masoretic text of the Torah. For information where this text varies from others, see the notes of the \textit{Mikra al pi Masorah} project, or notes in \textit{Min\d{h}at Shai}. The \textit{Mikra al pi Masorah} project also has information in its notes about variations in customs about where to divide aliyot. Kamatz Katan is indicated with a special symbol for the text of the Torah, but not in the text of the Targum.\\

I hope this text will be helpful to those who use it.

\end{english}

\end{minipage}


%\raggedright\chapter{Acknowledgments}
%
%\begin{minipage}{\textwidth}\englishfont\raggedright\beginL
%	
%The Hebrew text is from the "Mikra al pi Masorah`` project.  The Targum text is the text on Sefaria according to Yemenite editions.  The English is the 1917 JPS translation.  The Rashi text is from the Metzudah edition.
%
%\endL	
%\end{minipage}