%\setmainlanguage{english}
%\setotherlanguage{hebrew}

\fancyhead[C]{\englishfont \textsc{Appendix}}

\engsection{Blessings for Torah Reading}
\newcommand{\engcite}[2]{(\textit{#1} #2)}

\begin{center}
	\begin{large}
		\englishinst{The person with the aliya touches the beginning spot of the aliya with their tzitzit and kisses it, and says:}
		בָּרְכוּ אֶת־יְיָ הַמְבֹרָךְ׃\\
		
		\vspace{6pt}	\englishinst{The congregation responds:}
		בָּרוּךְ יְיָ הַמְבֹרָךְ לְעוֹלָם וָעֶד:\\
		
		\vspace{6pt}\englishinst{The person with the aliya repeats after them:}
		בָּרוּךְ יְיָ הַמְבֹרָךְ לְעוֹלָם וָעֶד:\\
		
		
		
		\vspace{6pt}
		\englishinst{The person with the aliya continues:}
		בָּרוּךְ אַתָּה יְיָ אֱלֹהֵֽינוּ מֶֽלֶךְ הָעוֹלָם אֲשֶׁר בָּֽחַר בָּֽנוּ מִכׇּל־הָעַמִּים
		וְנָֽתַן לָֽנוּ אֶת־תּוֹרָתוֹ׃ בָּרוּךְ אַתָּה יְיָ נוֹתֵן הַתּוֹרָה׃\\
		\vspace{8pt}
		\englishinst{After the reading, the person with the aliya touches the spot the aliya ended with their tzitzit, kisses it, and blesses as follows:}
		בָּרוּךְ אַתָּה יְיָ אֱלֹהֵֽינוּ מֶֽלֶךְ הָעוֹלָם אֲשֶׁר נָֽתַן לָֽנוּ תּוֹרַת אֱמֶת
		וְחַיֵּי עוֹלָם נָטַע בְּתוֹכֵֽנוּ׃ בָּרוּךְ אַתָּה יְיָ נוֹתֵן הַתּוֹרָה׃
	\end{large}
\end{center}

\engsection{\d{H}atzi Kaddish}

\englishinst{After the seventh aliya, the reader leads Kaddish.}
\begin{large}
	יִתְגַּדַּל וְיִתְקַדַּשׁ שְׁמֵיהּ רַבָּא (\kahal אָמֵן)
	בְּעָלְמָא דִּי בְרָא כִרְעוּתֵהּ וְיַמְלִיךְ מַלְכוּתֵהּ בְּחַיֵּיכוֹן וּבְיוֹמֵיכוֹן וּבְחַיֵּי דְכׇל־בֵּית יִשְׂרָאֵל בַּעֲגָלָא וּבִזְמַן קָרִיב׃ וְאִמְרוּ אָמֵן׃\\
	\textbf{\kahal
		אָמֵן׃ יְהֵא שְׁמֵיהּ רַבָּא מְבָרַךְ לְעָלַם וּלְעָלְמֵי עָלְמַיָּא}\\
	יִתְבָּרַךְ וְיִשְׁתַּבַּח וְיִתְפָּאַר וְיִתְרוֹמַם וְיִתְנַשֵּׂא וְיִתְהַדַּר וְיִתְעַלֶּה וְיִתְהַלַּל שְׁמֵיהּ דְּקֻדְשָׁא
	\textbf{בְּרִיךְ הוּא}
	׃ *לְעֵֽלָּא מִן כׇּל־בִּרְכָתָא
	(*\textsf{\footnotesize בעשי״ת׃}
	לְעֵֽלָּא לְעֵֽלָּא מִכׇּל־בִּרְכָתָא) וְשִׁירָתָא תֻּשְׁבְּחָתָא וְנֶחָמָתָא דַּאֲמִירָן בְּעָלְמָא וְאִמְרוּ אָמֵן׃ (\kahal אָמֵן)
\end{large}

\engsection{Blessings for Haftarah}

\begin{large}
	\englishinst{Before the Haftarah:}
	בָּר֙וּךְ אַתָּ֤ה יְ֙יָ אֱלֹ֙הֵֽינוּ֙ מֶ֣לֶךְ הָעוֹלָ֔ם אֲשֶׁ֤ר בָּחַר֙ בִּנְבִיאִ֣ים טוֹבִ֔ים וְרָצָ֥ה בְדִבְרֵיהֶ֖ם הַנֶּֽאֱמָרִ֣ים בֶּאֱמֶ֑ת בָּר֨וּךְ אַתָּ֜ה יְיָ֗ הַבּוֹחֵר֚ בַּתּוֹרָה֙ וּבְמֹשֶׁ֣ה עַבְדּ֔וֹ וּבְיִשְׂרָאֵ֣ל עַמּ֔וֹ וּבִנְבִיאֵ֥י הָֽאֱמֶ֖ת וָצֶֽדֶק׃\\
	
	\englishinst{After the Haftarah:}
	בָּרוּךְ אַתָּה יְיָ אֱלֹהֵֽינוּ מֶֽלֶךְ הָעוֹלָם צוּר כׇּל־הָעוֹלָמִים צַדִּיק בְּכׇל־הַדּוֹרוֹת הָאֵל הַנֶּאֱמָן הָאוֹמֵר וְעוֹשֶׂה הַמְדַבֵּר וּמְקַיֵּם שֶׁכׇּל־דְּבָרָיו אֱמֶת וָצֶֽדֶק׃ נֶאֱמָן אַתָּה הוּא יְיָ אֱלֹהֵֽינוּ וְנֶאֱמָנִים דְּבָרֶֽיךָ וְדָבָר אֶחָד מִדְּבָרֶֽיךָ אָחוֹר לֹא יָשׁוּב רֵיקָם כִּי אֵל מֶֽלֶךְ נֶאֱמָן וְרַחֲמָן אַֽתָּה׃ בָּרוּךְ אַתָּה יְיָ הָאֵל הַנֶּאֱמָן בְּכׇל־דְּבָרָיו׃\\
	רַחֵם עַל צִיּוֹן כִּי הִיא בֵּית חַיֵּֽינוּ וְלַעֲלֽוּבַת נֶֽפֶשׁ תּוֹשִֽׁיעַ בִּמְהֵרָה בְיָמֵֽינוּ׃ בָּרוּךְ אַתָּה יְיָ מְשַׂמֵּֽחַ צִיּוֹן בְּבָנֶֽיהָ׃\\
	שַׂמְּחֵֽנוּ יְיָ אֱלֹהֵֽינוּ בְּאֵלִיָּֽהוּ הַנָּבִיא עַבְדֶּֽךָ וּבְמַלְכוּת בֵּית דָּוִד מְשִׁיחֶֽךָ בִּמְהֵרָה יָבֹא וְיָגֵל לִבֵּֽנוּ עַל כִּסְאוֹ לֹא יֵשֵׁב זָר וְלֹא יִנְחֲלוּ עוֹד אֲחֵרִים אֶת־כְּבוֹדוֹ כִּי בְשֵׁם קׇדְשְׁךָ נִשְׁבַּעְתָּ לוֹ שֶׁלֹּא יִכְבֶּה נֵרוֹ לְעוֹלָם וָעֶד׃ בָּרוּךְ אַתָּה יְיָ מָגֵן דָּוִד׃\\	
	
	\englishinst{On Festivals and \d{H}ol HaMo'ed Sukkot, continue below.  On Shabbat, conclude as follows:}
	עַל הַתּוֹרָה וְעַל הָעֲבוֹדָה וְעַל הַנְּבִיאִים וְעַל יוֹם הַשַּׁבָּת הַזֶּה שֶׁנָּתַֽתָּ לָֽנוּ יְיָ אֱלֹהֵֽינוּ לִקְדֻשָּׁה וְלִמְנוּחָה לְכָבוֹד וּלְתִפְאָֽרֶת׃ עַל הַכֹּל יְיָ אֱלֹהֵֽינוּ אֲנַֽחְנוּ מוֹדִים לָךְ וּמְבָרְכִים אוֹתָךְ יִתְבָּרַךְ שִׁמְךָ בְּפִי כׇל־חַי תָּמִיד לְעוֹלָם וָעֶד׃ בָּרוּךְ אַתָּה יְיָ מְקַדֵּשׁ הַשַּׁבָּת׃\\
	
	\englishinst{On Festivals and \d{H}ol HaMo'ed Sukkot conclude as follows. On Shabbat include the words in parentheses:}
	עַל הַתּוֹרָה וְעַל הָעֲבוֹדָה וְעַל הַנְּבִיאִים וְעַל יוֹם (הַשַּׁבָּת וְעַל יוֹם)\\
	
	\begin{tabular}{c | c | c | c}
		
		\enginline{Pesa\d{h}}&\enginline{Shavu'ot}&\enginline{Sukkot}&\enginline{Shemini Atzeret}\\	
		חַג הַמַּצּוֹת & חַג הַשָּׁבֻעוֹת & חַג הַסֻּכּוֹת & שְׁמִינִי חַג הָעֲצֶֽרֶת\\ 
	\end{tabular}
	
	הַזֶּה שֶׁנָּתַֽתָּ לָֽנוּ יְיָ אֱלֹהֵֽינוּ (לִקְדֻשָּׁה וְלִמְנוּחָה) לְכָבוֹד וּלְתִפְאָֽרֶת׃ עַל הַכֹּל יְיָ אֱלֹהֵֽינוּ אֲנַֽחְנוּ מוֹדִים לָךְ וּמְבָרְכִים אוֹתָךְ יִתְבָּרַךְ שִׁמְךָ בְּפִי כׇל־חַי תָּמִיד לְעוֹלָם וָעֶד׃ בָּרוּךְ אַתָּה יְיָ מְקַדֵּשׁ (הַשַּׁבָּת וְ)יִשְׂרָאֵל וְהַזְּמַנִים׃
\end{large}
\newpage
\engsection{What Invalidates a Sefer Torah}

\englishcontent{
The following are considered errors in the Sefer Torah \engcite{Kitzur Shul\d{h}an Arukh}{24:1}:}\\
%\begin{itemize}
	\engitem{A missing letter that changes the meaning of the word, including grammatical gender. A change of spelling that does not affect the meaning does not invalidate the Torah, though it should be fixed. Note that\hebword{ י } can be part of the root even if it does not have an independent vowel associated with it, in which case it missing the letter would invalidate the Torah.  For example, if in the phrase \hebword{אל תירא הגר} it was spelled \hebword{תרא} the Torah is invalid \engcite{Kitzur Shul\d{h}an Arukh}{24:1}}
	\engitem{An extra letter (unless the mistake is between a \textit{\d{h}aser} and \textit{malei} spelling)}
	\engitem{One letter split such that it looks like two letters}
	\engitem{Two letters joined together to look like one letter}
	\engitem{One letter substituted for another}
	\engitem{An incorrect paragraph division, either a division missing, a division where there should be none, or thee wrong type of division}
	\engitem{The majority of the seam between two sheets is torn (though if no other Torah is available it may be used if the tear is in a different book of the Torah.  If it is in that book, it may be used if there is no other Torah if five seams remain)}
	\engitem{Wax or a similar substance obscures the words of the portion being read}
%\end{itemize}

\englishcontent{In a case of doubt whether a letter is formed properly (substituted, joined, or split), a child is shown the doubtful letter.  If they identify the letter, the Torah may be used, if not it is invalid.}

\engsection{When an Invalidation is Found}

\englishcontent{When a problem in a Sefer Torah is discovered between aliyot, the reading is continued with the next aliya from a kosher Sefer Torah from the point the previous aliya ended. If a problem is discovered in the middle of an aliya, a kosher Torah is taken out and the reading is resumed without interruption. If the error occurred in the middle of a sentence, the reading in the kosher Torah should resume from the beginning of that sentence.  No new initial berakha on the aliya should be recited (so long as the oleh does not do any unnecessary conversation while the scrolls are swapped), and the concluding blessing should be recited as usual on the kosher Sefer Torah. If there are less than three verses remaining in the aliya, the reader should go back and repeat verses from the kosher Sefer Torah so that three verses are read from the kosher Sefer Torah \engcite{Kitzur Shul\d{h}an Arukh}{24:8}.}\\

\englishcontent{The following are exceptions to the above:}\nopagebreak

	\engitem{If the mistake is in a place where the aliya may be ended (three or more verses from a paragraph break and not in the curses), customs vary. Some end the aliya there, the oleh recites the blessing, and the next aliya begins from the kosher Sefer Torah.  This is the recommendation of the Rema \engcite{Peninei Halakha}{22:2 citing OC 146}.}
	\engitem{If the problem is discovered in Maftir, a new Torah need not be brought out.  Instead, the Maftir is completed as usual, but no concluding berakha on the aliya is recited.  This applies only when the Maftir is a portion repeated from the last aliya.  If the Maftir is from a second Torah, the rules as on other occasions apply \engcite{Kitzur Shul\d{h}an Arukh}{78:8}.}
	\engitem{On a Shabbat Mincha, Monday, or Thursday, if the problem is found in the final aliya and in a location where the aliya can be ended (not within 3 verses of a paragraph break and after 3 verses), the reading can be concluded at that spot.  The concluding blessing is recited and the service continues as usual.}\\

\englishcontent{On Shabbat, if possible, Hosafot should be added to make the remaining reading into 7 aliyot from the kosher Sefer Torah \engcite{Kitzur Shul\d{h}an Arukh}{24:7}.}\\

\englishcontent{On a day when multiple Sifrei Torah are used, the other scrolls already used or planned to be used should not be used to replace the invalid scroll.  Instead a third or fourth Torah should be used, if available \engcite{Kitzur Shul\d{h}an Arukh}{78:10}. An invalid Sefer Torah may be used if the error is not in the book currently being read and no other Sefer Torah is available.  This leniency does not apply to Shabbat afternoon \engcite{Kitzur Shul\d{h}an Arukh}{24:10}. If no kosher Sefer Torah is available, the readings may proceed from an invalid one, but without the blessings on the aliyot.  In such a case the berakhot on the Haftarah are not recited either \engcite{Kitzur Shul\d{h}an Arukh}{79:10}.}

